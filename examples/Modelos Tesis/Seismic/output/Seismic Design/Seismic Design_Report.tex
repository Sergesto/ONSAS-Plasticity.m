\documentclass[a4paper,10pt]{article} 
\usepackage[a4paper,margin=20mm]{geometry} 
\usepackage{longtable} 
\usepackage{float} 
\usepackage{adjustbox} 
\usepackage{graphicx} 
\usepackage{color} 
\usepackage{booktabs} 
\aboverulesep=0ex 
\belowrulesep=0ex 
\usepackage{array} 
\newcolumntype{?}{!{\vrule width 1pt}}
\usepackage{fancyhdr} 
\pagestyle{fancy} 
\fancyhf{} 
\rhead{Problem: Seismic Design } 
\lfoot{Date: \today} 
\rfoot{Page \thepage} 
\renewcommand{\footrulewidth}{1pt} 
\renewcommand{\headrulewidth}{1.5pt} 
\setlength{\parindent}{0pt} 
\usepackage[T1]{fontenc} 
\usepackage{libertine} 
\usepackage{arydshln} 
\definecolor{miblue}{rgb}{0,0.1,0.38} 
\usepackage{titlesec} 
\titleformat{\section}{\normalfont\Large\color{miblue}\bfseries}{\color{miblue}\sectionmark\thesection}{0.5em}{}[{\color{miblue}\titlerule[0.5pt]}] 

\begin{document} 
This is an ONSAS automatically-generated report with part of the results obtained after the analysis. The user can access other magnitudes and results through the GNU-Octave/MATLAB console. The code is provided AS IS \textbf{WITHOUT WARRANTY of any kind}, express or implied.

\section{Analysis results}

\begin{longtable}{cccccc} 
$\#t$ & $t$ & its & $\| RHS \|$ & $\| \Delta u \|$ & flagExit \\ \hline 
 \endhead 
   1 &  0.00e+00 &    0 &           &           &   \\ 
 \hdashline 
     &           &    1 &  1.27e-08 &  1.25e-10 &      \\ 
     &           &    2 &  1.26e-08 &  3.61e-16 &      \\ 
     &           &    3 &  1.26e-08 &  3.61e-16 &      \\ 
     &           &    4 &  6.51e-08 &  3.60e-16 &      \\ 
     &           &    5 &  1.27e-08 &  1.86e-15 &      \\ 
     &           &    6 &  1.27e-08 &  3.62e-16 &      \\ 
     &           &    7 &  1.27e-08 &  3.62e-16 &      \\ 
     &           &    8 &  1.26e-08 &  3.61e-16 &      \\ 
     &           &    9 &  1.26e-08 &  3.61e-16 &      \\ 
     &           &   10 &  6.51e-08 &  3.60e-16 &      \\ 
   2 &  1.00e+01 &   10 &           &           & iters  \\ 
 \hdashline 
     &           &    1 &  4.43e-09 &  6.64e-11 &      \\ 
     &           &    2 &  4.42e-09 &  1.26e-16 &      \\ 
     &           &    3 &  4.42e-09 &  1.26e-16 &      \\ 
     &           &    4 &  4.41e-09 &  1.26e-16 &      \\ 
     &           &    5 &  4.40e-09 &  1.26e-16 &      \\ 
     &           &    6 &  4.40e-09 &  1.26e-16 &      \\ 
     &           &    7 &  4.39e-09 &  1.26e-16 &      \\ 
     &           &    8 &  4.39e-09 &  1.25e-16 &      \\ 
     &           &    9 &  4.38e-09 &  1.25e-16 &      \\ 
     &           &   10 &  4.37e-09 &  1.25e-16 &      \\ 
   3 &  2.00e+01 &   10 &           &           & iters  \\ 
 \hdashline 
     &           &    1 &  2.36e-08 &  2.24e-11 &      \\ 
     &           &    2 &  5.42e-08 &  6.72e-16 &      \\ 
     &           &    3 &  2.36e-08 &  1.55e-15 &      \\ 
     &           &    4 &  2.36e-08 &  6.73e-16 &      \\ 
     &           &    5 &  2.35e-08 &  6.72e-16 &      \\ 
     &           &    6 &  5.42e-08 &  6.71e-16 &      \\ 
     &           &    7 &  2.36e-08 &  1.55e-15 &      \\ 
     &           &    8 &  2.35e-08 &  6.73e-16 &      \\ 
     &           &    9 &  5.42e-08 &  6.72e-16 &      \\ 
     &           &   10 &  2.36e-08 &  1.55e-15 &      \\ 
   4 &  3.00e+01 &   10 &           &           & iters  \\ 
 \hdashline 
     &           &    1 &  1.39e-10 &  1.46e-11 &      \\ 
     &           &    2 &  1.38e-10 &  3.95e-18 &      \\ 
     &           &    3 &  1.38e-10 &  3.95e-18 &      \\ 
     &           &    4 &  1.38e-10 &  3.94e-18 &      \\ 
     &           &    5 &  1.38e-10 &  3.94e-18 &      \\ 
     &           &    6 &  1.38e-10 &  3.93e-18 &      \\ 
     &           &    7 &  1.37e-10 &  3.93e-18 &      \\ 
     &           &    8 &  1.37e-10 &  3.92e-18 &      \\ 
     &           &    9 &  1.37e-10 &  3.91e-18 &      \\ 
     &           &   10 &  1.37e-10 &  3.91e-18 &      \\ 
   5 &  4.00e+01 &   10 &           &           & iters  \\ 
 \hdashline 
     &           &    1 &  3.79e-08 &  3.22e-12 &      \\ 
     &           &    2 &  3.99e-08 &  1.08e-15 &      \\ 
     &           &    3 &  3.79e-08 &  1.14e-15 &      \\ 
     &           &    4 &  3.99e-08 &  1.08e-15 &      \\ 
     &           &    5 &  3.79e-08 &  1.14e-15 &      \\ 
     &           &    6 &  3.99e-08 &  1.08e-15 &      \\ 
     &           &    7 &  3.79e-08 &  1.14e-15 &      \\ 
     &           &    8 &  3.99e-08 &  1.08e-15 &      \\ 
     &           &    9 &  3.79e-08 &  1.14e-15 &      \\ 
     &           &   10 &  3.99e-08 &  1.08e-15 &      \\ 
   6 &  5.00e+01 &   10 &           &           & iters  \\ 
 \hdashline 
     &           &    1 &  3.98e-08 &  5.91e-11 &      \\ 
     &           &    2 &  3.80e-08 &  1.14e-15 &      \\ 
     &           &    3 &  3.98e-08 &  1.08e-15 &      \\ 
     &           &    4 &  3.80e-08 &  1.14e-15 &      \\ 
     &           &    5 &  3.98e-08 &  1.08e-15 &      \\ 
     &           &    6 &  3.80e-08 &  1.14e-15 &      \\ 
     &           &    7 &  3.98e-08 &  1.08e-15 &      \\ 
     &           &    8 &  3.80e-08 &  1.14e-15 &      \\ 
     &           &    9 &  3.98e-08 &  1.08e-15 &      \\ 
     &           &   10 &  3.80e-08 &  1.14e-15 &      \\ 
   7 &  6.00e+01 &   10 &           &           & iters  \\ 
 \hdashline 
     &           &    1 &  4.50e-08 &  1.49e-10 &      \\ 
     &           &    2 &  3.27e-08 &  1.29e-15 &      \\ 
     &           &    3 &  3.27e-08 &  9.34e-16 &      \\ 
     &           &    4 &  4.51e-08 &  9.32e-16 &      \\ 
     &           &    5 &  3.27e-08 &  1.29e-15 &      \\ 
     &           &    6 &  4.51e-08 &  9.33e-16 &      \\ 
     &           &    7 &  3.27e-08 &  1.29e-15 &      \\ 
     &           &    8 &  4.50e-08 &  9.33e-16 &      \\ 
     &           &    9 &  3.27e-08 &  1.29e-15 &      \\ 
     &           &   10 &  3.27e-08 &  9.34e-16 &      \\ 
   8 &  7.00e+01 &   10 &           &           & iters  \\ 
 \hdashline 
     &           &    1 &  5.82e-08 &  6.39e-11 &      \\ 
     &           &    2 &  1.96e-08 &  1.66e-15 &      \\ 
     &           &    3 &  1.96e-08 &  5.60e-16 &      \\ 
     &           &    4 &  1.96e-08 &  5.59e-16 &      \\ 
     &           &    5 &  5.82e-08 &  5.58e-16 &      \\ 
     &           &    6 &  1.96e-08 &  1.66e-15 &      \\ 
     &           &    7 &  1.96e-08 &  5.60e-16 &      \\ 
     &           &    8 &  1.96e-08 &  5.59e-16 &      \\ 
     &           &    9 &  5.82e-08 &  5.58e-16 &      \\ 
     &           &   10 &  1.96e-08 &  1.66e-15 &      \\ 
   9 &  8.00e+01 &   10 &           &           & iters  \\ 
 \hdashline 
     &           &    1 &  4.96e-08 &  1.93e-10 &      \\ 
     &           &    2 &  2.81e-08 &  1.42e-15 &      \\ 
     &           &    3 &  2.81e-08 &  8.03e-16 &      \\ 
     &           &    4 &  4.96e-08 &  8.02e-16 &      \\ 
     &           &    5 &  2.81e-08 &  1.42e-15 &      \\ 
     &           &    6 &  2.81e-08 &  8.03e-16 &      \\ 
     &           &    7 &  4.96e-08 &  8.02e-16 &      \\ 
     &           &    8 &  2.81e-08 &  1.42e-15 &      \\ 
     &           &    9 &  2.81e-08 &  8.03e-16 &      \\ 
     &           &   10 &  4.96e-08 &  8.02e-16 &      \\ 
  10 &  9.00e+01 &   10 &           &           & iters  \\ 
 \hdashline 
     &           &    1 &  3.47e-08 &  3.00e-10 &      \\ 
     &           &    2 &  4.30e-08 &  9.90e-16 &      \\ 
     &           &    3 &  3.47e-08 &  1.23e-15 &      \\ 
     &           &    4 &  4.30e-08 &  9.91e-16 &      \\ 
     &           &    5 &  3.47e-08 &  1.23e-15 &      \\ 
     &           &    6 &  3.47e-08 &  9.91e-16 &      \\ 
     &           &    7 &  4.31e-08 &  9.90e-16 &      \\ 
     &           &    8 &  3.47e-08 &  1.23e-15 &      \\ 
     &           &    9 &  4.31e-08 &  9.90e-16 &      \\ 
     &           &   10 &  3.47e-08 &  1.23e-15 &      \\ 
  11 &  1.00e+02 &   10 &           &           & iters  \\ 
 \hdashline 
     &           &    1 &  1.72e-08 &  1.02e-10 &      \\ 
     &           &    2 &  6.05e-08 &  4.92e-16 &      \\ 
     &           &    3 &  1.73e-08 &  1.73e-15 &      \\ 
     &           &    4 &  1.73e-08 &  4.94e-16 &      \\ 
     &           &    5 &  1.73e-08 &  4.93e-16 &      \\ 
     &           &    6 &  1.72e-08 &  4.92e-16 &      \\ 
     &           &    7 &  6.05e-08 &  4.92e-16 &      \\ 
     &           &    8 &  1.73e-08 &  1.73e-15 &      \\ 
     &           &    9 &  1.73e-08 &  4.94e-16 &      \\ 
     &           &   10 &  1.72e-08 &  4.93e-16 &      \\ 
  12 &  1.10e+02 &   10 &           &           & iters  \\ 
 \hdashline 
     &           &    1 &  5.56e-08 &  3.67e-10 &      \\ 
     &           &    2 &  2.21e-08 &  1.59e-15 &      \\ 
     &           &    3 &  2.21e-08 &  6.32e-16 &      \\ 
     &           &    4 &  2.21e-08 &  6.31e-16 &      \\ 
     &           &    5 &  5.56e-08 &  6.30e-16 &      \\ 
     &           &    6 &  2.21e-08 &  1.59e-15 &      \\ 
     &           &    7 &  2.21e-08 &  6.32e-16 &      \\ 
     &           &    8 &  5.56e-08 &  6.31e-16 &      \\ 
     &           &    9 &  2.22e-08 &  1.59e-15 &      \\ 
     &           &   10 &  2.21e-08 &  6.32e-16 &      \\ 
  13 &  1.20e+02 &   10 &           &           & iters  \\ 
 \hdashline 
     &           &    1 &  2.45e-08 &  6.34e-10 &      \\ 
     &           &    2 &  2.45e-08 &  6.99e-16 &      \\ 
     &           &    3 &  5.33e-08 &  6.98e-16 &      \\ 
     &           &    4 &  2.45e-08 &  1.52e-15 &      \\ 
     &           &    5 &  2.45e-08 &  6.99e-16 &      \\ 
     &           &    6 &  5.33e-08 &  6.98e-16 &      \\ 
     &           &    7 &  2.45e-08 &  1.52e-15 &      \\ 
     &           &    8 &  2.45e-08 &  7.00e-16 &      \\ 
     &           &    9 &  5.32e-08 &  6.99e-16 &      \\ 
     &           &   10 &  2.45e-08 &  1.52e-15 &      \\ 
  14 &  1.30e+02 &   10 &           &           & iters  \\ 
 \hdashline 
     &           &    1 &  6.38e-08 &  1.19e-11 &      \\ 
     &           &    2 &  1.40e-08 &  1.82e-15 &      \\ 
     &           &    3 &  1.40e-08 &  4.00e-16 &      \\ 
     &           &    4 &  1.40e-08 &  3.99e-16 &      \\ 
     &           &    5 &  1.39e-08 &  3.99e-16 &      \\ 
     &           &    6 &  1.39e-08 &  3.98e-16 &      \\ 
     &           &    7 &  6.38e-08 &  3.98e-16 &      \\ 
     &           &    8 &  1.40e-08 &  1.82e-15 &      \\ 
     &           &    9 &  1.40e-08 &  4.00e-16 &      \\ 
     &           &   10 &  1.40e-08 &  3.99e-16 &      \\ 
  15 &  1.40e+02 &   10 &           &           & iters  \\ 
 \hdashline 
     &           &    1 &  3.94e-08 &  9.32e-10 &      \\ 
     &           &    2 &  3.84e-08 &  1.12e-15 &      \\ 
     &           &    3 &  3.94e-08 &  1.10e-15 &      \\ 
     &           &    4 &  3.84e-08 &  1.12e-15 &      \\ 
     &           &    5 &  3.83e-08 &  1.10e-15 &      \\ 
     &           &    6 &  3.94e-08 &  1.09e-15 &      \\ 
     &           &    7 &  3.83e-08 &  1.12e-15 &      \\ 
     &           &    8 &  3.94e-08 &  1.09e-15 &      \\ 
     &           &    9 &  3.83e-08 &  1.12e-15 &      \\ 
     &           &   10 &  3.94e-08 &  1.09e-15 &      \\ 
  16 &  1.50e+02 &   10 &           &           & iters  \\ 
 \hdashline 
     &           &    1 &  1.81e-08 &  8.40e-10 &      \\ 
     &           &    2 &  1.80e-08 &  5.15e-16 &      \\ 
     &           &    3 &  5.97e-08 &  5.14e-16 &      \\ 
     &           &    4 &  1.81e-08 &  1.70e-15 &      \\ 
     &           &    5 &  1.81e-08 &  5.16e-16 &      \\ 
     &           &    6 &  1.80e-08 &  5.15e-16 &      \\ 
     &           &    7 &  5.97e-08 &  5.15e-16 &      \\ 
     &           &    8 &  1.81e-08 &  1.70e-15 &      \\ 
     &           &    9 &  1.81e-08 &  5.16e-16 &      \\ 
     &           &   10 &  1.80e-08 &  5.16e-16 &      \\ 
  17 &  1.60e+02 &   10 &           &           & iters  \\ 
 \hdashline 
     &           &    1 &  2.22e-08 &  2.28e-11 &      \\ 
     &           &    2 &  2.22e-08 &  6.35e-16 &      \\ 
     &           &    3 &  1.67e-08 &  6.34e-16 &      \\ 
     &           &    4 &  2.22e-08 &  4.76e-16 &      \\ 
     &           &    5 &  1.67e-08 &  6.34e-16 &      \\ 
     &           &    6 &  1.66e-08 &  4.76e-16 &      \\ 
     &           &    7 &  2.22e-08 &  4.75e-16 &      \\ 
     &           &    8 &  1.67e-08 &  6.34e-16 &      \\ 
     &           &    9 &  2.22e-08 &  4.75e-16 &      \\ 
     &           &   10 &  1.67e-08 &  6.34e-16 &      \\ 
  18 &  1.70e+02 &   10 &           &           & iters  \\ 
 \hdashline 
     &           &    1 &  4.24e-08 &  1.28e-09 &      \\ 
     &           &    2 &  3.53e-08 &  1.21e-15 &      \\ 
     &           &    3 &  3.53e-08 &  1.01e-15 &      \\ 
     &           &    4 &  4.25e-08 &  1.01e-15 &      \\ 
     &           &    5 &  3.53e-08 &  1.21e-15 &      \\ 
     &           &    6 &  4.24e-08 &  1.01e-15 &      \\ 
     &           &    7 &  3.53e-08 &  1.21e-15 &      \\ 
     &           &    8 &  4.24e-08 &  1.01e-15 &      \\ 
     &           &    9 &  3.53e-08 &  1.21e-15 &      \\ 
     &           &   10 &  4.24e-08 &  1.01e-15 &      \\ 
  19 &  1.80e+02 &   10 &           &           & iters  \\ 
 \hdashline 
     &           &    1 &  1.67e-10 &  1.60e-09 &      \\ 
     &           &    2 &  1.67e-10 &  4.76e-18 &      \\ 
     &           &    3 &  1.66e-10 &  4.76e-18 &      \\ 
     &           &    4 &  1.66e-10 &  4.75e-18 &      \\ 
     &           &    5 &  1.66e-10 &  4.74e-18 &      \\ 
     &           &    6 &  1.66e-10 &  4.74e-18 &      \\ 
     &           &    7 &  1.65e-10 &  4.73e-18 &      \\ 
     &           &    8 &  1.65e-10 &  4.72e-18 &      \\ 
     &           &    9 &  1.65e-10 &  4.72e-18 &      \\ 
     &           &   10 &  1.65e-10 &  4.71e-18 &      \\ 
  20 &  1.90e+02 &   10 &           &           & iters  \\ 
 \hdashline 
     &           &    1 &  8.25e-09 &  3.59e-10 &      \\ 
     &           &    2 &  8.24e-09 &  2.35e-16 &      \\ 
     &           &    3 &  8.23e-09 &  2.35e-16 &      \\ 
     &           &    4 &  8.22e-09 &  2.35e-16 &      \\ 
     &           &    5 &  8.20e-09 &  2.34e-16 &      \\ 
     &           &    6 &  8.19e-09 &  2.34e-16 &      \\ 
     &           &    7 &  8.18e-09 &  2.34e-16 &      \\ 
     &           &    8 &  8.17e-09 &  2.33e-16 &      \\ 
     &           &    9 &  8.16e-09 &  2.33e-16 &      \\ 
     &           &   10 &  8.14e-09 &  2.33e-16 &      \\ 
  21 &  2.00e+02 &   10 &           &           & iters  \\ 
 \hdashline 
     &           &    1 &  5.19e-08 &  2.24e-09 &      \\ 
     &           &    2 &  1.29e-08 &  1.48e-15 &      \\ 
     &           &    3 &  1.29e-08 &  3.69e-16 &      \\ 
     &           &    4 &  1.29e-08 &  3.69e-16 &      \\ 
     &           &    5 &  6.48e-08 &  3.68e-16 &      \\ 
     &           &    6 &  1.30e-08 &  1.85e-15 &      \\ 
     &           &    7 &  1.30e-08 &  3.71e-16 &      \\ 
     &           &    8 &  1.29e-08 &  3.70e-16 &      \\ 
     &           &    9 &  1.29e-08 &  3.69e-16 &      \\ 
     &           &   10 &  1.29e-08 &  3.69e-16 &      \\ 
  22 &  2.10e+02 &   10 &           &           & iters  \\ 
 \hdashline 
     &           &    1 &  4.84e-08 &  2.13e-09 &      \\ 
     &           &    2 &  9.52e-09 &  1.38e-15 &      \\ 
     &           &    3 &  9.51e-09 &  2.72e-16 &      \\ 
     &           &    4 &  9.49e-09 &  2.71e-16 &      \\ 
     &           &    5 &  9.48e-09 &  2.71e-16 &      \\ 
     &           &    6 &  6.82e-08 &  2.71e-16 &      \\ 
     &           &    7 &  9.56e-09 &  1.95e-15 &      \\ 
     &           &    8 &  9.55e-09 &  2.73e-16 &      \\ 
     &           &    9 &  9.54e-09 &  2.73e-16 &      \\ 
     &           &   10 &  9.52e-09 &  2.72e-16 &      \\ 
  23 &  2.20e+02 &   10 &           &           & iters  \\ 
 \hdashline 
     &           &    1 &  5.69e-09 &  7.11e-10 &      \\ 
     &           &    2 &  5.68e-09 &  1.62e-16 &      \\ 
     &           &    3 &  5.67e-09 &  1.62e-16 &      \\ 
     &           &    4 &  5.67e-09 &  1.62e-16 &      \\ 
     &           &    5 &  5.66e-09 &  1.62e-16 &      \\ 
     &           &    6 &  5.65e-09 &  1.61e-16 &      \\ 
     &           &    7 &  3.32e-08 &  1.61e-16 &      \\ 
     &           &    8 &  5.69e-09 &  9.48e-16 &      \\ 
     &           &    9 &  5.68e-09 &  1.62e-16 &      \\ 
     &           &   10 &  5.67e-09 &  1.62e-16 &      \\ 
  24 &  2.30e+02 &   10 &           &           & iters  \\ 
 \hdashline 
     &           &    1 &  2.49e-08 &  3.63e-09 &      \\ 
     &           &    2 &  5.29e-08 &  7.10e-16 &      \\ 
     &           &    3 &  2.49e-08 &  1.51e-15 &      \\ 
     &           &    4 &  2.49e-08 &  7.11e-16 &      \\ 
     &           &    5 &  5.29e-08 &  7.10e-16 &      \\ 
     &           &    6 &  2.49e-08 &  1.51e-15 &      \\ 
     &           &    7 &  2.49e-08 &  7.11e-16 &      \\ 
     &           &    8 &  5.28e-08 &  7.10e-16 &      \\ 
     &           &    9 &  2.49e-08 &  1.51e-15 &      \\ 
     &           &   10 &  2.49e-08 &  7.11e-16 &      \\ 
  25 &  2.40e+02 &   10 &           &           & iters  \\ 
 \hdashline 
     &           &    1 &  1.47e-08 &  1.50e-09 &      \\ 
     &           &    2 &  1.47e-08 &  4.19e-16 &      \\ 
     &           &    3 &  6.31e-08 &  4.18e-16 &      \\ 
     &           &    4 &  1.47e-08 &  1.80e-15 &      \\ 
     &           &    5 &  1.47e-08 &  4.20e-16 &      \\ 
     &           &    6 &  1.47e-08 &  4.20e-16 &      \\ 
     &           &    7 &  1.47e-08 &  4.19e-16 &      \\ 
     &           &    8 &  6.30e-08 &  4.18e-16 &      \\ 
     &           &    9 &  1.47e-08 &  1.80e-15 &      \\ 
     &           &   10 &  1.47e-08 &  4.20e-16 &      \\ 
  26 &  2.50e+02 &   10 &           &           & iters  \\ 
 \hdashline 
     &           &    1 &  3.06e-08 &  2.23e-09 &      \\ 
     &           &    2 &  4.71e-08 &  8.73e-16 &      \\ 
     &           &    3 &  3.06e-08 &  1.35e-15 &      \\ 
     &           &    4 &  4.71e-08 &  8.74e-16 &      \\ 
     &           &    5 &  3.06e-08 &  1.34e-15 &      \\ 
     &           &    6 &  3.06e-08 &  8.74e-16 &      \\ 
     &           &    7 &  4.71e-08 &  8.73e-16 &      \\ 
     &           &    8 &  3.06e-08 &  1.35e-15 &      \\ 
     &           &    9 &  4.71e-08 &  8.74e-16 &      \\ 
     &           &   10 &  3.06e-08 &  1.34e-15 &      \\ 
  27 &  2.60e+02 &   10 &           &           & iters  \\ 
 \hdashline 
     &           &    1 &  4.97e-08 &  2.05e-09 &      \\ 
     &           &    2 &  1.08e-08 &  1.42e-15 &      \\ 
     &           &    3 &  1.08e-08 &  3.08e-16 &      \\ 
     &           &    4 &  6.69e-08 &  3.07e-16 &      \\ 
     &           &    5 &  4.97e-08 &  1.91e-15 &      \\ 
     &           &    6 &  1.08e-08 &  1.42e-15 &      \\ 
     &           &    7 &  1.08e-08 &  3.08e-16 &      \\ 
     &           &    8 &  6.69e-08 &  3.07e-16 &      \\ 
     &           &    9 &  1.08e-08 &  1.91e-15 &      \\ 
     &           &   10 &  1.08e-08 &  3.10e-16 &      \\ 
  28 &  2.70e+02 &   10 &           &           & iters  \\ 
 \hdashline 
     &           &    1 &  2.55e-08 &  4.21e-10 &      \\ 
     &           &    2 &  5.22e-08 &  7.28e-16 &      \\ 
     &           &    3 &  2.55e-08 &  1.49e-15 &      \\ 
     &           &    4 &  2.55e-08 &  7.29e-16 &      \\ 
     &           &    5 &  5.22e-08 &  7.28e-16 &      \\ 
     &           &    6 &  2.55e-08 &  1.49e-15 &      \\ 
     &           &    7 &  2.55e-08 &  7.29e-16 &      \\ 
     &           &    8 &  5.22e-08 &  7.28e-16 &      \\ 
     &           &    9 &  2.55e-08 &  1.49e-15 &      \\ 
     &           &   10 &  2.55e-08 &  7.29e-16 &      \\ 
  29 &  2.80e+02 &   10 &           &           & iters  \\ 
 \hdashline 
     &           &    1 &  1.27e-08 &  1.50e-10 &      \\ 
     &           &    2 &  1.27e-08 &  3.63e-16 &      \\ 
     &           &    3 &  1.27e-08 &  3.62e-16 &      \\ 
     &           &    4 &  1.27e-08 &  3.62e-16 &      \\ 
     &           &    5 &  1.26e-08 &  3.61e-16 &      \\ 
     &           &    6 &  6.51e-08 &  3.61e-16 &      \\ 
     &           &    7 &  1.27e-08 &  1.86e-15 &      \\ 
     &           &    8 &  1.27e-08 &  3.63e-16 &      \\ 
     &           &    9 &  1.27e-08 &  3.62e-16 &      \\ 
     &           &   10 &  1.27e-08 &  3.62e-16 &      \\ 
  30 &  2.90e+02 &   10 &           &           & iters  \\ 
 \hdashline 
     &           &    1 &  7.38e-09 &  1.22e-09 &      \\ 
     &           &    2 &  7.37e-09 &  2.11e-16 &      \\ 
     &           &    3 &  7.36e-09 &  2.10e-16 &      \\ 
     &           &    4 &  7.35e-09 &  2.10e-16 &      \\ 
     &           &    5 &  7.34e-09 &  2.10e-16 &      \\ 
     &           &    6 &  7.33e-09 &  2.09e-16 &      \\ 
     &           &    7 &  7.32e-09 &  2.09e-16 &      \\ 
     &           &    8 &  7.31e-09 &  2.09e-16 &      \\ 
     &           &    9 &  7.04e-08 &  2.09e-16 &      \\ 
     &           &   10 &  4.62e-08 &  2.01e-15 &      \\ 
  31 &  3.00e+02 &   10 &           &           & iters  \\ 
 \hdashline 
     &           &    1 &  6.96e-08 &  2.64e-09 &      \\ 
     &           &    2 &  8.58e-08 &  1.99e-15 &      \\ 
     &           &    3 &  6.97e-08 &  2.45e-15 &      \\ 
     &           &    4 &  8.13e-09 &  1.99e-15 &      \\ 
     &           &    5 &  8.12e-09 &  2.32e-16 &      \\ 
     &           &    6 &  8.11e-09 &  2.32e-16 &      \\ 
     &           &    7 &  8.10e-09 &  2.31e-16 &      \\ 
     &           &    8 &  8.08e-09 &  2.31e-16 &      \\ 
     &           &    9 &  8.07e-09 &  2.31e-16 &      \\ 
     &           &   10 &  8.06e-09 &  2.30e-16 &      \\ 
  32 &  3.10e+02 &   10 &           &           & iters  \\ 
 \hdashline 
     &           &    1 &  8.49e-09 &  1.35e-09 &      \\ 
     &           &    2 &  8.48e-09 &  2.42e-16 &      \\ 
     &           &    3 &  8.47e-09 &  2.42e-16 &      \\ 
     &           &    4 &  8.45e-09 &  2.42e-16 &      \\ 
     &           &    5 &  8.44e-09 &  2.41e-16 &      \\ 
     &           &    6 &  6.93e-08 &  2.41e-16 &      \\ 
     &           &    7 &  8.53e-09 &  1.98e-15 &      \\ 
     &           &    8 &  8.52e-09 &  2.43e-16 &      \\ 
     &           &    9 &  8.50e-09 &  2.43e-16 &      \\ 
     &           &   10 &  8.49e-09 &  2.43e-16 &      \\ 
  33 &  3.20e+02 &   10 &           &           & iters  \\ 
 \hdashline 
     &           &    1 &  1.77e-08 &  8.69e-10 &      \\ 
     &           &    2 &  1.77e-08 &  5.06e-16 &      \\ 
     &           &    3 &  6.00e-08 &  5.05e-16 &      \\ 
     &           &    4 &  1.78e-08 &  1.71e-15 &      \\ 
     &           &    5 &  1.77e-08 &  5.07e-16 &      \\ 
     &           &    6 &  1.77e-08 &  5.06e-16 &      \\ 
     &           &    7 &  6.00e-08 &  5.06e-16 &      \\ 
     &           &    8 &  1.78e-08 &  1.71e-15 &      \\ 
     &           &    9 &  1.78e-08 &  5.07e-16 &      \\ 
     &           &   10 &  1.77e-08 &  5.07e-16 &      \\ 
  34 &  3.30e+02 &   10 &           &           & iters  \\ 
 \hdashline 
     &           &    1 &  6.96e-08 &  1.58e-09 &      \\ 
     &           &    2 &  8.17e-09 &  1.99e-15 &      \\ 
     &           &    3 &  8.16e-09 &  2.33e-16 &      \\ 
     &           &    4 &  8.14e-09 &  2.33e-16 &      \\ 
     &           &    5 &  8.13e-09 &  2.32e-16 &      \\ 
     &           &    6 &  8.12e-09 &  2.32e-16 &      \\ 
     &           &    7 &  8.11e-09 &  2.32e-16 &      \\ 
     &           &    8 &  8.10e-09 &  2.31e-16 &      \\ 
     &           &    9 &  6.96e-08 &  2.31e-16 &      \\ 
     &           &   10 &  8.59e-08 &  1.99e-15 &      \\ 
  35 &  3.40e+02 &   10 &           &           & iters  \\ 
 \hdashline 
     &           &    1 &  9.25e-09 &  2.00e-09 &      \\ 
     &           &    2 &  9.24e-09 &  2.64e-16 &      \\ 
     &           &    3 &  9.23e-09 &  2.64e-16 &      \\ 
     &           &    4 &  9.21e-09 &  2.63e-16 &      \\ 
     &           &    5 &  9.20e-09 &  2.63e-16 &      \\ 
     &           &    6 &  9.19e-09 &  2.63e-16 &      \\ 
     &           &    7 &  2.97e-08 &  2.62e-16 &      \\ 
     &           &    8 &  9.21e-09 &  8.47e-16 &      \\ 
     &           &    9 &  9.20e-09 &  2.63e-16 &      \\ 
     &           &   10 &  9.19e-09 &  2.63e-16 &      \\ 
  36 &  3.50e+02 &   10 &           &           & iters  \\ 
 \hdashline 
     &           &    1 &  2.36e-08 &  8.42e-10 &      \\ 
     &           &    2 &  1.52e-08 &  6.75e-16 &      \\ 
     &           &    3 &  1.52e-08 &  4.35e-16 &      \\ 
     &           &    4 &  2.37e-08 &  4.34e-16 &      \\ 
     &           &    5 &  1.52e-08 &  6.75e-16 &      \\ 
     &           &    6 &  1.52e-08 &  4.34e-16 &      \\ 
     &           &    7 &  2.37e-08 &  4.34e-16 &      \\ 
     &           &    8 &  1.52e-08 &  6.75e-16 &      \\ 
     &           &    9 &  2.37e-08 &  4.34e-16 &      \\ 
     &           &   10 &  1.52e-08 &  6.75e-16 &      \\ 
  37 &  3.60e+02 &   10 &           &           & iters  \\ 
 \hdashline 
     &           &    1 &  2.39e-08 &  8.13e-10 &      \\ 
     &           &    2 &  1.50e-08 &  6.81e-16 &      \\ 
     &           &    3 &  1.50e-08 &  4.29e-16 &      \\ 
     &           &    4 &  2.39e-08 &  4.28e-16 &      \\ 
     &           &    5 &  1.50e-08 &  6.81e-16 &      \\ 
     &           &    6 &  2.39e-08 &  4.28e-16 &      \\ 
     &           &    7 &  1.50e-08 &  6.81e-16 &      \\ 
     &           &    8 &  1.50e-08 &  4.29e-16 &      \\ 
     &           &    9 &  2.39e-08 &  4.28e-16 &      \\ 
     &           &   10 &  1.50e-08 &  6.81e-16 &      \\ 
  38 &  3.70e+02 &   10 &           &           & iters  \\ 
 \hdashline 
     &           &    1 &  1.83e-08 &  1.46e-09 &      \\ 
     &           &    2 &  1.83e-08 &  5.23e-16 &      \\ 
     &           &    3 &  1.83e-08 &  5.22e-16 &      \\ 
     &           &    4 &  5.94e-08 &  5.22e-16 &      \\ 
     &           &    5 &  1.83e-08 &  1.70e-15 &      \\ 
     &           &    6 &  1.83e-08 &  5.23e-16 &      \\ 
     &           &    7 &  1.83e-08 &  5.23e-16 &      \\ 
     &           &    8 &  5.94e-08 &  5.22e-16 &      \\ 
     &           &    9 &  1.83e-08 &  1.70e-15 &      \\ 
     &           &   10 &  1.83e-08 &  5.24e-16 &      \\ 
  39 &  3.80e+02 &   10 &           &           & iters  \\ 
 \hdashline 
     &           &    1 &  2.93e-08 &  1.45e-09 &      \\ 
     &           &    2 &  2.92e-08 &  8.35e-16 &      \\ 
     &           &    3 &  2.92e-08 &  8.34e-16 &      \\ 
     &           &    4 &  4.86e-08 &  8.33e-16 &      \\ 
     &           &    5 &  2.92e-08 &  1.39e-15 &      \\ 
     &           &    6 &  4.85e-08 &  8.34e-16 &      \\ 
     &           &    7 &  2.92e-08 &  1.38e-15 &      \\ 
     &           &    8 &  2.92e-08 &  8.34e-16 &      \\ 
     &           &    9 &  4.85e-08 &  8.33e-16 &      \\ 
     &           &   10 &  2.92e-08 &  1.39e-15 &      \\ 
  40 &  3.90e+02 &   10 &           &           & iters  \\ 
 \hdashline 
     &           &    1 &  3.64e-08 &  4.21e-11 &      \\ 
     &           &    2 &  4.13e-08 &  1.04e-15 &      \\ 
     &           &    3 &  3.65e-08 &  1.18e-15 &      \\ 
     &           &    4 &  4.13e-08 &  1.04e-15 &      \\ 
     &           &    5 &  3.65e-08 &  1.18e-15 &      \\ 
     &           &    6 &  4.13e-08 &  1.04e-15 &      \\ 
     &           &    7 &  3.65e-08 &  1.18e-15 &      \\ 
     &           &    8 &  3.64e-08 &  1.04e-15 &      \\ 
     &           &    9 &  4.13e-08 &  1.04e-15 &      \\ 
     &           &   10 &  3.64e-08 &  1.18e-15 &      \\ 
  41 &  4.00e+02 &   10 &           &           & iters  \\ 
 \hdashline 
     &           &    1 &  3.11e-08 &  1.92e-09 &      \\ 
     &           &    2 &  3.11e-08 &  8.89e-16 &      \\ 
     &           &    3 &  4.66e-08 &  8.88e-16 &      \\ 
     &           &    4 &  3.11e-08 &  1.33e-15 &      \\ 
     &           &    5 &  4.66e-08 &  8.88e-16 &      \\ 
     &           &    6 &  3.11e-08 &  1.33e-15 &      \\ 
     &           &    7 &  3.11e-08 &  8.89e-16 &      \\ 
     &           &    8 &  4.66e-08 &  8.88e-16 &      \\ 
     &           &    9 &  3.11e-08 &  1.33e-15 &      \\ 
     &           &   10 &  4.66e-08 &  8.88e-16 &      \\ 
  42 &  4.10e+02 &   10 &           &           & iters  \\ 
 \hdashline 
     &           &    1 &  1.41e-08 &  2.31e-09 &      \\ 
     &           &    2 &  6.36e-08 &  4.04e-16 &      \\ 
     &           &    3 &  1.42e-08 &  1.81e-15 &      \\ 
     &           &    4 &  1.42e-08 &  4.06e-16 &      \\ 
     &           &    5 &  1.42e-08 &  4.05e-16 &      \\ 
     &           &    6 &  1.42e-08 &  4.05e-16 &      \\ 
     &           &    7 &  1.41e-08 &  4.04e-16 &      \\ 
     &           &    8 &  6.36e-08 &  4.03e-16 &      \\ 
     &           &    9 &  1.42e-08 &  1.81e-15 &      \\ 
     &           &   10 &  1.42e-08 &  4.05e-16 &      \\ 
  43 &  4.20e+02 &   10 &           &           & iters  \\ 
 \hdashline 
     &           &    1 &  2.93e-08 &  6.47e-10 &      \\ 
     &           &    2 &  4.84e-08 &  8.37e-16 &      \\ 
     &           &    3 &  2.94e-08 &  1.38e-15 &      \\ 
     &           &    4 &  4.84e-08 &  8.38e-16 &      \\ 
     &           &    5 &  2.94e-08 &  1.38e-15 &      \\ 
     &           &    6 &  2.94e-08 &  8.39e-16 &      \\ 
     &           &    7 &  4.84e-08 &  8.38e-16 &      \\ 
     &           &    8 &  2.94e-08 &  1.38e-15 &      \\ 
     &           &    9 &  4.83e-08 &  8.39e-16 &      \\ 
     &           &   10 &  2.94e-08 &  1.38e-15 &      \\ 
  44 &  4.30e+02 &   10 &           &           & iters  \\ 
 \hdashline 
     &           &    1 &  4.72e-08 &  4.02e-09 &      \\ 
     &           &    2 &  3.06e-08 &  1.35e-15 &      \\ 
     &           &    3 &  4.71e-08 &  8.73e-16 &      \\ 
     &           &    4 &  3.06e-08 &  1.35e-15 &      \\ 
     &           &    5 &  3.06e-08 &  8.74e-16 &      \\ 
     &           &    6 &  4.72e-08 &  8.72e-16 &      \\ 
     &           &    7 &  3.06e-08 &  1.35e-15 &      \\ 
     &           &    8 &  4.71e-08 &  8.73e-16 &      \\ 
     &           &    9 &  3.06e-08 &  1.35e-15 &      \\ 
     &           &   10 &  3.06e-08 &  8.74e-16 &      \\ 
  45 &  4.40e+02 &   10 &           &           & iters  \\ 
 \hdashline 
     &           &    1 &  1.98e-08 &  4.68e-09 &      \\ 
     &           &    2 &  5.79e-08 &  5.65e-16 &      \\ 
     &           &    3 &  1.99e-08 &  1.65e-15 &      \\ 
     &           &    4 &  1.98e-08 &  5.67e-16 &      \\ 
     &           &    5 &  1.98e-08 &  5.66e-16 &      \\ 
     &           &    6 &  5.79e-08 &  5.65e-16 &      \\ 
     &           &    7 &  1.98e-08 &  1.65e-15 &      \\ 
     &           &    8 &  1.98e-08 &  5.67e-16 &      \\ 
     &           &    9 &  1.98e-08 &  5.66e-16 &      \\ 
     &           &   10 &  5.79e-08 &  5.65e-16 &      \\ 
  46 &  4.50e+02 &   10 &           &           & iters  \\ 
 \hdashline 
     &           &    1 &  3.94e-08 &  1.85e-09 &      \\ 
     &           &    2 &  3.83e-08 &  1.13e-15 &      \\ 
     &           &    3 &  3.94e-08 &  1.09e-15 &      \\ 
     &           &    4 &  3.83e-08 &  1.13e-15 &      \\ 
     &           &    5 &  3.94e-08 &  1.09e-15 &      \\ 
     &           &    6 &  3.83e-08 &  1.13e-15 &      \\ 
     &           &    7 &  3.94e-08 &  1.09e-15 &      \\ 
     &           &    8 &  3.83e-08 &  1.13e-15 &      \\ 
     &           &    9 &  3.94e-08 &  1.09e-15 &      \\ 
     &           &   10 &  3.83e-08 &  1.13e-15 &      \\ 
  47 &  4.60e+02 &   10 &           &           & iters  \\ 
 \hdashline 
     &           &    1 &  1.30e-08 &  4.14e-09 &      \\ 
     &           &    2 &  1.30e-08 &  3.70e-16 &      \\ 
     &           &    3 &  1.29e-08 &  3.70e-16 &      \\ 
     &           &    4 &  1.29e-08 &  3.69e-16 &      \\ 
     &           &    5 &  1.29e-08 &  3.69e-16 &      \\ 
     &           &    6 &  1.29e-08 &  3.68e-16 &      \\ 
     &           &    7 &  1.29e-08 &  3.68e-16 &      \\ 
     &           &    8 &  6.48e-08 &  3.67e-16 &      \\ 
     &           &    9 &  1.29e-08 &  1.85e-15 &      \\ 
     &           &   10 &  1.29e-08 &  3.69e-16 &      \\ 
  48 &  4.70e+02 &   10 &           &           & iters  \\ 
 \hdashline 
     &           &    1 &  7.03e-09 &  9.10e-09 &      \\ 
     &           &    2 &  7.02e-09 &  2.01e-16 &      \\ 
     &           &    3 &  7.07e-08 &  2.00e-16 &      \\ 
     &           &    4 &  4.60e-08 &  2.02e-15 &      \\ 
     &           &    5 &  7.05e-09 &  1.31e-15 &      \\ 
     &           &    6 &  7.04e-09 &  2.01e-16 &      \\ 
     &           &    7 &  7.03e-09 &  2.01e-16 &      \\ 
     &           &    8 &  7.02e-09 &  2.01e-16 &      \\ 
     &           &    9 &  7.07e-08 &  2.00e-16 &      \\ 
     &           &   10 &  4.60e-08 &  2.02e-15 &      \\ 
  49 &  4.80e+02 &   10 &           &           & iters  \\ 
 \hdashline 
     &           &    1 &  3.67e-08 &  8.18e-09 &      \\ 
     &           &    2 &  4.10e-08 &  1.05e-15 &      \\ 
     &           &    3 &  3.67e-08 &  1.17e-15 &      \\ 
     &           &    4 &  4.10e-08 &  1.05e-15 &      \\ 
     &           &    5 &  3.67e-08 &  1.17e-15 &      \\ 
     &           &    6 &  4.10e-08 &  1.05e-15 &      \\ 
     &           &    7 &  3.67e-08 &  1.17e-15 &      \\ 
     &           &    8 &  4.10e-08 &  1.05e-15 &      \\ 
     &           &    9 &  3.67e-08 &  1.17e-15 &      \\ 
     &           &   10 &  3.67e-08 &  1.05e-15 &      \\ 
  50 &  4.90e+02 &   10 &           &           & iters  \\ 
 \hdashline 
     &           &    1 &  6.69e-08 &  1.95e-09 &      \\ 
     &           &    2 &  1.09e-08 &  1.91e-15 &      \\ 
     &           &    3 &  1.09e-08 &  3.12e-16 &      \\ 
     &           &    4 &  1.09e-08 &  3.11e-16 &      \\ 
     &           &    5 &  1.09e-08 &  3.11e-16 &      \\ 
     &           &    6 &  6.68e-08 &  3.11e-16 &      \\ 
     &           &    7 &  8.87e-08 &  1.91e-15 &      \\ 
     &           &    8 &  6.69e-08 &  2.53e-15 &      \\ 
     &           &    9 &  1.09e-08 &  1.91e-15 &      \\ 
     &           &   10 &  1.09e-08 &  3.12e-16 &      \\ 
  51 &  5.00e+02 &   10 &           &           & iters  \\ 
 \hdashline 
     &           &    1 &  2.25e-08 &  1.12e-08 &      \\ 
     &           &    2 &  2.24e-08 &  6.41e-16 &      \\ 
     &           &    3 &  5.53e-08 &  6.40e-16 &      \\ 
     &           &    4 &  2.25e-08 &  1.58e-15 &      \\ 
     &           &    5 &  2.24e-08 &  6.41e-16 &      \\ 
     &           &    6 &  5.53e-08 &  6.40e-16 &      \\ 
     &           &    7 &  2.25e-08 &  1.58e-15 &      \\ 
     &           &    8 &  2.25e-08 &  6.42e-16 &      \\ 
     &           &    9 &  5.53e-08 &  6.41e-16 &      \\ 
     &           &   10 &  2.25e-08 &  1.58e-15 &      \\ 
  52 &  5.10e+02 &   10 &           &           & iters  \\ 
 \hdashline 
     &           &    1 &  9.42e-09 &  8.31e-09 &      \\ 
     &           &    2 &  9.41e-09 &  2.69e-16 &      \\ 
     &           &    3 &  6.83e-08 &  2.68e-16 &      \\ 
     &           &    4 &  8.72e-08 &  1.95e-15 &      \\ 
     &           &    5 &  6.83e-08 &  2.49e-15 &      \\ 
     &           &    6 &  9.46e-09 &  1.95e-15 &      \\ 
     &           &    7 &  9.45e-09 &  2.70e-16 &      \\ 
     &           &    8 &  9.44e-09 &  2.70e-16 &      \\ 
     &           &    9 &  9.42e-09 &  2.69e-16 &      \\ 
     &           &   10 &  9.41e-09 &  2.69e-16 &      \\ 
  53 &  5.20e+02 &   10 &           &           & iters  \\ 
 \hdashline 
     &           &    1 &  3.33e-08 &  4.66e-09 &      \\ 
     &           &    2 &  3.33e-08 &  9.51e-16 &      \\ 
     &           &    3 &  4.45e-08 &  9.50e-16 &      \\ 
     &           &    4 &  3.33e-08 &  1.27e-15 &      \\ 
     &           &    5 &  4.44e-08 &  9.50e-16 &      \\ 
     &           &    6 &  3.33e-08 &  1.27e-15 &      \\ 
     &           &    7 &  4.44e-08 &  9.50e-16 &      \\ 
     &           &    8 &  3.33e-08 &  1.27e-15 &      \\ 
     &           &    9 &  3.33e-08 &  9.51e-16 &      \\ 
     &           &   10 &  4.45e-08 &  9.50e-16 &      \\ 
  54 &  5.30e+02 &   10 &           &           & iters  \\ 
 \hdashline 
     &           &    1 &  1.61e-08 &  3.35e-09 &      \\ 
     &           &    2 &  1.61e-08 &  4.60e-16 &      \\ 
     &           &    3 &  1.61e-08 &  4.59e-16 &      \\ 
     &           &    4 &  1.60e-08 &  4.59e-16 &      \\ 
     &           &    5 &  6.17e-08 &  4.58e-16 &      \\ 
     &           &    6 &  1.61e-08 &  1.76e-15 &      \\ 
     &           &    7 &  1.61e-08 &  4.60e-16 &      \\ 
     &           &    8 &  1.61e-08 &  4.59e-16 &      \\ 
     &           &    9 &  1.60e-08 &  4.58e-16 &      \\ 
     &           &   10 &  6.17e-08 &  4.58e-16 &      \\ 
  55 &  5.40e+02 &   10 &           &           & iters  \\ 
 \hdashline 
     &           &    1 &  7.90e-09 &  2.46e-09 &      \\ 
     &           &    2 &  7.89e-09 &  2.26e-16 &      \\ 
     &           &    3 &  7.88e-09 &  2.25e-16 &      \\ 
     &           &    4 &  6.98e-08 &  2.25e-16 &      \\ 
     &           &    5 &  8.57e-08 &  1.99e-15 &      \\ 
     &           &    6 &  6.98e-08 &  2.44e-15 &      \\ 
     &           &    7 &  7.95e-09 &  1.99e-15 &      \\ 
     &           &    8 &  7.93e-09 &  2.27e-16 &      \\ 
     &           &    9 &  7.92e-09 &  2.26e-16 &      \\ 
     &           &   10 &  7.91e-09 &  2.26e-16 &      \\ 
  56 &  5.50e+02 &   10 &           &           & iters  \\ 
 \hdashline 
     &           &    1 &  4.50e-08 &  5.63e-09 &      \\ 
     &           &    2 &  3.27e-08 &  1.29e-15 &      \\ 
     &           &    3 &  3.27e-08 &  9.34e-16 &      \\ 
     &           &    4 &  4.51e-08 &  9.32e-16 &      \\ 
     &           &    5 &  3.27e-08 &  1.29e-15 &      \\ 
     &           &    6 &  4.50e-08 &  9.33e-16 &      \\ 
     &           &    7 &  3.27e-08 &  1.29e-15 &      \\ 
     &           &    8 &  4.50e-08 &  9.33e-16 &      \\ 
     &           &    9 &  3.27e-08 &  1.29e-15 &      \\ 
     &           &   10 &  3.27e-08 &  9.34e-16 &      \\ 
  57 &  5.60e+02 &   10 &           &           & iters  \\ 
 \hdashline 
     &           &    1 &  2.06e-08 &  4.77e-09 &      \\ 
     &           &    2 &  5.71e-08 &  5.88e-16 &      \\ 
     &           &    3 &  2.07e-08 &  1.63e-15 &      \\ 
     &           &    4 &  2.06e-08 &  5.90e-16 &      \\ 
     &           &    5 &  2.06e-08 &  5.89e-16 &      \\ 
     &           &    6 &  5.71e-08 &  5.88e-16 &      \\ 
     &           &    7 &  2.07e-08 &  1.63e-15 &      \\ 
     &           &    8 &  2.06e-08 &  5.90e-16 &      \\ 
     &           &    9 &  2.06e-08 &  5.89e-16 &      \\ 
     &           &   10 &  5.71e-08 &  5.88e-16 &      \\ 
  58 &  5.70e+02 &   10 &           &           & iters  \\ 
 \hdashline 
     &           &    1 &  5.07e-08 &  2.86e-09 &      \\ 
     &           &    2 &  2.70e-08 &  1.45e-15 &      \\ 
     &           &    3 &  2.70e-08 &  7.72e-16 &      \\ 
     &           &    4 &  5.07e-08 &  7.71e-16 &      \\ 
     &           &    5 &  2.70e-08 &  1.45e-15 &      \\ 
     &           &    6 &  2.70e-08 &  7.72e-16 &      \\ 
     &           &    7 &  5.07e-08 &  7.70e-16 &      \\ 
     &           &    8 &  2.70e-08 &  1.45e-15 &      \\ 
     &           &    9 &  2.70e-08 &  7.71e-16 &      \\ 
     &           &   10 &  5.07e-08 &  7.70e-16 &      \\ 
  59 &  5.80e+02 &   10 &           &           & iters  \\ 
 \hdashline 
     &           &    1 &  7.39e-08 &  1.14e-09 &      \\ 
     &           &    2 &  3.85e-09 &  2.11e-15 &      \\ 
     &           &    3 &  3.84e-09 &  1.10e-16 &      \\ 
     &           &    4 &  3.84e-09 &  1.10e-16 &      \\ 
     &           &    5 &  3.83e-09 &  1.10e-16 &      \\ 
     &           &    6 &  3.83e-09 &  1.09e-16 &      \\ 
     &           &    7 &  3.82e-09 &  1.09e-16 &      \\ 
     &           &    8 &  3.82e-09 &  1.09e-16 &      \\ 
     &           &    9 &  3.81e-09 &  1.09e-16 &      \\ 
     &           &   10 &  3.81e-09 &  1.09e-16 &      \\ 
  60 &  5.90e+02 &   10 &           &           & iters  \\ 
 \hdashline 
     &           &    1 &  2.33e-08 &  5.38e-09 &      \\ 
     &           &    2 &  2.33e-08 &  6.66e-16 &      \\ 
     &           &    3 &  5.44e-08 &  6.65e-16 &      \\ 
     &           &    4 &  2.34e-08 &  1.55e-15 &      \\ 
     &           &    5 &  2.33e-08 &  6.67e-16 &      \\ 
     &           &    6 &  5.44e-08 &  6.66e-16 &      \\ 
     &           &    7 &  2.34e-08 &  1.55e-15 &      \\ 
     &           &    8 &  2.33e-08 &  6.67e-16 &      \\ 
     &           &    9 &  5.44e-08 &  6.66e-16 &      \\ 
     &           &   10 &  2.34e-08 &  1.55e-15 &      \\ 
  61 &  6.00e+02 &   10 &           &           & iters  \\ 
 \hdashline 
     &           &    1 &  3.68e-09 &  8.04e-09 &      \\ 
     &           &    2 &  3.68e-09 &  1.05e-16 &      \\ 
     &           &    3 &  3.67e-09 &  1.05e-16 &      \\ 
     &           &    4 &  3.66e-09 &  1.05e-16 &      \\ 
     &           &    5 &  3.66e-09 &  1.05e-16 &      \\ 
     &           &    6 &  3.65e-09 &  1.04e-16 &      \\ 
     &           &    7 &  3.65e-09 &  1.04e-16 &      \\ 
     &           &    8 &  3.64e-09 &  1.04e-16 &      \\ 
     &           &    9 &  3.64e-09 &  1.04e-16 &      \\ 
     &           &   10 &  3.63e-09 &  1.04e-16 &      \\ 
  62 &  6.10e+02 &   10 &           &           & iters  \\ 
 \hdashline 
     &           &    1 &  7.86e-08 &  5.61e-12 &      \\ 
     &           &    2 &  7.69e-08 &  2.24e-15 &      \\ 
     &           &    3 &  7.86e-08 &  2.19e-15 &      \\ 
     &           &    4 &  7.69e-08 &  2.24e-15 &      \\ 
     &           &    5 &  7.86e-08 &  2.19e-15 &      \\ 
     &           &    6 &  7.69e-08 &  2.24e-15 &      \\ 
     &           &    7 &  8.96e-10 &  2.19e-15 &      \\ 
     &           &    8 &  8.95e-10 &  2.56e-17 &      \\ 
     &           &    9 &  8.94e-10 &  2.55e-17 &      \\ 
     &           &   10 &  8.93e-10 &  2.55e-17 &      \\ 
  63 &  6.20e+02 &   10 &           &           & iters  \\ 
 \hdashline 
     &           &    1 &  4.97e-08 &  5.71e-09 &      \\ 
     &           &    2 &  2.81e-08 &  1.42e-15 &      \\ 
     &           &    3 &  2.80e-08 &  8.01e-16 &      \\ 
     &           &    4 &  4.97e-08 &  8.00e-16 &      \\ 
     &           &    5 &  2.81e-08 &  1.42e-15 &      \\ 
     &           &    6 &  2.80e-08 &  8.01e-16 &      \\ 
     &           &    7 &  4.97e-08 &  8.00e-16 &      \\ 
     &           &    8 &  2.81e-08 &  1.42e-15 &      \\ 
     &           &    9 &  2.80e-08 &  8.01e-16 &      \\ 
     &           &   10 &  4.97e-08 &  8.00e-16 &      \\ 
  64 &  6.30e+02 &   10 &           &           & iters  \\ 
 \hdashline 
     &           &    1 &  3.35e-08 &  7.79e-10 &      \\ 
     &           &    2 &  4.43e-08 &  9.55e-16 &      \\ 
     &           &    3 &  3.35e-08 &  1.26e-15 &      \\ 
     &           &    4 &  4.43e-08 &  9.55e-16 &      \\ 
     &           &    5 &  3.35e-08 &  1.26e-15 &      \\ 
     &           &    6 &  4.43e-08 &  9.56e-16 &      \\ 
     &           &    7 &  3.35e-08 &  1.26e-15 &      \\ 
     &           &    8 &  4.42e-08 &  9.56e-16 &      \\ 
     &           &    9 &  3.35e-08 &  1.26e-15 &      \\ 
     &           &   10 &  3.35e-08 &  9.57e-16 &      \\ 
  65 &  6.40e+02 &   10 &           &           & iters  \\ 
 \hdashline 
     &           &    1 &  4.29e-08 &  7.32e-10 &      \\ 
     &           &    2 &  4.02e-09 &  1.23e-15 &      \\ 
     &           &    3 &  4.02e-09 &  1.15e-16 &      \\ 
     &           &    4 &  4.01e-09 &  1.15e-16 &      \\ 
     &           &    5 &  4.00e-09 &  1.14e-16 &      \\ 
     &           &    6 &  4.00e-09 &  1.14e-16 &      \\ 
     &           &    7 &  3.99e-09 &  1.14e-16 &      \\ 
     &           &    8 &  7.37e-08 &  1.14e-16 &      \\ 
     &           &    9 &  4.29e-08 &  2.10e-15 &      \\ 
     &           &   10 &  4.03e-09 &  1.23e-15 &      \\ 
  66 &  6.50e+02 &   10 &           &           & iters  \\ 
 \hdashline 
     &           &    1 &  3.09e-08 &  8.59e-09 &      \\ 
     &           &    2 &  4.69e-08 &  8.81e-16 &      \\ 
     &           &    3 &  3.09e-08 &  1.34e-15 &      \\ 
     &           &    4 &  4.69e-08 &  8.81e-16 &      \\ 
     &           &    5 &  3.09e-08 &  1.34e-15 &      \\ 
     &           &    6 &  3.09e-08 &  8.82e-16 &      \\ 
     &           &    7 &  4.69e-08 &  8.81e-16 &      \\ 
     &           &    8 &  3.09e-08 &  1.34e-15 &      \\ 
     &           &    9 &  4.69e-08 &  8.81e-16 &      \\ 
     &           &   10 &  3.09e-08 &  1.34e-15 &      \\ 
  67 &  6.60e+02 &   10 &           &           & iters  \\ 
 \hdashline 
     &           &    1 &  2.34e-08 &  7.77e-09 &      \\ 
     &           &    2 &  5.43e-08 &  6.68e-16 &      \\ 
     &           &    3 &  2.34e-08 &  1.55e-15 &      \\ 
     &           &    4 &  2.34e-08 &  6.69e-16 &      \\ 
     &           &    5 &  2.34e-08 &  6.68e-16 &      \\ 
     &           &    6 &  5.43e-08 &  6.67e-16 &      \\ 
     &           &    7 &  2.34e-08 &  1.55e-15 &      \\ 
     &           &    8 &  2.34e-08 &  6.68e-16 &      \\ 
     &           &    9 &  5.43e-08 &  6.68e-16 &      \\ 
     &           &   10 &  2.34e-08 &  1.55e-15 &      \\ 
  68 &  6.70e+02 &   10 &           &           & iters  \\ 
 \hdashline 
     &           &    1 &  1.89e-08 &  1.23e-09 &      \\ 
     &           &    2 &  1.89e-08 &  5.41e-16 &      \\ 
     &           &    3 &  1.89e-08 &  5.40e-16 &      \\ 
     &           &    4 &  1.89e-08 &  5.39e-16 &      \\ 
     &           &    5 &  5.89e-08 &  5.38e-16 &      \\ 
     &           &    6 &  1.89e-08 &  1.68e-15 &      \\ 
     &           &    7 &  1.89e-08 &  5.40e-16 &      \\ 
     &           &    8 &  1.89e-08 &  5.39e-16 &      \\ 
     &           &    9 &  5.89e-08 &  5.38e-16 &      \\ 
     &           &   10 &  1.89e-08 &  1.68e-15 &      \\ 
  69 &  6.80e+02 &   10 &           &           & iters  \\ 
 \hdashline 
     &           &    1 &  9.08e-09 &  2.65e-09 &      \\ 
     &           &    2 &  6.86e-08 &  2.59e-16 &      \\ 
     &           &    3 &  8.68e-08 &  1.96e-15 &      \\ 
     &           &    4 &  6.87e-08 &  2.48e-15 &      \\ 
     &           &    5 &  9.13e-09 &  1.96e-15 &      \\ 
     &           &    6 &  9.12e-09 &  2.61e-16 &      \\ 
     &           &    7 &  9.11e-09 &  2.60e-16 &      \\ 
     &           &    8 &  9.10e-09 &  2.60e-16 &      \\ 
     &           &    9 &  9.08e-09 &  2.60e-16 &      \\ 
     &           &   10 &  9.07e-09 &  2.59e-16 &      \\ 
  70 &  6.90e+02 &   10 &           &           & iters  \\ 
 \hdashline 
     &           &    1 &  4.31e-08 &  5.62e-09 &      \\ 
     &           &    2 &  3.47e-08 &  1.23e-15 &      \\ 
     &           &    3 &  3.46e-08 &  9.90e-16 &      \\ 
     &           &    4 &  4.31e-08 &  9.89e-16 &      \\ 
     &           &    5 &  3.47e-08 &  1.23e-15 &      \\ 
     &           &    6 &  4.31e-08 &  9.89e-16 &      \\ 
     &           &    7 &  3.47e-08 &  1.23e-15 &      \\ 
     &           &    8 &  4.31e-08 &  9.90e-16 &      \\ 
     &           &    9 &  3.47e-08 &  1.23e-15 &      \\ 
     &           &   10 &  4.31e-08 &  9.90e-16 &      \\ 
  71 &  7.00e+02 &   10 &           &           & iters  \\ 
 \hdashline 
     &           &    1 &  6.45e-09 &  3.00e-09 &      \\ 
     &           &    2 &  6.44e-09 &  1.84e-16 &      \\ 
     &           &    3 &  6.43e-09 &  1.84e-16 &      \\ 
     &           &    4 &  6.42e-09 &  1.84e-16 &      \\ 
     &           &    5 &  6.42e-09 &  1.83e-16 &      \\ 
     &           &    6 &  6.41e-09 &  1.83e-16 &      \\ 
     &           &    7 &  6.40e-09 &  1.83e-16 &      \\ 
     &           &    8 &  6.39e-09 &  1.83e-16 &      \\ 
     &           &    9 &  6.38e-09 &  1.82e-16 &      \\ 
     &           &   10 &  7.13e-08 &  1.82e-16 &      \\ 
  72 &  7.10e+02 &   10 &           &           & iters  \\ 
 \hdashline 
     &           &    1 &  7.74e-09 &  1.67e-09 &      \\ 
     &           &    2 &  7.73e-09 &  2.21e-16 &      \\ 
     &           &    3 &  7.72e-09 &  2.21e-16 &      \\ 
     &           &    4 &  7.71e-09 &  2.20e-16 &      \\ 
     &           &    5 &  7.00e-08 &  2.20e-16 &      \\ 
     &           &    6 &  7.79e-09 &  2.00e-15 &      \\ 
     &           &    7 &  7.78e-09 &  2.22e-16 &      \\ 
     &           &    8 &  7.77e-09 &  2.22e-16 &      \\ 
     &           &    9 &  7.76e-09 &  2.22e-16 &      \\ 
     &           &   10 &  7.75e-09 &  2.22e-16 &      \\ 
  73 &  7.20e+02 &   10 &           &           & iters  \\ 
 \hdashline 
     &           &    1 &  3.17e-08 &  1.56e-09 &      \\ 
     &           &    2 &  3.16e-08 &  9.04e-16 &      \\ 
     &           &    3 &  4.61e-08 &  9.03e-16 &      \\ 
     &           &    4 &  3.16e-08 &  1.32e-15 &      \\ 
     &           &    5 &  3.16e-08 &  9.03e-16 &      \\ 
     &           &    6 &  4.61e-08 &  9.02e-16 &      \\ 
     &           &    7 &  3.16e-08 &  1.32e-15 &      \\ 
     &           &    8 &  4.61e-08 &  9.02e-16 &      \\ 
     &           &    9 &  3.16e-08 &  1.32e-15 &      \\ 
     &           &   10 &  3.16e-08 &  9.03e-16 &      \\ 
  74 &  7.30e+02 &   10 &           &           & iters  \\ 
 \hdashline 
     &           &    1 &  4.75e-08 &  2.89e-09 &      \\ 
     &           &    2 &  4.74e-08 &  1.36e-15 &      \\ 
     &           &    3 &  3.03e-08 &  1.35e-15 &      \\ 
     &           &    4 &  4.74e-08 &  8.65e-16 &      \\ 
     &           &    5 &  3.03e-08 &  1.35e-15 &      \\ 
     &           &    6 &  3.03e-08 &  8.66e-16 &      \\ 
     &           &    7 &  4.74e-08 &  8.65e-16 &      \\ 
     &           &    8 &  3.03e-08 &  1.35e-15 &      \\ 
     &           &    9 &  4.74e-08 &  8.65e-16 &      \\ 
     &           &   10 &  3.03e-08 &  1.35e-15 &      \\ 
  75 &  7.40e+02 &   10 &           &           & iters  \\ 
 \hdashline 
     &           &    1 &  4.69e-08 &  5.96e-09 &      \\ 
     &           &    2 &  3.08e-08 &  1.34e-15 &      \\ 
     &           &    3 &  4.69e-08 &  8.80e-16 &      \\ 
     &           &    4 &  3.08e-08 &  1.34e-15 &      \\ 
     &           &    5 &  3.08e-08 &  8.80e-16 &      \\ 
     &           &    6 &  4.69e-08 &  8.79e-16 &      \\ 
     &           &    7 &  3.08e-08 &  1.34e-15 &      \\ 
     &           &    8 &  4.69e-08 &  8.80e-16 &      \\ 
     &           &    9 &  3.08e-08 &  1.34e-15 &      \\ 
     &           &   10 &  3.08e-08 &  8.80e-16 &      \\ 
  76 &  7.50e+02 &   10 &           &           & iters  \\ 
 \hdashline 
     &           &    1 &  4.31e-09 &  7.61e-10 &      \\ 
     &           &    2 &  4.31e-09 &  1.23e-16 &      \\ 
     &           &    3 &  7.34e-08 &  1.23e-16 &      \\ 
     &           &    4 &  4.32e-08 &  2.09e-15 &      \\ 
     &           &    5 &  4.34e-09 &  1.23e-15 &      \\ 
     &           &    6 &  4.34e-09 &  1.24e-16 &      \\ 
     &           &    7 &  4.33e-09 &  1.24e-16 &      \\ 
     &           &    8 &  4.32e-09 &  1.24e-16 &      \\ 
     &           &    9 &  4.32e-09 &  1.23e-16 &      \\ 
     &           &   10 &  4.31e-09 &  1.23e-16 &      \\ 
  77 &  7.60e+02 &   10 &           &           & iters  \\ 
 \hdashline 
     &           &    1 &  4.69e-08 &  4.67e-09 &      \\ 
     &           &    2 &  3.09e-08 &  1.34e-15 &      \\ 
     &           &    3 &  4.68e-08 &  8.81e-16 &      \\ 
     &           &    4 &  3.09e-08 &  1.34e-15 &      \\ 
     &           &    5 &  3.09e-08 &  8.82e-16 &      \\ 
     &           &    6 &  4.69e-08 &  8.81e-16 &      \\ 
     &           &    7 &  3.09e-08 &  1.34e-15 &      \\ 
     &           &    8 &  4.68e-08 &  8.81e-16 &      \\ 
     &           &    9 &  3.09e-08 &  1.34e-15 &      \\ 
     &           &   10 &  3.09e-08 &  8.82e-16 &      \\ 
  78 &  7.70e+02 &   10 &           &           & iters  \\ 
 \hdashline 
     &           &    1 &  1.84e-08 &  2.63e-09 &      \\ 
     &           &    2 &  1.84e-08 &  5.26e-16 &      \\ 
     &           &    3 &  5.93e-08 &  5.25e-16 &      \\ 
     &           &    4 &  1.84e-08 &  1.69e-15 &      \\ 
     &           &    5 &  1.84e-08 &  5.26e-16 &      \\ 
     &           &    6 &  1.84e-08 &  5.26e-16 &      \\ 
     &           &    7 &  5.93e-08 &  5.25e-16 &      \\ 
     &           &    8 &  1.85e-08 &  1.69e-15 &      \\ 
     &           &    9 &  1.84e-08 &  5.27e-16 &      \\ 
     &           &   10 &  1.84e-08 &  5.26e-16 &      \\ 
  79 &  7.80e+02 &   10 &           &           & iters  \\ 
 \hdashline 
     &           &    1 &  4.21e-08 &  8.55e-10 &      \\ 
     &           &    2 &  3.56e-08 &  1.20e-15 &      \\ 
     &           &    3 &  4.21e-08 &  1.02e-15 &      \\ 
     &           &    4 &  3.56e-08 &  1.20e-15 &      \\ 
     &           &    5 &  4.21e-08 &  1.02e-15 &      \\ 
     &           &    6 &  3.56e-08 &  1.20e-15 &      \\ 
     &           &    7 &  4.21e-08 &  1.02e-15 &      \\ 
     &           &    8 &  3.57e-08 &  1.20e-15 &      \\ 
     &           &    9 &  3.56e-08 &  1.02e-15 &      \\ 
     &           &   10 &  4.21e-08 &  1.02e-15 &      \\ 
  80 &  7.90e+02 &   10 &           &           & iters  \\ 
 \hdashline 
     &           &    1 &  6.00e-08 &  3.92e-09 &      \\ 
     &           &    2 &  1.78e-08 &  1.71e-15 &      \\ 
     &           &    3 &  1.78e-08 &  5.09e-16 &      \\ 
     &           &    4 &  5.99e-08 &  5.08e-16 &      \\ 
     &           &    5 &  1.79e-08 &  1.71e-15 &      \\ 
     &           &    6 &  1.78e-08 &  5.10e-16 &      \\ 
     &           &    7 &  1.78e-08 &  5.09e-16 &      \\ 
     &           &    8 &  5.99e-08 &  5.08e-16 &      \\ 
     &           &    9 &  1.79e-08 &  1.71e-15 &      \\ 
     &           &   10 &  1.78e-08 &  5.10e-16 &      \\ 
  81 &  8.00e+02 &   10 &           &           & iters  \\ 
 \hdashline 
     &           &    1 &  1.95e-08 &  6.11e-09 &      \\ 
     &           &    2 &  5.82e-08 &  5.57e-16 &      \\ 
     &           &    3 &  1.96e-08 &  1.66e-15 &      \\ 
     &           &    4 &  1.95e-08 &  5.58e-16 &      \\ 
     &           &    5 &  1.95e-08 &  5.57e-16 &      \\ 
     &           &    6 &  5.82e-08 &  5.57e-16 &      \\ 
     &           &    7 &  1.96e-08 &  1.66e-15 &      \\ 
     &           &    8 &  1.95e-08 &  5.58e-16 &      \\ 
     &           &    9 &  1.95e-08 &  5.57e-16 &      \\ 
     &           &   10 &  5.82e-08 &  5.57e-16 &      \\ 
  82 &  8.10e+02 &   10 &           &           & iters  \\ 
 \hdashline 
     &           &    1 &  1.07e-09 &  2.59e-09 &      \\ 
     &           &    2 &  1.06e-09 &  3.04e-17 &      \\ 
     &           &    3 &  1.06e-09 &  3.04e-17 &      \\ 
     &           &    4 &  1.06e-09 &  3.03e-17 &      \\ 
     &           &    5 &  1.06e-09 &  3.03e-17 &      \\ 
     &           &    6 &  1.06e-09 &  3.02e-17 &      \\ 
     &           &    7 &  1.06e-09 &  3.02e-17 &      \\ 
     &           &    8 &  1.06e-09 &  3.02e-17 &      \\ 
     &           &    9 &  1.05e-09 &  3.01e-17 &      \\ 
     &           &   10 &  1.05e-09 &  3.01e-17 &      \\ 
  83 &  8.20e+02 &   10 &           &           & iters  \\ 
 \hdashline 
     &           &    1 &  2.02e-08 &  6.77e-09 &      \\ 
     &           &    2 &  2.02e-08 &  5.77e-16 &      \\ 
     &           &    3 &  2.01e-08 &  5.76e-16 &      \\ 
     &           &    4 &  5.76e-08 &  5.75e-16 &      \\ 
     &           &    5 &  2.02e-08 &  1.64e-15 &      \\ 
     &           &    6 &  2.02e-08 &  5.76e-16 &      \\ 
     &           &    7 &  2.01e-08 &  5.76e-16 &      \\ 
     &           &    8 &  5.76e-08 &  5.75e-16 &      \\ 
     &           &    9 &  2.02e-08 &  1.64e-15 &      \\ 
     &           &   10 &  2.02e-08 &  5.76e-16 &      \\ 
  84 &  8.30e+02 &   10 &           &           & iters  \\ 
 \hdashline 
     &           &    1 &  3.08e-08 &  1.64e-08 &      \\ 
     &           &    2 &  4.69e-08 &  8.80e-16 &      \\ 
     &           &    3 &  3.09e-08 &  1.34e-15 &      \\ 
     &           &    4 &  3.08e-08 &  8.80e-16 &      \\ 
     &           &    5 &  4.69e-08 &  8.79e-16 &      \\ 
     &           &    6 &  3.08e-08 &  1.34e-15 &      \\ 
     &           &    7 &  4.69e-08 &  8.80e-16 &      \\ 
     &           &    8 &  3.09e-08 &  1.34e-15 &      \\ 
     &           &    9 &  3.08e-08 &  8.81e-16 &      \\ 
     &           &   10 &  4.69e-08 &  8.79e-16 &      \\ 
  85 &  8.40e+02 &   10 &           &           & iters  \\ 
 \hdashline 
     &           &    1 &  9.10e-09 &  1.67e-08 &      \\ 
     &           &    2 &  9.09e-09 &  2.60e-16 &      \\ 
     &           &    3 &  9.08e-09 &  2.59e-16 &      \\ 
     &           &    4 &  9.06e-09 &  2.59e-16 &      \\ 
     &           &    5 &  9.05e-09 &  2.59e-16 &      \\ 
     &           &    6 &  9.04e-09 &  2.58e-16 &      \\ 
     &           &    7 &  6.87e-08 &  2.58e-16 &      \\ 
     &           &    8 &  4.80e-08 &  1.96e-15 &      \\ 
     &           &    9 &  9.06e-09 &  1.37e-15 &      \\ 
     &           &   10 &  9.04e-09 &  2.58e-16 &      \\ 
  86 &  8.50e+02 &   10 &           &           & iters  \\ 
 \hdashline 
     &           &    1 &  1.40e-08 &  3.32e-10 &      \\ 
     &           &    2 &  1.40e-08 &  4.01e-16 &      \\ 
     &           &    3 &  1.40e-08 &  4.00e-16 &      \\ 
     &           &    4 &  1.40e-08 &  4.00e-16 &      \\ 
     &           &    5 &  6.37e-08 &  3.99e-16 &      \\ 
     &           &    6 &  1.40e-08 &  1.82e-15 &      \\ 
     &           &    7 &  1.40e-08 &  4.01e-16 &      \\ 
     &           &    8 &  1.40e-08 &  4.00e-16 &      \\ 
     &           &    9 &  1.40e-08 &  4.00e-16 &      \\ 
     &           &   10 &  1.40e-08 &  3.99e-16 &      \\ 
  87 &  8.60e+02 &   10 &           &           & iters  \\ 
 \hdashline 
     &           &    1 &  6.23e-08 &  1.74e-08 &      \\ 
     &           &    2 &  1.55e-08 &  1.78e-15 &      \\ 
     &           &    3 &  1.54e-08 &  4.41e-16 &      \\ 
     &           &    4 &  1.54e-08 &  4.41e-16 &      \\ 
     &           &    5 &  1.54e-08 &  4.40e-16 &      \\ 
     &           &    6 &  6.23e-08 &  4.39e-16 &      \\ 
     &           &    7 &  1.55e-08 &  1.78e-15 &      \\ 
     &           &    8 &  1.54e-08 &  4.41e-16 &      \\ 
     &           &    9 &  1.54e-08 &  4.41e-16 &      \\ 
     &           &   10 &  1.54e-08 &  4.40e-16 &      \\ 
  88 &  8.70e+02 &   10 &           &           & iters  \\ 
 \hdashline 
     &           &    1 &  1.81e-08 &  1.58e-08 &      \\ 
     &           &    2 &  5.96e-08 &  5.16e-16 &      \\ 
     &           &    3 &  1.81e-08 &  1.70e-15 &      \\ 
     &           &    4 &  1.81e-08 &  5.18e-16 &      \\ 
     &           &    5 &  1.81e-08 &  5.17e-16 &      \\ 
     &           &    6 &  5.96e-08 &  5.16e-16 &      \\ 
     &           &    7 &  1.81e-08 &  1.70e-15 &      \\ 
     &           &    8 &  1.81e-08 &  5.18e-16 &      \\ 
     &           &    9 &  1.81e-08 &  5.17e-16 &      \\ 
     &           &   10 &  1.81e-08 &  5.16e-16 &      \\ 
  89 &  8.80e+02 &   10 &           &           & iters  \\ 
 \hdashline 
     &           &    1 &  9.04e-08 &  1.50e-09 &      \\ 
     &           &    2 &  1.26e-08 &  2.58e-15 &      \\ 
     &           &    3 &  6.51e-08 &  3.60e-16 &      \\ 
     &           &    4 &  1.27e-08 &  1.86e-15 &      \\ 
     &           &    5 &  1.27e-08 &  3.62e-16 &      \\ 
     &           &    6 &  1.26e-08 &  3.61e-16 &      \\ 
     &           &    7 &  1.26e-08 &  3.61e-16 &      \\ 
     &           &    8 &  1.26e-08 &  3.60e-16 &      \\ 
     &           &    9 &  6.51e-08 &  3.60e-16 &      \\ 
     &           &   10 &  1.27e-08 &  1.86e-15 &      \\ 
  90 &  8.90e+02 &   10 &           &           & iters  \\ 
 \hdashline 
     &           &    1 &  1.64e-08 &  1.22e-08 &      \\ 
     &           &    2 &  1.64e-08 &  4.68e-16 &      \\ 
     &           &    3 &  6.13e-08 &  4.67e-16 &      \\ 
     &           &    4 &  1.64e-08 &  1.75e-15 &      \\ 
     &           &    5 &  1.64e-08 &  4.69e-16 &      \\ 
     &           &    6 &  1.64e-08 &  4.69e-16 &      \\ 
     &           &    7 &  1.64e-08 &  4.68e-16 &      \\ 
     &           &    8 &  6.13e-08 &  4.67e-16 &      \\ 
     &           &    9 &  1.64e-08 &  1.75e-15 &      \\ 
     &           &   10 &  1.64e-08 &  4.69e-16 &      \\ 
  91 &  9.00e+02 &   10 &           &           & iters  \\ 
 \hdashline 
     &           &    1 &  2.61e-08 &  4.61e-09 &      \\ 
     &           &    2 &  5.16e-08 &  7.46e-16 &      \\ 
     &           &    3 &  2.62e-08 &  1.47e-15 &      \\ 
     &           &    4 &  2.61e-08 &  7.47e-16 &      \\ 
     &           &    5 &  5.16e-08 &  7.46e-16 &      \\ 
     &           &    6 &  2.62e-08 &  1.47e-15 &      \\ 
     &           &    7 &  2.61e-08 &  7.47e-16 &      \\ 
     &           &    8 &  5.16e-08 &  7.46e-16 &      \\ 
     &           &    9 &  2.62e-08 &  1.47e-15 &      \\ 
     &           &   10 &  2.61e-08 &  7.47e-16 &      \\ 
  92 &  9.10e+02 &   10 &           &           & iters  \\ 
 \hdashline 
     &           &    1 &  1.31e-08 &  1.82e-08 &      \\ 
     &           &    2 &  6.46e-08 &  3.75e-16 &      \\ 
     &           &    3 &  1.32e-08 &  1.84e-15 &      \\ 
     &           &    4 &  1.32e-08 &  3.77e-16 &      \\ 
     &           &    5 &  1.32e-08 &  3.76e-16 &      \\ 
     &           &    6 &  1.32e-08 &  3.76e-16 &      \\ 
     &           &    7 &  1.31e-08 &  3.75e-16 &      \\ 
     &           &    8 &  6.46e-08 &  3.75e-16 &      \\ 
     &           &    9 &  1.32e-08 &  1.84e-15 &      \\ 
     &           &   10 &  1.32e-08 &  3.77e-16 &      \\ 
  93 &  9.20e+02 &   10 &           &           & iters  \\ 
 \hdashline 
     &           &    1 &  3.00e-08 &  1.67e-08 &      \\ 
     &           &    2 &  4.77e-08 &  8.57e-16 &      \\ 
     &           &    3 &  3.01e-08 &  1.36e-15 &      \\ 
     &           &    4 &  3.00e-08 &  8.58e-16 &      \\ 
     &           &    5 &  4.77e-08 &  8.57e-16 &      \\ 
     &           &    6 &  3.00e-08 &  1.36e-15 &      \\ 
     &           &    7 &  4.77e-08 &  8.57e-16 &      \\ 
     &           &    8 &  3.01e-08 &  1.36e-15 &      \\ 
     &           &    9 &  3.00e-08 &  8.58e-16 &      \\ 
     &           &   10 &  4.77e-08 &  8.57e-16 &      \\ 
  94 &  9.30e+02 &   10 &           &           & iters  \\ 
 \hdashline 
     &           &    1 &  2.25e-08 &  2.86e-09 &      \\ 
     &           &    2 &  2.24e-08 &  6.41e-16 &      \\ 
     &           &    3 &  5.53e-08 &  6.40e-16 &      \\ 
     &           &    4 &  2.25e-08 &  1.58e-15 &      \\ 
     &           &    5 &  2.24e-08 &  6.41e-16 &      \\ 
     &           &    6 &  2.24e-08 &  6.41e-16 &      \\ 
     &           &    7 &  5.53e-08 &  6.40e-16 &      \\ 
     &           &    8 &  2.25e-08 &  1.58e-15 &      \\ 
     &           &    9 &  2.24e-08 &  6.41e-16 &      \\ 
     &           &   10 &  5.53e-08 &  6.40e-16 &      \\ 
  95 &  9.40e+02 &   10 &           &           & iters  \\ 
 \hdashline 
     &           &    1 &  7.31e-09 &  8.21e-10 &      \\ 
     &           &    2 &  7.30e-09 &  2.09e-16 &      \\ 
     &           &    3 &  7.29e-09 &  2.08e-16 &      \\ 
     &           &    4 &  7.28e-09 &  2.08e-16 &      \\ 
     &           &    5 &  7.26e-09 &  2.08e-16 &      \\ 
     &           &    6 &  7.04e-08 &  2.07e-16 &      \\ 
     &           &    7 &  8.50e-08 &  2.01e-15 &      \\ 
     &           &    8 &  7.05e-08 &  2.43e-15 &      \\ 
     &           &    9 &  7.33e-09 &  2.01e-15 &      \\ 
     &           &   10 &  7.32e-09 &  2.09e-16 &      \\ 
  96 &  9.50e+02 &   10 &           &           & iters  \\ 
 \hdashline 
     &           &    1 &  2.87e-08 &  1.13e-08 &      \\ 
     &           &    2 &  4.90e-08 &  8.20e-16 &      \\ 
     &           &    3 &  2.88e-08 &  1.40e-15 &      \\ 
     &           &    4 &  4.90e-08 &  8.21e-16 &      \\ 
     &           &    5 &  2.88e-08 &  1.40e-15 &      \\ 
     &           &    6 &  2.88e-08 &  8.22e-16 &      \\ 
     &           &    7 &  4.90e-08 &  8.21e-16 &      \\ 
     &           &    8 &  2.88e-08 &  1.40e-15 &      \\ 
     &           &    9 &  2.87e-08 &  8.22e-16 &      \\ 
     &           &   10 &  4.90e-08 &  8.20e-16 &      \\ 
  97 &  9.60e+02 &   10 &           &           & iters  \\ 
 \hdashline 
     &           &    1 &  2.00e-08 &  2.44e-08 &      \\ 
     &           &    2 &  2.00e-08 &  5.71e-16 &      \\ 
     &           &    3 &  5.77e-08 &  5.70e-16 &      \\ 
     &           &    4 &  2.00e-08 &  1.65e-15 &      \\ 
     &           &    5 &  2.00e-08 &  5.72e-16 &      \\ 
     &           &    6 &  2.00e-08 &  5.71e-16 &      \\ 
     &           &    7 &  5.77e-08 &  5.70e-16 &      \\ 
     &           &    8 &  2.00e-08 &  1.65e-15 &      \\ 
     &           &    9 &  2.00e-08 &  5.72e-16 &      \\ 
     &           &   10 &  2.00e-08 &  5.71e-16 &      \\ 
  98 &  9.70e+02 &   10 &           &           & iters  \\ 
 \hdashline 
     &           &    1 &  5.82e-09 &  1.72e-08 &      \\ 
     &           &    2 &  5.81e-09 &  1.66e-16 &      \\ 
     &           &    3 &  5.81e-09 &  1.66e-16 &      \\ 
     &           &    4 &  5.80e-09 &  1.66e-16 &      \\ 
     &           &    5 &  5.79e-09 &  1.65e-16 &      \\ 
     &           &    6 &  5.78e-09 &  1.65e-16 &      \\ 
     &           &    7 &  5.77e-09 &  1.65e-16 &      \\ 
     &           &    8 &  5.76e-09 &  1.65e-16 &      \\ 
     &           &    9 &  5.76e-09 &  1.65e-16 &      \\ 
     &           &   10 &  3.31e-08 &  1.64e-16 &      \\ 
  99 &  9.80e+02 &   10 &           &           & iters  \\ 
 \hdashline 
     &           &    1 &  4.39e-08 &  6.79e-09 &      \\ 
     &           &    2 &  3.38e-08 &  1.25e-15 &      \\ 
     &           &    3 &  4.39e-08 &  9.65e-16 &      \\ 
     &           &    4 &  3.38e-08 &  1.25e-15 &      \\ 
     &           &    5 &  3.38e-08 &  9.65e-16 &      \\ 
     &           &    6 &  4.40e-08 &  9.64e-16 &      \\ 
     &           &    7 &  3.38e-08 &  1.25e-15 &      \\ 
     &           &    8 &  4.39e-08 &  9.64e-16 &      \\ 
     &           &    9 &  3.38e-08 &  1.25e-15 &      \\ 
     &           &   10 &  4.39e-08 &  9.65e-16 &      \\ 
 100 &  9.90e+02 &   10 &           &           & iters  \\ 
 \hdashline 
     &           &    1 &  1.71e-08 &  4.86e-09 &      \\ 
     &           &    2 &  1.71e-08 &  4.87e-16 &      \\ 
     &           &    3 &  1.70e-08 &  4.87e-16 &      \\ 
     &           &    4 &  1.70e-08 &  4.86e-16 &      \\ 
     &           &    5 &  6.07e-08 &  4.85e-16 &      \\ 
     &           &    6 &  1.71e-08 &  1.73e-15 &      \\ 
     &           &    7 &  1.70e-08 &  4.87e-16 &      \\ 
     &           &    8 &  1.70e-08 &  4.86e-16 &      \\ 
     &           &    9 &  1.70e-08 &  4.86e-16 &      \\ 
     &           &   10 &  6.07e-08 &  4.85e-16 &      \\ 
 101 &  1.00e+03 &   10 &           &           & iters  \\ 
 \hdashline 
     &           &    1 &  5.08e-08 &  1.38e-08 &      \\ 
     &           &    2 &  2.70e-08 &  1.45e-15 &      \\ 
     &           &    3 &  2.69e-08 &  7.70e-16 &      \\ 
     &           &    4 &  5.08e-08 &  7.69e-16 &      \\ 
     &           &    5 &  2.70e-08 &  1.45e-15 &      \\ 
     &           &    6 &  2.69e-08 &  7.70e-16 &      \\ 
     &           &    7 &  5.08e-08 &  7.69e-16 &      \\ 
     &           &    8 &  2.70e-08 &  1.45e-15 &      \\ 
     &           &    9 &  2.69e-08 &  7.70e-16 &      \\ 
     &           &   10 &  5.08e-08 &  7.69e-16 &      \\ 
 102 &  1.01e+03 &   10 &           &           & iters  \\ 
 \hdashline 
     &           &    1 &  1.52e-08 &  9.17e-09 &      \\ 
     &           &    2 &  1.52e-08 &  4.33e-16 &      \\ 
     &           &    3 &  2.37e-08 &  4.32e-16 &      \\ 
     &           &    4 &  1.52e-08 &  6.77e-16 &      \\ 
     &           &    5 &  1.51e-08 &  4.33e-16 &      \\ 
     &           &    6 &  2.37e-08 &  4.32e-16 &      \\ 
     &           &    7 &  1.52e-08 &  6.77e-16 &      \\ 
     &           &    8 &  2.37e-08 &  4.33e-16 &      \\ 
     &           &    9 &  1.52e-08 &  6.77e-16 &      \\ 
     &           &   10 &  1.51e-08 &  4.33e-16 &      \\ 
 103 &  1.02e+03 &   10 &           &           & iters  \\ 
 \hdashline 
     &           &    1 &  6.58e-08 &  2.81e-09 &      \\ 
     &           &    2 &  1.20e-08 &  1.88e-15 &      \\ 
     &           &    3 &  1.20e-08 &  3.42e-16 &      \\ 
     &           &    4 &  1.19e-08 &  3.41e-16 &      \\ 
     &           &    5 &  1.19e-08 &  3.41e-16 &      \\ 
     &           &    6 &  1.19e-08 &  3.40e-16 &      \\ 
     &           &    7 &  6.58e-08 &  3.40e-16 &      \\ 
     &           &    8 &  8.97e-08 &  1.88e-15 &      \\ 
     &           &    9 &  6.58e-08 &  2.56e-15 &      \\ 
     &           &   10 &  1.20e-08 &  1.88e-15 &      \\ 
 104 &  1.03e+03 &   10 &           &           & iters  \\ 
 \hdashline 
     &           &    1 &  5.64e-08 &  1.23e-09 &      \\ 
     &           &    2 &  2.14e-08 &  1.61e-15 &      \\ 
     &           &    3 &  2.13e-08 &  6.10e-16 &      \\ 
     &           &    4 &  2.13e-08 &  6.09e-16 &      \\ 
     &           &    5 &  5.64e-08 &  6.08e-16 &      \\ 
     &           &    6 &  2.14e-08 &  1.61e-15 &      \\ 
     &           &    7 &  2.13e-08 &  6.10e-16 &      \\ 
     &           &    8 &  2.13e-08 &  6.09e-16 &      \\ 
     &           &    9 &  5.64e-08 &  6.08e-16 &      \\ 
     &           &   10 &  2.13e-08 &  1.61e-15 &      \\ 
 105 &  1.04e+03 &   10 &           &           & iters  \\ 
 \hdashline 
     &           &    1 &  2.48e-08 &  4.40e-10 &      \\ 
     &           &    2 &  2.48e-08 &  7.08e-16 &      \\ 
     &           &    3 &  5.30e-08 &  7.07e-16 &      \\ 
     &           &    4 &  2.48e-08 &  1.51e-15 &      \\ 
     &           &    5 &  2.48e-08 &  7.08e-16 &      \\ 
     &           &    6 &  5.29e-08 &  7.07e-16 &      \\ 
     &           &    7 &  2.48e-08 &  1.51e-15 &      \\ 
     &           &    8 &  2.48e-08 &  7.08e-16 &      \\ 
     &           &    9 &  2.47e-08 &  7.07e-16 &      \\ 
     &           &   10 &  5.30e-08 &  7.06e-16 &      \\ 
 106 &  1.05e+03 &   10 &           &           & iters  \\ 
 \hdashline 
     &           &    1 &  3.17e-08 &  7.80e-09 &      \\ 
     &           &    2 &  4.60e-08 &  9.06e-16 &      \\ 
     &           &    3 &  3.18e-08 &  1.31e-15 &      \\ 
     &           &    4 &  3.17e-08 &  9.06e-16 &      \\ 
     &           &    5 &  4.60e-08 &  9.05e-16 &      \\ 
     &           &    6 &  3.17e-08 &  1.31e-15 &      \\ 
     &           &    7 &  4.60e-08 &  9.06e-16 &      \\ 
     &           &    8 &  3.18e-08 &  1.31e-15 &      \\ 
     &           &    9 &  3.17e-08 &  9.06e-16 &      \\ 
     &           &   10 &  4.60e-08 &  9.05e-16 &      \\ 
 107 &  1.06e+03 &   10 &           &           & iters  \\ 
 \hdashline 
     &           &    1 &  5.94e-10 &  1.17e-08 &      \\ 
     &           &    2 &  5.93e-10 &  1.70e-17 &      \\ 
     &           &    3 &  5.92e-10 &  1.69e-17 &      \\ 
     &           &    4 &  5.91e-10 &  1.69e-17 &      \\ 
     &           &    5 &  5.91e-10 &  1.69e-17 &      \\ 
     &           &    6 &  5.90e-10 &  1.69e-17 &      \\ 
     &           &    7 &  5.89e-10 &  1.68e-17 &      \\ 
     &           &    8 &  5.88e-10 &  1.68e-17 &      \\ 
     &           &    9 &  5.87e-10 &  1.68e-17 &      \\ 
     &           &   10 &  5.86e-10 &  1.68e-17 &      \\ 
 108 &  1.07e+03 &   10 &           &           & iters  \\ 
 \hdashline 
     &           &    1 &  6.20e-09 &  3.44e-09 &      \\ 
     &           &    2 &  6.19e-09 &  1.77e-16 &      \\ 
     &           &    3 &  6.18e-09 &  1.77e-16 &      \\ 
     &           &    4 &  6.17e-09 &  1.76e-16 &      \\ 
     &           &    5 &  6.17e-09 &  1.76e-16 &      \\ 
     &           &    6 &  6.16e-09 &  1.76e-16 &      \\ 
     &           &    7 &  6.15e-09 &  1.76e-16 &      \\ 
     &           &    8 &  6.14e-09 &  1.75e-16 &      \\ 
     &           &    9 &  6.13e-09 &  1.75e-16 &      \\ 
     &           &   10 &  7.16e-08 &  1.75e-16 &      \\ 
 109 &  1.08e+03 &   10 &           &           & iters  \\ 
 \hdashline 
     &           &    1 &  1.51e-08 &  1.43e-08 &      \\ 
     &           &    2 &  1.51e-08 &  4.32e-16 &      \\ 
     &           &    3 &  1.51e-08 &  4.31e-16 &      \\ 
     &           &    4 &  6.26e-08 &  4.31e-16 &      \\ 
     &           &    5 &  1.52e-08 &  1.79e-15 &      \\ 
     &           &    6 &  1.51e-08 &  4.32e-16 &      \\ 
     &           &    7 &  1.51e-08 &  4.32e-16 &      \\ 
     &           &    8 &  1.51e-08 &  4.31e-16 &      \\ 
     &           &    9 &  1.51e-08 &  4.31e-16 &      \\ 
     &           &   10 &  6.26e-08 &  4.30e-16 &      \\ 
 110 &  1.09e+03 &   10 &           &           & iters  \\ 
 \hdashline 
     &           &    1 &  2.80e-08 &  1.52e-08 &      \\ 
     &           &    2 &  1.09e-08 &  7.99e-16 &      \\ 
     &           &    3 &  1.09e-08 &  3.11e-16 &      \\ 
     &           &    4 &  1.09e-08 &  3.11e-16 &      \\ 
     &           &    5 &  2.80e-08 &  3.10e-16 &      \\ 
     &           &    6 &  1.09e-08 &  7.99e-16 &      \\ 
     &           &    7 &  1.09e-08 &  3.11e-16 &      \\ 
     &           &    8 &  2.80e-08 &  3.11e-16 &      \\ 
     &           &    9 &  1.09e-08 &  7.98e-16 &      \\ 
     &           &   10 &  1.09e-08 &  3.11e-16 &      \\ 
 111 &  1.10e+03 &   10 &           &           & iters  \\ 
 \hdashline 
     &           &    1 &  1.54e-08 &  5.49e-09 &      \\ 
     &           &    2 &  6.24e-08 &  4.38e-16 &      \\ 
     &           &    3 &  1.54e-08 &  1.78e-15 &      \\ 
     &           &    4 &  1.54e-08 &  4.40e-16 &      \\ 
     &           &    5 &  1.54e-08 &  4.39e-16 &      \\ 
     &           &    6 &  1.54e-08 &  4.39e-16 &      \\ 
     &           &    7 &  6.24e-08 &  4.38e-16 &      \\ 
     &           &    8 &  1.54e-08 &  1.78e-15 &      \\ 
     &           &    9 &  1.54e-08 &  4.40e-16 &      \\ 
     &           &   10 &  1.54e-08 &  4.39e-16 &      \\ 
 112 &  1.11e+03 &   10 &           &           & iters  \\ 
 \hdashline 
     &           &    1 &  1.37e-08 &  1.18e-08 &      \\ 
     &           &    2 &  1.37e-08 &  3.92e-16 &      \\ 
     &           &    3 &  1.37e-08 &  3.92e-16 &      \\ 
     &           &    4 &  6.40e-08 &  3.91e-16 &      \\ 
     &           &    5 &  9.15e-08 &  1.83e-15 &      \\ 
     &           &    6 &  6.40e-08 &  2.61e-15 &      \\ 
     &           &    7 &  1.37e-08 &  1.83e-15 &      \\ 
     &           &    8 &  1.37e-08 &  3.92e-16 &      \\ 
     &           &    9 &  1.37e-08 &  3.92e-16 &      \\ 
     &           &   10 &  6.40e-08 &  3.91e-16 &      \\ 
 113 &  1.12e+03 &   10 &           &           & iters  \\ 
 \hdashline 
     &           &    1 &  8.55e-09 &  2.60e-09 &      \\ 
     &           &    2 &  8.54e-09 &  2.44e-16 &      \\ 
     &           &    3 &  8.53e-09 &  2.44e-16 &      \\ 
     &           &    4 &  8.52e-09 &  2.43e-16 &      \\ 
     &           &    5 &  8.50e-09 &  2.43e-16 &      \\ 
     &           &    6 &  8.49e-09 &  2.43e-16 &      \\ 
     &           &    7 &  8.48e-09 &  2.42e-16 &      \\ 
     &           &    8 &  8.47e-09 &  2.42e-16 &      \\ 
     &           &    9 &  8.45e-09 &  2.42e-16 &      \\ 
     &           &   10 &  8.44e-09 &  2.41e-16 &      \\ 
 114 &  1.13e+03 &   10 &           &           & iters  \\ 
 \hdashline 
     &           &    1 &  1.01e-07 &  7.29e-10 &      \\ 
     &           &    2 &  5.49e-08 &  2.87e-15 &      \\ 
     &           &    3 &  2.29e-08 &  1.57e-15 &      \\ 
     &           &    4 &  5.48e-08 &  6.53e-16 &      \\ 
     &           &    5 &  2.29e-08 &  1.57e-15 &      \\ 
     &           &    6 &  2.29e-08 &  6.54e-16 &      \\ 
     &           &    7 &  2.29e-08 &  6.53e-16 &      \\ 
     &           &    8 &  5.49e-08 &  6.52e-16 &      \\ 
     &           &    9 &  2.29e-08 &  1.57e-15 &      \\ 
     &           &   10 &  2.29e-08 &  6.54e-16 &      \\ 
 115 &  1.14e+03 &   10 &           &           & iters  \\ 
 \hdashline 
     &           &    1 &  8.20e-09 &  1.55e-09 &      \\ 
     &           &    2 &  8.19e-09 &  2.34e-16 &      \\ 
     &           &    3 &  8.18e-09 &  2.34e-16 &      \\ 
     &           &    4 &  8.17e-09 &  2.33e-16 &      \\ 
     &           &    5 &  8.16e-09 &  2.33e-16 &      \\ 
     &           &    6 &  8.14e-09 &  2.33e-16 &      \\ 
     &           &    7 &  8.13e-09 &  2.32e-16 &      \\ 
     &           &    8 &  6.96e-08 &  2.32e-16 &      \\ 
     &           &    9 &  8.59e-08 &  1.99e-15 &      \\ 
     &           &   10 &  6.96e-08 &  2.45e-15 &      \\ 
 116 &  1.15e+03 &   10 &           &           & iters  \\ 
 \hdashline 
     &           &    1 &  3.73e-09 &  5.10e-10 &      \\ 
     &           &    2 &  3.73e-09 &  1.07e-16 &      \\ 
     &           &    3 &  3.72e-09 &  1.06e-16 &      \\ 
     &           &    4 &  3.72e-09 &  1.06e-16 &      \\ 
     &           &    5 &  3.71e-09 &  1.06e-16 &      \\ 
     &           &    6 &  3.71e-09 &  1.06e-16 &      \\ 
     &           &    7 &  3.70e-09 &  1.06e-16 &      \\ 
     &           &    8 &  3.70e-09 &  1.06e-16 &      \\ 
     &           &    9 &  3.69e-09 &  1.06e-16 &      \\ 
     &           &   10 &  3.69e-09 &  1.05e-16 &      \\ 
 117 &  1.16e+03 &   10 &           &           & iters  \\ 
 \hdashline 
     &           &    1 &  3.26e-08 &  4.85e-09 &      \\ 
     &           &    2 &  4.51e-08 &  9.31e-16 &      \\ 
     &           &    3 &  3.26e-08 &  1.29e-15 &      \\ 
     &           &    4 &  4.51e-08 &  9.32e-16 &      \\ 
     &           &    5 &  3.27e-08 &  1.29e-15 &      \\ 
     &           &    6 &  3.26e-08 &  9.32e-16 &      \\ 
     &           &    7 &  4.51e-08 &  9.31e-16 &      \\ 
     &           &    8 &  3.26e-08 &  1.29e-15 &      \\ 
     &           &    9 &  4.51e-08 &  9.31e-16 &      \\ 
     &           &   10 &  3.26e-08 &  1.29e-15 &      \\ 
 118 &  1.17e+03 &   10 &           &           & iters  \\ 
 \hdashline 
     &           &    1 &  4.82e-08 &  1.79e-09 &      \\ 
     &           &    2 &  2.95e-08 &  1.38e-15 &      \\ 
     &           &    3 &  2.95e-08 &  8.42e-16 &      \\ 
     &           &    4 &  4.83e-08 &  8.41e-16 &      \\ 
     &           &    5 &  2.95e-08 &  1.38e-15 &      \\ 
     &           &    6 &  4.82e-08 &  8.42e-16 &      \\ 
     &           &    7 &  2.95e-08 &  1.38e-15 &      \\ 
     &           &    8 &  2.95e-08 &  8.43e-16 &      \\ 
     &           &    9 &  4.82e-08 &  8.41e-16 &      \\ 
     &           &   10 &  2.95e-08 &  1.38e-15 &      \\ 
 119 &  1.18e+03 &   10 &           &           & iters  \\ 
 \hdashline 
     &           &    1 &  4.89e-08 &  1.81e-08 &      \\ 
     &           &    2 &  2.88e-08 &  1.40e-15 &      \\ 
     &           &    3 &  2.88e-08 &  8.23e-16 &      \\ 
     &           &    4 &  4.89e-08 &  8.22e-16 &      \\ 
     &           &    5 &  2.88e-08 &  1.40e-15 &      \\ 
     &           &    6 &  2.88e-08 &  8.22e-16 &      \\ 
     &           &    7 &  4.90e-08 &  8.21e-16 &      \\ 
     &           &    8 &  2.88e-08 &  1.40e-15 &      \\ 
     &           &    9 &  2.88e-08 &  8.22e-16 &      \\ 
     &           &   10 &  4.90e-08 &  8.21e-16 &      \\ 
 120 &  1.19e+03 &   10 &           &           & iters  \\ 
 \hdashline 
     &           &    1 &  1.30e-08 &  1.52e-08 &      \\ 
     &           &    2 &  1.30e-08 &  3.72e-16 &      \\ 
     &           &    3 &  1.30e-08 &  3.72e-16 &      \\ 
     &           &    4 &  6.47e-08 &  3.71e-16 &      \\ 
     &           &    5 &  1.31e-08 &  1.85e-15 &      \\ 
     &           &    6 &  1.31e-08 &  3.73e-16 &      \\ 
     &           &    7 &  1.30e-08 &  3.73e-16 &      \\ 
     &           &    8 &  1.30e-08 &  3.72e-16 &      \\ 
     &           &    9 &  1.30e-08 &  3.72e-16 &      \\ 
     &           &   10 &  6.47e-08 &  3.71e-16 &      \\ 
 121 &  1.20e+03 &   10 &           &           & iters  \\ 
 \hdashline 
     &           &    1 &  4.03e-08 &  3.08e-09 &      \\ 
     &           &    2 &  1.44e-09 &  1.15e-15 &      \\ 
     &           &    3 &  1.44e-09 &  4.12e-17 &      \\ 
     &           &    4 &  1.44e-09 &  4.11e-17 &      \\ 
     &           &    5 &  1.44e-09 &  4.11e-17 &      \\ 
     &           &    6 &  1.44e-09 &  4.10e-17 &      \\ 
     &           &    7 &  1.43e-09 &  4.10e-17 &      \\ 
     &           &    8 &  1.43e-09 &  4.09e-17 &      \\ 
     &           &    9 &  1.43e-09 &  4.08e-17 &      \\ 
     &           &   10 &  1.43e-09 &  4.08e-17 &      \\ 
 122 &  1.21e+03 &   10 &           &           & iters  \\ 
 \hdashline 
     &           &    1 &  2.97e-09 &  5.84e-09 &      \\ 
     &           &    2 &  2.96e-09 &  8.47e-17 &      \\ 
     &           &    3 &  2.96e-09 &  8.46e-17 &      \\ 
     &           &    4 &  2.95e-09 &  8.44e-17 &      \\ 
     &           &    5 &  2.95e-09 &  8.43e-17 &      \\ 
     &           &    6 &  2.95e-09 &  8.42e-17 &      \\ 
     &           &    7 &  2.94e-09 &  8.41e-17 &      \\ 
     &           &    8 &  2.94e-09 &  8.40e-17 &      \\ 
     &           &    9 &  2.93e-09 &  8.38e-17 &      \\ 
     &           &   10 &  2.93e-09 &  8.37e-17 &      \\ 
 123 &  1.22e+03 &   10 &           &           & iters  \\ 
 \hdashline 
     &           &    1 &  4.42e-08 &  5.87e-10 &      \\ 
     &           &    2 &  3.36e-08 &  1.26e-15 &      \\ 
     &           &    3 &  4.42e-08 &  9.58e-16 &      \\ 
     &           &    4 &  3.36e-08 &  1.26e-15 &      \\ 
     &           &    5 &  4.42e-08 &  9.58e-16 &      \\ 
     &           &    6 &  3.36e-08 &  1.26e-15 &      \\ 
     &           &    7 &  3.35e-08 &  9.59e-16 &      \\ 
     &           &    8 &  4.42e-08 &  9.57e-16 &      \\ 
     &           &    9 &  3.36e-08 &  1.26e-15 &      \\ 
     &           &   10 &  4.42e-08 &  9.58e-16 &      \\ 
 124 &  1.23e+03 &   10 &           &           & iters  \\ 
 \hdashline 
     &           &    1 &  7.07e-08 &  4.50e-09 &      \\ 
     &           &    2 &  7.09e-09 &  2.02e-15 &      \\ 
     &           &    3 &  7.08e-09 &  2.02e-16 &      \\ 
     &           &    4 &  7.07e-09 &  2.02e-16 &      \\ 
     &           &    5 &  7.06e-09 &  2.02e-16 &      \\ 
     &           &    6 &  7.05e-09 &  2.02e-16 &      \\ 
     &           &    7 &  7.04e-09 &  2.01e-16 &      \\ 
     &           &    8 &  7.03e-09 &  2.01e-16 &      \\ 
     &           &    9 &  7.02e-09 &  2.01e-16 &      \\ 
     &           &   10 &  7.01e-09 &  2.00e-16 &      \\ 
 125 &  1.24e+03 &   10 &           &           & iters  \\ 
 \hdashline 
     &           &    1 &  5.12e-08 &  7.71e-09 &      \\ 
     &           &    2 &  2.66e-08 &  1.46e-15 &      \\ 
     &           &    3 &  2.65e-08 &  7.59e-16 &      \\ 
     &           &    4 &  5.12e-08 &  7.58e-16 &      \\ 
     &           &    5 &  2.66e-08 &  1.46e-15 &      \\ 
     &           &    6 &  2.65e-08 &  7.59e-16 &      \\ 
     &           &    7 &  5.12e-08 &  7.58e-16 &      \\ 
     &           &    8 &  2.66e-08 &  1.46e-15 &      \\ 
     &           &    9 &  5.11e-08 &  7.59e-16 &      \\ 
     &           &   10 &  2.66e-08 &  1.46e-15 &      \\ 
 126 &  1.25e+03 &   10 &           &           & iters  \\ 
 \hdashline 
     &           &    1 &  6.22e-08 &  2.99e-09 &      \\ 
     &           &    2 &  1.56e-08 &  1.77e-15 &      \\ 
     &           &    3 &  1.56e-08 &  4.45e-16 &      \\ 
     &           &    4 &  1.56e-08 &  4.45e-16 &      \\ 
     &           &    5 &  1.55e-08 &  4.44e-16 &      \\ 
     &           &    6 &  6.22e-08 &  4.43e-16 &      \\ 
     &           &    7 &  1.56e-08 &  1.77e-15 &      \\ 
     &           &    8 &  1.56e-08 &  4.45e-16 &      \\ 
     &           &    9 &  1.56e-08 &  4.45e-16 &      \\ 
     &           &   10 &  1.55e-08 &  4.44e-16 &      \\ 
 127 &  1.26e+03 &   10 &           &           & iters  \\ 
 \hdashline 
     &           &    1 &  9.66e-09 &  4.51e-09 &      \\ 
     &           &    2 &  9.65e-09 &  2.76e-16 &      \\ 
     &           &    3 &  9.63e-09 &  2.75e-16 &      \\ 
     &           &    4 &  9.62e-09 &  2.75e-16 &      \\ 
     &           &    5 &  6.81e-08 &  2.75e-16 &      \\ 
     &           &    6 &  9.70e-09 &  1.94e-15 &      \\ 
     &           &    7 &  9.69e-09 &  2.77e-16 &      \\ 
     &           &    8 &  9.67e-09 &  2.77e-16 &      \\ 
     &           &    9 &  9.66e-09 &  2.76e-16 &      \\ 
     &           &   10 &  9.65e-09 &  2.76e-16 &      \\ 
 128 &  1.27e+03 &   10 &           &           & iters  \\ 
 \hdashline 
     &           &    1 &  2.77e-08 &  7.28e-09 &      \\ 
     &           &    2 &  2.77e-08 &  7.91e-16 &      \\ 
     &           &    3 &  2.76e-08 &  7.90e-16 &      \\ 
     &           &    4 &  5.01e-08 &  7.88e-16 &      \\ 
     &           &    5 &  2.77e-08 &  1.43e-15 &      \\ 
     &           &    6 &  5.01e-08 &  7.89e-16 &      \\ 
     &           &    7 &  2.77e-08 &  1.43e-15 &      \\ 
     &           &    8 &  2.76e-08 &  7.90e-16 &      \\ 
     &           &    9 &  5.01e-08 &  7.89e-16 &      \\ 
     &           &   10 &  2.77e-08 &  1.43e-15 &      \\ 
 129 &  1.28e+03 &   10 &           &           & iters  \\ 
 \hdashline 
     &           &    1 &  2.44e-08 &  2.50e-09 &      \\ 
     &           &    2 &  5.34e-08 &  6.95e-16 &      \\ 
     &           &    3 &  2.44e-08 &  1.52e-15 &      \\ 
     &           &    4 &  2.44e-08 &  6.96e-16 &      \\ 
     &           &    5 &  5.34e-08 &  6.95e-16 &      \\ 
     &           &    6 &  2.44e-08 &  1.52e-15 &      \\ 
     &           &    7 &  2.44e-08 &  6.97e-16 &      \\ 
     &           &    8 &  5.34e-08 &  6.96e-16 &      \\ 
     &           &    9 &  2.44e-08 &  1.52e-15 &      \\ 
     &           &   10 &  2.44e-08 &  6.97e-16 &      \\ 
 130 &  1.29e+03 &   10 &           &           & iters  \\ 
 \hdashline 
     &           &    1 &  1.61e-08 &  4.39e-09 &      \\ 
     &           &    2 &  1.60e-08 &  4.58e-16 &      \\ 
     &           &    3 &  1.60e-08 &  4.58e-16 &      \\ 
     &           &    4 &  6.17e-08 &  4.57e-16 &      \\ 
     &           &    5 &  1.61e-08 &  1.76e-15 &      \\ 
     &           &    6 &  1.61e-08 &  4.59e-16 &      \\ 
     &           &    7 &  1.60e-08 &  4.58e-16 &      \\ 
     &           &    8 &  1.60e-08 &  4.58e-16 &      \\ 
     &           &    9 &  6.17e-08 &  4.57e-16 &      \\ 
     &           &   10 &  1.61e-08 &  1.76e-15 &      \\ 
 131 &  1.30e+03 &   10 &           &           & iters  \\ 
 \hdashline 
     &           &    1 &  6.39e-09 &  9.40e-09 &      \\ 
     &           &    2 &  6.38e-09 &  1.82e-16 &      \\ 
     &           &    3 &  6.37e-09 &  1.82e-16 &      \\ 
     &           &    4 &  6.36e-09 &  1.82e-16 &      \\ 
     &           &    5 &  6.35e-09 &  1.82e-16 &      \\ 
     &           &    6 &  6.34e-09 &  1.81e-16 &      \\ 
     &           &    7 &  6.33e-09 &  1.81e-16 &      \\ 
     &           &    8 &  6.32e-09 &  1.81e-16 &      \\ 
     &           &    9 &  6.31e-09 &  1.80e-16 &      \\ 
     &           &   10 &  7.14e-08 &  1.80e-16 &      \\ 
 132 &  1.31e+03 &   10 &           &           & iters  \\ 
 \hdashline 
     &           &    1 &  1.02e-08 &  1.01e-08 &      \\ 
     &           &    2 &  1.01e-08 &  2.90e-16 &      \\ 
     &           &    3 &  6.76e-08 &  2.90e-16 &      \\ 
     &           &    4 &  1.02e-08 &  1.93e-15 &      \\ 
     &           &    5 &  1.02e-08 &  2.92e-16 &      \\ 
     &           &    6 &  1.02e-08 &  2.92e-16 &      \\ 
     &           &    7 &  1.02e-08 &  2.91e-16 &      \\ 
     &           &    8 &  1.02e-08 &  2.91e-16 &      \\ 
     &           &    9 &  1.02e-08 &  2.90e-16 &      \\ 
     &           &   10 &  6.75e-08 &  2.90e-16 &      \\ 
 133 &  1.32e+03 &   10 &           &           & iters  \\ 
 \hdashline 
     &           &    1 &  4.87e-08 &  1.40e-09 &      \\ 
     &           &    2 &  2.91e-08 &  1.39e-15 &      \\ 
     &           &    3 &  4.87e-08 &  8.30e-16 &      \\ 
     &           &    4 &  2.91e-08 &  1.39e-15 &      \\ 
     &           &    5 &  2.91e-08 &  8.31e-16 &      \\ 
     &           &    6 &  4.87e-08 &  8.29e-16 &      \\ 
     &           &    7 &  2.91e-08 &  1.39e-15 &      \\ 
     &           &    8 &  2.90e-08 &  8.30e-16 &      \\ 
     &           &    9 &  4.87e-08 &  8.29e-16 &      \\ 
     &           &   10 &  2.91e-08 &  1.39e-15 &      \\ 
 134 &  1.33e+03 &   10 &           &           & iters  \\ 
 \hdashline 
     &           &    1 &  5.87e-08 &  8.43e-09 &      \\ 
     &           &    2 &  1.91e-08 &  1.68e-15 &      \\ 
     &           &    3 &  1.90e-08 &  5.44e-16 &      \\ 
     &           &    4 &  5.87e-08 &  5.44e-16 &      \\ 
     &           &    5 &  1.91e-08 &  1.67e-15 &      \\ 
     &           &    6 &  1.91e-08 &  5.45e-16 &      \\ 
     &           &    7 &  1.90e-08 &  5.44e-16 &      \\ 
     &           &    8 &  1.90e-08 &  5.44e-16 &      \\ 
     &           &    9 &  5.87e-08 &  5.43e-16 &      \\ 
     &           &   10 &  1.91e-08 &  1.68e-15 &      \\ 
 135 &  1.34e+03 &   10 &           &           & iters  \\ 
 \hdashline 
     &           &    1 &  7.63e-09 &  5.77e-09 &      \\ 
     &           &    2 &  7.62e-09 &  2.18e-16 &      \\ 
     &           &    3 &  7.60e-09 &  2.17e-16 &      \\ 
     &           &    4 &  7.59e-09 &  2.17e-16 &      \\ 
     &           &    5 &  7.58e-09 &  2.17e-16 &      \\ 
     &           &    6 &  7.01e-08 &  2.16e-16 &      \\ 
     &           &    7 &  7.67e-09 &  2.00e-15 &      \\ 
     &           &    8 &  7.66e-09 &  2.19e-16 &      \\ 
     &           &    9 &  7.65e-09 &  2.19e-16 &      \\ 
     &           &   10 &  7.64e-09 &  2.18e-16 &      \\ 
 136 &  1.35e+03 &   10 &           &           & iters  \\ 
 \hdashline 
     &           &    1 &  7.90e-08 &  1.66e-09 &      \\ 
     &           &    2 &  7.65e-08 &  2.25e-15 &      \\ 
     &           &    3 &  7.90e-08 &  2.18e-15 &      \\ 
     &           &    4 &  7.65e-08 &  2.25e-15 &      \\ 
     &           &    5 &  7.90e-08 &  2.18e-15 &      \\ 
     &           &    6 &  7.65e-08 &  2.25e-15 &      \\ 
     &           &    7 &  1.29e-09 &  2.18e-15 &      \\ 
     &           &    8 &  1.29e-09 &  3.68e-17 &      \\ 
     &           &    9 &  1.28e-09 &  3.67e-17 &      \\ 
     &           &   10 &  1.28e-09 &  3.67e-17 &      \\ 
 137 &  1.36e+03 &   10 &           &           & iters  \\ 
 \hdashline 
     &           &    1 &  2.21e-08 &  1.20e-09 &      \\ 
     &           &    2 &  5.56e-08 &  6.31e-16 &      \\ 
     &           &    3 &  2.22e-08 &  1.59e-15 &      \\ 
     &           &    4 &  2.21e-08 &  6.32e-16 &      \\ 
     &           &    5 &  5.56e-08 &  6.32e-16 &      \\ 
     &           &    6 &  2.22e-08 &  1.59e-15 &      \\ 
     &           &    7 &  2.21e-08 &  6.33e-16 &      \\ 
     &           &    8 &  2.21e-08 &  6.32e-16 &      \\ 
     &           &    9 &  5.56e-08 &  6.31e-16 &      \\ 
     &           &   10 &  2.22e-08 &  1.59e-15 &      \\ 
 138 &  1.37e+03 &   10 &           &           & iters  \\ 
 \hdashline 
     &           &    1 &  2.84e-09 &  3.95e-09 &      \\ 
     &           &    2 &  2.84e-09 &  8.11e-17 &      \\ 
     &           &    3 &  2.84e-09 &  8.10e-17 &      \\ 
     &           &    4 &  2.83e-09 &  8.09e-17 &      \\ 
     &           &    5 &  2.83e-09 &  8.08e-17 &      \\ 
     &           &    6 &  2.82e-09 &  8.07e-17 &      \\ 
     &           &    7 &  2.82e-09 &  8.06e-17 &      \\ 
     &           &    8 &  2.81e-09 &  8.04e-17 &      \\ 
     &           &    9 &  2.81e-09 &  8.03e-17 &      \\ 
     &           &   10 &  2.81e-09 &  8.02e-17 &      \\ 
 139 &  1.38e+03 &   10 &           &           & iters  \\ 
 \hdashline 
     &           &    1 &  2.56e-08 &  2.28e-09 &      \\ 
     &           &    2 &  2.55e-08 &  7.30e-16 &      \\ 
     &           &    3 &  5.22e-08 &  7.29e-16 &      \\ 
     &           &    4 &  2.56e-08 &  1.49e-15 &      \\ 
     &           &    5 &  2.55e-08 &  7.30e-16 &      \\ 
     &           &    6 &  5.22e-08 &  7.29e-16 &      \\ 
     &           &    7 &  2.56e-08 &  1.49e-15 &      \\ 
     &           &    8 &  2.55e-08 &  7.30e-16 &      \\ 
     &           &    9 &  5.22e-08 &  7.29e-16 &      \\ 
     &           &   10 &  2.56e-08 &  1.49e-15 &      \\ 
 140 &  1.39e+03 &   10 &           &           & iters  \\ 
 \hdashline 
     &           &    1 &  5.48e-10 &  4.23e-09 &      \\ 
     &           &    2 &  5.47e-10 &  1.56e-17 &      \\ 
     &           &    3 &  5.46e-10 &  1.56e-17 &      \\ 
     &           &    4 &  5.46e-10 &  1.56e-17 &      \\ 
     &           &    5 &  5.45e-10 &  1.56e-17 &      \\ 
     &           &    6 &  5.44e-10 &  1.55e-17 &      \\ 
     &           &    7 &  5.43e-10 &  1.55e-17 &      \\ 
     &           &    8 &  5.42e-10 &  1.55e-17 &      \\ 
     &           &    9 &  5.42e-10 &  1.55e-17 &      \\ 
     &           &   10 &  5.41e-10 &  1.55e-17 &      \\ 
 141 &  1.40e+03 &   10 &           &           & iters  \\ 
 \hdashline 
     &           &    1 &  3.61e-08 &  5.85e-09 &      \\ 
     &           &    2 &  3.61e-08 &  1.03e-15 &      \\ 
     &           &    3 &  4.16e-08 &  1.03e-15 &      \\ 
     &           &    4 &  3.61e-08 &  1.19e-15 &      \\ 
     &           &    5 &  4.16e-08 &  1.03e-15 &      \\ 
     &           &    6 &  3.61e-08 &  1.19e-15 &      \\ 
     &           &    7 &  4.16e-08 &  1.03e-15 &      \\ 
     &           &    8 &  3.61e-08 &  1.19e-15 &      \\ 
     &           &    9 &  4.16e-08 &  1.03e-15 &      \\ 
     &           &   10 &  3.61e-08 &  1.19e-15 &      \\ 
 142 &  1.41e+03 &   10 &           &           & iters  \\ 
 \hdashline 
     &           &    1 &  5.06e-08 &  2.63e-10 &      \\ 
     &           &    2 &  2.72e-08 &  1.44e-15 &      \\ 
     &           &    3 &  2.71e-08 &  7.75e-16 &      \\ 
     &           &    4 &  5.06e-08 &  7.74e-16 &      \\ 
     &           &    5 &  2.72e-08 &  1.44e-15 &      \\ 
     &           &    6 &  2.71e-08 &  7.75e-16 &      \\ 
     &           &    7 &  5.06e-08 &  7.74e-16 &      \\ 
     &           &    8 &  2.71e-08 &  1.44e-15 &      \\ 
     &           &    9 &  2.71e-08 &  7.75e-16 &      \\ 
     &           &   10 &  5.06e-08 &  7.74e-16 &      \\ 
 143 &  1.42e+03 &   10 &           &           & iters  \\ 
 \hdashline 
     &           &    1 &  1.24e-09 &  8.28e-09 &      \\ 
     &           &    2 &  1.24e-09 &  3.55e-17 &      \\ 
     &           &    3 &  1.24e-09 &  3.54e-17 &      \\ 
     &           &    4 &  1.24e-09 &  3.54e-17 &      \\ 
     &           &    5 &  1.24e-09 &  3.53e-17 &      \\ 
     &           &    6 &  1.23e-09 &  3.53e-17 &      \\ 
     &           &    7 &  1.23e-09 &  3.52e-17 &      \\ 
     &           &    8 &  1.23e-09 &  3.52e-17 &      \\ 
     &           &    9 &  1.23e-09 &  3.51e-17 &      \\ 
     &           &   10 &  1.23e-09 &  3.51e-17 &      \\ 
 144 &  1.43e+03 &   10 &           &           & iters  \\ 
 \hdashline 
     &           &    1 &  1.27e-08 &  1.02e-08 &      \\ 
     &           &    2 &  1.26e-08 &  3.61e-16 &      \\ 
     &           &    3 &  6.51e-08 &  3.61e-16 &      \\ 
     &           &    4 &  1.27e-08 &  1.86e-15 &      \\ 
     &           &    5 &  1.27e-08 &  3.63e-16 &      \\ 
     &           &    6 &  1.27e-08 &  3.62e-16 &      \\ 
     &           &    7 &  1.27e-08 &  3.62e-16 &      \\ 
     &           &    8 &  1.26e-08 &  3.61e-16 &      \\ 
     &           &    9 &  6.51e-08 &  3.61e-16 &      \\ 
     &           &   10 &  9.04e-08 &  1.86e-15 &      \\ 
 145 &  1.44e+03 &   10 &           &           & iters  \\ 
 \hdashline 
     &           &    1 &  2.42e-08 &  2.66e-09 &      \\ 
     &           &    2 &  2.41e-08 &  6.90e-16 &      \\ 
     &           &    3 &  2.41e-08 &  6.89e-16 &      \\ 
     &           &    4 &  2.41e-08 &  6.88e-16 &      \\ 
     &           &    5 &  2.40e-08 &  6.87e-16 &      \\ 
     &           &    6 &  5.37e-08 &  6.86e-16 &      \\ 
     &           &    7 &  2.41e-08 &  1.53e-15 &      \\ 
     &           &    8 &  2.40e-08 &  6.87e-16 &      \\ 
     &           &    9 &  5.37e-08 &  6.86e-16 &      \\ 
     &           &   10 &  2.41e-08 &  1.53e-15 &      \\ 
 146 &  1.45e+03 &   10 &           &           & iters  \\ 
 \hdashline 
     &           &    1 &  5.83e-08 &  9.02e-09 &      \\ 
     &           &    2 &  1.94e-08 &  1.66e-15 &      \\ 
     &           &    3 &  1.94e-08 &  5.55e-16 &      \\ 
     &           &    4 &  1.94e-08 &  5.54e-16 &      \\ 
     &           &    5 &  5.83e-08 &  5.53e-16 &      \\ 
     &           &    6 &  1.94e-08 &  1.66e-15 &      \\ 
     &           &    7 &  1.94e-08 &  5.55e-16 &      \\ 
     &           &    8 &  1.94e-08 &  5.54e-16 &      \\ 
     &           &    9 &  5.83e-08 &  5.53e-16 &      \\ 
     &           &   10 &  1.94e-08 &  1.66e-15 &      \\ 
 147 &  1.46e+03 &   10 &           &           & iters  \\ 
 \hdashline 
     &           &    1 &  1.82e-08 &  9.11e-09 &      \\ 
     &           &    2 &  1.81e-08 &  5.19e-16 &      \\ 
     &           &    3 &  5.96e-08 &  5.18e-16 &      \\ 
     &           &    4 &  1.82e-08 &  1.70e-15 &      \\ 
     &           &    5 &  1.82e-08 &  5.20e-16 &      \\ 
     &           &    6 &  1.82e-08 &  5.19e-16 &      \\ 
     &           &    7 &  5.96e-08 &  5.18e-16 &      \\ 
     &           &    8 &  1.82e-08 &  1.70e-15 &      \\ 
     &           &    9 &  1.82e-08 &  5.20e-16 &      \\ 
     &           &   10 &  1.82e-08 &  5.19e-16 &      \\ 
 148 &  1.47e+03 &   10 &           &           & iters  \\ 
 \hdashline 
     &           &    1 &  3.93e-08 &  3.24e-09 &      \\ 
     &           &    2 &  3.84e-08 &  1.12e-15 &      \\ 
     &           &    3 &  3.93e-08 &  1.10e-15 &      \\ 
     &           &    4 &  3.84e-08 &  1.12e-15 &      \\ 
     &           &    5 &  3.93e-08 &  1.10e-15 &      \\ 
     &           &    6 &  3.84e-08 &  1.12e-15 &      \\ 
     &           &    7 &  3.93e-08 &  1.10e-15 &      \\ 
     &           &    8 &  3.84e-08 &  1.12e-15 &      \\ 
     &           &    9 &  3.93e-08 &  1.10e-15 &      \\ 
     &           &   10 &  3.84e-08 &  1.12e-15 &      \\ 
 149 &  1.48e+03 &   10 &           &           & iters  \\ 
 \hdashline 
     &           &    1 &  8.98e-08 &  3.58e-10 &      \\ 
     &           &    2 &  6.57e-08 &  2.56e-15 &      \\ 
     &           &    3 &  1.20e-08 &  1.88e-15 &      \\ 
     &           &    4 &  1.20e-08 &  3.44e-16 &      \\ 
     &           &    5 &  1.20e-08 &  3.43e-16 &      \\ 
     &           &    6 &  1.20e-08 &  3.43e-16 &      \\ 
     &           &    7 &  6.57e-08 &  3.42e-16 &      \\ 
     &           &    8 &  8.98e-08 &  1.88e-15 &      \\ 
     &           &    9 &  6.58e-08 &  2.56e-15 &      \\ 
     &           &   10 &  1.20e-08 &  1.88e-15 &      \\ 
 150 &  1.49e+03 &   10 &           &           & iters  \\ 
 \hdashline 
     &           &    1 &  4.72e-08 &  4.54e-09 &      \\ 
     &           &    2 &  3.05e-08 &  1.35e-15 &      \\ 
     &           &    3 &  3.05e-08 &  8.71e-16 &      \\ 
     &           &    4 &  4.73e-08 &  8.70e-16 &      \\ 
     &           &    5 &  3.05e-08 &  1.35e-15 &      \\ 
     &           &    6 &  4.72e-08 &  8.70e-16 &      \\ 
     &           &    7 &  3.05e-08 &  1.35e-15 &      \\ 
     &           &    8 &  3.05e-08 &  8.71e-16 &      \\ 
     &           &    9 &  4.73e-08 &  8.70e-16 &      \\ 
     &           &   10 &  3.05e-08 &  1.35e-15 &      \\ 
 151 &  1.50e+03 &   10 &           &           & iters  \\ 
 \hdashline 
     &           &    1 &  1.11e-08 &  1.62e-09 &      \\ 
     &           &    2 &  1.11e-08 &  3.18e-16 &      \\ 
     &           &    3 &  6.66e-08 &  3.17e-16 &      \\ 
     &           &    4 &  1.12e-08 &  1.90e-15 &      \\ 
     &           &    5 &  1.12e-08 &  3.19e-16 &      \\ 
     &           &    6 &  1.12e-08 &  3.19e-16 &      \\ 
     &           &    7 &  1.11e-08 &  3.18e-16 &      \\ 
     &           &    8 &  1.11e-08 &  3.18e-16 &      \\ 
     &           &    9 &  1.11e-08 &  3.18e-16 &      \\ 
     &           &   10 &  6.66e-08 &  3.17e-16 &      \\ 
 152 &  1.51e+03 &   10 &           &           & iters  \\ 
 \hdashline 
     &           &    1 &  4.46e-08 &  4.28e-09 &      \\ 
     &           &    2 &  3.31e-08 &  1.27e-15 &      \\ 
     &           &    3 &  4.46e-08 &  9.46e-16 &      \\ 
     &           &    4 &  3.32e-08 &  1.27e-15 &      \\ 
     &           &    5 &  4.46e-08 &  9.46e-16 &      \\ 
     &           &    6 &  3.32e-08 &  1.27e-15 &      \\ 
     &           &    7 &  3.31e-08 &  9.47e-16 &      \\ 
     &           &    8 &  4.46e-08 &  9.45e-16 &      \\ 
     &           &    9 &  3.31e-08 &  1.27e-15 &      \\ 
     &           &   10 &  4.46e-08 &  9.46e-16 &      \\ 
 153 &  1.52e+03 &   10 &           &           & iters  \\ 
 \hdashline 
     &           &    1 &  1.06e-08 &  2.75e-09 &      \\ 
     &           &    2 &  1.06e-08 &  3.03e-16 &      \\ 
     &           &    3 &  2.82e-08 &  3.03e-16 &      \\ 
     &           &    4 &  1.06e-08 &  8.06e-16 &      \\ 
     &           &    5 &  1.06e-08 &  3.04e-16 &      \\ 
     &           &    6 &  1.06e-08 &  3.03e-16 &      \\ 
     &           &    7 &  2.83e-08 &  3.03e-16 &      \\ 
     &           &    8 &  1.06e-08 &  8.06e-16 &      \\ 
     &           &    9 &  1.06e-08 &  3.03e-16 &      \\ 
     &           &   10 &  2.82e-08 &  3.03e-16 &      \\ 
 154 &  1.53e+03 &   10 &           &           & iters  \\ 
 \hdashline 
     &           &    1 &  7.33e-09 &  1.86e-09 &      \\ 
     &           &    2 &  7.32e-09 &  2.09e-16 &      \\ 
     &           &    3 &  7.31e-09 &  2.09e-16 &      \\ 
     &           &    4 &  7.04e-08 &  2.09e-16 &      \\ 
     &           &    5 &  8.51e-08 &  2.01e-15 &      \\ 
     &           &    6 &  7.04e-08 &  2.43e-15 &      \\ 
     &           &    7 &  7.38e-09 &  2.01e-15 &      \\ 
     &           &    8 &  7.37e-09 &  2.11e-16 &      \\ 
     &           &    9 &  7.36e-09 &  2.10e-16 &      \\ 
     &           &   10 &  7.35e-09 &  2.10e-16 &      \\ 
 155 &  1.54e+03 &   10 &           &           & iters  \\ 
 \hdashline 
     &           &    1 &  1.82e-08 &  4.84e-09 &      \\ 
     &           &    2 &  5.95e-08 &  5.19e-16 &      \\ 
     &           &    3 &  1.83e-08 &  1.70e-15 &      \\ 
     &           &    4 &  1.82e-08 &  5.21e-16 &      \\ 
     &           &    5 &  1.82e-08 &  5.20e-16 &      \\ 
     &           &    6 &  1.82e-08 &  5.19e-16 &      \\ 
     &           &    7 &  5.95e-08 &  5.19e-16 &      \\ 
     &           &    8 &  1.82e-08 &  1.70e-15 &      \\ 
     &           &    9 &  1.82e-08 &  5.20e-16 &      \\ 
     &           &   10 &  1.82e-08 &  5.20e-16 &      \\ 
 156 &  1.55e+03 &   10 &           &           & iters  \\ 
 \hdashline 
     &           &    1 &  8.14e-09 &  5.00e-09 &      \\ 
     &           &    2 &  8.13e-09 &  2.32e-16 &      \\ 
     &           &    3 &  8.12e-09 &  2.32e-16 &      \\ 
     &           &    4 &  6.96e-08 &  2.32e-16 &      \\ 
     &           &    5 &  8.59e-08 &  1.99e-15 &      \\ 
     &           &    6 &  6.96e-08 &  2.45e-15 &      \\ 
     &           &    7 &  8.18e-09 &  1.99e-15 &      \\ 
     &           &    8 &  8.17e-09 &  2.34e-16 &      \\ 
     &           &    9 &  8.16e-09 &  2.33e-16 &      \\ 
     &           &   10 &  8.15e-09 &  2.33e-16 &      \\ 
 157 &  1.56e+03 &   10 &           &           & iters  \\ 
 \hdashline 
     &           &    1 &  3.83e-08 &  6.67e-10 &      \\ 
     &           &    2 &  3.94e-08 &  1.09e-15 &      \\ 
     &           &    3 &  3.83e-08 &  1.12e-15 &      \\ 
     &           &    4 &  3.94e-08 &  1.09e-15 &      \\ 
     &           &    5 &  3.83e-08 &  1.12e-15 &      \\ 
     &           &    6 &  3.94e-08 &  1.09e-15 &      \\ 
     &           &    7 &  3.83e-08 &  1.12e-15 &      \\ 
     &           &    8 &  3.94e-08 &  1.09e-15 &      \\ 
     &           &    9 &  3.83e-08 &  1.12e-15 &      \\ 
     &           &   10 &  3.94e-08 &  1.09e-15 &      \\ 
 158 &  1.57e+03 &   10 &           &           & iters  \\ 
 \hdashline 
     &           &    1 &  3.81e-08 &  1.89e-09 &      \\ 
     &           &    2 &  3.96e-08 &  1.09e-15 &      \\ 
     &           &    3 &  3.81e-08 &  1.13e-15 &      \\ 
     &           &    4 &  3.96e-08 &  1.09e-15 &      \\ 
     &           &    5 &  3.81e-08 &  1.13e-15 &      \\ 
     &           &    6 &  3.96e-08 &  1.09e-15 &      \\ 
     &           &    7 &  3.81e-08 &  1.13e-15 &      \\ 
     &           &    8 &  3.96e-08 &  1.09e-15 &      \\ 
     &           &    9 &  3.81e-08 &  1.13e-15 &      \\ 
     &           &   10 &  3.96e-08 &  1.09e-15 &      \\ 
 159 &  1.58e+03 &   10 &           &           & iters  \\ 
 \hdashline 
     &           &    1 &  5.31e-08 &  3.75e-09 &      \\ 
     &           &    2 &  2.47e-08 &  1.52e-15 &      \\ 
     &           &    3 &  2.46e-08 &  7.04e-16 &      \\ 
     &           &    4 &  5.31e-08 &  7.03e-16 &      \\ 
     &           &    5 &  2.47e-08 &  1.51e-15 &      \\ 
     &           &    6 &  2.47e-08 &  7.05e-16 &      \\ 
     &           &    7 &  5.31e-08 &  7.04e-16 &      \\ 
     &           &    8 &  2.47e-08 &  1.51e-15 &      \\ 
     &           &    9 &  2.47e-08 &  7.05e-16 &      \\ 
     &           &   10 &  5.31e-08 &  7.04e-16 &      \\ 
 160 &  1.59e+03 &   10 &           &           & iters  \\ 
 \hdashline 
     &           &    1 &  2.77e-08 &  8.49e-09 &      \\ 
     &           &    2 &  2.76e-08 &  7.90e-16 &      \\ 
     &           &    3 &  1.12e-08 &  7.89e-16 &      \\ 
     &           &    4 &  1.12e-08 &  3.21e-16 &      \\ 
     &           &    5 &  2.76e-08 &  3.21e-16 &      \\ 
     &           &    6 &  1.13e-08 &  7.89e-16 &      \\ 
     &           &    7 &  1.12e-08 &  3.21e-16 &      \\ 
     &           &    8 &  2.76e-08 &  3.21e-16 &      \\ 
     &           &    9 &  1.13e-08 &  7.88e-16 &      \\ 
     &           &   10 &  1.12e-08 &  3.21e-16 &      \\ 
 161 &  1.60e+03 &   10 &           &           & iters  \\ 
 \hdashline 
     &           &    1 &  1.62e-08 &  5.84e-09 &      \\ 
     &           &    2 &  1.62e-08 &  4.64e-16 &      \\ 
     &           &    3 &  6.15e-08 &  4.63e-16 &      \\ 
     &           &    4 &  1.63e-08 &  1.76e-15 &      \\ 
     &           &    5 &  1.63e-08 &  4.65e-16 &      \\ 
     &           &    6 &  1.62e-08 &  4.64e-16 &      \\ 
     &           &    7 &  6.15e-08 &  4.63e-16 &      \\ 
     &           &    8 &  1.63e-08 &  1.75e-15 &      \\ 
     &           &    9 &  1.63e-08 &  4.65e-16 &      \\ 
     &           &   10 &  1.63e-08 &  4.65e-16 &      \\ 
 162 &  1.61e+03 &   10 &           &           & iters  \\ 
 \hdashline 
     &           &    1 &  2.03e-08 &  1.24e-10 &      \\ 
     &           &    2 &  5.74e-08 &  5.80e-16 &      \\ 
     &           &    3 &  2.04e-08 &  1.64e-15 &      \\ 
     &           &    4 &  2.03e-08 &  5.81e-16 &      \\ 
     &           &    5 &  2.03e-08 &  5.81e-16 &      \\ 
     &           &    6 &  5.74e-08 &  5.80e-16 &      \\ 
     &           &    7 &  2.04e-08 &  1.64e-15 &      \\ 
     &           &    8 &  2.03e-08 &  5.81e-16 &      \\ 
     &           &    9 &  2.03e-08 &  5.80e-16 &      \\ 
     &           &   10 &  5.74e-08 &  5.80e-16 &      \\ 
 163 &  1.62e+03 &   10 &           &           & iters  \\ 
 \hdashline 
     &           &    1 &  4.19e-08 &  4.21e-09 &      \\ 
     &           &    2 &  3.58e-08 &  1.20e-15 &      \\ 
     &           &    3 &  4.19e-08 &  1.02e-15 &      \\ 
     &           &    4 &  3.59e-08 &  1.20e-15 &      \\ 
     &           &    5 &  4.19e-08 &  1.02e-15 &      \\ 
     &           &    6 &  3.59e-08 &  1.20e-15 &      \\ 
     &           &    7 &  4.19e-08 &  1.02e-15 &      \\ 
     &           &    8 &  3.59e-08 &  1.20e-15 &      \\ 
     &           &    9 &  4.19e-08 &  1.02e-15 &      \\ 
     &           &   10 &  3.59e-08 &  1.19e-15 &      \\ 
 164 &  1.63e+03 &   10 &           &           & iters  \\ 
 \hdashline 
     &           &    1 &  1.96e-08 &  4.19e-09 &      \\ 
     &           &    2 &  5.81e-08 &  5.61e-16 &      \\ 
     &           &    3 &  1.97e-08 &  1.66e-15 &      \\ 
     &           &    4 &  1.97e-08 &  5.62e-16 &      \\ 
     &           &    5 &  1.96e-08 &  5.61e-16 &      \\ 
     &           &    6 &  5.81e-08 &  5.61e-16 &      \\ 
     &           &    7 &  1.97e-08 &  1.66e-15 &      \\ 
     &           &    8 &  1.97e-08 &  5.62e-16 &      \\ 
     &           &    9 &  1.96e-08 &  5.61e-16 &      \\ 
     &           &   10 &  5.81e-08 &  5.61e-16 &      \\ 
 165 &  1.64e+03 &   10 &           &           & iters  \\ 
 \hdashline 
     &           &    1 &  2.27e-08 &  7.73e-09 &      \\ 
     &           &    2 &  2.27e-08 &  6.49e-16 &      \\ 
     &           &    3 &  1.62e-08 &  6.48e-16 &      \\ 
     &           &    4 &  2.27e-08 &  4.61e-16 &      \\ 
     &           &    5 &  1.62e-08 &  6.48e-16 &      \\ 
     &           &    6 &  2.27e-08 &  4.62e-16 &      \\ 
     &           &    7 &  1.62e-08 &  6.48e-16 &      \\ 
     &           &    8 &  1.62e-08 &  4.62e-16 &      \\ 
     &           &    9 &  2.27e-08 &  4.61e-16 &      \\ 
     &           &   10 &  1.62e-08 &  6.48e-16 &      \\ 
 166 &  1.65e+03 &   10 &           &           & iters  \\ 
 \hdashline 
     &           &    1 &  3.90e-08 &  1.37e-08 &      \\ 
     &           &    2 &  3.88e-08 &  1.11e-15 &      \\ 
     &           &    3 &  3.90e-08 &  1.11e-15 &      \\ 
     &           &    4 &  3.88e-08 &  1.11e-15 &      \\ 
     &           &    5 &  3.90e-08 &  1.11e-15 &      \\ 
     &           &    6 &  3.88e-08 &  1.11e-15 &      \\ 
     &           &    7 &  3.90e-08 &  1.11e-15 &      \\ 
     &           &    8 &  3.88e-08 &  1.11e-15 &      \\ 
     &           &    9 &  3.90e-08 &  1.11e-15 &      \\ 
     &           &   10 &  3.88e-08 &  1.11e-15 &      \\ 
 167 &  1.66e+03 &   10 &           &           & iters  \\ 
 \hdashline 
     &           &    1 &  2.85e-08 &  1.62e-08 &      \\ 
     &           &    2 &  4.93e-08 &  8.12e-16 &      \\ 
     &           &    3 &  2.85e-08 &  1.41e-15 &      \\ 
     &           &    4 &  2.85e-08 &  8.13e-16 &      \\ 
     &           &    5 &  4.93e-08 &  8.12e-16 &      \\ 
     &           &    6 &  2.85e-08 &  1.41e-15 &      \\ 
     &           &    7 &  2.84e-08 &  8.13e-16 &      \\ 
     &           &    8 &  4.93e-08 &  8.12e-16 &      \\ 
     &           &    9 &  2.85e-08 &  1.41e-15 &      \\ 
     &           &   10 &  2.84e-08 &  8.13e-16 &      \\ 
 168 &  1.67e+03 &   10 &           &           & iters  \\ 
 \hdashline 
     &           &    1 &  4.88e-08 &  1.62e-08 &      \\ 
     &           &    2 &  2.89e-08 &  1.39e-15 &      \\ 
     &           &    3 &  4.88e-08 &  8.25e-16 &      \\ 
     &           &    4 &  2.90e-08 &  1.39e-15 &      \\ 
     &           &    5 &  2.89e-08 &  8.26e-16 &      \\ 
     &           &    6 &  4.88e-08 &  8.25e-16 &      \\ 
     &           &    7 &  2.89e-08 &  1.39e-15 &      \\ 
     &           &    8 &  2.89e-08 &  8.26e-16 &      \\ 
     &           &    9 &  4.88e-08 &  8.25e-16 &      \\ 
     &           &   10 &  2.89e-08 &  1.39e-15 &      \\ 
 169 &  1.68e+03 &   10 &           &           & iters  \\ 
 \hdashline 
     &           &    1 &  7.89e-08 &  1.25e-09 &      \\ 
     &           &    2 &  1.10e-09 &  2.25e-15 &      \\ 
     &           &    3 &  1.10e-09 &  3.14e-17 &      \\ 
     &           &    4 &  1.10e-09 &  3.13e-17 &      \\ 
     &           &    5 &  1.09e-09 &  3.13e-17 &      \\ 
     &           &    6 &  1.09e-09 &  3.12e-17 &      \\ 
     &           &    7 &  1.09e-09 &  3.12e-17 &      \\ 
     &           &    8 &  7.66e-08 &  3.12e-17 &      \\ 
     &           &    9 &  7.89e-08 &  2.19e-15 &      \\ 
     &           &   10 &  7.66e-08 &  2.25e-15 &      \\ 
 170 &  1.69e+03 &   10 &           &           & iters  \\ 
 \hdashline 
     &           &    1 &  7.88e-08 &  9.05e-09 &      \\ 
     &           &    2 &  7.67e-08 &  2.25e-15 &      \\ 
     &           &    3 &  7.88e-08 &  2.19e-15 &      \\ 
     &           &    4 &  7.67e-08 &  2.25e-15 &      \\ 
     &           &    5 &  7.88e-08 &  2.19e-15 &      \\ 
     &           &    6 &  7.67e-08 &  2.25e-15 &      \\ 
     &           &    7 &  7.88e-08 &  2.19e-15 &      \\ 
     &           &    8 &  7.67e-08 &  2.25e-15 &      \\ 
     &           &    9 &  7.88e-08 &  2.19e-15 &      \\ 
     &           &   10 &  7.67e-08 &  2.25e-15 &      \\ 
 171 &  1.70e+03 &   10 &           &           & iters  \\ 
 \hdashline 
     &           &    1 &  1.28e-08 &  1.98e-09 &      \\ 
     &           &    2 &  1.28e-08 &  3.65e-16 &      \\ 
     &           &    3 &  1.28e-08 &  3.65e-16 &      \\ 
     &           &    4 &  6.49e-08 &  3.64e-16 &      \\ 
     &           &    5 &  1.28e-08 &  1.85e-15 &      \\ 
     &           &    6 &  1.28e-08 &  3.66e-16 &      \\ 
     &           &    7 &  1.28e-08 &  3.66e-16 &      \\ 
     &           &    8 &  1.28e-08 &  3.65e-16 &      \\ 
     &           &    9 &  1.28e-08 &  3.65e-16 &      \\ 
     &           &   10 &  6.49e-08 &  3.64e-16 &      \\ 
 172 &  1.71e+03 &   10 &           &           & iters  \\ 
 \hdashline 
     &           &    1 &  5.78e-08 &  3.69e-09 &      \\ 
     &           &    2 &  2.00e-08 &  1.65e-15 &      \\ 
     &           &    3 &  2.00e-08 &  5.70e-16 &      \\ 
     &           &    4 &  1.99e-08 &  5.70e-16 &      \\ 
     &           &    5 &  5.78e-08 &  5.69e-16 &      \\ 
     &           &    6 &  2.00e-08 &  1.65e-15 &      \\ 
     &           &    7 &  2.00e-08 &  5.70e-16 &      \\ 
     &           &    8 &  1.99e-08 &  5.70e-16 &      \\ 
     &           &    9 &  5.78e-08 &  5.69e-16 &      \\ 
     &           &   10 &  2.00e-08 &  1.65e-15 &      \\ 
 173 &  1.72e+03 &   10 &           &           & iters  \\ 
 \hdashline 
     &           &    1 &  5.26e-09 &  6.31e-09 &      \\ 
     &           &    2 &  5.25e-09 &  1.50e-16 &      \\ 
     &           &    3 &  5.24e-09 &  1.50e-16 &      \\ 
     &           &    4 &  5.23e-09 &  1.50e-16 &      \\ 
     &           &    5 &  5.23e-09 &  1.49e-16 &      \\ 
     &           &    6 &  5.22e-09 &  1.49e-16 &      \\ 
     &           &    7 &  5.21e-09 &  1.49e-16 &      \\ 
     &           &    8 &  5.20e-09 &  1.49e-16 &      \\ 
     &           &    9 &  5.20e-09 &  1.49e-16 &      \\ 
     &           &   10 &  5.19e-09 &  1.48e-16 &      \\ 
 174 &  1.73e+03 &   10 &           &           & iters  \\ 
 \hdashline 
     &           &    1 &  7.28e-08 &  1.53e-08 &      \\ 
     &           &    2 &  5.01e-09 &  2.08e-15 &      \\ 
     &           &    3 &  5.00e-09 &  1.43e-16 &      \\ 
     &           &    4 &  5.00e-09 &  1.43e-16 &      \\ 
     &           &    5 &  4.99e-09 &  1.43e-16 &      \\ 
     &           &    6 &  4.98e-09 &  1.42e-16 &      \\ 
     &           &    7 &  4.98e-09 &  1.42e-16 &      \\ 
     &           &    8 &  4.97e-09 &  1.42e-16 &      \\ 
     &           &    9 &  4.96e-09 &  1.42e-16 &      \\ 
     &           &   10 &  4.95e-09 &  1.42e-16 &      \\ 
 175 &  1.74e+03 &   10 &           &           & iters  \\ 
 \hdashline 
     &           &    1 &  4.69e-08 &  8.75e-09 &      \\ 
     &           &    2 &  3.09e-08 &  1.34e-15 &      \\ 
     &           &    3 &  3.08e-08 &  8.82e-16 &      \\ 
     &           &    4 &  4.69e-08 &  8.80e-16 &      \\ 
     &           &    5 &  3.09e-08 &  1.34e-15 &      \\ 
     &           &    6 &  4.69e-08 &  8.81e-16 &      \\ 
     &           &    7 &  3.09e-08 &  1.34e-15 &      \\ 
     &           &    8 &  3.08e-08 &  8.82e-16 &      \\ 
     &           &    9 &  4.69e-08 &  8.80e-16 &      \\ 
     &           &   10 &  3.09e-08 &  1.34e-15 &      \\ 
 176 &  1.75e+03 &   10 &           &           & iters  \\ 
 \hdashline 
     &           &    1 &  4.88e-08 &  1.83e-09 &      \\ 
     &           &    2 &  2.90e-08 &  1.39e-15 &      \\ 
     &           &    3 &  2.89e-08 &  8.27e-16 &      \\ 
     &           &    4 &  4.88e-08 &  8.26e-16 &      \\ 
     &           &    5 &  2.90e-08 &  1.39e-15 &      \\ 
     &           &    6 &  2.89e-08 &  8.26e-16 &      \\ 
     &           &    7 &  4.88e-08 &  8.25e-16 &      \\ 
     &           &    8 &  2.89e-08 &  1.39e-15 &      \\ 
     &           &    9 &  4.88e-08 &  8.26e-16 &      \\ 
     &           &   10 &  2.90e-08 &  1.39e-15 &      \\ 
 177 &  1.76e+03 &   10 &           &           & iters  \\ 
 \hdashline 
     &           &    1 &  1.46e-10 &  4.23e-09 &      \\ 
     &           &    2 &  1.46e-10 &  4.16e-18 &      \\ 
     &           &    3 &  1.45e-10 &  4.16e-18 &      \\ 
     &           &    4 &  1.45e-10 &  4.15e-18 &      \\ 
     &           &    5 &  1.45e-10 &  4.15e-18 &      \\ 
     &           &    6 &  1.45e-10 &  4.14e-18 &      \\ 
     &           &    7 &  1.45e-10 &  4.13e-18 &      \\ 
     &           &    8 &  1.44e-10 &  4.13e-18 &      \\ 
     &           &    9 &  1.44e-10 &  4.12e-18 &      \\ 
     &           &   10 &  1.44e-10 &  4.12e-18 &      \\ 
 178 &  1.77e+03 &   10 &           &           & iters  \\ 
 \hdashline 
     &           &    1 &  9.98e-09 &  3.79e-09 &      \\ 
     &           &    2 &  9.97e-09 &  2.85e-16 &      \\ 
     &           &    3 &  6.77e-08 &  2.85e-16 &      \\ 
     &           &    4 &  1.01e-08 &  1.93e-15 &      \\ 
     &           &    5 &  1.00e-08 &  2.87e-16 &      \\ 
     &           &    6 &  1.00e-08 &  2.86e-16 &      \\ 
     &           &    7 &  1.00e-08 &  2.86e-16 &      \\ 
     &           &    8 &  9.99e-09 &  2.86e-16 &      \\ 
     &           &    9 &  9.98e-09 &  2.85e-16 &      \\ 
     &           &   10 &  9.97e-09 &  2.85e-16 &      \\ 
 179 &  1.78e+03 &   10 &           &           & iters  \\ 
 \hdashline 
     &           &    1 &  3.87e-08 &  8.65e-09 &      \\ 
     &           &    2 &  3.90e-08 &  1.11e-15 &      \\ 
     &           &    3 &  3.87e-08 &  1.11e-15 &      \\ 
     &           &    4 &  3.90e-08 &  1.11e-15 &      \\ 
     &           &    5 &  3.87e-08 &  1.11e-15 &      \\ 
     &           &    6 &  3.90e-08 &  1.11e-15 &      \\ 
     &           &    7 &  3.87e-08 &  1.11e-15 &      \\ 
     &           &    8 &  3.90e-08 &  1.11e-15 &      \\ 
     &           &    9 &  3.87e-08 &  1.11e-15 &      \\ 
     &           &   10 &  3.90e-08 &  1.11e-15 &      \\ 
 180 &  1.79e+03 &   10 &           &           & iters  \\ 
 \hdashline 
     &           &    1 &  5.15e-08 &  9.08e-09 &      \\ 
     &           &    2 &  2.62e-08 &  1.47e-15 &      \\ 
     &           &    3 &  2.62e-08 &  7.49e-16 &      \\ 
     &           &    4 &  5.15e-08 &  7.48e-16 &      \\ 
     &           &    5 &  2.62e-08 &  1.47e-15 &      \\ 
     &           &    6 &  2.62e-08 &  7.49e-16 &      \\ 
     &           &    7 &  5.15e-08 &  7.48e-16 &      \\ 
     &           &    8 &  2.62e-08 &  1.47e-15 &      \\ 
     &           &    9 &  2.62e-08 &  7.49e-16 &      \\ 
     &           &   10 &  5.15e-08 &  7.48e-16 &      \\ 
 181 &  1.80e+03 &   10 &           &           & iters  \\ 
 \hdashline 
     &           &    1 &  1.46e-08 &  3.18e-09 &      \\ 
     &           &    2 &  1.46e-08 &  4.16e-16 &      \\ 
     &           &    3 &  1.45e-08 &  4.16e-16 &      \\ 
     &           &    4 &  6.32e-08 &  4.15e-16 &      \\ 
     &           &    5 &  1.46e-08 &  1.80e-15 &      \\ 
     &           &    6 &  1.46e-08 &  4.17e-16 &      \\ 
     &           &    7 &  1.46e-08 &  4.17e-16 &      \\ 
     &           &    8 &  1.46e-08 &  4.16e-16 &      \\ 
     &           &    9 &  1.45e-08 &  4.15e-16 &      \\ 
     &           &   10 &  6.32e-08 &  4.15e-16 &      \\ 
 182 &  1.81e+03 &   10 &           &           & iters  \\ 
 \hdashline 
     &           &    1 &  4.50e-08 &  5.85e-09 &      \\ 
     &           &    2 &  3.28e-08 &  1.28e-15 &      \\ 
     &           &    3 &  4.50e-08 &  9.35e-16 &      \\ 
     &           &    4 &  3.28e-08 &  1.28e-15 &      \\ 
     &           &    5 &  3.27e-08 &  9.36e-16 &      \\ 
     &           &    6 &  4.50e-08 &  9.35e-16 &      \\ 
     &           &    7 &  3.28e-08 &  1.28e-15 &      \\ 
     &           &    8 &  4.50e-08 &  9.35e-16 &      \\ 
     &           &    9 &  3.28e-08 &  1.28e-15 &      \\ 
     &           &   10 &  4.50e-08 &  9.36e-16 &      \\ 
 183 &  1.82e+03 &   10 &           &           & iters  \\ 
 \hdashline 
     &           &    1 &  3.11e-08 &  7.33e-09 &      \\ 
     &           &    2 &  7.79e-09 &  8.88e-16 &      \\ 
     &           &    3 &  3.11e-08 &  2.22e-16 &      \\ 
     &           &    4 &  7.82e-09 &  8.87e-16 &      \\ 
     &           &    5 &  7.81e-09 &  2.23e-16 &      \\ 
     &           &    6 &  7.80e-09 &  2.23e-16 &      \\ 
     &           &    7 &  7.79e-09 &  2.23e-16 &      \\ 
     &           &    8 &  3.11e-08 &  2.22e-16 &      \\ 
     &           &    9 &  7.82e-09 &  8.87e-16 &      \\ 
     &           &   10 &  7.81e-09 &  2.23e-16 &      \\ 
 184 &  1.83e+03 &   10 &           &           & iters  \\ 
 \hdashline 
     &           &    1 &  1.83e-09 &  4.57e-09 &      \\ 
     &           &    2 &  1.82e-09 &  5.21e-17 &      \\ 
     &           &    3 &  1.82e-09 &  5.21e-17 &      \\ 
     &           &    4 &  1.82e-09 &  5.20e-17 &      \\ 
     &           &    5 &  1.82e-09 &  5.19e-17 &      \\ 
     &           &    6 &  1.81e-09 &  5.18e-17 &      \\ 
     &           &    7 &  1.81e-09 &  5.18e-17 &      \\ 
     &           &    8 &  1.81e-09 &  5.17e-17 &      \\ 
     &           &    9 &  1.81e-09 &  5.16e-17 &      \\ 
     &           &   10 &  1.80e-09 &  5.15e-17 &      \\ 
 185 &  1.84e+03 &   10 &           &           & iters  \\ 
 \hdashline 
     &           &    1 &  1.84e-08 &  5.08e-09 &      \\ 
     &           &    2 &  5.93e-08 &  5.26e-16 &      \\ 
     &           &    3 &  1.85e-08 &  1.69e-15 &      \\ 
     &           &    4 &  1.85e-08 &  5.28e-16 &      \\ 
     &           &    5 &  1.84e-08 &  5.27e-16 &      \\ 
     &           &    6 &  5.93e-08 &  5.27e-16 &      \\ 
     &           &    7 &  1.85e-08 &  1.69e-15 &      \\ 
     &           &    8 &  1.85e-08 &  5.28e-16 &      \\ 
     &           &    9 &  1.85e-08 &  5.27e-16 &      \\ 
     &           &   10 &  5.93e-08 &  5.27e-16 &      \\ 
 186 &  1.85e+03 &   10 &           &           & iters  \\ 
 \hdashline 
     &           &    1 &  1.25e-08 &  1.30e-08 &      \\ 
     &           &    2 &  6.52e-08 &  3.56e-16 &      \\ 
     &           &    3 &  1.26e-08 &  1.86e-15 &      \\ 
     &           &    4 &  1.25e-08 &  3.58e-16 &      \\ 
     &           &    5 &  1.25e-08 &  3.58e-16 &      \\ 
     &           &    6 &  1.25e-08 &  3.57e-16 &      \\ 
     &           &    7 &  1.25e-08 &  3.57e-16 &      \\ 
     &           &    8 &  6.52e-08 &  3.56e-16 &      \\ 
     &           &    9 &  5.14e-08 &  1.86e-15 &      \\ 
     &           &   10 &  1.25e-08 &  1.47e-15 &      \\ 
 187 &  1.86e+03 &   10 &           &           & iters  \\ 
 \hdashline 
     &           &    1 &  2.33e-09 &  7.62e-09 &      \\ 
     &           &    2 &  2.32e-09 &  6.64e-17 &      \\ 
     &           &    3 &  2.32e-09 &  6.63e-17 &      \\ 
     &           &    4 &  2.32e-09 &  6.62e-17 &      \\ 
     &           &    5 &  2.31e-09 &  6.61e-17 &      \\ 
     &           &    6 &  2.31e-09 &  6.61e-17 &      \\ 
     &           &    7 &  2.31e-09 &  6.60e-17 &      \\ 
     &           &    8 &  2.30e-09 &  6.59e-17 &      \\ 
     &           &    9 &  2.30e-09 &  6.58e-17 &      \\ 
     &           &   10 &  2.30e-09 &  6.57e-17 &      \\ 
 188 &  1.87e+03 &   10 &           &           & iters  \\ 
 \hdashline 
     &           &    1 &  1.36e-08 &  3.32e-09 &      \\ 
     &           &    2 &  6.41e-08 &  3.88e-16 &      \\ 
     &           &    3 &  1.37e-08 &  1.83e-15 &      \\ 
     &           &    4 &  1.36e-08 &  3.90e-16 &      \\ 
     &           &    5 &  1.36e-08 &  3.89e-16 &      \\ 
     &           &    6 &  1.36e-08 &  3.89e-16 &      \\ 
     &           &    7 &  1.36e-08 &  3.88e-16 &      \\ 
     &           &    8 &  6.41e-08 &  3.88e-16 &      \\ 
     &           &    9 &  1.37e-08 &  1.83e-15 &      \\ 
     &           &   10 &  1.36e-08 &  3.90e-16 &      \\ 
 189 &  1.88e+03 &   10 &           &           & iters  \\ 
 \hdashline 
     &           &    1 &  4.87e-08 &  7.79e-09 &      \\ 
     &           &    2 &  2.91e-08 &  1.39e-15 &      \\ 
     &           &    3 &  4.86e-08 &  8.30e-16 &      \\ 
     &           &    4 &  2.91e-08 &  1.39e-15 &      \\ 
     &           &    5 &  2.91e-08 &  8.31e-16 &      \\ 
     &           &    6 &  4.87e-08 &  8.30e-16 &      \\ 
     &           &    7 &  2.91e-08 &  1.39e-15 &      \\ 
     &           &    8 &  2.91e-08 &  8.31e-16 &      \\ 
     &           &    9 &  4.87e-08 &  8.29e-16 &      \\ 
     &           &   10 &  2.91e-08 &  1.39e-15 &      \\ 
 190 &  1.89e+03 &   10 &           &           & iters  \\ 
 \hdashline 
     &           &    1 &  5.51e-08 &  1.09e-08 &      \\ 
     &           &    2 &  2.27e-08 &  1.57e-15 &      \\ 
     &           &    3 &  2.26e-08 &  6.47e-16 &      \\ 
     &           &    4 &  2.26e-08 &  6.46e-16 &      \\ 
     &           &    5 &  5.51e-08 &  6.45e-16 &      \\ 
     &           &    6 &  2.26e-08 &  1.57e-15 &      \\ 
     &           &    7 &  2.26e-08 &  6.46e-16 &      \\ 
     &           &    8 &  5.51e-08 &  6.45e-16 &      \\ 
     &           &    9 &  2.27e-08 &  1.57e-15 &      \\ 
     &           &   10 &  2.26e-08 &  6.47e-16 &      \\ 
 191 &  1.90e+03 &   10 &           &           & iters  \\ 
 \hdashline 
     &           &    1 &  8.45e-09 &  1.20e-08 &      \\ 
     &           &    2 &  8.44e-09 &  2.41e-16 &      \\ 
     &           &    3 &  8.43e-09 &  2.41e-16 &      \\ 
     &           &    4 &  8.42e-09 &  2.41e-16 &      \\ 
     &           &    5 &  8.40e-09 &  2.40e-16 &      \\ 
     &           &    6 &  8.39e-09 &  2.40e-16 &      \\ 
     &           &    7 &  8.38e-09 &  2.39e-16 &      \\ 
     &           &    8 &  8.37e-09 &  2.39e-16 &      \\ 
     &           &    9 &  6.93e-08 &  2.39e-16 &      \\ 
     &           &   10 &  8.45e-09 &  1.98e-15 &      \\ 
 192 &  1.91e+03 &   10 &           &           & iters  \\ 
 \hdashline 
     &           &    1 &  3.98e-10 &  7.64e-09 &      \\ 
     &           &    2 &  3.97e-10 &  1.14e-17 &      \\ 
     &           &    3 &  3.97e-10 &  1.13e-17 &      \\ 
     &           &    4 &  3.96e-10 &  1.13e-17 &      \\ 
     &           &    5 &  3.95e-10 &  1.13e-17 &      \\ 
     &           &    6 &  3.95e-10 &  1.13e-17 &      \\ 
     &           &    7 &  3.94e-10 &  1.13e-17 &      \\ 
     &           &    8 &  3.94e-10 &  1.13e-17 &      \\ 
     &           &    9 &  3.93e-10 &  1.12e-17 &      \\ 
     &           &   10 &  3.93e-10 &  1.12e-17 &      \\ 
 193 &  1.92e+03 &   10 &           &           & iters  \\ 
 \hdashline 
     &           &    1 &  3.66e-08 &  4.13e-08 &      \\ 
     &           &    2 &  4.11e-08 &  1.04e-15 &      \\ 
     &           &    3 &  3.66e-08 &  1.17e-15 &      \\ 
     &           &    4 &  4.11e-08 &  1.04e-15 &      \\ 
     &           &    5 &  3.66e-08 &  1.17e-15 &      \\ 
     &           &    6 &  4.11e-08 &  1.05e-15 &      \\ 
     &           &    7 &  3.66e-08 &  1.17e-15 &      \\ 
     &           &    8 &  4.11e-08 &  1.05e-15 &      \\ 
     &           &    9 &  3.66e-08 &  1.17e-15 &      \\ 
     &           &   10 &  4.11e-08 &  1.05e-15 &      \\ 
 194 &  1.93e+03 &   10 &           &           & iters  \\ 
 \hdashline 
     &           &    1 &  5.60e-08 &  2.98e-08 &      \\ 
     &           &    2 &  2.18e-08 &  1.60e-15 &      \\ 
     &           &    3 &  2.18e-08 &  6.23e-16 &      \\ 
     &           &    4 &  2.18e-08 &  6.22e-16 &      \\ 
     &           &    5 &  5.60e-08 &  6.21e-16 &      \\ 
     &           &    6 &  2.18e-08 &  1.60e-15 &      \\ 
     &           &    7 &  2.18e-08 &  6.22e-16 &      \\ 
     &           &    8 &  5.59e-08 &  6.21e-16 &      \\ 
     &           &    9 &  2.18e-08 &  1.60e-15 &      \\ 
     &           &   10 &  2.18e-08 &  6.23e-16 &      \\ 
 195 &  1.94e+03 &   10 &           &           & iters  \\ 
 \hdashline 
     &           &    1 &  7.71e-09 &  1.17e-09 &      \\ 
     &           &    2 &  3.11e-08 &  2.20e-16 &      \\ 
     &           &    3 &  7.74e-09 &  8.89e-16 &      \\ 
     &           &    4 &  7.73e-09 &  2.21e-16 &      \\ 
     &           &    5 &  7.72e-09 &  2.21e-16 &      \\ 
     &           &    6 &  7.71e-09 &  2.20e-16 &      \\ 
     &           &    7 &  3.11e-08 &  2.20e-16 &      \\ 
     &           &    8 &  7.74e-09 &  8.89e-16 &      \\ 
     &           &    9 &  7.73e-09 &  2.21e-16 &      \\ 
     &           &   10 &  7.72e-09 &  2.21e-16 &      \\ 
 196 &  1.95e+03 &   10 &           &           & iters  \\ 
 \hdashline 
     &           &    1 &  1.74e-08 &  5.72e-09 &      \\ 
     &           &    2 &  2.14e-08 &  4.97e-16 &      \\ 
     &           &    3 &  1.74e-08 &  6.12e-16 &      \\ 
     &           &    4 &  2.14e-08 &  4.98e-16 &      \\ 
     &           &    5 &  1.74e-08 &  6.12e-16 &      \\ 
     &           &    6 &  2.14e-08 &  4.98e-16 &      \\ 
     &           &    7 &  1.74e-08 &  6.12e-16 &      \\ 
     &           &    8 &  2.14e-08 &  4.98e-16 &      \\ 
     &           &    9 &  1.75e-08 &  6.11e-16 &      \\ 
     &           &   10 &  1.74e-08 &  4.98e-16 &      \\ 
 197 &  1.96e+03 &   10 &           &           & iters  \\ 
 \hdashline 
     &           &    1 &  1.86e-08 &  8.96e-09 &      \\ 
     &           &    2 &  5.91e-08 &  5.32e-16 &      \\ 
     &           &    3 &  1.87e-08 &  1.69e-15 &      \\ 
     &           &    4 &  1.87e-08 &  5.34e-16 &      \\ 
     &           &    5 &  1.86e-08 &  5.33e-16 &      \\ 
     &           &    6 &  5.91e-08 &  5.32e-16 &      \\ 
     &           &    7 &  1.87e-08 &  1.69e-15 &      \\ 
     &           &    8 &  1.87e-08 &  5.34e-16 &      \\ 
     &           &    9 &  1.86e-08 &  5.33e-16 &      \\ 
     &           &   10 &  5.91e-08 &  5.32e-16 &      \\ 
 198 &  1.97e+03 &   10 &           &           & iters  \\ 
 \hdashline 
     &           &    1 &  2.30e-08 &  1.07e-08 &      \\ 
     &           &    2 &  2.29e-08 &  6.56e-16 &      \\ 
     &           &    3 &  1.59e-08 &  6.55e-16 &      \\ 
     &           &    4 &  2.29e-08 &  4.55e-16 &      \\ 
     &           &    5 &  1.59e-08 &  6.54e-16 &      \\ 
     &           &    6 &  1.59e-08 &  4.55e-16 &      \\ 
     &           &    7 &  2.29e-08 &  4.54e-16 &      \\ 
     &           &    8 &  1.59e-08 &  6.55e-16 &      \\ 
     &           &    9 &  2.29e-08 &  4.55e-16 &      \\ 
     &           &   10 &  1.59e-08 &  6.55e-16 &      \\ 
 199 &  1.98e+03 &   10 &           &           & iters  \\ 
 \hdashline 
     &           &    1 &  2.06e-08 &  5.56e-09 &      \\ 
     &           &    2 &  5.71e-08 &  5.89e-16 &      \\ 
     &           &    3 &  2.07e-08 &  1.63e-15 &      \\ 
     &           &    4 &  2.07e-08 &  5.90e-16 &      \\ 
     &           &    5 &  5.71e-08 &  5.89e-16 &      \\ 
     &           &    6 &  2.07e-08 &  1.63e-15 &      \\ 
     &           &    7 &  2.07e-08 &  5.91e-16 &      \\ 
     &           &    8 &  2.06e-08 &  5.90e-16 &      \\ 
     &           &    9 &  5.71e-08 &  5.89e-16 &      \\ 
     &           &   10 &  2.07e-08 &  1.63e-15 &      \\ 
 200 &  1.99e+03 &   10 &           &           & iters  \\ 
 \hdashline 
     &           &    1 &  4.65e-08 &  2.62e-09 &      \\ 
     &           &    2 &  3.12e-08 &  1.33e-15 &      \\ 
     &           &    3 &  4.65e-08 &  8.91e-16 &      \\ 
     &           &    4 &  3.12e-08 &  1.33e-15 &      \\ 
     &           &    5 &  3.12e-08 &  8.91e-16 &      \\ 
     &           &    6 &  4.65e-08 &  8.90e-16 &      \\ 
     &           &    7 &  3.12e-08 &  1.33e-15 &      \\ 
     &           &    8 &  4.65e-08 &  8.91e-16 &      \\ 
     &           &    9 &  3.12e-08 &  1.33e-15 &      \\ 
     &           &   10 &  3.12e-08 &  8.91e-16 &      \\ 
 201 &  2.00e+03 &   10 &           &           & iters  \\ 
 \hdashline 
     &           &    1 &  1.03e-08 &  4.87e-09 &      \\ 
     &           &    2 &  1.03e-08 &  2.94e-16 &      \\ 
     &           &    3 &  1.03e-08 &  2.94e-16 &      \\ 
     &           &    4 &  1.03e-08 &  2.93e-16 &      \\ 
     &           &    5 &  1.02e-08 &  2.93e-16 &      \\ 
     &           &    6 &  1.02e-08 &  2.92e-16 &      \\ 
     &           &    7 &  6.75e-08 &  2.92e-16 &      \\ 
     &           &    8 &  4.92e-08 &  1.93e-15 &      \\ 
     &           &    9 &  1.02e-08 &  1.40e-15 &      \\ 
     &           &   10 &  1.02e-08 &  2.92e-16 &      \\ 
 202 &  2.01e+03 &   10 &           &           & iters  \\ 
 \hdashline 
     &           &    1 &  5.37e-10 &  1.42e-08 &      \\ 
     &           &    2 &  5.36e-10 &  1.53e-17 &      \\ 
     &           &    3 &  5.35e-10 &  1.53e-17 &      \\ 
     &           &    4 &  5.34e-10 &  1.53e-17 &      \\ 
     &           &    5 &  5.34e-10 &  1.53e-17 &      \\ 
     &           &    6 &  5.33e-10 &  1.52e-17 &      \\ 
     &           &    7 &  5.32e-10 &  1.52e-17 &      \\ 
     &           &    8 &  5.31e-10 &  1.52e-17 &      \\ 
     &           &    9 &  7.72e-08 &  1.52e-17 &      \\ 
     &           &   10 &  7.83e-08 &  2.20e-15 &      \\ 
 203 &  2.02e+03 &   10 &           &           & iters  \\ 
 \hdashline 
     &           &    1 &  7.72e-08 &  1.88e-08 &      \\ 
     &           &    2 &  6.05e-10 &  2.20e-15 &      \\ 
     &           &    3 &  6.04e-10 &  1.73e-17 &      \\ 
     &           &    4 &  6.03e-10 &  1.72e-17 &      \\ 
     &           &    5 &  6.02e-10 &  1.72e-17 &      \\ 
     &           &    6 &  6.01e-10 &  1.72e-17 &      \\ 
     &           &    7 &  6.01e-10 &  1.72e-17 &      \\ 
     &           &    8 &  6.00e-10 &  1.71e-17 &      \\ 
     &           &    9 &  5.99e-10 &  1.71e-17 &      \\ 
     &           &   10 &  5.98e-10 &  1.71e-17 &      \\ 
 204 &  2.03e+03 &   10 &           &           & iters  \\ 
 \hdashline 
     &           &    1 &  3.06e-08 &  1.43e-08 &      \\ 
     &           &    2 &  3.05e-08 &  8.73e-16 &      \\ 
     &           &    3 &  4.72e-08 &  8.71e-16 &      \\ 
     &           &    4 &  3.06e-08 &  1.35e-15 &      \\ 
     &           &    5 &  4.72e-08 &  8.72e-16 &      \\ 
     &           &    6 &  3.06e-08 &  1.35e-15 &      \\ 
     &           &    7 &  3.05e-08 &  8.73e-16 &      \\ 
     &           &    8 &  4.72e-08 &  8.72e-16 &      \\ 
     &           &    9 &  3.06e-08 &  1.35e-15 &      \\ 
     &           &   10 &  3.05e-08 &  8.72e-16 &      \\ 
 205 &  2.04e+03 &   10 &           &           & iters  \\ 
 \hdashline 
     &           &    1 &  3.64e-08 &  8.17e-09 &      \\ 
     &           &    2 &  3.63e-08 &  1.04e-15 &      \\ 
     &           &    3 &  4.14e-08 &  1.04e-15 &      \\ 
     &           &    4 &  3.63e-08 &  1.18e-15 &      \\ 
     &           &    5 &  4.14e-08 &  1.04e-15 &      \\ 
     &           &    6 &  3.63e-08 &  1.18e-15 &      \\ 
     &           &    7 &  4.14e-08 &  1.04e-15 &      \\ 
     &           &    8 &  3.64e-08 &  1.18e-15 &      \\ 
     &           &    9 &  4.14e-08 &  1.04e-15 &      \\ 
     &           &   10 &  3.64e-08 &  1.18e-15 &      \\ 
 206 &  2.05e+03 &   10 &           &           & iters  \\ 
 \hdashline 
     &           &    1 &  1.38e-08 &  3.43e-08 &      \\ 
     &           &    2 &  2.50e-08 &  3.94e-16 &      \\ 
     &           &    3 &  1.38e-08 &  7.15e-16 &      \\ 
     &           &    4 &  1.38e-08 &  3.95e-16 &      \\ 
     &           &    5 &  2.51e-08 &  3.94e-16 &      \\ 
     &           &    6 &  1.38e-08 &  7.15e-16 &      \\ 
     &           &    7 &  1.38e-08 &  3.95e-16 &      \\ 
     &           &    8 &  2.51e-08 &  3.94e-16 &      \\ 
     &           &    9 &  1.38e-08 &  7.15e-16 &      \\ 
     &           &   10 &  1.38e-08 &  3.95e-16 &      \\ 
 207 &  2.06e+03 &   10 &           &           & iters  \\ 
 \hdashline 
     &           &    1 &  2.65e-08 &  1.65e-08 &      \\ 
     &           &    2 &  2.64e-08 &  7.56e-16 &      \\ 
     &           &    3 &  5.13e-08 &  7.55e-16 &      \\ 
     &           &    4 &  2.65e-08 &  1.46e-15 &      \\ 
     &           &    5 &  2.64e-08 &  7.56e-16 &      \\ 
     &           &    6 &  5.13e-08 &  7.55e-16 &      \\ 
     &           &    7 &  2.65e-08 &  1.46e-15 &      \\ 
     &           &    8 &  2.64e-08 &  7.56e-16 &      \\ 
     &           &    9 &  5.13e-08 &  7.54e-16 &      \\ 
     &           &   10 &  2.65e-08 &  1.46e-15 &      \\ 
 208 &  2.07e+03 &   10 &           &           & iters  \\ 
 \hdashline 
     &           &    1 &  4.28e-08 &  2.02e-08 &      \\ 
     &           &    2 &  3.49e-08 &  1.22e-15 &      \\ 
     &           &    3 &  4.28e-08 &  9.97e-16 &      \\ 
     &           &    4 &  3.50e-08 &  1.22e-15 &      \\ 
     &           &    5 &  3.49e-08 &  9.98e-16 &      \\ 
     &           &    6 &  4.28e-08 &  9.96e-16 &      \\ 
     &           &    7 &  3.49e-08 &  1.22e-15 &      \\ 
     &           &    8 &  4.28e-08 &  9.97e-16 &      \\ 
     &           &    9 &  3.49e-08 &  1.22e-15 &      \\ 
     &           &   10 &  4.28e-08 &  9.97e-16 &      \\ 
 209 &  2.08e+03 &   10 &           &           & iters  \\ 
 \hdashline 
     &           &    1 &  2.25e-08 &  2.16e-08 &      \\ 
     &           &    2 &  2.25e-08 &  6.43e-16 &      \\ 
     &           &    3 &  2.25e-08 &  6.42e-16 &      \\ 
     &           &    4 &  5.53e-08 &  6.41e-16 &      \\ 
     &           &    5 &  2.25e-08 &  1.58e-15 &      \\ 
     &           &    6 &  2.25e-08 &  6.42e-16 &      \\ 
     &           &    7 &  5.53e-08 &  6.41e-16 &      \\ 
     &           &    8 &  2.25e-08 &  1.58e-15 &      \\ 
     &           &    9 &  2.25e-08 &  6.43e-16 &      \\ 
     &           &   10 &  2.25e-08 &  6.42e-16 &      \\ 
 210 &  2.09e+03 &   10 &           &           & iters  \\ 
 \hdashline 
     &           &    1 &  1.95e-08 &  7.06e-09 &      \\ 
     &           &    2 &  1.95e-08 &  5.56e-16 &      \\ 
     &           &    3 &  1.94e-08 &  5.56e-16 &      \\ 
     &           &    4 &  1.94e-08 &  5.55e-16 &      \\ 
     &           &    5 &  1.94e-08 &  5.54e-16 &      \\ 
     &           &    6 &  5.83e-08 &  5.53e-16 &      \\ 
     &           &    7 &  1.94e-08 &  1.66e-15 &      \\ 
     &           &    8 &  1.94e-08 &  5.55e-16 &      \\ 
     &           &    9 &  1.94e-08 &  5.54e-16 &      \\ 
     &           &   10 &  5.83e-08 &  5.53e-16 &      \\ 
 211 &  2.10e+03 &   10 &           &           & iters  \\ 
 \hdashline 
     &           &    1 &  1.38e-08 &  1.05e-08 &      \\ 
     &           &    2 &  1.38e-08 &  3.93e-16 &      \\ 
     &           &    3 &  1.37e-08 &  3.92e-16 &      \\ 
     &           &    4 &  1.37e-08 &  3.92e-16 &      \\ 
     &           &    5 &  1.37e-08 &  3.91e-16 &      \\ 
     &           &    6 &  6.40e-08 &  3.91e-16 &      \\ 
     &           &    7 &  1.38e-08 &  1.83e-15 &      \\ 
     &           &    8 &  1.37e-08 &  3.93e-16 &      \\ 
     &           &    9 &  1.37e-08 &  3.92e-16 &      \\ 
     &           &   10 &  1.37e-08 &  3.92e-16 &      \\ 
 212 &  2.11e+03 &   10 &           &           & iters  \\ 
 \hdashline 
     &           &    1 &  6.86e-08 &  1.39e-08 &      \\ 
     &           &    2 &  9.20e-09 &  1.96e-15 &      \\ 
     &           &    3 &  9.19e-09 &  2.63e-16 &      \\ 
     &           &    4 &  9.17e-09 &  2.62e-16 &      \\ 
     &           &    5 &  6.85e-08 &  2.62e-16 &      \\ 
     &           &    6 &  8.69e-08 &  1.96e-15 &      \\ 
     &           &    7 &  6.86e-08 &  2.48e-15 &      \\ 
     &           &    8 &  9.23e-09 &  1.96e-15 &      \\ 
     &           &    9 &  9.22e-09 &  2.64e-16 &      \\ 
     &           &   10 &  9.21e-09 &  2.63e-16 &      \\ 
 213 &  2.12e+03 &   10 &           &           & iters  \\ 
 \hdashline 
     &           &    1 &  9.65e-09 &  3.97e-09 &      \\ 
     &           &    2 &  9.64e-09 &  2.76e-16 &      \\ 
     &           &    3 &  9.63e-09 &  2.75e-16 &      \\ 
     &           &    4 &  6.81e-08 &  2.75e-16 &      \\ 
     &           &    5 &  9.71e-09 &  1.94e-15 &      \\ 
     &           &    6 &  9.70e-09 &  2.77e-16 &      \\ 
     &           &    7 &  9.68e-09 &  2.77e-16 &      \\ 
     &           &    8 &  9.67e-09 &  2.76e-16 &      \\ 
     &           &    9 &  9.66e-09 &  2.76e-16 &      \\ 
     &           &   10 &  9.64e-09 &  2.76e-16 &      \\ 
 214 &  2.13e+03 &   10 &           &           & iters  \\ 
 \hdashline 
     &           &    1 &  5.88e-08 &  3.22e-09 &      \\ 
     &           &    2 &  1.90e-08 &  1.68e-15 &      \\ 
     &           &    3 &  1.90e-08 &  5.42e-16 &      \\ 
     &           &    4 &  1.89e-08 &  5.41e-16 &      \\ 
     &           &    5 &  5.88e-08 &  5.40e-16 &      \\ 
     &           &    6 &  1.90e-08 &  1.68e-15 &      \\ 
     &           &    7 &  1.90e-08 &  5.42e-16 &      \\ 
     &           &    8 &  1.89e-08 &  5.41e-16 &      \\ 
     &           &    9 &  5.88e-08 &  5.40e-16 &      \\ 
     &           &   10 &  1.90e-08 &  1.68e-15 &      \\ 
 215 &  2.14e+03 &   10 &           &           & iters  \\ 
 \hdashline 
     &           &    1 &  5.61e-08 &  9.28e-10 &      \\ 
     &           &    2 &  2.17e-08 &  1.60e-15 &      \\ 
     &           &    3 &  2.16e-08 &  6.19e-16 &      \\ 
     &           &    4 &  5.61e-08 &  6.18e-16 &      \\ 
     &           &    5 &  2.17e-08 &  1.60e-15 &      \\ 
     &           &    6 &  2.17e-08 &  6.19e-16 &      \\ 
     &           &    7 &  2.16e-08 &  6.18e-16 &      \\ 
     &           &    8 &  5.61e-08 &  6.17e-16 &      \\ 
     &           &    9 &  2.17e-08 &  1.60e-15 &      \\ 
     &           &   10 &  2.17e-08 &  6.19e-16 &      \\ 
 216 &  2.15e+03 &   10 &           &           & iters  \\ 
 \hdashline 
     &           &    1 &  6.52e-08 &  1.94e-09 &      \\ 
     &           &    2 &  1.26e-08 &  1.86e-15 &      \\ 
     &           &    3 &  1.26e-08 &  3.60e-16 &      \\ 
     &           &    4 &  1.26e-08 &  3.59e-16 &      \\ 
     &           &    5 &  1.25e-08 &  3.59e-16 &      \\ 
     &           &    6 &  1.25e-08 &  3.58e-16 &      \\ 
     &           &    7 &  6.52e-08 &  3.58e-16 &      \\ 
     &           &    8 &  9.03e-08 &  1.86e-15 &      \\ 
     &           &    9 &  6.52e-08 &  2.58e-15 &      \\ 
     &           &   10 &  1.26e-08 &  1.86e-15 &      \\ 
 217 &  2.16e+03 &   10 &           &           & iters  \\ 
 \hdashline 
     &           &    1 &  1.12e-08 &  2.26e-09 &      \\ 
     &           &    2 &  1.11e-08 &  3.19e-16 &      \\ 
     &           &    3 &  1.11e-08 &  3.18e-16 &      \\ 
     &           &    4 &  6.66e-08 &  3.18e-16 &      \\ 
     &           &    5 &  1.12e-08 &  1.90e-15 &      \\ 
     &           &    6 &  1.12e-08 &  3.20e-16 &      \\ 
     &           &    7 &  1.12e-08 &  3.20e-16 &      \\ 
     &           &    8 &  1.12e-08 &  3.19e-16 &      \\ 
     &           &    9 &  1.11e-08 &  3.19e-16 &      \\ 
     &           &   10 &  1.11e-08 &  3.18e-16 &      \\ 
 218 &  2.17e+03 &   10 &           &           & iters  \\ 
 \hdashline 
     &           &    1 &  2.32e-08 &  4.77e-09 &      \\ 
     &           &    2 &  2.31e-08 &  6.61e-16 &      \\ 
     &           &    3 &  5.46e-08 &  6.60e-16 &      \\ 
     &           &    4 &  2.32e-08 &  1.56e-15 &      \\ 
     &           &    5 &  2.31e-08 &  6.61e-16 &      \\ 
     &           &    6 &  5.46e-08 &  6.60e-16 &      \\ 
     &           &    7 &  2.32e-08 &  1.56e-15 &      \\ 
     &           &    8 &  2.32e-08 &  6.62e-16 &      \\ 
     &           &    9 &  2.31e-08 &  6.61e-16 &      \\ 
     &           &   10 &  5.46e-08 &  6.60e-16 &      \\ 
 219 &  2.18e+03 &   10 &           &           & iters  \\ 
 \hdashline 
     &           &    1 &  4.52e-08 &  2.98e-09 &      \\ 
     &           &    2 &  3.25e-08 &  1.29e-15 &      \\ 
     &           &    3 &  4.52e-08 &  9.28e-16 &      \\ 
     &           &    4 &  3.25e-08 &  1.29e-15 &      \\ 
     &           &    5 &  4.52e-08 &  9.29e-16 &      \\ 
     &           &    6 &  3.26e-08 &  1.29e-15 &      \\ 
     &           &    7 &  3.25e-08 &  9.29e-16 &      \\ 
     &           &    8 &  4.52e-08 &  9.28e-16 &      \\ 
     &           &    9 &  3.25e-08 &  1.29e-15 &      \\ 
     &           &   10 &  4.52e-08 &  9.28e-16 &      \\ 
 220 &  2.19e+03 &   10 &           &           & iters  \\ 
 \hdashline 
     &           &    1 &  2.56e-08 &  6.93e-09 &      \\ 
     &           &    2 &  2.56e-08 &  7.31e-16 &      \\ 
     &           &    3 &  2.55e-08 &  7.30e-16 &      \\ 
     &           &    4 &  5.22e-08 &  7.29e-16 &      \\ 
     &           &    5 &  2.56e-08 &  1.49e-15 &      \\ 
     &           &    6 &  2.55e-08 &  7.30e-16 &      \\ 
     &           &    7 &  5.22e-08 &  7.29e-16 &      \\ 
     &           &    8 &  2.56e-08 &  1.49e-15 &      \\ 
     &           &    9 &  2.55e-08 &  7.30e-16 &      \\ 
     &           &   10 &  5.22e-08 &  7.29e-16 &      \\ 
 221 &  2.20e+03 &   10 &           &           & iters  \\ 
 \hdashline 
     &           &    1 &  4.79e-09 &  1.39e-08 &      \\ 
     &           &    2 &  4.78e-09 &  1.37e-16 &      \\ 
     &           &    3 &  4.78e-09 &  1.37e-16 &      \\ 
     &           &    4 &  4.77e-09 &  1.36e-16 &      \\ 
     &           &    5 &  4.76e-09 &  1.36e-16 &      \\ 
     &           &    6 &  4.76e-09 &  1.36e-16 &      \\ 
     &           &    7 &  7.29e-08 &  1.36e-16 &      \\ 
     &           &    8 &  8.25e-08 &  2.08e-15 &      \\ 
     &           &    9 &  7.30e-08 &  2.36e-15 &      \\ 
     &           &   10 &  4.84e-09 &  2.08e-15 &      \\ 
 222 &  2.21e+03 &   10 &           &           & iters  \\ 
 \hdashline 
     &           &    1 &  3.58e-08 &  1.34e-08 &      \\ 
     &           &    2 &  4.20e-08 &  1.02e-15 &      \\ 
     &           &    3 &  3.58e-08 &  1.20e-15 &      \\ 
     &           &    4 &  4.20e-08 &  1.02e-15 &      \\ 
     &           &    5 &  3.58e-08 &  1.20e-15 &      \\ 
     &           &    6 &  4.20e-08 &  1.02e-15 &      \\ 
     &           &    7 &  3.58e-08 &  1.20e-15 &      \\ 
     &           &    8 &  4.19e-08 &  1.02e-15 &      \\ 
     &           &    9 &  3.58e-08 &  1.20e-15 &      \\ 
     &           &   10 &  3.58e-08 &  1.02e-15 &      \\ 
 223 &  2.22e+03 &   10 &           &           & iters  \\ 
 \hdashline 
     &           &    1 &  7.30e-08 &  2.63e-09 &      \\ 
     &           &    2 &  4.78e-09 &  2.08e-15 &      \\ 
     &           &    3 &  4.77e-09 &  1.36e-16 &      \\ 
     &           &    4 &  4.77e-09 &  1.36e-16 &      \\ 
     &           &    5 &  4.76e-09 &  1.36e-16 &      \\ 
     &           &    6 &  4.75e-09 &  1.36e-16 &      \\ 
     &           &    7 &  4.75e-09 &  1.36e-16 &      \\ 
     &           &    8 &  4.74e-09 &  1.35e-16 &      \\ 
     &           &    9 &  4.73e-09 &  1.35e-16 &      \\ 
     &           &   10 &  4.73e-09 &  1.35e-16 &      \\ 
 224 &  2.23e+03 &   10 &           &           & iters  \\ 
 \hdashline 
     &           &    1 &  6.38e-08 &  8.04e-09 &      \\ 
     &           &    2 &  1.40e-08 &  1.82e-15 &      \\ 
     &           &    3 &  1.40e-08 &  4.00e-16 &      \\ 
     &           &    4 &  1.40e-08 &  4.00e-16 &      \\ 
     &           &    5 &  1.40e-08 &  3.99e-16 &      \\ 
     &           &    6 &  6.37e-08 &  3.99e-16 &      \\ 
     &           &    7 &  1.40e-08 &  1.82e-15 &      \\ 
     &           &    8 &  1.40e-08 &  4.01e-16 &      \\ 
     &           &    9 &  1.40e-08 &  4.00e-16 &      \\ 
     &           &   10 &  1.40e-08 &  4.00e-16 &      \\ 
 225 &  2.24e+03 &   10 &           &           & iters  \\ 
 \hdashline 
     &           &    1 &  1.02e-08 &  4.92e-09 &      \\ 
     &           &    2 &  1.02e-08 &  2.91e-16 &      \\ 
     &           &    3 &  1.02e-08 &  2.91e-16 &      \\ 
     &           &    4 &  1.02e-08 &  2.90e-16 &      \\ 
     &           &    5 &  1.01e-08 &  2.90e-16 &      \\ 
     &           &    6 &  6.76e-08 &  2.89e-16 &      \\ 
     &           &    7 &  8.79e-08 &  1.93e-15 &      \\ 
     &           &    8 &  6.76e-08 &  2.51e-15 &      \\ 
     &           &    9 &  1.02e-08 &  1.93e-15 &      \\ 
     &           &   10 &  1.02e-08 &  2.91e-16 &      \\ 
 226 &  2.25e+03 &   10 &           &           & iters  \\ 
 \hdashline 
     &           &    1 &  2.13e-08 &  1.24e-09 &      \\ 
     &           &    2 &  5.64e-08 &  6.08e-16 &      \\ 
     &           &    3 &  2.14e-08 &  1.61e-15 &      \\ 
     &           &    4 &  2.13e-08 &  6.09e-16 &      \\ 
     &           &    5 &  5.64e-08 &  6.09e-16 &      \\ 
     &           &    6 &  2.14e-08 &  1.61e-15 &      \\ 
     &           &    7 &  2.13e-08 &  6.10e-16 &      \\ 
     &           &    8 &  2.13e-08 &  6.09e-16 &      \\ 
     &           &    9 &  5.64e-08 &  6.08e-16 &      \\ 
     &           &   10 &  2.14e-08 &  1.61e-15 &      \\ 
 227 &  2.26e+03 &   10 &           &           & iters  \\ 
 \hdashline 
     &           &    1 &  4.80e-08 &  3.91e-10 &      \\ 
     &           &    2 &  2.98e-08 &  1.37e-15 &      \\ 
     &           &    3 &  4.79e-08 &  8.51e-16 &      \\ 
     &           &    4 &  2.98e-08 &  1.37e-15 &      \\ 
     &           &    5 &  2.98e-08 &  8.51e-16 &      \\ 
     &           &    6 &  4.79e-08 &  8.50e-16 &      \\ 
     &           &    7 &  2.98e-08 &  1.37e-15 &      \\ 
     &           &    8 &  4.79e-08 &  8.51e-16 &      \\ 
     &           &    9 &  2.98e-08 &  1.37e-15 &      \\ 
     &           &   10 &  2.98e-08 &  8.52e-16 &      \\ 
 228 &  2.27e+03 &   10 &           &           & iters  \\ 
 \hdashline 
     &           &    1 &  4.99e-08 &  9.82e-09 &      \\ 
     &           &    2 &  2.78e-08 &  1.42e-15 &      \\ 
     &           &    3 &  2.78e-08 &  7.94e-16 &      \\ 
     &           &    4 &  4.99e-08 &  7.93e-16 &      \\ 
     &           &    5 &  2.78e-08 &  1.43e-15 &      \\ 
     &           &    6 &  2.78e-08 &  7.94e-16 &      \\ 
     &           &    7 &  4.99e-08 &  7.93e-16 &      \\ 
     &           &    8 &  2.78e-08 &  1.43e-15 &      \\ 
     &           &    9 &  4.99e-08 &  7.94e-16 &      \\ 
     &           &   10 &  2.78e-08 &  1.42e-15 &      \\ 
 229 &  2.28e+03 &   10 &           &           & iters  \\ 
 \hdashline 
     &           &    1 &  5.44e-09 &  1.13e-08 &      \\ 
     &           &    2 &  5.43e-09 &  1.55e-16 &      \\ 
     &           &    3 &  5.43e-09 &  1.55e-16 &      \\ 
     &           &    4 &  5.42e-09 &  1.55e-16 &      \\ 
     &           &    5 &  5.41e-09 &  1.55e-16 &      \\ 
     &           &    6 &  5.40e-09 &  1.54e-16 &      \\ 
     &           &    7 &  5.40e-09 &  1.54e-16 &      \\ 
     &           &    8 &  5.39e-09 &  1.54e-16 &      \\ 
     &           &    9 &  5.38e-09 &  1.54e-16 &      \\ 
     &           &   10 &  5.37e-09 &  1.54e-16 &      \\ 
 230 &  2.29e+03 &   10 &           &           & iters  \\ 
 \hdashline 
     &           &    1 &  3.12e-08 &  9.43e-09 &      \\ 
     &           &    2 &  7.72e-09 &  8.90e-16 &      \\ 
     &           &    3 &  7.70e-09 &  2.20e-16 &      \\ 
     &           &    4 &  3.12e-08 &  2.20e-16 &      \\ 
     &           &    5 &  7.74e-09 &  8.89e-16 &      \\ 
     &           &    6 &  7.73e-09 &  2.21e-16 &      \\ 
     &           &    7 &  7.72e-09 &  2.21e-16 &      \\ 
     &           &    8 &  7.70e-09 &  2.20e-16 &      \\ 
     &           &    9 &  3.12e-08 &  2.20e-16 &      \\ 
     &           &   10 &  7.74e-09 &  8.89e-16 &      \\ 
 231 &  2.30e+03 &   10 &           &           & iters  \\ 
 \hdashline 
     &           &    1 &  2.47e-09 &  1.26e-09 &      \\ 
     &           &    2 &  2.46e-09 &  7.04e-17 &      \\ 
     &           &    3 &  2.46e-09 &  7.03e-17 &      \\ 
     &           &    4 &  2.45e-09 &  7.02e-17 &      \\ 
     &           &    5 &  2.45e-09 &  7.01e-17 &      \\ 
     &           &    6 &  2.45e-09 &  7.00e-17 &      \\ 
     &           &    7 &  2.44e-09 &  6.99e-17 &      \\ 
     &           &    8 &  2.44e-09 &  6.98e-17 &      \\ 
     &           &    9 &  2.44e-09 &  6.97e-17 &      \\ 
     &           &   10 &  2.43e-09 &  6.96e-17 &      \\ 
 232 &  2.31e+03 &   10 &           &           & iters  \\ 
 \hdashline 
     &           &    1 &  8.46e-09 &  2.31e-08 &      \\ 
     &           &    2 &  8.45e-09 &  2.42e-16 &      \\ 
     &           &    3 &  8.44e-09 &  2.41e-16 &      \\ 
     &           &    4 &  8.43e-09 &  2.41e-16 &      \\ 
     &           &    5 &  6.93e-08 &  2.40e-16 &      \\ 
     &           &    6 &  8.62e-08 &  1.98e-15 &      \\ 
     &           &    7 &  6.93e-08 &  2.46e-15 &      \\ 
     &           &    8 &  8.49e-09 &  1.98e-15 &      \\ 
     &           &    9 &  8.48e-09 &  2.42e-16 &      \\ 
     &           &   10 &  8.46e-09 &  2.42e-16 &      \\ 
 233 &  2.32e+03 &   10 &           &           & iters  \\ 
 \hdashline 
     &           &    1 &  4.54e-08 &  2.61e-08 &      \\ 
     &           &    2 &  3.24e-08 &  1.30e-15 &      \\ 
     &           &    3 &  4.54e-08 &  9.24e-16 &      \\ 
     &           &    4 &  3.24e-08 &  1.29e-15 &      \\ 
     &           &    5 &  3.23e-08 &  9.25e-16 &      \\ 
     &           &    6 &  4.54e-08 &  9.23e-16 &      \\ 
     &           &    7 &  3.24e-08 &  1.30e-15 &      \\ 
     &           &    8 &  4.54e-08 &  9.24e-16 &      \\ 
     &           &    9 &  3.24e-08 &  1.29e-15 &      \\ 
     &           &   10 &  3.23e-08 &  9.24e-16 &      \\ 
 234 &  2.33e+03 &   10 &           &           & iters  \\ 
 \hdashline 
     &           &    1 &  2.55e-08 &  1.16e-08 &      \\ 
     &           &    2 &  2.54e-08 &  7.27e-16 &      \\ 
     &           &    3 &  2.54e-08 &  7.26e-16 &      \\ 
     &           &    4 &  5.23e-08 &  7.25e-16 &      \\ 
     &           &    5 &  2.54e-08 &  1.49e-15 &      \\ 
     &           &    6 &  2.54e-08 &  7.26e-16 &      \\ 
     &           &    7 &  5.23e-08 &  7.25e-16 &      \\ 
     &           &    8 &  2.54e-08 &  1.49e-15 &      \\ 
     &           &    9 &  2.54e-08 &  7.26e-16 &      \\ 
     &           &   10 &  5.23e-08 &  7.25e-16 &      \\ 
 235 &  2.34e+03 &   10 &           &           & iters  \\ 
 \hdashline 
     &           &    1 &  2.74e-08 &  1.23e-08 &      \\ 
     &           &    2 &  2.74e-08 &  7.83e-16 &      \\ 
     &           &    3 &  5.03e-08 &  7.82e-16 &      \\ 
     &           &    4 &  2.74e-08 &  1.44e-15 &      \\ 
     &           &    5 &  2.74e-08 &  7.82e-16 &      \\ 
     &           &    6 &  5.04e-08 &  7.81e-16 &      \\ 
     &           &    7 &  2.74e-08 &  1.44e-15 &      \\ 
     &           &    8 &  5.03e-08 &  7.82e-16 &      \\ 
     &           &    9 &  2.74e-08 &  1.44e-15 &      \\ 
     &           &   10 &  2.74e-08 &  7.83e-16 &      \\ 
 236 &  2.35e+03 &   10 &           &           & iters  \\ 
 \hdashline 
     &           &    1 &  2.90e-08 &  3.67e-09 &      \\ 
     &           &    2 &  2.89e-08 &  8.27e-16 &      \\ 
     &           &    3 &  2.89e-08 &  8.26e-16 &      \\ 
     &           &    4 &  4.88e-08 &  8.24e-16 &      \\ 
     &           &    5 &  2.89e-08 &  1.39e-15 &      \\ 
     &           &    6 &  4.88e-08 &  8.25e-16 &      \\ 
     &           &    7 &  2.89e-08 &  1.39e-15 &      \\ 
     &           &    8 &  2.89e-08 &  8.26e-16 &      \\ 
     &           &    9 &  4.88e-08 &  8.25e-16 &      \\ 
     &           &   10 &  2.89e-08 &  1.39e-15 &      \\ 
 237 &  2.36e+03 &   10 &           &           & iters  \\ 
 \hdashline 
     &           &    1 &  2.75e-09 &  1.23e-08 &      \\ 
     &           &    2 &  2.75e-09 &  7.85e-17 &      \\ 
     &           &    3 &  2.74e-09 &  7.84e-17 &      \\ 
     &           &    4 &  2.74e-09 &  7.82e-17 &      \\ 
     &           &    5 &  2.73e-09 &  7.81e-17 &      \\ 
     &           &    6 &  2.73e-09 &  7.80e-17 &      \\ 
     &           &    7 &  2.73e-09 &  7.79e-17 &      \\ 
     &           &    8 &  7.50e-08 &  7.78e-17 &      \\ 
     &           &    9 &  8.05e-08 &  2.14e-15 &      \\ 
     &           &   10 &  7.50e-08 &  2.30e-15 &      \\ 
 238 &  2.37e+03 &   10 &           &           & iters  \\ 
 \hdashline 
     &           &    1 &  5.23e-08 &  1.08e-08 &      \\ 
     &           &    2 &  2.54e-08 &  1.49e-15 &      \\ 
     &           &    3 &  5.23e-08 &  7.26e-16 &      \\ 
     &           &    4 &  2.55e-08 &  1.49e-15 &      \\ 
     &           &    5 &  2.54e-08 &  7.27e-16 &      \\ 
     &           &    6 &  5.23e-08 &  7.26e-16 &      \\ 
     &           &    7 &  2.55e-08 &  1.49e-15 &      \\ 
     &           &    8 &  2.54e-08 &  7.27e-16 &      \\ 
     &           &    9 &  5.23e-08 &  7.26e-16 &      \\ 
     &           &   10 &  2.55e-08 &  1.49e-15 &      \\ 
 239 &  2.38e+03 &   10 &           &           & iters  \\ 
 \hdashline 
     &           &    1 &  7.41e-08 &  9.41e-09 &      \\ 
     &           &    2 &  8.13e-08 &  2.12e-15 &      \\ 
     &           &    3 &  7.41e-08 &  2.32e-15 &      \\ 
     &           &    4 &  8.13e-08 &  2.12e-15 &      \\ 
     &           &    5 &  7.42e-08 &  2.32e-15 &      \\ 
     &           &    6 &  3.64e-09 &  2.12e-15 &      \\ 
     &           &    7 &  3.63e-09 &  1.04e-16 &      \\ 
     &           &    8 &  3.63e-09 &  1.04e-16 &      \\ 
     &           &    9 &  3.62e-09 &  1.04e-16 &      \\ 
     &           &   10 &  3.62e-09 &  1.03e-16 &      \\ 
 240 &  2.39e+03 &   10 &           &           & iters  \\ 
 \hdashline 
     &           &    1 &  2.12e-08 &  1.21e-08 &      \\ 
     &           &    2 &  2.12e-08 &  6.05e-16 &      \\ 
     &           &    3 &  2.11e-08 &  6.04e-16 &      \\ 
     &           &    4 &  2.11e-08 &  6.03e-16 &      \\ 
     &           &    5 &  5.66e-08 &  6.03e-16 &      \\ 
     &           &    6 &  2.12e-08 &  1.62e-15 &      \\ 
     &           &    7 &  2.11e-08 &  6.04e-16 &      \\ 
     &           &    8 &  5.66e-08 &  6.03e-16 &      \\ 
     &           &    9 &  2.12e-08 &  1.61e-15 &      \\ 
     &           &   10 &  2.12e-08 &  6.05e-16 &      \\ 
 241 &  2.40e+03 &   10 &           &           & iters  \\ 
 \hdashline 
     &           &    1 &  1.91e-09 &  1.42e-08 &      \\ 
     &           &    2 &  1.91e-09 &  5.46e-17 &      \\ 
     &           &    3 &  1.91e-09 &  5.45e-17 &      \\ 
     &           &    4 &  1.90e-09 &  5.44e-17 &      \\ 
     &           &    5 &  1.90e-09 &  5.43e-17 &      \\ 
     &           &    6 &  1.90e-09 &  5.42e-17 &      \\ 
     &           &    7 &  1.90e-09 &  5.42e-17 &      \\ 
     &           &    8 &  1.89e-09 &  5.41e-17 &      \\ 
     &           &    9 &  1.89e-09 &  5.40e-17 &      \\ 
     &           &   10 &  1.89e-09 &  5.39e-17 &      \\ 
 242 &  2.41e+03 &   10 &           &           & iters  \\ 
 \hdashline 
     &           &    1 &  2.96e-08 &  2.06e-08 &      \\ 
     &           &    2 &  4.81e-08 &  8.45e-16 &      \\ 
     &           &    3 &  2.96e-08 &  1.37e-15 &      \\ 
     &           &    4 &  4.81e-08 &  8.46e-16 &      \\ 
     &           &    5 &  2.97e-08 &  1.37e-15 &      \\ 
     &           &    6 &  2.96e-08 &  8.47e-16 &      \\ 
     &           &    7 &  4.81e-08 &  8.45e-16 &      \\ 
     &           &    8 &  2.96e-08 &  1.37e-15 &      \\ 
     &           &    9 &  2.96e-08 &  8.46e-16 &      \\ 
     &           &   10 &  4.81e-08 &  8.45e-16 &      \\ 
 243 &  2.42e+03 &   10 &           &           & iters  \\ 
 \hdashline 
     &           &    1 &  2.59e-08 &  2.43e-08 &      \\ 
     &           &    2 &  2.58e-08 &  7.38e-16 &      \\ 
     &           &    3 &  5.19e-08 &  7.37e-16 &      \\ 
     &           &    4 &  2.59e-08 &  1.48e-15 &      \\ 
     &           &    5 &  2.58e-08 &  7.38e-16 &      \\ 
     &           &    6 &  5.19e-08 &  7.37e-16 &      \\ 
     &           &    7 &  2.59e-08 &  1.48e-15 &      \\ 
     &           &    8 &  2.58e-08 &  7.38e-16 &      \\ 
     &           &    9 &  5.19e-08 &  7.37e-16 &      \\ 
     &           &   10 &  2.59e-08 &  1.48e-15 &      \\ 
 244 &  2.43e+03 &   10 &           &           & iters  \\ 
 \hdashline 
     &           &    1 &  5.80e-09 &  2.19e-08 &      \\ 
     &           &    2 &  5.79e-09 &  1.66e-16 &      \\ 
     &           &    3 &  5.79e-09 &  1.65e-16 &      \\ 
     &           &    4 &  5.78e-09 &  1.65e-16 &      \\ 
     &           &    5 &  7.19e-08 &  1.65e-16 &      \\ 
     &           &    6 &  8.36e-08 &  2.05e-15 &      \\ 
     &           &    7 &  7.19e-08 &  2.38e-15 &      \\ 
     &           &    8 &  5.86e-09 &  2.05e-15 &      \\ 
     &           &    9 &  5.85e-09 &  1.67e-16 &      \\ 
     &           &   10 &  5.84e-09 &  1.67e-16 &      \\ 
 245 &  2.44e+03 &   10 &           &           & iters  \\ 
 \hdashline 
     &           &    1 &  8.44e-09 &  1.58e-08 &      \\ 
     &           &    2 &  8.43e-09 &  2.41e-16 &      \\ 
     &           &    3 &  8.41e-09 &  2.40e-16 &      \\ 
     &           &    4 &  8.40e-09 &  2.40e-16 &      \\ 
     &           &    5 &  8.39e-09 &  2.40e-16 &      \\ 
     &           &    6 &  6.93e-08 &  2.39e-16 &      \\ 
     &           &    7 &  8.48e-09 &  1.98e-15 &      \\ 
     &           &    8 &  8.47e-09 &  2.42e-16 &      \\ 
     &           &    9 &  8.45e-09 &  2.42e-16 &      \\ 
     &           &   10 &  8.44e-09 &  2.41e-16 &      \\ 
 246 &  2.45e+03 &   10 &           &           & iters  \\ 
 \hdashline 
     &           &    1 &  2.47e-10 &  6.07e-09 &      \\ 
 247 &  2.46e+03 &    1 &           &           & forces  \\ 
 \hdashline 
     &           &    1 &  4.38e-08 &  1.08e-08 &      \\ 
     &           &    2 &  3.39e-08 &  1.25e-15 &      \\ 
     &           &    3 &  3.39e-08 &  9.69e-16 &      \\ 
     &           &    4 &  4.38e-08 &  9.67e-16 &      \\ 
     &           &    5 &  3.39e-08 &  1.25e-15 &      \\ 
     &           &    6 &  4.38e-08 &  9.68e-16 &      \\ 
     &           &    7 &  3.39e-08 &  1.25e-15 &      \\ 
     &           &    8 &  4.38e-08 &  9.68e-16 &      \\ 
     &           &    9 &  3.39e-08 &  1.25e-15 &      \\ 
     &           &   10 &  3.39e-08 &  9.69e-16 &      \\ 
 248 &  2.47e+03 &   10 &           &           & iters  \\ 
 \hdashline 
     &           &    1 &  3.81e-08 &  2.04e-08 &      \\ 
     &           &    2 &  3.97e-08 &  1.09e-15 &      \\ 
     &           &    3 &  3.81e-08 &  1.13e-15 &      \\ 
     &           &    4 &  3.97e-08 &  1.09e-15 &      \\ 
     &           &    5 &  3.81e-08 &  1.13e-15 &      \\ 
     &           &    6 &  3.97e-08 &  1.09e-15 &      \\ 
     &           &    7 &  3.81e-08 &  1.13e-15 &      \\ 
     &           &    8 &  3.97e-08 &  1.09e-15 &      \\ 
     &           &    9 &  3.81e-08 &  1.13e-15 &      \\ 
     &           &   10 &  3.80e-08 &  1.09e-15 &      \\ 
 249 &  2.48e+03 &   10 &           &           & iters  \\ 
 \hdashline 
     &           &    1 &  4.30e-09 &  8.68e-09 &      \\ 
     &           &    2 &  4.29e-09 &  1.23e-16 &      \\ 
     &           &    3 &  4.29e-09 &  1.23e-16 &      \\ 
     &           &    4 &  4.28e-09 &  1.22e-16 &      \\ 
     &           &    5 &  4.28e-09 &  1.22e-16 &      \\ 
     &           &    6 &  4.27e-09 &  1.22e-16 &      \\ 
     &           &    7 &  4.26e-09 &  1.22e-16 &      \\ 
     &           &    8 &  4.26e-09 &  1.22e-16 &      \\ 
     &           &    9 &  4.25e-09 &  1.22e-16 &      \\ 
     &           &   10 &  4.25e-09 &  1.21e-16 &      \\ 
 250 &  2.49e+03 &   10 &           &           & iters  \\ 
 \hdashline 
     &           &    1 &  4.84e-08 &  8.00e-09 &      \\ 
     &           &    2 &  2.94e-08 &  1.38e-15 &      \\ 
     &           &    3 &  2.93e-08 &  8.38e-16 &      \\ 
     &           &    4 &  4.84e-08 &  8.37e-16 &      \\ 
     &           &    5 &  2.94e-08 &  1.38e-15 &      \\ 
     &           &    6 &  2.93e-08 &  8.38e-16 &      \\ 
     &           &    7 &  4.84e-08 &  8.37e-16 &      \\ 
     &           &    8 &  2.93e-08 &  1.38e-15 &      \\ 
     &           &    9 &  4.84e-08 &  8.37e-16 &      \\ 
     &           &   10 &  2.94e-08 &  1.38e-15 &      \\ 
 251 &  2.50e+03 &   10 &           &           & iters  \\ 
 \hdashline 
     &           &    1 &  4.32e-08 &  1.24e-08 &      \\ 
     &           &    2 &  3.45e-08 &  1.23e-15 &      \\ 
     &           &    3 &  3.45e-08 &  9.86e-16 &      \\ 
     &           &    4 &  4.33e-08 &  9.84e-16 &      \\ 
     &           &    5 &  3.45e-08 &  1.23e-15 &      \\ 
     &           &    6 &  4.32e-08 &  9.84e-16 &      \\ 
     &           &    7 &  3.45e-08 &  1.23e-15 &      \\ 
     &           &    8 &  4.32e-08 &  9.85e-16 &      \\ 
     &           &    9 &  3.45e-08 &  1.23e-15 &      \\ 
     &           &   10 &  4.32e-08 &  9.85e-16 &      \\ 
 252 &  2.51e+03 &   10 &           &           & iters  \\ 
 \hdashline 
     &           &    1 &  2.30e-08 &  9.42e-09 &      \\ 
     &           &    2 &  2.30e-08 &  6.57e-16 &      \\ 
     &           &    3 &  1.59e-08 &  6.56e-16 &      \\ 
     &           &    4 &  1.59e-08 &  4.53e-16 &      \\ 
     &           &    5 &  2.30e-08 &  4.53e-16 &      \\ 
     &           &    6 &  1.59e-08 &  6.57e-16 &      \\ 
     &           &    7 &  2.30e-08 &  4.53e-16 &      \\ 
     &           &    8 &  1.59e-08 &  6.56e-16 &      \\ 
     &           &    9 &  1.59e-08 &  4.53e-16 &      \\ 
     &           &   10 &  2.30e-08 &  4.53e-16 &      \\ 
 253 &  2.52e+03 &   10 &           &           & iters  \\ 
 \hdashline 
     &           &    1 &  1.55e-08 &  1.02e-08 &      \\ 
     &           &    2 &  2.34e-08 &  4.42e-16 &      \\ 
     &           &    3 &  1.55e-08 &  6.67e-16 &      \\ 
     &           &    4 &  2.34e-08 &  4.43e-16 &      \\ 
     &           &    5 &  1.55e-08 &  6.67e-16 &      \\ 
     &           &    6 &  1.55e-08 &  4.43e-16 &      \\ 
     &           &    7 &  2.34e-08 &  4.42e-16 &      \\ 
     &           &    8 &  1.55e-08 &  6.67e-16 &      \\ 
     &           &    9 &  2.34e-08 &  4.43e-16 &      \\ 
     &           &   10 &  1.55e-08 &  6.67e-16 &      \\ 
 254 &  2.53e+03 &   10 &           &           & iters  \\ 
 \hdashline 
     &           &    1 &  4.03e-08 &  7.56e-09 &      \\ 
     &           &    2 &  3.75e-08 &  1.15e-15 &      \\ 
     &           &    3 &  4.03e-08 &  1.07e-15 &      \\ 
     &           &    4 &  3.75e-08 &  1.15e-15 &      \\ 
     &           &    5 &  4.03e-08 &  1.07e-15 &      \\ 
     &           &    6 &  3.75e-08 &  1.15e-15 &      \\ 
     &           &    7 &  4.03e-08 &  1.07e-15 &      \\ 
     &           &    8 &  3.75e-08 &  1.15e-15 &      \\ 
     &           &    9 &  4.02e-08 &  1.07e-15 &      \\ 
     &           &   10 &  3.75e-08 &  1.15e-15 &      \\ 
 255 &  2.54e+03 &   10 &           &           & iters  \\ 
 \hdashline 
     &           &    1 &  3.16e-08 &  1.02e-09 &      \\ 
     &           &    2 &  4.61e-08 &  9.02e-16 &      \\ 
     &           &    3 &  3.16e-08 &  1.32e-15 &      \\ 
     &           &    4 &  4.61e-08 &  9.03e-16 &      \\ 
     &           &    5 &  3.16e-08 &  1.32e-15 &      \\ 
     &           &    6 &  3.16e-08 &  9.03e-16 &      \\ 
     &           &    7 &  4.61e-08 &  9.02e-16 &      \\ 
     &           &    8 &  3.16e-08 &  1.32e-15 &      \\ 
     &           &    9 &  4.61e-08 &  9.02e-16 &      \\ 
     &           &   10 &  3.16e-08 &  1.32e-15 &      \\ 
 256 &  2.55e+03 &   10 &           &           & iters  \\ 
 \hdashline 
     &           &    1 &  3.86e-08 &  3.78e-09 &      \\ 
     &           &    2 &  3.92e-08 &  1.10e-15 &      \\ 
     &           &    3 &  3.86e-08 &  1.12e-15 &      \\ 
     &           &    4 &  3.92e-08 &  1.10e-15 &      \\ 
     &           &    5 &  3.86e-08 &  1.12e-15 &      \\ 
     &           &    6 &  3.92e-08 &  1.10e-15 &      \\ 
     &           &    7 &  3.86e-08 &  1.12e-15 &      \\ 
     &           &    8 &  3.92e-08 &  1.10e-15 &      \\ 
     &           &    9 &  3.86e-08 &  1.12e-15 &      \\ 
     &           &   10 &  3.92e-08 &  1.10e-15 &      \\ 
 257 &  2.56e+03 &   10 &           &           & iters  \\ 
 \hdashline 
     &           &    1 &  6.86e-10 &  7.94e-09 &      \\ 
     &           &    2 &  6.85e-10 &  1.96e-17 &      \\ 
     &           &    3 &  6.84e-10 &  1.95e-17 &      \\ 
     &           &    4 &  6.83e-10 &  1.95e-17 &      \\ 
     &           &    5 &  6.82e-10 &  1.95e-17 &      \\ 
     &           &    6 &  6.81e-10 &  1.95e-17 &      \\ 
     &           &    7 &  6.80e-10 &  1.94e-17 &      \\ 
     &           &    8 &  6.79e-10 &  1.94e-17 &      \\ 
     &           &    9 &  6.78e-10 &  1.94e-17 &      \\ 
     &           &   10 &  6.77e-10 &  1.93e-17 &      \\ 
 258 &  2.57e+03 &   10 &           &           & iters  \\ 
 \hdashline 
     &           &    1 &  1.09e-08 &  5.30e-09 &      \\ 
     &           &    2 &  1.09e-08 &  3.12e-16 &      \\ 
     &           &    3 &  6.68e-08 &  3.11e-16 &      \\ 
     &           &    4 &  1.10e-08 &  1.91e-15 &      \\ 
     &           &    5 &  1.10e-08 &  3.14e-16 &      \\ 
     &           &    6 &  1.10e-08 &  3.13e-16 &      \\ 
     &           &    7 &  1.09e-08 &  3.13e-16 &      \\ 
     &           &    8 &  1.09e-08 &  3.12e-16 &      \\ 
     &           &    9 &  1.09e-08 &  3.12e-16 &      \\ 
     &           &   10 &  6.68e-08 &  3.11e-16 &      \\ 
 259 &  2.58e+03 &   10 &           &           & iters  \\ 
 \hdashline 
     &           &    1 &  2.10e-08 &  3.80e-09 &      \\ 
     &           &    2 &  2.10e-08 &  6.00e-16 &      \\ 
     &           &    3 &  1.79e-08 &  5.99e-16 &      \\ 
     &           &    4 &  2.10e-08 &  5.10e-16 &      \\ 
     &           &    5 &  1.79e-08 &  5.99e-16 &      \\ 
     &           &    6 &  1.79e-08 &  5.10e-16 &      \\ 
     &           &    7 &  2.10e-08 &  5.10e-16 &      \\ 
     &           &    8 &  1.79e-08 &  6.00e-16 &      \\ 
     &           &    9 &  2.10e-08 &  5.10e-16 &      \\ 
     &           &   10 &  1.79e-08 &  6.00e-16 &      \\ 
 260 &  2.59e+03 &   10 &           &           & iters  \\ 
 \hdashline 
     &           &    1 &  8.29e-09 &  8.66e-09 &      \\ 
     &           &    2 &  8.28e-09 &  2.37e-16 &      \\ 
     &           &    3 &  8.27e-09 &  2.36e-16 &      \\ 
     &           &    4 &  8.26e-09 &  2.36e-16 &      \\ 
     &           &    5 &  8.24e-09 &  2.36e-16 &      \\ 
     &           &    6 &  8.23e-09 &  2.35e-16 &      \\ 
     &           &    7 &  8.22e-09 &  2.35e-16 &      \\ 
     &           &    8 &  6.95e-08 &  2.35e-16 &      \\ 
     &           &    9 &  4.72e-08 &  1.98e-15 &      \\ 
     &           &   10 &  8.24e-09 &  1.35e-15 &      \\ 
 261 &  2.60e+03 &   10 &           &           & iters  \\ 
 \hdashline 
     &           &    1 &  1.37e-08 &  9.49e-09 &      \\ 
     &           &    2 &  1.37e-08 &  3.90e-16 &      \\ 
     &           &    3 &  1.36e-08 &  3.90e-16 &      \\ 
     &           &    4 &  1.36e-08 &  3.89e-16 &      \\ 
     &           &    5 &  6.41e-08 &  3.89e-16 &      \\ 
     &           &    6 &  1.37e-08 &  1.83e-15 &      \\ 
     &           &    7 &  1.37e-08 &  3.91e-16 &      \\ 
     &           &    8 &  1.36e-08 &  3.90e-16 &      \\ 
     &           &    9 &  1.36e-08 &  3.90e-16 &      \\ 
     &           &   10 &  1.36e-08 &  3.89e-16 &      \\ 
 262 &  2.61e+03 &   10 &           &           & iters  \\ 
 \hdashline 
     &           &    1 &  1.81e-08 &  4.96e-09 &      \\ 
     &           &    2 &  1.81e-08 &  5.18e-16 &      \\ 
     &           &    3 &  1.81e-08 &  5.17e-16 &      \\ 
     &           &    4 &  5.96e-08 &  5.16e-16 &      \\ 
     &           &    5 &  1.81e-08 &  1.70e-15 &      \\ 
     &           &    6 &  1.81e-08 &  5.18e-16 &      \\ 
     &           &    7 &  1.81e-08 &  5.17e-16 &      \\ 
     &           &    8 &  5.96e-08 &  5.16e-16 &      \\ 
     &           &    9 &  1.82e-08 &  1.70e-15 &      \\ 
     &           &   10 &  1.81e-08 &  5.18e-16 &      \\ 
 263 &  2.62e+03 &   10 &           &           & iters  \\ 
 \hdashline 
     &           &    1 &  1.34e-08 &  1.88e-09 &      \\ 
     &           &    2 &  2.55e-08 &  3.83e-16 &      \\ 
     &           &    3 &  1.34e-08 &  7.27e-16 &      \\ 
     &           &    4 &  1.34e-08 &  3.83e-16 &      \\ 
     &           &    5 &  2.55e-08 &  3.83e-16 &      \\ 
     &           &    6 &  1.34e-08 &  7.27e-16 &      \\ 
     &           &    7 &  1.34e-08 &  3.83e-16 &      \\ 
     &           &    8 &  2.55e-08 &  3.83e-16 &      \\ 
     &           &    9 &  1.34e-08 &  7.27e-16 &      \\ 
     &           &   10 &  1.34e-08 &  3.83e-16 &      \\ 
 264 &  2.63e+03 &   10 &           &           & iters  \\ 
 \hdashline 
     &           &    1 &  2.13e-08 &  6.26e-09 &      \\ 
     &           &    2 &  2.13e-08 &  6.07e-16 &      \\ 
     &           &    3 &  5.65e-08 &  6.07e-16 &      \\ 
     &           &    4 &  2.13e-08 &  1.61e-15 &      \\ 
     &           &    5 &  2.13e-08 &  6.08e-16 &      \\ 
     &           &    6 &  5.64e-08 &  6.07e-16 &      \\ 
     &           &    7 &  2.13e-08 &  1.61e-15 &      \\ 
     &           &    8 &  2.13e-08 &  6.09e-16 &      \\ 
     &           &    9 &  2.13e-08 &  6.08e-16 &      \\ 
     &           &   10 &  5.65e-08 &  6.07e-16 &      \\ 
 265 &  2.64e+03 &   10 &           &           & iters  \\ 
 \hdashline 
     &           &    1 &  6.38e-08 &  3.80e-09 &      \\ 
     &           &    2 &  1.39e-08 &  1.82e-15 &      \\ 
     &           &    3 &  1.39e-08 &  3.98e-16 &      \\ 
     &           &    4 &  1.39e-08 &  3.97e-16 &      \\ 
     &           &    5 &  1.39e-08 &  3.97e-16 &      \\ 
     &           &    6 &  1.39e-08 &  3.96e-16 &      \\ 
     &           &    7 &  6.39e-08 &  3.95e-16 &      \\ 
     &           &    8 &  1.39e-08 &  1.82e-15 &      \\ 
     &           &    9 &  1.39e-08 &  3.98e-16 &      \\ 
     &           &   10 &  1.39e-08 &  3.97e-16 &      \\ 
 266 &  2.65e+03 &   10 &           &           & iters  \\ 
 \hdashline 
     &           &    1 &  2.85e-08 &  6.62e-09 &      \\ 
     &           &    2 &  1.04e-08 &  8.14e-16 &      \\ 
     &           &    3 &  1.04e-08 &  2.96e-16 &      \\ 
     &           &    4 &  2.85e-08 &  2.96e-16 &      \\ 
     &           &    5 &  1.04e-08 &  8.13e-16 &      \\ 
     &           &    6 &  1.04e-08 &  2.97e-16 &      \\ 
     &           &    7 &  1.04e-08 &  2.96e-16 &      \\ 
     &           &    8 &  2.85e-08 &  2.96e-16 &      \\ 
     &           &    9 &  1.04e-08 &  8.13e-16 &      \\ 
     &           &   10 &  1.04e-08 &  2.96e-16 &      \\ 
 267 &  2.66e+03 &   10 &           &           & iters  \\ 
 \hdashline 
     &           &    1 &  2.75e-08 &  2.06e-09 &      \\ 
     &           &    2 &  5.02e-08 &  7.85e-16 &      \\ 
     &           &    3 &  2.75e-08 &  1.43e-15 &      \\ 
     &           &    4 &  2.75e-08 &  7.86e-16 &      \\ 
     &           &    5 &  5.02e-08 &  7.85e-16 &      \\ 
     &           &    6 &  2.75e-08 &  1.43e-15 &      \\ 
     &           &    7 &  2.75e-08 &  7.86e-16 &      \\ 
     &           &    8 &  5.02e-08 &  7.85e-16 &      \\ 
     &           &    9 &  2.75e-08 &  1.43e-15 &      \\ 
     &           &   10 &  5.02e-08 &  7.86e-16 &      \\ 
 268 &  2.67e+03 &   10 &           &           & iters  \\ 
 \hdashline 
     &           &    1 &  3.15e-08 &  7.77e-09 &      \\ 
     &           &    2 &  7.42e-09 &  8.98e-16 &      \\ 
     &           &    3 &  3.14e-08 &  2.12e-16 &      \\ 
     &           &    4 &  7.45e-09 &  8.97e-16 &      \\ 
     &           &    5 &  7.44e-09 &  2.13e-16 &      \\ 
     &           &    6 &  7.43e-09 &  2.12e-16 &      \\ 
     &           &    7 &  7.42e-09 &  2.12e-16 &      \\ 
     &           &    8 &  3.14e-08 &  2.12e-16 &      \\ 
     &           &    9 &  7.45e-09 &  8.97e-16 &      \\ 
     &           &   10 &  7.44e-09 &  2.13e-16 &      \\ 
 269 &  2.68e+03 &   10 &           &           & iters  \\ 
 \hdashline 
     &           &    1 &  3.75e-08 &  8.84e-09 &      \\ 
     &           &    2 &  4.02e-08 &  1.07e-15 &      \\ 
     &           &    3 &  3.75e-08 &  1.15e-15 &      \\ 
     &           &    4 &  4.02e-08 &  1.07e-15 &      \\ 
     &           &    5 &  3.75e-08 &  1.15e-15 &      \\ 
     &           &    6 &  4.02e-08 &  1.07e-15 &      \\ 
     &           &    7 &  3.75e-08 &  1.15e-15 &      \\ 
     &           &    8 &  4.02e-08 &  1.07e-15 &      \\ 
     &           &    9 &  3.75e-08 &  1.15e-15 &      \\ 
     &           &   10 &  4.02e-08 &  1.07e-15 &      \\ 
 270 &  2.69e+03 &   10 &           &           & iters  \\ 
 \hdashline 
     &           &    1 &  3.73e-09 &  7.91e-09 &      \\ 
     &           &    2 &  3.72e-09 &  1.06e-16 &      \\ 
     &           &    3 &  3.72e-09 &  1.06e-16 &      \\ 
     &           &    4 &  3.71e-09 &  1.06e-16 &      \\ 
     &           &    5 &  3.71e-09 &  1.06e-16 &      \\ 
     &           &    6 &  3.70e-09 &  1.06e-16 &      \\ 
     &           &    7 &  3.70e-09 &  1.06e-16 &      \\ 
     &           &    8 &  3.69e-09 &  1.06e-16 &      \\ 
     &           &    9 &  3.69e-09 &  1.05e-16 &      \\ 
     &           &   10 &  7.40e-08 &  1.05e-16 &      \\ 
 271 &  2.70e+03 &   10 &           &           & iters  \\ 
 \hdashline 
     &           &    1 &  8.09e-09 &  3.34e-08 &      \\ 
     &           &    2 &  8.08e-09 &  2.31e-16 &      \\ 
     &           &    3 &  6.96e-08 &  2.31e-16 &      \\ 
     &           &    4 &  4.70e-08 &  1.99e-15 &      \\ 
     &           &    5 &  8.10e-09 &  1.34e-15 &      \\ 
     &           &    6 &  8.09e-09 &  2.31e-16 &      \\ 
     &           &    7 &  8.08e-09 &  2.31e-16 &      \\ 
     &           &    8 &  6.96e-08 &  2.31e-16 &      \\ 
     &           &    9 &  4.70e-08 &  1.99e-15 &      \\ 
     &           &   10 &  8.10e-09 &  1.34e-15 &      \\ 
 272 &  2.71e+03 &   10 &           &           & iters  \\ 
 \hdashline 
     &           &    1 &  2.17e-08 &  3.56e-08 &      \\ 
     &           &    2 &  2.17e-08 &  6.19e-16 &      \\ 
     &           &    3 &  5.61e-08 &  6.18e-16 &      \\ 
     &           &    4 &  2.17e-08 &  1.60e-15 &      \\ 
     &           &    5 &  2.17e-08 &  6.20e-16 &      \\ 
     &           &    6 &  2.17e-08 &  6.19e-16 &      \\ 
     &           &    7 &  5.61e-08 &  6.18e-16 &      \\ 
     &           &    8 &  2.17e-08 &  1.60e-15 &      \\ 
     &           &    9 &  2.17e-08 &  6.19e-16 &      \\ 
     &           &   10 &  2.16e-08 &  6.19e-16 &      \\ 
 273 &  2.72e+03 &   10 &           &           & iters  \\ 
 \hdashline 
     &           &    1 &  4.42e-08 &  1.66e-08 &      \\ 
     &           &    2 &  5.26e-09 &  1.26e-15 &      \\ 
     &           &    3 &  5.26e-09 &  1.50e-16 &      \\ 
     &           &    4 &  5.25e-09 &  1.50e-16 &      \\ 
     &           &    5 &  5.24e-09 &  1.50e-16 &      \\ 
     &           &    6 &  5.23e-09 &  1.50e-16 &      \\ 
     &           &    7 &  5.23e-09 &  1.49e-16 &      \\ 
     &           &    8 &  5.22e-09 &  1.49e-16 &      \\ 
     &           &    9 &  5.21e-09 &  1.49e-16 &      \\ 
     &           &   10 &  5.20e-09 &  1.49e-16 &      \\ 
 274 &  2.73e+03 &   10 &           &           & iters  \\ 
 \hdashline 
     &           &    1 &  6.90e-09 &  6.61e-10 &      \\ 
     &           &    2 &  7.08e-08 &  1.97e-16 &      \\ 
     &           &    3 &  4.58e-08 &  2.02e-15 &      \\ 
     &           &    4 &  6.92e-09 &  1.31e-15 &      \\ 
     &           &    5 &  6.91e-09 &  1.98e-16 &      \\ 
     &           &    6 &  6.90e-09 &  1.97e-16 &      \\ 
     &           &    7 &  7.08e-08 &  1.97e-16 &      \\ 
     &           &    8 &  4.58e-08 &  2.02e-15 &      \\ 
     &           &    9 &  6.93e-09 &  1.31e-15 &      \\ 
     &           &   10 &  6.92e-09 &  1.98e-16 &      \\ 
 275 &  2.74e+03 &   10 &           &           & iters  \\ 
 \hdashline 
     &           &    1 &  3.39e-08 &  9.41e-09 &      \\ 
     &           &    2 &  5.01e-09 &  9.67e-16 &      \\ 
     &           &    3 &  5.01e-09 &  1.43e-16 &      \\ 
     &           &    4 &  5.00e-09 &  1.43e-16 &      \\ 
     &           &    5 &  4.99e-09 &  1.43e-16 &      \\ 
     &           &    6 &  4.98e-09 &  1.42e-16 &      \\ 
     &           &    7 &  4.98e-09 &  1.42e-16 &      \\ 
     &           &    8 &  3.39e-08 &  1.42e-16 &      \\ 
     &           &    9 &  5.02e-09 &  9.67e-16 &      \\ 
     &           &   10 &  5.01e-09 &  1.43e-16 &      \\ 
 276 &  2.75e+03 &   10 &           &           & iters  \\ 
 \hdashline 
     &           &    1 &  1.38e-08 &  2.05e-08 &      \\ 
     &           &    2 &  1.38e-08 &  3.93e-16 &      \\ 
     &           &    3 &  1.37e-08 &  3.93e-16 &      \\ 
     &           &    4 &  6.40e-08 &  3.92e-16 &      \\ 
     &           &    5 &  1.38e-08 &  1.83e-15 &      \\ 
     &           &    6 &  1.38e-08 &  3.94e-16 &      \\ 
     &           &    7 &  1.38e-08 &  3.94e-16 &      \\ 
     &           &    8 &  1.38e-08 &  3.93e-16 &      \\ 
     &           &    9 &  6.39e-08 &  3.93e-16 &      \\ 
     &           &   10 &  1.38e-08 &  1.83e-15 &      \\ 
 277 &  2.76e+03 &   10 &           &           & iters  \\ 
 \hdashline 
     &           &    1 &  5.42e-08 &  8.86e-09 &      \\ 
     &           &    2 &  2.35e-08 &  1.55e-15 &      \\ 
     &           &    3 &  2.35e-08 &  6.72e-16 &      \\ 
     &           &    4 &  5.42e-08 &  6.71e-16 &      \\ 
     &           &    5 &  2.35e-08 &  1.55e-15 &      \\ 
     &           &    6 &  2.35e-08 &  6.72e-16 &      \\ 
     &           &    7 &  5.42e-08 &  6.71e-16 &      \\ 
     &           &    8 &  2.36e-08 &  1.55e-15 &      \\ 
     &           &    9 &  2.35e-08 &  6.72e-16 &      \\ 
     &           &   10 &  5.42e-08 &  6.71e-16 &      \\ 
 278 &  2.77e+03 &   10 &           &           & iters  \\ 
 \hdashline 
     &           &    1 &  2.01e-08 &  9.90e-09 &      \\ 
     &           &    2 &  2.01e-08 &  5.73e-16 &      \\ 
     &           &    3 &  2.00e-08 &  5.73e-16 &      \\ 
     &           &    4 &  5.77e-08 &  5.72e-16 &      \\ 
     &           &    5 &  2.01e-08 &  1.65e-15 &      \\ 
     &           &    6 &  2.01e-08 &  5.73e-16 &      \\ 
     &           &    7 &  2.00e-08 &  5.72e-16 &      \\ 
     &           &    8 &  5.77e-08 &  5.72e-16 &      \\ 
     &           &    9 &  2.01e-08 &  1.65e-15 &      \\ 
     &           &   10 &  2.01e-08 &  5.73e-16 &      \\ 
 279 &  2.78e+03 &   10 &           &           & iters  \\ 
 \hdashline 
     &           &    1 &  1.07e-09 &  2.00e-08 &      \\ 
     &           &    2 &  1.07e-09 &  3.06e-17 &      \\ 
     &           &    3 &  1.07e-09 &  3.06e-17 &      \\ 
     &           &    4 &  3.78e-08 &  3.05e-17 &      \\ 
     &           &    5 &  1.12e-09 &  1.08e-15 &      \\ 
     &           &    6 &  1.12e-09 &  3.20e-17 &      \\ 
     &           &    7 &  1.12e-09 &  3.20e-17 &      \\ 
     &           &    8 &  1.12e-09 &  3.19e-17 &      \\ 
     &           &    9 &  1.12e-09 &  3.19e-17 &      \\ 
     &           &   10 &  1.11e-09 &  3.18e-17 &      \\ 
 280 &  2.79e+03 &   10 &           &           & iters  \\ 
 \hdashline 
     &           &    1 &  1.82e-08 &  5.33e-09 &      \\ 
     &           &    2 &  5.96e-08 &  5.18e-16 &      \\ 
     &           &    3 &  1.82e-08 &  1.70e-15 &      \\ 
     &           &    4 &  1.82e-08 &  5.20e-16 &      \\ 
     &           &    5 &  1.82e-08 &  5.19e-16 &      \\ 
     &           &    6 &  1.81e-08 &  5.19e-16 &      \\ 
     &           &    7 &  5.96e-08 &  5.18e-16 &      \\ 
     &           &    8 &  1.82e-08 &  1.70e-15 &      \\ 
     &           &    9 &  1.82e-08 &  5.20e-16 &      \\ 
     &           &   10 &  1.82e-08 &  5.19e-16 &      \\ 
 281 &  2.80e+03 &   10 &           &           & iters  \\ 
 \hdashline 
     &           &    1 &  7.79e-08 &  9.10e-10 &      \\ 
     &           &    2 &  3.88e-08 &  2.22e-15 &      \\ 
     &           &    3 &  1.27e-10 &  1.11e-15 &      \\ 
 282 &  2.81e+03 &    3 &           &           & forces  \\ 
 \hdashline 
     &           &    1 &  6.67e-09 &  3.44e-08 &      \\ 
     &           &    2 &  6.66e-09 &  1.90e-16 &      \\ 
     &           &    3 &  6.65e-09 &  1.90e-16 &      \\ 
     &           &    4 &  6.64e-09 &  1.90e-16 &      \\ 
     &           &    5 &  3.22e-08 &  1.90e-16 &      \\ 
     &           &    6 &  6.68e-09 &  9.19e-16 &      \\ 
     &           &    7 &  6.67e-09 &  1.91e-16 &      \\ 
     &           &    8 &  6.66e-09 &  1.90e-16 &      \\ 
     &           &    9 &  6.65e-09 &  1.90e-16 &      \\ 
     &           &   10 &  6.64e-09 &  1.90e-16 &      \\ 
 283 &  2.82e+03 &   10 &           &           & iters  \\ 
 \hdashline 
     &           &    1 &  8.08e-09 &  4.32e-08 &      \\ 
     &           &    2 &  8.07e-09 &  2.31e-16 &      \\ 
     &           &    3 &  3.08e-08 &  2.30e-16 &      \\ 
     &           &    4 &  8.10e-09 &  8.79e-16 &      \\ 
     &           &    5 &  8.09e-09 &  2.31e-16 &      \\ 
     &           &    6 &  8.08e-09 &  2.31e-16 &      \\ 
     &           &    7 &  8.07e-09 &  2.31e-16 &      \\ 
     &           &    8 &  3.08e-08 &  2.30e-16 &      \\ 
     &           &    9 &  8.10e-09 &  8.79e-16 &      \\ 
     &           &   10 &  8.09e-09 &  2.31e-16 &      \\ 
 284 &  2.83e+03 &   10 &           &           & iters  \\ 
 \hdashline 
     &           &    1 &  1.92e-09 &  2.54e-08 &      \\ 
     &           &    2 &  1.92e-09 &  5.48e-17 &      \\ 
     &           &    3 &  1.91e-09 &  5.47e-17 &      \\ 
     &           &    4 &  1.91e-09 &  5.46e-17 &      \\ 
     &           &    5 &  1.91e-09 &  5.45e-17 &      \\ 
     &           &    6 &  1.91e-09 &  5.45e-17 &      \\ 
     &           &    7 &  1.90e-09 &  5.44e-17 &      \\ 
     &           &    8 &  1.90e-09 &  5.43e-17 &      \\ 
     &           &    9 &  1.90e-09 &  5.42e-17 &      \\ 
     &           &   10 &  1.89e-09 &  5.42e-17 &      \\ 
 285 &  2.84e+03 &   10 &           &           & iters  \\ 
 \hdashline 
     &           &    1 &  5.03e-08 &  2.55e-08 &      \\ 
     &           &    2 &  2.75e-08 &  1.44e-15 &      \\ 
     &           &    3 &  5.03e-08 &  7.84e-16 &      \\ 
     &           &    4 &  2.75e-08 &  1.43e-15 &      \\ 
     &           &    5 &  2.75e-08 &  7.85e-16 &      \\ 
     &           &    6 &  5.03e-08 &  7.84e-16 &      \\ 
     &           &    7 &  2.75e-08 &  1.43e-15 &      \\ 
     &           &    8 &  2.75e-08 &  7.85e-16 &      \\ 
     &           &    9 &  5.03e-08 &  7.84e-16 &      \\ 
     &           &   10 &  2.75e-08 &  1.43e-15 &      \\ 
 286 &  2.85e+03 &   10 &           &           & iters  \\ 
 \hdashline 
     &           &    1 &  3.59e-08 &  2.88e-08 &      \\ 
     &           &    2 &  4.18e-08 &  1.03e-15 &      \\ 
     &           &    3 &  3.59e-08 &  1.19e-15 &      \\ 
     &           &    4 &  4.18e-08 &  1.03e-15 &      \\ 
     &           &    5 &  3.59e-08 &  1.19e-15 &      \\ 
     &           &    6 &  4.18e-08 &  1.03e-15 &      \\ 
     &           &    7 &  3.59e-08 &  1.19e-15 &      \\ 
     &           &    8 &  4.18e-08 &  1.03e-15 &      \\ 
     &           &    9 &  3.60e-08 &  1.19e-15 &      \\ 
     &           &   10 &  3.59e-08 &  1.03e-15 &      \\ 
 287 &  2.86e+03 &   10 &           &           & iters  \\ 
 \hdashline 
     &           &    1 &  1.92e-08 &  2.98e-08 &      \\ 
     &           &    2 &  5.85e-08 &  5.47e-16 &      \\ 
     &           &    3 &  1.92e-08 &  1.67e-15 &      \\ 
     &           &    4 &  1.92e-08 &  5.49e-16 &      \\ 
     &           &    5 &  1.92e-08 &  5.48e-16 &      \\ 
     &           &    6 &  5.85e-08 &  5.47e-16 &      \\ 
     &           &    7 &  1.92e-08 &  1.67e-15 &      \\ 
     &           &    8 &  1.92e-08 &  5.49e-16 &      \\ 
     &           &    9 &  1.92e-08 &  5.48e-16 &      \\ 
     &           &   10 &  1.92e-08 &  5.47e-16 &      \\ 
 288 &  2.87e+03 &   10 &           &           & iters  \\ 
 \hdashline 
     &           &    1 &  5.74e-09 &  1.49e-08 &      \\ 
     &           &    2 &  5.73e-09 &  1.64e-16 &      \\ 
     &           &    3 &  5.73e-09 &  1.64e-16 &      \\ 
     &           &    4 &  5.72e-09 &  1.63e-16 &      \\ 
     &           &    5 &  7.20e-08 &  1.63e-16 &      \\ 
     &           &    6 &  8.35e-08 &  2.05e-15 &      \\ 
     &           &    7 &  7.20e-08 &  2.38e-15 &      \\ 
     &           &    8 &  5.80e-09 &  2.05e-15 &      \\ 
     &           &    9 &  5.79e-09 &  1.65e-16 &      \\ 
     &           &   10 &  5.78e-09 &  1.65e-16 &      \\ 
 289 &  2.88e+03 &   10 &           &           & iters  \\ 
 \hdashline 
     &           &    1 &  9.09e-09 &  8.25e-08 &      \\ 
     &           &    2 &  9.08e-09 &  2.60e-16 &      \\ 
     &           &    3 &  9.07e-09 &  2.59e-16 &      \\ 
     &           &    4 &  9.05e-09 &  2.59e-16 &      \\ 
     &           &    5 &  9.04e-09 &  2.58e-16 &      \\ 
     &           &    6 &  9.03e-09 &  2.58e-16 &      \\ 
     &           &    7 &  6.87e-08 &  2.58e-16 &      \\ 
     &           &    8 &  8.68e-08 &  1.96e-15 &      \\ 
     &           &    9 &  6.87e-08 &  2.48e-15 &      \\ 
     &           &   10 &  9.09e-09 &  1.96e-15 &      \\ 
 290 &  2.89e+03 &   10 &           &           & iters  \\ 
 \hdashline 
     &           &    1 &  3.49e-08 &  6.99e-08 &      \\ 
     &           &    2 &  3.48e-08 &  9.95e-16 &      \\ 
     &           &    3 &  4.29e-08 &  9.93e-16 &      \\ 
     &           &    4 &  3.48e-08 &  1.23e-15 &      \\ 
     &           &    5 &  4.29e-08 &  9.94e-16 &      \\ 
     &           &    6 &  3.48e-08 &  1.23e-15 &      \\ 
     &           &    7 &  4.29e-08 &  9.94e-16 &      \\ 
     &           &    8 &  3.48e-08 &  1.22e-15 &      \\ 
     &           &    9 &  3.48e-08 &  9.94e-16 &      \\ 
     &           &   10 &  4.30e-08 &  9.93e-16 &      \\ 
 291 &  2.90e+03 &   10 &           &           & iters  \\ 
 \hdashline 
     &           &    1 &  3.27e-08 &  9.62e-09 &      \\ 
     &           &    2 &  4.51e-08 &  9.33e-16 &      \\ 
     &           &    3 &  3.27e-08 &  1.29e-15 &      \\ 
     &           &    4 &  4.50e-08 &  9.33e-16 &      \\ 
     &           &    5 &  3.27e-08 &  1.29e-15 &      \\ 
     &           &    6 &  3.27e-08 &  9.34e-16 &      \\ 
     &           &    7 &  4.51e-08 &  9.32e-16 &      \\ 
     &           &    8 &  3.27e-08 &  1.29e-15 &      \\ 
     &           &    9 &  4.50e-08 &  9.33e-16 &      \\ 
     &           &   10 &  3.27e-08 &  1.29e-15 &      \\ 
 292 &  2.91e+03 &   10 &           &           & iters  \\ 
 \hdashline 
     &           &    1 &  2.98e-08 &  2.59e-08 &      \\ 
     &           &    2 &  9.08e-09 &  8.51e-16 &      \\ 
     &           &    3 &  9.07e-09 &  2.59e-16 &      \\ 
     &           &    4 &  2.98e-08 &  2.59e-16 &      \\ 
     &           &    5 &  9.10e-09 &  8.50e-16 &      \\ 
     &           &    6 &  9.09e-09 &  2.60e-16 &      \\ 
     &           &    7 &  9.07e-09 &  2.59e-16 &      \\ 
     &           &    8 &  2.98e-08 &  2.59e-16 &      \\ 
     &           &    9 &  9.10e-09 &  8.50e-16 &      \\ 
     &           &   10 &  9.09e-09 &  2.60e-16 &      \\ 
 293 &  2.92e+03 &   10 &           &           & iters  \\ 
 \hdashline 
     &           &    1 &  7.02e-09 &  1.31e-08 &      \\ 
     &           &    2 &  7.01e-09 &  2.00e-16 &      \\ 
     &           &    3 &  7.00e-09 &  2.00e-16 &      \\ 
     &           &    4 &  3.18e-08 &  2.00e-16 &      \\ 
     &           &    5 &  7.04e-09 &  9.09e-16 &      \\ 
     &           &    6 &  7.03e-09 &  2.01e-16 &      \\ 
     &           &    7 &  7.02e-09 &  2.01e-16 &      \\ 
     &           &    8 &  7.01e-09 &  2.00e-16 &      \\ 
     &           &    9 &  3.18e-08 &  2.00e-16 &      \\ 
     &           &   10 &  7.05e-09 &  9.09e-16 &      \\ 
 294 &  2.93e+03 &   10 &           &           & iters  \\ 
 \hdashline 
     &           &    1 &  4.20e-08 &  3.13e-08 &      \\ 
     &           &    2 &  3.58e-08 &  1.20e-15 &      \\ 
     &           &    3 &  4.20e-08 &  1.02e-15 &      \\ 
     &           &    4 &  3.58e-08 &  1.20e-15 &      \\ 
     &           &    5 &  4.20e-08 &  1.02e-15 &      \\ 
     &           &    6 &  3.58e-08 &  1.20e-15 &      \\ 
     &           &    7 &  4.20e-08 &  1.02e-15 &      \\ 
     &           &    8 &  3.58e-08 &  1.20e-15 &      \\ 
     &           &    9 &  3.57e-08 &  1.02e-15 &      \\ 
     &           &   10 &  4.20e-08 &  1.02e-15 &      \\ 
 295 &  2.94e+03 &   10 &           &           & iters  \\ 
 \hdashline 
     &           &    1 &  2.87e-09 &  3.50e-08 &      \\ 
     &           &    2 &  2.86e-09 &  8.19e-17 &      \\ 
     &           &    3 &  2.86e-09 &  8.17e-17 &      \\ 
     &           &    4 &  2.86e-09 &  8.16e-17 &      \\ 
     &           &    5 &  2.85e-09 &  8.15e-17 &      \\ 
     &           &    6 &  2.85e-09 &  8.14e-17 &      \\ 
     &           &    7 &  2.84e-09 &  8.13e-17 &      \\ 
     &           &    8 &  2.84e-09 &  8.12e-17 &      \\ 
     &           &    9 &  2.84e-09 &  8.10e-17 &      \\ 
     &           &   10 &  2.83e-09 &  8.09e-17 &      \\ 
 296 &  2.95e+03 &   10 &           &           & iters  \\ 
 \hdashline 
     &           &    1 &  2.39e-08 &  6.97e-09 &      \\ 
     &           &    2 &  2.39e-08 &  6.82e-16 &      \\ 
     &           &    3 &  2.38e-08 &  6.81e-16 &      \\ 
     &           &    4 &  5.39e-08 &  6.80e-16 &      \\ 
     &           &    5 &  2.39e-08 &  1.54e-15 &      \\ 
     &           &    6 &  2.39e-08 &  6.82e-16 &      \\ 
     &           &    7 &  5.39e-08 &  6.81e-16 &      \\ 
     &           &    8 &  2.39e-08 &  1.54e-15 &      \\ 
     &           &    9 &  2.39e-08 &  6.82e-16 &      \\ 
     &           &   10 &  5.39e-08 &  6.81e-16 &      \\ 
 297 &  2.96e+03 &   10 &           &           & iters  \\ 
 \hdashline 
     &           &    1 &  3.49e-09 &  2.03e-08 &      \\ 
     &           &    2 &  3.49e-09 &  9.97e-17 &      \\ 
     &           &    3 &  3.48e-09 &  9.96e-17 &      \\ 
     &           &    4 &  3.48e-09 &  9.94e-17 &      \\ 
     &           &    5 &  3.47e-09 &  9.93e-17 &      \\ 
     &           &    6 &  3.47e-09 &  9.91e-17 &      \\ 
     &           &    7 &  3.46e-09 &  9.90e-17 &      \\ 
     &           &    8 &  3.46e-09 &  9.88e-17 &      \\ 
     &           &    9 &  3.45e-09 &  9.87e-17 &      \\ 
     &           &   10 &  3.45e-09 &  9.86e-17 &      \\ 
 298 &  2.97e+03 &   10 &           &           & iters  \\ 
 \hdashline 
     &           &    1 &  6.08e-08 &  8.99e-09 &      \\ 
     &           &    2 &  1.70e-08 &  1.74e-15 &      \\ 
     &           &    3 &  1.70e-08 &  4.84e-16 &      \\ 
     &           &    4 &  1.69e-08 &  4.84e-16 &      \\ 
     &           &    5 &  6.08e-08 &  4.83e-16 &      \\ 
     &           &    6 &  1.70e-08 &  1.73e-15 &      \\ 
     &           &    7 &  1.70e-08 &  4.85e-16 &      \\ 
     &           &    8 &  1.69e-08 &  4.84e-16 &      \\ 
     &           &    9 &  6.08e-08 &  4.83e-16 &      \\ 
     &           &   10 &  1.70e-08 &  1.73e-15 &      \\ 
 299 &  2.98e+03 &   10 &           &           & iters  \\ 
 \hdashline 
     &           &    1 &  5.03e-08 &  1.53e-08 &      \\ 
     &           &    2 &  2.74e-08 &  1.44e-15 &      \\ 
     &           &    3 &  2.74e-08 &  7.83e-16 &      \\ 
     &           &    4 &  5.03e-08 &  7.82e-16 &      \\ 
     &           &    5 &  2.74e-08 &  1.44e-15 &      \\ 
     &           &    6 &  2.74e-08 &  7.83e-16 &      \\ 
     &           &    7 &  5.03e-08 &  7.82e-16 &      \\ 
     &           &    8 &  2.74e-08 &  1.44e-15 &      \\ 
     &           &    9 &  2.74e-08 &  7.82e-16 &      \\ 
     &           &   10 &  5.04e-08 &  7.81e-16 &      \\ 
 300 &  2.99e+03 &   10 &           &           & iters  \\ 
 \hdashline 
     &           &    1 &  3.12e-08 &  2.48e-08 &      \\ 
     &           &    2 &  4.65e-08 &  8.92e-16 &      \\ 
     &           &    3 &  3.13e-08 &  1.33e-15 &      \\ 
     &           &    4 &  4.65e-08 &  8.92e-16 &      \\ 
     &           &    5 &  3.13e-08 &  1.33e-15 &      \\ 
     &           &    6 &  3.12e-08 &  8.93e-16 &      \\ 
     &           &    7 &  4.65e-08 &  8.92e-16 &      \\ 
     &           &    8 &  3.13e-08 &  1.33e-15 &      \\ 
     &           &    9 &  4.65e-08 &  8.92e-16 &      \\ 
     &           &   10 &  3.13e-08 &  1.33e-15 &      \\ 
 301 &  3.00e+03 &   10 &           &           & iters  \\ 
 \hdashline 
     &           &    1 &  2.66e-08 &  1.60e-08 &      \\ 
     &           &    2 &  5.11e-08 &  7.59e-16 &      \\ 
     &           &    3 &  2.66e-08 &  1.46e-15 &      \\ 
     &           &    4 &  2.66e-08 &  7.60e-16 &      \\ 
     &           &    5 &  5.12e-08 &  7.58e-16 &      \\ 
     &           &    6 &  2.66e-08 &  1.46e-15 &      \\ 
     &           &    7 &  5.11e-08 &  7.59e-16 &      \\ 
     &           &    8 &  2.66e-08 &  1.46e-15 &      \\ 
     &           &    9 &  2.66e-08 &  7.60e-16 &      \\ 
     &           &   10 &  5.11e-08 &  7.59e-16 &      \\ 
 302 &  3.01e+03 &   10 &           &           & iters  \\ 
 \hdashline 
     &           &    1 &  2.93e-08 &  5.12e-09 &      \\ 
     &           &    2 &  2.92e-08 &  8.35e-16 &      \\ 
     &           &    3 &  4.85e-08 &  8.34e-16 &      \\ 
     &           &    4 &  2.93e-08 &  1.38e-15 &      \\ 
     &           &    5 &  2.92e-08 &  8.35e-16 &      \\ 
     &           &    6 &  4.85e-08 &  8.34e-16 &      \\ 
     &           &    7 &  2.92e-08 &  1.38e-15 &      \\ 
     &           &    8 &  4.85e-08 &  8.35e-16 &      \\ 
     &           &    9 &  2.93e-08 &  1.38e-15 &      \\ 
     &           &   10 &  2.92e-08 &  8.35e-16 &      \\ 
 303 &  3.02e+03 &   10 &           &           & iters  \\ 
 \hdashline 
     &           &    1 &  1.19e-08 &  1.38e-08 &      \\ 
     &           &    2 &  1.19e-08 &  3.39e-16 &      \\ 
     &           &    3 &  6.58e-08 &  3.39e-16 &      \\ 
     &           &    4 &  5.08e-08 &  1.88e-15 &      \\ 
     &           &    5 &  1.19e-08 &  1.45e-15 &      \\ 
     &           &    6 &  1.19e-08 &  3.39e-16 &      \\ 
     &           &    7 &  6.59e-08 &  3.38e-16 &      \\ 
     &           &    8 &  1.19e-08 &  1.88e-15 &      \\ 
     &           &    9 &  1.19e-08 &  3.40e-16 &      \\ 
     &           &   10 &  1.19e-08 &  3.40e-16 &      \\ 
 304 &  3.03e+03 &   10 &           &           & iters  \\ 
 \hdashline 
     &           &    1 &  8.06e-09 &  1.89e-08 &      \\ 
     &           &    2 &  8.04e-09 &  2.30e-16 &      \\ 
     &           &    3 &  8.03e-09 &  2.30e-16 &      \\ 
     &           &    4 &  8.02e-09 &  2.29e-16 &      \\ 
     &           &    5 &  8.01e-09 &  2.29e-16 &      \\ 
     &           &    6 &  8.00e-09 &  2.29e-16 &      \\ 
     &           &    7 &  6.97e-08 &  2.28e-16 &      \\ 
     &           &    8 &  4.69e-08 &  1.99e-15 &      \\ 
     &           &    9 &  8.02e-09 &  1.34e-15 &      \\ 
     &           &   10 &  8.01e-09 &  2.29e-16 &      \\ 
 305 &  3.04e+03 &   10 &           &           & iters  \\ 
 \hdashline 
     &           &    1 &  4.56e-08 &  1.12e-08 &      \\ 
     &           &    2 &  3.22e-08 &  1.30e-15 &      \\ 
     &           &    3 &  4.56e-08 &  9.18e-16 &      \\ 
     &           &    4 &  3.22e-08 &  1.30e-15 &      \\ 
     &           &    5 &  3.21e-08 &  9.18e-16 &      \\ 
     &           &    6 &  4.56e-08 &  9.17e-16 &      \\ 
     &           &    7 &  3.21e-08 &  1.30e-15 &      \\ 
     &           &    8 &  4.56e-08 &  9.17e-16 &      \\ 
     &           &    9 &  3.22e-08 &  1.30e-15 &      \\ 
     &           &   10 &  3.21e-08 &  9.18e-16 &      \\ 
 306 &  3.05e+03 &   10 &           &           & iters  \\ 
 \hdashline 
     &           &    1 &  4.30e-08 &  1.54e-08 &      \\ 
     &           &    2 &  3.47e-08 &  1.23e-15 &      \\ 
     &           &    3 &  4.30e-08 &  9.92e-16 &      \\ 
     &           &    4 &  3.48e-08 &  1.23e-15 &      \\ 
     &           &    5 &  3.47e-08 &  9.92e-16 &      \\ 
     &           &    6 &  4.30e-08 &  9.91e-16 &      \\ 
     &           &    7 &  3.47e-08 &  1.23e-15 &      \\ 
     &           &    8 &  4.30e-08 &  9.91e-16 &      \\ 
     &           &    9 &  3.47e-08 &  1.23e-15 &      \\ 
     &           &   10 &  4.30e-08 &  9.91e-16 &      \\ 
 307 &  3.06e+03 &   10 &           &           & iters  \\ 
 \hdashline 
     &           &    1 &  1.65e-09 &  1.79e-08 &      \\ 
     &           &    2 &  1.65e-09 &  4.70e-17 &      \\ 
     &           &    3 &  1.64e-09 &  4.70e-17 &      \\ 
     &           &    4 &  1.64e-09 &  4.69e-17 &      \\ 
     &           &    5 &  1.64e-09 &  4.68e-17 &      \\ 
     &           &    6 &  1.64e-09 &  4.68e-17 &      \\ 
     &           &    7 &  1.63e-09 &  4.67e-17 &      \\ 
     &           &    8 &  1.63e-09 &  4.66e-17 &      \\ 
     &           &    9 &  1.63e-09 &  4.66e-17 &      \\ 
     &           &   10 &  1.63e-09 &  4.65e-17 &      \\ 
 308 &  3.07e+03 &   10 &           &           & iters  \\ 
 \hdashline 
     &           &    1 &  4.07e-08 &  1.24e-08 &      \\ 
     &           &    2 &  3.70e-08 &  1.16e-15 &      \\ 
     &           &    3 &  4.07e-08 &  1.06e-15 &      \\ 
     &           &    4 &  3.71e-08 &  1.16e-15 &      \\ 
     &           &    5 &  4.07e-08 &  1.06e-15 &      \\ 
     &           &    6 &  3.71e-08 &  1.16e-15 &      \\ 
     &           &    7 &  4.07e-08 &  1.06e-15 &      \\ 
     &           &    8 &  3.71e-08 &  1.16e-15 &      \\ 
     &           &    9 &  4.07e-08 &  1.06e-15 &      \\ 
     &           &   10 &  3.71e-08 &  1.16e-15 &      \\ 
 309 &  3.08e+03 &   10 &           &           & iters  \\ 
 \hdashline 
     &           &    1 &  2.46e-08 &  1.32e-08 &      \\ 
     &           &    2 &  5.31e-08 &  7.03e-16 &      \\ 
     &           &    3 &  2.47e-08 &  1.52e-15 &      \\ 
     &           &    4 &  2.46e-08 &  7.04e-16 &      \\ 
     &           &    5 &  5.31e-08 &  7.03e-16 &      \\ 
     &           &    6 &  2.47e-08 &  1.52e-15 &      \\ 
     &           &    7 &  2.46e-08 &  7.04e-16 &      \\ 
     &           &    8 &  2.46e-08 &  7.03e-16 &      \\ 
     &           &    9 &  5.31e-08 &  7.02e-16 &      \\ 
     &           &   10 &  2.46e-08 &  1.52e-15 &      \\ 
 310 &  3.09e+03 &   10 &           &           & iters  \\ 
 \hdashline 
     &           &    1 &  3.73e-08 &  2.14e-09 &      \\ 
     &           &    2 &  4.05e-08 &  1.06e-15 &      \\ 
     &           &    3 &  3.73e-08 &  1.16e-15 &      \\ 
     &           &    4 &  4.05e-08 &  1.06e-15 &      \\ 
     &           &    5 &  3.73e-08 &  1.16e-15 &      \\ 
     &           &    6 &  4.05e-08 &  1.06e-15 &      \\ 
     &           &    7 &  3.73e-08 &  1.16e-15 &      \\ 
     &           &    8 &  4.05e-08 &  1.06e-15 &      \\ 
     &           &    9 &  3.73e-08 &  1.15e-15 &      \\ 
     &           &   10 &  4.05e-08 &  1.06e-15 &      \\ 
 311 &  3.10e+03 &   10 &           &           & iters  \\ 
 \hdashline 
     &           &    1 &  2.17e-08 &  1.47e-09 &      \\ 
     &           &    2 &  2.17e-08 &  6.19e-16 &      \\ 
     &           &    3 &  2.16e-08 &  6.18e-16 &      \\ 
     &           &    4 &  5.61e-08 &  6.17e-16 &      \\ 
     &           &    5 &  2.17e-08 &  1.60e-15 &      \\ 
     &           &    6 &  2.16e-08 &  6.19e-16 &      \\ 
     &           &    7 &  2.16e-08 &  6.18e-16 &      \\ 
     &           &    8 &  5.61e-08 &  6.17e-16 &      \\ 
     &           &    9 &  2.17e-08 &  1.60e-15 &      \\ 
     &           &   10 &  2.16e-08 &  6.18e-16 &      \\ 
 312 &  3.11e+03 &   10 &           &           & iters  \\ 
 \hdashline 
     &           &    1 &  4.03e-08 &  2.33e-08 &      \\ 
     &           &    2 &  3.74e-08 &  1.15e-15 &      \\ 
     &           &    3 &  4.03e-08 &  1.07e-15 &      \\ 
     &           &    4 &  3.74e-08 &  1.15e-15 &      \\ 
     &           &    5 &  4.03e-08 &  1.07e-15 &      \\ 
     &           &    6 &  3.74e-08 &  1.15e-15 &      \\ 
     &           &    7 &  4.03e-08 &  1.07e-15 &      \\ 
     &           &    8 &  3.74e-08 &  1.15e-15 &      \\ 
     &           &    9 &  4.03e-08 &  1.07e-15 &      \\ 
     &           &   10 &  3.75e-08 &  1.15e-15 &      \\ 
 313 &  3.12e+03 &   10 &           &           & iters  \\ 
 \hdashline 
     &           &    1 &  2.40e-08 &  2.89e-08 &      \\ 
     &           &    2 &  5.37e-08 &  6.84e-16 &      \\ 
     &           &    3 &  2.40e-08 &  1.53e-15 &      \\ 
     &           &    4 &  2.40e-08 &  6.86e-16 &      \\ 
     &           &    5 &  5.37e-08 &  6.85e-16 &      \\ 
     &           &    6 &  2.40e-08 &  1.53e-15 &      \\ 
     &           &    7 &  2.40e-08 &  6.86e-16 &      \\ 
     &           &    8 &  5.37e-08 &  6.85e-16 &      \\ 
     &           &    9 &  2.40e-08 &  1.53e-15 &      \\ 
     &           &   10 &  2.40e-08 &  6.86e-16 &      \\ 
 314 &  3.13e+03 &   10 &           &           & iters  \\ 
 \hdashline 
     &           &    1 &  7.43e-09 &  1.46e-08 &      \\ 
     &           &    2 &  7.42e-09 &  2.12e-16 &      \\ 
     &           &    3 &  7.41e-09 &  2.12e-16 &      \\ 
     &           &    4 &  7.40e-09 &  2.11e-16 &      \\ 
     &           &    5 &  7.39e-09 &  2.11e-16 &      \\ 
     &           &    6 &  7.38e-09 &  2.11e-16 &      \\ 
     &           &    7 &  7.03e-08 &  2.11e-16 &      \\ 
     &           &    8 &  7.47e-09 &  2.01e-15 &      \\ 
     &           &    9 &  7.46e-09 &  2.13e-16 &      \\ 
     &           &   10 &  7.45e-09 &  2.13e-16 &      \\ 
 315 &  3.14e+03 &   10 &           &           & iters  \\ 
 \hdashline 
     &           &    1 &  4.51e-08 &  2.55e-08 &      \\ 
     &           &    2 &  3.27e-08 &  1.29e-15 &      \\ 
     &           &    3 &  4.51e-08 &  9.32e-16 &      \\ 
     &           &    4 &  3.27e-08 &  1.29e-15 &      \\ 
     &           &    5 &  4.50e-08 &  9.33e-16 &      \\ 
     &           &    6 &  3.27e-08 &  1.29e-15 &      \\ 
     &           &    7 &  4.50e-08 &  9.33e-16 &      \\ 
     &           &    8 &  3.27e-08 &  1.29e-15 &      \\ 
     &           &    9 &  3.27e-08 &  9.34e-16 &      \\ 
     &           &   10 &  4.51e-08 &  9.33e-16 &      \\ 
 316 &  3.15e+03 &   10 &           &           & iters  \\ 
 \hdashline 
     &           &    1 &  1.65e-09 &  3.24e-08 &      \\ 
     &           &    2 &  1.65e-09 &  4.71e-17 &      \\ 
     &           &    3 &  1.65e-09 &  4.71e-17 &      \\ 
     &           &    4 &  1.64e-09 &  4.70e-17 &      \\ 
     &           &    5 &  1.64e-09 &  4.69e-17 &      \\ 
     &           &    6 &  1.64e-09 &  4.69e-17 &      \\ 
     &           &    7 &  1.64e-09 &  4.68e-17 &      \\ 
     &           &    8 &  1.64e-09 &  4.67e-17 &      \\ 
     &           &    9 &  1.63e-09 &  4.67e-17 &      \\ 
     &           &   10 &  1.63e-09 &  4.66e-17 &      \\ 
 317 &  3.16e+03 &   10 &           &           & iters  \\ 
 \hdashline 
     &           &    1 &  1.40e-09 &  1.98e-08 &      \\ 
     &           &    2 &  1.40e-09 &  4.01e-17 &      \\ 
     &           &    3 &  1.40e-09 &  4.00e-17 &      \\ 
     &           &    4 &  1.40e-09 &  4.00e-17 &      \\ 
     &           &    5 &  1.40e-09 &  3.99e-17 &      \\ 
     &           &    6 &  1.39e-09 &  3.98e-17 &      \\ 
     &           &    7 &  1.39e-09 &  3.98e-17 &      \\ 
     &           &    8 &  1.39e-09 &  3.97e-17 &      \\ 
     &           &    9 &  1.39e-09 &  3.97e-17 &      \\ 
     &           &   10 &  7.63e-08 &  3.96e-17 &      \\ 
 318 &  3.17e+03 &   10 &           &           & iters  \\ 
 \hdashline 
     &           &    1 &  3.06e-08 &  1.82e-08 &      \\ 
     &           &    2 &  4.72e-08 &  8.72e-16 &      \\ 
     &           &    3 &  3.06e-08 &  1.35e-15 &      \\ 
     &           &    4 &  3.05e-08 &  8.73e-16 &      \\ 
     &           &    5 &  4.72e-08 &  8.72e-16 &      \\ 
     &           &    6 &  3.06e-08 &  1.35e-15 &      \\ 
     &           &    7 &  4.72e-08 &  8.72e-16 &      \\ 
     &           &    8 &  3.06e-08 &  1.35e-15 &      \\ 
     &           &    9 &  3.05e-08 &  8.73e-16 &      \\ 
     &           &   10 &  4.72e-08 &  8.72e-16 &      \\ 
 319 &  3.18e+03 &   10 &           &           & iters  \\ 
 \hdashline 
     &           &    1 &  3.71e-08 &  1.84e-08 &      \\ 
     &           &    2 &  3.70e-08 &  1.06e-15 &      \\ 
     &           &    3 &  4.07e-08 &  1.06e-15 &      \\ 
     &           &    4 &  3.70e-08 &  1.16e-15 &      \\ 
     &           &    5 &  4.07e-08 &  1.06e-15 &      \\ 
     &           &    6 &  3.70e-08 &  1.16e-15 &      \\ 
     &           &    7 &  4.07e-08 &  1.06e-15 &      \\ 
     &           &    8 &  3.70e-08 &  1.16e-15 &      \\ 
     &           &    9 &  4.07e-08 &  1.06e-15 &      \\ 
     &           &   10 &  3.70e-08 &  1.16e-15 &      \\ 
 320 &  3.19e+03 &   10 &           &           & iters  \\ 
 \hdashline 
     &           &    1 &  2.73e-08 &  1.51e-08 &      \\ 
     &           &    2 &  5.04e-08 &  7.79e-16 &      \\ 
     &           &    3 &  2.73e-08 &  1.44e-15 &      \\ 
     &           &    4 &  2.73e-08 &  7.80e-16 &      \\ 
     &           &    5 &  5.04e-08 &  7.79e-16 &      \\ 
     &           &    6 &  2.73e-08 &  1.44e-15 &      \\ 
     &           &    7 &  2.73e-08 &  7.80e-16 &      \\ 
     &           &    8 &  5.04e-08 &  7.79e-16 &      \\ 
     &           &    9 &  2.73e-08 &  1.44e-15 &      \\ 
     &           &   10 &  2.73e-08 &  7.80e-16 &      \\ 
 321 &  3.20e+03 &   10 &           &           & iters  \\ 
 \hdashline 
     &           &    1 &  3.59e-09 &  7.81e-09 &      \\ 
     &           &    2 &  3.59e-09 &  1.03e-16 &      \\ 
     &           &    3 &  7.41e-08 &  1.02e-16 &      \\ 
     &           &    4 &  8.14e-08 &  2.11e-15 &      \\ 
     &           &    5 &  7.41e-08 &  2.32e-15 &      \\ 
     &           &    6 &  8.14e-08 &  2.12e-15 &      \\ 
     &           &    7 &  7.41e-08 &  2.32e-15 &      \\ 
     &           &    8 &  3.67e-09 &  2.12e-15 &      \\ 
     &           &    9 &  3.66e-09 &  1.05e-16 &      \\ 
     &           &   10 &  3.66e-09 &  1.05e-16 &      \\ 
 322 &  3.21e+03 &   10 &           &           & iters  \\ 
 \hdashline 
     &           &    1 &  1.47e-09 &  4.78e-09 &      \\ 
     &           &    2 &  1.46e-09 &  4.19e-17 &      \\ 
     &           &    3 &  1.46e-09 &  4.18e-17 &      \\ 
     &           &    4 &  1.46e-09 &  4.17e-17 &      \\ 
     &           &    5 &  1.46e-09 &  4.17e-17 &      \\ 
     &           &    6 &  1.46e-09 &  4.16e-17 &      \\ 
     &           &    7 &  1.45e-09 &  4.16e-17 &      \\ 
     &           &    8 &  1.45e-09 &  4.15e-17 &      \\ 
     &           &    9 &  1.45e-09 &  4.14e-17 &      \\ 
     &           &   10 &  1.45e-09 &  4.14e-17 &      \\ 
 323 &  3.22e+03 &   10 &           &           & iters  \\ 
 \hdashline 
     &           &    1 &  1.67e-08 &  9.45e-09 &      \\ 
     &           &    2 &  1.67e-08 &  4.77e-16 &      \\ 
     &           &    3 &  6.10e-08 &  4.76e-16 &      \\ 
     &           &    4 &  1.67e-08 &  1.74e-15 &      \\ 
     &           &    5 &  1.67e-08 &  4.78e-16 &      \\ 
     &           &    6 &  1.67e-08 &  4.77e-16 &      \\ 
     &           &    7 &  1.67e-08 &  4.77e-16 &      \\ 
     &           &    8 &  6.10e-08 &  4.76e-16 &      \\ 
     &           &    9 &  1.67e-08 &  1.74e-15 &      \\ 
     &           &   10 &  1.67e-08 &  4.78e-16 &      \\ 
 324 &  3.23e+03 &   10 &           &           & iters  \\ 
 \hdashline 
     &           &    1 &  2.83e-09 &  6.32e-09 &      \\ 
     &           &    2 &  2.82e-09 &  8.07e-17 &      \\ 
     &           &    3 &  2.82e-09 &  8.06e-17 &      \\ 
     &           &    4 &  2.81e-09 &  8.04e-17 &      \\ 
     &           &    5 &  2.81e-09 &  8.03e-17 &      \\ 
     &           &    6 &  2.81e-09 &  8.02e-17 &      \\ 
     &           &    7 &  7.49e-08 &  8.01e-17 &      \\ 
     &           &    8 &  8.06e-08 &  2.14e-15 &      \\ 
     &           &    9 &  7.49e-08 &  2.30e-15 &      \\ 
     &           &   10 &  8.06e-08 &  2.14e-15 &      \\ 
 325 &  3.24e+03 &   10 &           &           & iters  \\ 
 \hdashline 
     &           &    1 &  3.18e-08 &  9.85e-09 &      \\ 
     &           &    2 &  4.59e-08 &  9.08e-16 &      \\ 
     &           &    3 &  3.18e-08 &  1.31e-15 &      \\ 
     &           &    4 &  3.18e-08 &  9.09e-16 &      \\ 
     &           &    5 &  4.59e-08 &  9.08e-16 &      \\ 
     &           &    6 &  3.18e-08 &  1.31e-15 &      \\ 
     &           &    7 &  4.59e-08 &  9.08e-16 &      \\ 
     &           &    8 &  3.18e-08 &  1.31e-15 &      \\ 
     &           &    9 &  3.18e-08 &  9.09e-16 &      \\ 
     &           &   10 &  4.59e-08 &  9.08e-16 &      \\ 
 326 &  3.25e+03 &   10 &           &           & iters  \\ 
 \hdashline 
     &           &    1 &  2.35e-08 &  5.76e-09 &      \\ 
     &           &    2 &  2.35e-08 &  6.72e-16 &      \\ 
     &           &    3 &  5.42e-08 &  6.71e-16 &      \\ 
     &           &    4 &  2.35e-08 &  1.55e-15 &      \\ 
     &           &    5 &  2.35e-08 &  6.72e-16 &      \\ 
     &           &    6 &  5.42e-08 &  6.71e-16 &      \\ 
     &           &    7 &  2.36e-08 &  1.55e-15 &      \\ 
     &           &    8 &  2.35e-08 &  6.72e-16 &      \\ 
     &           &    9 &  5.42e-08 &  6.71e-16 &      \\ 
     &           &   10 &  2.36e-08 &  1.55e-15 &      \\ 
 327 &  3.26e+03 &   10 &           &           & iters  \\ 
 \hdashline 
     &           &    1 &  4.41e-08 &  2.47e-08 &      \\ 
     &           &    2 &  3.36e-08 &  1.26e-15 &      \\ 
     &           &    3 &  3.36e-08 &  9.59e-16 &      \\ 
     &           &    4 &  4.42e-08 &  9.58e-16 &      \\ 
     &           &    5 &  3.36e-08 &  1.26e-15 &      \\ 
     &           &    6 &  4.42e-08 &  9.58e-16 &      \\ 
     &           &    7 &  3.36e-08 &  1.26e-15 &      \\ 
     &           &    8 &  4.41e-08 &  9.59e-16 &      \\ 
     &           &    9 &  3.36e-08 &  1.26e-15 &      \\ 
     &           &   10 &  4.41e-08 &  9.59e-16 &      \\ 
 328 &  3.27e+03 &   10 &           &           & iters  \\ 
 \hdashline 
     &           &    1 &  1.59e-09 &  2.01e-08 &      \\ 
     &           &    2 &  1.59e-09 &  4.53e-17 &      \\ 
     &           &    3 &  1.58e-09 &  4.53e-17 &      \\ 
     &           &    4 &  1.58e-09 &  4.52e-17 &      \\ 
     &           &    5 &  1.58e-09 &  4.51e-17 &      \\ 
     &           &    6 &  1.58e-09 &  4.51e-17 &      \\ 
     &           &    7 &  1.57e-09 &  4.50e-17 &      \\ 
     &           &    8 &  1.57e-09 &  4.49e-17 &      \\ 
     &           &    9 &  1.57e-09 &  4.49e-17 &      \\ 
     &           &   10 &  1.57e-09 &  4.48e-17 &      \\ 
 329 &  3.28e+03 &   10 &           &           & iters  \\ 
 \hdashline 
     &           &    1 &  5.60e-08 &  2.98e-08 &      \\ 
     &           &    2 &  2.17e-08 &  1.60e-15 &      \\ 
     &           &    3 &  2.17e-08 &  6.20e-16 &      \\ 
     &           &    4 &  5.60e-08 &  6.19e-16 &      \\ 
     &           &    5 &  2.17e-08 &  1.60e-15 &      \\ 
     &           &    6 &  2.17e-08 &  6.21e-16 &      \\ 
     &           &    7 &  5.60e-08 &  6.20e-16 &      \\ 
     &           &    8 &  2.18e-08 &  1.60e-15 &      \\ 
     &           &    9 &  2.17e-08 &  6.21e-16 &      \\ 
     &           &   10 &  2.17e-08 &  6.20e-16 &      \\ 
 330 &  3.29e+03 &   10 &           &           & iters  \\ 
 \hdashline 
     &           &    1 &  2.62e-08 &  1.88e-08 &      \\ 
     &           &    2 &  5.15e-08 &  7.47e-16 &      \\ 
     &           &    3 &  2.62e-08 &  1.47e-15 &      \\ 
     &           &    4 &  2.62e-08 &  7.48e-16 &      \\ 
     &           &    5 &  5.15e-08 &  7.47e-16 &      \\ 
     &           &    6 &  2.62e-08 &  1.47e-15 &      \\ 
     &           &    7 &  2.62e-08 &  7.48e-16 &      \\ 
     &           &    8 &  5.15e-08 &  7.47e-16 &      \\ 
     &           &    9 &  2.62e-08 &  1.47e-15 &      \\ 
     &           &   10 &  2.62e-08 &  7.48e-16 &      \\ 
 331 &  3.30e+03 &   10 &           &           & iters  \\ 
 \hdashline 
     &           &    1 &  1.39e-08 &  2.52e-08 &      \\ 
     &           &    2 &  6.38e-08 &  3.96e-16 &      \\ 
     &           &    3 &  1.40e-08 &  1.82e-15 &      \\ 
     &           &    4 &  1.39e-08 &  3.98e-16 &      \\ 
     &           &    5 &  1.39e-08 &  3.98e-16 &      \\ 
     &           &    6 &  1.39e-08 &  3.97e-16 &      \\ 
     &           &    7 &  1.39e-08 &  3.96e-16 &      \\ 
     &           &    8 &  6.38e-08 &  3.96e-16 &      \\ 
     &           &    9 &  1.39e-08 &  1.82e-15 &      \\ 
     &           &   10 &  1.39e-08 &  3.98e-16 &      \\ 
 332 &  3.31e+03 &   10 &           &           & iters  \\ 
 \hdashline 
     &           &    1 &  2.28e-09 &  2.51e-08 &      \\ 
     &           &    2 &  2.28e-09 &  6.51e-17 &      \\ 
     &           &    3 &  2.27e-09 &  6.50e-17 &      \\ 
     &           &    4 &  2.27e-09 &  6.49e-17 &      \\ 
     &           &    5 &  2.27e-09 &  6.48e-17 &      \\ 
     &           &    6 &  2.26e-09 &  6.47e-17 &      \\ 
     &           &    7 &  2.26e-09 &  6.46e-17 &      \\ 
     &           &    8 &  2.26e-09 &  6.45e-17 &      \\ 
     &           &    9 &  2.25e-09 &  6.44e-17 &      \\ 
     &           &   10 &  2.25e-09 &  6.43e-17 &      \\ 
 333 &  3.32e+03 &   10 &           &           & iters  \\ 
 \hdashline 
     &           &    1 &  4.94e-08 &  1.68e-08 &      \\ 
     &           &    2 &  4.93e-08 &  1.41e-15 &      \\ 
     &           &    3 &  2.85e-08 &  1.41e-15 &      \\ 
     &           &    4 &  2.84e-08 &  8.13e-16 &      \\ 
     &           &    5 &  4.93e-08 &  8.12e-16 &      \\ 
     &           &    6 &  2.85e-08 &  1.41e-15 &      \\ 
     &           &    7 &  4.93e-08 &  8.13e-16 &      \\ 
     &           &    8 &  2.85e-08 &  1.41e-15 &      \\ 
     &           &    9 &  2.85e-08 &  8.13e-16 &      \\ 
     &           &   10 &  4.93e-08 &  8.12e-16 &      \\ 
 334 &  3.33e+03 &   10 &           &           & iters  \\ 
 \hdashline 
     &           &    1 &  5.91e-08 &  3.96e-08 &      \\ 
     &           &    2 &  5.90e-08 &  1.69e-15 &      \\ 
     &           &    3 &  1.88e-08 &  1.68e-15 &      \\ 
     &           &    4 &  1.87e-08 &  5.35e-16 &      \\ 
     &           &    5 &  1.87e-08 &  5.34e-16 &      \\ 
     &           &    6 &  5.90e-08 &  5.34e-16 &      \\ 
     &           &    7 &  1.88e-08 &  1.68e-15 &      \\ 
     &           &    8 &  1.87e-08 &  5.35e-16 &      \\ 
     &           &    9 &  1.87e-08 &  5.35e-16 &      \\ 
     &           &   10 &  5.90e-08 &  5.34e-16 &      \\ 
 335 &  3.34e+03 &   10 &           &           & iters  \\ 
 \hdashline 
     &           &    1 &  5.62e-09 &  2.42e-08 &      \\ 
     &           &    2 &  5.61e-09 &  1.60e-16 &      \\ 
     &           &    3 &  5.60e-09 &  1.60e-16 &      \\ 
     &           &    4 &  5.59e-09 &  1.60e-16 &      \\ 
     &           &    5 &  5.59e-09 &  1.60e-16 &      \\ 
     &           &    6 &  5.58e-09 &  1.59e-16 &      \\ 
     &           &    7 &  5.57e-09 &  1.59e-16 &      \\ 
     &           &    8 &  5.56e-09 &  1.59e-16 &      \\ 
     &           &    9 &  5.55e-09 &  1.59e-16 &      \\ 
     &           &   10 &  5.55e-09 &  1.59e-16 &      \\ 
 336 &  3.35e+03 &   10 &           &           & iters  \\ 
 \hdashline 
     &           &    1 &  2.44e-08 &  9.34e-09 &      \\ 
     &           &    2 &  5.33e-08 &  6.97e-16 &      \\ 
     &           &    3 &  2.45e-08 &  1.52e-15 &      \\ 
     &           &    4 &  2.44e-08 &  6.99e-16 &      \\ 
     &           &    5 &  2.44e-08 &  6.98e-16 &      \\ 
     &           &    6 &  5.33e-08 &  6.97e-16 &      \\ 
     &           &    7 &  2.45e-08 &  1.52e-15 &      \\ 
     &           &    8 &  2.44e-08 &  6.98e-16 &      \\ 
     &           &    9 &  5.33e-08 &  6.97e-16 &      \\ 
     &           &   10 &  2.45e-08 &  1.52e-15 &      \\ 
 337 &  3.36e+03 &   10 &           &           & iters  \\ 
 \hdashline 
     &           &    1 &  3.40e-08 &  4.76e-08 &      \\ 
     &           &    2 &  4.37e-08 &  9.72e-16 &      \\ 
     &           &    3 &  3.41e-08 &  1.25e-15 &      \\ 
     &           &    4 &  4.37e-08 &  9.72e-16 &      \\ 
     &           &    5 &  3.41e-08 &  1.25e-15 &      \\ 
     &           &    6 &  4.37e-08 &  9.72e-16 &      \\ 
     &           &    7 &  3.41e-08 &  1.25e-15 &      \\ 
     &           &    8 &  3.40e-08 &  9.73e-16 &      \\ 
     &           &    9 &  4.37e-08 &  9.71e-16 &      \\ 
     &           &   10 &  3.41e-08 &  1.25e-15 &      \\ 
 338 &  3.37e+03 &   10 &           &           & iters  \\ 
 \hdashline 
     &           &    1 &  5.84e-08 &  4.64e-08 &      \\ 
     &           &    2 &  1.94e-08 &  1.67e-15 &      \\ 
     &           &    3 &  1.93e-08 &  5.52e-16 &      \\ 
     &           &    4 &  5.84e-08 &  5.52e-16 &      \\ 
     &           &    5 &  1.94e-08 &  1.67e-15 &      \\ 
     &           &    6 &  1.94e-08 &  5.53e-16 &      \\ 
     &           &    7 &  1.93e-08 &  5.52e-16 &      \\ 
     &           &    8 &  5.84e-08 &  5.52e-16 &      \\ 
     &           &    9 &  1.94e-08 &  1.67e-15 &      \\ 
     &           &   10 &  1.94e-08 &  5.53e-16 &      \\ 
 339 &  3.38e+03 &   10 &           &           & iters  \\ 
 \hdashline 
     &           &    1 &  1.39e-08 &  4.64e-09 &      \\ 
     &           &    2 &  6.38e-08 &  3.96e-16 &      \\ 
     &           &    3 &  1.40e-08 &  1.82e-15 &      \\ 
     &           &    4 &  1.39e-08 &  3.98e-16 &      \\ 
     &           &    5 &  1.39e-08 &  3.98e-16 &      \\ 
     &           &    6 &  1.39e-08 &  3.97e-16 &      \\ 
     &           &    7 &  1.39e-08 &  3.97e-16 &      \\ 
     &           &    8 &  6.38e-08 &  3.96e-16 &      \\ 
     &           &    9 &  1.40e-08 &  1.82e-15 &      \\ 
     &           &   10 &  1.39e-08 &  3.98e-16 &      \\ 
 340 &  3.39e+03 &   10 &           &           & iters  \\ 
 \hdashline 
     &           &    1 &  5.49e-08 &  3.60e-09 &      \\ 
     &           &    2 &  2.29e-08 &  1.57e-15 &      \\ 
     &           &    3 &  2.28e-08 &  6.52e-16 &      \\ 
     &           &    4 &  5.49e-08 &  6.51e-16 &      \\ 
     &           &    5 &  2.29e-08 &  1.57e-15 &      \\ 
     &           &    6 &  2.28e-08 &  6.53e-16 &      \\ 
     &           &    7 &  5.49e-08 &  6.52e-16 &      \\ 
     &           &    8 &  2.29e-08 &  1.57e-15 &      \\ 
     &           &    9 &  2.28e-08 &  6.53e-16 &      \\ 
     &           &   10 &  2.28e-08 &  6.52e-16 &      \\ 
 341 &  3.40e+03 &   10 &           &           & iters  \\ 
 \hdashline 
     &           &    1 &  5.57e-09 &  2.36e-08 &      \\ 
     &           &    2 &  5.56e-09 &  1.59e-16 &      \\ 
     &           &    3 &  5.55e-09 &  1.59e-16 &      \\ 
     &           &    4 &  5.55e-09 &  1.59e-16 &      \\ 
     &           &    5 &  5.54e-09 &  1.58e-16 &      \\ 
     &           &    6 &  5.53e-09 &  1.58e-16 &      \\ 
     &           &    7 &  5.52e-09 &  1.58e-16 &      \\ 
     &           &    8 &  5.51e-09 &  1.58e-16 &      \\ 
     &           &    9 &  5.51e-09 &  1.57e-16 &      \\ 
     &           &   10 &  5.50e-09 &  1.57e-16 &      \\ 
 342 &  3.41e+03 &   10 &           &           & iters  \\ 
 \hdashline 
     &           &    1 &  2.85e-08 &  1.75e-08 &      \\ 
     &           &    2 &  2.85e-08 &  8.13e-16 &      \\ 
     &           &    3 &  4.93e-08 &  8.12e-16 &      \\ 
     &           &    4 &  2.85e-08 &  1.41e-15 &      \\ 
     &           &    5 &  2.84e-08 &  8.13e-16 &      \\ 
     &           &    6 &  4.93e-08 &  8.12e-16 &      \\ 
     &           &    7 &  2.85e-08 &  1.41e-15 &      \\ 
     &           &    8 &  4.93e-08 &  8.13e-16 &      \\ 
     &           &    9 &  2.85e-08 &  1.41e-15 &      \\ 
     &           &   10 &  2.85e-08 &  8.14e-16 &      \\ 
 343 &  3.42e+03 &   10 &           &           & iters  \\ 
 \hdashline 
     &           &    1 &  5.14e-08 &  7.55e-10 &      \\ 
     &           &    2 &  2.64e-08 &  1.47e-15 &      \\ 
     &           &    3 &  5.13e-08 &  7.53e-16 &      \\ 
     &           &    4 &  2.64e-08 &  1.46e-15 &      \\ 
     &           &    5 &  2.64e-08 &  7.54e-16 &      \\ 
     &           &    6 &  5.13e-08 &  7.53e-16 &      \\ 
     &           &    7 &  2.64e-08 &  1.47e-15 &      \\ 
     &           &    8 &  2.64e-08 &  7.54e-16 &      \\ 
     &           &    9 &  5.13e-08 &  7.53e-16 &      \\ 
     &           &   10 &  2.64e-08 &  1.47e-15 &      \\ 
 344 &  3.43e+03 &   10 &           &           & iters  \\ 
 \hdashline 
     &           &    1 &  1.60e-08 &  1.25e-08 &      \\ 
     &           &    2 &  1.59e-08 &  4.55e-16 &      \\ 
     &           &    3 &  1.59e-08 &  4.55e-16 &      \\ 
     &           &    4 &  1.59e-08 &  4.54e-16 &      \\ 
     &           &    5 &  1.59e-08 &  4.54e-16 &      \\ 
     &           &    6 &  6.18e-08 &  4.53e-16 &      \\ 
     &           &    7 &  1.59e-08 &  1.77e-15 &      \\ 
     &           &    8 &  1.59e-08 &  4.55e-16 &      \\ 
     &           &    9 &  1.59e-08 &  4.54e-16 &      \\ 
     &           &   10 &  1.59e-08 &  4.53e-16 &      \\ 
 345 &  3.44e+03 &   10 &           &           & iters  \\ 
 \hdashline 
     &           &    1 &  3.12e-08 &  4.64e-09 &      \\ 
     &           &    2 &  3.11e-08 &  8.90e-16 &      \\ 
     &           &    3 &  3.11e-08 &  8.89e-16 &      \\ 
     &           &    4 &  4.66e-08 &  8.87e-16 &      \\ 
     &           &    5 &  3.11e-08 &  1.33e-15 &      \\ 
     &           &    6 &  3.11e-08 &  8.88e-16 &      \\ 
     &           &    7 &  4.67e-08 &  8.87e-16 &      \\ 
     &           &    8 &  3.11e-08 &  1.33e-15 &      \\ 
     &           &    9 &  4.66e-08 &  8.87e-16 &      \\ 
     &           &   10 &  3.11e-08 &  1.33e-15 &      \\ 
 346 &  3.45e+03 &   10 &           &           & iters  \\ 
 \hdashline 
     &           &    1 &  5.48e-09 &  2.88e-11 &      \\ 
     &           &    2 &  5.47e-09 &  1.56e-16 &      \\ 
     &           &    3 &  5.47e-09 &  1.56e-16 &      \\ 
     &           &    4 &  5.46e-09 &  1.56e-16 &      \\ 
     &           &    5 &  5.45e-09 &  1.56e-16 &      \\ 
     &           &    6 &  7.22e-08 &  1.56e-16 &      \\ 
     &           &    7 &  8.32e-08 &  2.06e-15 &      \\ 
     &           &    8 &  7.23e-08 &  2.38e-15 &      \\ 
     &           &    9 &  5.53e-09 &  2.06e-15 &      \\ 
     &           &   10 &  5.52e-09 &  1.58e-16 &      \\ 
 347 &  3.46e+03 &   10 &           &           & iters  \\ 
 \hdashline 
     &           &    1 &  4.47e-08 &  8.25e-10 &      \\ 
     &           &    2 &  3.31e-08 &  1.28e-15 &      \\ 
     &           &    3 &  4.47e-08 &  9.44e-16 &      \\ 
     &           &    4 &  3.31e-08 &  1.27e-15 &      \\ 
     &           &    5 &  3.30e-08 &  9.44e-16 &      \\ 
     &           &    6 &  4.47e-08 &  9.43e-16 &      \\ 
     &           &    7 &  3.30e-08 &  1.28e-15 &      \\ 
     &           &    8 &  4.47e-08 &  9.43e-16 &      \\ 
     &           &    9 &  3.31e-08 &  1.28e-15 &      \\ 
     &           &   10 &  3.30e-08 &  9.44e-16 &      \\ 
 348 &  3.47e+03 &   10 &           &           & iters  \\ 
 \hdashline 
     &           &    1 &  6.19e-09 &  1.29e-08 &      \\ 
     &           &    2 &  6.18e-09 &  1.77e-16 &      \\ 
     &           &    3 &  6.17e-09 &  1.76e-16 &      \\ 
     &           &    4 &  6.17e-09 &  1.76e-16 &      \\ 
     &           &    5 &  6.16e-09 &  1.76e-16 &      \\ 
     &           &    6 &  6.15e-09 &  1.76e-16 &      \\ 
     &           &    7 &  6.14e-09 &  1.75e-16 &      \\ 
     &           &    8 &  7.16e-08 &  1.75e-16 &      \\ 
     &           &    9 &  8.39e-08 &  2.04e-15 &      \\ 
     &           &   10 &  7.16e-08 &  2.40e-15 &      \\ 
 349 &  3.48e+03 &   10 &           &           & iters  \\ 
 \hdashline 
     &           &    1 &  1.14e-08 &  1.46e-08 &      \\ 
     &           &    2 &  6.63e-08 &  3.26e-16 &      \\ 
     &           &    3 &  8.92e-08 &  1.89e-15 &      \\ 
     &           &    4 &  6.63e-08 &  2.55e-15 &      \\ 
     &           &    5 &  1.15e-08 &  1.89e-15 &      \\ 
     &           &    6 &  1.14e-08 &  3.27e-16 &      \\ 
     &           &    7 &  1.14e-08 &  3.27e-16 &      \\ 
     &           &    8 &  1.14e-08 &  3.26e-16 &      \\ 
     &           &    9 &  6.63e-08 &  3.26e-16 &      \\ 
     &           &   10 &  1.15e-08 &  1.89e-15 &      \\ 
 350 &  3.49e+03 &   10 &           &           & iters  \\ 
 \hdashline 
     &           &    1 &  1.65e-08 &  1.54e-08 &      \\ 
     &           &    2 &  6.13e-08 &  4.70e-16 &      \\ 
     &           &    3 &  1.65e-08 &  1.75e-15 &      \\ 
     &           &    4 &  1.65e-08 &  4.71e-16 &      \\ 
     &           &    5 &  1.65e-08 &  4.71e-16 &      \\ 
     &           &    6 &  1.64e-08 &  4.70e-16 &      \\ 
     &           &    7 &  6.13e-08 &  4.69e-16 &      \\ 
     &           &    8 &  1.65e-08 &  1.75e-15 &      \\ 
     &           &    9 &  1.65e-08 &  4.71e-16 &      \\ 
     &           &   10 &  1.65e-08 &  4.70e-16 &      \\ 
 351 &  3.50e+03 &   10 &           &           & iters  \\ 
 \hdashline 
     &           &    1 &  2.09e-08 &  2.51e-08 &      \\ 
     &           &    2 &  2.08e-08 &  5.95e-16 &      \\ 
     &           &    3 &  5.69e-08 &  5.94e-16 &      \\ 
     &           &    4 &  2.09e-08 &  1.62e-15 &      \\ 
     &           &    5 &  2.08e-08 &  5.96e-16 &      \\ 
     &           &    6 &  5.69e-08 &  5.95e-16 &      \\ 
     &           &    7 &  2.09e-08 &  1.62e-15 &      \\ 
     &           &    8 &  2.09e-08 &  5.96e-16 &      \\ 
     &           &    9 &  2.08e-08 &  5.96e-16 &      \\ 
     &           &   10 &  5.69e-08 &  5.95e-16 &      \\ 
 352 &  3.51e+03 &   10 &           &           & iters  \\ 
 \hdashline 
     &           &    1 &  2.48e-08 &  2.36e-09 &      \\ 
     &           &    2 &  2.48e-08 &  7.09e-16 &      \\ 
     &           &    3 &  1.41e-08 &  7.08e-16 &      \\ 
     &           &    4 &  1.40e-08 &  4.02e-16 &      \\ 
     &           &    5 &  2.48e-08 &  4.01e-16 &      \\ 
     &           &    6 &  1.41e-08 &  7.08e-16 &      \\ 
     &           &    7 &  1.40e-08 &  4.01e-16 &      \\ 
     &           &    8 &  2.48e-08 &  4.01e-16 &      \\ 
     &           &    9 &  1.41e-08 &  7.08e-16 &      \\ 
     &           &   10 &  2.48e-08 &  4.01e-16 &      \\ 
 353 &  3.52e+03 &   10 &           &           & iters  \\ 
 \hdashline 
     &           &    1 &  5.81e-08 &  3.39e-08 &      \\ 
     &           &    2 &  1.96e-08 &  1.66e-15 &      \\ 
     &           &    3 &  1.96e-08 &  5.60e-16 &      \\ 
     &           &    4 &  5.81e-08 &  5.59e-16 &      \\ 
     &           &    5 &  1.97e-08 &  1.66e-15 &      \\ 
     &           &    6 &  1.96e-08 &  5.61e-16 &      \\ 
     &           &    7 &  1.96e-08 &  5.60e-16 &      \\ 
     &           &    8 &  5.81e-08 &  5.59e-16 &      \\ 
     &           &    9 &  1.97e-08 &  1.66e-15 &      \\ 
     &           &   10 &  1.96e-08 &  5.61e-16 &      \\ 
 354 &  3.53e+03 &   10 &           &           & iters  \\ 
 \hdashline 
     &           &    1 &  2.06e-08 &  2.74e-08 &      \\ 
     &           &    2 &  5.71e-08 &  5.88e-16 &      \\ 
     &           &    3 &  2.06e-08 &  1.63e-15 &      \\ 
     &           &    4 &  2.06e-08 &  5.89e-16 &      \\ 
     &           &    5 &  2.06e-08 &  5.88e-16 &      \\ 
     &           &    6 &  5.71e-08 &  5.88e-16 &      \\ 
     &           &    7 &  2.06e-08 &  1.63e-15 &      \\ 
     &           &    8 &  2.06e-08 &  5.89e-16 &      \\ 
     &           &    9 &  2.06e-08 &  5.88e-16 &      \\ 
     &           &   10 &  5.71e-08 &  5.87e-16 &      \\ 
 355 &  3.54e+03 &   10 &           &           & iters  \\ 
 \hdashline 
     &           &    1 &  1.74e-08 &  2.07e-09 &      \\ 
     &           &    2 &  1.74e-08 &  4.97e-16 &      \\ 
     &           &    3 &  6.03e-08 &  4.97e-16 &      \\ 
     &           &    4 &  1.75e-08 &  1.72e-15 &      \\ 
     &           &    5 &  1.74e-08 &  4.98e-16 &      \\ 
     &           &    6 &  1.74e-08 &  4.98e-16 &      \\ 
     &           &    7 &  1.74e-08 &  4.97e-16 &      \\ 
     &           &    8 &  6.03e-08 &  4.96e-16 &      \\ 
     &           &    9 &  1.74e-08 &  1.72e-15 &      \\ 
     &           &   10 &  1.74e-08 &  4.98e-16 &      \\ 
 356 &  3.55e+03 &   10 &           &           & iters  \\ 
 \hdashline 
     &           &    1 &  3.08e-09 &  2.37e-08 &      \\ 
     &           &    2 &  3.08e-09 &  8.80e-17 &      \\ 
     &           &    3 &  3.07e-09 &  8.79e-17 &      \\ 
     &           &    4 &  3.07e-09 &  8.78e-17 &      \\ 
     &           &    5 &  3.07e-09 &  8.76e-17 &      \\ 
     &           &    6 &  3.06e-09 &  8.75e-17 &      \\ 
     &           &    7 &  3.06e-09 &  8.74e-17 &      \\ 
     &           &    8 &  3.05e-09 &  8.73e-17 &      \\ 
     &           &    9 &  3.05e-09 &  8.71e-17 &      \\ 
     &           &   10 &  7.46e-08 &  8.70e-17 &      \\ 
 357 &  3.56e+03 &   10 &           &           & iters  \\ 
 \hdashline 
     &           &    1 &  3.04e-08 &  3.17e-08 &      \\ 
     &           &    2 &  8.44e-09 &  8.69e-16 &      \\ 
     &           &    3 &  8.43e-09 &  2.41e-16 &      \\ 
     &           &    4 &  8.42e-09 &  2.41e-16 &      \\ 
     &           &    5 &  8.40e-09 &  2.40e-16 &      \\ 
     &           &    6 &  3.05e-08 &  2.40e-16 &      \\ 
     &           &    7 &  8.44e-09 &  8.69e-16 &      \\ 
     &           &    8 &  8.42e-09 &  2.41e-16 &      \\ 
     &           &    9 &  8.41e-09 &  2.40e-16 &      \\ 
     &           &   10 &  8.40e-09 &  2.40e-16 &      \\ 
 358 &  3.57e+03 &   10 &           &           & iters  \\ 
 \hdashline 
     &           &    1 &  2.88e-08 &  2.20e-08 &      \\ 
     &           &    2 &  2.88e-08 &  8.22e-16 &      \\ 
     &           &    3 &  4.90e-08 &  8.21e-16 &      \\ 
     &           &    4 &  2.88e-08 &  1.40e-15 &      \\ 
     &           &    5 &  2.88e-08 &  8.22e-16 &      \\ 
     &           &    6 &  4.90e-08 &  8.21e-16 &      \\ 
     &           &    7 &  2.88e-08 &  1.40e-15 &      \\ 
     &           &    8 &  4.89e-08 &  8.21e-16 &      \\ 
     &           &    9 &  2.88e-08 &  1.40e-15 &      \\ 
     &           &   10 &  2.88e-08 &  8.22e-16 &      \\ 
 359 &  3.58e+03 &   10 &           &           & iters  \\ 
 \hdashline 
     &           &    1 &  3.97e-08 &  8.95e-09 &      \\ 
     &           &    2 &  7.85e-10 &  1.13e-15 &      \\ 
     &           &    3 &  7.84e-10 &  2.24e-17 &      \\ 
     &           &    4 &  7.83e-10 &  2.24e-17 &      \\ 
     &           &    5 &  7.82e-10 &  2.23e-17 &      \\ 
     &           &    6 &  7.81e-10 &  2.23e-17 &      \\ 
     &           &    7 &  7.80e-10 &  2.23e-17 &      \\ 
     &           &    8 &  7.78e-10 &  2.23e-17 &      \\ 
     &           &    9 &  7.77e-10 &  2.22e-17 &      \\ 
     &           &   10 &  7.76e-10 &  2.22e-17 &      \\ 
 360 &  3.59e+03 &   10 &           &           & iters  \\ 
 \hdashline 
     &           &    1 &  1.05e-08 &  4.59e-09 &      \\ 
     &           &    2 &  1.05e-08 &  2.99e-16 &      \\ 
     &           &    3 &  1.04e-08 &  2.99e-16 &      \\ 
     &           &    4 &  2.84e-08 &  2.98e-16 &      \\ 
     &           &    5 &  1.05e-08 &  8.11e-16 &      \\ 
     &           &    6 &  1.05e-08 &  2.99e-16 &      \\ 
     &           &    7 &  1.04e-08 &  2.98e-16 &      \\ 
     &           &    8 &  2.84e-08 &  2.98e-16 &      \\ 
     &           &    9 &  1.05e-08 &  8.11e-16 &      \\ 
     &           &   10 &  1.05e-08 &  2.99e-16 &      \\ 
 361 &  3.60e+03 &   10 &           &           & iters  \\ 
 \hdashline 
     &           &    1 &  7.58e-09 &  1.66e-08 &      \\ 
     &           &    2 &  7.57e-09 &  2.16e-16 &      \\ 
     &           &    3 &  7.56e-09 &  2.16e-16 &      \\ 
     &           &    4 &  3.13e-08 &  2.16e-16 &      \\ 
     &           &    5 &  7.59e-09 &  8.93e-16 &      \\ 
     &           &    6 &  7.58e-09 &  2.17e-16 &      \\ 
     &           &    7 &  7.57e-09 &  2.16e-16 &      \\ 
     &           &    8 &  7.56e-09 &  2.16e-16 &      \\ 
     &           &    9 &  3.13e-08 &  2.16e-16 &      \\ 
     &           &   10 &  7.59e-09 &  8.93e-16 &      \\ 
 362 &  3.61e+03 &   10 &           &           & iters  \\ 
 \hdashline 
     &           &    1 &  3.25e-08 &  3.53e-09 &      \\ 
     &           &    2 &  6.42e-09 &  9.27e-16 &      \\ 
     &           &    3 &  6.41e-09 &  1.83e-16 &      \\ 
     &           &    4 &  6.40e-09 &  1.83e-16 &      \\ 
     &           &    5 &  6.39e-09 &  1.83e-16 &      \\ 
     &           &    6 &  6.38e-09 &  1.82e-16 &      \\ 
     &           &    7 &  3.25e-08 &  1.82e-16 &      \\ 
     &           &    8 &  6.42e-09 &  9.27e-16 &      \\ 
     &           &    9 &  6.41e-09 &  1.83e-16 &      \\ 
     &           &   10 &  6.40e-09 &  1.83e-16 &      \\ 
 363 &  3.62e+03 &   10 &           &           & iters  \\ 
 \hdashline 
     &           &    1 &  3.29e-08 &  8.96e-09 &      \\ 
     &           &    2 &  4.48e-08 &  9.40e-16 &      \\ 
     &           &    3 &  3.29e-08 &  1.28e-15 &      \\ 
     &           &    4 &  4.48e-08 &  9.40e-16 &      \\ 
     &           &    5 &  3.30e-08 &  1.28e-15 &      \\ 
     &           &    6 &  3.29e-08 &  9.41e-16 &      \\ 
     &           &    7 &  4.48e-08 &  9.40e-16 &      \\ 
     &           &    8 &  3.29e-08 &  1.28e-15 &      \\ 
     &           &    9 &  4.48e-08 &  9.40e-16 &      \\ 
     &           &   10 &  3.30e-08 &  1.28e-15 &      \\ 
 364 &  3.63e+03 &   10 &           &           & iters  \\ 
 \hdashline 
     &           &    1 &  4.89e-08 &  9.85e-09 &      \\ 
     &           &    2 &  2.89e-08 &  1.39e-15 &      \\ 
     &           &    3 &  2.89e-08 &  8.25e-16 &      \\ 
     &           &    4 &  4.89e-08 &  8.24e-16 &      \\ 
     &           &    5 &  2.89e-08 &  1.39e-15 &      \\ 
     &           &    6 &  2.88e-08 &  8.24e-16 &      \\ 
     &           &    7 &  4.89e-08 &  8.23e-16 &      \\ 
     &           &    8 &  2.89e-08 &  1.40e-15 &      \\ 
     &           &    9 &  2.88e-08 &  8.24e-16 &      \\ 
     &           &   10 &  4.89e-08 &  8.23e-16 &      \\ 
 365 &  3.64e+03 &   10 &           &           & iters  \\ 
 \hdashline 
     &           &    1 &  3.35e-08 &  3.35e-08 &      \\ 
     &           &    2 &  4.43e-08 &  9.55e-16 &      \\ 
     &           &    3 &  3.35e-08 &  1.26e-15 &      \\ 
     &           &    4 &  4.43e-08 &  9.55e-16 &      \\ 
     &           &    5 &  3.35e-08 &  1.26e-15 &      \\ 
     &           &    6 &  4.43e-08 &  9.56e-16 &      \\ 
     &           &    7 &  3.35e-08 &  1.26e-15 &      \\ 
     &           &    8 &  3.35e-08 &  9.56e-16 &      \\ 
     &           &    9 &  4.43e-08 &  9.55e-16 &      \\ 
     &           &   10 &  3.35e-08 &  1.26e-15 &      \\ 
 366 &  3.65e+03 &   10 &           &           & iters  \\ 
 \hdashline 
     &           &    1 &  3.32e-08 &  2.85e-08 &      \\ 
     &           &    2 &  4.45e-08 &  9.49e-16 &      \\ 
     &           &    3 &  3.33e-08 &  1.27e-15 &      \\ 
     &           &    4 &  4.45e-08 &  9.49e-16 &      \\ 
     &           &    5 &  3.33e-08 &  1.27e-15 &      \\ 
     &           &    6 &  3.32e-08 &  9.50e-16 &      \\ 
     &           &    7 &  4.45e-08 &  9.48e-16 &      \\ 
     &           &    8 &  3.32e-08 &  1.27e-15 &      \\ 
     &           &    9 &  4.45e-08 &  9.49e-16 &      \\ 
     &           &   10 &  3.33e-08 &  1.27e-15 &      \\ 
 367 &  3.66e+03 &   10 &           &           & iters  \\ 
 \hdashline 
     &           &    1 &  2.57e-10 &  1.63e-09 &      \\ 
     &           &    2 &  2.56e-10 &  7.32e-18 &      \\ 
     &           &    3 &  2.56e-10 &  7.31e-18 &      \\ 
     &           &    4 &  2.56e-10 &  7.30e-18 &      \\ 
     &           &    5 &  2.55e-10 &  7.29e-18 &      \\ 
     &           &    6 &  2.55e-10 &  7.28e-18 &      \\ 
     &           &    7 &  2.54e-10 &  7.27e-18 &      \\ 
     &           &    8 &  2.54e-10 &  7.26e-18 &      \\ 
     &           &    9 &  2.54e-10 &  7.25e-18 &      \\ 
     &           &   10 &  2.53e-10 &  7.24e-18 &      \\ 
 368 &  3.67e+03 &   10 &           &           & iters  \\ 
 \hdashline 
     &           &    1 &  2.79e-08 &  1.40e-08 &      \\ 
     &           &    2 &  4.99e-08 &  7.96e-16 &      \\ 
     &           &    3 &  2.79e-08 &  1.42e-15 &      \\ 
     &           &    4 &  2.79e-08 &  7.97e-16 &      \\ 
     &           &    5 &  4.99e-08 &  7.95e-16 &      \\ 
     &           &    6 &  2.79e-08 &  1.42e-15 &      \\ 
     &           &    7 &  2.79e-08 &  7.96e-16 &      \\ 
     &           &    8 &  4.99e-08 &  7.95e-16 &      \\ 
     &           &    9 &  2.79e-08 &  1.42e-15 &      \\ 
     &           &   10 &  2.79e-08 &  7.96e-16 &      \\ 
 369 &  3.68e+03 &   10 &           &           & iters  \\ 
 \hdashline 
     &           &    1 &  5.48e-08 &  4.25e-09 &      \\ 
     &           &    2 &  2.30e-08 &  1.56e-15 &      \\ 
     &           &    3 &  5.47e-08 &  6.57e-16 &      \\ 
     &           &    4 &  2.31e-08 &  1.56e-15 &      \\ 
     &           &    5 &  2.30e-08 &  6.58e-16 &      \\ 
     &           &    6 &  2.30e-08 &  6.57e-16 &      \\ 
     &           &    7 &  5.47e-08 &  6.56e-16 &      \\ 
     &           &    8 &  2.30e-08 &  1.56e-15 &      \\ 
     &           &    9 &  2.30e-08 &  6.57e-16 &      \\ 
     &           &   10 &  5.47e-08 &  6.56e-16 &      \\ 
 370 &  3.69e+03 &   10 &           &           & iters  \\ 
 \hdashline 
     &           &    1 &  2.63e-08 &  1.30e-08 &      \\ 
     &           &    2 &  5.14e-08 &  7.50e-16 &      \\ 
     &           &    3 &  2.63e-08 &  1.47e-15 &      \\ 
     &           &    4 &  2.63e-08 &  7.51e-16 &      \\ 
     &           &    5 &  5.14e-08 &  7.50e-16 &      \\ 
     &           &    6 &  2.63e-08 &  1.47e-15 &      \\ 
     &           &    7 &  2.63e-08 &  7.51e-16 &      \\ 
     &           &    8 &  5.14e-08 &  7.50e-16 &      \\ 
     &           &    9 &  2.63e-08 &  1.47e-15 &      \\ 
     &           &   10 &  2.63e-08 &  7.51e-16 &      \\ 
 371 &  3.70e+03 &   10 &           &           & iters  \\ 
 \hdashline 
     &           &    1 &  3.76e-08 &  5.86e-09 &      \\ 
     &           &    2 &  3.76e-08 &  1.07e-15 &      \\ 
     &           &    3 &  3.75e-08 &  1.07e-15 &      \\ 
     &           &    4 &  4.02e-08 &  1.07e-15 &      \\ 
     &           &    5 &  3.75e-08 &  1.15e-15 &      \\ 
     &           &    6 &  4.02e-08 &  1.07e-15 &      \\ 
     &           &    7 &  3.75e-08 &  1.15e-15 &      \\ 
     &           &    8 &  4.02e-08 &  1.07e-15 &      \\ 
     &           &    9 &  3.75e-08 &  1.15e-15 &      \\ 
     &           &   10 &  4.02e-08 &  1.07e-15 &      \\ 
 372 &  3.71e+03 &   10 &           &           & iters  \\ 
 \hdashline 
     &           &    1 &  6.48e-08 &  1.82e-08 &      \\ 
     &           &    2 &  1.29e-08 &  1.85e-15 &      \\ 
     &           &    3 &  1.29e-08 &  3.70e-16 &      \\ 
     &           &    4 &  1.29e-08 &  3.69e-16 &      \\ 
     &           &    5 &  1.29e-08 &  3.68e-16 &      \\ 
     &           &    6 &  1.29e-08 &  3.68e-16 &      \\ 
     &           &    7 &  6.48e-08 &  3.67e-16 &      \\ 
     &           &    8 &  1.29e-08 &  1.85e-15 &      \\ 
     &           &    9 &  1.29e-08 &  3.70e-16 &      \\ 
     &           &   10 &  1.29e-08 &  3.69e-16 &      \\ 
 373 &  3.72e+03 &   10 &           &           & iters  \\ 
 \hdashline 
     &           &    1 &  2.08e-08 &  2.34e-08 &      \\ 
     &           &    2 &  2.08e-08 &  5.94e-16 &      \\ 
     &           &    3 &  2.07e-08 &  5.93e-16 &      \\ 
     &           &    4 &  5.70e-08 &  5.92e-16 &      \\ 
     &           &    5 &  2.08e-08 &  1.63e-15 &      \\ 
     &           &    6 &  2.08e-08 &  5.94e-16 &      \\ 
     &           &    7 &  2.07e-08 &  5.93e-16 &      \\ 
     &           &    8 &  5.70e-08 &  5.92e-16 &      \\ 
     &           &    9 &  2.08e-08 &  1.63e-15 &      \\ 
     &           &   10 &  2.08e-08 &  5.93e-16 &      \\ 
 374 &  3.73e+03 &   10 &           &           & iters  \\ 
 \hdashline 
     &           &    1 &  4.75e-08 &  3.42e-08 &      \\ 
     &           &    2 &  3.03e-08 &  1.36e-15 &      \\ 
     &           &    3 &  4.75e-08 &  8.64e-16 &      \\ 
     &           &    4 &  3.03e-08 &  1.35e-15 &      \\ 
     &           &    5 &  3.02e-08 &  8.64e-16 &      \\ 
     &           &    6 &  4.75e-08 &  8.63e-16 &      \\ 
     &           &    7 &  3.03e-08 &  1.36e-15 &      \\ 
     &           &    8 &  4.75e-08 &  8.64e-16 &      \\ 
     &           &    9 &  3.03e-08 &  1.35e-15 &      \\ 
     &           &   10 &  3.02e-08 &  8.65e-16 &      \\ 
 375 &  3.74e+03 &   10 &           &           & iters  \\ 
 \hdashline 
     &           &    1 &  2.42e-08 &  1.95e-08 &      \\ 
     &           &    2 &  2.42e-08 &  6.92e-16 &      \\ 
     &           &    3 &  5.35e-08 &  6.91e-16 &      \\ 
     &           &    4 &  2.42e-08 &  1.53e-15 &      \\ 
     &           &    5 &  2.42e-08 &  6.92e-16 &      \\ 
     &           &    6 &  2.42e-08 &  6.91e-16 &      \\ 
     &           &    7 &  5.36e-08 &  6.90e-16 &      \\ 
     &           &    8 &  2.42e-08 &  1.53e-15 &      \\ 
     &           &    9 &  2.42e-08 &  6.91e-16 &      \\ 
     &           &   10 &  5.35e-08 &  6.90e-16 &      \\ 
 376 &  3.75e+03 &   10 &           &           & iters  \\ 
 \hdashline 
     &           &    1 &  6.00e-08 &  3.70e-09 &      \\ 
     &           &    2 &  1.78e-08 &  1.71e-15 &      \\ 
     &           &    3 &  1.78e-08 &  5.09e-16 &      \\ 
     &           &    4 &  1.78e-08 &  5.08e-16 &      \\ 
     &           &    5 &  5.99e-08 &  5.07e-16 &      \\ 
     &           &    6 &  1.78e-08 &  1.71e-15 &      \\ 
     &           &    7 &  1.78e-08 &  5.09e-16 &      \\ 
     &           &    8 &  1.78e-08 &  5.08e-16 &      \\ 
     &           &    9 &  1.78e-08 &  5.07e-16 &      \\ 
     &           &   10 &  6.00e-08 &  5.07e-16 &      \\ 
 377 &  3.76e+03 &   10 &           &           & iters  \\ 
 \hdashline 
     &           &    1 &  7.46e-09 &  9.02e-09 &      \\ 
     &           &    2 &  7.02e-08 &  2.13e-16 &      \\ 
     &           &    3 &  4.64e-08 &  2.00e-15 &      \\ 
     &           &    4 &  7.48e-09 &  1.32e-15 &      \\ 
     &           &    5 &  7.47e-09 &  2.14e-16 &      \\ 
     &           &    6 &  7.46e-09 &  2.13e-16 &      \\ 
     &           &    7 &  7.45e-09 &  2.13e-16 &      \\ 
     &           &    8 &  7.02e-08 &  2.13e-16 &      \\ 
     &           &    9 &  4.64e-08 &  2.00e-15 &      \\ 
     &           &   10 &  7.48e-09 &  1.32e-15 &      \\ 
 378 &  3.77e+03 &   10 &           &           & iters  \\ 
 \hdashline 
     &           &    1 &  4.16e-08 &  4.27e-09 &      \\ 
     &           &    2 &  3.61e-08 &  1.19e-15 &      \\ 
     &           &    3 &  4.16e-08 &  1.03e-15 &      \\ 
     &           &    4 &  3.61e-08 &  1.19e-15 &      \\ 
     &           &    5 &  4.16e-08 &  1.03e-15 &      \\ 
     &           &    6 &  3.61e-08 &  1.19e-15 &      \\ 
     &           &    7 &  4.16e-08 &  1.03e-15 &      \\ 
     &           &    8 &  3.62e-08 &  1.19e-15 &      \\ 
     &           &    9 &  4.16e-08 &  1.03e-15 &      \\ 
     &           &   10 &  3.62e-08 &  1.19e-15 &      \\ 
 379 &  3.78e+03 &   10 &           &           & iters  \\ 
 \hdashline 
     &           &    1 &  8.50e-09 &  7.68e-10 &      \\ 
     &           &    2 &  8.49e-09 &  2.43e-16 &      \\ 
     &           &    3 &  8.48e-09 &  2.42e-16 &      \\ 
     &           &    4 &  8.47e-09 &  2.42e-16 &      \\ 
     &           &    5 &  6.92e-08 &  2.42e-16 &      \\ 
     &           &    6 &  4.74e-08 &  1.98e-15 &      \\ 
     &           &    7 &  8.49e-09 &  1.35e-15 &      \\ 
     &           &    8 &  8.47e-09 &  2.42e-16 &      \\ 
     &           &    9 &  8.46e-09 &  2.42e-16 &      \\ 
     &           &   10 &  6.92e-08 &  2.42e-16 &      \\ 
 380 &  3.79e+03 &   10 &           &           & iters  \\ 
 \hdashline 
     &           &    1 &  5.62e-08 &  6.84e-09 &      \\ 
     &           &    2 &  2.16e-08 &  1.60e-15 &      \\ 
     &           &    3 &  2.16e-08 &  6.16e-16 &      \\ 
     &           &    4 &  2.15e-08 &  6.15e-16 &      \\ 
     &           &    5 &  5.62e-08 &  6.14e-16 &      \\ 
     &           &    6 &  2.16e-08 &  1.60e-15 &      \\ 
     &           &    7 &  2.15e-08 &  6.16e-16 &      \\ 
     &           &    8 &  5.62e-08 &  6.15e-16 &      \\ 
     &           &    9 &  2.16e-08 &  1.60e-15 &      \\ 
     &           &   10 &  2.16e-08 &  6.16e-16 &      \\ 
 381 &  3.80e+03 &   10 &           &           & iters  \\ 
 \hdashline 
     &           &    1 &  1.65e-08 &  4.15e-09 &      \\ 
     &           &    2 &  2.23e-08 &  4.72e-16 &      \\ 
     &           &    3 &  1.65e-08 &  6.37e-16 &      \\ 
     &           &    4 &  1.65e-08 &  4.72e-16 &      \\ 
     &           &    5 &  2.24e-08 &  4.71e-16 &      \\ 
     &           &    6 &  1.65e-08 &  6.38e-16 &      \\ 
     &           &    7 &  2.23e-08 &  4.72e-16 &      \\ 
     &           &    8 &  1.65e-08 &  6.38e-16 &      \\ 
     &           &    9 &  2.23e-08 &  4.72e-16 &      \\ 
     &           &   10 &  1.65e-08 &  6.37e-16 &      \\ 
 382 &  3.81e+03 &   10 &           &           & iters  \\ 
 \hdashline 
     &           &    1 &  1.39e-08 &  4.63e-09 &      \\ 
     &           &    2 &  1.39e-08 &  3.96e-16 &      \\ 
     &           &    3 &  6.39e-08 &  3.95e-16 &      \\ 
     &           &    4 &  1.39e-08 &  1.82e-15 &      \\ 
     &           &    5 &  1.39e-08 &  3.97e-16 &      \\ 
     &           &    6 &  1.39e-08 &  3.97e-16 &      \\ 
     &           &    7 &  1.39e-08 &  3.96e-16 &      \\ 
     &           &    8 &  1.38e-08 &  3.96e-16 &      \\ 
     &           &    9 &  6.39e-08 &  3.95e-16 &      \\ 
     &           &   10 &  1.39e-08 &  1.82e-15 &      \\ 
 383 &  3.82e+03 &   10 &           &           & iters  \\ 
 \hdashline 
     &           &    1 &  5.67e-08 &  6.25e-09 &      \\ 
     &           &    2 &  2.10e-08 &  1.62e-15 &      \\ 
     &           &    3 &  2.10e-08 &  6.00e-16 &      \\ 
     &           &    4 &  2.10e-08 &  6.00e-16 &      \\ 
     &           &    5 &  5.67e-08 &  5.99e-16 &      \\ 
     &           &    6 &  2.10e-08 &  1.62e-15 &      \\ 
     &           &    7 &  2.10e-08 &  6.00e-16 &      \\ 
     &           &    8 &  2.10e-08 &  5.99e-16 &      \\ 
     &           &    9 &  5.67e-08 &  5.98e-16 &      \\ 
     &           &   10 &  2.10e-08 &  1.62e-15 &      \\ 
 384 &  3.83e+03 &   10 &           &           & iters  \\ 
 \hdashline 
     &           &    1 &  1.99e-10 &  2.00e-09 &      \\ 
 385 &  3.84e+03 &    1 &           &           & forces  \\ 
 \hdashline 
     &           &    1 &  6.42e-09 &  4.89e-10 &      \\ 
     &           &    2 &  6.42e-09 &  1.83e-16 &      \\ 
     &           &    3 &  6.41e-09 &  1.83e-16 &      \\ 
     &           &    4 &  3.24e-08 &  1.83e-16 &      \\ 
     &           &    5 &  6.44e-09 &  9.26e-16 &      \\ 
     &           &    6 &  6.43e-09 &  1.84e-16 &      \\ 
     &           &    7 &  6.43e-09 &  1.84e-16 &      \\ 
     &           &    8 &  6.42e-09 &  1.83e-16 &      \\ 
     &           &    9 &  6.41e-09 &  1.83e-16 &      \\ 
     &           &   10 &  3.24e-08 &  1.83e-16 &      \\ 
 386 &  3.85e+03 &   10 &           &           & iters  \\ 
 \hdashline 
     &           &    1 &  2.15e-08 &  3.57e-10 &      \\ 
     &           &    2 &  1.74e-08 &  6.14e-16 &      \\ 
     &           &    3 &  2.15e-08 &  4.96e-16 &      \\ 
     &           &    4 &  1.74e-08 &  6.13e-16 &      \\ 
     &           &    5 &  1.74e-08 &  4.96e-16 &      \\ 
     &           &    6 &  2.15e-08 &  4.95e-16 &      \\ 
     &           &    7 &  1.74e-08 &  6.14e-16 &      \\ 
     &           &    8 &  2.15e-08 &  4.96e-16 &      \\ 
     &           &    9 &  1.74e-08 &  6.14e-16 &      \\ 
     &           &   10 &  2.15e-08 &  4.96e-16 &      \\ 
 387 &  3.86e+03 &   10 &           &           & iters  \\ 
 \hdashline 
     &           &    1 &  1.18e-08 &  9.27e-10 &      \\ 
     &           &    2 &  1.18e-08 &  3.38e-16 &      \\ 
     &           &    3 &  2.70e-08 &  3.37e-16 &      \\ 
     &           &    4 &  1.18e-08 &  7.72e-16 &      \\ 
     &           &    5 &  1.18e-08 &  3.38e-16 &      \\ 
     &           &    6 &  2.70e-08 &  3.37e-16 &      \\ 
     &           &    7 &  1.18e-08 &  7.72e-16 &      \\ 
     &           &    8 &  1.18e-08 &  3.38e-16 &      \\ 
     &           &    9 &  1.18e-08 &  3.37e-16 &      \\ 
     &           &   10 &  2.71e-08 &  3.37e-16 &      \\ 
 388 &  3.87e+03 &   10 &           &           & iters  \\ 
 \hdashline 
     &           &    1 &  2.92e-08 &  5.07e-09 &      \\ 
     &           &    2 &  4.86e-08 &  8.32e-16 &      \\ 
     &           &    3 &  2.92e-08 &  1.39e-15 &      \\ 
     &           &    4 &  2.91e-08 &  8.33e-16 &      \\ 
     &           &    5 &  4.86e-08 &  8.32e-16 &      \\ 
     &           &    6 &  2.92e-08 &  1.39e-15 &      \\ 
     &           &    7 &  4.86e-08 &  8.33e-16 &      \\ 
     &           &    8 &  2.92e-08 &  1.39e-15 &      \\ 
     &           &    9 &  2.92e-08 &  8.33e-16 &      \\ 
     &           &   10 &  4.86e-08 &  8.32e-16 &      \\ 
 389 &  3.88e+03 &   10 &           &           & iters  \\ 
 \hdashline 
     &           &    1 &  1.37e-08 &  4.25e-09 &      \\ 
     &           &    2 &  6.40e-08 &  3.90e-16 &      \\ 
     &           &    3 &  1.37e-08 &  1.83e-15 &      \\ 
     &           &    4 &  1.37e-08 &  3.92e-16 &      \\ 
     &           &    5 &  1.37e-08 &  3.92e-16 &      \\ 
     &           &    6 &  1.37e-08 &  3.91e-16 &      \\ 
     &           &    7 &  1.37e-08 &  3.90e-16 &      \\ 
     &           &    8 &  6.40e-08 &  3.90e-16 &      \\ 
     &           &    9 &  1.37e-08 &  1.83e-15 &      \\ 
     &           &   10 &  1.37e-08 &  3.92e-16 &      \\ 
 390 &  3.89e+03 &   10 &           &           & iters  \\ 
 \hdashline 
     &           &    1 &  1.40e-08 &  2.87e-09 &      \\ 
     &           &    2 &  2.49e-08 &  3.99e-16 &      \\ 
     &           &    3 &  1.40e-08 &  7.10e-16 &      \\ 
     &           &    4 &  2.49e-08 &  3.99e-16 &      \\ 
     &           &    5 &  1.40e-08 &  7.10e-16 &      \\ 
     &           &    6 &  1.40e-08 &  4.00e-16 &      \\ 
     &           &    7 &  2.49e-08 &  3.99e-16 &      \\ 
     &           &    8 &  1.40e-08 &  7.10e-16 &      \\ 
     &           &    9 &  1.40e-08 &  4.00e-16 &      \\ 
     &           &   10 &  2.49e-08 &  3.99e-16 &      \\ 
 391 &  3.90e+03 &   10 &           &           & iters  \\ 
 \hdashline 
     &           &    1 &  4.58e-11 &  4.96e-09 &      \\ 
 392 &  3.91e+03 &    1 &           &           & forces  \\ 
 \hdashline 
     &           &    1 &  3.41e-08 &  8.49e-10 &      \\ 
     &           &    2 &  4.37e-08 &  9.72e-16 &      \\ 
     &           &    3 &  3.41e-08 &  1.25e-15 &      \\ 
     &           &    4 &  4.37e-08 &  9.73e-16 &      \\ 
     &           &    5 &  3.41e-08 &  1.25e-15 &      \\ 
     &           &    6 &  4.36e-08 &  9.73e-16 &      \\ 
     &           &    7 &  3.41e-08 &  1.25e-15 &      \\ 
     &           &    8 &  3.41e-08 &  9.73e-16 &      \\ 
     &           &    9 &  4.37e-08 &  9.72e-16 &      \\ 
     &           &   10 &  3.41e-08 &  1.25e-15 &      \\ 
 393 &  3.92e+03 &   10 &           &           & iters  \\ 
 \hdashline 
     &           &    1 &  2.71e-08 &  6.51e-09 &      \\ 
     &           &    2 &  1.18e-08 &  7.72e-16 &      \\ 
     &           &    3 &  2.70e-08 &  3.37e-16 &      \\ 
     &           &    4 &  1.18e-08 &  7.72e-16 &      \\ 
     &           &    5 &  1.18e-08 &  3.38e-16 &      \\ 
     &           &    6 &  2.70e-08 &  3.37e-16 &      \\ 
     &           &    7 &  1.18e-08 &  7.72e-16 &      \\ 
     &           &    8 &  1.18e-08 &  3.38e-16 &      \\ 
     &           &    9 &  2.70e-08 &  3.38e-16 &      \\ 
     &           &   10 &  1.19e-08 &  7.72e-16 &      \\ 
 394 &  3.93e+03 &   10 &           &           & iters  \\ 
 \hdashline 
     &           &    1 &  9.77e-09 &  7.63e-09 &      \\ 
     &           &    2 &  9.76e-09 &  2.79e-16 &      \\ 
     &           &    3 &  9.74e-09 &  2.78e-16 &      \\ 
     &           &    4 &  9.73e-09 &  2.78e-16 &      \\ 
     &           &    5 &  6.80e-08 &  2.78e-16 &      \\ 
     &           &    6 &  9.81e-09 &  1.94e-15 &      \\ 
     &           &    7 &  9.80e-09 &  2.80e-16 &      \\ 
     &           &    8 &  9.78e-09 &  2.80e-16 &      \\ 
     &           &    9 &  9.77e-09 &  2.79e-16 &      \\ 
     &           &   10 &  9.76e-09 &  2.79e-16 &      \\ 
 395 &  3.94e+03 &   10 &           &           & iters  \\ 
 \hdashline 
     &           &    1 &  4.77e-08 &  7.61e-09 &      \\ 
     &           &    2 &  3.00e-08 &  1.36e-15 &      \\ 
     &           &    3 &  3.00e-08 &  8.57e-16 &      \\ 
     &           &    4 &  4.77e-08 &  8.56e-16 &      \\ 
     &           &    5 &  3.00e-08 &  1.36e-15 &      \\ 
     &           &    6 &  3.00e-08 &  8.57e-16 &      \\ 
     &           &    7 &  4.78e-08 &  8.56e-16 &      \\ 
     &           &    8 &  3.00e-08 &  1.36e-15 &      \\ 
     &           &    9 &  4.77e-08 &  8.56e-16 &      \\ 
     &           &   10 &  3.00e-08 &  1.36e-15 &      \\ 
 396 &  3.95e+03 &   10 &           &           & iters  \\ 
 \hdashline 
     &           &    1 &  3.77e-09 &  7.37e-09 &      \\ 
     &           &    2 &  3.76e-09 &  1.08e-16 &      \\ 
     &           &    3 &  3.76e-09 &  1.07e-16 &      \\ 
     &           &    4 &  3.75e-09 &  1.07e-16 &      \\ 
     &           &    5 &  3.75e-09 &  1.07e-16 &      \\ 
     &           &    6 &  3.74e-09 &  1.07e-16 &      \\ 
     &           &    7 &  3.74e-09 &  1.07e-16 &      \\ 
     &           &    8 &  3.73e-09 &  1.07e-16 &      \\ 
     &           &    9 &  3.72e-09 &  1.06e-16 &      \\ 
     &           &   10 &  7.40e-08 &  1.06e-16 &      \\ 
 397 &  3.96e+03 &   10 &           &           & iters  \\ 
 \hdashline 
     &           &    1 &  7.36e-08 &  1.16e-09 &      \\ 
     &           &    2 &  4.30e-08 &  2.10e-15 &      \\ 
     &           &    3 &  4.08e-09 &  1.23e-15 &      \\ 
     &           &    4 &  4.08e-09 &  1.17e-16 &      \\ 
     &           &    5 &  4.07e-09 &  1.16e-16 &      \\ 
     &           &    6 &  4.07e-09 &  1.16e-16 &      \\ 
     &           &    7 &  4.06e-09 &  1.16e-16 &      \\ 
     &           &    8 &  4.06e-09 &  1.16e-16 &      \\ 
     &           &    9 &  4.05e-09 &  1.16e-16 &      \\ 
     &           &   10 &  4.04e-09 &  1.16e-16 &      \\ 
 398 &  3.97e+03 &   10 &           &           & iters  \\ 
 \hdashline 
     &           &    1 &  2.14e-08 &  4.02e-09 &      \\ 
     &           &    2 &  2.14e-08 &  6.11e-16 &      \\ 
     &           &    3 &  2.14e-08 &  6.11e-16 &      \\ 
     &           &    4 &  1.75e-08 &  6.10e-16 &      \\ 
     &           &    5 &  2.14e-08 &  5.00e-16 &      \\ 
     &           &    6 &  1.75e-08 &  6.09e-16 &      \\ 
     &           &    7 &  1.75e-08 &  5.00e-16 &      \\ 
     &           &    8 &  2.14e-08 &  4.99e-16 &      \\ 
     &           &    9 &  1.75e-08 &  6.10e-16 &      \\ 
     &           &   10 &  2.14e-08 &  4.99e-16 &      \\ 
 399 &  3.98e+03 &   10 &           &           & iters  \\ 
 \hdashline 
     &           &    1 &  6.77e-09 &  3.10e-09 &      \\ 
     &           &    2 &  6.76e-09 &  1.93e-16 &      \\ 
     &           &    3 &  6.76e-09 &  1.93e-16 &      \\ 
     &           &    4 &  6.75e-09 &  1.93e-16 &      \\ 
     &           &    5 &  6.74e-09 &  1.93e-16 &      \\ 
     &           &    6 &  6.73e-09 &  1.92e-16 &      \\ 
     &           &    7 &  6.72e-09 &  1.92e-16 &      \\ 
     &           &    8 &  7.10e-08 &  1.92e-16 &      \\ 
     &           &    9 &  4.57e-08 &  2.03e-15 &      \\ 
     &           &   10 &  6.74e-09 &  1.30e-15 &      \\ 
 400 &  3.99e+03 &   10 &           &           & iters  \\ 
 \hdashline 
     &           &    1 &  1.02e-10 &  6.48e-09 &      \\ 
 401 &  4.00e+03 &    1 &           &           & forces  \\ 
 \hdashline 
     &           &    1 &  2.21e-09 &  8.75e-09 &      \\ 
     &           &    2 &  2.21e-09 &  6.31e-17 &      \\ 
     &           &    3 &  2.21e-09 &  6.31e-17 &      \\ 
     &           &    4 &  2.20e-09 &  6.30e-17 &      \\ 
     &           &    5 &  2.20e-09 &  6.29e-17 &      \\ 
     &           &    6 &  2.20e-09 &  6.28e-17 &      \\ 
     &           &    7 &  2.19e-09 &  6.27e-17 &      \\ 
     &           &    8 &  2.19e-09 &  6.26e-17 &      \\ 
     &           &    9 &  2.19e-09 &  6.25e-17 &      \\ 
     &           &   10 &  2.18e-09 &  6.24e-17 &      \\ 
 402 &  4.01e+03 &   10 &           &           & iters  \\ 
 \hdashline 
     &           &    1 &  2.09e-08 &  5.82e-09 &      \\ 
     &           &    2 &  5.68e-08 &  5.96e-16 &      \\ 
     &           &    3 &  2.09e-08 &  1.62e-15 &      \\ 
     &           &    4 &  2.09e-08 &  5.98e-16 &      \\ 
     &           &    5 &  2.09e-08 &  5.97e-16 &      \\ 
     &           &    6 &  5.68e-08 &  5.96e-16 &      \\ 
     &           &    7 &  2.09e-08 &  1.62e-15 &      \\ 
     &           &    8 &  2.09e-08 &  5.98e-16 &      \\ 
     &           &    9 &  2.09e-08 &  5.97e-16 &      \\ 
     &           &   10 &  5.68e-08 &  5.96e-16 &      \\ 
 403 &  4.02e+03 &   10 &           &           & iters  \\ 
 \hdashline 
     &           &    1 &  6.13e-09 &  7.94e-09 &      \\ 
     &           &    2 &  6.12e-09 &  1.75e-16 &      \\ 
     &           &    3 &  6.11e-09 &  1.75e-16 &      \\ 
     &           &    4 &  6.10e-09 &  1.74e-16 &      \\ 
     &           &    5 &  6.09e-09 &  1.74e-16 &      \\ 
     &           &    6 &  6.08e-09 &  1.74e-16 &      \\ 
     &           &    7 &  6.07e-09 &  1.74e-16 &      \\ 
     &           &    8 &  6.06e-09 &  1.73e-16 &      \\ 
     &           &    9 &  7.16e-08 &  1.73e-16 &      \\ 
     &           &   10 &  8.38e-08 &  2.04e-15 &      \\ 
 404 &  4.03e+03 &   10 &           &           & iters  \\ 
 \hdashline 
     &           &    1 &  6.70e-08 &  6.71e-09 &      \\ 
     &           &    2 &  1.07e-08 &  1.91e-15 &      \\ 
     &           &    3 &  1.07e-08 &  3.06e-16 &      \\ 
     &           &    4 &  1.07e-08 &  3.06e-16 &      \\ 
     &           &    5 &  1.07e-08 &  3.06e-16 &      \\ 
     &           &    6 &  1.07e-08 &  3.05e-16 &      \\ 
     &           &    7 &  1.07e-08 &  3.05e-16 &      \\ 
     &           &    8 &  6.70e-08 &  3.04e-16 &      \\ 
     &           &    9 &  1.07e-08 &  1.91e-15 &      \\ 
     &           &   10 &  1.07e-08 &  3.07e-16 &      \\ 
 405 &  4.04e+03 &   10 &           &           & iters  \\ 
 \hdashline 
     &           &    1 &  2.41e-08 &  1.83e-09 &      \\ 
     &           &    2 &  5.36e-08 &  6.88e-16 &      \\ 
     &           &    3 &  2.41e-08 &  1.53e-15 &      \\ 
     &           &    4 &  2.41e-08 &  6.89e-16 &      \\ 
     &           &    5 &  5.36e-08 &  6.88e-16 &      \\ 
     &           &    6 &  2.42e-08 &  1.53e-15 &      \\ 
     &           &    7 &  2.41e-08 &  6.89e-16 &      \\ 
     &           &    8 &  2.41e-08 &  6.88e-16 &      \\ 
     &           &    9 &  5.36e-08 &  6.87e-16 &      \\ 
     &           &   10 &  2.41e-08 &  1.53e-15 &      \\ 
 406 &  4.05e+03 &   10 &           &           & iters  \\ 
 \hdashline 
     &           &    1 &  2.63e-08 &  3.60e-09 &      \\ 
     &           &    2 &  1.26e-08 &  7.51e-16 &      \\ 
     &           &    3 &  1.25e-08 &  3.59e-16 &      \\ 
     &           &    4 &  2.63e-08 &  3.58e-16 &      \\ 
     &           &    5 &  1.26e-08 &  7.51e-16 &      \\ 
     &           &    6 &  1.26e-08 &  3.59e-16 &      \\ 
     &           &    7 &  2.63e-08 &  3.58e-16 &      \\ 
     &           &    8 &  1.26e-08 &  7.51e-16 &      \\ 
     &           &    9 &  1.26e-08 &  3.59e-16 &      \\ 
     &           &   10 &  2.63e-08 &  3.58e-16 &      \\ 
 407 &  4.06e+03 &   10 &           &           & iters  \\ 
 \hdashline 
     &           &    1 &  3.53e-08 &  5.97e-09 &      \\ 
     &           &    2 &  4.25e-08 &  1.01e-15 &      \\ 
     &           &    3 &  3.53e-08 &  1.21e-15 &      \\ 
     &           &    4 &  4.25e-08 &  1.01e-15 &      \\ 
     &           &    5 &  3.53e-08 &  1.21e-15 &      \\ 
     &           &    6 &  3.52e-08 &  1.01e-15 &      \\ 
     &           &    7 &  4.25e-08 &  1.01e-15 &      \\ 
     &           &    8 &  3.53e-08 &  1.21e-15 &      \\ 
     &           &    9 &  4.25e-08 &  1.01e-15 &      \\ 
     &           &   10 &  3.53e-08 &  1.21e-15 &      \\ 
 408 &  4.07e+03 &   10 &           &           & iters  \\ 
 \hdashline 
     &           &    1 &  3.06e-08 &  3.65e-09 &      \\ 
     &           &    2 &  3.05e-08 &  8.72e-16 &      \\ 
     &           &    3 &  4.72e-08 &  8.71e-16 &      \\ 
     &           &    4 &  3.05e-08 &  1.35e-15 &      \\ 
     &           &    5 &  4.72e-08 &  8.71e-16 &      \\ 
     &           &    6 &  3.06e-08 &  1.35e-15 &      \\ 
     &           &    7 &  3.05e-08 &  8.72e-16 &      \\ 
     &           &    8 &  4.72e-08 &  8.71e-16 &      \\ 
     &           &    9 &  3.05e-08 &  1.35e-15 &      \\ 
     &           &   10 &  3.05e-08 &  8.72e-16 &      \\ 
 409 &  4.08e+03 &   10 &           &           & iters  \\ 
 \hdashline 
     &           &    1 &  6.47e-08 &  8.84e-10 &      \\ 
     &           &    2 &  1.31e-08 &  1.85e-15 &      \\ 
     &           &    3 &  1.31e-08 &  3.74e-16 &      \\ 
     &           &    4 &  1.31e-08 &  3.73e-16 &      \\ 
     &           &    5 &  1.30e-08 &  3.73e-16 &      \\ 
     &           &    6 &  6.47e-08 &  3.72e-16 &      \\ 
     &           &    7 &  1.31e-08 &  1.85e-15 &      \\ 
     &           &    8 &  1.31e-08 &  3.74e-16 &      \\ 
     &           &    9 &  1.31e-08 &  3.74e-16 &      \\ 
     &           &   10 &  1.31e-08 &  3.73e-16 &      \\ 
 410 &  4.09e+03 &   10 &           &           & iters  \\ 
 \hdashline 
     &           &    1 &  2.18e-08 &  3.17e-10 &      \\ 
     &           &    2 &  2.18e-08 &  6.23e-16 &      \\ 
     &           &    3 &  2.18e-08 &  6.22e-16 &      \\ 
     &           &    4 &  5.60e-08 &  6.21e-16 &      \\ 
     &           &    5 &  2.18e-08 &  1.60e-15 &      \\ 
     &           &    6 &  2.18e-08 &  6.22e-16 &      \\ 
     &           &    7 &  5.59e-08 &  6.21e-16 &      \\ 
     &           &    8 &  2.18e-08 &  1.60e-15 &      \\ 
     &           &    9 &  2.18e-08 &  6.23e-16 &      \\ 
     &           &   10 &  2.18e-08 &  6.22e-16 &      \\ 
 411 &  4.10e+03 &   10 &           &           & iters  \\ 
 \hdashline 
     &           &    1 &  1.33e-08 &  2.65e-09 &      \\ 
     &           &    2 &  1.33e-08 &  3.79e-16 &      \\ 
     &           &    3 &  1.32e-08 &  3.78e-16 &      \\ 
     &           &    4 &  1.32e-08 &  3.78e-16 &      \\ 
     &           &    5 &  2.56e-08 &  3.77e-16 &      \\ 
     &           &    6 &  1.32e-08 &  7.32e-16 &      \\ 
     &           &    7 &  1.32e-08 &  3.78e-16 &      \\ 
     &           &    8 &  2.56e-08 &  3.77e-16 &      \\ 
     &           &    9 &  1.32e-08 &  7.32e-16 &      \\ 
     &           &   10 &  1.32e-08 &  3.78e-16 &      \\ 
 412 &  4.11e+03 &   10 &           &           & iters  \\ 
 \hdashline 
     &           &    1 &  1.36e-09 &  3.94e-09 &      \\ 
     &           &    2 &  1.35e-09 &  3.87e-17 &      \\ 
     &           &    3 &  1.35e-09 &  3.86e-17 &      \\ 
     &           &    4 &  1.35e-09 &  3.86e-17 &      \\ 
     &           &    5 &  1.35e-09 &  3.85e-17 &      \\ 
     &           &    6 &  1.35e-09 &  3.85e-17 &      \\ 
     &           &    7 &  1.34e-09 &  3.84e-17 &      \\ 
     &           &    8 &  1.34e-09 &  3.84e-17 &      \\ 
     &           &    9 &  1.34e-09 &  3.83e-17 &      \\ 
     &           &   10 &  1.34e-09 &  3.83e-17 &      \\ 
 413 &  4.12e+03 &   10 &           &           & iters  \\ 
 \hdashline 
     &           &    1 &  1.16e-08 &  1.78e-09 &      \\ 
     &           &    2 &  1.16e-08 &  3.31e-16 &      \\ 
     &           &    3 &  6.61e-08 &  3.31e-16 &      \\ 
     &           &    4 &  1.17e-08 &  1.89e-15 &      \\ 
     &           &    5 &  1.16e-08 &  3.33e-16 &      \\ 
     &           &    6 &  1.16e-08 &  3.32e-16 &      \\ 
     &           &    7 &  1.16e-08 &  3.32e-16 &      \\ 
     &           &    8 &  1.16e-08 &  3.32e-16 &      \\ 
     &           &    9 &  1.16e-08 &  3.31e-16 &      \\ 
     &           &   10 &  6.61e-08 &  3.31e-16 &      \\ 
 414 &  4.13e+03 &   10 &           &           & iters  \\ 
 \hdashline 
     &           &    1 &  4.85e-08 &  5.84e-10 &      \\ 
     &           &    2 &  2.93e-08 &  1.38e-15 &      \\ 
     &           &    3 &  4.84e-08 &  8.36e-16 &      \\ 
     &           &    4 &  2.93e-08 &  1.38e-15 &      \\ 
     &           &    5 &  2.93e-08 &  8.37e-16 &      \\ 
     &           &    6 &  4.84e-08 &  8.36e-16 &      \\ 
     &           &    7 &  2.93e-08 &  1.38e-15 &      \\ 
     &           &    8 &  2.93e-08 &  8.37e-16 &      \\ 
     &           &    9 &  4.85e-08 &  8.35e-16 &      \\ 
     &           &   10 &  2.93e-08 &  1.38e-15 &      \\ 
 415 &  4.14e+03 &   10 &           &           & iters  \\ 
 \hdashline 
     &           &    1 &  1.01e-08 &  4.75e-09 &      \\ 
     &           &    2 &  1.01e-08 &  2.88e-16 &      \\ 
     &           &    3 &  1.01e-08 &  2.88e-16 &      \\ 
     &           &    4 &  1.01e-08 &  2.87e-16 &      \\ 
     &           &    5 &  2.88e-08 &  2.87e-16 &      \\ 
     &           &    6 &  1.01e-08 &  8.22e-16 &      \\ 
     &           &    7 &  1.01e-08 &  2.88e-16 &      \\ 
     &           &    8 &  1.01e-08 &  2.87e-16 &      \\ 
     &           &    9 &  2.88e-08 &  2.87e-16 &      \\ 
     &           &   10 &  1.01e-08 &  8.22e-16 &      \\ 
 416 &  4.15e+03 &   10 &           &           & iters  \\ 
 \hdashline 
     &           &    1 &  2.23e-09 &  7.80e-09 &      \\ 
     &           &    2 &  2.23e-09 &  6.38e-17 &      \\ 
     &           &    3 &  2.23e-09 &  6.37e-17 &      \\ 
     &           &    4 &  2.22e-09 &  6.36e-17 &      \\ 
     &           &    5 &  2.22e-09 &  6.35e-17 &      \\ 
     &           &    6 &  3.66e-08 &  6.34e-17 &      \\ 
     &           &    7 &  2.27e-09 &  1.05e-15 &      \\ 
     &           &    8 &  2.27e-09 &  6.48e-17 &      \\ 
     &           &    9 &  2.26e-09 &  6.47e-17 &      \\ 
     &           &   10 &  2.26e-09 &  6.46e-17 &      \\ 
 417 &  4.16e+03 &   10 &           &           & iters  \\ 
 \hdashline 
     &           &    1 &  4.28e-08 &  2.49e-08 &      \\ 
     &           &    2 &  3.50e-08 &  1.22e-15 &      \\ 
     &           &    3 &  4.28e-08 &  9.98e-16 &      \\ 
     &           &    4 &  3.50e-08 &  1.22e-15 &      \\ 
     &           &    5 &  4.28e-08 &  9.98e-16 &      \\ 
     &           &    6 &  3.50e-08 &  1.22e-15 &      \\ 
     &           &    7 &  4.28e-08 &  9.98e-16 &      \\ 
     &           &    8 &  3.50e-08 &  1.22e-15 &      \\ 
     &           &    9 &  4.28e-08 &  9.98e-16 &      \\ 
     &           &   10 &  3.50e-08 &  1.22e-15 &      \\ 
 418 &  4.17e+03 &   10 &           &           & iters  \\ 
 \hdashline 
     &           &    1 &  6.50e-09 &  2.13e-08 &      \\ 
     &           &    2 &  6.49e-09 &  1.86e-16 &      \\ 
     &           &    3 &  6.49e-09 &  1.85e-16 &      \\ 
     &           &    4 &  6.48e-09 &  1.85e-16 &      \\ 
     &           &    5 &  6.47e-09 &  1.85e-16 &      \\ 
     &           &    6 &  7.12e-08 &  1.85e-16 &      \\ 
     &           &    7 &  6.56e-09 &  2.03e-15 &      \\ 
     &           &    8 &  6.55e-09 &  1.87e-16 &      \\ 
     &           &    9 &  6.54e-09 &  1.87e-16 &      \\ 
     &           &   10 &  6.53e-09 &  1.87e-16 &      \\ 
 419 &  4.18e+03 &   10 &           &           & iters  \\ 
 \hdashline 
     &           &    1 &  3.36e-08 &  1.54e-08 &      \\ 
     &           &    2 &  4.41e-08 &  9.60e-16 &      \\ 
     &           &    3 &  3.37e-08 &  1.26e-15 &      \\ 
     &           &    4 &  3.36e-08 &  9.61e-16 &      \\ 
     &           &    5 &  4.41e-08 &  9.59e-16 &      \\ 
     &           &    6 &  3.36e-08 &  1.26e-15 &      \\ 
     &           &    7 &  4.41e-08 &  9.60e-16 &      \\ 
     &           &    8 &  3.36e-08 &  1.26e-15 &      \\ 
     &           &    9 &  4.41e-08 &  9.60e-16 &      \\ 
     &           &   10 &  3.37e-08 &  1.26e-15 &      \\ 
 420 &  4.19e+03 &   10 &           &           & iters  \\ 
 \hdashline 
     &           &    1 &  3.33e-08 &  1.45e-08 &      \\ 
     &           &    2 &  4.44e-08 &  9.51e-16 &      \\ 
     &           &    3 &  3.33e-08 &  1.27e-15 &      \\ 
     &           &    4 &  4.44e-08 &  9.52e-16 &      \\ 
     &           &    5 &  3.34e-08 &  1.27e-15 &      \\ 
     &           &    6 &  4.44e-08 &  9.52e-16 &      \\ 
     &           &    7 &  3.34e-08 &  1.27e-15 &      \\ 
     &           &    8 &  3.33e-08 &  9.52e-16 &      \\ 
     &           &    9 &  4.44e-08 &  9.51e-16 &      \\ 
     &           &   10 &  3.33e-08 &  1.27e-15 &      \\ 
 421 &  4.20e+03 &   10 &           &           & iters  \\ 
 \hdashline 
     &           &    1 &  1.22e-08 &  1.43e-09 &      \\ 
     &           &    2 &  1.21e-08 &  3.47e-16 &      \\ 
     &           &    3 &  1.21e-08 &  3.46e-16 &      \\ 
     &           &    4 &  1.21e-08 &  3.46e-16 &      \\ 
     &           &    5 &  1.21e-08 &  3.45e-16 &      \\ 
     &           &    6 &  1.21e-08 &  3.45e-16 &      \\ 
     &           &    7 &  1.20e-08 &  3.44e-16 &      \\ 
     &           &    8 &  6.57e-08 &  3.44e-16 &      \\ 
     &           &    9 &  1.21e-08 &  1.87e-15 &      \\ 
     &           &   10 &  1.21e-08 &  3.46e-16 &      \\ 
 422 &  4.21e+03 &   10 &           &           & iters  \\ 
 \hdashline 
     &           &    1 &  4.44e-08 &  9.15e-09 &      \\ 
     &           &    2 &  3.34e-08 &  1.27e-15 &      \\ 
     &           &    3 &  3.33e-08 &  9.52e-16 &      \\ 
     &           &    4 &  4.44e-08 &  9.50e-16 &      \\ 
     &           &    5 &  3.33e-08 &  1.27e-15 &      \\ 
     &           &    6 &  4.44e-08 &  9.51e-16 &      \\ 
     &           &    7 &  3.33e-08 &  1.27e-15 &      \\ 
     &           &    8 &  4.44e-08 &  9.51e-16 &      \\ 
     &           &    9 &  3.34e-08 &  1.27e-15 &      \\ 
     &           &   10 &  3.33e-08 &  9.52e-16 &      \\ 
 423 &  4.22e+03 &   10 &           &           & iters  \\ 
 \hdashline 
     &           &    1 &  1.27e-08 &  4.31e-09 &      \\ 
     &           &    2 &  1.27e-08 &  3.62e-16 &      \\ 
     &           &    3 &  6.50e-08 &  3.62e-16 &      \\ 
     &           &    4 &  1.27e-08 &  1.86e-15 &      \\ 
     &           &    5 &  1.27e-08 &  3.64e-16 &      \\ 
     &           &    6 &  1.27e-08 &  3.63e-16 &      \\ 
     &           &    7 &  1.27e-08 &  3.63e-16 &      \\ 
     &           &    8 &  1.27e-08 &  3.62e-16 &      \\ 
     &           &    9 &  6.50e-08 &  3.62e-16 &      \\ 
     &           &   10 &  1.27e-08 &  1.86e-15 &      \\ 
 424 &  4.23e+03 &   10 &           &           & iters  \\ 
 \hdashline 
     &           &    1 &  2.84e-08 &  7.46e-09 &      \\ 
     &           &    2 &  2.83e-08 &  8.10e-16 &      \\ 
     &           &    3 &  4.94e-08 &  8.09e-16 &      \\ 
     &           &    4 &  2.84e-08 &  1.41e-15 &      \\ 
     &           &    5 &  2.83e-08 &  8.10e-16 &      \\ 
     &           &    6 &  4.94e-08 &  8.09e-16 &      \\ 
     &           &    7 &  2.84e-08 &  1.41e-15 &      \\ 
     &           &    8 &  4.94e-08 &  8.10e-16 &      \\ 
     &           &    9 &  2.84e-08 &  1.41e-15 &      \\ 
     &           &   10 &  2.84e-08 &  8.10e-16 &      \\ 
 425 &  4.24e+03 &   10 &           &           & iters  \\ 
 \hdashline 
     &           &    1 &  3.97e-08 &  1.03e-08 &      \\ 
     &           &    2 &  3.81e-08 &  1.13e-15 &      \\ 
     &           &    3 &  3.97e-08 &  1.09e-15 &      \\ 
     &           &    4 &  3.81e-08 &  1.13e-15 &      \\ 
     &           &    5 &  3.97e-08 &  1.09e-15 &      \\ 
     &           &    6 &  3.81e-08 &  1.13e-15 &      \\ 
     &           &    7 &  3.97e-08 &  1.09e-15 &      \\ 
     &           &    8 &  3.81e-08 &  1.13e-15 &      \\ 
     &           &    9 &  3.97e-08 &  1.09e-15 &      \\ 
     &           &   10 &  3.81e-08 &  1.13e-15 &      \\ 
 426 &  4.25e+03 &   10 &           &           & iters  \\ 
 \hdashline 
     &           &    1 &  2.65e-08 &  8.87e-09 &      \\ 
     &           &    2 &  2.65e-08 &  7.58e-16 &      \\ 
     &           &    3 &  5.12e-08 &  7.57e-16 &      \\ 
     &           &    4 &  2.65e-08 &  1.46e-15 &      \\ 
     &           &    5 &  2.65e-08 &  7.58e-16 &      \\ 
     &           &    6 &  5.12e-08 &  7.56e-16 &      \\ 
     &           &    7 &  2.65e-08 &  1.46e-15 &      \\ 
     &           &    8 &  2.65e-08 &  7.57e-16 &      \\ 
     &           &    9 &  5.12e-08 &  7.56e-16 &      \\ 
     &           &   10 &  2.65e-08 &  1.46e-15 &      \\ 
 427 &  4.26e+03 &   10 &           &           & iters  \\ 
 \hdashline 
     &           &    1 &  3.55e-08 &  1.24e-09 &      \\ 
     &           &    2 &  4.22e-08 &  1.01e-15 &      \\ 
     &           &    3 &  3.56e-08 &  1.20e-15 &      \\ 
     &           &    4 &  3.55e-08 &  1.01e-15 &      \\ 
     &           &    5 &  4.22e-08 &  1.01e-15 &      \\ 
     &           &    6 &  3.55e-08 &  1.21e-15 &      \\ 
     &           &    7 &  4.22e-08 &  1.01e-15 &      \\ 
     &           &    8 &  3.55e-08 &  1.21e-15 &      \\ 
     &           &    9 &  4.22e-08 &  1.01e-15 &      \\ 
     &           &   10 &  3.55e-08 &  1.20e-15 &      \\ 
 428 &  4.27e+03 &   10 &           &           & iters  \\ 
 \hdashline 
     &           &    1 &  5.00e-09 &  1.77e-08 &      \\ 
     &           &    2 &  4.99e-09 &  1.43e-16 &      \\ 
     &           &    3 &  4.98e-09 &  1.42e-16 &      \\ 
     &           &    4 &  4.97e-09 &  1.42e-16 &      \\ 
     &           &    5 &  4.97e-09 &  1.42e-16 &      \\ 
     &           &    6 &  7.27e-08 &  1.42e-16 &      \\ 
     &           &    7 &  5.06e-09 &  2.08e-15 &      \\ 
     &           &    8 &  5.06e-09 &  1.45e-16 &      \\ 
     &           &    9 &  5.05e-09 &  1.44e-16 &      \\ 
     &           &   10 &  5.04e-09 &  1.44e-16 &      \\ 
 429 &  4.28e+03 &   10 &           &           & iters  \\ 
 \hdashline 
     &           &    1 &  4.94e-09 &  2.08e-08 &      \\ 
     &           &    2 &  4.94e-09 &  1.41e-16 &      \\ 
     &           &    3 &  4.93e-09 &  1.41e-16 &      \\ 
     &           &    4 &  4.92e-09 &  1.41e-16 &      \\ 
     &           &    5 &  4.91e-09 &  1.40e-16 &      \\ 
     &           &    6 &  4.91e-09 &  1.40e-16 &      \\ 
     &           &    7 &  4.90e-09 &  1.40e-16 &      \\ 
     &           &    8 &  4.89e-09 &  1.40e-16 &      \\ 
     &           &    9 &  4.89e-09 &  1.40e-16 &      \\ 
     &           &   10 &  4.88e-09 &  1.39e-16 &      \\ 
 430 &  4.29e+03 &   10 &           &           & iters  \\ 
 \hdashline 
     &           &    1 &  4.23e-08 &  2.71e-08 &      \\ 
     &           &    2 &  4.22e-08 &  1.21e-15 &      \\ 
     &           &    3 &  3.55e-08 &  1.21e-15 &      \\ 
     &           &    4 &  4.22e-08 &  1.01e-15 &      \\ 
     &           &    5 &  3.55e-08 &  1.21e-15 &      \\ 
     &           &    6 &  3.55e-08 &  1.01e-15 &      \\ 
     &           &    7 &  4.23e-08 &  1.01e-15 &      \\ 
     &           &    8 &  3.55e-08 &  1.21e-15 &      \\ 
     &           &    9 &  4.23e-08 &  1.01e-15 &      \\ 
     &           &   10 &  3.55e-08 &  1.21e-15 &      \\ 
 431 &  4.30e+03 &   10 &           &           & iters  \\ 
 \hdashline 
     &           &    1 &  4.04e-08 &  4.47e-08 &      \\ 
     &           &    2 &  3.74e-08 &  1.15e-15 &      \\ 
     &           &    3 &  4.04e-08 &  1.07e-15 &      \\ 
     &           &    4 &  3.74e-08 &  1.15e-15 &      \\ 
     &           &    5 &  3.73e-08 &  1.07e-15 &      \\ 
     &           &    6 &  4.04e-08 &  1.07e-15 &      \\ 
     &           &    7 &  3.73e-08 &  1.15e-15 &      \\ 
     &           &    8 &  4.04e-08 &  1.07e-15 &      \\ 
     &           &    9 &  3.73e-08 &  1.15e-15 &      \\ 
     &           &   10 &  4.04e-08 &  1.07e-15 &      \\ 
 432 &  4.31e+03 &   10 &           &           & iters  \\ 
 \hdashline 
     &           &    1 &  1.96e-08 &  4.61e-08 &      \\ 
     &           &    2 &  1.96e-08 &  5.59e-16 &      \\ 
     &           &    3 &  5.81e-08 &  5.58e-16 &      \\ 
     &           &    4 &  1.96e-08 &  1.66e-15 &      \\ 
     &           &    5 &  1.96e-08 &  5.60e-16 &      \\ 
     &           &    6 &  1.96e-08 &  5.59e-16 &      \\ 
     &           &    7 &  5.81e-08 &  5.58e-16 &      \\ 
     &           &    8 &  1.96e-08 &  1.66e-15 &      \\ 
     &           &    9 &  1.96e-08 &  5.60e-16 &      \\ 
     &           &   10 &  1.96e-08 &  5.59e-16 &      \\ 
 433 &  4.32e+03 &   10 &           &           & iters  \\ 
 \hdashline 
     &           &    1 &  1.73e-08 &  3.30e-08 &      \\ 
     &           &    2 &  6.04e-08 &  4.94e-16 &      \\ 
     &           &    3 &  1.74e-08 &  1.72e-15 &      \\ 
     &           &    4 &  1.73e-08 &  4.96e-16 &      \\ 
     &           &    5 &  1.73e-08 &  4.95e-16 &      \\ 
     &           &    6 &  6.04e-08 &  4.94e-16 &      \\ 
     &           &    7 &  1.74e-08 &  1.72e-15 &      \\ 
     &           &    8 &  1.73e-08 &  4.96e-16 &      \\ 
     &           &    9 &  1.73e-08 &  4.95e-16 &      \\ 
     &           &   10 &  1.73e-08 &  4.94e-16 &      \\ 
 434 &  4.33e+03 &   10 &           &           & iters  \\ 
 \hdashline 
     &           &    1 &  4.24e-08 &  1.61e-08 &      \\ 
     &           &    2 &  3.54e-09 &  1.21e-15 &      \\ 
     &           &    3 &  3.54e-09 &  1.01e-16 &      \\ 
     &           &    4 &  3.53e-09 &  1.01e-16 &      \\ 
     &           &    5 &  3.53e-09 &  1.01e-16 &      \\ 
     &           &    6 &  3.52e-09 &  1.01e-16 &      \\ 
     &           &    7 &  3.52e-09 &  1.01e-16 &      \\ 
     &           &    8 &  3.51e-09 &  1.00e-16 &      \\ 
     &           &    9 &  3.51e-09 &  1.00e-16 &      \\ 
     &           &   10 &  3.50e-09 &  1.00e-16 &      \\ 
 435 &  4.34e+03 &   10 &           &           & iters  \\ 
 \hdashline 
     &           &    1 &  3.92e-08 &  2.44e-09 &      \\ 
     &           &    2 &  3.04e-10 &  1.12e-15 &      \\ 
 436 &  4.35e+03 &    2 &           &           & forces  \\ 
 \hdashline 
     &           &    1 &  1.59e-08 &  2.43e-08 &      \\ 
     &           &    2 &  1.59e-08 &  4.55e-16 &      \\ 
     &           &    3 &  6.18e-08 &  4.54e-16 &      \\ 
     &           &    4 &  1.60e-08 &  1.76e-15 &      \\ 
     &           &    5 &  1.60e-08 &  4.56e-16 &      \\ 
     &           &    6 &  1.59e-08 &  4.55e-16 &      \\ 
     &           &    7 &  1.59e-08 &  4.55e-16 &      \\ 
     &           &    8 &  6.18e-08 &  4.54e-16 &      \\ 
     &           &    9 &  1.60e-08 &  1.76e-15 &      \\ 
     &           &   10 &  1.59e-08 &  4.56e-16 &      \\ 
 437 &  4.36e+03 &   10 &           &           & iters  \\ 
 \hdashline 
     &           &    1 &  2.07e-09 &  2.65e-08 &      \\ 
     &           &    2 &  2.07e-09 &  5.92e-17 &      \\ 
     &           &    3 &  2.07e-09 &  5.91e-17 &      \\ 
     &           &    4 &  2.06e-09 &  5.90e-17 &      \\ 
     &           &    5 &  2.06e-09 &  5.89e-17 &      \\ 
     &           &    6 &  2.06e-09 &  5.88e-17 &      \\ 
     &           &    7 &  2.06e-09 &  5.88e-17 &      \\ 
     &           &    8 &  2.05e-09 &  5.87e-17 &      \\ 
     &           &    9 &  2.05e-09 &  5.86e-17 &      \\ 
     &           &   10 &  2.05e-09 &  5.85e-17 &      \\ 
 438 &  4.37e+03 &   10 &           &           & iters  \\ 
 \hdashline 
     &           &    1 &  1.88e-08 &  2.23e-08 &      \\ 
     &           &    2 &  1.88e-08 &  5.37e-16 &      \\ 
     &           &    3 &  5.89e-08 &  5.36e-16 &      \\ 
     &           &    4 &  1.89e-08 &  1.68e-15 &      \\ 
     &           &    5 &  1.88e-08 &  5.38e-16 &      \\ 
     &           &    6 &  1.88e-08 &  5.37e-16 &      \\ 
     &           &    7 &  5.89e-08 &  5.37e-16 &      \\ 
     &           &    8 &  1.89e-08 &  1.68e-15 &      \\ 
     &           &    9 &  1.88e-08 &  5.38e-16 &      \\ 
     &           &   10 &  1.88e-08 &  5.37e-16 &      \\ 
 439 &  4.38e+03 &   10 &           &           & iters  \\ 
 \hdashline 
     &           &    1 &  2.46e-09 &  3.21e-08 &      \\ 
     &           &    2 &  2.46e-09 &  7.02e-17 &      \\ 
     &           &    3 &  2.45e-09 &  7.01e-17 &      \\ 
     &           &    4 &  7.52e-08 &  7.00e-17 &      \\ 
     &           &    5 &  4.14e-08 &  2.15e-15 &      \\ 
     &           &    6 &  2.50e-09 &  1.18e-15 &      \\ 
     &           &    7 &  2.49e-09 &  7.13e-17 &      \\ 
     &           &    8 &  2.49e-09 &  7.12e-17 &      \\ 
     &           &    9 &  2.49e-09 &  7.11e-17 &      \\ 
     &           &   10 &  2.48e-09 &  7.10e-17 &      \\ 
 440 &  4.39e+03 &   10 &           &           & iters  \\ 
 \hdashline 
     &           &    1 &  5.05e-08 &  2.61e-08 &      \\ 
     &           &    2 &  2.72e-08 &  1.44e-15 &      \\ 
     &           &    3 &  5.05e-08 &  7.77e-16 &      \\ 
     &           &    4 &  2.73e-08 &  1.44e-15 &      \\ 
     &           &    5 &  2.72e-08 &  7.78e-16 &      \\ 
     &           &    6 &  5.05e-08 &  7.77e-16 &      \\ 
     &           &    7 &  2.73e-08 &  1.44e-15 &      \\ 
     &           &    8 &  2.72e-08 &  7.78e-16 &      \\ 
     &           &    9 &  5.05e-08 &  7.77e-16 &      \\ 
     &           &   10 &  2.73e-08 &  1.44e-15 &      \\ 
 441 &  4.40e+03 &   10 &           &           & iters  \\ 
 \hdashline 
     &           &    1 &  1.24e-08 &  1.27e-08 &      \\ 
     &           &    2 &  6.53e-08 &  3.54e-16 &      \\ 
     &           &    3 &  1.25e-08 &  1.86e-15 &      \\ 
     &           &    4 &  1.25e-08 &  3.56e-16 &      \\ 
     &           &    5 &  1.24e-08 &  3.55e-16 &      \\ 
     &           &    6 &  1.24e-08 &  3.55e-16 &      \\ 
     &           &    7 &  1.24e-08 &  3.54e-16 &      \\ 
     &           &    8 &  6.53e-08 &  3.54e-16 &      \\ 
     &           &    9 &  1.25e-08 &  1.86e-15 &      \\ 
     &           &   10 &  1.25e-08 &  3.56e-16 &      \\ 
 442 &  4.41e+03 &   10 &           &           & iters  \\ 
 \hdashline 
     &           &    1 &  9.80e-09 &  1.34e-08 &      \\ 
     &           &    2 &  9.78e-09 &  2.80e-16 &      \\ 
     &           &    3 &  9.77e-09 &  2.79e-16 &      \\ 
     &           &    4 &  9.75e-09 &  2.79e-16 &      \\ 
     &           &    5 &  6.79e-08 &  2.78e-16 &      \\ 
     &           &    6 &  8.75e-08 &  1.94e-15 &      \\ 
     &           &    7 &  6.80e-08 &  2.50e-15 &      \\ 
     &           &    8 &  9.81e-09 &  1.94e-15 &      \\ 
     &           &    9 &  9.80e-09 &  2.80e-16 &      \\ 
     &           &   10 &  9.78e-09 &  2.80e-16 &      \\ 
 443 &  4.42e+03 &   10 &           &           & iters  \\ 
 \hdashline 
     &           &    1 &  9.59e-09 &  1.14e-08 &      \\ 
     &           &    2 &  6.81e-08 &  2.74e-16 &      \\ 
     &           &    3 &  9.68e-09 &  1.94e-15 &      \\ 
     &           &    4 &  9.66e-09 &  2.76e-16 &      \\ 
     &           &    5 &  9.65e-09 &  2.76e-16 &      \\ 
     &           &    6 &  9.64e-09 &  2.75e-16 &      \\ 
     &           &    7 &  9.62e-09 &  2.75e-16 &      \\ 
     &           &    8 &  9.61e-09 &  2.75e-16 &      \\ 
     &           &    9 &  9.59e-09 &  2.74e-16 &      \\ 
     &           &   10 &  6.81e-08 &  2.74e-16 &      \\ 
 444 &  4.43e+03 &   10 &           &           & iters  \\ 
 \hdashline 
     &           &    1 &  2.71e-08 &  2.97e-09 &      \\ 
     &           &    2 &  2.70e-08 &  7.73e-16 &      \\ 
     &           &    3 &  2.70e-08 &  7.72e-16 &      \\ 
     &           &    4 &  2.70e-08 &  7.70e-16 &      \\ 
     &           &    5 &  5.08e-08 &  7.69e-16 &      \\ 
     &           &    6 &  2.70e-08 &  1.45e-15 &      \\ 
     &           &    7 &  2.70e-08 &  7.70e-16 &      \\ 
     &           &    8 &  5.08e-08 &  7.69e-16 &      \\ 
     &           &    9 &  2.70e-08 &  1.45e-15 &      \\ 
     &           &   10 &  2.69e-08 &  7.70e-16 &      \\ 
 445 &  4.44e+03 &   10 &           &           & iters  \\ 
 \hdashline 
     &           &    1 &  3.41e-08 &  5.99e-10 &      \\ 
     &           &    2 &  3.41e-08 &  9.74e-16 &      \\ 
     &           &    3 &  3.40e-08 &  9.72e-16 &      \\ 
     &           &    4 &  4.37e-08 &  9.71e-16 &      \\ 
     &           &    5 &  3.40e-08 &  1.25e-15 &      \\ 
     &           &    6 &  4.37e-08 &  9.71e-16 &      \\ 
     &           &    7 &  3.40e-08 &  1.25e-15 &      \\ 
     &           &    8 &  3.40e-08 &  9.72e-16 &      \\ 
     &           &    9 &  4.37e-08 &  9.70e-16 &      \\ 
     &           &   10 &  3.40e-08 &  1.25e-15 &      \\ 
 446 &  4.45e+03 &   10 &           &           & iters  \\ 
 \hdashline 
     &           &    1 &  2.88e-08 &  7.98e-09 &      \\ 
     &           &    2 &  2.87e-08 &  8.21e-16 &      \\ 
     &           &    3 &  4.90e-08 &  8.19e-16 &      \\ 
     &           &    4 &  2.87e-08 &  1.40e-15 &      \\ 
     &           &    5 &  2.87e-08 &  8.20e-16 &      \\ 
     &           &    6 &  4.90e-08 &  8.19e-16 &      \\ 
     &           &    7 &  2.87e-08 &  1.40e-15 &      \\ 
     &           &    8 &  2.87e-08 &  8.20e-16 &      \\ 
     &           &    9 &  4.90e-08 &  8.19e-16 &      \\ 
     &           &   10 &  2.87e-08 &  1.40e-15 &      \\ 
 447 &  4.46e+03 &   10 &           &           & iters  \\ 
 \hdashline 
     &           &    1 &  3.73e-10 &  2.22e-08 &      \\ 
     &           &    2 &  3.72e-10 &  1.06e-17 &      \\ 
     &           &    3 &  3.72e-10 &  1.06e-17 &      \\ 
     &           &    4 &  3.71e-10 &  1.06e-17 &      \\ 
     &           &    5 &  3.71e-10 &  1.06e-17 &      \\ 
     &           &    6 &  3.70e-10 &  1.06e-17 &      \\ 
     &           &    7 &  3.70e-10 &  1.06e-17 &      \\ 
     &           &    8 &  3.69e-10 &  1.06e-17 &      \\ 
     &           &    9 &  3.69e-10 &  1.05e-17 &      \\ 
     &           &   10 &  3.68e-10 &  1.05e-17 &      \\ 
 448 &  4.47e+03 &   10 &           &           & iters  \\ 
 \hdashline 
     &           &    1 &  8.88e-10 &  8.30e-09 &      \\ 
     &           &    2 &  8.87e-10 &  2.53e-17 &      \\ 
     &           &    3 &  8.85e-10 &  2.53e-17 &      \\ 
     &           &    4 &  8.84e-10 &  2.53e-17 &      \\ 
     &           &    5 &  8.83e-10 &  2.52e-17 &      \\ 
     &           &    6 &  8.82e-10 &  2.52e-17 &      \\ 
     &           &    7 &  8.80e-10 &  2.52e-17 &      \\ 
     &           &    8 &  8.79e-10 &  2.51e-17 &      \\ 
     &           &    9 &  8.78e-10 &  2.51e-17 &      \\ 
     &           &   10 &  8.77e-10 &  2.51e-17 &      \\ 
 449 &  4.48e+03 &   10 &           &           & iters  \\ 
 \hdashline 
     &           &    1 &  3.30e-08 &  2.58e-08 &      \\ 
     &           &    2 &  3.29e-08 &  9.42e-16 &      \\ 
     &           &    3 &  4.48e-08 &  9.40e-16 &      \\ 
     &           &    4 &  3.30e-08 &  1.28e-15 &      \\ 
     &           &    5 &  4.48e-08 &  9.41e-16 &      \\ 
     &           &    6 &  3.30e-08 &  1.28e-15 &      \\ 
     &           &    7 &  3.29e-08 &  9.41e-16 &      \\ 
     &           &    8 &  4.48e-08 &  9.40e-16 &      \\ 
     &           &    9 &  3.29e-08 &  1.28e-15 &      \\ 
     &           &   10 &  4.48e-08 &  9.40e-16 &      \\ 
 450 &  4.49e+03 &   10 &           &           & iters  \\ 
 \hdashline 
     &           &    1 &  1.58e-08 &  1.59e-08 &      \\ 
     &           &    2 &  1.57e-08 &  4.50e-16 &      \\ 
     &           &    3 &  6.20e-08 &  4.49e-16 &      \\ 
     &           &    4 &  1.58e-08 &  1.77e-15 &      \\ 
     &           &    5 &  1.58e-08 &  4.51e-16 &      \\ 
     &           &    6 &  1.58e-08 &  4.51e-16 &      \\ 
     &           &    7 &  1.57e-08 &  4.50e-16 &      \\ 
     &           &    8 &  6.20e-08 &  4.49e-16 &      \\ 
     &           &    9 &  1.58e-08 &  1.77e-15 &      \\ 
     &           &   10 &  1.58e-08 &  4.51e-16 &      \\ 
 451 &  4.50e+03 &   10 &           &           & iters  \\ 
 \hdashline 
     &           &    1 &  5.35e-08 &  5.62e-09 &      \\ 
     &           &    2 &  2.42e-08 &  1.53e-15 &      \\ 
     &           &    3 &  2.42e-08 &  6.92e-16 &      \\ 
     &           &    4 &  2.42e-08 &  6.91e-16 &      \\ 
     &           &    5 &  5.36e-08 &  6.90e-16 &      \\ 
     &           &    6 &  2.42e-08 &  1.53e-15 &      \\ 
     &           &    7 &  2.42e-08 &  6.91e-16 &      \\ 
     &           &    8 &  5.35e-08 &  6.90e-16 &      \\ 
     &           &    9 &  2.42e-08 &  1.53e-15 &      \\ 
     &           &   10 &  2.42e-08 &  6.91e-16 &      \\ 
 452 &  4.51e+03 &   10 &           &           & iters  \\ 
 \hdashline 
     &           &    1 &  7.55e-08 &  3.55e-09 &      \\ 
     &           &    2 &  8.00e-08 &  2.16e-15 &      \\ 
     &           &    3 &  7.55e-08 &  2.28e-15 &      \\ 
     &           &    4 &  8.00e-08 &  2.16e-15 &      \\ 
     &           &    5 &  7.55e-08 &  2.28e-15 &      \\ 
     &           &    6 &  8.00e-08 &  2.16e-15 &      \\ 
     &           &    7 &  7.55e-08 &  2.28e-15 &      \\ 
     &           &    8 &  2.27e-09 &  2.16e-15 &      \\ 
     &           &    9 &  2.27e-09 &  6.48e-17 &      \\ 
     &           &   10 &  2.26e-09 &  6.47e-17 &      \\ 
 453 &  4.52e+03 &   10 &           &           & iters  \\ 
 \hdashline 
     &           &    1 &  4.15e-08 &  9.74e-09 &      \\ 
     &           &    2 &  4.15e-08 &  1.19e-15 &      \\ 
     &           &    3 &  3.63e-08 &  1.18e-15 &      \\ 
     &           &    4 &  4.15e-08 &  1.04e-15 &      \\ 
     &           &    5 &  3.63e-08 &  1.18e-15 &      \\ 
     &           &    6 &  4.15e-08 &  1.04e-15 &      \\ 
     &           &    7 &  3.63e-08 &  1.18e-15 &      \\ 
     &           &    8 &  4.14e-08 &  1.04e-15 &      \\ 
     &           &    9 &  3.63e-08 &  1.18e-15 &      \\ 
     &           &   10 &  4.14e-08 &  1.04e-15 &      \\ 
 454 &  4.53e+03 &   10 &           &           & iters  \\ 
 \hdashline 
     &           &    1 &  6.85e-08 &  1.73e-08 &      \\ 
     &           &    2 &  9.28e-09 &  1.96e-15 &      \\ 
     &           &    3 &  9.27e-09 &  2.65e-16 &      \\ 
     &           &    4 &  9.26e-09 &  2.65e-16 &      \\ 
     &           &    5 &  9.24e-09 &  2.64e-16 &      \\ 
     &           &    6 &  9.23e-09 &  2.64e-16 &      \\ 
     &           &    7 &  9.22e-09 &  2.63e-16 &      \\ 
     &           &    8 &  6.85e-08 &  2.63e-16 &      \\ 
     &           &    9 &  9.30e-09 &  1.95e-15 &      \\ 
     &           &   10 &  9.29e-09 &  2.65e-16 &      \\ 
 455 &  4.54e+03 &   10 &           &           & iters  \\ 
 \hdashline 
     &           &    1 &  1.51e-08 &  8.29e-09 &      \\ 
     &           &    2 &  6.26e-08 &  4.32e-16 &      \\ 
     &           &    3 &  1.52e-08 &  1.79e-15 &      \\ 
     &           &    4 &  1.52e-08 &  4.34e-16 &      \\ 
     &           &    5 &  1.52e-08 &  4.34e-16 &      \\ 
     &           &    6 &  1.51e-08 &  4.33e-16 &      \\ 
     &           &    7 &  6.26e-08 &  4.32e-16 &      \\ 
     &           &    8 &  1.52e-08 &  1.79e-15 &      \\ 
     &           &    9 &  1.52e-08 &  4.34e-16 &      \\ 
     &           &   10 &  1.52e-08 &  4.34e-16 &      \\ 
 456 &  4.55e+03 &   10 &           &           & iters  \\ 
 \hdashline 
     &           &    1 &  2.59e-08 &  7.78e-09 &      \\ 
     &           &    2 &  2.59e-08 &  7.40e-16 &      \\ 
     &           &    3 &  5.18e-08 &  7.39e-16 &      \\ 
     &           &    4 &  2.59e-08 &  1.48e-15 &      \\ 
     &           &    5 &  2.59e-08 &  7.40e-16 &      \\ 
     &           &    6 &  5.18e-08 &  7.39e-16 &      \\ 
     &           &    7 &  2.59e-08 &  1.48e-15 &      \\ 
     &           &    8 &  2.59e-08 &  7.40e-16 &      \\ 
     &           &    9 &  5.18e-08 &  7.39e-16 &      \\ 
     &           &   10 &  2.59e-08 &  1.48e-15 &      \\ 
 457 &  4.56e+03 &   10 &           &           & iters  \\ 
 \hdashline 
     &           &    1 &  6.62e-08 &  1.39e-08 &      \\ 
     &           &    2 &  1.16e-08 &  1.89e-15 &      \\ 
     &           &    3 &  1.16e-08 &  3.32e-16 &      \\ 
     &           &    4 &  1.16e-08 &  3.31e-16 &      \\ 
     &           &    5 &  1.16e-08 &  3.31e-16 &      \\ 
     &           &    6 &  1.16e-08 &  3.30e-16 &      \\ 
     &           &    7 &  1.15e-08 &  3.30e-16 &      \\ 
     &           &    8 &  6.62e-08 &  3.29e-16 &      \\ 
     &           &    9 &  1.16e-08 &  1.89e-15 &      \\ 
     &           &   10 &  1.16e-08 &  3.31e-16 &      \\ 
 458 &  4.57e+03 &   10 &           &           & iters  \\ 
 \hdashline 
     &           &    1 &  5.07e-08 &  3.82e-09 &      \\ 
     &           &    2 &  2.70e-08 &  1.45e-15 &      \\ 
     &           &    3 &  2.70e-08 &  7.72e-16 &      \\ 
     &           &    4 &  5.07e-08 &  7.71e-16 &      \\ 
     &           &    5 &  2.70e-08 &  1.45e-15 &      \\ 
     &           &    6 &  5.07e-08 &  7.72e-16 &      \\ 
     &           &    7 &  2.71e-08 &  1.45e-15 &      \\ 
     &           &    8 &  2.70e-08 &  7.73e-16 &      \\ 
     &           &    9 &  5.07e-08 &  7.71e-16 &      \\ 
     &           &   10 &  2.71e-08 &  1.45e-15 &      \\ 
 459 &  4.58e+03 &   10 &           &           & iters  \\ 
 \hdashline 
     &           &    1 &  1.81e-09 &  1.24e-10 &      \\ 
     &           &    2 &  1.81e-09 &  5.17e-17 &      \\ 
     &           &    3 &  1.81e-09 &  5.16e-17 &      \\ 
     &           &    4 &  1.80e-09 &  5.15e-17 &      \\ 
     &           &    5 &  1.80e-09 &  5.15e-17 &      \\ 
     &           &    6 &  1.80e-09 &  5.14e-17 &      \\ 
     &           &    7 &  1.80e-09 &  5.13e-17 &      \\ 
     &           &    8 &  1.79e-09 &  5.12e-17 &      \\ 
     &           &    9 &  1.79e-09 &  5.12e-17 &      \\ 
     &           &   10 &  7.59e-08 &  5.11e-17 &      \\ 
 460 &  4.59e+03 &   10 &           &           & iters  \\ 
 \hdashline 
     &           &    1 &  3.91e-08 &  2.57e-08 &      \\ 
     &           &    2 &  3.87e-08 &  1.12e-15 &      \\ 
     &           &    3 &  3.91e-08 &  1.10e-15 &      \\ 
     &           &    4 &  3.87e-08 &  1.12e-15 &      \\ 
     &           &    5 &  3.91e-08 &  1.10e-15 &      \\ 
     &           &    6 &  3.87e-08 &  1.12e-15 &      \\ 
     &           &    7 &  3.91e-08 &  1.10e-15 &      \\ 
     &           &    8 &  3.87e-08 &  1.12e-15 &      \\ 
     &           &    9 &  3.91e-08 &  1.10e-15 &      \\ 
     &           &   10 &  3.87e-08 &  1.12e-15 &      \\ 
 461 &  4.60e+03 &   10 &           &           & iters  \\ 
 \hdashline 
     &           &    1 &  8.71e-10 &  6.26e-08 &      \\ 
     &           &    2 &  8.70e-10 &  2.49e-17 &      \\ 
     &           &    3 &  8.69e-10 &  2.48e-17 &      \\ 
     &           &    4 &  8.67e-10 &  2.48e-17 &      \\ 
     &           &    5 &  8.66e-10 &  2.48e-17 &      \\ 
     &           &    6 &  8.65e-10 &  2.47e-17 &      \\ 
     &           &    7 &  8.64e-10 &  2.47e-17 &      \\ 
     &           &    8 &  8.62e-10 &  2.46e-17 &      \\ 
     &           &    9 &  8.61e-10 &  2.46e-17 &      \\ 
     &           &   10 &  7.68e-08 &  2.46e-17 &      \\ 
 462 &  4.61e+03 &   10 &           &           & iters  \\ 
 \hdashline 
     &           &    1 &  2.86e-08 &  5.93e-08 &      \\ 
     &           &    2 &  2.86e-08 &  8.18e-16 &      \\ 
     &           &    3 &  1.03e-08 &  8.16e-16 &      \\ 
     &           &    4 &  1.03e-08 &  2.93e-16 &      \\ 
     &           &    5 &  1.02e-08 &  2.93e-16 &      \\ 
     &           &    6 &  2.86e-08 &  2.93e-16 &      \\ 
     &           &    7 &  1.03e-08 &  8.17e-16 &      \\ 
     &           &    8 &  1.03e-08 &  2.93e-16 &      \\ 
     &           &    9 &  1.02e-08 &  2.93e-16 &      \\ 
     &           &   10 &  2.86e-08 &  2.92e-16 &      \\ 
 463 &  4.62e+03 &   10 &           &           & iters  \\ 
 \hdashline 
     &           &    1 &  6.13e-09 &  2.11e-08 &      \\ 
     &           &    2 &  6.12e-09 &  1.75e-16 &      \\ 
     &           &    3 &  7.16e-08 &  1.75e-16 &      \\ 
     &           &    4 &  4.51e-08 &  2.04e-15 &      \\ 
     &           &    5 &  6.15e-09 &  1.29e-15 &      \\ 
     &           &    6 &  6.14e-09 &  1.75e-16 &      \\ 
     &           &    7 &  6.13e-09 &  1.75e-16 &      \\ 
     &           &    8 &  6.12e-09 &  1.75e-16 &      \\ 
     &           &    9 &  7.16e-08 &  1.75e-16 &      \\ 
     &           &   10 &  4.51e-08 &  2.04e-15 &      \\ 
 464 &  4.63e+03 &   10 &           &           & iters  \\ 
 \hdashline 
     &           &    1 &  4.16e-08 &  1.67e-08 &      \\ 
     &           &    2 &  3.62e-08 &  1.19e-15 &      \\ 
     &           &    3 &  4.16e-08 &  1.03e-15 &      \\ 
     &           &    4 &  3.62e-08 &  1.19e-15 &      \\ 
     &           &    5 &  4.16e-08 &  1.03e-15 &      \\ 
     &           &    6 &  3.62e-08 &  1.19e-15 &      \\ 
     &           &    7 &  4.16e-08 &  1.03e-15 &      \\ 
     &           &    8 &  3.62e-08 &  1.19e-15 &      \\ 
     &           &    9 &  3.61e-08 &  1.03e-15 &      \\ 
     &           &   10 &  4.16e-08 &  1.03e-15 &      \\ 
 465 &  4.64e+03 &   10 &           &           & iters  \\ 
 \hdashline 
     &           &    1 &  1.37e-08 &  2.21e-08 &      \\ 
     &           &    2 &  1.37e-08 &  3.92e-16 &      \\ 
     &           &    3 &  1.37e-08 &  3.91e-16 &      \\ 
     &           &    4 &  6.40e-08 &  3.91e-16 &      \\ 
     &           &    5 &  1.38e-08 &  1.83e-15 &      \\ 
     &           &    6 &  1.37e-08 &  3.93e-16 &      \\ 
     &           &    7 &  1.37e-08 &  3.92e-16 &      \\ 
     &           &    8 &  1.37e-08 &  3.92e-16 &      \\ 
     &           &    9 &  1.37e-08 &  3.91e-16 &      \\ 
     &           &   10 &  6.40e-08 &  3.90e-16 &      \\ 
 466 &  4.65e+03 &   10 &           &           & iters  \\ 
 \hdashline 
     &           &    1 &  8.19e-11 &  3.14e-08 &      \\ 
 467 &  4.66e+03 &    1 &           &           & forces  \\ 
 \hdashline 
     &           &    1 &  4.09e-08 &  5.95e-08 &      \\ 
     &           &    2 &  3.68e-08 &  1.17e-15 &      \\ 
     &           &    3 &  4.09e-08 &  1.05e-15 &      \\ 
     &           &    4 &  3.68e-08 &  1.17e-15 &      \\ 
     &           &    5 &  4.09e-08 &  1.05e-15 &      \\ 
     &           &    6 &  3.68e-08 &  1.17e-15 &      \\ 
     &           &    7 &  3.68e-08 &  1.05e-15 &      \\ 
     &           &    8 &  4.10e-08 &  1.05e-15 &      \\ 
     &           &    9 &  3.68e-08 &  1.17e-15 &      \\ 
     &           &   10 &  4.10e-08 &  1.05e-15 &      \\ 
 468 &  4.67e+03 &   10 &           &           & iters  \\ 
 \hdashline 
     &           &    1 &  1.36e-08 &  2.00e-08 &      \\ 
     &           &    2 &  2.52e-08 &  3.89e-16 &      \\ 
     &           &    3 &  1.37e-08 &  7.20e-16 &      \\ 
     &           &    4 &  1.36e-08 &  3.90e-16 &      \\ 
     &           &    5 &  2.52e-08 &  3.89e-16 &      \\ 
     &           &    6 &  1.37e-08 &  7.20e-16 &      \\ 
     &           &    7 &  1.36e-08 &  3.90e-16 &      \\ 
     &           &    8 &  2.52e-08 &  3.89e-16 &      \\ 
     &           &    9 &  1.37e-08 &  7.20e-16 &      \\ 
     &           &   10 &  2.52e-08 &  3.90e-16 &      \\ 
 469 &  4.68e+03 &   10 &           &           & iters  \\ 
 \hdashline 
     &           &    1 &  6.33e-09 &  1.54e-08 &      \\ 
     &           &    2 &  6.32e-09 &  1.81e-16 &      \\ 
     &           &    3 &  6.31e-09 &  1.80e-16 &      \\ 
     &           &    4 &  6.30e-09 &  1.80e-16 &      \\ 
     &           &    5 &  6.29e-09 &  1.80e-16 &      \\ 
     &           &    6 &  3.26e-08 &  1.80e-16 &      \\ 
     &           &    7 &  6.33e-09 &  9.29e-16 &      \\ 
     &           &    8 &  6.32e-09 &  1.81e-16 &      \\ 
     &           &    9 &  6.31e-09 &  1.80e-16 &      \\ 
     &           &   10 &  6.30e-09 &  1.80e-16 &      \\ 
 470 &  4.69e+03 &   10 &           &           & iters  \\ 
 \hdashline 
     &           &    1 &  3.61e-09 &  1.68e-08 &      \\ 
     &           &    2 &  3.60e-09 &  1.03e-16 &      \\ 
     &           &    3 &  3.60e-09 &  1.03e-16 &      \\ 
     &           &    4 &  3.59e-09 &  1.03e-16 &      \\ 
     &           &    5 &  3.59e-09 &  1.03e-16 &      \\ 
     &           &    6 &  3.53e-08 &  1.02e-16 &      \\ 
     &           &    7 &  3.63e-09 &  1.01e-15 &      \\ 
     &           &    8 &  3.63e-09 &  1.04e-16 &      \\ 
     &           &    9 &  3.62e-09 &  1.04e-16 &      \\ 
     &           &   10 &  3.62e-09 &  1.03e-16 &      \\ 
 471 &  4.70e+03 &   10 &           &           & iters  \\ 
 \hdashline 
     &           &    1 &  5.42e-09 &  9.52e-09 &      \\ 
     &           &    2 &  5.41e-09 &  1.55e-16 &      \\ 
     &           &    3 &  5.41e-09 &  1.54e-16 &      \\ 
     &           &    4 &  5.40e-09 &  1.54e-16 &      \\ 
     &           &    5 &  7.23e-08 &  1.54e-16 &      \\ 
     &           &    6 &  8.32e-08 &  2.06e-15 &      \\ 
     &           &    7 &  7.23e-08 &  2.37e-15 &      \\ 
     &           &    8 &  5.48e-09 &  2.06e-15 &      \\ 
     &           &    9 &  5.47e-09 &  1.56e-16 &      \\ 
     &           &   10 &  5.46e-09 &  1.56e-16 &      \\ 
 472 &  4.71e+03 &   10 &           &           & iters  \\ 
 \hdashline 
     &           &    1 &  3.76e-08 &  3.03e-09 &      \\ 
     &           &    2 &  1.25e-09 &  1.07e-15 &      \\ 
     &           &    3 &  1.25e-09 &  3.58e-17 &      \\ 
     &           &    4 &  1.25e-09 &  3.57e-17 &      \\ 
     &           &    5 &  1.25e-09 &  3.57e-17 &      \\ 
     &           &    6 &  1.25e-09 &  3.56e-17 &      \\ 
     &           &    7 &  1.25e-09 &  3.56e-17 &      \\ 
     &           &    8 &  1.24e-09 &  3.55e-17 &      \\ 
     &           &    9 &  1.24e-09 &  3.55e-17 &      \\ 
     &           &   10 &  1.24e-09 &  3.54e-17 &      \\ 
 473 &  4.72e+03 &   10 &           &           & iters  \\ 
 \hdashline 
     &           &    1 &  4.50e-08 &  4.53e-09 &      \\ 
     &           &    2 &  3.28e-08 &  1.28e-15 &      \\ 
     &           &    3 &  3.27e-08 &  9.35e-16 &      \\ 
     &           &    4 &  4.50e-08 &  9.34e-16 &      \\ 
     &           &    5 &  3.27e-08 &  1.29e-15 &      \\ 
     &           &    6 &  4.50e-08 &  9.34e-16 &      \\ 
     &           &    7 &  3.27e-08 &  1.28e-15 &      \\ 
     &           &    8 &  3.27e-08 &  9.35e-16 &      \\ 
     &           &    9 &  4.50e-08 &  9.33e-16 &      \\ 
     &           &   10 &  3.27e-08 &  1.29e-15 &      \\ 
 474 &  4.73e+03 &   10 &           &           & iters  \\ 
 \hdashline 
     &           &    1 &  6.50e-09 &  1.14e-08 &      \\ 
     &           &    2 &  6.49e-09 &  1.86e-16 &      \\ 
     &           &    3 &  6.48e-09 &  1.85e-16 &      \\ 
     &           &    4 &  7.12e-08 &  1.85e-16 &      \\ 
     &           &    5 &  4.54e-08 &  2.03e-15 &      \\ 
     &           &    6 &  6.51e-09 &  1.30e-15 &      \\ 
     &           &    7 &  6.50e-09 &  1.86e-16 &      \\ 
     &           &    8 &  6.49e-09 &  1.86e-16 &      \\ 
     &           &    9 &  6.48e-09 &  1.85e-16 &      \\ 
     &           &   10 &  7.12e-08 &  1.85e-16 &      \\ 
 475 &  4.74e+03 &   10 &           &           & iters  \\ 
 \hdashline 
     &           &    1 &  3.44e-08 &  4.17e-09 &      \\ 
     &           &    2 &  3.43e-08 &  9.81e-16 &      \\ 
     &           &    3 &  4.34e-08 &  9.80e-16 &      \\ 
     &           &    4 &  3.43e-08 &  1.24e-15 &      \\ 
     &           &    5 &  4.34e-08 &  9.80e-16 &      \\ 
     &           &    6 &  3.43e-08 &  1.24e-15 &      \\ 
     &           &    7 &  4.34e-08 &  9.80e-16 &      \\ 
     &           &    8 &  3.44e-08 &  1.24e-15 &      \\ 
     &           &    9 &  4.34e-08 &  9.81e-16 &      \\ 
     &           &   10 &  3.44e-08 &  1.24e-15 &      \\ 
 476 &  4.75e+03 &   10 &           &           & iters  \\ 
 \hdashline 
     &           &    1 &  1.16e-08 &  1.29e-08 &      \\ 
     &           &    2 &  1.16e-08 &  3.32e-16 &      \\ 
     &           &    3 &  1.16e-08 &  3.32e-16 &      \\ 
     &           &    4 &  6.61e-08 &  3.31e-16 &      \\ 
     &           &    5 &  5.05e-08 &  1.89e-15 &      \\ 
     &           &    6 &  1.16e-08 &  1.44e-15 &      \\ 
     &           &    7 &  1.16e-08 &  3.31e-16 &      \\ 
     &           &    8 &  6.61e-08 &  3.31e-16 &      \\ 
     &           &    9 &  1.17e-08 &  1.89e-15 &      \\ 
     &           &   10 &  1.17e-08 &  3.33e-16 &      \\ 
 477 &  4.76e+03 &   10 &           &           & iters  \\ 
 \hdashline 
     &           &    1 &  6.89e-08 &  1.45e-08 &      \\ 
     &           &    2 &  8.66e-08 &  1.97e-15 &      \\ 
     &           &    3 &  6.90e-08 &  2.47e-15 &      \\ 
     &           &    4 &  8.84e-09 &  1.97e-15 &      \\ 
     &           &    5 &  8.82e-09 &  2.52e-16 &      \\ 
     &           &    6 &  8.81e-09 &  2.52e-16 &      \\ 
     &           &    7 &  8.80e-09 &  2.51e-16 &      \\ 
     &           &    8 &  8.79e-09 &  2.51e-16 &      \\ 
     &           &    9 &  8.77e-09 &  2.51e-16 &      \\ 
     &           &   10 &  6.89e-08 &  2.50e-16 &      \\ 
 478 &  4.77e+03 &   10 &           &           & iters  \\ 
 \hdashline 
     &           &    1 &  7.64e-08 &  2.63e-08 &      \\ 
     &           &    2 &  1.40e-09 &  2.18e-15 &      \\ 
     &           &    3 &  1.40e-09 &  4.00e-17 &      \\ 
     &           &    4 &  1.40e-09 &  3.99e-17 &      \\ 
     &           &    5 &  1.39e-09 &  3.99e-17 &      \\ 
     &           &    6 &  1.39e-09 &  3.98e-17 &      \\ 
     &           &    7 &  1.39e-09 &  3.97e-17 &      \\ 
     &           &    8 &  1.39e-09 &  3.97e-17 &      \\ 
     &           &    9 &  1.39e-09 &  3.96e-17 &      \\ 
     &           &   10 &  1.38e-09 &  3.96e-17 &      \\ 
 479 &  4.78e+03 &   10 &           &           & iters  \\ 
 \hdashline 
     &           &    1 &  3.37e-08 &  2.93e-08 &      \\ 
     &           &    2 &  4.40e-08 &  9.62e-16 &      \\ 
     &           &    3 &  3.37e-08 &  1.26e-15 &      \\ 
     &           &    4 &  4.40e-08 &  9.62e-16 &      \\ 
     &           &    5 &  3.37e-08 &  1.26e-15 &      \\ 
     &           &    6 &  3.37e-08 &  9.62e-16 &      \\ 
     &           &    7 &  4.41e-08 &  9.61e-16 &      \\ 
     &           &    8 &  3.37e-08 &  1.26e-15 &      \\ 
     &           &    9 &  4.40e-08 &  9.62e-16 &      \\ 
     &           &   10 &  3.37e-08 &  1.26e-15 &      \\ 
 480 &  4.79e+03 &   10 &           &           & iters  \\ 
 \hdashline 
     &           &    1 &  5.70e-08 &  3.25e-09 &      \\ 
     &           &    2 &  2.08e-08 &  1.63e-15 &      \\ 
     &           &    3 &  2.07e-08 &  5.92e-16 &      \\ 
     &           &    4 &  5.70e-08 &  5.91e-16 &      \\ 
     &           &    5 &  2.08e-08 &  1.63e-15 &      \\ 
     &           &    6 &  2.07e-08 &  5.93e-16 &      \\ 
     &           &    7 &  2.07e-08 &  5.92e-16 &      \\ 
     &           &    8 &  5.70e-08 &  5.91e-16 &      \\ 
     &           &    9 &  2.08e-08 &  1.63e-15 &      \\ 
     &           &   10 &  2.07e-08 &  5.93e-16 &      \\ 
 481 &  4.80e+03 &   10 &           &           & iters  \\ 
 \hdashline 
     &           &    1 &  3.76e-08 &  7.77e-09 &      \\ 
     &           &    2 &  4.01e-08 &  1.07e-15 &      \\ 
     &           &    3 &  3.77e-08 &  1.14e-15 &      \\ 
     &           &    4 &  4.01e-08 &  1.07e-15 &      \\ 
     &           &    5 &  3.77e-08 &  1.14e-15 &      \\ 
     &           &    6 &  4.01e-08 &  1.07e-15 &      \\ 
     &           &    7 &  3.77e-08 &  1.14e-15 &      \\ 
     &           &    8 &  4.01e-08 &  1.07e-15 &      \\ 
     &           &    9 &  3.77e-08 &  1.14e-15 &      \\ 
     &           &   10 &  4.01e-08 &  1.07e-15 &      \\ 
 482 &  4.81e+03 &   10 &           &           & iters  \\ 
 \hdashline 
     &           &    1 &  1.35e-08 &  1.92e-09 &      \\ 
     &           &    2 &  1.35e-08 &  3.85e-16 &      \\ 
     &           &    3 &  1.34e-08 &  3.84e-16 &      \\ 
     &           &    4 &  6.43e-08 &  3.83e-16 &      \\ 
     &           &    5 &  1.35e-08 &  1.83e-15 &      \\ 
     &           &    6 &  1.35e-08 &  3.86e-16 &      \\ 
     &           &    7 &  1.35e-08 &  3.85e-16 &      \\ 
     &           &    8 &  1.35e-08 &  3.84e-16 &      \\ 
     &           &    9 &  1.34e-08 &  3.84e-16 &      \\ 
     &           &   10 &  6.43e-08 &  3.83e-16 &      \\ 
 483 &  4.82e+03 &   10 &           &           & iters  \\ 
 \hdashline 
     &           &    1 &  4.90e-10 &  2.29e-10 &      \\ 
     &           &    2 &  4.89e-10 &  1.40e-17 &      \\ 
     &           &    3 &  4.89e-10 &  1.40e-17 &      \\ 
     &           &    4 &  4.88e-10 &  1.39e-17 &      \\ 
     &           &    5 &  4.87e-10 &  1.39e-17 &      \\ 
     &           &    6 &  4.87e-10 &  1.39e-17 &      \\ 
     &           &    7 &  4.86e-10 &  1.39e-17 &      \\ 
     &           &    8 &  4.85e-10 &  1.39e-17 &      \\ 
     &           &    9 &  4.84e-10 &  1.38e-17 &      \\ 
     &           &   10 &  4.84e-10 &  1.38e-17 &      \\ 
 484 &  4.83e+03 &   10 &           &           & iters  \\ 
 \hdashline 
     &           &    1 &  5.41e-08 &  7.79e-09 &      \\ 
     &           &    2 &  2.37e-08 &  1.54e-15 &      \\ 
     &           &    3 &  2.37e-08 &  6.77e-16 &      \\ 
     &           &    4 &  2.36e-08 &  6.76e-16 &      \\ 
     &           &    5 &  5.41e-08 &  6.75e-16 &      \\ 
     &           &    6 &  2.37e-08 &  1.54e-15 &      \\ 
     &           &    7 &  2.37e-08 &  6.76e-16 &      \\ 
     &           &    8 &  5.41e-08 &  6.75e-16 &      \\ 
     &           &    9 &  2.37e-08 &  1.54e-15 &      \\ 
     &           &   10 &  2.37e-08 &  6.76e-16 &      \\ 
 485 &  4.84e+03 &   10 &           &           & iters  \\ 
 \hdashline 
     &           &    1 &  4.14e-08 &  1.56e-09 &      \\ 
     &           &    2 &  3.63e-08 &  1.18e-15 &      \\ 
     &           &    3 &  4.14e-08 &  1.04e-15 &      \\ 
     &           &    4 &  3.63e-08 &  1.18e-15 &      \\ 
     &           &    5 &  4.14e-08 &  1.04e-15 &      \\ 
     &           &    6 &  3.64e-08 &  1.18e-15 &      \\ 
     &           &    7 &  4.14e-08 &  1.04e-15 &      \\ 
     &           &    8 &  3.64e-08 &  1.18e-15 &      \\ 
     &           &    9 &  4.14e-08 &  1.04e-15 &      \\ 
     &           &   10 &  3.64e-08 &  1.18e-15 &      \\ 
 486 &  4.85e+03 &   10 &           &           & iters  \\ 
 \hdashline 
     &           &    1 &  7.00e-08 &  2.02e-08 &      \\ 
     &           &    2 &  8.54e-08 &  2.00e-15 &      \\ 
     &           &    3 &  7.01e-08 &  2.44e-15 &      \\ 
     &           &    4 &  7.73e-09 &  2.00e-15 &      \\ 
     &           &    5 &  7.71e-09 &  2.20e-16 &      \\ 
     &           &    6 &  7.70e-09 &  2.20e-16 &      \\ 
     &           &    7 &  7.69e-09 &  2.20e-16 &      \\ 
     &           &    8 &  7.68e-09 &  2.20e-16 &      \\ 
     &           &    9 &  7.67e-09 &  2.19e-16 &      \\ 
     &           &   10 &  7.66e-09 &  2.19e-16 &      \\ 
 487 &  4.86e+03 &   10 &           &           & iters  \\ 
 \hdashline 
     &           &    1 &  8.88e-08 &  2.96e-09 &      \\ 
     &           &    2 &  6.67e-08 &  2.53e-15 &      \\ 
     &           &    3 &  1.11e-08 &  1.90e-15 &      \\ 
     &           &    4 &  1.11e-08 &  3.16e-16 &      \\ 
     &           &    5 &  1.11e-08 &  3.16e-16 &      \\ 
     &           &    6 &  1.10e-08 &  3.16e-16 &      \\ 
     &           &    7 &  6.67e-08 &  3.15e-16 &      \\ 
     &           &    8 &  8.88e-08 &  1.90e-15 &      \\ 
     &           &    9 &  6.67e-08 &  2.53e-15 &      \\ 
     &           &   10 &  1.11e-08 &  1.90e-15 &      \\ 
 488 &  4.87e+03 &   10 &           &           & iters  \\ 
 \hdashline 
     &           &    1 &  2.75e-08 &  1.31e-08 &      \\ 
     &           &    2 &  5.02e-08 &  7.85e-16 &      \\ 
     &           &    3 &  2.75e-08 &  1.43e-15 &      \\ 
     &           &    4 &  2.75e-08 &  7.86e-16 &      \\ 
     &           &    5 &  5.02e-08 &  7.85e-16 &      \\ 
     &           &    6 &  2.75e-08 &  1.43e-15 &      \\ 
     &           &    7 &  2.75e-08 &  7.86e-16 &      \\ 
     &           &    8 &  5.02e-08 &  7.85e-16 &      \\ 
     &           &    9 &  2.75e-08 &  1.43e-15 &      \\ 
     &           &   10 &  2.75e-08 &  7.86e-16 &      \\ 
 489 &  4.88e+03 &   10 &           &           & iters  \\ 
 \hdashline 
     &           &    1 &  5.35e-09 &  4.35e-09 &      \\ 
     &           &    2 &  5.34e-09 &  1.53e-16 &      \\ 
     &           &    3 &  5.34e-09 &  1.53e-16 &      \\ 
     &           &    4 &  5.33e-09 &  1.52e-16 &      \\ 
     &           &    5 &  5.32e-09 &  1.52e-16 &      \\ 
     &           &    6 &  7.24e-08 &  1.52e-16 &      \\ 
     &           &    7 &  5.42e-09 &  2.07e-15 &      \\ 
     &           &    8 &  5.41e-09 &  1.55e-16 &      \\ 
     &           &    9 &  5.40e-09 &  1.54e-16 &      \\ 
     &           &   10 &  5.39e-09 &  1.54e-16 &      \\ 
 490 &  4.89e+03 &   10 &           &           & iters  \\ 
 \hdashline 
     &           &    1 &  1.43e-08 &  3.35e-09 &      \\ 
     &           &    2 &  1.43e-08 &  4.09e-16 &      \\ 
     &           &    3 &  6.34e-08 &  4.08e-16 &      \\ 
     &           &    4 &  9.21e-08 &  1.81e-15 &      \\ 
     &           &    5 &  6.34e-08 &  2.63e-15 &      \\ 
     &           &    6 &  1.43e-08 &  1.81e-15 &      \\ 
     &           &    7 &  1.43e-08 &  4.09e-16 &      \\ 
     &           &    8 &  1.43e-08 &  4.09e-16 &      \\ 
     &           &    9 &  6.34e-08 &  4.08e-16 &      \\ 
     &           &   10 &  1.44e-08 &  1.81e-15 &      \\ 
 491 &  4.90e+03 &   10 &           &           & iters  \\ 
 \hdashline 
     &           &    1 &  7.34e-09 &  1.46e-08 &      \\ 
     &           &    2 &  7.33e-09 &  2.09e-16 &      \\ 
     &           &    3 &  7.32e-09 &  2.09e-16 &      \\ 
     &           &    4 &  7.31e-09 &  2.09e-16 &      \\ 
     &           &    5 &  7.29e-09 &  2.08e-16 &      \\ 
     &           &    6 &  7.04e-08 &  2.08e-16 &      \\ 
     &           &    7 &  8.51e-08 &  2.01e-15 &      \\ 
     &           &    8 &  7.04e-08 &  2.43e-15 &      \\ 
     &           &    9 &  7.36e-09 &  2.01e-15 &      \\ 
     &           &   10 &  7.35e-09 &  2.10e-16 &      \\ 
 492 &  4.91e+03 &   10 &           &           & iters  \\ 
 \hdashline 
     &           &    1 &  1.99e-09 &  1.20e-08 &      \\ 
     &           &    2 &  1.99e-09 &  5.69e-17 &      \\ 
     &           &    3 &  1.99e-09 &  5.68e-17 &      \\ 
     &           &    4 &  1.98e-09 &  5.67e-17 &      \\ 
     &           &    5 &  7.57e-08 &  5.66e-17 &      \\ 
     &           &    6 &  7.98e-08 &  2.16e-15 &      \\ 
     &           &    7 &  7.57e-08 &  2.28e-15 &      \\ 
     &           &    8 &  7.98e-08 &  2.16e-15 &      \\ 
     &           &    9 &  7.57e-08 &  2.28e-15 &      \\ 
     &           &   10 &  7.98e-08 &  2.16e-15 &      \\ 
 493 &  4.92e+03 &   10 &           &           & iters  \\ 
 \hdashline 
     &           &    1 &  4.02e-09 &  7.41e-09 &      \\ 
     &           &    2 &  4.01e-09 &  1.15e-16 &      \\ 
     &           &    3 &  4.01e-09 &  1.15e-16 &      \\ 
     &           &    4 &  4.00e-09 &  1.14e-16 &      \\ 
     &           &    5 &  4.00e-09 &  1.14e-16 &      \\ 
     &           &    6 &  7.37e-08 &  1.14e-16 &      \\ 
     &           &    7 &  4.10e-09 &  2.10e-15 &      \\ 
     &           &    8 &  4.09e-09 &  1.17e-16 &      \\ 
     &           &    9 &  4.08e-09 &  1.17e-16 &      \\ 
     &           &   10 &  4.08e-09 &  1.17e-16 &      \\ 
 494 &  4.93e+03 &   10 &           &           & iters  \\ 
 \hdashline 
     &           &    1 &  6.73e-09 &  8.65e-10 &      \\ 
     &           &    2 &  6.72e-09 &  1.92e-16 &      \\ 
     &           &    3 &  6.71e-09 &  1.92e-16 &      \\ 
     &           &    4 &  6.70e-09 &  1.91e-16 &      \\ 
     &           &    5 &  6.69e-09 &  1.91e-16 &      \\ 
     &           &    6 &  7.10e-08 &  1.91e-16 &      \\ 
     &           &    7 &  8.45e-08 &  2.03e-15 &      \\ 
     &           &    8 &  7.10e-08 &  2.41e-15 &      \\ 
     &           &    9 &  6.76e-09 &  2.03e-15 &      \\ 
     &           &   10 &  6.75e-09 &  1.93e-16 &      \\ 
 495 &  4.94e+03 &   10 &           &           & iters  \\ 
 \hdashline 
     &           &    1 &  4.51e-08 &  1.01e-08 &      \\ 
     &           &    2 &  3.27e-08 &  1.29e-15 &      \\ 
     &           &    3 &  4.51e-08 &  9.33e-16 &      \\ 
     &           &    4 &  3.27e-08 &  1.29e-15 &      \\ 
     &           &    5 &  4.50e-08 &  9.33e-16 &      \\ 
     &           &    6 &  3.27e-08 &  1.29e-15 &      \\ 
     &           &    7 &  3.27e-08 &  9.34e-16 &      \\ 
     &           &    8 &  4.51e-08 &  9.32e-16 &      \\ 
     &           &    9 &  3.27e-08 &  1.29e-15 &      \\ 
     &           &   10 &  4.51e-08 &  9.33e-16 &      \\ 
 496 &  4.95e+03 &   10 &           &           & iters  \\ 
 \hdashline 
     &           &    1 &  7.09e-08 &  1.44e-09 &      \\ 
     &           &    2 &  6.89e-09 &  2.02e-15 &      \\ 
     &           &    3 &  6.88e-09 &  1.97e-16 &      \\ 
     &           &    4 &  6.87e-09 &  1.96e-16 &      \\ 
     &           &    5 &  6.86e-09 &  1.96e-16 &      \\ 
     &           &    6 &  6.85e-09 &  1.96e-16 &      \\ 
     &           &    7 &  6.84e-09 &  1.95e-16 &      \\ 
     &           &    8 &  6.83e-09 &  1.95e-16 &      \\ 
     &           &    9 &  6.82e-09 &  1.95e-16 &      \\ 
     &           &   10 &  7.09e-08 &  1.95e-16 &      \\ 
 497 &  4.96e+03 &   10 &           &           & iters  \\ 
 \hdashline 
     &           &    1 &  5.55e-08 &  4.07e-09 &      \\ 
     &           &    2 &  2.22e-08 &  1.58e-15 &      \\ 
     &           &    3 &  2.22e-08 &  6.35e-16 &      \\ 
     &           &    4 &  2.22e-08 &  6.34e-16 &      \\ 
     &           &    5 &  5.56e-08 &  6.33e-16 &      \\ 
     &           &    6 &  2.22e-08 &  1.59e-15 &      \\ 
     &           &    7 &  2.22e-08 &  6.34e-16 &      \\ 
     &           &    8 &  5.55e-08 &  6.33e-16 &      \\ 
     &           &    9 &  2.22e-08 &  1.58e-15 &      \\ 
     &           &   10 &  2.22e-08 &  6.35e-16 &      \\ 
 498 &  4.97e+03 &   10 &           &           & iters  \\ 
 \hdashline 
     &           &    1 &  6.94e-08 &  1.85e-08 &      \\ 
     &           &    2 &  8.36e-09 &  1.98e-15 &      \\ 
     &           &    3 &  8.35e-09 &  2.39e-16 &      \\ 
     &           &    4 &  8.34e-09 &  2.38e-16 &      \\ 
     &           &    5 &  8.32e-09 &  2.38e-16 &      \\ 
     &           &    6 &  8.31e-09 &  2.38e-16 &      \\ 
     &           &    7 &  6.94e-08 &  2.37e-16 &      \\ 
     &           &    8 &  8.61e-08 &  1.98e-15 &      \\ 
     &           &    9 &  6.94e-08 &  2.46e-15 &      \\ 
     &           &   10 &  8.38e-09 &  1.98e-15 &      \\ 
 499 &  4.98e+03 &   10 &           &           & iters  \\ 
 \hdashline 
     &           &    1 &  5.13e-08 &  1.87e-08 &      \\ 
     &           &    2 &  2.64e-08 &  1.46e-15 &      \\ 
     &           &    3 &  5.13e-08 &  7.54e-16 &      \\ 
     &           &    4 &  2.65e-08 &  1.46e-15 &      \\ 
     &           &    5 &  2.64e-08 &  7.55e-16 &      \\ 
     &           &    6 &  5.13e-08 &  7.54e-16 &      \\ 
     &           &    7 &  2.65e-08 &  1.46e-15 &      \\ 
     &           &    8 &  2.64e-08 &  7.55e-16 &      \\ 
     &           &    9 &  5.13e-08 &  7.54e-16 &      \\ 
     &           &   10 &  2.65e-08 &  1.46e-15 &      \\ 
 500 &  4.99e+03 &   10 &           &           & iters  \\ 
 \hdashline 
     &           &    1 &  3.25e-08 &  1.63e-08 &      \\ 
     &           &    2 &  4.52e-08 &  9.28e-16 &      \\ 
     &           &    3 &  3.25e-08 &  1.29e-15 &      \\ 
     &           &    4 &  3.25e-08 &  9.28e-16 &      \\ 
     &           &    5 &  4.53e-08 &  9.27e-16 &      \\ 
     &           &    6 &  3.25e-08 &  1.29e-15 &      \\ 
     &           &    7 &  4.52e-08 &  9.27e-16 &      \\ 
     &           &    8 &  3.25e-08 &  1.29e-15 &      \\ 
     &           &    9 &  4.52e-08 &  9.28e-16 &      \\ 
     &           &   10 &  3.25e-08 &  1.29e-15 &      \\ 
 501 &  5.00e+03 &   10 &           &           & iters  \\ 
 \hdashline 
     &           &    1 &  9.84e-09 &  5.16e-09 &      \\ 
     &           &    2 &  9.83e-09 &  2.81e-16 &      \\ 
     &           &    3 &  9.81e-09 &  2.80e-16 &      \\ 
     &           &    4 &  9.80e-09 &  2.80e-16 &      \\ 
     &           &    5 &  6.79e-08 &  2.80e-16 &      \\ 
     &           &    6 &  9.88e-09 &  1.94e-15 &      \\ 
     &           &    7 &  9.87e-09 &  2.82e-16 &      \\ 
     &           &    8 &  9.85e-09 &  2.82e-16 &      \\ 
     &           &    9 &  9.84e-09 &  2.81e-16 &      \\ 
     &           &   10 &  9.83e-09 &  2.81e-16 &      \\ 
 502 &  5.01e+03 &   10 &           &           & iters  \\ 
 \hdashline 
     &           &    1 &  4.02e-08 &  1.95e-08 &      \\ 
     &           &    2 &  3.76e-08 &  1.15e-15 &      \\ 
     &           &    3 &  4.02e-08 &  1.07e-15 &      \\ 
     &           &    4 &  3.76e-08 &  1.15e-15 &      \\ 
     &           &    5 &  4.02e-08 &  1.07e-15 &      \\ 
     &           &    6 &  3.76e-08 &  1.15e-15 &      \\ 
     &           &    7 &  4.02e-08 &  1.07e-15 &      \\ 
     &           &    8 &  3.76e-08 &  1.15e-15 &      \\ 
     &           &    9 &  4.02e-08 &  1.07e-15 &      \\ 
     &           &   10 &  3.76e-08 &  1.15e-15 &      \\ 
 503 &  5.02e+03 &   10 &           &           & iters  \\ 
 \hdashline 
     &           &    1 &  4.91e-08 &  1.86e-08 &      \\ 
     &           &    2 &  2.86e-08 &  1.40e-15 &      \\ 
     &           &    3 &  2.86e-08 &  8.17e-16 &      \\ 
     &           &    4 &  4.91e-08 &  8.16e-16 &      \\ 
     &           &    5 &  2.86e-08 &  1.40e-15 &      \\ 
     &           &    6 &  4.91e-08 &  8.17e-16 &      \\ 
     &           &    7 &  2.87e-08 &  1.40e-15 &      \\ 
     &           &    8 &  2.86e-08 &  8.18e-16 &      \\ 
     &           &    9 &  4.91e-08 &  8.17e-16 &      \\ 
     &           &   10 &  2.86e-08 &  1.40e-15 &      \\ 
 504 &  5.03e+03 &   10 &           &           & iters  \\ 
 \hdashline 
     &           &    1 &  4.35e-08 &  1.14e-08 &      \\ 
     &           &    2 &  3.43e-08 &  1.24e-15 &      \\ 
     &           &    3 &  4.34e-08 &  9.79e-16 &      \\ 
     &           &    4 &  3.43e-08 &  1.24e-15 &      \\ 
     &           &    5 &  4.34e-08 &  9.79e-16 &      \\ 
     &           &    6 &  3.43e-08 &  1.24e-15 &      \\ 
     &           &    7 &  4.34e-08 &  9.79e-16 &      \\ 
     &           &    8 &  3.43e-08 &  1.24e-15 &      \\ 
     &           &    9 &  4.34e-08 &  9.80e-16 &      \\ 
     &           &   10 &  3.43e-08 &  1.24e-15 &      \\ 
 505 &  5.04e+03 &   10 &           &           & iters  \\ 
 \hdashline 
     &           &    1 &  9.84e-09 &  2.87e-08 &      \\ 
     &           &    2 &  9.83e-09 &  2.81e-16 &      \\ 
     &           &    3 &  9.82e-09 &  2.81e-16 &      \\ 
     &           &    4 &  9.80e-09 &  2.80e-16 &      \\ 
     &           &    5 &  9.79e-09 &  2.80e-16 &      \\ 
     &           &    6 &  9.77e-09 &  2.79e-16 &      \\ 
     &           &    7 &  6.79e-08 &  2.79e-16 &      \\ 
     &           &    8 &  4.87e-08 &  1.94e-15 &      \\ 
     &           &    9 &  9.79e-09 &  1.39e-15 &      \\ 
     &           &   10 &  9.77e-09 &  2.79e-16 &      \\ 
 506 &  5.05e+03 &   10 &           &           & iters  \\ 
 \hdashline 
     &           &    1 &  3.09e-08 &  2.85e-08 &      \\ 
     &           &    2 &  3.08e-08 &  8.82e-16 &      \\ 
     &           &    3 &  4.69e-08 &  8.80e-16 &      \\ 
     &           &    4 &  3.09e-08 &  1.34e-15 &      \\ 
     &           &    5 &  4.69e-08 &  8.81e-16 &      \\ 
     &           &    6 &  3.09e-08 &  1.34e-15 &      \\ 
     &           &    7 &  3.08e-08 &  8.82e-16 &      \\ 
     &           &    8 &  4.69e-08 &  8.80e-16 &      \\ 
     &           &    9 &  3.09e-08 &  1.34e-15 &      \\ 
     &           &   10 &  4.69e-08 &  8.81e-16 &      \\ 
 507 &  5.06e+03 &   10 &           &           & iters  \\ 
 \hdashline 
     &           &    1 &  5.32e-08 &  3.11e-08 &      \\ 
     &           &    2 &  1.43e-08 &  1.52e-15 &      \\ 
     &           &    3 &  6.34e-08 &  4.09e-16 &      \\ 
     &           &    4 &  5.32e-08 &  1.81e-15 &      \\ 
     &           &    5 &  1.43e-08 &  1.52e-15 &      \\ 
     &           &    6 &  6.34e-08 &  4.09e-16 &      \\ 
     &           &    7 &  1.44e-08 &  1.81e-15 &      \\ 
     &           &    8 &  1.44e-08 &  4.11e-16 &      \\ 
     &           &    9 &  1.43e-08 &  4.10e-16 &      \\ 
     &           &   10 &  1.43e-08 &  4.10e-16 &      \\ 
 508 &  5.07e+03 &   10 &           &           & iters  \\ 
 \hdashline 
     &           &    1 &  6.00e-09 &  2.55e-09 &      \\ 
     &           &    2 &  5.99e-09 &  1.71e-16 &      \\ 
     &           &    3 &  5.98e-09 &  1.71e-16 &      \\ 
     &           &    4 &  5.97e-09 &  1.71e-16 &      \\ 
     &           &    5 &  5.96e-09 &  1.70e-16 &      \\ 
     &           &    6 &  5.95e-09 &  1.70e-16 &      \\ 
     &           &    7 &  3.29e-08 &  1.70e-16 &      \\ 
     &           &    8 &  5.99e-09 &  9.39e-16 &      \\ 
     &           &    9 &  5.98e-09 &  1.71e-16 &      \\ 
     &           &   10 &  5.97e-09 &  1.71e-16 &      \\ 
 509 &  5.08e+03 &   10 &           &           & iters  \\ 
 \hdashline 
     &           &    1 &  2.29e-08 &  3.23e-08 &      \\ 
     &           &    2 &  2.29e-08 &  6.54e-16 &      \\ 
     &           &    3 &  5.48e-08 &  6.53e-16 &      \\ 
     &           &    4 &  2.29e-08 &  1.57e-15 &      \\ 
     &           &    5 &  2.29e-08 &  6.54e-16 &      \\ 
     &           &    6 &  5.48e-08 &  6.54e-16 &      \\ 
     &           &    7 &  2.29e-08 &  1.56e-15 &      \\ 
     &           &    8 &  2.29e-08 &  6.55e-16 &      \\ 
     &           &    9 &  5.48e-08 &  6.54e-16 &      \\ 
     &           &   10 &  2.30e-08 &  1.56e-15 &      \\ 
 510 &  5.09e+03 &   10 &           &           & iters  \\ 
 \hdashline 
     &           &    1 &  2.89e-08 &  9.35e-09 &      \\ 
     &           &    2 &  4.88e-08 &  8.26e-16 &      \\ 
     &           &    3 &  2.90e-08 &  1.39e-15 &      \\ 
     &           &    4 &  2.89e-08 &  8.27e-16 &      \\ 
     &           &    5 &  4.88e-08 &  8.26e-16 &      \\ 
     &           &    6 &  2.90e-08 &  1.39e-15 &      \\ 
     &           &    7 &  2.89e-08 &  8.26e-16 &      \\ 
     &           &    8 &  4.88e-08 &  8.25e-16 &      \\ 
     &           &    9 &  2.89e-08 &  1.39e-15 &      \\ 
     &           &   10 &  4.88e-08 &  8.26e-16 &      \\ 
 511 &  5.10e+03 &   10 &           &           & iters  \\ 
 \hdashline 
     &           &    1 &  2.71e-08 &  6.28e-09 &      \\ 
     &           &    2 &  2.71e-08 &  7.74e-16 &      \\ 
     &           &    3 &  1.18e-08 &  7.73e-16 &      \\ 
     &           &    4 &  1.18e-08 &  3.37e-16 &      \\ 
     &           &    5 &  1.18e-08 &  3.37e-16 &      \\ 
     &           &    6 &  2.71e-08 &  3.36e-16 &      \\ 
     &           &    7 &  1.18e-08 &  7.73e-16 &      \\ 
     &           &    8 &  1.18e-08 &  3.37e-16 &      \\ 
     &           &    9 &  2.71e-08 &  3.36e-16 &      \\ 
     &           &   10 &  1.18e-08 &  7.73e-16 &      \\ 
 512 &  5.11e+03 &   10 &           &           & iters  \\ 
 \hdashline 
     &           &    1 &  1.56e-08 &  1.86e-08 &      \\ 
     &           &    2 &  1.56e-08 &  4.45e-16 &      \\ 
     &           &    3 &  1.55e-08 &  4.44e-16 &      \\ 
     &           &    4 &  6.22e-08 &  4.44e-16 &      \\ 
     &           &    5 &  1.56e-08 &  1.77e-15 &      \\ 
     &           &    6 &  1.56e-08 &  4.45e-16 &      \\ 
     &           &    7 &  1.56e-08 &  4.45e-16 &      \\ 
     &           &    8 &  1.55e-08 &  4.44e-16 &      \\ 
     &           &    9 &  6.22e-08 &  4.44e-16 &      \\ 
     &           &   10 &  1.56e-08 &  1.77e-15 &      \\ 
 513 &  5.12e+03 &   10 &           &           & iters  \\ 
 \hdashline 
     &           &    1 &  4.02e-08 &  4.12e-08 &      \\ 
     &           &    2 &  3.76e-08 &  1.15e-15 &      \\ 
     &           &    3 &  4.02e-08 &  1.07e-15 &      \\ 
     &           &    4 &  3.76e-08 &  1.15e-15 &      \\ 
     &           &    5 &  4.02e-08 &  1.07e-15 &      \\ 
     &           &    6 &  3.76e-08 &  1.15e-15 &      \\ 
     &           &    7 &  4.02e-08 &  1.07e-15 &      \\ 
     &           &    8 &  3.76e-08 &  1.15e-15 &      \\ 
     &           &    9 &  4.02e-08 &  1.07e-15 &      \\ 
     &           &   10 &  3.76e-08 &  1.15e-15 &      \\ 
 514 &  5.13e+03 &   10 &           &           & iters  \\ 
 \hdashline 
     &           &    1 &  1.78e-08 &  3.37e-08 &      \\ 
     &           &    2 &  1.78e-08 &  5.09e-16 &      \\ 
     &           &    3 &  5.99e-08 &  5.09e-16 &      \\ 
     &           &    4 &  1.79e-08 &  1.71e-15 &      \\ 
     &           &    5 &  1.79e-08 &  5.10e-16 &      \\ 
     &           &    6 &  1.78e-08 &  5.10e-16 &      \\ 
     &           &    7 &  5.99e-08 &  5.09e-16 &      \\ 
     &           &    8 &  1.79e-08 &  1.71e-15 &      \\ 
     &           &    9 &  1.79e-08 &  5.11e-16 &      \\ 
     &           &   10 &  1.78e-08 &  5.10e-16 &      \\ 
 515 &  5.14e+03 &   10 &           &           & iters  \\ 
 \hdashline 
     &           &    1 &  8.63e-08 &  1.27e-08 &      \\ 
     &           &    2 &  6.92e-08 &  2.46e-15 &      \\ 
     &           &    3 &  8.56e-09 &  1.98e-15 &      \\ 
     &           &    4 &  8.55e-09 &  2.44e-16 &      \\ 
     &           &    5 &  8.54e-09 &  2.44e-16 &      \\ 
     &           &    6 &  8.53e-09 &  2.44e-16 &      \\ 
     &           &    7 &  8.52e-09 &  2.43e-16 &      \\ 
     &           &    8 &  8.50e-09 &  2.43e-16 &      \\ 
     &           &    9 &  6.92e-08 &  2.43e-16 &      \\ 
     &           &   10 &  8.63e-08 &  1.97e-15 &      \\ 
 516 &  5.15e+03 &   10 &           &           & iters  \\ 
 \hdashline 
     &           &    1 &  4.67e-08 &  1.40e-08 &      \\ 
     &           &    2 &  4.66e-08 &  1.33e-15 &      \\ 
     &           &    3 &  3.11e-08 &  1.33e-15 &      \\ 
     &           &    4 &  4.66e-08 &  8.89e-16 &      \\ 
     &           &    5 &  3.12e-08 &  1.33e-15 &      \\ 
     &           &    6 &  4.66e-08 &  8.89e-16 &      \\ 
     &           &    7 &  3.12e-08 &  1.33e-15 &      \\ 
     &           &    8 &  3.11e-08 &  8.90e-16 &      \\ 
     &           &    9 &  4.66e-08 &  8.89e-16 &      \\ 
     &           &   10 &  3.12e-08 &  1.33e-15 &      \\ 
 517 &  5.16e+03 &   10 &           &           & iters  \\ 
 \hdashline 
     &           &    1 &  6.19e-09 &  2.38e-09 &      \\ 
     &           &    2 &  6.18e-09 &  1.77e-16 &      \\ 
     &           &    3 &  6.17e-09 &  1.76e-16 &      \\ 
     &           &    4 &  6.16e-09 &  1.76e-16 &      \\ 
     &           &    5 &  7.15e-08 &  1.76e-16 &      \\ 
     &           &    6 &  8.39e-08 &  2.04e-15 &      \\ 
     &           &    7 &  7.16e-08 &  2.40e-15 &      \\ 
     &           &    8 &  6.24e-09 &  2.04e-15 &      \\ 
     &           &    9 &  6.23e-09 &  1.78e-16 &      \\ 
     &           &   10 &  6.22e-09 &  1.78e-16 &      \\ 
 518 &  5.17e+03 &   10 &           &           & iters  \\ 
 \hdashline 
     &           &    1 &  1.97e-08 &  2.53e-08 &      \\ 
     &           &    2 &  5.81e-08 &  5.61e-16 &      \\ 
     &           &    3 &  1.97e-08 &  1.66e-15 &      \\ 
     &           &    4 &  1.97e-08 &  5.62e-16 &      \\ 
     &           &    5 &  1.97e-08 &  5.62e-16 &      \\ 
     &           &    6 &  5.81e-08 &  5.61e-16 &      \\ 
     &           &    7 &  1.97e-08 &  1.66e-15 &      \\ 
     &           &    8 &  1.97e-08 &  5.62e-16 &      \\ 
     &           &    9 &  1.97e-08 &  5.62e-16 &      \\ 
     &           &   10 &  5.81e-08 &  5.61e-16 &      \\ 
 519 &  5.18e+03 &   10 &           &           & iters  \\ 
 \hdashline 
     &           &    1 &  4.41e-08 &  3.57e-10 &      \\ 
     &           &    2 &  3.36e-08 &  1.26e-15 &      \\ 
     &           &    3 &  4.41e-08 &  9.59e-16 &      \\ 
     &           &    4 &  3.36e-08 &  1.26e-15 &      \\ 
     &           &    5 &  3.36e-08 &  9.59e-16 &      \\ 
     &           &    6 &  4.42e-08 &  9.58e-16 &      \\ 
     &           &    7 &  3.36e-08 &  1.26e-15 &      \\ 
     &           &    8 &  4.42e-08 &  9.59e-16 &      \\ 
     &           &    9 &  3.36e-08 &  1.26e-15 &      \\ 
     &           &   10 &  4.41e-08 &  9.59e-16 &      \\ 
 520 &  5.19e+03 &   10 &           &           & iters  \\ 
 \hdashline 
     &           &    1 &  3.05e-08 &  7.36e-09 &      \\ 
     &           &    2 &  3.05e-08 &  8.71e-16 &      \\ 
     &           &    3 &  3.04e-08 &  8.70e-16 &      \\ 
     &           &    4 &  4.73e-08 &  8.69e-16 &      \\ 
     &           &    5 &  3.05e-08 &  1.35e-15 &      \\ 
     &           &    6 &  3.04e-08 &  8.70e-16 &      \\ 
     &           &    7 &  4.73e-08 &  8.68e-16 &      \\ 
     &           &    8 &  3.04e-08 &  1.35e-15 &      \\ 
     &           &    9 &  4.73e-08 &  8.69e-16 &      \\ 
     &           &   10 &  3.05e-08 &  1.35e-15 &      \\ 
 521 &  5.20e+03 &   10 &           &           & iters  \\ 
 \hdashline 
     &           &    1 &  7.90e-09 &  3.08e-08 &      \\ 
     &           &    2 &  7.89e-09 &  2.26e-16 &      \\ 
     &           &    3 &  7.88e-09 &  2.25e-16 &      \\ 
     &           &    4 &  6.98e-08 &  2.25e-16 &      \\ 
     &           &    5 &  8.57e-08 &  1.99e-15 &      \\ 
     &           &    6 &  6.98e-08 &  2.44e-15 &      \\ 
     &           &    7 &  7.95e-09 &  1.99e-15 &      \\ 
     &           &    8 &  7.93e-09 &  2.27e-16 &      \\ 
     &           &    9 &  7.92e-09 &  2.26e-16 &      \\ 
     &           &   10 &  7.91e-09 &  2.26e-16 &      \\ 
 522 &  5.21e+03 &   10 &           &           & iters  \\ 
 \hdashline 
     &           &    1 &  1.80e-08 &  2.97e-08 &      \\ 
     &           &    2 &  1.80e-08 &  5.14e-16 &      \\ 
     &           &    3 &  1.80e-08 &  5.13e-16 &      \\ 
     &           &    4 &  5.98e-08 &  5.13e-16 &      \\ 
     &           &    5 &  1.80e-08 &  1.71e-15 &      \\ 
     &           &    6 &  1.80e-08 &  5.14e-16 &      \\ 
     &           &    7 &  1.80e-08 &  5.14e-16 &      \\ 
     &           &    8 &  1.79e-08 &  5.13e-16 &      \\ 
     &           &    9 &  5.98e-08 &  5.12e-16 &      \\ 
     &           &   10 &  1.80e-08 &  1.71e-15 &      \\ 
 523 &  5.22e+03 &   10 &           &           & iters  \\ 
 \hdashline 
     &           &    1 &  4.22e-09 &  5.26e-09 &      \\ 
     &           &    2 &  4.22e-09 &  1.21e-16 &      \\ 
     &           &    3 &  4.21e-09 &  1.20e-16 &      \\ 
     &           &    4 &  4.21e-09 &  1.20e-16 &      \\ 
     &           &    5 &  4.20e-09 &  1.20e-16 &      \\ 
     &           &    6 &  4.19e-09 &  1.20e-16 &      \\ 
     &           &    7 &  4.19e-09 &  1.20e-16 &      \\ 
     &           &    8 &  4.18e-09 &  1.20e-16 &      \\ 
     &           &    9 &  4.18e-09 &  1.19e-16 &      \\ 
     &           &   10 &  4.17e-09 &  1.19e-16 &      \\ 
 524 &  5.23e+03 &   10 &           &           & iters  \\ 
 \hdashline 
     &           &    1 &  4.77e-08 &  2.93e-08 &      \\ 
     &           &    2 &  3.01e-08 &  1.36e-15 &      \\ 
     &           &    3 &  4.76e-08 &  8.59e-16 &      \\ 
     &           &    4 &  3.01e-08 &  1.36e-15 &      \\ 
     &           &    5 &  4.76e-08 &  8.59e-16 &      \\ 
     &           &    6 &  3.01e-08 &  1.36e-15 &      \\ 
     &           &    7 &  3.01e-08 &  8.60e-16 &      \\ 
     &           &    8 &  4.76e-08 &  8.59e-16 &      \\ 
     &           &    9 &  3.01e-08 &  1.36e-15 &      \\ 
     &           &   10 &  4.76e-08 &  8.60e-16 &      \\ 
 525 &  5.24e+03 &   10 &           &           & iters  \\ 
 \hdashline 
     &           &    1 &  4.46e-08 &  2.04e-08 &      \\ 
     &           &    2 &  5.68e-09 &  1.27e-15 &      \\ 
     &           &    3 &  5.67e-09 &  1.62e-16 &      \\ 
     &           &    4 &  5.67e-09 &  1.62e-16 &      \\ 
     &           &    5 &  5.66e-09 &  1.62e-16 &      \\ 
     &           &    6 &  5.65e-09 &  1.61e-16 &      \\ 
     &           &    7 &  5.64e-09 &  1.61e-16 &      \\ 
     &           &    8 &  5.63e-09 &  1.61e-16 &      \\ 
     &           &    9 &  7.21e-08 &  1.61e-16 &      \\ 
     &           &   10 &  4.46e-08 &  2.06e-15 &      \\ 
 526 &  5.25e+03 &   10 &           &           & iters  \\ 
 \hdashline 
     &           &    1 &  3.38e-08 &  4.30e-08 &      \\ 
     &           &    2 &  4.39e-08 &  9.65e-16 &      \\ 
     &           &    3 &  3.38e-08 &  1.25e-15 &      \\ 
     &           &    4 &  3.38e-08 &  9.66e-16 &      \\ 
     &           &    5 &  4.39e-08 &  9.64e-16 &      \\ 
     &           &    6 &  3.38e-08 &  1.25e-15 &      \\ 
     &           &    7 &  4.39e-08 &  9.65e-16 &      \\ 
     &           &    8 &  3.38e-08 &  1.25e-15 &      \\ 
     &           &    9 &  4.39e-08 &  9.65e-16 &      \\ 
     &           &   10 &  3.38e-08 &  1.25e-15 &      \\ 
 527 &  5.26e+03 &   10 &           &           & iters  \\ 
 \hdashline 
     &           &    1 &  2.20e-08 &  4.28e-08 &      \\ 
     &           &    2 &  2.19e-08 &  6.27e-16 &      \\ 
     &           &    3 &  5.58e-08 &  6.26e-16 &      \\ 
     &           &    4 &  2.20e-08 &  1.59e-15 &      \\ 
     &           &    5 &  2.19e-08 &  6.27e-16 &      \\ 
     &           &    6 &  2.19e-08 &  6.26e-16 &      \\ 
     &           &    7 &  5.58e-08 &  6.26e-16 &      \\ 
     &           &    8 &  2.20e-08 &  1.59e-15 &      \\ 
     &           &    9 &  2.19e-08 &  6.27e-16 &      \\ 
     &           &   10 &  2.19e-08 &  6.26e-16 &      \\ 
 528 &  5.27e+03 &   10 &           &           & iters  \\ 
 \hdashline 
     &           &    1 &  1.30e-09 &  1.63e-08 &      \\ 
     &           &    2 &  1.30e-09 &  3.72e-17 &      \\ 
     &           &    3 &  1.30e-09 &  3.71e-17 &      \\ 
     &           &    4 &  1.30e-09 &  3.71e-17 &      \\ 
     &           &    5 &  1.29e-09 &  3.70e-17 &      \\ 
     &           &    6 &  1.29e-09 &  3.70e-17 &      \\ 
     &           &    7 &  1.29e-09 &  3.69e-17 &      \\ 
     &           &    8 &  1.29e-09 &  3.68e-17 &      \\ 
     &           &    9 &  1.29e-09 &  3.68e-17 &      \\ 
     &           &   10 &  1.29e-09 &  3.67e-17 &      \\ 
 529 &  5.28e+03 &   10 &           &           & iters  \\ 
 \hdashline 
     &           &    1 &  1.69e-08 &  1.05e-08 &      \\ 
     &           &    2 &  1.68e-08 &  4.81e-16 &      \\ 
     &           &    3 &  6.09e-08 &  4.80e-16 &      \\ 
     &           &    4 &  1.69e-08 &  1.74e-15 &      \\ 
     &           &    5 &  1.69e-08 &  4.82e-16 &      \\ 
     &           &    6 &  1.68e-08 &  4.82e-16 &      \\ 
     &           &    7 &  1.68e-08 &  4.81e-16 &      \\ 
     &           &    8 &  6.09e-08 &  4.80e-16 &      \\ 
     &           &    9 &  1.69e-08 &  1.74e-15 &      \\ 
     &           &   10 &  1.69e-08 &  4.82e-16 &      \\ 
 530 &  5.29e+03 &   10 &           &           & iters  \\ 
 \hdashline 
     &           &    1 &  1.94e-08 &  2.16e-08 &      \\ 
     &           &    2 &  1.94e-08 &  5.54e-16 &      \\ 
     &           &    3 &  1.94e-08 &  5.53e-16 &      \\ 
     &           &    4 &  5.84e-08 &  5.52e-16 &      \\ 
     &           &    5 &  1.94e-08 &  1.67e-15 &      \\ 
     &           &    6 &  1.94e-08 &  5.54e-16 &      \\ 
     &           &    7 &  1.94e-08 &  5.53e-16 &      \\ 
     &           &    8 &  5.84e-08 &  5.52e-16 &      \\ 
     &           &    9 &  1.94e-08 &  1.67e-15 &      \\ 
     &           &   10 &  1.94e-08 &  5.54e-16 &      \\ 
 531 &  5.30e+03 &   10 &           &           & iters  \\ 
 \hdashline 
     &           &    1 &  6.99e-08 &  2.65e-08 &      \\ 
     &           &    2 &  7.93e-09 &  1.99e-15 &      \\ 
     &           &    3 &  7.91e-09 &  2.26e-16 &      \\ 
     &           &    4 &  7.90e-09 &  2.26e-16 &      \\ 
     &           &    5 &  7.89e-09 &  2.26e-16 &      \\ 
     &           &    6 &  7.88e-09 &  2.25e-16 &      \\ 
     &           &    7 &  7.87e-09 &  2.25e-16 &      \\ 
     &           &    8 &  7.86e-09 &  2.25e-16 &      \\ 
     &           &    9 &  7.85e-09 &  2.24e-16 &      \\ 
     &           &   10 &  6.99e-08 &  2.24e-16 &      \\ 
 532 &  5.31e+03 &   10 &           &           & iters  \\ 
 \hdashline 
     &           &    1 &  1.27e-08 &  3.82e-09 &      \\ 
     &           &    2 &  1.27e-08 &  3.64e-16 &      \\ 
     &           &    3 &  1.27e-08 &  3.63e-16 &      \\ 
     &           &    4 &  1.27e-08 &  3.63e-16 &      \\ 
     &           &    5 &  1.27e-08 &  3.62e-16 &      \\ 
     &           &    6 &  1.26e-08 &  3.61e-16 &      \\ 
     &           &    7 &  1.26e-08 &  3.61e-16 &      \\ 
     &           &    8 &  1.26e-08 &  3.60e-16 &      \\ 
     &           &    9 &  1.26e-08 &  3.60e-16 &      \\ 
     &           &   10 &  6.51e-08 &  3.59e-16 &      \\ 
 533 &  5.32e+03 &   10 &           &           & iters  \\ 
 \hdashline 
     &           &    1 &  4.38e-08 &  6.49e-09 &      \\ 
     &           &    2 &  3.40e-08 &  1.25e-15 &      \\ 
     &           &    3 &  4.38e-08 &  9.70e-16 &      \\ 
     &           &    4 &  3.40e-08 &  1.25e-15 &      \\ 
     &           &    5 &  3.39e-08 &  9.70e-16 &      \\ 
     &           &    6 &  4.38e-08 &  9.69e-16 &      \\ 
     &           &    7 &  3.40e-08 &  1.25e-15 &      \\ 
     &           &    8 &  4.38e-08 &  9.69e-16 &      \\ 
     &           &    9 &  3.40e-08 &  1.25e-15 &      \\ 
     &           &   10 &  4.38e-08 &  9.70e-16 &      \\ 
 534 &  5.33e+03 &   10 &           &           & iters  \\ 
 \hdashline 
     &           &    1 &  2.31e-08 &  3.38e-10 &      \\ 
     &           &    2 &  2.31e-08 &  6.61e-16 &      \\ 
     &           &    3 &  5.46e-08 &  6.60e-16 &      \\ 
     &           &    4 &  2.32e-08 &  1.56e-15 &      \\ 
     &           &    5 &  2.31e-08 &  6.61e-16 &      \\ 
     &           &    6 &  5.46e-08 &  6.60e-16 &      \\ 
     &           &    7 &  2.32e-08 &  1.56e-15 &      \\ 
     &           &    8 &  2.31e-08 &  6.61e-16 &      \\ 
     &           &    9 &  2.31e-08 &  6.60e-16 &      \\ 
     &           &   10 &  5.46e-08 &  6.59e-16 &      \\ 
 535 &  5.34e+03 &   10 &           &           & iters  \\ 
 \hdashline 
     &           &    1 &  3.97e-08 &  5.31e-09 &      \\ 
     &           &    2 &  3.80e-08 &  1.13e-15 &      \\ 
     &           &    3 &  3.97e-08 &  1.09e-15 &      \\ 
     &           &    4 &  3.80e-08 &  1.13e-15 &      \\ 
     &           &    5 &  3.97e-08 &  1.09e-15 &      \\ 
     &           &    6 &  3.80e-08 &  1.13e-15 &      \\ 
     &           &    7 &  3.97e-08 &  1.09e-15 &      \\ 
     &           &    8 &  3.80e-08 &  1.13e-15 &      \\ 
     &           &    9 &  3.97e-08 &  1.09e-15 &      \\ 
     &           &   10 &  3.80e-08 &  1.13e-15 &      \\ 
 536 &  5.35e+03 &   10 &           &           & iters  \\ 
 \hdashline 
     &           &    1 &  6.57e-09 &  5.54e-10 &      \\ 
     &           &    2 &  6.56e-09 &  1.87e-16 &      \\ 
     &           &    3 &  7.11e-08 &  1.87e-16 &      \\ 
     &           &    4 &  8.43e-08 &  2.03e-15 &      \\ 
     &           &    5 &  7.12e-08 &  2.41e-15 &      \\ 
     &           &    6 &  6.63e-09 &  2.03e-15 &      \\ 
     &           &    7 &  6.62e-09 &  1.89e-16 &      \\ 
     &           &    8 &  6.61e-09 &  1.89e-16 &      \\ 
     &           &    9 &  6.60e-09 &  1.89e-16 &      \\ 
     &           &   10 &  6.60e-09 &  1.89e-16 &      \\ 
 537 &  5.36e+03 &   10 &           &           & iters  \\ 
 \hdashline 
     &           &    1 &  4.20e-09 &  2.30e-08 &      \\ 
     &           &    2 &  4.20e-09 &  1.20e-16 &      \\ 
     &           &    3 &  4.19e-09 &  1.20e-16 &      \\ 
     &           &    4 &  4.18e-09 &  1.20e-16 &      \\ 
     &           &    5 &  4.18e-09 &  1.19e-16 &      \\ 
     &           &    6 &  4.17e-09 &  1.19e-16 &      \\ 
     &           &    7 &  4.17e-09 &  1.19e-16 &      \\ 
     &           &    8 &  4.16e-09 &  1.19e-16 &      \\ 
     &           &    9 &  4.15e-09 &  1.19e-16 &      \\ 
     &           &   10 &  4.15e-09 &  1.19e-16 &      \\ 
 538 &  5.37e+03 &   10 &           &           & iters  \\ 
 \hdashline 
     &           &    1 &  6.89e-08 &  1.55e-08 &      \\ 
     &           &    2 &  8.94e-09 &  1.97e-15 &      \\ 
     &           &    3 &  8.92e-09 &  2.55e-16 &      \\ 
     &           &    4 &  8.91e-09 &  2.55e-16 &      \\ 
     &           &    5 &  8.90e-09 &  2.54e-16 &      \\ 
     &           &    6 &  8.88e-09 &  2.54e-16 &      \\ 
     &           &    7 &  6.88e-08 &  2.54e-16 &      \\ 
     &           &    8 &  4.78e-08 &  1.96e-15 &      \\ 
     &           &    9 &  8.90e-09 &  1.36e-15 &      \\ 
     &           &   10 &  8.89e-09 &  2.54e-16 &      \\ 
 539 &  5.38e+03 &   10 &           &           & iters  \\ 
 \hdashline 
     &           &    1 &  3.86e-08 &  2.09e-08 &      \\ 
     &           &    2 &  3.92e-08 &  1.10e-15 &      \\ 
     &           &    3 &  3.86e-08 &  1.12e-15 &      \\ 
     &           &    4 &  3.92e-08 &  1.10e-15 &      \\ 
     &           &    5 &  3.86e-08 &  1.12e-15 &      \\ 
     &           &    6 &  3.92e-08 &  1.10e-15 &      \\ 
     &           &    7 &  3.86e-08 &  1.12e-15 &      \\ 
     &           &    8 &  3.92e-08 &  1.10e-15 &      \\ 
     &           &    9 &  3.86e-08 &  1.12e-15 &      \\ 
     &           &   10 &  3.92e-08 &  1.10e-15 &      \\ 
 540 &  5.39e+03 &   10 &           &           & iters  \\ 
 \hdashline 
     &           &    1 &  4.79e-08 &  1.78e-08 &      \\ 
     &           &    2 &  2.99e-08 &  1.37e-15 &      \\ 
     &           &    3 &  2.98e-08 &  8.52e-16 &      \\ 
     &           &    4 &  4.79e-08 &  8.51e-16 &      \\ 
     &           &    5 &  2.98e-08 &  1.37e-15 &      \\ 
     &           &    6 &  2.98e-08 &  8.52e-16 &      \\ 
     &           &    7 &  4.79e-08 &  8.50e-16 &      \\ 
     &           &    8 &  2.98e-08 &  1.37e-15 &      \\ 
     &           &    9 &  4.79e-08 &  8.51e-16 &      \\ 
     &           &   10 &  2.98e-08 &  1.37e-15 &      \\ 
 541 &  5.40e+03 &   10 &           &           & iters  \\ 
 \hdashline 
     &           &    1 &  3.81e-08 &  1.62e-08 &      \\ 
     &           &    2 &  3.80e-08 &  1.09e-15 &      \\ 
     &           &    3 &  3.97e-08 &  1.09e-15 &      \\ 
     &           &    4 &  3.80e-08 &  1.13e-15 &      \\ 
     &           &    5 &  3.97e-08 &  1.09e-15 &      \\ 
     &           &    6 &  3.80e-08 &  1.13e-15 &      \\ 
     &           &    7 &  3.97e-08 &  1.09e-15 &      \\ 
     &           &    8 &  3.80e-08 &  1.13e-15 &      \\ 
     &           &    9 &  3.97e-08 &  1.09e-15 &      \\ 
     &           &   10 &  3.80e-08 &  1.13e-15 &      \\ 
 542 &  5.41e+03 &   10 &           &           & iters  \\ 
 \hdashline 
     &           &    1 &  4.59e-08 &  2.83e-08 &      \\ 
     &           &    2 &  3.18e-08 &  1.31e-15 &      \\ 
     &           &    3 &  4.59e-08 &  9.09e-16 &      \\ 
     &           &    4 &  3.19e-08 &  1.31e-15 &      \\ 
     &           &    5 &  4.59e-08 &  9.10e-16 &      \\ 
     &           &    6 &  3.19e-08 &  1.31e-15 &      \\ 
     &           &    7 &  3.18e-08 &  9.10e-16 &      \\ 
     &           &    8 &  4.59e-08 &  9.09e-16 &      \\ 
     &           &    9 &  3.19e-08 &  1.31e-15 &      \\ 
     &           &   10 &  4.59e-08 &  9.09e-16 &      \\ 
 543 &  5.42e+03 &   10 &           &           & iters  \\ 
 \hdashline 
     &           &    1 &  7.03e-08 &  7.57e-09 &      \\ 
     &           &    2 &  7.45e-09 &  2.01e-15 &      \\ 
     &           &    3 &  7.44e-09 &  2.13e-16 &      \\ 
     &           &    4 &  7.43e-09 &  2.12e-16 &      \\ 
     &           &    5 &  7.42e-09 &  2.12e-16 &      \\ 
     &           &    6 &  7.41e-09 &  2.12e-16 &      \\ 
     &           &    7 &  7.40e-09 &  2.11e-16 &      \\ 
     &           &    8 &  7.39e-09 &  2.11e-16 &      \\ 
     &           &    9 &  7.03e-08 &  2.11e-16 &      \\ 
     &           &   10 &  8.52e-08 &  2.01e-15 &      \\ 
 544 &  5.43e+03 &   10 &           &           & iters  \\ 
 \hdashline 
     &           &    1 &  3.92e-08 &  3.71e-08 &      \\ 
     &           &    2 &  3.85e-08 &  1.12e-15 &      \\ 
     &           &    3 &  3.92e-08 &  1.10e-15 &      \\ 
     &           &    4 &  3.85e-08 &  1.12e-15 &      \\ 
     &           &    5 &  3.92e-08 &  1.10e-15 &      \\ 
     &           &    6 &  3.85e-08 &  1.12e-15 &      \\ 
     &           &    7 &  3.92e-08 &  1.10e-15 &      \\ 
     &           &    8 &  3.85e-08 &  1.12e-15 &      \\ 
     &           &    9 &  3.92e-08 &  1.10e-15 &      \\ 
     &           &   10 &  3.85e-08 &  1.12e-15 &      \\ 
 545 &  5.44e+03 &   10 &           &           & iters  \\ 
 \hdashline 
     &           &    1 &  1.40e-08 &  2.17e-08 &      \\ 
     &           &    2 &  1.40e-08 &  4.00e-16 &      \\ 
     &           &    3 &  6.37e-08 &  3.99e-16 &      \\ 
     &           &    4 &  1.41e-08 &  1.82e-15 &      \\ 
     &           &    5 &  1.40e-08 &  4.01e-16 &      \\ 
     &           &    6 &  1.40e-08 &  4.01e-16 &      \\ 
     &           &    7 &  1.40e-08 &  4.00e-16 &      \\ 
     &           &    8 &  6.37e-08 &  4.00e-16 &      \\ 
     &           &    9 &  1.41e-08 &  1.82e-15 &      \\ 
     &           &   10 &  1.41e-08 &  4.02e-16 &      \\ 
 546 &  5.45e+03 &   10 &           &           & iters  \\ 
 \hdashline 
     &           &    1 &  4.84e-09 &  2.79e-09 &      \\ 
     &           &    2 &  4.83e-09 &  1.38e-16 &      \\ 
     &           &    3 &  4.83e-09 &  1.38e-16 &      \\ 
     &           &    4 &  4.82e-09 &  1.38e-16 &      \\ 
     &           &    5 &  4.81e-09 &  1.38e-16 &      \\ 
     &           &    6 &  4.80e-09 &  1.37e-16 &      \\ 
     &           &    7 &  7.29e-08 &  1.37e-16 &      \\ 
     &           &    8 &  8.26e-08 &  2.08e-15 &      \\ 
     &           &    9 &  7.29e-08 &  2.36e-15 &      \\ 
     &           &   10 &  8.26e-08 &  2.08e-15 &      \\ 
 547 &  5.46e+03 &   10 &           &           & iters  \\ 
 \hdashline 
     &           &    1 &  4.99e-08 &  5.77e-09 &      \\ 
     &           &    2 &  4.99e-08 &  1.43e-15 &      \\ 
     &           &    3 &  2.79e-08 &  1.42e-15 &      \\ 
     &           &    4 &  2.78e-08 &  7.96e-16 &      \\ 
     &           &    5 &  4.99e-08 &  7.95e-16 &      \\ 
     &           &    6 &  2.79e-08 &  1.42e-15 &      \\ 
     &           &    7 &  2.78e-08 &  7.96e-16 &      \\ 
     &           &    8 &  4.99e-08 &  7.94e-16 &      \\ 
     &           &    9 &  2.79e-08 &  1.42e-15 &      \\ 
     &           &   10 &  2.78e-08 &  7.95e-16 &      \\ 
 548 &  5.47e+03 &   10 &           &           & iters  \\ 
 \hdashline 
     &           &    1 &  2.13e-08 &  5.51e-09 &      \\ 
     &           &    2 &  5.64e-08 &  6.08e-16 &      \\ 
     &           &    3 &  2.13e-08 &  1.61e-15 &      \\ 
     &           &    4 &  2.13e-08 &  6.09e-16 &      \\ 
     &           &    5 &  2.13e-08 &  6.08e-16 &      \\ 
     &           &    6 &  5.64e-08 &  6.07e-16 &      \\ 
     &           &    7 &  2.13e-08 &  1.61e-15 &      \\ 
     &           &    8 &  2.13e-08 &  6.09e-16 &      \\ 
     &           &    9 &  5.64e-08 &  6.08e-16 &      \\ 
     &           &   10 &  2.14e-08 &  1.61e-15 &      \\ 
 549 &  5.48e+03 &   10 &           &           & iters  \\ 
 \hdashline 
     &           &    1 &  1.97e-08 &  6.72e-09 &      \\ 
     &           &    2 &  1.97e-08 &  5.62e-16 &      \\ 
     &           &    3 &  5.81e-08 &  5.61e-16 &      \\ 
     &           &    4 &  1.97e-08 &  1.66e-15 &      \\ 
     &           &    5 &  1.97e-08 &  5.63e-16 &      \\ 
     &           &    6 &  1.97e-08 &  5.62e-16 &      \\ 
     &           &    7 &  5.81e-08 &  5.61e-16 &      \\ 
     &           &    8 &  1.97e-08 &  1.66e-15 &      \\ 
     &           &    9 &  1.97e-08 &  5.63e-16 &      \\ 
     &           &   10 &  1.97e-08 &  5.62e-16 &      \\ 
 550 &  5.49e+03 &   10 &           &           & iters  \\ 
 \hdashline 
     &           &    1 &  3.72e-08 &  1.84e-08 &      \\ 
     &           &    2 &  4.05e-08 &  1.06e-15 &      \\ 
     &           &    3 &  3.72e-08 &  1.16e-15 &      \\ 
     &           &    4 &  4.05e-08 &  1.06e-15 &      \\ 
     &           &    5 &  3.73e-08 &  1.16e-15 &      \\ 
     &           &    6 &  4.05e-08 &  1.06e-15 &      \\ 
     &           &    7 &  3.73e-08 &  1.16e-15 &      \\ 
     &           &    8 &  4.05e-08 &  1.06e-15 &      \\ 
     &           &    9 &  3.73e-08 &  1.16e-15 &      \\ 
     &           &   10 &  4.05e-08 &  1.06e-15 &      \\ 
 551 &  5.50e+03 &   10 &           &           & iters  \\ 
 \hdashline 
     &           &    1 &  9.91e-09 &  7.20e-09 &      \\ 
     &           &    2 &  9.90e-09 &  2.83e-16 &      \\ 
     &           &    3 &  6.78e-08 &  2.83e-16 &      \\ 
     &           &    4 &  9.98e-09 &  1.94e-15 &      \\ 
     &           &    5 &  9.97e-09 &  2.85e-16 &      \\ 
     &           &    6 &  9.95e-09 &  2.84e-16 &      \\ 
     &           &    7 &  9.94e-09 &  2.84e-16 &      \\ 
     &           &    8 &  9.92e-09 &  2.84e-16 &      \\ 
     &           &    9 &  9.91e-09 &  2.83e-16 &      \\ 
     &           &   10 &  9.90e-09 &  2.83e-16 &      \\ 
 552 &  5.51e+03 &   10 &           &           & iters  \\ 
 \hdashline 
     &           &    1 &  2.17e-08 &  1.19e-08 &      \\ 
     &           &    2 &  2.17e-08 &  6.20e-16 &      \\ 
     &           &    3 &  2.16e-08 &  6.19e-16 &      \\ 
     &           &    4 &  5.61e-08 &  6.18e-16 &      \\ 
     &           &    5 &  2.17e-08 &  1.60e-15 &      \\ 
     &           &    6 &  2.17e-08 &  6.19e-16 &      \\ 
     &           &    7 &  5.61e-08 &  6.18e-16 &      \\ 
     &           &    8 &  2.17e-08 &  1.60e-15 &      \\ 
     &           &    9 &  2.17e-08 &  6.20e-16 &      \\ 
     &           &   10 &  2.17e-08 &  6.19e-16 &      \\ 
 553 &  5.52e+03 &   10 &           &           & iters  \\ 
 \hdashline 
     &           &    1 &  1.73e-08 &  4.30e-09 &      \\ 
     &           &    2 &  6.04e-08 &  4.95e-16 &      \\ 
     &           &    3 &  1.74e-08 &  1.72e-15 &      \\ 
     &           &    4 &  1.74e-08 &  4.96e-16 &      \\ 
     &           &    5 &  1.73e-08 &  4.96e-16 &      \\ 
     &           &    6 &  1.73e-08 &  4.95e-16 &      \\ 
     &           &    7 &  6.04e-08 &  4.94e-16 &      \\ 
     &           &    8 &  1.74e-08 &  1.72e-15 &      \\ 
     &           &    9 &  1.74e-08 &  4.96e-16 &      \\ 
     &           &   10 &  1.73e-08 &  4.95e-16 &      \\ 
 554 &  5.53e+03 &   10 &           &           & iters  \\ 
 \hdashline 
     &           &    1 &  3.33e-08 &  1.52e-08 &      \\ 
     &           &    2 &  3.33e-08 &  9.51e-16 &      \\ 
     &           &    3 &  4.45e-08 &  9.50e-16 &      \\ 
     &           &    4 &  3.33e-08 &  1.27e-15 &      \\ 
     &           &    5 &  4.44e-08 &  9.50e-16 &      \\ 
     &           &    6 &  3.33e-08 &  1.27e-15 &      \\ 
     &           &    7 &  3.33e-08 &  9.51e-16 &      \\ 
     &           &    8 &  4.45e-08 &  9.50e-16 &      \\ 
     &           &    9 &  3.33e-08 &  1.27e-15 &      \\ 
     &           &   10 &  4.45e-08 &  9.50e-16 &      \\ 
 555 &  5.54e+03 &   10 &           &           & iters  \\ 
 \hdashline 
     &           &    1 &  7.96e-09 &  1.68e-08 &      \\ 
     &           &    2 &  7.94e-09 &  2.27e-16 &      \\ 
     &           &    3 &  7.93e-09 &  2.27e-16 &      \\ 
     &           &    4 &  7.92e-09 &  2.26e-16 &      \\ 
     &           &    5 &  7.91e-09 &  2.26e-16 &      \\ 
     &           &    6 &  3.09e-08 &  2.26e-16 &      \\ 
     &           &    7 &  7.94e-09 &  8.83e-16 &      \\ 
     &           &    8 &  7.93e-09 &  2.27e-16 &      \\ 
     &           &    9 &  7.92e-09 &  2.26e-16 &      \\ 
     &           &   10 &  7.91e-09 &  2.26e-16 &      \\ 
 556 &  5.55e+03 &   10 &           &           & iters  \\ 
 \hdashline 
     &           &    1 &  1.62e-08 &  1.52e-08 &      \\ 
     &           &    2 &  6.15e-08 &  4.62e-16 &      \\ 
     &           &    3 &  1.62e-08 &  1.76e-15 &      \\ 
     &           &    4 &  1.62e-08 &  4.64e-16 &      \\ 
     &           &    5 &  1.62e-08 &  4.63e-16 &      \\ 
     &           &    6 &  1.62e-08 &  4.62e-16 &      \\ 
     &           &    7 &  6.15e-08 &  4.62e-16 &      \\ 
     &           &    8 &  1.62e-08 &  1.76e-15 &      \\ 
     &           &    9 &  1.62e-08 &  4.64e-16 &      \\ 
     &           &   10 &  1.62e-08 &  4.63e-16 &      \\ 
 557 &  5.56e+03 &   10 &           &           & iters  \\ 
 \hdashline 
     &           &    1 &  1.16e-08 &  1.53e-08 &      \\ 
     &           &    2 &  1.16e-08 &  3.31e-16 &      \\ 
     &           &    3 &  2.73e-08 &  3.31e-16 &      \\ 
     &           &    4 &  1.16e-08 &  7.78e-16 &      \\ 
     &           &    5 &  1.16e-08 &  3.32e-16 &      \\ 
     &           &    6 &  2.73e-08 &  3.31e-16 &      \\ 
     &           &    7 &  1.16e-08 &  7.78e-16 &      \\ 
     &           &    8 &  1.16e-08 &  3.32e-16 &      \\ 
     &           &    9 &  1.16e-08 &  3.31e-16 &      \\ 
     &           &   10 &  2.73e-08 &  3.31e-16 &      \\ 
 558 &  5.57e+03 &   10 &           &           & iters  \\ 
 \hdashline 
     &           &    1 &  3.12e-10 &  1.26e-09 &      \\ 
     &           &    2 &  3.11e-10 &  8.90e-18 &      \\ 
     &           &    3 &  3.11e-10 &  8.88e-18 &      \\ 
     &           &    4 &  3.10e-10 &  8.87e-18 &      \\ 
     &           &    5 &  3.10e-10 &  8.86e-18 &      \\ 
     &           &    6 &  3.09e-10 &  8.84e-18 &      \\ 
     &           &    7 &  3.09e-10 &  8.83e-18 &      \\ 
     &           &    8 &  3.09e-10 &  8.82e-18 &      \\ 
     &           &    9 &  3.08e-10 &  8.81e-18 &      \\ 
     &           &   10 &  3.08e-10 &  8.79e-18 &      \\ 
 559 &  5.58e+03 &   10 &           &           & iters  \\ 
 \hdashline 
     &           &    1 &  2.15e-08 &  1.49e-08 &      \\ 
     &           &    2 &  1.74e-08 &  6.13e-16 &      \\ 
     &           &    3 &  2.15e-08 &  4.96e-16 &      \\ 
     &           &    4 &  1.74e-08 &  6.13e-16 &      \\ 
     &           &    5 &  2.15e-08 &  4.96e-16 &      \\ 
     &           &    6 &  1.74e-08 &  6.13e-16 &      \\ 
     &           &    7 &  2.15e-08 &  4.97e-16 &      \\ 
     &           &    8 &  1.74e-08 &  6.13e-16 &      \\ 
     &           &    9 &  1.74e-08 &  4.97e-16 &      \\ 
     &           &   10 &  2.15e-08 &  4.96e-16 &      \\ 
 560 &  5.59e+03 &   10 &           &           & iters  \\ 
 \hdashline 
     &           &    1 &  1.99e-09 &  1.07e-08 &      \\ 
     &           &    2 &  1.99e-09 &  5.68e-17 &      \\ 
     &           &    3 &  1.99e-09 &  5.68e-17 &      \\ 
     &           &    4 &  1.98e-09 &  5.67e-17 &      \\ 
     &           &    5 &  1.98e-09 &  5.66e-17 &      \\ 
     &           &    6 &  1.98e-09 &  5.65e-17 &      \\ 
     &           &    7 &  1.97e-09 &  5.64e-17 &      \\ 
     &           &    8 &  1.97e-09 &  5.64e-17 &      \\ 
     &           &    9 &  1.97e-09 &  5.63e-17 &      \\ 
     &           &   10 &  3.69e-08 &  5.62e-17 &      \\ 
 561 &  5.60e+03 &   10 &           &           & iters  \\ 
 \hdashline 
     &           &    1 &  2.43e-08 &  5.62e-10 &      \\ 
     &           &    2 &  2.43e-08 &  6.95e-16 &      \\ 
     &           &    3 &  5.34e-08 &  6.94e-16 &      \\ 
     &           &    4 &  2.44e-08 &  1.52e-15 &      \\ 
     &           &    5 &  2.43e-08 &  6.95e-16 &      \\ 
     &           &    6 &  5.34e-08 &  6.94e-16 &      \\ 
     &           &    7 &  2.44e-08 &  1.52e-15 &      \\ 
     &           &    8 &  2.43e-08 &  6.95e-16 &      \\ 
     &           &    9 &  5.34e-08 &  6.94e-16 &      \\ 
     &           &   10 &  2.44e-08 &  1.52e-15 &      \\ 
 562 &  5.61e+03 &   10 &           &           & iters  \\ 
 \hdashline 
     &           &    1 &  4.10e-08 &  4.91e-09 &      \\ 
     &           &    2 &  3.68e-08 &  1.17e-15 &      \\ 
     &           &    3 &  4.10e-08 &  1.05e-15 &      \\ 
     &           &    4 &  3.68e-08 &  1.17e-15 &      \\ 
     &           &    5 &  4.09e-08 &  1.05e-15 &      \\ 
     &           &    6 &  3.68e-08 &  1.17e-15 &      \\ 
     &           &    7 &  3.67e-08 &  1.05e-15 &      \\ 
     &           &    8 &  4.10e-08 &  1.05e-15 &      \\ 
     &           &    9 &  3.68e-08 &  1.17e-15 &      \\ 
     &           &   10 &  4.10e-08 &  1.05e-15 &      \\ 
 563 &  5.62e+03 &   10 &           &           & iters  \\ 
 \hdashline 
     &           &    1 &  2.57e-08 &  5.92e-09 &      \\ 
     &           &    2 &  2.56e-08 &  7.33e-16 &      \\ 
     &           &    3 &  5.21e-08 &  7.32e-16 &      \\ 
     &           &    4 &  2.57e-08 &  1.49e-15 &      \\ 
     &           &    5 &  2.56e-08 &  7.33e-16 &      \\ 
     &           &    6 &  5.21e-08 &  7.32e-16 &      \\ 
     &           &    7 &  2.57e-08 &  1.49e-15 &      \\ 
     &           &    8 &  2.56e-08 &  7.33e-16 &      \\ 
     &           &    9 &  5.21e-08 &  7.32e-16 &      \\ 
     &           &   10 &  2.57e-08 &  1.49e-15 &      \\ 
 564 &  5.63e+03 &   10 &           &           & iters  \\ 
 \hdashline 
     &           &    1 &  1.20e-08 &  2.02e-09 &      \\ 
     &           &    2 &  1.19e-08 &  3.41e-16 &      \\ 
     &           &    3 &  1.19e-08 &  3.41e-16 &      \\ 
     &           &    4 &  1.19e-08 &  3.40e-16 &      \\ 
     &           &    5 &  6.58e-08 &  3.40e-16 &      \\ 
     &           &    6 &  1.20e-08 &  1.88e-15 &      \\ 
     &           &    7 &  1.20e-08 &  3.42e-16 &      \\ 
     &           &    8 &  1.20e-08 &  3.42e-16 &      \\ 
     &           &    9 &  1.19e-08 &  3.41e-16 &      \\ 
     &           &   10 &  1.19e-08 &  3.41e-16 &      \\ 
 565 &  5.64e+03 &   10 &           &           & iters  \\ 
 \hdashline 
     &           &    1 &  4.97e-08 &  7.89e-09 &      \\ 
     &           &    2 &  2.81e-08 &  1.42e-15 &      \\ 
     &           &    3 &  4.96e-08 &  8.02e-16 &      \\ 
     &           &    4 &  2.81e-08 &  1.42e-15 &      \\ 
     &           &    5 &  2.81e-08 &  8.02e-16 &      \\ 
     &           &    6 &  4.97e-08 &  8.01e-16 &      \\ 
     &           &    7 &  2.81e-08 &  1.42e-15 &      \\ 
     &           &    8 &  2.81e-08 &  8.02e-16 &      \\ 
     &           &    9 &  4.97e-08 &  8.01e-16 &      \\ 
     &           &   10 &  2.81e-08 &  1.42e-15 &      \\ 
 566 &  5.65e+03 &   10 &           &           & iters  \\ 
 \hdashline 
     &           &    1 &  1.12e-08 &  1.11e-08 &      \\ 
     &           &    2 &  1.12e-08 &  3.19e-16 &      \\ 
     &           &    3 &  6.65e-08 &  3.19e-16 &      \\ 
     &           &    4 &  1.12e-08 &  1.90e-15 &      \\ 
     &           &    5 &  1.12e-08 &  3.21e-16 &      \\ 
     &           &    6 &  1.12e-08 &  3.20e-16 &      \\ 
     &           &    7 &  1.12e-08 &  3.20e-16 &      \\ 
     &           &    8 &  1.12e-08 &  3.19e-16 &      \\ 
     &           &    9 &  1.12e-08 &  3.19e-16 &      \\ 
     &           &   10 &  6.65e-08 &  3.19e-16 &      \\ 
 567 &  5.66e+03 &   10 &           &           & iters  \\ 
 \hdashline 
     &           &    1 &  3.14e-08 &  1.23e-08 &      \\ 
     &           &    2 &  3.14e-08 &  8.96e-16 &      \\ 
     &           &    3 &  4.64e-08 &  8.95e-16 &      \\ 
     &           &    4 &  3.14e-08 &  1.32e-15 &      \\ 
     &           &    5 &  4.64e-08 &  8.95e-16 &      \\ 
     &           &    6 &  3.14e-08 &  1.32e-15 &      \\ 
     &           &    7 &  3.14e-08 &  8.96e-16 &      \\ 
     &           &    8 &  4.64e-08 &  8.95e-16 &      \\ 
     &           &    9 &  3.14e-08 &  1.32e-15 &      \\ 
     &           &   10 &  4.64e-08 &  8.95e-16 &      \\ 
 568 &  5.67e+03 &   10 &           &           & iters  \\ 
 \hdashline 
     &           &    1 &  3.91e-08 &  1.46e-08 &      \\ 
     &           &    2 &  3.87e-08 &  1.12e-15 &      \\ 
     &           &    3 &  3.91e-08 &  1.10e-15 &      \\ 
     &           &    4 &  3.87e-08 &  1.12e-15 &      \\ 
     &           &    5 &  3.91e-08 &  1.10e-15 &      \\ 
     &           &    6 &  3.87e-08 &  1.12e-15 &      \\ 
     &           &    7 &  3.91e-08 &  1.10e-15 &      \\ 
     &           &    8 &  3.87e-08 &  1.12e-15 &      \\ 
     &           &    9 &  3.91e-08 &  1.10e-15 &      \\ 
     &           &   10 &  3.87e-08 &  1.12e-15 &      \\ 
 569 &  5.68e+03 &   10 &           &           & iters  \\ 
 \hdashline 
     &           &    1 &  1.14e-08 &  1.25e-08 &      \\ 
     &           &    2 &  1.14e-08 &  3.26e-16 &      \\ 
     &           &    3 &  1.14e-08 &  3.25e-16 &      \\ 
     &           &    4 &  2.75e-08 &  3.25e-16 &      \\ 
     &           &    5 &  1.14e-08 &  7.84e-16 &      \\ 
     &           &    6 &  1.14e-08 &  3.26e-16 &      \\ 
     &           &    7 &  2.75e-08 &  3.25e-16 &      \\ 
     &           &    8 &  1.14e-08 &  7.84e-16 &      \\ 
     &           &    9 &  1.14e-08 &  3.26e-16 &      \\ 
     &           &   10 &  1.14e-08 &  3.25e-16 &      \\ 
 570 &  5.69e+03 &   10 &           &           & iters  \\ 
 \hdashline 
     &           &    1 &  1.44e-08 &  4.19e-09 &      \\ 
     &           &    2 &  1.43e-08 &  4.10e-16 &      \\ 
     &           &    3 &  2.45e-08 &  4.09e-16 &      \\ 
     &           &    4 &  1.44e-08 &  7.00e-16 &      \\ 
     &           &    5 &  2.45e-08 &  4.10e-16 &      \\ 
     &           &    6 &  1.44e-08 &  6.99e-16 &      \\ 
     &           &    7 &  1.44e-08 &  4.10e-16 &      \\ 
     &           &    8 &  2.45e-08 &  4.10e-16 &      \\ 
     &           &    9 &  1.44e-08 &  7.00e-16 &      \\ 
     &           &   10 &  1.43e-08 &  4.10e-16 &      \\ 
 571 &  5.70e+03 &   10 &           &           & iters  \\ 
 \hdashline 
     &           &    1 &  1.74e-08 &  1.20e-09 &      \\ 
     &           &    2 &  6.03e-08 &  4.96e-16 &      \\ 
     &           &    3 &  1.74e-08 &  1.72e-15 &      \\ 
     &           &    4 &  1.74e-08 &  4.98e-16 &      \\ 
     &           &    5 &  1.74e-08 &  4.97e-16 &      \\ 
     &           &    6 &  1.74e-08 &  4.96e-16 &      \\ 
     &           &    7 &  6.03e-08 &  4.96e-16 &      \\ 
     &           &    8 &  1.74e-08 &  1.72e-15 &      \\ 
     &           &    9 &  1.74e-08 &  4.98e-16 &      \\ 
     &           &   10 &  1.74e-08 &  4.97e-16 &      \\ 
 572 &  5.71e+03 &   10 &           &           & iters  \\ 
 \hdashline 
     &           &    1 &  1.48e-08 &  7.46e-10 &      \\ 
     &           &    2 &  1.48e-08 &  4.24e-16 &      \\ 
     &           &    3 &  1.48e-08 &  4.23e-16 &      \\ 
     &           &    4 &  2.41e-08 &  4.23e-16 &      \\ 
     &           &    5 &  1.48e-08 &  6.87e-16 &      \\ 
     &           &    6 &  2.40e-08 &  4.23e-16 &      \\ 
     &           &    7 &  1.48e-08 &  6.86e-16 &      \\ 
     &           &    8 &  1.48e-08 &  4.23e-16 &      \\ 
     &           &    9 &  2.41e-08 &  4.23e-16 &      \\ 
     &           &   10 &  1.48e-08 &  6.87e-16 &      \\ 
 573 &  5.72e+03 &   10 &           &           & iters  \\ 
 \hdashline 
     &           &    1 &  2.89e-08 &  2.16e-09 &      \\ 
     &           &    2 &  4.88e-08 &  8.25e-16 &      \\ 
     &           &    3 &  2.89e-08 &  1.39e-15 &      \\ 
     &           &    4 &  2.89e-08 &  8.26e-16 &      \\ 
     &           &    5 &  4.88e-08 &  8.25e-16 &      \\ 
     &           &    6 &  2.89e-08 &  1.39e-15 &      \\ 
     &           &    7 &  4.88e-08 &  8.26e-16 &      \\ 
     &           &    8 &  2.90e-08 &  1.39e-15 &      \\ 
     &           &    9 &  2.89e-08 &  8.26e-16 &      \\ 
     &           &   10 &  4.88e-08 &  8.25e-16 &      \\ 
 574 &  5.73e+03 &   10 &           &           & iters  \\ 
 \hdashline 
     &           &    1 &  8.26e-09 &  4.79e-09 &      \\ 
     &           &    2 &  8.25e-09 &  2.36e-16 &      \\ 
     &           &    3 &  8.24e-09 &  2.35e-16 &      \\ 
     &           &    4 &  8.23e-09 &  2.35e-16 &      \\ 
     &           &    5 &  8.22e-09 &  2.35e-16 &      \\ 
     &           &    6 &  8.20e-09 &  2.34e-16 &      \\ 
     &           &    7 &  8.19e-09 &  2.34e-16 &      \\ 
     &           &    8 &  8.18e-09 &  2.34e-16 &      \\ 
     &           &    9 &  6.95e-08 &  2.33e-16 &      \\ 
     &           &   10 &  4.71e-08 &  1.98e-15 &      \\ 
 575 &  5.74e+03 &   10 &           &           & iters  \\ 
 \hdashline 
     &           &    1 &  5.77e-09 &  1.06e-08 &      \\ 
     &           &    2 &  5.76e-09 &  1.65e-16 &      \\ 
     &           &    3 &  5.76e-09 &  1.64e-16 &      \\ 
     &           &    4 &  5.75e-09 &  1.64e-16 &      \\ 
     &           &    5 &  5.74e-09 &  1.64e-16 &      \\ 
     &           &    6 &  5.73e-09 &  1.64e-16 &      \\ 
     &           &    7 &  5.72e-09 &  1.64e-16 &      \\ 
     &           &    8 &  5.71e-09 &  1.63e-16 &      \\ 
     &           &    9 &  5.71e-09 &  1.63e-16 &      \\ 
     &           &   10 &  7.20e-08 &  1.63e-16 &      \\ 
 576 &  5.75e+03 &   10 &           &           & iters  \\ 
 \hdashline 
     &           &    1 &  5.13e-08 &  1.65e-08 &      \\ 
     &           &    2 &  1.24e-08 &  1.47e-15 &      \\ 
     &           &    3 &  1.24e-08 &  3.55e-16 &      \\ 
     &           &    4 &  6.53e-08 &  3.54e-16 &      \\ 
     &           &    5 &  1.25e-08 &  1.86e-15 &      \\ 
     &           &    6 &  1.25e-08 &  3.56e-16 &      \\ 
     &           &    7 &  1.24e-08 &  3.56e-16 &      \\ 
     &           &    8 &  1.24e-08 &  3.55e-16 &      \\ 
     &           &    9 &  1.24e-08 &  3.55e-16 &      \\ 
     &           &   10 &  6.53e-08 &  3.54e-16 &      \\ 
 577 &  5.76e+03 &   10 &           &           & iters  \\ 
 \hdashline 
     &           &    1 &  2.09e-08 &  1.83e-08 &      \\ 
     &           &    2 &  2.09e-08 &  5.96e-16 &      \\ 
     &           &    3 &  1.80e-08 &  5.95e-16 &      \\ 
     &           &    4 &  2.09e-08 &  5.14e-16 &      \\ 
     &           &    5 &  1.80e-08 &  5.95e-16 &      \\ 
     &           &    6 &  1.80e-08 &  5.14e-16 &      \\ 
     &           &    7 &  2.09e-08 &  5.13e-16 &      \\ 
     &           &    8 &  1.80e-08 &  5.96e-16 &      \\ 
     &           &    9 &  2.09e-08 &  5.13e-16 &      \\ 
     &           &   10 &  1.80e-08 &  5.96e-16 &      \\ 
 578 &  5.77e+03 &   10 &           &           & iters  \\ 
 \hdashline 
     &           &    1 &  8.08e-09 &  1.81e-08 &      \\ 
     &           &    2 &  3.08e-08 &  2.31e-16 &      \\ 
     &           &    3 &  8.11e-09 &  8.78e-16 &      \\ 
     &           &    4 &  8.10e-09 &  2.31e-16 &      \\ 
     &           &    5 &  8.09e-09 &  2.31e-16 &      \\ 
     &           &    6 &  8.08e-09 &  2.31e-16 &      \\ 
     &           &    7 &  3.08e-08 &  2.30e-16 &      \\ 
     &           &    8 &  8.11e-09 &  8.78e-16 &      \\ 
     &           &    9 &  8.10e-09 &  2.31e-16 &      \\ 
     &           &   10 &  8.08e-09 &  2.31e-16 &      \\ 
 579 &  5.78e+03 &   10 &           &           & iters  \\ 
 \hdashline 
     &           &    1 &  1.11e-08 &  9.26e-09 &      \\ 
     &           &    2 &  1.11e-08 &  3.17e-16 &      \\ 
     &           &    3 &  1.11e-08 &  3.16e-16 &      \\ 
     &           &    4 &  1.10e-08 &  3.16e-16 &      \\ 
     &           &    5 &  1.10e-08 &  3.15e-16 &      \\ 
     &           &    6 &  6.67e-08 &  3.15e-16 &      \\ 
     &           &    7 &  1.11e-08 &  1.90e-15 &      \\ 
     &           &    8 &  1.11e-08 &  3.17e-16 &      \\ 
     &           &    9 &  1.11e-08 &  3.17e-16 &      \\ 
     &           &   10 &  1.11e-08 &  3.16e-16 &      \\ 
 580 &  5.79e+03 &   10 &           &           & iters  \\ 
 \hdashline 
     &           &    1 &  2.43e-08 &  5.58e-11 &      \\ 
     &           &    2 &  2.42e-08 &  6.93e-16 &      \\ 
     &           &    3 &  5.35e-08 &  6.92e-16 &      \\ 
     &           &    4 &  2.43e-08 &  1.53e-15 &      \\ 
     &           &    5 &  2.42e-08 &  6.93e-16 &      \\ 
     &           &    6 &  2.42e-08 &  6.92e-16 &      \\ 
     &           &    7 &  5.35e-08 &  6.91e-16 &      \\ 
     &           &    8 &  2.43e-08 &  1.53e-15 &      \\ 
     &           &    9 &  2.42e-08 &  6.92e-16 &      \\ 
     &           &   10 &  5.35e-08 &  6.91e-16 &      \\ 
 581 &  5.80e+03 &   10 &           &           & iters  \\ 
 \hdashline 
     &           &    1 &  4.49e-09 &  5.04e-09 &      \\ 
     &           &    2 &  4.48e-09 &  1.28e-16 &      \\ 
     &           &    3 &  4.48e-09 &  1.28e-16 &      \\ 
     &           &    4 &  4.47e-09 &  1.28e-16 &      \\ 
     &           &    5 &  4.46e-09 &  1.28e-16 &      \\ 
     &           &    6 &  4.46e-09 &  1.27e-16 &      \\ 
     &           &    7 &  4.45e-09 &  1.27e-16 &      \\ 
     &           &    8 &  7.32e-08 &  1.27e-16 &      \\ 
     &           &    9 &  4.34e-08 &  2.09e-15 &      \\ 
     &           &   10 &  4.49e-09 &  1.24e-15 &      \\ 
 582 &  5.81e+03 &   10 &           &           & iters  \\ 
 \hdashline 
     &           &    1 &  2.00e-09 &  1.08e-08 &      \\ 
     &           &    2 &  2.00e-09 &  5.71e-17 &      \\ 
     &           &    3 &  2.00e-09 &  5.70e-17 &      \\ 
     &           &    4 &  1.99e-09 &  5.70e-17 &      \\ 
     &           &    5 &  1.99e-09 &  5.69e-17 &      \\ 
     &           &    6 &  1.99e-09 &  5.68e-17 &      \\ 
     &           &    7 &  1.98e-09 &  5.67e-17 &      \\ 
     &           &    8 &  1.98e-09 &  5.66e-17 &      \\ 
     &           &    9 &  1.98e-09 &  5.65e-17 &      \\ 
     &           &   10 &  1.98e-09 &  5.65e-17 &      \\ 
 583 &  5.82e+03 &   10 &           &           & iters  \\ 
 \hdashline 
     &           &    1 &  3.97e-08 &  2.23e-09 &      \\ 
     &           &    2 &  3.97e-08 &  1.13e-15 &      \\ 
     &           &    3 &  3.81e-08 &  1.13e-15 &      \\ 
     &           &    4 &  3.97e-08 &  1.09e-15 &      \\ 
     &           &    5 &  3.81e-08 &  1.13e-15 &      \\ 
     &           &    6 &  3.97e-08 &  1.09e-15 &      \\ 
     &           &    7 &  3.81e-08 &  1.13e-15 &      \\ 
     &           &    8 &  3.97e-08 &  1.09e-15 &      \\ 
     &           &    9 &  3.81e-08 &  1.13e-15 &      \\ 
     &           &   10 &  3.97e-08 &  1.09e-15 &      \\ 
 584 &  5.83e+03 &   10 &           &           & iters  \\ 
 \hdashline 
     &           &    1 &  3.44e-08 &  1.44e-08 &      \\ 
     &           &    2 &  3.43e-08 &  9.81e-16 &      \\ 
     &           &    3 &  4.56e-09 &  9.80e-16 &      \\ 
     &           &    4 &  4.55e-09 &  1.30e-16 &      \\ 
     &           &    5 &  4.54e-09 &  1.30e-16 &      \\ 
     &           &    6 &  4.54e-09 &  1.30e-16 &      \\ 
     &           &    7 &  4.53e-09 &  1.29e-16 &      \\ 
     &           &    8 &  4.52e-09 &  1.29e-16 &      \\ 
     &           &    9 &  4.52e-09 &  1.29e-16 &      \\ 
     &           &   10 &  4.51e-09 &  1.29e-16 &      \\ 
 585 &  5.84e+03 &   10 &           &           & iters  \\ 
 \hdashline 
     &           &    1 &  1.44e-08 &  3.37e-09 &      \\ 
     &           &    2 &  6.33e-08 &  4.12e-16 &      \\ 
     &           &    3 &  1.45e-08 &  1.81e-15 &      \\ 
     &           &    4 &  1.45e-08 &  4.14e-16 &      \\ 
     &           &    5 &  1.45e-08 &  4.13e-16 &      \\ 
     &           &    6 &  1.44e-08 &  4.13e-16 &      \\ 
     &           &    7 &  6.33e-08 &  4.12e-16 &      \\ 
     &           &    8 &  1.45e-08 &  1.81e-15 &      \\ 
     &           &    9 &  1.45e-08 &  4.14e-16 &      \\ 
     &           &   10 &  1.45e-08 &  4.13e-16 &      \\ 
 586 &  5.85e+03 &   10 &           &           & iters  \\ 
 \hdashline 
     &           &    1 &  6.90e-09 &  1.76e-09 &      \\ 
     &           &    2 &  6.89e-09 &  1.97e-16 &      \\ 
     &           &    3 &  3.20e-08 &  1.97e-16 &      \\ 
     &           &    4 &  6.93e-09 &  9.12e-16 &      \\ 
     &           &    5 &  6.92e-09 &  1.98e-16 &      \\ 
     &           &    6 &  6.91e-09 &  1.97e-16 &      \\ 
     &           &    7 &  6.90e-09 &  1.97e-16 &      \\ 
     &           &    8 &  6.89e-09 &  1.97e-16 &      \\ 
     &           &    9 &  3.20e-08 &  1.97e-16 &      \\ 
     &           &   10 &  6.92e-09 &  9.12e-16 &      \\ 
 587 &  5.86e+03 &   10 &           &           & iters  \\ 
 \hdashline 
     &           &    1 &  9.05e-09 &  9.07e-09 &      \\ 
     &           &    2 &  9.04e-09 &  2.58e-16 &      \\ 
     &           &    3 &  9.03e-09 &  2.58e-16 &      \\ 
     &           &    4 &  9.01e-09 &  2.58e-16 &      \\ 
     &           &    5 &  9.00e-09 &  2.57e-16 &      \\ 
     &           &    6 &  6.87e-08 &  2.57e-16 &      \\ 
     &           &    7 &  4.79e-08 &  1.96e-15 &      \\ 
     &           &    8 &  9.02e-09 &  1.37e-15 &      \\ 
     &           &    9 &  9.00e-09 &  2.57e-16 &      \\ 
     &           &   10 &  8.99e-09 &  2.57e-16 &      \\ 
 588 &  5.87e+03 &   10 &           &           & iters  \\ 
 \hdashline 
     &           &    1 &  7.81e-08 &  1.04e-08 &      \\ 
     &           &    2 &  7.74e-08 &  2.23e-15 &      \\ 
     &           &    3 &  7.81e-08 &  2.21e-15 &      \\ 
     &           &    4 &  7.74e-08 &  2.23e-15 &      \\ 
     &           &    5 &  7.81e-08 &  2.21e-15 &      \\ 
     &           &    6 &  7.74e-08 &  2.23e-15 &      \\ 
     &           &    7 &  7.81e-08 &  2.21e-15 &      \\ 
     &           &    8 &  7.74e-08 &  2.23e-15 &      \\ 
     &           &    9 &  7.81e-08 &  2.21e-15 &      \\ 
     &           &   10 &  7.74e-08 &  2.23e-15 &      \\ 
 589 &  5.88e+03 &   10 &           &           & iters  \\ 
 \hdashline 
     &           &    1 &  6.05e-09 &  3.86e-09 &      \\ 
     &           &    2 &  6.04e-09 &  1.73e-16 &      \\ 
     &           &    3 &  6.04e-09 &  1.73e-16 &      \\ 
     &           &    4 &  6.03e-09 &  1.72e-16 &      \\ 
     &           &    5 &  6.02e-09 &  1.72e-16 &      \\ 
     &           &    6 &  6.01e-09 &  1.72e-16 &      \\ 
     &           &    7 &  6.00e-09 &  1.72e-16 &      \\ 
     &           &    8 &  5.99e-09 &  1.71e-16 &      \\ 
     &           &    9 &  5.98e-09 &  1.71e-16 &      \\ 
     &           &   10 &  5.98e-09 &  1.71e-16 &      \\ 
 590 &  5.89e+03 &   10 &           &           & iters  \\ 
 \hdashline 
     &           &    1 &  1.58e-08 &  1.61e-08 &      \\ 
     &           &    2 &  6.19e-08 &  4.50e-16 &      \\ 
     &           &    3 &  1.58e-08 &  1.77e-15 &      \\ 
     &           &    4 &  1.58e-08 &  4.52e-16 &      \\ 
     &           &    5 &  1.58e-08 &  4.51e-16 &      \\ 
     &           &    6 &  1.58e-08 &  4.51e-16 &      \\ 
     &           &    7 &  6.19e-08 &  4.50e-16 &      \\ 
     &           &    8 &  1.58e-08 &  1.77e-15 &      \\ 
     &           &    9 &  1.58e-08 &  4.52e-16 &      \\ 
     &           &   10 &  1.58e-08 &  4.51e-16 &      \\ 
 591 &  5.90e+03 &   10 &           &           & iters  \\ 
 \hdashline 
     &           &    1 &  2.55e-08 &  4.26e-09 &      \\ 
     &           &    2 &  1.34e-08 &  7.28e-16 &      \\ 
     &           &    3 &  2.55e-08 &  3.82e-16 &      \\ 
     &           &    4 &  1.34e-08 &  7.28e-16 &      \\ 
     &           &    5 &  1.34e-08 &  3.82e-16 &      \\ 
     &           &    6 &  2.55e-08 &  3.82e-16 &      \\ 
     &           &    7 &  1.34e-08 &  7.28e-16 &      \\ 
     &           &    8 &  1.34e-08 &  3.82e-16 &      \\ 
     &           &    9 &  2.55e-08 &  3.81e-16 &      \\ 
     &           &   10 &  1.34e-08 &  7.28e-16 &      \\ 
 592 &  5.91e+03 &   10 &           &           & iters  \\ 
 \hdashline 
     &           &    1 &  2.12e-08 &  1.18e-08 &      \\ 
     &           &    2 &  5.65e-08 &  6.06e-16 &      \\ 
     &           &    3 &  2.13e-08 &  1.61e-15 &      \\ 
     &           &    4 &  2.13e-08 &  6.07e-16 &      \\ 
     &           &    5 &  2.12e-08 &  6.07e-16 &      \\ 
     &           &    6 &  5.65e-08 &  6.06e-16 &      \\ 
     &           &    7 &  2.13e-08 &  1.61e-15 &      \\ 
     &           &    8 &  2.12e-08 &  6.07e-16 &      \\ 
     &           &    9 &  2.12e-08 &  6.06e-16 &      \\ 
     &           &   10 &  5.65e-08 &  6.05e-16 &      \\ 
 593 &  5.92e+03 &   10 &           &           & iters  \\ 
 \hdashline 
     &           &    1 &  1.78e-08 &  8.56e-11 &      \\ 
     &           &    2 &  1.78e-08 &  5.09e-16 &      \\ 
     &           &    3 &  5.99e-08 &  5.08e-16 &      \\ 
     &           &    4 &  1.79e-08 &  1.71e-15 &      \\ 
     &           &    5 &  1.78e-08 &  5.10e-16 &      \\ 
     &           &    6 &  1.78e-08 &  5.09e-16 &      \\ 
     &           &    7 &  1.78e-08 &  5.09e-16 &      \\ 
     &           &    8 &  5.99e-08 &  5.08e-16 &      \\ 
     &           &    9 &  1.79e-08 &  1.71e-15 &      \\ 
     &           &   10 &  1.78e-08 &  5.10e-16 &      \\ 
 594 &  5.93e+03 &   10 &           &           & iters  \\ 
 \hdashline 
     &           &    1 &  3.88e-08 &  3.63e-08 &      \\ 
     &           &    2 &  3.89e-08 &  1.11e-15 &      \\ 
     &           &    3 &  3.88e-08 &  1.11e-15 &      \\ 
     &           &    4 &  3.89e-08 &  1.11e-15 &      \\ 
     &           &    5 &  3.88e-08 &  1.11e-15 &      \\ 
     &           &    6 &  3.89e-08 &  1.11e-15 &      \\ 
     &           &    7 &  3.88e-08 &  1.11e-15 &      \\ 
     &           &    8 &  3.89e-08 &  1.11e-15 &      \\ 
     &           &    9 &  3.88e-08 &  1.11e-15 &      \\ 
     &           &   10 &  3.89e-08 &  1.11e-15 &      \\ 
 595 &  5.94e+03 &   10 &           &           & iters  \\ 
 \hdashline 
     &           &    1 &  2.86e-08 &  6.35e-08 &      \\ 
     &           &    2 &  4.91e-08 &  8.18e-16 &      \\ 
     &           &    3 &  2.87e-08 &  1.40e-15 &      \\ 
     &           &    4 &  4.91e-08 &  8.18e-16 &      \\ 
     &           &    5 &  2.87e-08 &  1.40e-15 &      \\ 
     &           &    6 &  2.87e-08 &  8.19e-16 &      \\ 
     &           &    7 &  4.91e-08 &  8.18e-16 &      \\ 
     &           &    8 &  2.87e-08 &  1.40e-15 &      \\ 
     &           &    9 &  2.87e-08 &  8.19e-16 &      \\ 
     &           &   10 &  4.91e-08 &  8.18e-16 &      \\ 
 596 &  5.95e+03 &   10 &           &           & iters  \\ 
 \hdashline 
     &           &    1 &  6.19e-08 &  3.29e-08 &      \\ 
     &           &    2 &  1.59e-08 &  1.77e-15 &      \\ 
     &           &    3 &  1.59e-08 &  4.54e-16 &      \\ 
     &           &    4 &  1.59e-08 &  4.53e-16 &      \\ 
     &           &    5 &  6.19e-08 &  4.53e-16 &      \\ 
     &           &    6 &  1.59e-08 &  1.77e-15 &      \\ 
     &           &    7 &  1.59e-08 &  4.55e-16 &      \\ 
     &           &    8 &  1.59e-08 &  4.54e-16 &      \\ 
     &           &    9 &  1.59e-08 &  4.53e-16 &      \\ 
     &           &   10 &  6.19e-08 &  4.53e-16 &      \\ 
 597 &  5.96e+03 &   10 &           &           & iters  \\ 
 \hdashline 
     &           &    1 &  2.48e-09 &  7.28e-08 &      \\ 
     &           &    2 &  7.52e-08 &  7.07e-17 &      \\ 
     &           &    3 &  8.03e-08 &  2.15e-15 &      \\ 
     &           &    4 &  7.52e-08 &  2.29e-15 &      \\ 
     &           &    5 &  8.03e-08 &  2.15e-15 &      \\ 
     &           &    6 &  7.52e-08 &  2.29e-15 &      \\ 
     &           &    7 &  8.03e-08 &  2.15e-15 &      \\ 
     &           &    8 &  7.52e-08 &  2.29e-15 &      \\ 
     &           &    9 &  2.56e-09 &  2.15e-15 &      \\ 
     &           &   10 &  2.55e-09 &  7.30e-17 &      \\ 
 598 &  5.97e+03 &   10 &           &           & iters  \\ 
 \hdashline 
     &           &    1 &  1.27e-08 &  9.48e-08 &      \\ 
     &           &    2 &  1.27e-08 &  3.64e-16 &      \\ 
     &           &    3 &  2.61e-08 &  3.63e-16 &      \\ 
     &           &    4 &  1.27e-08 &  7.46e-16 &      \\ 
     &           &    5 &  1.27e-08 &  3.64e-16 &      \\ 
     &           &    6 &  2.61e-08 &  3.63e-16 &      \\ 
     &           &    7 &  1.27e-08 &  7.46e-16 &      \\ 
     &           &    8 &  1.27e-08 &  3.64e-16 &      \\ 
     &           &    9 &  2.61e-08 &  3.63e-16 &      \\ 
     &           &   10 &  1.27e-08 &  7.46e-16 &      \\ 
 599 &  5.98e+03 &   10 &           &           & iters  \\ 
 \hdashline 
     &           &    1 &  1.13e-08 &  3.21e-08 &      \\ 
     &           &    2 &  2.76e-08 &  3.22e-16 &      \\ 
     &           &    3 &  1.13e-08 &  7.87e-16 &      \\ 
     &           &    4 &  1.13e-08 &  3.22e-16 &      \\ 
     &           &    5 &  1.13e-08 &  3.22e-16 &      \\ 
     &           &    6 &  2.76e-08 &  3.21e-16 &      \\ 
     &           &    7 &  1.13e-08 &  7.88e-16 &      \\ 
     &           &    8 &  1.13e-08 &  3.22e-16 &      \\ 
     &           &    9 &  2.76e-08 &  3.22e-16 &      \\ 
     &           &   10 &  1.13e-08 &  7.87e-16 &      \\ 
 600 &  5.99e+03 &   10 &           &           & iters  \\ 
 \hdashline 
     &           &    1 &  2.05e-08 &  5.92e-08 &      \\ 
     &           &    2 &  2.04e-08 &  5.84e-16 &      \\ 
     &           &    3 &  2.04e-08 &  5.83e-16 &      \\ 
     &           &    4 &  5.73e-08 &  5.82e-16 &      \\ 
     &           &    5 &  2.04e-08 &  1.64e-15 &      \\ 
     &           &    6 &  2.04e-08 &  5.84e-16 &      \\ 
     &           &    7 &  2.04e-08 &  5.83e-16 &      \\ 
     &           &    8 &  5.73e-08 &  5.82e-16 &      \\ 
     &           &    9 &  2.04e-08 &  1.64e-15 &      \\ 
     &           &   10 &  2.04e-08 &  5.83e-16 &      \\ 
 601 &  6.00e+03 &   10 &           &           & iters  \\ 
 \hdashline 
     &           &    1 &  4.18e-08 &  1.69e-08 &      \\ 
     &           &    2 &  3.59e-08 &  1.19e-15 &      \\ 
     &           &    3 &  4.18e-08 &  1.02e-15 &      \\ 
     &           &    4 &  3.59e-08 &  1.19e-15 &      \\ 
     &           &    5 &  4.18e-08 &  1.03e-15 &      \\ 
     &           &    6 &  3.59e-08 &  1.19e-15 &      \\ 
     &           &    7 &  3.59e-08 &  1.03e-15 &      \\ 
     &           &    8 &  4.19e-08 &  1.02e-15 &      \\ 
     &           &    9 &  3.59e-08 &  1.19e-15 &      \\ 
     &           &   10 &  4.19e-08 &  1.02e-15 &      \\ 
 602 &  6.01e+03 &   10 &           &           & iters  \\ 
 \hdashline 
     &           &    1 &  2.52e-08 &  2.03e-08 &      \\ 
     &           &    2 &  5.25e-08 &  7.20e-16 &      \\ 
     &           &    3 &  2.53e-08 &  1.50e-15 &      \\ 
     &           &    4 &  2.52e-08 &  7.21e-16 &      \\ 
     &           &    5 &  5.25e-08 &  7.20e-16 &      \\ 
     &           &    6 &  2.53e-08 &  1.50e-15 &      \\ 
     &           &    7 &  2.52e-08 &  7.21e-16 &      \\ 
     &           &    8 &  5.25e-08 &  7.20e-16 &      \\ 
     &           &    9 &  2.53e-08 &  1.50e-15 &      \\ 
     &           &   10 &  2.52e-08 &  7.22e-16 &      \\ 
 603 &  6.02e+03 &   10 &           &           & iters  \\ 
 \hdashline 
     &           &    1 &  1.88e-08 &  4.86e-09 &      \\ 
     &           &    2 &  1.88e-08 &  5.37e-16 &      \\ 
     &           &    3 &  5.89e-08 &  5.36e-16 &      \\ 
     &           &    4 &  1.88e-08 &  1.68e-15 &      \\ 
     &           &    5 &  1.88e-08 &  5.37e-16 &      \\ 
     &           &    6 &  1.88e-08 &  5.37e-16 &      \\ 
     &           &    7 &  5.89e-08 &  5.36e-16 &      \\ 
     &           &    8 &  1.88e-08 &  1.68e-15 &      \\ 
     &           &    9 &  1.88e-08 &  5.38e-16 &      \\ 
     &           &   10 &  1.88e-08 &  5.37e-16 &      \\ 
 604 &  6.03e+03 &   10 &           &           & iters  \\ 
 \hdashline 
     &           &    1 &  1.35e-08 &  1.59e-08 &      \\ 
     &           &    2 &  1.35e-08 &  3.85e-16 &      \\ 
     &           &    3 &  1.35e-08 &  3.85e-16 &      \\ 
     &           &    4 &  1.34e-08 &  3.84e-16 &      \\ 
     &           &    5 &  6.43e-08 &  3.84e-16 &      \\ 
     &           &    6 &  1.35e-08 &  1.83e-15 &      \\ 
     &           &    7 &  1.35e-08 &  3.86e-16 &      \\ 
     &           &    8 &  1.35e-08 &  3.85e-16 &      \\ 
     &           &    9 &  1.35e-08 &  3.84e-16 &      \\ 
     &           &   10 &  1.34e-08 &  3.84e-16 &      \\ 
 605 &  6.04e+03 &   10 &           &           & iters  \\ 
 \hdashline 
     &           &    1 &  1.18e-08 &  3.07e-08 &      \\ 
     &           &    2 &  1.18e-08 &  3.37e-16 &      \\ 
     &           &    3 &  1.18e-08 &  3.37e-16 &      \\ 
     &           &    4 &  1.18e-08 &  3.36e-16 &      \\ 
     &           &    5 &  1.18e-08 &  3.36e-16 &      \\ 
     &           &    6 &  1.17e-08 &  3.35e-16 &      \\ 
     &           &    7 &  1.17e-08 &  3.35e-16 &      \\ 
     &           &    8 &  1.17e-08 &  3.34e-16 &      \\ 
     &           &    9 &  1.17e-08 &  3.34e-16 &      \\ 
     &           &   10 &  6.60e-08 &  3.34e-16 &      \\ 
 606 &  6.05e+03 &   10 &           &           & iters  \\ 
 \hdashline 
     &           &    1 &  3.06e-08 &  4.31e-08 &      \\ 
     &           &    2 &  3.06e-08 &  8.74e-16 &      \\ 
     &           &    3 &  4.72e-08 &  8.72e-16 &      \\ 
     &           &    4 &  3.06e-08 &  1.35e-15 &      \\ 
     &           &    5 &  4.71e-08 &  8.73e-16 &      \\ 
     &           &    6 &  3.06e-08 &  1.35e-15 &      \\ 
     &           &    7 &  3.06e-08 &  8.74e-16 &      \\ 
     &           &    8 &  4.72e-08 &  8.73e-16 &      \\ 
     &           &    9 &  3.06e-08 &  1.35e-15 &      \\ 
     &           &   10 &  4.71e-08 &  8.73e-16 &      \\ 
 607 &  6.06e+03 &   10 &           &           & iters  \\ 
 \hdashline 
     &           &    1 &  3.99e-08 &  3.78e-08 &      \\ 
     &           &    2 &  3.78e-08 &  1.14e-15 &      \\ 
     &           &    3 &  3.78e-08 &  1.08e-15 &      \\ 
     &           &    4 &  4.00e-08 &  1.08e-15 &      \\ 
     &           &    5 &  3.78e-08 &  1.14e-15 &      \\ 
     &           &    6 &  4.00e-08 &  1.08e-15 &      \\ 
     &           &    7 &  3.78e-08 &  1.14e-15 &      \\ 
     &           &    8 &  4.00e-08 &  1.08e-15 &      \\ 
     &           &    9 &  3.78e-08 &  1.14e-15 &      \\ 
     &           &   10 &  4.00e-08 &  1.08e-15 &      \\ 
 608 &  6.07e+03 &   10 &           &           & iters  \\ 
 \hdashline 
     &           &    1 &  1.85e-08 &  3.07e-09 &      \\ 
     &           &    2 &  1.85e-08 &  5.28e-16 &      \\ 
     &           &    3 &  5.92e-08 &  5.27e-16 &      \\ 
     &           &    4 &  1.85e-08 &  1.69e-15 &      \\ 
     &           &    5 &  1.85e-08 &  5.29e-16 &      \\ 
     &           &    6 &  1.85e-08 &  5.28e-16 &      \\ 
     &           &    7 &  5.92e-08 &  5.27e-16 &      \\ 
     &           &    8 &  1.85e-08 &  1.69e-15 &      \\ 
     &           &    9 &  1.85e-08 &  5.29e-16 &      \\ 
     &           &   10 &  1.85e-08 &  5.28e-16 &      \\ 
 609 &  6.08e+03 &   10 &           &           & iters  \\ 
 \hdashline 
     &           &    1 &  3.56e-08 &  1.32e-08 &      \\ 
     &           &    2 &  3.56e-08 &  1.02e-15 &      \\ 
     &           &    3 &  4.22e-08 &  1.02e-15 &      \\ 
     &           &    4 &  3.56e-08 &  1.20e-15 &      \\ 
     &           &    5 &  4.22e-08 &  1.02e-15 &      \\ 
     &           &    6 &  3.56e-08 &  1.20e-15 &      \\ 
     &           &    7 &  4.21e-08 &  1.02e-15 &      \\ 
     &           &    8 &  3.56e-08 &  1.20e-15 &      \\ 
     &           &    9 &  3.55e-08 &  1.02e-15 &      \\ 
     &           &   10 &  4.22e-08 &  1.01e-15 &      \\ 
 610 &  6.09e+03 &   10 &           &           & iters  \\ 
 \hdashline 
     &           &    1 &  3.25e-08 &  2.85e-09 &      \\ 
     &           &    2 &  4.52e-08 &  9.29e-16 &      \\ 
     &           &    3 &  3.26e-08 &  1.29e-15 &      \\ 
     &           &    4 &  4.52e-08 &  9.29e-16 &      \\ 
     &           &    5 &  3.26e-08 &  1.29e-15 &      \\ 
     &           &    6 &  3.25e-08 &  9.30e-16 &      \\ 
     &           &    7 &  4.52e-08 &  9.29e-16 &      \\ 
     &           &    8 &  3.26e-08 &  1.29e-15 &      \\ 
     &           &    9 &  4.52e-08 &  9.29e-16 &      \\ 
     &           &   10 &  3.26e-08 &  1.29e-15 &      \\ 
 611 &  6.10e+03 &   10 &           &           & iters  \\ 
 \hdashline 
     &           &    1 &  1.65e-08 &  1.94e-08 &      \\ 
     &           &    2 &  1.64e-08 &  4.70e-16 &      \\ 
     &           &    3 &  1.64e-08 &  4.69e-16 &      \\ 
     &           &    4 &  6.13e-08 &  4.69e-16 &      \\ 
     &           &    5 &  1.65e-08 &  1.75e-15 &      \\ 
     &           &    6 &  1.65e-08 &  4.70e-16 &      \\ 
     &           &    7 &  1.64e-08 &  4.70e-16 &      \\ 
     &           &    8 &  6.13e-08 &  4.69e-16 &      \\ 
     &           &    9 &  1.65e-08 &  1.75e-15 &      \\ 
     &           &   10 &  1.65e-08 &  4.71e-16 &      \\ 
 612 &  6.11e+03 &   10 &           &           & iters  \\ 
 \hdashline 
     &           &    1 &  2.39e-08 &  5.10e-08 &      \\ 
     &           &    2 &  2.39e-08 &  6.82e-16 &      \\ 
     &           &    3 &  5.39e-08 &  6.81e-16 &      \\ 
     &           &    4 &  2.39e-08 &  1.54e-15 &      \\ 
     &           &    5 &  2.39e-08 &  6.82e-16 &      \\ 
     &           &    6 &  5.38e-08 &  6.81e-16 &      \\ 
     &           &    7 &  2.39e-08 &  1.54e-15 &      \\ 
     &           &    8 &  2.39e-08 &  6.83e-16 &      \\ 
     &           &    9 &  5.38e-08 &  6.82e-16 &      \\ 
     &           &   10 &  2.39e-08 &  1.54e-15 &      \\ 
 613 &  6.12e+03 &   10 &           &           & iters  \\ 
 \hdashline 
     &           &    1 &  1.87e-08 &  1.90e-08 &      \\ 
     &           &    2 &  5.90e-08 &  5.33e-16 &      \\ 
     &           &    3 &  1.87e-08 &  1.68e-15 &      \\ 
     &           &    4 &  1.87e-08 &  5.35e-16 &      \\ 
     &           &    5 &  1.87e-08 &  5.34e-16 &      \\ 
     &           &    6 &  5.90e-08 &  5.33e-16 &      \\ 
     &           &    7 &  1.87e-08 &  1.68e-15 &      \\ 
     &           &    8 &  1.87e-08 &  5.35e-16 &      \\ 
     &           &    9 &  1.87e-08 &  5.34e-16 &      \\ 
     &           &   10 &  5.90e-08 &  5.33e-16 &      \\ 
 614 &  6.13e+03 &   10 &           &           & iters  \\ 
 \hdashline 
     &           &    1 &  5.53e-08 &  3.07e-08 &      \\ 
     &           &    2 &  2.25e-08 &  1.58e-15 &      \\ 
     &           &    3 &  2.25e-08 &  6.42e-16 &      \\ 
     &           &    4 &  5.52e-08 &  6.42e-16 &      \\ 
     &           &    5 &  2.25e-08 &  1.58e-15 &      \\ 
     &           &    6 &  2.25e-08 &  6.43e-16 &      \\ 
     &           &    7 &  2.25e-08 &  6.42e-16 &      \\ 
     &           &    8 &  5.53e-08 &  6.41e-16 &      \\ 
     &           &    9 &  2.25e-08 &  1.58e-15 &      \\ 
     &           &   10 &  2.25e-08 &  6.42e-16 &      \\ 
 615 &  6.14e+03 &   10 &           &           & iters  \\ 
 \hdashline 
     &           &    1 &  2.46e-08 &  2.34e-08 &      \\ 
     &           &    2 &  2.46e-08 &  7.03e-16 &      \\ 
     &           &    3 &  2.46e-08 &  7.02e-16 &      \\ 
     &           &    4 &  2.45e-08 &  7.01e-16 &      \\ 
     &           &    5 &  5.32e-08 &  7.00e-16 &      \\ 
     &           &    6 &  2.46e-08 &  1.52e-15 &      \\ 
     &           &    7 &  2.45e-08 &  7.02e-16 &      \\ 
     &           &    8 &  5.32e-08 &  7.01e-16 &      \\ 
     &           &    9 &  2.46e-08 &  1.52e-15 &      \\ 
     &           &   10 &  2.46e-08 &  7.02e-16 &      \\ 
 616 &  6.15e+03 &   10 &           &           & iters  \\ 
 \hdashline 
     &           &    1 &  3.23e-08 &  1.91e-08 &      \\ 
     &           &    2 &  3.23e-08 &  9.23e-16 &      \\ 
     &           &    3 &  3.23e-08 &  9.22e-16 &      \\ 
     &           &    4 &  4.55e-08 &  9.21e-16 &      \\ 
     &           &    5 &  3.23e-08 &  1.30e-15 &      \\ 
     &           &    6 &  4.55e-08 &  9.21e-16 &      \\ 
     &           &    7 &  3.23e-08 &  1.30e-15 &      \\ 
     &           &    8 &  3.22e-08 &  9.22e-16 &      \\ 
     &           &    9 &  4.55e-08 &  9.20e-16 &      \\ 
     &           &   10 &  3.23e-08 &  1.30e-15 &      \\ 
 617 &  6.16e+03 &   10 &           &           & iters  \\ 
 \hdashline 
     &           &    1 &  4.48e-08 &  2.87e-08 &      \\ 
     &           &    2 &  3.29e-08 &  1.28e-15 &      \\ 
     &           &    3 &  4.48e-08 &  9.40e-16 &      \\ 
     &           &    4 &  3.30e-08 &  1.28e-15 &      \\ 
     &           &    5 &  4.48e-08 &  9.41e-16 &      \\ 
     &           &    6 &  3.30e-08 &  1.28e-15 &      \\ 
     &           &    7 &  4.48e-08 &  9.41e-16 &      \\ 
     &           &    8 &  3.30e-08 &  1.28e-15 &      \\ 
     &           &    9 &  3.29e-08 &  9.42e-16 &      \\ 
     &           &   10 &  4.48e-08 &  9.40e-16 &      \\ 
 618 &  6.17e+03 &   10 &           &           & iters  \\ 
 \hdashline 
     &           &    1 &  3.21e-09 &  2.35e-08 &      \\ 
     &           &    2 &  3.21e-09 &  9.18e-17 &      \\ 
     &           &    3 &  3.21e-09 &  9.16e-17 &      \\ 
     &           &    4 &  3.20e-09 &  9.15e-17 &      \\ 
     &           &    5 &  3.20e-09 &  9.14e-17 &      \\ 
     &           &    6 &  3.19e-09 &  9.12e-17 &      \\ 
     &           &    7 &  7.45e-08 &  9.11e-17 &      \\ 
     &           &    8 &  8.10e-08 &  2.13e-15 &      \\ 
     &           &    9 &  7.45e-08 &  2.31e-15 &      \\ 
     &           &   10 &  8.10e-08 &  2.13e-15 &      \\ 
 619 &  6.18e+03 &   10 &           &           & iters  \\ 
 \hdashline 
     &           &    1 &  4.32e-08 &  3.59e-08 &      \\ 
     &           &    2 &  3.46e-08 &  1.23e-15 &      \\ 
     &           &    3 &  4.32e-08 &  9.86e-16 &      \\ 
     &           &    4 &  3.46e-08 &  1.23e-15 &      \\ 
     &           &    5 &  3.45e-08 &  9.87e-16 &      \\ 
     &           &    6 &  4.32e-08 &  9.85e-16 &      \\ 
     &           &    7 &  3.45e-08 &  1.23e-15 &      \\ 
     &           &    8 &  4.32e-08 &  9.86e-16 &      \\ 
     &           &    9 &  3.45e-08 &  1.23e-15 &      \\ 
     &           &   10 &  4.32e-08 &  9.86e-16 &      \\ 
 620 &  6.19e+03 &   10 &           &           & iters  \\ 
 \hdashline 
     &           &    1 &  7.41e-09 &  4.42e-08 &      \\ 
     &           &    2 &  7.40e-09 &  2.11e-16 &      \\ 
     &           &    3 &  7.39e-09 &  2.11e-16 &      \\ 
     &           &    4 &  7.37e-09 &  2.11e-16 &      \\ 
     &           &    5 &  7.36e-09 &  2.10e-16 &      \\ 
     &           &    6 &  7.35e-09 &  2.10e-16 &      \\ 
     &           &    7 &  7.34e-09 &  2.10e-16 &      \\ 
     &           &    8 &  7.33e-09 &  2.10e-16 &      \\ 
     &           &    9 &  7.04e-08 &  2.09e-16 &      \\ 
     &           &   10 &  4.63e-08 &  2.01e-15 &      \\ 
 621 &  6.20e+03 &   10 &           &           & iters  \\ 
 \hdashline 
     &           &    1 &  1.47e-08 &  5.49e-08 &      \\ 
     &           &    2 &  1.47e-08 &  4.20e-16 &      \\ 
     &           &    3 &  6.30e-08 &  4.19e-16 &      \\ 
     &           &    4 &  5.36e-08 &  1.80e-15 &      \\ 
     &           &    5 &  1.47e-08 &  1.53e-15 &      \\ 
     &           &    6 &  6.30e-08 &  4.19e-16 &      \\ 
     &           &    7 &  1.47e-08 &  1.80e-15 &      \\ 
     &           &    8 &  1.47e-08 &  4.21e-16 &      \\ 
     &           &    9 &  1.47e-08 &  4.20e-16 &      \\ 
     &           &   10 &  1.47e-08 &  4.19e-16 &      \\ 
 622 &  6.21e+03 &   10 &           &           & iters  \\ 
 \hdashline 
     &           &    1 &  3.26e-08 &  6.93e-08 &      \\ 
     &           &    2 &  4.52e-08 &  9.29e-16 &      \\ 
     &           &    3 &  3.26e-08 &  1.29e-15 &      \\ 
     &           &    4 &  3.25e-08 &  9.30e-16 &      \\ 
     &           &    5 &  4.52e-08 &  9.29e-16 &      \\ 
     &           &    6 &  3.26e-08 &  1.29e-15 &      \\ 
     &           &    7 &  4.52e-08 &  9.29e-16 &      \\ 
     &           &    8 &  3.26e-08 &  1.29e-15 &      \\ 
     &           &    9 &  4.52e-08 &  9.30e-16 &      \\ 
     &           &   10 &  3.26e-08 &  1.29e-15 &      \\ 
 623 &  6.22e+03 &   10 &           &           & iters  \\ 
 \hdashline 
     &           &    1 &  2.33e-08 &  3.14e-08 &      \\ 
     &           &    2 &  2.33e-08 &  6.66e-16 &      \\ 
     &           &    3 &  5.44e-08 &  6.65e-16 &      \\ 
     &           &    4 &  2.33e-08 &  1.55e-15 &      \\ 
     &           &    5 &  2.33e-08 &  6.66e-16 &      \\ 
     &           &    6 &  2.33e-08 &  6.65e-16 &      \\ 
     &           &    7 &  5.44e-08 &  6.64e-16 &      \\ 
     &           &    8 &  2.33e-08 &  1.55e-15 &      \\ 
     &           &    9 &  2.33e-08 &  6.66e-16 &      \\ 
     &           &   10 &  5.44e-08 &  6.65e-16 &      \\ 
 624 &  6.23e+03 &   10 &           &           & iters  \\ 
 \hdashline 
     &           &    1 &  4.28e-08 &  4.58e-08 &      \\ 
     &           &    2 &  3.50e-08 &  1.22e-15 &      \\ 
     &           &    3 &  4.28e-08 &  9.98e-16 &      \\ 
     &           &    4 &  3.50e-08 &  1.22e-15 &      \\ 
     &           &    5 &  4.28e-08 &  9.99e-16 &      \\ 
     &           &    6 &  3.50e-08 &  1.22e-15 &      \\ 
     &           &    7 &  3.49e-08 &  9.99e-16 &      \\ 
     &           &    8 &  4.28e-08 &  9.97e-16 &      \\ 
     &           &    9 &  3.50e-08 &  1.22e-15 &      \\ 
     &           &   10 &  4.28e-08 &  9.98e-16 &      \\ 
 625 &  6.24e+03 &   10 &           &           & iters  \\ 
 \hdashline 
     &           &    1 &  2.70e-08 &  4.99e-08 &      \\ 
     &           &    2 &  5.07e-08 &  7.70e-16 &      \\ 
     &           &    3 &  2.70e-08 &  1.45e-15 &      \\ 
     &           &    4 &  2.70e-08 &  7.71e-16 &      \\ 
     &           &    5 &  5.08e-08 &  7.70e-16 &      \\ 
     &           &    6 &  2.70e-08 &  1.45e-15 &      \\ 
     &           &    7 &  2.70e-08 &  7.71e-16 &      \\ 
     &           &    8 &  5.08e-08 &  7.70e-16 &      \\ 
     &           &    9 &  2.70e-08 &  1.45e-15 &      \\ 
     &           &   10 &  2.70e-08 &  7.71e-16 &      \\ 
 626 &  6.25e+03 &   10 &           &           & iters  \\ 
 \hdashline 
     &           &    1 &  4.89e-08 &  1.11e-08 &      \\ 
     &           &    2 &  1.00e-08 &  1.40e-15 &      \\ 
     &           &    3 &  9.98e-09 &  2.85e-16 &      \\ 
     &           &    4 &  9.97e-09 &  2.85e-16 &      \\ 
     &           &    5 &  6.77e-08 &  2.84e-16 &      \\ 
     &           &    6 &  4.89e-08 &  1.93e-15 &      \\ 
     &           &    7 &  9.98e-09 &  1.40e-15 &      \\ 
     &           &    8 &  9.97e-09 &  2.85e-16 &      \\ 
     &           &    9 &  6.77e-08 &  2.84e-16 &      \\ 
     &           &   10 &  4.89e-08 &  1.93e-15 &      \\ 
 627 &  6.26e+03 &   10 &           &           & iters  \\ 
 \hdashline 
     &           &    1 &  1.99e-08 &  2.36e-08 &      \\ 
     &           &    2 &  1.99e-08 &  5.68e-16 &      \\ 
     &           &    3 &  5.79e-08 &  5.67e-16 &      \\ 
     &           &    4 &  1.99e-08 &  1.65e-15 &      \\ 
     &           &    5 &  1.99e-08 &  5.69e-16 &      \\ 
     &           &    6 &  1.99e-08 &  5.68e-16 &      \\ 
     &           &    7 &  5.79e-08 &  5.67e-16 &      \\ 
     &           &    8 &  1.99e-08 &  1.65e-15 &      \\ 
     &           &    9 &  1.99e-08 &  5.68e-16 &      \\ 
     &           &   10 &  1.99e-08 &  5.68e-16 &      \\ 
 628 &  6.27e+03 &   10 &           &           & iters  \\ 
 \hdashline 
     &           &    1 &  3.47e-08 &  3.94e-08 &      \\ 
     &           &    2 &  4.20e-09 &  9.90e-16 &      \\ 
     &           &    3 &  4.20e-09 &  1.20e-16 &      \\ 
     &           &    4 &  3.47e-08 &  1.20e-16 &      \\ 
     &           &    5 &  4.24e-09 &  9.89e-16 &      \\ 
     &           &    6 &  4.23e-09 &  1.21e-16 &      \\ 
     &           &    7 &  4.23e-09 &  1.21e-16 &      \\ 
     &           &    8 &  4.22e-09 &  1.21e-16 &      \\ 
     &           &    9 &  4.21e-09 &  1.20e-16 &      \\ 
     &           &   10 &  4.21e-09 &  1.20e-16 &      \\ 
 629 &  6.28e+03 &   10 &           &           & iters  \\ 
 \hdashline 
     &           &    1 &  5.26e-09 &  3.86e-08 &      \\ 
     &           &    2 &  5.25e-09 &  1.50e-16 &      \\ 
     &           &    3 &  3.36e-08 &  1.50e-16 &      \\ 
     &           &    4 &  5.29e-09 &  9.59e-16 &      \\ 
     &           &    5 &  5.28e-09 &  1.51e-16 &      \\ 
     &           &    6 &  5.28e-09 &  1.51e-16 &      \\ 
     &           &    7 &  5.27e-09 &  1.51e-16 &      \\ 
     &           &    8 &  5.26e-09 &  1.50e-16 &      \\ 
     &           &    9 &  5.25e-09 &  1.50e-16 &      \\ 
     &           &   10 &  5.25e-09 &  1.50e-16 &      \\ 
 630 &  6.29e+03 &   10 &           &           & iters  \\ 
 \hdashline 
     &           &    1 &  2.98e-08 &  4.02e-08 &      \\ 
     &           &    2 &  2.98e-08 &  8.52e-16 &      \\ 
     &           &    3 &  4.79e-08 &  8.50e-16 &      \\ 
     &           &    4 &  2.98e-08 &  1.37e-15 &      \\ 
     &           &    5 &  4.79e-08 &  8.51e-16 &      \\ 
     &           &    6 &  2.98e-08 &  1.37e-15 &      \\ 
     &           &    7 &  2.98e-08 &  8.52e-16 &      \\ 
     &           &    8 &  4.79e-08 &  8.51e-16 &      \\ 
     &           &    9 &  2.98e-08 &  1.37e-15 &      \\ 
     &           &   10 &  2.98e-08 &  8.51e-16 &      \\ 
 631 &  6.30e+03 &   10 &           &           & iters  \\ 
 \hdashline 
     &           &    1 &  9.12e-08 &  1.57e-08 &      \\ 
     &           &    2 &  6.43e-08 &  2.60e-15 &      \\ 
     &           &    3 &  1.34e-08 &  1.84e-15 &      \\ 
     &           &    4 &  1.34e-08 &  3.84e-16 &      \\ 
     &           &    5 &  1.34e-08 &  3.83e-16 &      \\ 
     &           &    6 &  1.34e-08 &  3.83e-16 &      \\ 
     &           &    7 &  6.43e-08 &  3.82e-16 &      \\ 
     &           &    8 &  9.12e-08 &  1.84e-15 &      \\ 
     &           &    9 &  6.44e-08 &  2.60e-15 &      \\ 
     &           &   10 &  1.34e-08 &  1.84e-15 &      \\ 
 632 &  6.31e+03 &   10 &           &           & iters  \\ 
 \hdashline 
     &           &    1 &  3.29e-09 &  2.03e-08 &      \\ 
     &           &    2 &  3.28e-09 &  9.39e-17 &      \\ 
     &           &    3 &  3.28e-09 &  9.37e-17 &      \\ 
     &           &    4 &  3.27e-09 &  9.36e-17 &      \\ 
     &           &    5 &  3.27e-09 &  9.35e-17 &      \\ 
     &           &    6 &  3.27e-09 &  9.33e-17 &      \\ 
     &           &    7 &  3.26e-09 &  9.32e-17 &      \\ 
     &           &    8 &  3.26e-09 &  9.31e-17 &      \\ 
     &           &    9 &  3.25e-09 &  9.29e-17 &      \\ 
     &           &   10 &  3.25e-09 &  9.28e-17 &      \\ 
 633 &  6.32e+03 &   10 &           &           & iters  \\ 
 \hdashline 
     &           &    1 &  4.30e-08 &  4.95e-08 &      \\ 
     &           &    2 &  3.47e-08 &  1.23e-15 &      \\ 
     &           &    3 &  3.47e-08 &  9.91e-16 &      \\ 
     &           &    4 &  4.31e-08 &  9.90e-16 &      \\ 
     &           &    5 &  3.47e-08 &  1.23e-15 &      \\ 
     &           &    6 &  4.30e-08 &  9.90e-16 &      \\ 
     &           &    7 &  3.47e-08 &  1.23e-15 &      \\ 
     &           &    8 &  4.30e-08 &  9.91e-16 &      \\ 
     &           &    9 &  3.47e-08 &  1.23e-15 &      \\ 
     &           &   10 &  4.30e-08 &  9.91e-16 &      \\ 
 634 &  6.33e+03 &   10 &           &           & iters  \\ 
 \hdashline 
     &           &    1 &  4.04e-08 &  9.89e-09 &      \\ 
     &           &    2 &  1.52e-09 &  1.15e-15 &      \\ 
     &           &    3 &  1.51e-09 &  4.33e-17 &      \\ 
     &           &    4 &  1.51e-09 &  4.32e-17 &      \\ 
     &           &    5 &  1.51e-09 &  4.31e-17 &      \\ 
     &           &    6 &  1.51e-09 &  4.31e-17 &      \\ 
     &           &    7 &  1.50e-09 &  4.30e-17 &      \\ 
     &           &    8 &  1.50e-09 &  4.29e-17 &      \\ 
     &           &    9 &  1.50e-09 &  4.29e-17 &      \\ 
     &           &   10 &  1.50e-09 &  4.28e-17 &      \\ 
 635 &  6.34e+03 &   10 &           &           & iters  \\ 
 \hdashline 
     &           &    1 &  3.00e-08 &  8.28e-08 &      \\ 
     &           &    2 &  4.77e-08 &  8.57e-16 &      \\ 
     &           &    3 &  3.00e-08 &  1.36e-15 &      \\ 
     &           &    4 &  3.00e-08 &  8.57e-16 &      \\ 
     &           &    5 &  4.77e-08 &  8.56e-16 &      \\ 
     &           &    6 &  3.00e-08 &  1.36e-15 &      \\ 
     &           &    7 &  4.77e-08 &  8.57e-16 &      \\ 
     &           &    8 &  3.01e-08 &  1.36e-15 &      \\ 
     &           &    9 &  3.00e-08 &  8.58e-16 &      \\ 
     &           &   10 &  4.77e-08 &  8.56e-16 &      \\ 
 636 &  6.35e+03 &   10 &           &           & iters  \\ 
 \hdashline 
     &           &    1 &  8.95e-08 &  7.31e-08 &      \\ 
     &           &    2 &  6.60e-08 &  2.55e-15 &      \\ 
     &           &    3 &  1.18e-08 &  1.88e-15 &      \\ 
     &           &    4 &  1.17e-08 &  3.36e-16 &      \\ 
     &           &    5 &  1.17e-08 &  3.35e-16 &      \\ 
     &           &    6 &  1.17e-08 &  3.35e-16 &      \\ 
     &           &    7 &  1.17e-08 &  3.34e-16 &      \\ 
     &           &    8 &  6.60e-08 &  3.34e-16 &      \\ 
     &           &    9 &  1.18e-08 &  1.88e-15 &      \\ 
     &           &   10 &  1.18e-08 &  3.36e-16 &      \\ 
 637 &  6.36e+03 &   10 &           &           & iters  \\ 
 \hdashline 
     &           &    1 &  7.17e-08 &  2.97e-09 &      \\ 
     &           &    2 &  8.38e-08 &  2.05e-15 &      \\ 
     &           &    3 &  7.17e-08 &  2.39e-15 &      \\ 
     &           &    4 &  6.06e-09 &  2.05e-15 &      \\ 
     &           &    5 &  6.05e-09 &  1.73e-16 &      \\ 
     &           &    6 &  6.05e-09 &  1.73e-16 &      \\ 
     &           &    7 &  6.04e-09 &  1.73e-16 &      \\ 
     &           &    8 &  6.03e-09 &  1.72e-16 &      \\ 
     &           &    9 &  6.02e-09 &  1.72e-16 &      \\ 
     &           &   10 &  6.01e-09 &  1.72e-16 &      \\ 
 638 &  6.37e+03 &   10 &           &           & iters  \\ 
 \hdashline 
     &           &    1 &  6.02e-09 &  2.71e-08 &      \\ 
     &           &    2 &  6.01e-09 &  1.72e-16 &      \\ 
     &           &    3 &  6.00e-09 &  1.71e-16 &      \\ 
     &           &    4 &  5.99e-09 &  1.71e-16 &      \\ 
     &           &    5 &  5.98e-09 &  1.71e-16 &      \\ 
     &           &    6 &  7.17e-08 &  1.71e-16 &      \\ 
     &           &    7 &  8.38e-08 &  2.05e-15 &      \\ 
     &           &    8 &  7.17e-08 &  2.39e-15 &      \\ 
     &           &    9 &  6.06e-09 &  2.05e-15 &      \\ 
     &           &   10 &  6.05e-09 &  1.73e-16 &      \\ 
 639 &  6.38e+03 &   10 &           &           & iters  \\ 
 \hdashline 
     &           &    1 &  1.43e-08 &  7.85e-09 &      \\ 
     &           &    2 &  1.42e-08 &  4.07e-16 &      \\ 
     &           &    3 &  1.42e-08 &  4.06e-16 &      \\ 
     &           &    4 &  1.42e-08 &  4.06e-16 &      \\ 
     &           &    5 &  6.35e-08 &  4.05e-16 &      \\ 
     &           &    6 &  1.43e-08 &  1.81e-15 &      \\ 
     &           &    7 &  1.42e-08 &  4.07e-16 &      \\ 
     &           &    8 &  1.42e-08 &  4.07e-16 &      \\ 
     &           &    9 &  1.42e-08 &  4.06e-16 &      \\ 
     &           &   10 &  6.35e-08 &  4.05e-16 &      \\ 
 640 &  6.39e+03 &   10 &           &           & iters  \\ 
 \hdashline 
     &           &    1 &  1.64e-08 &  2.45e-08 &      \\ 
     &           &    2 &  6.13e-08 &  4.69e-16 &      \\ 
     &           &    3 &  1.65e-08 &  1.75e-15 &      \\ 
     &           &    4 &  1.65e-08 &  4.71e-16 &      \\ 
     &           &    5 &  1.64e-08 &  4.70e-16 &      \\ 
     &           &    6 &  1.64e-08 &  4.69e-16 &      \\ 
     &           &    7 &  6.13e-08 &  4.69e-16 &      \\ 
     &           &    8 &  1.65e-08 &  1.75e-15 &      \\ 
     &           &    9 &  1.65e-08 &  4.70e-16 &      \\ 
     &           &   10 &  1.64e-08 &  4.70e-16 &      \\ 
 641 &  6.40e+03 &   10 &           &           & iters  \\ 
 \hdashline 
     &           &    1 &  6.60e-08 &  2.03e-08 &      \\ 
     &           &    2 &  8.95e-08 &  1.88e-15 &      \\ 
     &           &    3 &  6.60e-08 &  2.55e-15 &      \\ 
     &           &    4 &  1.17e-08 &  1.88e-15 &      \\ 
     &           &    5 &  1.17e-08 &  3.35e-16 &      \\ 
     &           &    6 &  1.17e-08 &  3.34e-16 &      \\ 
     &           &    7 &  1.17e-08 &  3.34e-16 &      \\ 
     &           &    8 &  6.60e-08 &  3.34e-16 &      \\ 
     &           &    9 &  1.18e-08 &  1.88e-15 &      \\ 
     &           &   10 &  1.17e-08 &  3.36e-16 &      \\ 
 642 &  6.41e+03 &   10 &           &           & iters  \\ 
 \hdashline 
     &           &    1 &  1.43e-08 &  3.69e-08 &      \\ 
     &           &    2 &  1.42e-08 &  4.07e-16 &      \\ 
     &           &    3 &  1.42e-08 &  4.06e-16 &      \\ 
     &           &    4 &  6.35e-08 &  4.06e-16 &      \\ 
     &           &    5 &  1.43e-08 &  1.81e-15 &      \\ 
     &           &    6 &  1.43e-08 &  4.08e-16 &      \\ 
     &           &    7 &  1.42e-08 &  4.07e-16 &      \\ 
     &           &    8 &  1.42e-08 &  4.07e-16 &      \\ 
     &           &    9 &  1.42e-08 &  4.06e-16 &      \\ 
     &           &   10 &  6.35e-08 &  4.05e-16 &      \\ 
 643 &  6.42e+03 &   10 &           &           & iters  \\ 
 \hdashline 
     &           &    1 &  4.26e-08 &  4.45e-09 &      \\ 
     &           &    2 &  4.25e-08 &  1.22e-15 &      \\ 
     &           &    3 &  3.52e-08 &  1.21e-15 &      \\ 
     &           &    4 &  3.52e-08 &  1.01e-15 &      \\ 
     &           &    5 &  4.26e-08 &  1.00e-15 &      \\ 
     &           &    6 &  3.52e-08 &  1.21e-15 &      \\ 
     &           &    7 &  4.25e-08 &  1.00e-15 &      \\ 
     &           &    8 &  3.52e-08 &  1.21e-15 &      \\ 
     &           &    9 &  4.25e-08 &  1.00e-15 &      \\ 
     &           &   10 &  3.52e-08 &  1.21e-15 &      \\ 
 644 &  6.43e+03 &   10 &           &           & iters  \\ 
 \hdashline 
     &           &    1 &  1.95e-08 &  1.76e-08 &      \\ 
     &           &    2 &  5.82e-08 &  5.56e-16 &      \\ 
     &           &    3 &  1.95e-08 &  1.66e-15 &      \\ 
     &           &    4 &  1.95e-08 &  5.57e-16 &      \\ 
     &           &    5 &  1.95e-08 &  5.56e-16 &      \\ 
     &           &    6 &  5.82e-08 &  5.56e-16 &      \\ 
     &           &    7 &  1.95e-08 &  1.66e-15 &      \\ 
     &           &    8 &  1.95e-08 &  5.57e-16 &      \\ 
     &           &    9 &  1.95e-08 &  5.56e-16 &      \\ 
     &           &   10 &  5.82e-08 &  5.56e-16 &      \\ 
 645 &  6.44e+03 &   10 &           &           & iters  \\ 
 \hdashline 
     &           &    1 &  2.57e-08 &  4.16e-08 &      \\ 
     &           &    2 &  2.57e-08 &  7.34e-16 &      \\ 
     &           &    3 &  5.21e-08 &  7.33e-16 &      \\ 
     &           &    4 &  2.57e-08 &  1.49e-15 &      \\ 
     &           &    5 &  2.57e-08 &  7.34e-16 &      \\ 
     &           &    6 &  5.21e-08 &  7.33e-16 &      \\ 
     &           &    7 &  2.57e-08 &  1.49e-15 &      \\ 
     &           &    8 &  2.57e-08 &  7.34e-16 &      \\ 
     &           &    9 &  5.21e-08 &  7.33e-16 &      \\ 
     &           &   10 &  2.57e-08 &  1.49e-15 &      \\ 
 646 &  6.45e+03 &   10 &           &           & iters  \\ 
 \hdashline 
     &           &    1 &  2.70e-08 &  1.84e-08 &      \\ 
     &           &    2 &  5.07e-08 &  7.72e-16 &      \\ 
     &           &    3 &  2.71e-08 &  1.45e-15 &      \\ 
     &           &    4 &  2.70e-08 &  7.73e-16 &      \\ 
     &           &    5 &  5.07e-08 &  7.72e-16 &      \\ 
     &           &    6 &  2.71e-08 &  1.45e-15 &      \\ 
     &           &    7 &  2.70e-08 &  7.73e-16 &      \\ 
     &           &    8 &  5.07e-08 &  7.72e-16 &      \\ 
     &           &    9 &  2.71e-08 &  1.45e-15 &      \\ 
     &           &   10 &  2.70e-08 &  7.72e-16 &      \\ 
 647 &  6.46e+03 &   10 &           &           & iters  \\ 
 \hdashline 
     &           &    1 &  4.01e-08 &  1.98e-09 &      \\ 
     &           &    2 &  3.77e-08 &  1.14e-15 &      \\ 
     &           &    3 &  4.01e-08 &  1.08e-15 &      \\ 
     &           &    4 &  3.77e-08 &  1.14e-15 &      \\ 
     &           &    5 &  4.01e-08 &  1.08e-15 &      \\ 
     &           &    6 &  3.77e-08 &  1.14e-15 &      \\ 
     &           &    7 &  4.01e-08 &  1.08e-15 &      \\ 
     &           &    8 &  3.77e-08 &  1.14e-15 &      \\ 
     &           &    9 &  4.00e-08 &  1.08e-15 &      \\ 
     &           &   10 &  3.77e-08 &  1.14e-15 &      \\ 
 648 &  6.47e+03 &   10 &           &           & iters  \\ 
 \hdashline 
     &           &    1 &  5.46e-08 &  9.76e-09 &      \\ 
     &           &    2 &  2.32e-08 &  1.56e-15 &      \\ 
     &           &    3 &  2.32e-08 &  6.62e-16 &      \\ 
     &           &    4 &  5.45e-08 &  6.61e-16 &      \\ 
     &           &    5 &  2.32e-08 &  1.56e-15 &      \\ 
     &           &    6 &  2.32e-08 &  6.63e-16 &      \\ 
     &           &    7 &  2.32e-08 &  6.62e-16 &      \\ 
     &           &    8 &  5.46e-08 &  6.61e-16 &      \\ 
     &           &    9 &  2.32e-08 &  1.56e-15 &      \\ 
     &           &   10 &  2.32e-08 &  6.62e-16 &      \\ 
 649 &  6.48e+03 &   10 &           &           & iters  \\ 
 \hdashline 
     &           &    1 &  2.79e-08 &  9.04e-09 &      \\ 
     &           &    2 &  4.98e-08 &  7.97e-16 &      \\ 
     &           &    3 &  2.80e-08 &  1.42e-15 &      \\ 
     &           &    4 &  2.79e-08 &  7.98e-16 &      \\ 
     &           &    5 &  4.98e-08 &  7.97e-16 &      \\ 
     &           &    6 &  2.80e-08 &  1.42e-15 &      \\ 
     &           &    7 &  2.79e-08 &  7.98e-16 &      \\ 
     &           &    8 &  4.98e-08 &  7.97e-16 &      \\ 
     &           &    9 &  2.79e-08 &  1.42e-15 &      \\ 
     &           &   10 &  2.79e-08 &  7.98e-16 &      \\ 
 650 &  6.49e+03 &   10 &           &           & iters  \\ 
 \hdashline 
     &           &    1 &  4.54e-09 &  1.84e-08 &      \\ 
     &           &    2 &  4.54e-09 &  1.30e-16 &      \\ 
     &           &    3 &  4.53e-09 &  1.29e-16 &      \\ 
     &           &    4 &  4.52e-09 &  1.29e-16 &      \\ 
     &           &    5 &  4.52e-09 &  1.29e-16 &      \\ 
     &           &    6 &  4.51e-09 &  1.29e-16 &      \\ 
     &           &    7 &  4.50e-09 &  1.29e-16 &      \\ 
     &           &    8 &  4.50e-09 &  1.29e-16 &      \\ 
     &           &    9 &  4.49e-09 &  1.28e-16 &      \\ 
     &           &   10 &  3.44e-08 &  1.28e-16 &      \\ 
 651 &  6.50e+03 &   10 &           &           & iters  \\ 
 \hdashline 
     &           &    1 &  1.44e-08 &  1.32e-08 &      \\ 
     &           &    2 &  1.44e-08 &  4.11e-16 &      \\ 
     &           &    3 &  1.44e-08 &  4.10e-16 &      \\ 
     &           &    4 &  2.45e-08 &  4.10e-16 &      \\ 
     &           &    5 &  1.44e-08 &  6.99e-16 &      \\ 
     &           &    6 &  2.45e-08 &  4.10e-16 &      \\ 
     &           &    7 &  1.44e-08 &  6.99e-16 &      \\ 
     &           &    8 &  1.44e-08 &  4.11e-16 &      \\ 
     &           &    9 &  2.45e-08 &  4.10e-16 &      \\ 
     &           &   10 &  1.44e-08 &  6.99e-16 &      \\ 
 652 &  6.51e+03 &   10 &           &           & iters  \\ 
 \hdashline 
     &           &    1 &  1.02e-08 &  1.99e-08 &      \\ 
     &           &    2 &  1.02e-08 &  2.91e-16 &      \\ 
     &           &    3 &  1.02e-08 &  2.90e-16 &      \\ 
     &           &    4 &  6.75e-08 &  2.90e-16 &      \\ 
     &           &    5 &  1.02e-08 &  1.93e-15 &      \\ 
     &           &    6 &  1.02e-08 &  2.92e-16 &      \\ 
     &           &    7 &  1.02e-08 &  2.92e-16 &      \\ 
     &           &    8 &  1.02e-08 &  2.91e-16 &      \\ 
     &           &    9 &  1.02e-08 &  2.91e-16 &      \\ 
     &           &   10 &  1.02e-08 &  2.91e-16 &      \\ 
 653 &  6.52e+03 &   10 &           &           & iters  \\ 
 \hdashline 
     &           &    1 &  9.09e-09 &  8.67e-09 &      \\ 
     &           &    2 &  9.08e-09 &  2.59e-16 &      \\ 
     &           &    3 &  9.07e-09 &  2.59e-16 &      \\ 
     &           &    4 &  9.05e-09 &  2.59e-16 &      \\ 
     &           &    5 &  6.86e-08 &  2.58e-16 &      \\ 
     &           &    6 &  9.14e-09 &  1.96e-15 &      \\ 
     &           &    7 &  9.12e-09 &  2.61e-16 &      \\ 
     &           &    8 &  9.11e-09 &  2.60e-16 &      \\ 
     &           &    9 &  9.10e-09 &  2.60e-16 &      \\ 
     &           &   10 &  9.09e-09 &  2.60e-16 &      \\ 
 654 &  6.53e+03 &   10 &           &           & iters  \\ 
 \hdashline 
     &           &    1 &  2.87e-08 &  1.21e-08 &      \\ 
     &           &    2 &  4.90e-08 &  8.19e-16 &      \\ 
     &           &    3 &  2.87e-08 &  1.40e-15 &      \\ 
     &           &    4 &  4.90e-08 &  8.20e-16 &      \\ 
     &           &    5 &  2.88e-08 &  1.40e-15 &      \\ 
     &           &    6 &  2.87e-08 &  8.21e-16 &      \\ 
     &           &    7 &  4.90e-08 &  8.20e-16 &      \\ 
     &           &    8 &  2.88e-08 &  1.40e-15 &      \\ 
     &           &    9 &  2.87e-08 &  8.21e-16 &      \\ 
     &           &   10 &  4.90e-08 &  8.19e-16 &      \\ 
 655 &  6.54e+03 &   10 &           &           & iters  \\ 
 \hdashline 
     &           &    1 &  7.20e-09 &  8.55e-09 &      \\ 
     &           &    2 &  7.19e-09 &  2.06e-16 &      \\ 
     &           &    3 &  7.18e-09 &  2.05e-16 &      \\ 
     &           &    4 &  7.17e-09 &  2.05e-16 &      \\ 
     &           &    5 &  7.16e-09 &  2.05e-16 &      \\ 
     &           &    6 &  7.15e-09 &  2.04e-16 &      \\ 
     &           &    7 &  7.14e-09 &  2.04e-16 &      \\ 
     &           &    8 &  7.06e-08 &  2.04e-16 &      \\ 
     &           &    9 &  7.23e-09 &  2.01e-15 &      \\ 
     &           &   10 &  7.22e-09 &  2.06e-16 &      \\ 
 656 &  6.55e+03 &   10 &           &           & iters  \\ 
 \hdashline 
     &           &    1 &  3.66e-09 &  8.20e-09 &      \\ 
     &           &    2 &  3.66e-09 &  1.05e-16 &      \\ 
     &           &    3 &  3.65e-09 &  1.04e-16 &      \\ 
     &           &    4 &  3.65e-09 &  1.04e-16 &      \\ 
     &           &    5 &  3.64e-09 &  1.04e-16 &      \\ 
     &           &    6 &  3.64e-09 &  1.04e-16 &      \\ 
     &           &    7 &  3.63e-09 &  1.04e-16 &      \\ 
     &           &    8 &  3.63e-09 &  1.04e-16 &      \\ 
     &           &    9 &  3.62e-09 &  1.04e-16 &      \\ 
     &           &   10 &  3.62e-09 &  1.03e-16 &      \\ 
 657 &  6.56e+03 &   10 &           &           & iters  \\ 
 \hdashline 
     &           &    1 &  7.72e-08 &  8.56e-09 &      \\ 
     &           &    2 &  7.83e-08 &  2.20e-15 &      \\ 
     &           &    3 &  7.72e-08 &  2.23e-15 &      \\ 
     &           &    4 &  7.83e-08 &  2.20e-15 &      \\ 
     &           &    5 &  7.72e-08 &  2.23e-15 &      \\ 
     &           &    6 &  7.83e-08 &  2.20e-15 &      \\ 
     &           &    7 &  7.72e-08 &  2.23e-15 &      \\ 
     &           &    8 &  7.83e-08 &  2.20e-15 &      \\ 
     &           &    9 &  7.72e-08 &  2.23e-15 &      \\ 
     &           &   10 &  7.83e-08 &  2.20e-15 &      \\ 
 658 &  6.57e+03 &   10 &           &           & iters  \\ 
 \hdashline 
     &           &    1 &  5.63e-08 &  6.64e-08 &      \\ 
     &           &    2 &  2.15e-08 &  1.61e-15 &      \\ 
     &           &    3 &  2.14e-08 &  6.13e-16 &      \\ 
     &           &    4 &  5.63e-08 &  6.12e-16 &      \\ 
     &           &    5 &  2.15e-08 &  1.61e-15 &      \\ 
     &           &    6 &  2.15e-08 &  6.13e-16 &      \\ 
     &           &    7 &  2.14e-08 &  6.12e-16 &      \\ 
     &           &    8 &  5.63e-08 &  6.11e-16 &      \\ 
     &           &    9 &  2.15e-08 &  1.61e-15 &      \\ 
     &           &   10 &  2.14e-08 &  6.13e-16 &      \\ 
 659 &  6.58e+03 &   10 &           &           & iters  \\ 
 \hdashline 
     &           &    1 &  3.58e-08 &  6.89e-08 &      \\ 
     &           &    2 &  4.20e-08 &  1.02e-15 &      \\ 
     &           &    3 &  3.58e-08 &  1.20e-15 &      \\ 
     &           &    4 &  3.57e-08 &  1.02e-15 &      \\ 
     &           &    5 &  4.20e-08 &  1.02e-15 &      \\ 
     &           &    6 &  3.57e-08 &  1.20e-15 &      \\ 
     &           &    7 &  4.20e-08 &  1.02e-15 &      \\ 
     &           &    8 &  3.58e-08 &  1.20e-15 &      \\ 
     &           &    9 &  4.20e-08 &  1.02e-15 &      \\ 
     &           &   10 &  3.58e-08 &  1.20e-15 &      \\ 
 660 &  6.59e+03 &   10 &           &           & iters  \\ 
 \hdashline 
     &           &    1 &  6.98e-09 &  8.24e-09 &      \\ 
     &           &    2 &  6.97e-09 &  1.99e-16 &      \\ 
     &           &    3 &  3.19e-08 &  1.99e-16 &      \\ 
     &           &    4 &  7.01e-09 &  9.10e-16 &      \\ 
     &           &    5 &  7.00e-09 &  2.00e-16 &      \\ 
     &           &    6 &  6.99e-09 &  2.00e-16 &      \\ 
     &           &    7 &  6.98e-09 &  1.99e-16 &      \\ 
     &           &    8 &  6.97e-09 &  1.99e-16 &      \\ 
     &           &    9 &  3.19e-08 &  1.99e-16 &      \\ 
     &           &   10 &  7.01e-09 &  9.10e-16 &      \\ 
 661 &  6.60e+03 &   10 &           &           & iters  \\ 
 \hdashline 
     &           &    1 &  2.31e-08 &  2.98e-08 &      \\ 
     &           &    2 &  1.58e-08 &  6.59e-16 &      \\ 
     &           &    3 &  2.31e-08 &  4.51e-16 &      \\ 
     &           &    4 &  1.58e-08 &  6.58e-16 &      \\ 
     &           &    5 &  1.58e-08 &  4.51e-16 &      \\ 
     &           &    6 &  2.31e-08 &  4.51e-16 &      \\ 
     &           &    7 &  1.58e-08 &  6.59e-16 &      \\ 
     &           &    8 &  2.31e-08 &  4.51e-16 &      \\ 
     &           &    9 &  1.58e-08 &  6.58e-16 &      \\ 
     &           &   10 &  1.58e-08 &  4.51e-16 &      \\ 
 662 &  6.61e+03 &   10 &           &           & iters  \\ 
 \hdashline 
     &           &    1 &  9.03e-09 &  4.27e-08 &      \\ 
     &           &    2 &  9.02e-09 &  2.58e-16 &      \\ 
     &           &    3 &  9.00e-09 &  2.57e-16 &      \\ 
     &           &    4 &  8.99e-09 &  2.57e-16 &      \\ 
     &           &    5 &  2.99e-08 &  2.57e-16 &      \\ 
     &           &    6 &  9.02e-09 &  8.52e-16 &      \\ 
     &           &    7 &  9.01e-09 &  2.57e-16 &      \\ 
     &           &    8 &  9.00e-09 &  2.57e-16 &      \\ 
     &           &    9 &  8.98e-09 &  2.57e-16 &      \\ 
     &           &   10 &  2.99e-08 &  2.56e-16 &      \\ 
 663 &  6.62e+03 &   10 &           &           & iters  \\ 
 \hdashline 
     &           &    1 &  1.76e-08 &  2.94e-08 &      \\ 
     &           &    2 &  1.76e-08 &  5.02e-16 &      \\ 
     &           &    3 &  1.75e-08 &  5.01e-16 &      \\ 
     &           &    4 &  6.02e-08 &  5.00e-16 &      \\ 
     &           &    5 &  1.76e-08 &  1.72e-15 &      \\ 
     &           &    6 &  1.76e-08 &  5.02e-16 &      \\ 
     &           &    7 &  1.75e-08 &  5.01e-16 &      \\ 
     &           &    8 &  6.02e-08 &  5.01e-16 &      \\ 
     &           &    9 &  1.76e-08 &  1.72e-15 &      \\ 
     &           &   10 &  1.76e-08 &  5.02e-16 &      \\ 
 664 &  6.63e+03 &   10 &           &           & iters  \\ 
 \hdashline 
     &           &    1 &  2.91e-08 &  1.56e-08 &      \\ 
     &           &    2 &  4.86e-08 &  8.31e-16 &      \\ 
     &           &    3 &  2.91e-08 &  1.39e-15 &      \\ 
     &           &    4 &  2.91e-08 &  8.32e-16 &      \\ 
     &           &    5 &  4.86e-08 &  8.31e-16 &      \\ 
     &           &    6 &  2.91e-08 &  1.39e-15 &      \\ 
     &           &    7 &  4.86e-08 &  8.31e-16 &      \\ 
     &           &    8 &  2.92e-08 &  1.39e-15 &      \\ 
     &           &    9 &  2.91e-08 &  8.32e-16 &      \\ 
     &           &   10 &  4.86e-08 &  8.31e-16 &      \\ 
 665 &  6.64e+03 &   10 &           &           & iters  \\ 
 \hdashline 
     &           &    1 &  8.89e-09 &  1.86e-08 &      \\ 
     &           &    2 &  8.88e-09 &  2.54e-16 &      \\ 
     &           &    3 &  8.87e-09 &  2.53e-16 &      \\ 
     &           &    4 &  8.85e-09 &  2.53e-16 &      \\ 
     &           &    5 &  8.84e-09 &  2.53e-16 &      \\ 
     &           &    6 &  8.83e-09 &  2.52e-16 &      \\ 
     &           &    7 &  8.82e-09 &  2.52e-16 &      \\ 
     &           &    8 &  8.80e-09 &  2.52e-16 &      \\ 
     &           &    9 &  8.79e-09 &  2.51e-16 &      \\ 
     &           &   10 &  8.78e-09 &  2.51e-16 &      \\ 
 666 &  6.65e+03 &   10 &           &           & iters  \\ 
 \hdashline 
     &           &    1 &  6.28e-08 &  1.36e-08 &      \\ 
     &           &    2 &  1.49e-08 &  1.79e-15 &      \\ 
     &           &    3 &  1.49e-08 &  4.26e-16 &      \\ 
     &           &    4 &  1.49e-08 &  4.26e-16 &      \\ 
     &           &    5 &  6.28e-08 &  4.25e-16 &      \\ 
     &           &    6 &  1.50e-08 &  1.79e-15 &      \\ 
     &           &    7 &  1.49e-08 &  4.27e-16 &      \\ 
     &           &    8 &  1.49e-08 &  4.26e-16 &      \\ 
     &           &    9 &  1.49e-08 &  4.26e-16 &      \\ 
     &           &   10 &  6.28e-08 &  4.25e-16 &      \\ 
 667 &  6.66e+03 &   10 &           &           & iters  \\ 
 \hdashline 
     &           &    1 &  3.65e-08 &  1.01e-08 &      \\ 
     &           &    2 &  3.65e-08 &  1.04e-15 &      \\ 
     &           &    3 &  4.13e-08 &  1.04e-15 &      \\ 
     &           &    4 &  3.65e-08 &  1.18e-15 &      \\ 
     &           &    5 &  4.13e-08 &  1.04e-15 &      \\ 
     &           &    6 &  3.65e-08 &  1.18e-15 &      \\ 
     &           &    7 &  4.13e-08 &  1.04e-15 &      \\ 
     &           &    8 &  3.65e-08 &  1.18e-15 &      \\ 
     &           &    9 &  4.13e-08 &  1.04e-15 &      \\ 
     &           &   10 &  3.65e-08 &  1.18e-15 &      \\ 
 668 &  6.67e+03 &   10 &           &           & iters  \\ 
 \hdashline 
     &           &    1 &  3.22e-08 &  1.24e-08 &      \\ 
     &           &    2 &  3.22e-08 &  9.20e-16 &      \\ 
     &           &    3 &  4.55e-08 &  9.19e-16 &      \\ 
     &           &    4 &  3.22e-08 &  1.30e-15 &      \\ 
     &           &    5 &  4.55e-08 &  9.19e-16 &      \\ 
     &           &    6 &  3.22e-08 &  1.30e-15 &      \\ 
     &           &    7 &  4.55e-08 &  9.20e-16 &      \\ 
     &           &    8 &  3.23e-08 &  1.30e-15 &      \\ 
     &           &    9 &  3.22e-08 &  9.20e-16 &      \\ 
     &           &   10 &  4.55e-08 &  9.19e-16 &      \\ 
 669 &  6.68e+03 &   10 &           &           & iters  \\ 
 \hdashline 
     &           &    1 &  1.78e-08 &  9.17e-09 &      \\ 
     &           &    2 &  1.78e-08 &  5.09e-16 &      \\ 
     &           &    3 &  5.99e-08 &  5.08e-16 &      \\ 
     &           &    4 &  1.79e-08 &  1.71e-15 &      \\ 
     &           &    5 &  1.78e-08 &  5.10e-16 &      \\ 
     &           &    6 &  1.78e-08 &  5.09e-16 &      \\ 
     &           &    7 &  1.78e-08 &  5.08e-16 &      \\ 
     &           &    8 &  5.99e-08 &  5.07e-16 &      \\ 
     &           &    9 &  1.78e-08 &  1.71e-15 &      \\ 
     &           &   10 &  1.78e-08 &  5.09e-16 &      \\ 
 670 &  6.69e+03 &   10 &           &           & iters  \\ 
 \hdashline 
     &           &    1 &  3.03e-08 &  3.83e-09 &      \\ 
     &           &    2 &  4.75e-08 &  8.63e-16 &      \\ 
     &           &    3 &  3.03e-08 &  1.36e-15 &      \\ 
     &           &    4 &  4.75e-08 &  8.64e-16 &      \\ 
     &           &    5 &  3.03e-08 &  1.35e-15 &      \\ 
     &           &    6 &  3.03e-08 &  8.65e-16 &      \\ 
     &           &    7 &  4.75e-08 &  8.64e-16 &      \\ 
     &           &    8 &  3.03e-08 &  1.35e-15 &      \\ 
     &           &    9 &  4.74e-08 &  8.64e-16 &      \\ 
     &           &   10 &  3.03e-08 &  1.35e-15 &      \\ 
 671 &  6.70e+03 &   10 &           &           & iters  \\ 
 \hdashline 
     &           &    1 &  9.35e-09 &  7.14e-10 &      \\ 
     &           &    2 &  9.34e-09 &  2.67e-16 &      \\ 
     &           &    3 &  9.32e-09 &  2.66e-16 &      \\ 
     &           &    4 &  9.31e-09 &  2.66e-16 &      \\ 
     &           &    5 &  9.30e-09 &  2.66e-16 &      \\ 
     &           &    6 &  6.84e-08 &  2.65e-16 &      \\ 
     &           &    7 &  8.71e-08 &  1.95e-15 &      \\ 
     &           &    8 &  6.84e-08 &  2.49e-15 &      \\ 
     &           &    9 &  9.35e-09 &  1.95e-15 &      \\ 
     &           &   10 &  9.34e-09 &  2.67e-16 &      \\ 
 672 &  6.71e+03 &   10 &           &           & iters  \\ 
 \hdashline 
     &           &    1 &  6.43e-08 &  1.18e-08 &      \\ 
     &           &    2 &  1.35e-08 &  1.83e-15 &      \\ 
     &           &    3 &  1.35e-08 &  3.85e-16 &      \\ 
     &           &    4 &  1.35e-08 &  3.85e-16 &      \\ 
     &           &    5 &  1.34e-08 &  3.84e-16 &      \\ 
     &           &    6 &  1.34e-08 &  3.84e-16 &      \\ 
     &           &    7 &  6.43e-08 &  3.83e-16 &      \\ 
     &           &    8 &  1.35e-08 &  1.83e-15 &      \\ 
     &           &    9 &  1.35e-08 &  3.85e-16 &      \\ 
     &           &   10 &  1.35e-08 &  3.85e-16 &      \\ 
 673 &  6.72e+03 &   10 &           &           & iters  \\ 
 \hdashline 
     &           &    1 &  2.70e-08 &  2.25e-08 &      \\ 
     &           &    2 &  2.70e-08 &  7.70e-16 &      \\ 
     &           &    3 &  5.08e-08 &  7.69e-16 &      \\ 
     &           &    4 &  2.70e-08 &  1.45e-15 &      \\ 
     &           &    5 &  2.69e-08 &  7.70e-16 &      \\ 
     &           &    6 &  5.08e-08 &  7.69e-16 &      \\ 
     &           &    7 &  2.70e-08 &  1.45e-15 &      \\ 
     &           &    8 &  2.69e-08 &  7.70e-16 &      \\ 
     &           &    9 &  5.08e-08 &  7.69e-16 &      \\ 
     &           &   10 &  2.70e-08 &  1.45e-15 &      \\ 
 674 &  6.73e+03 &   10 &           &           & iters  \\ 
 \hdashline 
     &           &    1 &  5.78e-08 &  1.86e-08 &      \\ 
     &           &    2 &  2.00e-08 &  1.65e-15 &      \\ 
     &           &    3 &  2.00e-08 &  5.70e-16 &      \\ 
     &           &    4 &  1.99e-08 &  5.70e-16 &      \\ 
     &           &    5 &  5.78e-08 &  5.69e-16 &      \\ 
     &           &    6 &  2.00e-08 &  1.65e-15 &      \\ 
     &           &    7 &  2.00e-08 &  5.70e-16 &      \\ 
     &           &    8 &  1.99e-08 &  5.69e-16 &      \\ 
     &           &    9 &  5.78e-08 &  5.69e-16 &      \\ 
     &           &   10 &  2.00e-08 &  1.65e-15 &      \\ 
 675 &  6.74e+03 &   10 &           &           & iters  \\ 
 \hdashline 
     &           &    1 &  5.01e-08 &  3.49e-09 &      \\ 
     &           &    2 &  2.77e-08 &  1.43e-15 &      \\ 
     &           &    3 &  2.77e-08 &  7.90e-16 &      \\ 
     &           &    4 &  5.01e-08 &  7.89e-16 &      \\ 
     &           &    5 &  2.77e-08 &  1.43e-15 &      \\ 
     &           &    6 &  2.76e-08 &  7.90e-16 &      \\ 
     &           &    7 &  5.01e-08 &  7.89e-16 &      \\ 
     &           &    8 &  2.77e-08 &  1.43e-15 &      \\ 
     &           &    9 &  5.01e-08 &  7.90e-16 &      \\ 
     &           &   10 &  2.77e-08 &  1.43e-15 &      \\ 
 676 &  6.75e+03 &   10 &           &           & iters  \\ 
 \hdashline 
     &           &    1 &  3.24e-08 &  1.21e-08 &      \\ 
     &           &    2 &  4.53e-08 &  9.25e-16 &      \\ 
     &           &    3 &  3.24e-08 &  1.29e-15 &      \\ 
     &           &    4 &  4.53e-08 &  9.26e-16 &      \\ 
     &           &    5 &  3.25e-08 &  1.29e-15 &      \\ 
     &           &    6 &  3.24e-08 &  9.26e-16 &      \\ 
     &           &    7 &  4.53e-08 &  9.25e-16 &      \\ 
     &           &    8 &  3.24e-08 &  1.29e-15 &      \\ 
     &           &    9 &  4.53e-08 &  9.26e-16 &      \\ 
     &           &   10 &  3.25e-08 &  1.29e-15 &      \\ 
 677 &  6.76e+03 &   10 &           &           & iters  \\ 
 \hdashline 
     &           &    1 &  4.20e-08 &  2.44e-08 &      \\ 
     &           &    2 &  3.57e-08 &  1.20e-15 &      \\ 
     &           &    3 &  3.57e-08 &  1.02e-15 &      \\ 
     &           &    4 &  4.21e-08 &  1.02e-15 &      \\ 
     &           &    5 &  3.57e-08 &  1.20e-15 &      \\ 
     &           &    6 &  4.21e-08 &  1.02e-15 &      \\ 
     &           &    7 &  3.57e-08 &  1.20e-15 &      \\ 
     &           &    8 &  4.21e-08 &  1.02e-15 &      \\ 
     &           &    9 &  3.57e-08 &  1.20e-15 &      \\ 
     &           &   10 &  4.21e-08 &  1.02e-15 &      \\ 
 678 &  6.77e+03 &   10 &           &           & iters  \\ 
 \hdashline 
     &           &    1 &  5.30e-08 &  4.42e-08 &      \\ 
     &           &    2 &  2.47e-08 &  1.51e-15 &      \\ 
     &           &    3 &  5.30e-08 &  7.06e-16 &      \\ 
     &           &    4 &  2.48e-08 &  1.51e-15 &      \\ 
     &           &    5 &  2.47e-08 &  7.07e-16 &      \\ 
     &           &    6 &  5.30e-08 &  7.06e-16 &      \\ 
     &           &    7 &  2.48e-08 &  1.51e-15 &      \\ 
     &           &    8 &  2.48e-08 &  7.07e-16 &      \\ 
     &           &    9 &  5.30e-08 &  7.06e-16 &      \\ 
     &           &   10 &  2.48e-08 &  1.51e-15 &      \\ 
 679 &  6.78e+03 &   10 &           &           & iters  \\ 
 \hdashline 
     &           &    1 &  2.74e-08 &  1.53e-08 &      \\ 
     &           &    2 &  2.73e-08 &  7.81e-16 &      \\ 
     &           &    3 &  2.73e-08 &  7.80e-16 &      \\ 
     &           &    4 &  5.04e-08 &  7.79e-16 &      \\ 
     &           &    5 &  2.73e-08 &  1.44e-15 &      \\ 
     &           &    6 &  2.73e-08 &  7.80e-16 &      \\ 
     &           &    7 &  5.04e-08 &  7.79e-16 &      \\ 
     &           &    8 &  2.73e-08 &  1.44e-15 &      \\ 
     &           &    9 &  2.73e-08 &  7.80e-16 &      \\ 
     &           &   10 &  5.04e-08 &  7.79e-16 &      \\ 
 680 &  6.79e+03 &   10 &           &           & iters  \\ 
 \hdashline 
     &           &    1 &  2.66e-08 &  6.80e-08 &      \\ 
     &           &    2 &  2.65e-08 &  7.58e-16 &      \\ 
     &           &    3 &  5.12e-08 &  7.57e-16 &      \\ 
     &           &    4 &  2.66e-08 &  1.46e-15 &      \\ 
     &           &    5 &  2.65e-08 &  7.58e-16 &      \\ 
     &           &    6 &  5.12e-08 &  7.57e-16 &      \\ 
     &           &    7 &  2.65e-08 &  1.46e-15 &      \\ 
     &           &    8 &  2.65e-08 &  7.58e-16 &      \\ 
     &           &    9 &  5.12e-08 &  7.57e-16 &      \\ 
     &           &   10 &  2.65e-08 &  1.46e-15 &      \\ 
 681 &  6.80e+03 &   10 &           &           & iters  \\ 
 \hdashline 
     &           &    1 &  2.75e-08 &  5.17e-08 &      \\ 
     &           &    2 &  2.75e-08 &  7.85e-16 &      \\ 
     &           &    3 &  5.03e-08 &  7.84e-16 &      \\ 
     &           &    4 &  2.75e-08 &  1.43e-15 &      \\ 
     &           &    5 &  2.75e-08 &  7.85e-16 &      \\ 
     &           &    6 &  5.03e-08 &  7.83e-16 &      \\ 
     &           &    7 &  2.75e-08 &  1.43e-15 &      \\ 
     &           &    8 &  2.74e-08 &  7.84e-16 &      \\ 
     &           &    9 &  5.03e-08 &  7.83e-16 &      \\ 
     &           &   10 &  2.75e-08 &  1.44e-15 &      \\ 
 682 &  6.81e+03 &   10 &           &           & iters  \\ 
 \hdashline 
     &           &    1 &  5.69e-09 &  1.95e-08 &      \\ 
     &           &    2 &  5.69e-09 &  1.63e-16 &      \\ 
     &           &    3 &  5.68e-09 &  1.62e-16 &      \\ 
     &           &    4 &  5.67e-09 &  1.62e-16 &      \\ 
     &           &    5 &  5.66e-09 &  1.62e-16 &      \\ 
     &           &    6 &  5.65e-09 &  1.62e-16 &      \\ 
     &           &    7 &  7.20e-08 &  1.61e-16 &      \\ 
     &           &    8 &  8.34e-08 &  2.06e-15 &      \\ 
     &           &    9 &  7.21e-08 &  2.38e-15 &      \\ 
     &           &   10 &  5.73e-09 &  2.06e-15 &      \\ 
 683 &  6.82e+03 &   10 &           &           & iters  \\ 
 \hdashline 
     &           &    1 &  5.68e-08 &  4.17e-10 &      \\ 
     &           &    2 &  2.10e-08 &  1.62e-15 &      \\ 
     &           &    3 &  5.67e-08 &  5.99e-16 &      \\ 
     &           &    4 &  2.10e-08 &  1.62e-15 &      \\ 
     &           &    5 &  2.10e-08 &  6.00e-16 &      \\ 
     &           &    6 &  2.10e-08 &  5.99e-16 &      \\ 
     &           &    7 &  5.67e-08 &  5.98e-16 &      \\ 
     &           &    8 &  2.10e-08 &  1.62e-15 &      \\ 
     &           &    9 &  2.10e-08 &  6.00e-16 &      \\ 
     &           &   10 &  5.67e-08 &  5.99e-16 &      \\ 
 684 &  6.83e+03 &   10 &           &           & iters  \\ 
 \hdashline 
     &           &    1 &  8.40e-08 &  2.87e-08 &      \\ 
     &           &    2 &  3.27e-08 &  2.40e-15 &      \\ 
     &           &    3 &  4.50e-08 &  9.33e-16 &      \\ 
     &           &    4 &  3.27e-08 &  1.29e-15 &      \\ 
     &           &    5 &  3.27e-08 &  9.33e-16 &      \\ 
     &           &    6 &  4.51e-08 &  9.32e-16 &      \\ 
     &           &    7 &  3.27e-08 &  1.29e-15 &      \\ 
     &           &    8 &  4.51e-08 &  9.33e-16 &      \\ 
     &           &    9 &  3.27e-08 &  1.29e-15 &      \\ 
     &           &   10 &  4.50e-08 &  9.33e-16 &      \\ 
 685 &  6.84e+03 &   10 &           &           & iters  \\ 
 \hdashline 
     &           &    1 &  1.38e-08 &  7.32e-09 &      \\ 
     &           &    2 &  1.38e-08 &  3.94e-16 &      \\ 
     &           &    3 &  1.38e-08 &  3.94e-16 &      \\ 
     &           &    4 &  1.37e-08 &  3.93e-16 &      \\ 
     &           &    5 &  1.37e-08 &  3.92e-16 &      \\ 
     &           &    6 &  6.40e-08 &  3.92e-16 &      \\ 
     &           &    7 &  1.38e-08 &  1.83e-15 &      \\ 
     &           &    8 &  1.38e-08 &  3.94e-16 &      \\ 
     &           &    9 &  1.38e-08 &  3.93e-16 &      \\ 
     &           &   10 &  1.37e-08 &  3.93e-16 &      \\ 
 686 &  6.85e+03 &   10 &           &           & iters  \\ 
 \hdashline 
     &           &    1 &  2.15e-08 &  8.54e-09 &      \\ 
     &           &    2 &  2.15e-08 &  6.14e-16 &      \\ 
     &           &    3 &  5.63e-08 &  6.13e-16 &      \\ 
     &           &    4 &  2.15e-08 &  1.61e-15 &      \\ 
     &           &    5 &  2.15e-08 &  6.14e-16 &      \\ 
     &           &    6 &  2.15e-08 &  6.13e-16 &      \\ 
     &           &    7 &  5.63e-08 &  6.12e-16 &      \\ 
     &           &    8 &  2.15e-08 &  1.61e-15 &      \\ 
     &           &    9 &  2.15e-08 &  6.14e-16 &      \\ 
     &           &   10 &  5.62e-08 &  6.13e-16 &      \\ 
 687 &  6.86e+03 &   10 &           &           & iters  \\ 
 \hdashline 
     &           &    1 &  6.73e-08 &  2.19e-08 &      \\ 
     &           &    2 &  1.04e-08 &  1.92e-15 &      \\ 
     &           &    3 &  1.04e-08 &  2.98e-16 &      \\ 
     &           &    4 &  1.04e-08 &  2.97e-16 &      \\ 
     &           &    5 &  1.04e-08 &  2.97e-16 &      \\ 
     &           &    6 &  1.04e-08 &  2.97e-16 &      \\ 
     &           &    7 &  1.04e-08 &  2.96e-16 &      \\ 
     &           &    8 &  6.73e-08 &  2.96e-16 &      \\ 
     &           &    9 &  4.93e-08 &  1.92e-15 &      \\ 
     &           &   10 &  1.04e-08 &  1.41e-15 &      \\ 
 688 &  6.87e+03 &   10 &           &           & iters  \\ 
 \hdashline 
     &           &    1 &  1.80e-08 &  1.45e-08 &      \\ 
     &           &    2 &  1.80e-08 &  5.15e-16 &      \\ 
     &           &    3 &  2.08e-08 &  5.14e-16 &      \\ 
     &           &    4 &  1.80e-08 &  5.95e-16 &      \\ 
     &           &    5 &  2.08e-08 &  5.15e-16 &      \\ 
     &           &    6 &  1.80e-08 &  5.95e-16 &      \\ 
     &           &    7 &  2.08e-08 &  5.15e-16 &      \\ 
     &           &    8 &  1.80e-08 &  5.95e-16 &      \\ 
     &           &    9 &  2.08e-08 &  5.15e-16 &      \\ 
     &           &   10 &  1.80e-08 &  5.95e-16 &      \\ 
 689 &  6.88e+03 &   10 &           &           & iters  \\ 
 \hdashline 
     &           &    1 &  2.90e-08 &  4.12e-08 &      \\ 
     &           &    2 &  4.87e-08 &  8.27e-16 &      \\ 
     &           &    3 &  2.90e-08 &  1.39e-15 &      \\ 
     &           &    4 &  4.87e-08 &  8.28e-16 &      \\ 
     &           &    5 &  2.90e-08 &  1.39e-15 &      \\ 
     &           &    6 &  2.90e-08 &  8.29e-16 &      \\ 
     &           &    7 &  4.87e-08 &  8.28e-16 &      \\ 
     &           &    8 &  2.90e-08 &  1.39e-15 &      \\ 
     &           &    9 &  2.90e-08 &  8.28e-16 &      \\ 
     &           &   10 &  4.87e-08 &  8.27e-16 &      \\ 
 690 &  6.89e+03 &   10 &           &           & iters  \\ 
 \hdashline 
     &           &    1 &  3.00e-08 &  2.83e-10 &      \\ 
     &           &    2 &  2.99e-08 &  8.56e-16 &      \\ 
     &           &    3 &  4.78e-08 &  8.54e-16 &      \\ 
     &           &    4 &  3.00e-08 &  1.36e-15 &      \\ 
     &           &    5 &  2.99e-08 &  8.55e-16 &      \\ 
     &           &    6 &  4.78e-08 &  8.54e-16 &      \\ 
     &           &    7 &  2.99e-08 &  1.36e-15 &      \\ 
     &           &    8 &  4.78e-08 &  8.55e-16 &      \\ 
     &           &    9 &  3.00e-08 &  1.36e-15 &      \\ 
     &           &   10 &  2.99e-08 &  8.55e-16 &      \\ 
 691 &  6.90e+03 &   10 &           &           & iters  \\ 
 \hdashline 
     &           &    1 &  5.46e-08 &  2.76e-08 &      \\ 
     &           &    2 &  2.32e-08 &  1.56e-15 &      \\ 
     &           &    3 &  2.32e-08 &  6.62e-16 &      \\ 
     &           &    4 &  5.45e-08 &  6.62e-16 &      \\ 
     &           &    5 &  2.32e-08 &  1.56e-15 &      \\ 
     &           &    6 &  2.32e-08 &  6.63e-16 &      \\ 
     &           &    7 &  5.45e-08 &  6.62e-16 &      \\ 
     &           &    8 &  2.32e-08 &  1.56e-15 &      \\ 
     &           &    9 &  2.32e-08 &  6.63e-16 &      \\ 
     &           &   10 &  5.45e-08 &  6.62e-16 &      \\ 
 692 &  6.91e+03 &   10 &           &           & iters  \\ 
 \hdashline 
     &           &    1 &  1.77e-08 &  1.62e-08 &      \\ 
     &           &    2 &  1.76e-08 &  5.04e-16 &      \\ 
     &           &    3 &  1.76e-08 &  5.04e-16 &      \\ 
     &           &    4 &  2.12e-08 &  5.03e-16 &      \\ 
     &           &    5 &  1.76e-08 &  6.06e-16 &      \\ 
     &           &    6 &  2.12e-08 &  5.03e-16 &      \\ 
     &           &    7 &  1.76e-08 &  6.06e-16 &      \\ 
     &           &    8 &  2.12e-08 &  5.03e-16 &      \\ 
     &           &    9 &  1.76e-08 &  6.06e-16 &      \\ 
     &           &   10 &  2.12e-08 &  5.03e-16 &      \\ 
 693 &  6.92e+03 &   10 &           &           & iters  \\ 
 \hdashline 
     &           &    1 &  2.58e-08 &  4.67e-08 &      \\ 
     &           &    2 &  2.58e-08 &  7.37e-16 &      \\ 
     &           &    3 &  5.19e-08 &  7.36e-16 &      \\ 
     &           &    4 &  2.58e-08 &  1.48e-15 &      \\ 
     &           &    5 &  2.58e-08 &  7.37e-16 &      \\ 
     &           &    6 &  5.19e-08 &  7.36e-16 &      \\ 
     &           &    7 &  2.58e-08 &  1.48e-15 &      \\ 
     &           &    8 &  2.58e-08 &  7.37e-16 &      \\ 
     &           &    9 &  5.19e-08 &  7.36e-16 &      \\ 
     &           &   10 &  2.58e-08 &  1.48e-15 &      \\ 
 694 &  6.93e+03 &   10 &           &           & iters  \\ 
 \hdashline 
     &           &    1 &  1.11e-09 &  2.77e-08 &      \\ 
     &           &    2 &  1.11e-09 &  3.16e-17 &      \\ 
     &           &    3 &  1.10e-09 &  3.16e-17 &      \\ 
     &           &    4 &  1.10e-09 &  3.15e-17 &      \\ 
     &           &    5 &  1.10e-09 &  3.15e-17 &      \\ 
     &           &    6 &  1.10e-09 &  3.14e-17 &      \\ 
     &           &    7 &  1.10e-09 &  3.14e-17 &      \\ 
     &           &    8 &  1.10e-09 &  3.14e-17 &      \\ 
     &           &    9 &  1.10e-09 &  3.13e-17 &      \\ 
     &           &   10 &  1.09e-09 &  3.13e-17 &      \\ 
 695 &  6.94e+03 &   10 &           &           & iters  \\ 
 \hdashline 
     &           &    1 &  1.46e-08 &  3.39e-08 &      \\ 
     &           &    2 &  1.45e-08 &  4.16e-16 &      \\ 
     &           &    3 &  1.45e-08 &  4.15e-16 &      \\ 
     &           &    4 &  6.32e-08 &  4.14e-16 &      \\ 
     &           &    5 &  1.46e-08 &  1.80e-15 &      \\ 
     &           &    6 &  1.46e-08 &  4.16e-16 &      \\ 
     &           &    7 &  1.45e-08 &  4.16e-16 &      \\ 
     &           &    8 &  1.45e-08 &  4.15e-16 &      \\ 
     &           &    9 &  6.32e-08 &  4.15e-16 &      \\ 
     &           &   10 &  1.46e-08 &  1.80e-15 &      \\ 
 696 &  6.95e+03 &   10 &           &           & iters  \\ 
 \hdashline 
     &           &    1 &  7.95e-08 &  1.97e-08 &      \\ 
     &           &    2 &  1.74e-09 &  2.27e-15 &      \\ 
     &           &    3 &  7.60e-08 &  4.96e-17 &      \\ 
     &           &    4 &  7.95e-08 &  2.17e-15 &      \\ 
     &           &    5 &  7.60e-08 &  2.27e-15 &      \\ 
     &           &    6 &  7.95e-08 &  2.17e-15 &      \\ 
     &           &    7 &  7.60e-08 &  2.27e-15 &      \\ 
     &           &    8 &  7.95e-08 &  2.17e-15 &      \\ 
     &           &    9 &  7.60e-08 &  2.27e-15 &      \\ 
     &           &   10 &  7.95e-08 &  2.17e-15 &      \\ 
 697 &  6.96e+03 &   10 &           &           & iters  \\ 
 \hdashline 
     &           &    1 &  1.35e-07 &  9.31e-08 &      \\ 
     &           &    2 &  2.10e-08 &  3.84e-15 &      \\ 
     &           &    3 &  2.10e-08 &  5.99e-16 &      \\ 
     &           &    4 &  2.09e-08 &  5.99e-16 &      \\ 
     &           &    5 &  5.68e-08 &  5.98e-16 &      \\ 
     &           &    6 &  2.10e-08 &  1.62e-15 &      \\ 
     &           &    7 &  2.10e-08 &  5.99e-16 &      \\ 
     &           &    8 &  2.09e-08 &  5.98e-16 &      \\ 
     &           &    9 &  5.68e-08 &  5.97e-16 &      \\ 
     &           &   10 &  2.10e-08 &  1.62e-15 &      \\ 
 698 &  6.97e+03 &   10 &           &           & iters  \\ 
 \hdashline 
     &           &    1 &  6.45e-08 &  1.16e-07 &      \\ 
     &           &    2 &  1.33e-08 &  1.84e-15 &      \\ 
     &           &    3 &  1.32e-08 &  3.78e-16 &      \\ 
     &           &    4 &  1.32e-08 &  3.78e-16 &      \\ 
     &           &    5 &  1.32e-08 &  3.77e-16 &      \\ 
     &           &    6 &  6.45e-08 &  3.77e-16 &      \\ 
     &           &    7 &  1.33e-08 &  1.84e-15 &      \\ 
     &           &    8 &  1.33e-08 &  3.79e-16 &      \\ 
     &           &    9 &  1.32e-08 &  3.78e-16 &      \\ 
     &           &   10 &  1.32e-08 &  3.78e-16 &      \\ 
 699 &  6.98e+03 &   10 &           &           & iters  \\ 
 \hdashline 
     &           &    1 &  1.51e-08 &  2.99e-08 &      \\ 
     &           &    2 &  1.51e-08 &  4.31e-16 &      \\ 
     &           &    3 &  6.26e-08 &  4.31e-16 &      \\ 
     &           &    4 &  1.52e-08 &  1.79e-15 &      \\ 
     &           &    5 &  1.51e-08 &  4.33e-16 &      \\ 
     &           &    6 &  1.51e-08 &  4.32e-16 &      \\ 
     &           &    7 &  1.51e-08 &  4.31e-16 &      \\ 
     &           &    8 &  6.26e-08 &  4.31e-16 &      \\ 
     &           &    9 &  1.52e-08 &  1.79e-15 &      \\ 
     &           &   10 &  1.51e-08 &  4.33e-16 &      \\ 
 700 &  6.99e+03 &   10 &           &           & iters  \\ 
 \hdashline 
     &           &    1 &  1.80e-08 &  5.84e-08 &      \\ 
     &           &    2 &  2.09e-08 &  5.12e-16 &      \\ 
     &           &    3 &  1.80e-08 &  5.97e-16 &      \\ 
     &           &    4 &  2.09e-08 &  5.12e-16 &      \\ 
     &           &    5 &  1.80e-08 &  5.97e-16 &      \\ 
     &           &    6 &  2.09e-08 &  5.13e-16 &      \\ 
     &           &    7 &  1.80e-08 &  5.97e-16 &      \\ 
     &           &    8 &  2.09e-08 &  5.13e-16 &      \\ 
     &           &    9 &  1.80e-08 &  5.97e-16 &      \\ 
     &           &   10 &  1.79e-08 &  5.13e-16 &      \\ 
 701 &  7.00e+03 &   10 &           &           & iters  \\ 
 \hdashline 
     &           &    1 &  2.93e-08 &  1.97e-08 &      \\ 
     &           &    2 &  2.92e-08 &  8.35e-16 &      \\ 
     &           &    3 &  4.85e-08 &  8.34e-16 &      \\ 
     &           &    4 &  2.92e-08 &  1.38e-15 &      \\ 
     &           &    5 &  4.85e-08 &  8.35e-16 &      \\ 
     &           &    6 &  2.93e-08 &  1.38e-15 &      \\ 
     &           &    7 &  2.92e-08 &  8.35e-16 &      \\ 
     &           &    8 &  4.85e-08 &  8.34e-16 &      \\ 
     &           &    9 &  2.93e-08 &  1.38e-15 &      \\ 
     &           &   10 &  2.92e-08 &  8.35e-16 &      \\ 
 702 &  7.01e+03 &   10 &           &           & iters  \\ 
 \hdashline 
     &           &    1 &  6.00e-08 &  9.83e-09 &      \\ 
     &           &    2 &  2.10e-08 &  1.71e-15 &      \\ 
     &           &    3 &  5.67e-08 &  6.00e-16 &      \\ 
     &           &    4 &  2.11e-08 &  1.62e-15 &      \\ 
     &           &    5 &  2.11e-08 &  6.02e-16 &      \\ 
     &           &    6 &  2.10e-08 &  6.01e-16 &      \\ 
     &           &    7 &  5.67e-08 &  6.00e-16 &      \\ 
     &           &    8 &  2.11e-08 &  1.62e-15 &      \\ 
     &           &    9 &  2.10e-08 &  6.02e-16 &      \\ 
     &           &   10 &  5.67e-08 &  6.01e-16 &      \\ 
 703 &  7.02e+03 &   10 &           &           & iters  \\ 
 \hdashline 
     &           &    1 &  4.02e-08 &  5.95e-09 &      \\ 
     &           &    2 &  3.76e-08 &  1.15e-15 &      \\ 
     &           &    3 &  4.02e-08 &  1.07e-15 &      \\ 
     &           &    4 &  3.76e-08 &  1.15e-15 &      \\ 
     &           &    5 &  4.02e-08 &  1.07e-15 &      \\ 
     &           &    6 &  3.76e-08 &  1.15e-15 &      \\ 
     &           &    7 &  4.02e-08 &  1.07e-15 &      \\ 
     &           &    8 &  3.76e-08 &  1.15e-15 &      \\ 
     &           &    9 &  4.02e-08 &  1.07e-15 &      \\ 
     &           &   10 &  3.76e-08 &  1.15e-15 &      \\ 
 704 &  7.03e+03 &   10 &           &           & iters  \\ 
 \hdashline 
     &           &    1 &  5.06e-08 &  9.63e-09 &      \\ 
     &           &    2 &  2.71e-08 &  1.44e-15 &      \\ 
     &           &    3 &  2.71e-08 &  7.75e-16 &      \\ 
     &           &    4 &  5.06e-08 &  7.74e-16 &      \\ 
     &           &    5 &  2.71e-08 &  1.44e-15 &      \\ 
     &           &    6 &  2.71e-08 &  7.75e-16 &      \\ 
     &           &    7 &  5.06e-08 &  7.74e-16 &      \\ 
     &           &    8 &  2.71e-08 &  1.44e-15 &      \\ 
     &           &    9 &  2.71e-08 &  7.75e-16 &      \\ 
     &           &   10 &  5.06e-08 &  7.73e-16 &      \\ 
 705 &  7.04e+03 &   10 &           &           & iters  \\ 
 \hdashline 
     &           &    1 &  6.57e-09 &  8.40e-09 &      \\ 
     &           &    2 &  7.11e-08 &  1.88e-16 &      \\ 
     &           &    3 &  4.55e-08 &  2.03e-15 &      \\ 
     &           &    4 &  6.60e-09 &  1.30e-15 &      \\ 
     &           &    5 &  6.59e-09 &  1.88e-16 &      \\ 
     &           &    6 &  6.58e-09 &  1.88e-16 &      \\ 
     &           &    7 &  6.57e-09 &  1.88e-16 &      \\ 
     &           &    8 &  7.11e-08 &  1.88e-16 &      \\ 
     &           &    9 &  4.55e-08 &  2.03e-15 &      \\ 
     &           &   10 &  6.60e-09 &  1.30e-15 &      \\ 
 706 &  7.05e+03 &   10 &           &           & iters  \\ 
 \hdashline 
     &           &    1 &  1.35e-08 &  7.62e-09 &      \\ 
     &           &    2 &  1.35e-08 &  3.86e-16 &      \\ 
     &           &    3 &  2.54e-08 &  3.85e-16 &      \\ 
     &           &    4 &  1.35e-08 &  7.24e-16 &      \\ 
     &           &    5 &  1.35e-08 &  3.86e-16 &      \\ 
     &           &    6 &  2.54e-08 &  3.85e-16 &      \\ 
     &           &    7 &  1.35e-08 &  7.24e-16 &      \\ 
     &           &    8 &  1.35e-08 &  3.86e-16 &      \\ 
     &           &    9 &  2.54e-08 &  3.85e-16 &      \\ 
     &           &   10 &  1.35e-08 &  7.24e-16 &      \\ 
 707 &  7.06e+03 &   10 &           &           & iters  \\ 
 \hdashline 
     &           &    1 &  2.87e-08 &  1.88e-08 &      \\ 
     &           &    2 &  4.90e-08 &  8.19e-16 &      \\ 
     &           &    3 &  2.87e-08 &  1.40e-15 &      \\ 
     &           &    4 &  2.87e-08 &  8.20e-16 &      \\ 
     &           &    5 &  4.91e-08 &  8.18e-16 &      \\ 
     &           &    6 &  2.87e-08 &  1.40e-15 &      \\ 
     &           &    7 &  4.90e-08 &  8.19e-16 &      \\ 
     &           &    8 &  2.87e-08 &  1.40e-15 &      \\ 
     &           &    9 &  2.87e-08 &  8.20e-16 &      \\ 
     &           &   10 &  4.90e-08 &  8.19e-16 &      \\ 
 708 &  7.07e+03 &   10 &           &           & iters  \\ 
 \hdashline 
     &           &    1 &  2.08e-08 &  1.94e-08 &      \\ 
     &           &    2 &  2.08e-08 &  5.95e-16 &      \\ 
     &           &    3 &  2.08e-08 &  5.94e-16 &      \\ 
     &           &    4 &  5.69e-08 &  5.93e-16 &      \\ 
     &           &    5 &  2.08e-08 &  1.62e-15 &      \\ 
     &           &    6 &  2.08e-08 &  5.95e-16 &      \\ 
     &           &    7 &  2.08e-08 &  5.94e-16 &      \\ 
     &           &    8 &  5.69e-08 &  5.93e-16 &      \\ 
     &           &    9 &  2.08e-08 &  1.63e-15 &      \\ 
     &           &   10 &  2.08e-08 &  5.95e-16 &      \\ 
 709 &  7.08e+03 &   10 &           &           & iters  \\ 
 \hdashline 
     &           &    1 &  1.95e-08 &  2.88e-09 &      \\ 
     &           &    2 &  1.95e-08 &  5.58e-16 &      \\ 
     &           &    3 &  1.95e-08 &  5.57e-16 &      \\ 
     &           &    4 &  5.82e-08 &  5.56e-16 &      \\ 
     &           &    5 &  1.95e-08 &  1.66e-15 &      \\ 
     &           &    6 &  1.95e-08 &  5.58e-16 &      \\ 
     &           &    7 &  1.95e-08 &  5.57e-16 &      \\ 
     &           &    8 &  5.82e-08 &  5.56e-16 &      \\ 
     &           &    9 &  1.95e-08 &  1.66e-15 &      \\ 
     &           &   10 &  1.95e-08 &  5.58e-16 &      \\ 
 710 &  7.09e+03 &   10 &           &           & iters  \\ 
 \hdashline 
     &           &    1 &  2.88e-08 &  7.33e-09 &      \\ 
     &           &    2 &  4.89e-08 &  8.22e-16 &      \\ 
     &           &    3 &  2.88e-08 &  1.40e-15 &      \\ 
     &           &    4 &  2.88e-08 &  8.23e-16 &      \\ 
     &           &    5 &  4.89e-08 &  8.22e-16 &      \\ 
     &           &    6 &  2.88e-08 &  1.40e-15 &      \\ 
     &           &    7 &  2.88e-08 &  8.23e-16 &      \\ 
     &           &    8 &  4.89e-08 &  8.22e-16 &      \\ 
     &           &    9 &  2.88e-08 &  1.40e-15 &      \\ 
     &           &   10 &  4.89e-08 &  8.22e-16 &      \\ 
 711 &  7.10e+03 &   10 &           &           & iters  \\ 
 \hdashline 
     &           &    1 &  5.34e-08 &  2.06e-08 &      \\ 
     &           &    2 &  2.44e-08 &  1.52e-15 &      \\ 
     &           &    3 &  2.44e-08 &  6.96e-16 &      \\ 
     &           &    4 &  5.34e-08 &  6.95e-16 &      \\ 
     &           &    5 &  2.44e-08 &  1.52e-15 &      \\ 
     &           &    6 &  2.44e-08 &  6.97e-16 &      \\ 
     &           &    7 &  5.34e-08 &  6.96e-16 &      \\ 
     &           &    8 &  2.44e-08 &  1.52e-15 &      \\ 
     &           &    9 &  2.44e-08 &  6.97e-16 &      \\ 
     &           &   10 &  2.43e-08 &  6.96e-16 &      \\ 
 712 &  7.11e+03 &   10 &           &           & iters  \\ 
 \hdashline 
     &           &    1 &  2.76e-08 &  3.03e-08 &      \\ 
     &           &    2 &  2.76e-08 &  7.88e-16 &      \\ 
     &           &    3 &  5.02e-08 &  7.86e-16 &      \\ 
     &           &    4 &  2.76e-08 &  1.43e-15 &      \\ 
     &           &    5 &  2.75e-08 &  7.87e-16 &      \\ 
     &           &    6 &  5.02e-08 &  7.86e-16 &      \\ 
     &           &    7 &  2.76e-08 &  1.43e-15 &      \\ 
     &           &    8 &  2.75e-08 &  7.87e-16 &      \\ 
     &           &    9 &  5.02e-08 &  7.86e-16 &      \\ 
     &           &   10 &  2.76e-08 &  1.43e-15 &      \\ 
 713 &  7.12e+03 &   10 &           &           & iters  \\ 
 \hdashline 
     &           &    1 &  1.66e-08 &  1.09e-08 &      \\ 
     &           &    2 &  2.23e-08 &  4.74e-16 &      \\ 
     &           &    3 &  1.66e-08 &  6.35e-16 &      \\ 
     &           &    4 &  2.23e-08 &  4.74e-16 &      \\ 
     &           &    5 &  1.66e-08 &  6.35e-16 &      \\ 
     &           &    6 &  2.22e-08 &  4.74e-16 &      \\ 
     &           &    7 &  1.66e-08 &  6.35e-16 &      \\ 
     &           &    8 &  1.66e-08 &  4.75e-16 &      \\ 
     &           &    9 &  2.23e-08 &  4.74e-16 &      \\ 
     &           &   10 &  1.66e-08 &  6.35e-16 &      \\ 
 714 &  7.13e+03 &   10 &           &           & iters  \\ 
 \hdashline 
     &           &    1 &  9.24e-09 &  3.66e-09 &      \\ 
     &           &    2 &  9.23e-09 &  2.64e-16 &      \\ 
     &           &    3 &  9.21e-09 &  2.63e-16 &      \\ 
     &           &    4 &  9.20e-09 &  2.63e-16 &      \\ 
     &           &    5 &  6.85e-08 &  2.63e-16 &      \\ 
     &           &    6 &  8.70e-08 &  1.96e-15 &      \\ 
     &           &    7 &  6.85e-08 &  2.48e-15 &      \\ 
     &           &    8 &  9.26e-09 &  1.96e-15 &      \\ 
     &           &    9 &  9.25e-09 &  2.64e-16 &      \\ 
     &           &   10 &  9.23e-09 &  2.64e-16 &      \\ 
 715 &  7.14e+03 &   10 &           &           & iters  \\ 
 \hdashline 
     &           &    1 &  4.27e-08 &  2.39e-08 &      \\ 
     &           &    2 &  3.51e-08 &  1.22e-15 &      \\ 
     &           &    3 &  4.27e-08 &  1.00e-15 &      \\ 
     &           &    4 &  3.51e-08 &  1.22e-15 &      \\ 
     &           &    5 &  4.27e-08 &  1.00e-15 &      \\ 
     &           &    6 &  3.51e-08 &  1.22e-15 &      \\ 
     &           &    7 &  3.50e-08 &  1.00e-15 &      \\ 
     &           &    8 &  4.27e-08 &  1.00e-15 &      \\ 
     &           &    9 &  3.50e-08 &  1.22e-15 &      \\ 
     &           &   10 &  4.27e-08 &  1.00e-15 &      \\ 
 716 &  7.15e+03 &   10 &           &           & iters  \\ 
 \hdashline 
     &           &    1 &  3.64e-09 &  3.12e-08 &      \\ 
     &           &    2 &  3.64e-09 &  1.04e-16 &      \\ 
     &           &    3 &  3.63e-09 &  1.04e-16 &      \\ 
     &           &    4 &  3.63e-09 &  1.04e-16 &      \\ 
     &           &    5 &  3.62e-09 &  1.03e-16 &      \\ 
     &           &    6 &  3.61e-09 &  1.03e-16 &      \\ 
     &           &    7 &  3.61e-09 &  1.03e-16 &      \\ 
     &           &    8 &  3.60e-09 &  1.03e-16 &      \\ 
     &           &    9 &  3.60e-09 &  1.03e-16 &      \\ 
     &           &   10 &  3.59e-09 &  1.03e-16 &      \\ 
 717 &  7.16e+03 &   10 &           &           & iters  \\ 
 \hdashline 
     &           &    1 &  6.31e-08 &  9.52e-09 &      \\ 
     &           &    2 &  1.46e-08 &  1.80e-15 &      \\ 
     &           &    3 &  1.46e-08 &  4.18e-16 &      \\ 
     &           &    4 &  1.46e-08 &  4.17e-16 &      \\ 
     &           &    5 &  1.46e-08 &  4.16e-16 &      \\ 
     &           &    6 &  1.45e-08 &  4.16e-16 &      \\ 
     &           &    7 &  6.32e-08 &  4.15e-16 &      \\ 
     &           &    8 &  1.46e-08 &  1.80e-15 &      \\ 
     &           &    9 &  1.46e-08 &  4.17e-16 &      \\ 
     &           &   10 &  1.46e-08 &  4.17e-16 &      \\ 
 718 &  7.17e+03 &   10 &           &           & iters  \\ 
 \hdashline 
     &           &    1 &  5.14e-08 &  9.35e-09 &      \\ 
     &           &    2 &  2.63e-08 &  1.47e-15 &      \\ 
     &           &    3 &  2.63e-08 &  7.51e-16 &      \\ 
     &           &    4 &  5.14e-08 &  7.50e-16 &      \\ 
     &           &    5 &  2.63e-08 &  1.47e-15 &      \\ 
     &           &    6 &  2.63e-08 &  7.51e-16 &      \\ 
     &           &    7 &  5.14e-08 &  7.50e-16 &      \\ 
     &           &    8 &  2.63e-08 &  1.47e-15 &      \\ 
     &           &    9 &  2.63e-08 &  7.51e-16 &      \\ 
     &           &   10 &  5.14e-08 &  7.50e-16 &      \\ 
 719 &  7.18e+03 &   10 &           &           & iters  \\ 
 \hdashline 
     &           &    1 &  1.37e-08 &  2.61e-09 &      \\ 
     &           &    2 &  6.41e-08 &  3.90e-16 &      \\ 
     &           &    3 &  1.37e-08 &  1.83e-15 &      \\ 
     &           &    4 &  1.37e-08 &  3.92e-16 &      \\ 
     &           &    5 &  1.37e-08 &  3.91e-16 &      \\ 
     &           &    6 &  1.37e-08 &  3.91e-16 &      \\ 
     &           &    7 &  1.36e-08 &  3.90e-16 &      \\ 
     &           &    8 &  6.41e-08 &  3.90e-16 &      \\ 
     &           &    9 &  1.37e-08 &  1.83e-15 &      \\ 
     &           &   10 &  1.37e-08 &  3.92e-16 &      \\ 
 720 &  7.19e+03 &   10 &           &           & iters  \\ 
 \hdashline 
     &           &    1 &  6.75e-08 &  2.29e-08 &      \\ 
     &           &    2 &  1.02e-08 &  1.93e-15 &      \\ 
     &           &    3 &  1.02e-08 &  2.92e-16 &      \\ 
     &           &    4 &  1.02e-08 &  2.92e-16 &      \\ 
     &           &    5 &  1.02e-08 &  2.91e-16 &      \\ 
     &           &    6 &  1.02e-08 &  2.91e-16 &      \\ 
     &           &    7 &  1.02e-08 &  2.90e-16 &      \\ 
     &           &    8 &  6.75e-08 &  2.90e-16 &      \\ 
     &           &    9 &  8.79e-08 &  1.93e-15 &      \\ 
     &           &   10 &  6.76e-08 &  2.51e-15 &      \\ 
 721 &  7.20e+03 &   10 &           &           & iters  \\ 
 \hdashline 
     &           &    1 &  3.32e-08 &  1.69e-08 &      \\ 
     &           &    2 &  4.45e-08 &  9.49e-16 &      \\ 
     &           &    3 &  3.33e-08 &  1.27e-15 &      \\ 
     &           &    4 &  4.45e-08 &  9.49e-16 &      \\ 
     &           &    5 &  3.33e-08 &  1.27e-15 &      \\ 
     &           &    6 &  3.32e-08 &  9.50e-16 &      \\ 
     &           &    7 &  4.45e-08 &  9.48e-16 &      \\ 
     &           &    8 &  3.32e-08 &  1.27e-15 &      \\ 
     &           &    9 &  4.45e-08 &  9.49e-16 &      \\ 
     &           &   10 &  3.33e-08 &  1.27e-15 &      \\ 
 722 &  7.21e+03 &   10 &           &           & iters  \\ 
 \hdashline 
     &           &    1 &  8.03e-09 &  1.67e-10 &      \\ 
     &           &    2 &  8.02e-09 &  2.29e-16 &      \\ 
     &           &    3 &  8.01e-09 &  2.29e-16 &      \\ 
     &           &    4 &  6.97e-08 &  2.29e-16 &      \\ 
     &           &    5 &  8.58e-08 &  1.99e-15 &      \\ 
     &           &    6 &  6.97e-08 &  2.45e-15 &      \\ 
     &           &    7 &  8.07e-09 &  1.99e-15 &      \\ 
     &           &    8 &  8.06e-09 &  2.30e-16 &      \\ 
     &           &    9 &  8.05e-09 &  2.30e-16 &      \\ 
     &           &   10 &  8.04e-09 &  2.30e-16 &      \\ 
 723 &  7.22e+03 &   10 &           &           & iters  \\ 
 \hdashline 
     &           &    1 &  2.02e-09 &  3.58e-09 &      \\ 
     &           &    2 &  2.02e-09 &  5.76e-17 &      \\ 
     &           &    3 &  2.01e-09 &  5.75e-17 &      \\ 
     &           &    4 &  2.01e-09 &  5.75e-17 &      \\ 
     &           &    5 &  2.01e-09 &  5.74e-17 &      \\ 
     &           &    6 &  2.00e-09 &  5.73e-17 &      \\ 
     &           &    7 &  2.00e-09 &  5.72e-17 &      \\ 
     &           &    8 &  2.00e-09 &  5.71e-17 &      \\ 
     &           &    9 &  2.00e-09 &  5.70e-17 &      \\ 
     &           &   10 &  1.99e-09 &  5.70e-17 &      \\ 
 724 &  7.23e+03 &   10 &           &           & iters  \\ 
 \hdashline 
     &           &    1 &  3.62e-08 &  1.23e-08 &      \\ 
     &           &    2 &  4.16e-08 &  1.03e-15 &      \\ 
     &           &    3 &  3.62e-08 &  1.19e-15 &      \\ 
     &           &    4 &  4.16e-08 &  1.03e-15 &      \\ 
     &           &    5 &  3.62e-08 &  1.19e-15 &      \\ 
     &           &    6 &  4.15e-08 &  1.03e-15 &      \\ 
     &           &    7 &  3.62e-08 &  1.19e-15 &      \\ 
     &           &    8 &  4.15e-08 &  1.03e-15 &      \\ 
     &           &    9 &  3.62e-08 &  1.19e-15 &      \\ 
     &           &   10 &  4.15e-08 &  1.03e-15 &      \\ 
 725 &  7.24e+03 &   10 &           &           & iters  \\ 
 \hdashline 
     &           &    1 &  1.47e-08 &  8.13e-09 &      \\ 
     &           &    2 &  6.30e-08 &  4.20e-16 &      \\ 
     &           &    3 &  1.48e-08 &  1.80e-15 &      \\ 
     &           &    4 &  1.48e-08 &  4.22e-16 &      \\ 
     &           &    5 &  1.48e-08 &  4.22e-16 &      \\ 
     &           &    6 &  1.47e-08 &  4.21e-16 &      \\ 
     &           &    7 &  6.30e-08 &  4.20e-16 &      \\ 
     &           &    8 &  1.48e-08 &  1.80e-15 &      \\ 
     &           &    9 &  1.48e-08 &  4.22e-16 &      \\ 
     &           &   10 &  1.48e-08 &  4.22e-16 &      \\ 
 726 &  7.25e+03 &   10 &           &           & iters  \\ 
 \hdashline 
     &           &    1 &  3.45e-08 &  7.82e-09 &      \\ 
     &           &    2 &  4.32e-08 &  9.86e-16 &      \\ 
     &           &    3 &  3.46e-08 &  1.23e-15 &      \\ 
     &           &    4 &  3.45e-08 &  9.86e-16 &      \\ 
     &           &    5 &  4.32e-08 &  9.85e-16 &      \\ 
     &           &    6 &  3.45e-08 &  1.23e-15 &      \\ 
     &           &    7 &  4.32e-08 &  9.85e-16 &      \\ 
     &           &    8 &  3.45e-08 &  1.23e-15 &      \\ 
     &           &    9 &  4.32e-08 &  9.86e-16 &      \\ 
     &           &   10 &  3.45e-08 &  1.23e-15 &      \\ 
 727 &  7.26e+03 &   10 &           &           & iters  \\ 
 \hdashline 
     &           &    1 &  5.50e-08 &  4.01e-08 &      \\ 
     &           &    2 &  2.28e-08 &  1.57e-15 &      \\ 
     &           &    3 &  2.28e-08 &  6.50e-16 &      \\ 
     &           &    4 &  2.27e-08 &  6.49e-16 &      \\ 
     &           &    5 &  5.50e-08 &  6.48e-16 &      \\ 
     &           &    6 &  2.28e-08 &  1.57e-15 &      \\ 
     &           &    7 &  2.27e-08 &  6.50e-16 &      \\ 
     &           &    8 &  5.50e-08 &  6.49e-16 &      \\ 
     &           &    9 &  2.28e-08 &  1.57e-15 &      \\ 
     &           &   10 &  2.27e-08 &  6.50e-16 &      \\ 
 728 &  7.27e+03 &   10 &           &           & iters  \\ 
 \hdashline 
     &           &    1 &  1.88e-08 &  4.78e-08 &      \\ 
     &           &    2 &  5.90e-08 &  5.35e-16 &      \\ 
     &           &    3 &  1.88e-08 &  1.68e-15 &      \\ 
     &           &    4 &  1.88e-08 &  5.37e-16 &      \\ 
     &           &    5 &  1.88e-08 &  5.36e-16 &      \\ 
     &           &    6 &  5.90e-08 &  5.36e-16 &      \\ 
     &           &    7 &  1.88e-08 &  1.68e-15 &      \\ 
     &           &    8 &  1.88e-08 &  5.37e-16 &      \\ 
     &           &    9 &  1.88e-08 &  5.36e-16 &      \\ 
     &           &   10 &  1.87e-08 &  5.36e-16 &      \\ 
 729 &  7.28e+03 &   10 &           &           & iters  \\ 
 \hdashline 
     &           &    1 &  6.38e-08 &  1.26e-08 &      \\ 
     &           &    2 &  2.49e-08 &  1.82e-15 &      \\ 
     &           &    3 &  5.29e-08 &  7.09e-16 &      \\ 
     &           &    4 &  2.49e-08 &  1.51e-15 &      \\ 
     &           &    5 &  2.49e-08 &  7.10e-16 &      \\ 
     &           &    6 &  5.29e-08 &  7.09e-16 &      \\ 
     &           &    7 &  2.49e-08 &  1.51e-15 &      \\ 
     &           &    8 &  2.49e-08 &  7.11e-16 &      \\ 
     &           &    9 &  5.29e-08 &  7.10e-16 &      \\ 
     &           &   10 &  2.49e-08 &  1.51e-15 &      \\ 
 730 &  7.29e+03 &   10 &           &           & iters  \\ 
 \hdashline 
     &           &    1 &  2.06e-08 &  1.51e-08 &      \\ 
     &           &    2 &  2.06e-08 &  5.88e-16 &      \\ 
     &           &    3 &  5.72e-08 &  5.87e-16 &      \\ 
     &           &    4 &  2.06e-08 &  1.63e-15 &      \\ 
     &           &    5 &  2.06e-08 &  5.88e-16 &      \\ 
     &           &    6 &  2.06e-08 &  5.88e-16 &      \\ 
     &           &    7 &  5.72e-08 &  5.87e-16 &      \\ 
     &           &    8 &  2.06e-08 &  1.63e-15 &      \\ 
     &           &    9 &  2.06e-08 &  5.88e-16 &      \\ 
     &           &   10 &  2.06e-08 &  5.87e-16 &      \\ 
 731 &  7.30e+03 &   10 &           &           & iters  \\ 
 \hdashline 
     &           &    1 &  4.30e-08 &  7.40e-08 &      \\ 
     &           &    2 &  3.48e-08 &  1.23e-15 &      \\ 
     &           &    3 &  4.29e-08 &  9.93e-16 &      \\ 
     &           &    4 &  3.48e-08 &  1.23e-15 &      \\ 
     &           &    5 &  3.48e-08 &  9.93e-16 &      \\ 
     &           &    6 &  4.30e-08 &  9.92e-16 &      \\ 
     &           &    7 &  3.48e-08 &  1.23e-15 &      \\ 
     &           &    8 &  4.30e-08 &  9.92e-16 &      \\ 
     &           &    9 &  3.48e-08 &  1.23e-15 &      \\ 
     &           &   10 &  4.30e-08 &  9.92e-16 &      \\ 
 732 &  7.31e+03 &   10 &           &           & iters  \\ 
 \hdashline 
     &           &    1 &  7.44e-08 &  1.08e-07 &      \\ 
     &           &    2 &  3.41e-09 &  2.12e-15 &      \\ 
     &           &    3 &  3.41e-09 &  9.73e-17 &      \\ 
     &           &    4 &  3.40e-09 &  9.72e-17 &      \\ 
     &           &    5 &  3.40e-09 &  9.71e-17 &      \\ 
     &           &    6 &  3.39e-09 &  9.69e-17 &      \\ 
     &           &    7 &  3.39e-09 &  9.68e-17 &      \\ 
     &           &    8 &  3.38e-09 &  9.66e-17 &      \\ 
     &           &    9 &  3.38e-09 &  9.65e-17 &      \\ 
     &           &   10 &  3.37e-09 &  9.64e-17 &      \\ 
 733 &  7.32e+03 &   10 &           &           & iters  \\ 
 \hdashline 
     &           &    1 &  1.10e-07 &  5.07e-08 &      \\ 
     &           &    2 &  6.49e-09 &  3.15e-15 &      \\ 
     &           &    3 &  6.48e-09 &  1.85e-16 &      \\ 
     &           &    4 &  6.47e-09 &  1.85e-16 &      \\ 
     &           &    5 &  6.46e-09 &  1.85e-16 &      \\ 
     &           &    6 &  6.45e-09 &  1.84e-16 &      \\ 
     &           &    7 &  6.44e-09 &  1.84e-16 &      \\ 
     &           &    8 &  6.43e-09 &  1.84e-16 &      \\ 
     &           &    9 &  6.42e-09 &  1.84e-16 &      \\ 
     &           &   10 &  3.24e-08 &  1.83e-16 &      \\ 
 734 &  7.33e+03 &   10 &           &           & iters  \\ 
 \hdashline 
     &           &    1 &  3.30e-08 &  3.02e-08 &      \\ 
     &           &    2 &  3.30e-08 &  9.42e-16 &      \\ 
     &           &    3 &  5.93e-09 &  9.41e-16 &      \\ 
     &           &    4 &  5.92e-09 &  1.69e-16 &      \\ 
     &           &    5 &  5.92e-09 &  1.69e-16 &      \\ 
     &           &    6 &  5.91e-09 &  1.69e-16 &      \\ 
     &           &    7 &  5.90e-09 &  1.69e-16 &      \\ 
     &           &    8 &  5.89e-09 &  1.68e-16 &      \\ 
     &           &    9 &  3.30e-08 &  1.68e-16 &      \\ 
     &           &   10 &  5.93e-09 &  9.41e-16 &      \\ 
 735 &  7.34e+03 &   10 &           &           & iters  \\ 
 \hdashline 
     &           &    1 &  1.21e-08 &  3.07e-08 &      \\ 
     &           &    2 &  1.21e-08 &  3.44e-16 &      \\ 
     &           &    3 &  1.20e-08 &  3.44e-16 &      \\ 
     &           &    4 &  1.20e-08 &  3.44e-16 &      \\ 
     &           &    5 &  6.57e-08 &  3.43e-16 &      \\ 
     &           &    6 &  1.21e-08 &  1.87e-15 &      \\ 
     &           &    7 &  1.21e-08 &  3.45e-16 &      \\ 
     &           &    8 &  1.21e-08 &  3.45e-16 &      \\ 
     &           &    9 &  1.20e-08 &  3.44e-16 &      \\ 
     &           &   10 &  1.20e-08 &  3.44e-16 &      \\ 
 736 &  7.35e+03 &   10 &           &           & iters  \\ 
 \hdashline 
     &           &    1 &  4.96e-08 &  2.27e-08 &      \\ 
     &           &    2 &  2.81e-08 &  1.42e-15 &      \\ 
     &           &    3 &  2.81e-08 &  8.02e-16 &      \\ 
     &           &    4 &  4.97e-08 &  8.01e-16 &      \\ 
     &           &    5 &  2.81e-08 &  1.42e-15 &      \\ 
     &           &    6 &  4.96e-08 &  8.02e-16 &      \\ 
     &           &    7 &  2.81e-08 &  1.42e-15 &      \\ 
     &           &    8 &  2.81e-08 &  8.03e-16 &      \\ 
     &           &    9 &  4.96e-08 &  8.02e-16 &      \\ 
     &           &   10 &  2.81e-08 &  1.42e-15 &      \\ 
 737 &  7.36e+03 &   10 &           &           & iters  \\ 
 \hdashline 
     &           &    1 &  2.12e-08 &  4.58e-08 &      \\ 
     &           &    2 &  2.11e-08 &  6.04e-16 &      \\ 
     &           &    3 &  2.11e-08 &  6.03e-16 &      \\ 
     &           &    4 &  5.66e-08 &  6.02e-16 &      \\ 
     &           &    5 &  2.12e-08 &  1.62e-15 &      \\ 
     &           &    6 &  2.11e-08 &  6.04e-16 &      \\ 
     &           &    7 &  2.11e-08 &  6.03e-16 &      \\ 
     &           &    8 &  5.66e-08 &  6.02e-16 &      \\ 
     &           &    9 &  2.12e-08 &  1.62e-15 &      \\ 
     &           &   10 &  2.11e-08 &  6.04e-16 &      \\ 
 738 &  7.37e+03 &   10 &           &           & iters  \\ 
 \hdashline 
     &           &    1 &  1.26e-08 &  3.21e-08 &      \\ 
     &           &    2 &  2.63e-08 &  3.59e-16 &      \\ 
     &           &    3 &  1.26e-08 &  7.50e-16 &      \\ 
     &           &    4 &  1.26e-08 &  3.60e-16 &      \\ 
     &           &    5 &  2.63e-08 &  3.59e-16 &      \\ 
     &           &    6 &  1.26e-08 &  7.50e-16 &      \\ 
     &           &    7 &  1.26e-08 &  3.60e-16 &      \\ 
     &           &    8 &  2.63e-08 &  3.59e-16 &      \\ 
     &           &    9 &  1.26e-08 &  7.50e-16 &      \\ 
     &           &   10 &  1.26e-08 &  3.60e-16 &      \\ 
 739 &  7.38e+03 &   10 &           &           & iters  \\ 
 \hdashline 
     &           &    1 &  3.85e-08 &  1.05e-08 &      \\ 
     &           &    2 &  3.93e-08 &  1.10e-15 &      \\ 
     &           &    3 &  3.85e-08 &  1.12e-15 &      \\ 
     &           &    4 &  3.93e-08 &  1.10e-15 &      \\ 
     &           &    5 &  3.85e-08 &  1.12e-15 &      \\ 
     &           &    6 &  3.92e-08 &  1.10e-15 &      \\ 
     &           &    7 &  3.85e-08 &  1.12e-15 &      \\ 
     &           &    8 &  3.92e-08 &  1.10e-15 &      \\ 
     &           &    9 &  3.85e-08 &  1.12e-15 &      \\ 
     &           &   10 &  3.92e-08 &  1.10e-15 &      \\ 
 740 &  7.39e+03 &   10 &           &           & iters  \\ 
 \hdashline 
     &           &    1 &  1.65e-08 &  2.46e-08 &      \\ 
     &           &    2 &  2.24e-08 &  4.71e-16 &      \\ 
     &           &    3 &  1.65e-08 &  6.38e-16 &      \\ 
     &           &    4 &  2.24e-08 &  4.71e-16 &      \\ 
     &           &    5 &  1.65e-08 &  6.38e-16 &      \\ 
     &           &    6 &  1.65e-08 &  4.71e-16 &      \\ 
     &           &    7 &  2.24e-08 &  4.71e-16 &      \\ 
     &           &    8 &  1.65e-08 &  6.39e-16 &      \\ 
     &           &    9 &  2.24e-08 &  4.71e-16 &      \\ 
     &           &   10 &  1.65e-08 &  6.38e-16 &      \\ 
 741 &  7.40e+03 &   10 &           &           & iters  \\ 
 \hdashline 
     &           &    1 &  1.45e-08 &  3.30e-08 &      \\ 
     &           &    2 &  1.44e-08 &  4.13e-16 &      \\ 
     &           &    3 &  6.33e-08 &  4.12e-16 &      \\ 
     &           &    4 &  1.45e-08 &  1.81e-15 &      \\ 
     &           &    5 &  1.45e-08 &  4.14e-16 &      \\ 
     &           &    6 &  1.45e-08 &  4.13e-16 &      \\ 
     &           &    7 &  1.44e-08 &  4.13e-16 &      \\ 
     &           &    8 &  6.33e-08 &  4.12e-16 &      \\ 
     &           &    9 &  1.45e-08 &  1.81e-15 &      \\ 
     &           &   10 &  1.45e-08 &  4.14e-16 &      \\ 
 742 &  7.41e+03 &   10 &           &           & iters  \\ 
 \hdashline 
     &           &    1 &  5.05e-09 &  1.93e-08 &      \\ 
     &           &    2 &  5.05e-09 &  1.44e-16 &      \\ 
     &           &    3 &  5.04e-09 &  1.44e-16 &      \\ 
     &           &    4 &  5.03e-09 &  1.44e-16 &      \\ 
     &           &    5 &  5.02e-09 &  1.44e-16 &      \\ 
     &           &    6 &  5.02e-09 &  1.43e-16 &      \\ 
     &           &    7 &  5.01e-09 &  1.43e-16 &      \\ 
     &           &    8 &  5.00e-09 &  1.43e-16 &      \\ 
     &           &    9 &  5.00e-09 &  1.43e-16 &      \\ 
     &           &   10 &  4.99e-09 &  1.43e-16 &      \\ 
 743 &  7.42e+03 &   10 &           &           & iters  \\ 
 \hdashline 
     &           &    1 &  4.29e-08 &  1.19e-08 &      \\ 
     &           &    2 &  4.29e-08 &  1.23e-15 &      \\ 
     &           &    3 &  3.49e-08 &  1.22e-15 &      \\ 
     &           &    4 &  3.48e-08 &  9.95e-16 &      \\ 
     &           &    5 &  4.29e-08 &  9.94e-16 &      \\ 
     &           &    6 &  3.48e-08 &  1.22e-15 &      \\ 
     &           &    7 &  4.29e-08 &  9.94e-16 &      \\ 
     &           &    8 &  3.49e-08 &  1.22e-15 &      \\ 
     &           &    9 &  4.29e-08 &  9.95e-16 &      \\ 
     &           &   10 &  3.49e-08 &  1.22e-15 &      \\ 
 744 &  7.43e+03 &   10 &           &           & iters  \\ 
 \hdashline 
     &           &    1 &  5.40e-08 &  7.40e-09 &      \\ 
     &           &    2 &  2.38e-08 &  1.54e-15 &      \\ 
     &           &    3 &  2.37e-08 &  6.78e-16 &      \\ 
     &           &    4 &  5.40e-08 &  6.77e-16 &      \\ 
     &           &    5 &  2.38e-08 &  1.54e-15 &      \\ 
     &           &    6 &  2.37e-08 &  6.78e-16 &      \\ 
     &           &    7 &  2.37e-08 &  6.77e-16 &      \\ 
     &           &    8 &  5.40e-08 &  6.76e-16 &      \\ 
     &           &    9 &  2.37e-08 &  1.54e-15 &      \\ 
     &           &   10 &  2.37e-08 &  6.78e-16 &      \\ 
 745 &  7.44e+03 &   10 &           &           & iters  \\ 
 \hdashline 
     &           &    1 &  5.49e-08 &  6.17e-09 &      \\ 
     &           &    2 &  2.29e-08 &  1.57e-15 &      \\ 
     &           &    3 &  2.29e-08 &  6.54e-16 &      \\ 
     &           &    4 &  5.49e-08 &  6.53e-16 &      \\ 
     &           &    5 &  2.29e-08 &  1.57e-15 &      \\ 
     &           &    6 &  2.29e-08 &  6.54e-16 &      \\ 
     &           &    7 &  2.28e-08 &  6.53e-16 &      \\ 
     &           &    8 &  5.49e-08 &  6.52e-16 &      \\ 
     &           &    9 &  2.29e-08 &  1.57e-15 &      \\ 
     &           &   10 &  2.29e-08 &  6.53e-16 &      \\ 
 746 &  7.45e+03 &   10 &           &           & iters  \\ 
 \hdashline 
     &           &    1 &  5.85e-08 &  4.89e-09 &      \\ 
     &           &    2 &  1.92e-08 &  1.67e-15 &      \\ 
     &           &    3 &  1.92e-08 &  5.49e-16 &      \\ 
     &           &    4 &  1.92e-08 &  5.48e-16 &      \\ 
     &           &    5 &  1.91e-08 &  5.47e-16 &      \\ 
     &           &    6 &  5.86e-08 &  5.46e-16 &      \\ 
     &           &    7 &  1.92e-08 &  1.67e-15 &      \\ 
     &           &    8 &  1.92e-08 &  5.48e-16 &      \\ 
     &           &    9 &  1.91e-08 &  5.47e-16 &      \\ 
     &           &   10 &  5.86e-08 &  5.46e-16 &      \\ 
 747 &  7.46e+03 &   10 &           &           & iters  \\ 
 \hdashline 
     &           &    1 &  1.06e-07 &  5.73e-09 &      \\ 
     &           &    2 &  4.94e-08 &  3.03e-15 &      \\ 
     &           &    3 &  2.84e-08 &  1.41e-15 &      \\ 
     &           &    4 &  4.93e-08 &  8.11e-16 &      \\ 
     &           &    5 &  2.84e-08 &  1.41e-15 &      \\ 
     &           &    6 &  4.93e-08 &  8.12e-16 &      \\ 
     &           &    7 &  2.85e-08 &  1.41e-15 &      \\ 
     &           &    8 &  2.84e-08 &  8.12e-16 &      \\ 
     &           &    9 &  4.93e-08 &  8.11e-16 &      \\ 
     &           &   10 &  2.85e-08 &  1.41e-15 &      \\ 
 748 &  7.47e+03 &   10 &           &           & iters  \\ 
 \hdashline 
     &           &    1 &  2.57e-09 &  9.31e-09 &      \\ 
     &           &    2 &  2.57e-09 &  7.33e-17 &      \\ 
     &           &    3 &  2.56e-09 &  7.32e-17 &      \\ 
     &           &    4 &  2.56e-09 &  7.31e-17 &      \\ 
     &           &    5 &  2.55e-09 &  7.30e-17 &      \\ 
     &           &    6 &  2.55e-09 &  7.29e-17 &      \\ 
     &           &    7 &  2.55e-09 &  7.28e-17 &      \\ 
     &           &    8 &  2.54e-09 &  7.27e-17 &      \\ 
     &           &    9 &  2.54e-09 &  7.26e-17 &      \\ 
     &           &   10 &  7.52e-08 &  7.25e-17 &      \\ 
 749 &  7.48e+03 &   10 &           &           & iters  \\ 
 \hdashline 
     &           &    1 &  8.80e-08 &  2.24e-08 &      \\ 
     &           &    2 &  6.75e-08 &  2.51e-15 &      \\ 
     &           &    3 &  1.03e-08 &  1.93e-15 &      \\ 
     &           &    4 &  1.03e-08 &  2.93e-16 &      \\ 
     &           &    5 &  1.03e-08 &  2.93e-16 &      \\ 
     &           &    6 &  1.02e-08 &  2.93e-16 &      \\ 
     &           &    7 &  1.02e-08 &  2.92e-16 &      \\ 
     &           &    8 &  6.75e-08 &  2.92e-16 &      \\ 
     &           &    9 &  1.03e-08 &  1.93e-15 &      \\ 
     &           &   10 &  1.03e-08 &  2.94e-16 &      \\ 
 750 &  7.49e+03 &   10 &           &           & iters  \\ 
 \hdashline 
     &           &    1 &  1.91e-08 &  6.99e-09 &      \\ 
     &           &    2 &  1.91e-08 &  5.45e-16 &      \\ 
     &           &    3 &  1.91e-08 &  5.45e-16 &      \\ 
     &           &    4 &  5.87e-08 &  5.44e-16 &      \\ 
     &           &    5 &  1.91e-08 &  1.67e-15 &      \\ 
     &           &    6 &  1.91e-08 &  5.45e-16 &      \\ 
     &           &    7 &  1.91e-08 &  5.45e-16 &      \\ 
     &           &    8 &  5.87e-08 &  5.44e-16 &      \\ 
     &           &    9 &  1.91e-08 &  1.67e-15 &      \\ 
     &           &   10 &  1.91e-08 &  5.46e-16 &      \\ 
 751 &  7.50e+03 &   10 &           &           & iters  \\ 
 \hdashline 
     &           &    1 &  7.92e-09 &  2.26e-08 &      \\ 
     &           &    2 &  3.09e-08 &  2.26e-16 &      \\ 
     &           &    3 &  7.96e-09 &  8.83e-16 &      \\ 
     &           &    4 &  7.94e-09 &  2.27e-16 &      \\ 
     &           &    5 &  7.93e-09 &  2.27e-16 &      \\ 
     &           &    6 &  7.92e-09 &  2.26e-16 &      \\ 
     &           &    7 &  3.09e-08 &  2.26e-16 &      \\ 
     &           &    8 &  7.95e-09 &  8.83e-16 &      \\ 
     &           &    9 &  7.94e-09 &  2.27e-16 &      \\ 
     &           &   10 &  7.93e-09 &  2.27e-16 &      \\ 
 752 &  7.51e+03 &   10 &           &           & iters  \\ 
 \hdashline 
     &           &    1 &  1.09e-08 &  4.89e-08 &      \\ 
     &           &    2 &  1.09e-08 &  3.12e-16 &      \\ 
     &           &    3 &  1.09e-08 &  3.11e-16 &      \\ 
     &           &    4 &  6.68e-08 &  3.11e-16 &      \\ 
     &           &    5 &  1.10e-08 &  1.91e-15 &      \\ 
     &           &    6 &  1.10e-08 &  3.13e-16 &      \\ 
     &           &    7 &  1.09e-08 &  3.13e-16 &      \\ 
     &           &    8 &  1.09e-08 &  3.12e-16 &      \\ 
     &           &    9 &  1.09e-08 &  3.12e-16 &      \\ 
     &           &   10 &  1.09e-08 &  3.12e-16 &      \\ 
 753 &  7.52e+03 &   10 &           &           & iters  \\ 
 \hdashline 
     &           &    1 &  1.96e-09 &  4.35e-08 &      \\ 
     &           &    2 &  1.96e-09 &  5.61e-17 &      \\ 
     &           &    3 &  1.96e-09 &  5.60e-17 &      \\ 
     &           &    4 &  1.96e-09 &  5.59e-17 &      \\ 
     &           &    5 &  1.95e-09 &  5.58e-17 &      \\ 
     &           &    6 &  1.95e-09 &  5.57e-17 &      \\ 
     &           &    7 &  1.95e-09 &  5.57e-17 &      \\ 
     &           &    8 &  1.94e-09 &  5.56e-17 &      \\ 
     &           &    9 &  1.94e-09 &  5.55e-17 &      \\ 
     &           &   10 &  1.94e-09 &  5.54e-17 &      \\ 
 754 &  7.53e+03 &   10 &           &           & iters  \\ 
 \hdashline 
     &           &    1 &  1.14e-08 &  9.15e-09 &      \\ 
     &           &    2 &  1.14e-08 &  3.25e-16 &      \\ 
     &           &    3 &  6.63e-08 &  3.25e-16 &      \\ 
     &           &    4 &  1.15e-08 &  1.89e-15 &      \\ 
     &           &    5 &  1.15e-08 &  3.27e-16 &      \\ 
     &           &    6 &  1.14e-08 &  3.27e-16 &      \\ 
     &           &    7 &  1.14e-08 &  3.26e-16 &      \\ 
     &           &    8 &  1.14e-08 &  3.26e-16 &      \\ 
     &           &    9 &  1.14e-08 &  3.25e-16 &      \\ 
     &           &   10 &  6.63e-08 &  3.25e-16 &      \\ 
 755 &  7.54e+03 &   10 &           &           & iters  \\ 
 \hdashline 
     &           &    1 &  2.32e-08 &  2.99e-09 &      \\ 
     &           &    2 &  2.32e-08 &  6.62e-16 &      \\ 
     &           &    3 &  2.31e-08 &  6.61e-16 &      \\ 
     &           &    4 &  5.46e-08 &  6.60e-16 &      \\ 
     &           &    5 &  2.32e-08 &  1.56e-15 &      \\ 
     &           &    6 &  2.31e-08 &  6.61e-16 &      \\ 
     &           &    7 &  5.46e-08 &  6.60e-16 &      \\ 
     &           &    8 &  2.32e-08 &  1.56e-15 &      \\ 
     &           &    9 &  2.31e-08 &  6.61e-16 &      \\ 
     &           &   10 &  2.31e-08 &  6.60e-16 &      \\ 
 756 &  7.55e+03 &   10 &           &           & iters  \\ 
 \hdashline 
     &           &    1 &  7.95e-08 &  2.32e-09 &      \\ 
     &           &    2 &  1.72e-09 &  2.27e-15 &      \\ 
     &           &    3 &  1.71e-09 &  4.90e-17 &      \\ 
     &           &    4 &  1.71e-09 &  4.89e-17 &      \\ 
     &           &    5 &  1.71e-09 &  4.88e-17 &      \\ 
     &           &    6 &  1.71e-09 &  4.88e-17 &      \\ 
     &           &    7 &  1.70e-09 &  4.87e-17 &      \\ 
     &           &    8 &  1.70e-09 &  4.86e-17 &      \\ 
     &           &    9 &  1.70e-09 &  4.86e-17 &      \\ 
     &           &   10 &  1.70e-09 &  4.85e-17 &      \\ 
 757 &  7.56e+03 &   10 &           &           & iters  \\ 
 \hdashline 
     &           &    1 &  3.70e-08 &  2.14e-08 &      \\ 
     &           &    2 &  4.07e-08 &  1.06e-15 &      \\ 
     &           &    3 &  3.71e-08 &  1.16e-15 &      \\ 
     &           &    4 &  4.07e-08 &  1.06e-15 &      \\ 
     &           &    5 &  3.71e-08 &  1.16e-15 &      \\ 
     &           &    6 &  4.07e-08 &  1.06e-15 &      \\ 
     &           &    7 &  3.71e-08 &  1.16e-15 &      \\ 
     &           &    8 &  3.70e-08 &  1.06e-15 &      \\ 
     &           &    9 &  4.07e-08 &  1.06e-15 &      \\ 
     &           &   10 &  3.70e-08 &  1.16e-15 &      \\ 
 758 &  7.57e+03 &   10 &           &           & iters  \\ 
 \hdashline 
     &           &    1 &  1.69e-08 &  4.08e-09 &      \\ 
     &           &    2 &  6.08e-08 &  4.83e-16 &      \\ 
     &           &    3 &  1.70e-08 &  1.74e-15 &      \\ 
     &           &    4 &  1.70e-08 &  4.85e-16 &      \\ 
     &           &    5 &  1.69e-08 &  4.84e-16 &      \\ 
     &           &    6 &  6.08e-08 &  4.83e-16 &      \\ 
     &           &    7 &  1.70e-08 &  1.73e-15 &      \\ 
     &           &    8 &  1.70e-08 &  4.85e-16 &      \\ 
     &           &    9 &  1.69e-08 &  4.84e-16 &      \\ 
     &           &   10 &  1.69e-08 &  4.84e-16 &      \\ 
 759 &  7.58e+03 &   10 &           &           & iters  \\ 
 \hdashline 
     &           &    1 &  3.54e-08 &  3.76e-08 &      \\ 
     &           &    2 &  3.53e-08 &  1.01e-15 &      \\ 
     &           &    3 &  4.24e-08 &  1.01e-15 &      \\ 
     &           &    4 &  3.53e-08 &  1.21e-15 &      \\ 
     &           &    5 &  4.24e-08 &  1.01e-15 &      \\ 
     &           &    6 &  3.53e-08 &  1.21e-15 &      \\ 
     &           &    7 &  3.53e-08 &  1.01e-15 &      \\ 
     &           &    8 &  4.25e-08 &  1.01e-15 &      \\ 
     &           &    9 &  3.53e-08 &  1.21e-15 &      \\ 
     &           &   10 &  4.24e-08 &  1.01e-15 &      \\ 
 760 &  7.59e+03 &   10 &           &           & iters  \\ 
 \hdashline 
     &           &    1 &  3.41e-08 &  3.54e-08 &      \\ 
     &           &    2 &  4.36e-08 &  9.73e-16 &      \\ 
     &           &    3 &  3.41e-08 &  1.25e-15 &      \\ 
     &           &    4 &  4.36e-08 &  9.74e-16 &      \\ 
     &           &    5 &  3.41e-08 &  1.25e-15 &      \\ 
     &           &    6 &  4.36e-08 &  9.74e-16 &      \\ 
     &           &    7 &  3.41e-08 &  1.24e-15 &      \\ 
     &           &    8 &  3.41e-08 &  9.74e-16 &      \\ 
     &           &    9 &  4.36e-08 &  9.73e-16 &      \\ 
     &           &   10 &  3.41e-08 &  1.25e-15 &      \\ 
 761 &  7.60e+03 &   10 &           &           & iters  \\ 
 \hdashline 
     &           &    1 &  1.43e-09 &  4.47e-09 &      \\ 
     &           &    2 &  1.43e-09 &  4.09e-17 &      \\ 
     &           &    3 &  1.43e-09 &  4.09e-17 &      \\ 
     &           &    4 &  1.43e-09 &  4.08e-17 &      \\ 
     &           &    5 &  1.43e-09 &  4.08e-17 &      \\ 
     &           &    6 &  1.42e-09 &  4.07e-17 &      \\ 
     &           &    7 &  1.42e-09 &  4.06e-17 &      \\ 
     &           &    8 &  1.42e-09 &  4.06e-17 &      \\ 
     &           &    9 &  1.42e-09 &  4.05e-17 &      \\ 
     &           &   10 &  1.42e-09 &  4.05e-17 &      \\ 
 762 &  7.61e+03 &   10 &           &           & iters  \\ 
 \hdashline 
     &           &    1 &  3.32e-08 &  2.90e-08 &      \\ 
     &           &    2 &  4.45e-08 &  9.49e-16 &      \\ 
     &           &    3 &  3.33e-08 &  1.27e-15 &      \\ 
     &           &    4 &  3.32e-08 &  9.49e-16 &      \\ 
     &           &    5 &  4.45e-08 &  9.48e-16 &      \\ 
     &           &    6 &  3.32e-08 &  1.27e-15 &      \\ 
     &           &    7 &  4.45e-08 &  9.48e-16 &      \\ 
     &           &    8 &  3.32e-08 &  1.27e-15 &      \\ 
     &           &    9 &  4.45e-08 &  9.49e-16 &      \\ 
     &           &   10 &  3.33e-08 &  1.27e-15 &      \\ 
 763 &  7.62e+03 &   10 &           &           & iters  \\ 
 \hdashline 
     &           &    1 &  3.70e-08 &  3.17e-08 &      \\ 
     &           &    2 &  4.08e-08 &  1.06e-15 &      \\ 
     &           &    3 &  3.70e-08 &  1.16e-15 &      \\ 
     &           &    4 &  4.08e-08 &  1.06e-15 &      \\ 
     &           &    5 &  3.70e-08 &  1.16e-15 &      \\ 
     &           &    6 &  4.08e-08 &  1.06e-15 &      \\ 
     &           &    7 &  3.70e-08 &  1.16e-15 &      \\ 
     &           &    8 &  3.69e-08 &  1.06e-15 &      \\ 
     &           &    9 &  4.08e-08 &  1.05e-15 &      \\ 
     &           &   10 &  3.69e-08 &  1.16e-15 &      \\ 
 764 &  7.63e+03 &   10 &           &           & iters  \\ 
 \hdashline 
     &           &    1 &  3.78e-09 &  7.60e-10 &      \\ 
     &           &    2 &  3.78e-09 &  1.08e-16 &      \\ 
     &           &    3 &  3.77e-09 &  1.08e-16 &      \\ 
     &           &    4 &  3.77e-09 &  1.08e-16 &      \\ 
     &           &    5 &  3.76e-09 &  1.08e-16 &      \\ 
     &           &    6 &  3.76e-09 &  1.07e-16 &      \\ 
     &           &    7 &  3.75e-09 &  1.07e-16 &      \\ 
     &           &    8 &  3.75e-09 &  1.07e-16 &      \\ 
     &           &    9 &  3.74e-09 &  1.07e-16 &      \\ 
     &           &   10 &  3.74e-09 &  1.07e-16 &      \\ 
 765 &  7.64e+03 &   10 &           &           & iters  \\ 
 \hdashline 
     &           &    1 &  1.17e-08 &  1.03e-08 &      \\ 
     &           &    2 &  1.16e-08 &  3.33e-16 &      \\ 
     &           &    3 &  1.16e-08 &  3.32e-16 &      \\ 
     &           &    4 &  1.16e-08 &  3.32e-16 &      \\ 
     &           &    5 &  6.61e-08 &  3.31e-16 &      \\ 
     &           &    6 &  8.94e-08 &  1.89e-15 &      \\ 
     &           &    7 &  6.61e-08 &  2.55e-15 &      \\ 
     &           &    8 &  1.17e-08 &  1.89e-15 &      \\ 
     &           &    9 &  1.16e-08 &  3.33e-16 &      \\ 
     &           &   10 &  1.16e-08 &  3.32e-16 &      \\ 
 766 &  7.65e+03 &   10 &           &           & iters  \\ 
 \hdashline 
     &           &    1 &  6.74e-09 &  1.12e-08 &      \\ 
     &           &    2 &  6.73e-09 &  1.92e-16 &      \\ 
     &           &    3 &  6.72e-09 &  1.92e-16 &      \\ 
     &           &    4 &  6.71e-09 &  1.92e-16 &      \\ 
     &           &    5 &  6.71e-09 &  1.92e-16 &      \\ 
     &           &    6 &  6.70e-09 &  1.91e-16 &      \\ 
     &           &    7 &  6.69e-09 &  1.91e-16 &      \\ 
     &           &    8 &  6.68e-09 &  1.91e-16 &      \\ 
     &           &    9 &  6.67e-09 &  1.91e-16 &      \\ 
     &           &   10 &  6.66e-09 &  1.90e-16 &      \\ 
 767 &  7.66e+03 &   10 &           &           & iters  \\ 
 \hdashline 
     &           &    1 &  5.68e-09 &  2.14e-08 &      \\ 
     &           &    2 &  5.67e-09 &  1.62e-16 &      \\ 
     &           &    3 &  5.66e-09 &  1.62e-16 &      \\ 
     &           &    4 &  5.65e-09 &  1.62e-16 &      \\ 
     &           &    5 &  5.64e-09 &  1.61e-16 &      \\ 
     &           &    6 &  5.64e-09 &  1.61e-16 &      \\ 
     &           &    7 &  5.63e-09 &  1.61e-16 &      \\ 
     &           &    8 &  5.62e-09 &  1.61e-16 &      \\ 
     &           &    9 &  5.61e-09 &  1.60e-16 &      \\ 
     &           &   10 &  5.60e-09 &  1.60e-16 &      \\ 
 768 &  7.67e+03 &   10 &           &           & iters  \\ 
 \hdashline 
     &           &    1 &  4.04e-08 &  9.63e-09 &      \\ 
     &           &    2 &  3.73e-08 &  1.15e-15 &      \\ 
     &           &    3 &  4.04e-08 &  1.06e-15 &      \\ 
     &           &    4 &  3.73e-08 &  1.15e-15 &      \\ 
     &           &    5 &  4.04e-08 &  1.06e-15 &      \\ 
     &           &    6 &  3.73e-08 &  1.15e-15 &      \\ 
     &           &    7 &  4.04e-08 &  1.06e-15 &      \\ 
     &           &    8 &  3.73e-08 &  1.15e-15 &      \\ 
     &           &    9 &  4.04e-08 &  1.06e-15 &      \\ 
     &           &   10 &  3.73e-08 &  1.15e-15 &      \\ 
 769 &  7.68e+03 &   10 &           &           & iters  \\ 
 \hdashline 
     &           &    1 &  1.02e-09 &  4.15e-08 &      \\ 
     &           &    2 &  1.02e-09 &  2.91e-17 &      \\ 
     &           &    3 &  1.02e-09 &  2.90e-17 &      \\ 
     &           &    4 &  1.01e-09 &  2.90e-17 &      \\ 
     &           &    5 &  1.01e-09 &  2.90e-17 &      \\ 
     &           &    6 &  1.01e-09 &  2.89e-17 &      \\ 
     &           &    7 &  1.01e-09 &  2.89e-17 &      \\ 
     &           &    8 &  1.01e-09 &  2.88e-17 &      \\ 
     &           &    9 &  1.01e-09 &  2.88e-17 &      \\ 
     &           &   10 &  1.01e-09 &  2.88e-17 &      \\ 
 770 &  7.69e+03 &   10 &           &           & iters  \\ 
 \hdashline 
     &           &    1 &  1.86e-08 &  1.18e-08 &      \\ 
     &           &    2 &  1.86e-08 &  5.32e-16 &      \\ 
     &           &    3 &  1.86e-08 &  5.31e-16 &      \\ 
     &           &    4 &  5.91e-08 &  5.30e-16 &      \\ 
     &           &    5 &  1.86e-08 &  1.69e-15 &      \\ 
     &           &    6 &  1.86e-08 &  5.32e-16 &      \\ 
     &           &    7 &  1.86e-08 &  5.31e-16 &      \\ 
     &           &    8 &  5.91e-08 &  5.30e-16 &      \\ 
     &           &    9 &  1.86e-08 &  1.69e-15 &      \\ 
     &           &   10 &  1.86e-08 &  5.32e-16 &      \\ 
 771 &  7.70e+03 &   10 &           &           & iters  \\ 
 \hdashline 
     &           &    1 &  2.80e-08 &  6.05e-09 &      \\ 
     &           &    2 &  4.97e-08 &  8.00e-16 &      \\ 
     &           &    3 &  2.81e-08 &  1.42e-15 &      \\ 
     &           &    4 &  4.97e-08 &  8.01e-16 &      \\ 
     &           &    5 &  2.81e-08 &  1.42e-15 &      \\ 
     &           &    6 &  2.81e-08 &  8.02e-16 &      \\ 
     &           &    7 &  4.97e-08 &  8.01e-16 &      \\ 
     &           &    8 &  2.81e-08 &  1.42e-15 &      \\ 
     &           &    9 &  2.80e-08 &  8.02e-16 &      \\ 
     &           &   10 &  4.97e-08 &  8.00e-16 &      \\ 
 772 &  7.71e+03 &   10 &           &           & iters  \\ 
 \hdashline 
     &           &    1 &  5.20e-09 &  4.36e-08 &      \\ 
     &           &    2 &  5.19e-09 &  1.48e-16 &      \\ 
     &           &    3 &  5.19e-09 &  1.48e-16 &      \\ 
     &           &    4 &  5.18e-09 &  1.48e-16 &      \\ 
     &           &    5 &  7.25e-08 &  1.48e-16 &      \\ 
     &           &    6 &  5.28e-09 &  2.07e-15 &      \\ 
     &           &    7 &  5.27e-09 &  1.51e-16 &      \\ 
     &           &    8 &  5.26e-09 &  1.50e-16 &      \\ 
     &           &    9 &  5.25e-09 &  1.50e-16 &      \\ 
     &           &   10 &  5.25e-09 &  1.50e-16 &      \\ 
 773 &  7.72e+03 &   10 &           &           & iters  \\ 
 \hdashline 
     &           &    1 &  4.31e-08 &  1.20e-07 &      \\ 
     &           &    2 &  4.19e-09 &  1.23e-15 &      \\ 
     &           &    3 &  4.18e-09 &  1.20e-16 &      \\ 
     &           &    4 &  4.18e-09 &  1.19e-16 &      \\ 
     &           &    5 &  4.17e-09 &  1.19e-16 &      \\ 
     &           &    6 &  4.17e-09 &  1.19e-16 &      \\ 
     &           &    7 &  4.16e-09 &  1.19e-16 &      \\ 
     &           &    8 &  4.15e-09 &  1.19e-16 &      \\ 
     &           &    9 &  7.35e-08 &  1.19e-16 &      \\ 
     &           &   10 &  4.31e-08 &  2.10e-15 &      \\ 
 774 &  7.73e+03 &   10 &           &           & iters  \\ 
 \hdashline 
     &           &    1 &  1.29e-07 &  1.70e-07 &      \\ 
     &           &    2 &  1.26e-08 &  3.69e-15 &      \\ 
     &           &    3 &  6.51e-08 &  3.59e-16 &      \\ 
     &           &    4 &  5.15e-08 &  1.86e-15 &      \\ 
     &           &    5 &  1.26e-08 &  1.47e-15 &      \\ 
     &           &    6 &  6.51e-08 &  3.60e-16 &      \\ 
     &           &    7 &  5.15e-08 &  1.86e-15 &      \\ 
     &           &    8 &  1.26e-08 &  1.47e-15 &      \\ 
     &           &    9 &  6.51e-08 &  3.60e-16 &      \\ 
     &           &   10 &  5.15e-08 &  1.86e-15 &      \\ 
 775 &  7.74e+03 &   10 &           &           & iters  \\ 
 \hdashline 
     &           &    1 &  8.63e-08 &  6.15e-08 &      \\ 
     &           &    2 &  6.92e-08 &  2.46e-15 &      \\ 
     &           &    3 &  8.54e-09 &  1.98e-15 &      \\ 
     &           &    4 &  8.53e-09 &  2.44e-16 &      \\ 
     &           &    5 &  8.51e-09 &  2.43e-16 &      \\ 
     &           &    6 &  8.50e-09 &  2.43e-16 &      \\ 
     &           &    7 &  8.49e-09 &  2.43e-16 &      \\ 
     &           &    8 &  8.48e-09 &  2.42e-16 &      \\ 
     &           &    9 &  8.47e-09 &  2.42e-16 &      \\ 
     &           &   10 &  8.45e-09 &  2.42e-16 &      \\ 
 776 &  7.75e+03 &   10 &           &           & iters  \\ 
 \hdashline 
     &           &    1 &  9.08e-08 &  4.58e-08 &      \\ 
     &           &    2 &  1.30e-08 &  2.59e-15 &      \\ 
     &           &    3 &  6.47e-08 &  3.71e-16 &      \\ 
     &           &    4 &  1.31e-08 &  1.85e-15 &      \\ 
     &           &    5 &  1.31e-08 &  3.73e-16 &      \\ 
     &           &    6 &  1.30e-08 &  3.73e-16 &      \\ 
     &           &    7 &  1.30e-08 &  3.72e-16 &      \\ 
     &           &    8 &  1.30e-08 &  3.71e-16 &      \\ 
     &           &    9 &  6.47e-08 &  3.71e-16 &      \\ 
     &           &   10 &  1.31e-08 &  1.85e-15 &      \\ 
 777 &  7.76e+03 &   10 &           &           & iters  \\ 
 \hdashline 
     &           &    1 &  9.04e-08 &  5.59e-08 &      \\ 
     &           &    2 &  2.62e-08 &  2.58e-15 &      \\ 
     &           &    3 &  5.15e-08 &  7.49e-16 &      \\ 
     &           &    4 &  2.63e-08 &  1.47e-15 &      \\ 
     &           &    5 &  2.62e-08 &  7.50e-16 &      \\ 
     &           &    6 &  5.15e-08 &  7.49e-16 &      \\ 
     &           &    7 &  2.63e-08 &  1.47e-15 &      \\ 
     &           &    8 &  2.62e-08 &  7.50e-16 &      \\ 
     &           &    9 &  5.15e-08 &  7.49e-16 &      \\ 
     &           &   10 &  2.63e-08 &  1.47e-15 &      \\ 
 778 &  7.77e+03 &   10 &           &           & iters  \\ 
 \hdashline 
     &           &    1 &  6.52e-08 &  7.10e-08 &      \\ 
     &           &    2 &  1.26e-08 &  1.86e-15 &      \\ 
     &           &    3 &  1.26e-08 &  3.60e-16 &      \\ 
     &           &    4 &  1.26e-08 &  3.60e-16 &      \\ 
     &           &    5 &  1.26e-08 &  3.59e-16 &      \\ 
     &           &    6 &  1.25e-08 &  3.59e-16 &      \\ 
     &           &    7 &  6.52e-08 &  3.58e-16 &      \\ 
     &           &    8 &  1.26e-08 &  1.86e-15 &      \\ 
     &           &    9 &  1.26e-08 &  3.60e-16 &      \\ 
     &           &   10 &  1.26e-08 &  3.60e-16 &      \\ 
 779 &  7.78e+03 &   10 &           &           & iters  \\ 
 \hdashline 
     &           &    1 &  5.00e-08 &  4.90e-08 &      \\ 
     &           &    2 &  2.77e-08 &  1.43e-15 &      \\ 
     &           &    3 &  5.00e-08 &  7.91e-16 &      \\ 
     &           &    4 &  2.78e-08 &  1.43e-15 &      \\ 
     &           &    5 &  2.77e-08 &  7.92e-16 &      \\ 
     &           &    6 &  5.00e-08 &  7.91e-16 &      \\ 
     &           &    7 &  2.78e-08 &  1.43e-15 &      \\ 
     &           &    8 &  2.77e-08 &  7.92e-16 &      \\ 
     &           &    9 &  5.00e-08 &  7.91e-16 &      \\ 
     &           &   10 &  2.77e-08 &  1.43e-15 &      \\ 
 780 &  7.79e+03 &   10 &           &           & iters  \\ 
 \hdashline 
     &           &    1 &  1.01e-08 &  9.31e-09 &      \\ 
     &           &    2 &  2.88e-08 &  2.87e-16 &      \\ 
     &           &    3 &  1.01e-08 &  8.22e-16 &      \\ 
     &           &    4 &  1.01e-08 &  2.88e-16 &      \\ 
     &           &    5 &  1.01e-08 &  2.88e-16 &      \\ 
     &           &    6 &  2.88e-08 &  2.87e-16 &      \\ 
     &           &    7 &  1.01e-08 &  8.22e-16 &      \\ 
     &           &    8 &  1.01e-08 &  2.88e-16 &      \\ 
     &           &    9 &  1.01e-08 &  2.88e-16 &      \\ 
     &           &   10 &  2.88e-08 &  2.87e-16 &      \\ 
 781 &  7.80e+03 &   10 &           &           & iters  \\ 
 \hdashline 
     &           &    1 &  9.50e-09 &  2.38e-08 &      \\ 
     &           &    2 &  9.48e-09 &  2.71e-16 &      \\ 
     &           &    3 &  9.47e-09 &  2.71e-16 &      \\ 
     &           &    4 &  9.45e-09 &  2.70e-16 &      \\ 
     &           &    5 &  9.44e-09 &  2.70e-16 &      \\ 
     &           &    6 &  6.83e-08 &  2.69e-16 &      \\ 
     &           &    7 &  4.84e-08 &  1.95e-15 &      \\ 
     &           &    8 &  9.46e-09 &  1.38e-15 &      \\ 
     &           &    9 &  9.44e-09 &  2.70e-16 &      \\ 
     &           &   10 &  6.83e-08 &  2.69e-16 &      \\ 
 782 &  7.81e+03 &   10 &           &           & iters  \\ 
 \hdashline 
     &           &    1 &  4.62e-08 &  2.98e-08 &      \\ 
     &           &    2 &  7.33e-09 &  1.32e-15 &      \\ 
     &           &    3 &  7.31e-09 &  2.09e-16 &      \\ 
     &           &    4 &  7.30e-09 &  2.09e-16 &      \\ 
     &           &    5 &  7.29e-09 &  2.08e-16 &      \\ 
     &           &    6 &  7.28e-09 &  2.08e-16 &      \\ 
     &           &    7 &  7.27e-09 &  2.08e-16 &      \\ 
     &           &    8 &  7.04e-08 &  2.08e-16 &      \\ 
     &           &    9 &  4.62e-08 &  2.01e-15 &      \\ 
     &           &   10 &  7.30e-09 &  1.32e-15 &      \\ 
 783 &  7.82e+03 &   10 &           &           & iters  \\ 
 \hdashline 
     &           &    1 &  1.52e-08 &  9.80e-08 &      \\ 
     &           &    2 &  1.52e-08 &  4.33e-16 &      \\ 
     &           &    3 &  6.25e-08 &  4.33e-16 &      \\ 
     &           &    4 &  1.52e-08 &  1.79e-15 &      \\ 
     &           &    5 &  1.52e-08 &  4.35e-16 &      \\ 
     &           &    6 &  1.52e-08 &  4.34e-16 &      \\ 
     &           &    7 &  1.52e-08 &  4.34e-16 &      \\ 
     &           &    8 &  1.51e-08 &  4.33e-16 &      \\ 
     &           &    9 &  6.26e-08 &  4.32e-16 &      \\ 
     &           &   10 &  1.52e-08 &  1.79e-15 &      \\ 
 784 &  7.83e+03 &   10 &           &           & iters  \\ 
 \hdashline 
     &           &    1 &  4.65e-08 &  7.32e-08 &      \\ 
     &           &    2 &  3.12e-08 &  1.33e-15 &      \\ 
     &           &    3 &  3.12e-08 &  8.91e-16 &      \\ 
     &           &    4 &  4.66e-08 &  8.90e-16 &      \\ 
     &           &    5 &  3.12e-08 &  1.33e-15 &      \\ 
     &           &    6 &  4.65e-08 &  8.91e-16 &      \\ 
     &           &    7 &  3.12e-08 &  1.33e-15 &      \\ 
     &           &    8 &  3.12e-08 &  8.91e-16 &      \\ 
     &           &    9 &  4.66e-08 &  8.90e-16 &      \\ 
     &           &   10 &  3.12e-08 &  1.33e-15 &      \\ 
 785 &  7.84e+03 &   10 &           &           & iters  \\ 
 \hdashline 
     &           &    1 &  3.06e-08 &  3.46e-08 &      \\ 
     &           &    2 &  4.71e-08 &  8.73e-16 &      \\ 
     &           &    3 &  3.06e-08 &  1.35e-15 &      \\ 
     &           &    4 &  3.06e-08 &  8.74e-16 &      \\ 
     &           &    5 &  4.72e-08 &  8.73e-16 &      \\ 
     &           &    6 &  3.06e-08 &  1.35e-15 &      \\ 
     &           &    7 &  4.71e-08 &  8.73e-16 &      \\ 
     &           &    8 &  3.06e-08 &  1.35e-15 &      \\ 
     &           &    9 &  3.06e-08 &  8.74e-16 &      \\ 
     &           &   10 &  4.71e-08 &  8.73e-16 &      \\ 
 786 &  7.85e+03 &   10 &           &           & iters  \\ 
 \hdashline 
     &           &    1 &  3.02e-09 &  7.16e-08 &      \\ 
     &           &    2 &  3.01e-09 &  8.61e-17 &      \\ 
     &           &    3 &  3.01e-09 &  8.60e-17 &      \\ 
     &           &    4 &  3.00e-09 &  8.59e-17 &      \\ 
     &           &    5 &  3.00e-09 &  8.57e-17 &      \\ 
     &           &    6 &  3.00e-09 &  8.56e-17 &      \\ 
     &           &    7 &  2.99e-09 &  8.55e-17 &      \\ 
     &           &    8 &  2.99e-09 &  8.54e-17 &      \\ 
     &           &    9 &  2.98e-09 &  8.52e-17 &      \\ 
     &           &   10 &  7.47e-08 &  8.51e-17 &      \\ 
 787 &  7.86e+03 &   10 &           &           & iters  \\ 
 \hdashline 
     &           &    1 &  5.27e-08 &  8.14e-09 &      \\ 
     &           &    2 &  2.51e-08 &  1.50e-15 &      \\ 
     &           &    3 &  2.50e-08 &  7.15e-16 &      \\ 
     &           &    4 &  5.27e-08 &  7.14e-16 &      \\ 
     &           &    5 &  2.51e-08 &  1.50e-15 &      \\ 
     &           &    6 &  2.50e-08 &  7.15e-16 &      \\ 
     &           &    7 &  5.27e-08 &  7.14e-16 &      \\ 
     &           &    8 &  2.51e-08 &  1.50e-15 &      \\ 
     &           &    9 &  2.50e-08 &  7.15e-16 &      \\ 
     &           &   10 &  5.27e-08 &  7.14e-16 &      \\ 
 788 &  7.87e+03 &   10 &           &           & iters  \\ 
 \hdashline 
     &           &    1 &  5.19e-09 &  2.79e-08 &      \\ 
     &           &    2 &  5.19e-09 &  1.48e-16 &      \\ 
     &           &    3 &  7.25e-08 &  1.48e-16 &      \\ 
     &           &    4 &  8.30e-08 &  2.07e-15 &      \\ 
     &           &    5 &  7.25e-08 &  2.37e-15 &      \\ 
     &           &    6 &  5.27e-09 &  2.07e-15 &      \\ 
     &           &    7 &  5.26e-09 &  1.50e-16 &      \\ 
     &           &    8 &  5.25e-09 &  1.50e-16 &      \\ 
     &           &    9 &  5.24e-09 &  1.50e-16 &      \\ 
     &           &   10 &  5.24e-09 &  1.50e-16 &      \\ 
 789 &  7.88e+03 &   10 &           &           & iters  \\ 
 \hdashline 
     &           &    1 &  3.79e-08 &  9.03e-09 &      \\ 
     &           &    2 &  3.98e-08 &  1.08e-15 &      \\ 
     &           &    3 &  3.79e-08 &  1.14e-15 &      \\ 
     &           &    4 &  3.98e-08 &  1.08e-15 &      \\ 
     &           &    5 &  3.79e-08 &  1.14e-15 &      \\ 
     &           &    6 &  3.98e-08 &  1.08e-15 &      \\ 
     &           &    7 &  3.79e-08 &  1.14e-15 &      \\ 
     &           &    8 &  3.98e-08 &  1.08e-15 &      \\ 
     &           &    9 &  3.79e-08 &  1.14e-15 &      \\ 
     &           &   10 &  3.98e-08 &  1.08e-15 &      \\ 
 790 &  7.89e+03 &   10 &           &           & iters  \\ 
 \hdashline 
     &           &    1 &  2.18e-08 &  1.43e-08 &      \\ 
     &           &    2 &  2.18e-08 &  6.22e-16 &      \\ 
     &           &    3 &  5.59e-08 &  6.21e-16 &      \\ 
     &           &    4 &  2.18e-08 &  1.60e-15 &      \\ 
     &           &    5 &  2.18e-08 &  6.23e-16 &      \\ 
     &           &    6 &  2.18e-08 &  6.22e-16 &      \\ 
     &           &    7 &  5.60e-08 &  6.21e-16 &      \\ 
     &           &    8 &  2.18e-08 &  1.60e-15 &      \\ 
     &           &    9 &  2.18e-08 &  6.22e-16 &      \\ 
     &           &   10 &  5.59e-08 &  6.21e-16 &      \\ 
 791 &  7.90e+03 &   10 &           &           & iters  \\ 
 \hdashline 
     &           &    1 &  6.21e-08 &  3.24e-08 &      \\ 
     &           &    2 &  1.56e-08 &  1.77e-15 &      \\ 
     &           &    3 &  1.56e-08 &  4.46e-16 &      \\ 
     &           &    4 &  6.21e-08 &  4.45e-16 &      \\ 
     &           &    5 &  1.57e-08 &  1.77e-15 &      \\ 
     &           &    6 &  1.57e-08 &  4.47e-16 &      \\ 
     &           &    7 &  1.56e-08 &  4.47e-16 &      \\ 
     &           &    8 &  1.56e-08 &  4.46e-16 &      \\ 
     &           &    9 &  6.21e-08 &  4.45e-16 &      \\ 
     &           &   10 &  1.57e-08 &  1.77e-15 &      \\ 
 792 &  7.91e+03 &   10 &           &           & iters  \\ 
 \hdashline 
     &           &    1 &  6.07e-08 &  2.88e-09 &      \\ 
     &           &    2 &  1.71e-08 &  1.73e-15 &      \\ 
     &           &    3 &  1.71e-08 &  4.88e-16 &      \\ 
     &           &    4 &  6.06e-08 &  4.88e-16 &      \\ 
     &           &    5 &  1.72e-08 &  1.73e-15 &      \\ 
     &           &    6 &  1.71e-08 &  4.89e-16 &      \\ 
     &           &    7 &  1.71e-08 &  4.89e-16 &      \\ 
     &           &    8 &  1.71e-08 &  4.88e-16 &      \\ 
     &           &    9 &  6.06e-08 &  4.87e-16 &      \\ 
     &           &   10 &  1.71e-08 &  1.73e-15 &      \\ 
 793 &  7.92e+03 &   10 &           &           & iters  \\ 
 \hdashline 
     &           &    1 &  6.47e-09 &  2.55e-08 &      \\ 
     &           &    2 &  6.46e-09 &  1.85e-16 &      \\ 
     &           &    3 &  6.45e-09 &  1.84e-16 &      \\ 
     &           &    4 &  7.12e-08 &  1.84e-16 &      \\ 
     &           &    5 &  4.54e-08 &  2.03e-15 &      \\ 
     &           &    6 &  6.48e-09 &  1.30e-15 &      \\ 
     &           &    7 &  6.47e-09 &  1.85e-16 &      \\ 
     &           &    8 &  6.46e-09 &  1.85e-16 &      \\ 
     &           &    9 &  6.45e-09 &  1.84e-16 &      \\ 
     &           &   10 &  7.12e-08 &  1.84e-16 &      \\ 
 794 &  7.93e+03 &   10 &           &           & iters  \\ 
 \hdashline 
     &           &    1 &  6.10e-08 &  2.47e-09 &      \\ 
     &           &    2 &  1.68e-08 &  1.74e-15 &      \\ 
     &           &    3 &  1.68e-08 &  4.79e-16 &      \\ 
     &           &    4 &  1.68e-08 &  4.79e-16 &      \\ 
     &           &    5 &  1.67e-08 &  4.78e-16 &      \\ 
     &           &    6 &  6.10e-08 &  4.77e-16 &      \\ 
     &           &    7 &  1.68e-08 &  1.74e-15 &      \\ 
     &           &    8 &  1.68e-08 &  4.79e-16 &      \\ 
     &           &    9 &  1.67e-08 &  4.78e-16 &      \\ 
     &           &   10 &  6.10e-08 &  4.78e-16 &      \\ 
 795 &  7.94e+03 &   10 &           &           & iters  \\ 
 \hdashline 
     &           &    1 &  4.49e-08 &  3.56e-08 &      \\ 
     &           &    2 &  3.28e-08 &  1.28e-15 &      \\ 
     &           &    3 &  4.49e-08 &  9.36e-16 &      \\ 
     &           &    4 &  3.28e-08 &  1.28e-15 &      \\ 
     &           &    5 &  3.28e-08 &  9.37e-16 &      \\ 
     &           &    6 &  4.50e-08 &  9.35e-16 &      \\ 
     &           &    7 &  3.28e-08 &  1.28e-15 &      \\ 
     &           &    8 &  4.49e-08 &  9.36e-16 &      \\ 
     &           &    9 &  3.28e-08 &  1.28e-15 &      \\ 
     &           &   10 &  4.49e-08 &  9.36e-16 &      \\ 
 796 &  7.95e+03 &   10 &           &           & iters  \\ 
 \hdashline 
     &           &    1 &  3.12e-08 &  3.59e-08 &      \\ 
     &           &    2 &  3.11e-08 &  8.90e-16 &      \\ 
     &           &    3 &  4.66e-08 &  8.89e-16 &      \\ 
     &           &    4 &  3.12e-08 &  1.33e-15 &      \\ 
     &           &    5 &  4.66e-08 &  8.89e-16 &      \\ 
     &           &    6 &  3.12e-08 &  1.33e-15 &      \\ 
     &           &    7 &  3.11e-08 &  8.90e-16 &      \\ 
     &           &    8 &  4.66e-08 &  8.89e-16 &      \\ 
     &           &    9 &  3.12e-08 &  1.33e-15 &      \\ 
     &           &   10 &  4.66e-08 &  8.89e-16 &      \\ 
 797 &  7.96e+03 &   10 &           &           & iters  \\ 
 \hdashline 
     &           &    1 &  4.99e-08 &  3.69e-08 &      \\ 
     &           &    2 &  2.79e-08 &  1.42e-15 &      \\ 
     &           &    3 &  4.99e-08 &  7.95e-16 &      \\ 
     &           &    4 &  2.79e-08 &  1.42e-15 &      \\ 
     &           &    5 &  2.79e-08 &  7.96e-16 &      \\ 
     &           &    6 &  4.99e-08 &  7.95e-16 &      \\ 
     &           &    7 &  2.79e-08 &  1.42e-15 &      \\ 
     &           &    8 &  2.78e-08 &  7.96e-16 &      \\ 
     &           &    9 &  4.99e-08 &  7.95e-16 &      \\ 
     &           &   10 &  2.79e-08 &  1.42e-15 &      \\ 
 798 &  7.97e+03 &   10 &           &           & iters  \\ 
 \hdashline 
     &           &    1 &  4.92e-08 &  3.61e-08 &      \\ 
     &           &    2 &  2.85e-08 &  1.40e-15 &      \\ 
     &           &    3 &  4.92e-08 &  8.14e-16 &      \\ 
     &           &    4 &  2.86e-08 &  1.40e-15 &      \\ 
     &           &    5 &  4.92e-08 &  8.15e-16 &      \\ 
     &           &    6 &  2.86e-08 &  1.40e-15 &      \\ 
     &           &    7 &  2.86e-08 &  8.16e-16 &      \\ 
     &           &    8 &  4.92e-08 &  8.15e-16 &      \\ 
     &           &    9 &  2.86e-08 &  1.40e-15 &      \\ 
     &           &   10 &  2.85e-08 &  8.16e-16 &      \\ 
 799 &  7.98e+03 &   10 &           &           & iters  \\ 
 \hdashline 
     &           &    1 &  7.47e-08 &  2.45e-08 &      \\ 
     &           &    2 &  8.08e-08 &  2.13e-15 &      \\ 
     &           &    3 &  7.47e-08 &  2.31e-15 &      \\ 
     &           &    4 &  8.08e-08 &  2.13e-15 &      \\ 
     &           &    5 &  7.47e-08 &  2.31e-15 &      \\ 
     &           &    6 &  3.09e-09 &  2.13e-15 &      \\ 
     &           &    7 &  3.08e-09 &  8.82e-17 &      \\ 
     &           &    8 &  3.08e-09 &  8.80e-17 &      \\ 
     &           &    9 &  3.08e-09 &  8.79e-17 &      \\ 
     &           &   10 &  3.07e-09 &  8.78e-17 &      \\ 
 800 &  7.99e+03 &   10 &           &           & iters  \\ 
 \hdashline 
     &           &    1 &  3.71e-08 &  7.99e-09 &      \\ 
     &           &    2 &  3.71e-08 &  1.06e-15 &      \\ 
     &           &    3 &  4.07e-08 &  1.06e-15 &      \\ 
     &           &    4 &  3.71e-08 &  1.16e-15 &      \\ 
     &           &    5 &  4.07e-08 &  1.06e-15 &      \\ 
     &           &    6 &  3.71e-08 &  1.16e-15 &      \\ 
     &           &    7 &  4.06e-08 &  1.06e-15 &      \\ 
     &           &    8 &  3.71e-08 &  1.16e-15 &      \\ 
     &           &    9 &  4.06e-08 &  1.06e-15 &      \\ 
     &           &   10 &  3.71e-08 &  1.16e-15 &      \\ 
 801 &  8.00e+03 &   10 &           &           & iters  \\ 
 \hdashline 
     &           &    1 &  2.91e-08 &  4.19e-08 &      \\ 
     &           &    2 &  2.91e-08 &  8.31e-16 &      \\ 
     &           &    3 &  4.87e-08 &  8.29e-16 &      \\ 
     &           &    4 &  2.91e-08 &  1.39e-15 &      \\ 
     &           &    5 &  4.86e-08 &  8.30e-16 &      \\ 
     &           &    6 &  2.91e-08 &  1.39e-15 &      \\ 
     &           &    7 &  2.91e-08 &  8.31e-16 &      \\ 
     &           &    8 &  4.87e-08 &  8.30e-16 &      \\ 
     &           &    9 &  2.91e-08 &  1.39e-15 &      \\ 
     &           &   10 &  2.91e-08 &  8.31e-16 &      \\ 
 802 &  8.01e+03 &   10 &           &           & iters  \\ 
 \hdashline 
     &           &    1 &  4.54e-08 &  5.45e-08 &      \\ 
     &           &    2 &  4.54e-08 &  1.30e-15 &      \\ 
     &           &    3 &  3.24e-08 &  1.29e-15 &      \\ 
     &           &    4 &  4.54e-08 &  9.24e-16 &      \\ 
     &           &    5 &  3.24e-08 &  1.29e-15 &      \\ 
     &           &    6 &  4.53e-08 &  9.25e-16 &      \\ 
     &           &    7 &  3.24e-08 &  1.29e-15 &      \\ 
     &           &    8 &  3.24e-08 &  9.25e-16 &      \\ 
     &           &    9 &  4.54e-08 &  9.24e-16 &      \\ 
     &           &   10 &  3.24e-08 &  1.29e-15 &      \\ 
 803 &  8.02e+03 &   10 &           &           & iters  \\ 
 \hdashline 
     &           &    1 &  3.34e-08 &  8.41e-09 &      \\ 
     &           &    2 &  4.43e-08 &  9.55e-16 &      \\ 
     &           &    3 &  3.35e-08 &  1.26e-15 &      \\ 
     &           &    4 &  4.43e-08 &  9.55e-16 &      \\ 
     &           &    5 &  3.35e-08 &  1.26e-15 &      \\ 
     &           &    6 &  3.34e-08 &  9.55e-16 &      \\ 
     &           &    7 &  4.43e-08 &  9.54e-16 &      \\ 
     &           &    8 &  3.34e-08 &  1.26e-15 &      \\ 
     &           &    9 &  4.43e-08 &  9.55e-16 &      \\ 
     &           &   10 &  3.35e-08 &  1.26e-15 &      \\ 
 804 &  8.03e+03 &   10 &           &           & iters  \\ 
 \hdashline 
     &           &    1 &  2.80e-08 &  1.74e-08 &      \\ 
     &           &    2 &  4.97e-08 &  7.99e-16 &      \\ 
     &           &    3 &  2.80e-08 &  1.42e-15 &      \\ 
     &           &    4 &  4.97e-08 &  8.00e-16 &      \\ 
     &           &    5 &  2.81e-08 &  1.42e-15 &      \\ 
     &           &    6 &  2.80e-08 &  8.01e-16 &      \\ 
     &           &    7 &  4.97e-08 &  8.00e-16 &      \\ 
     &           &    8 &  2.80e-08 &  1.42e-15 &      \\ 
     &           &    9 &  2.80e-08 &  8.00e-16 &      \\ 
     &           &   10 &  4.97e-08 &  7.99e-16 &      \\ 
 805 &  8.04e+03 &   10 &           &           & iters  \\ 
 \hdashline 
     &           &    1 &  5.10e-08 &  4.21e-08 &      \\ 
     &           &    2 &  5.09e-08 &  1.46e-15 &      \\ 
     &           &    3 &  2.68e-08 &  1.45e-15 &      \\ 
     &           &    4 &  2.68e-08 &  7.66e-16 &      \\ 
     &           &    5 &  5.09e-08 &  7.65e-16 &      \\ 
     &           &    6 &  2.68e-08 &  1.45e-15 &      \\ 
     &           &    7 &  2.68e-08 &  7.66e-16 &      \\ 
     &           &    8 &  5.09e-08 &  7.65e-16 &      \\ 
     &           &    9 &  2.68e-08 &  1.45e-15 &      \\ 
     &           &   10 &  5.09e-08 &  7.66e-16 &      \\ 
 806 &  8.05e+03 &   10 &           &           & iters  \\ 
 \hdashline 
     &           &    1 &  3.89e-08 &  6.52e-08 &      \\ 
     &           &    2 &  3.89e-08 &  1.11e-15 &      \\ 
     &           &    3 &  3.89e-08 &  1.11e-15 &      \\ 
     &           &    4 &  3.89e-08 &  1.11e-15 &      \\ 
     &           &    5 &  3.89e-08 &  1.11e-15 &      \\ 
     &           &    6 &  3.89e-08 &  1.11e-15 &      \\ 
     &           &    7 &  3.89e-08 &  1.11e-15 &      \\ 
     &           &    8 &  3.89e-08 &  1.11e-15 &      \\ 
     &           &    9 &  3.89e-08 &  1.11e-15 &      \\ 
     &           &   10 &  3.89e-08 &  1.11e-15 &      \\ 
 807 &  8.06e+03 &   10 &           &           & iters  \\ 
 \hdashline 
     &           &    1 &  3.44e-08 &  4.14e-08 &      \\ 
     &           &    2 &  4.33e-08 &  9.82e-16 &      \\ 
     &           &    3 &  3.44e-08 &  1.24e-15 &      \\ 
     &           &    4 &  3.44e-08 &  9.82e-16 &      \\ 
     &           &    5 &  4.34e-08 &  9.81e-16 &      \\ 
     &           &    6 &  3.44e-08 &  1.24e-15 &      \\ 
     &           &    7 &  4.34e-08 &  9.81e-16 &      \\ 
     &           &    8 &  3.44e-08 &  1.24e-15 &      \\ 
     &           &    9 &  4.33e-08 &  9.82e-16 &      \\ 
     &           &   10 &  3.44e-08 &  1.24e-15 &      \\ 
 808 &  8.07e+03 &   10 &           &           & iters  \\ 
 \hdashline 
     &           &    1 &  2.17e-08 &  1.37e-08 &      \\ 
     &           &    2 &  1.71e-08 &  6.20e-16 &      \\ 
     &           &    3 &  2.17e-08 &  4.89e-16 &      \\ 
     &           &    4 &  1.71e-08 &  6.20e-16 &      \\ 
     &           &    5 &  2.17e-08 &  4.89e-16 &      \\ 
     &           &    6 &  1.72e-08 &  6.20e-16 &      \\ 
     &           &    7 &  1.71e-08 &  4.90e-16 &      \\ 
     &           &    8 &  2.17e-08 &  4.89e-16 &      \\ 
     &           &    9 &  1.71e-08 &  6.20e-16 &      \\ 
     &           &   10 &  2.17e-08 &  4.89e-16 &      \\ 
 809 &  8.08e+03 &   10 &           &           & iters  \\ 
 \hdashline 
     &           &    1 &  2.19e-08 &  2.09e-08 &      \\ 
     &           &    2 &  1.70e-08 &  6.25e-16 &      \\ 
     &           &    3 &  1.70e-08 &  4.85e-16 &      \\ 
     &           &    4 &  2.19e-08 &  4.84e-16 &      \\ 
     &           &    5 &  1.70e-08 &  6.25e-16 &      \\ 
     &           &    6 &  2.19e-08 &  4.84e-16 &      \\ 
     &           &    7 &  1.70e-08 &  6.25e-16 &      \\ 
     &           &    8 &  2.19e-08 &  4.84e-16 &      \\ 
     &           &    9 &  1.70e-08 &  6.25e-16 &      \\ 
     &           &   10 &  2.19e-08 &  4.85e-16 &      \\ 
 810 &  8.09e+03 &   10 &           &           & iters  \\ 
 \hdashline 
     &           &    1 &  3.75e-08 &  3.05e-08 &      \\ 
     &           &    2 &  4.02e-08 &  1.07e-15 &      \\ 
     &           &    3 &  3.75e-08 &  1.15e-15 &      \\ 
     &           &    4 &  4.02e-08 &  1.07e-15 &      \\ 
     &           &    5 &  3.75e-08 &  1.15e-15 &      \\ 
     &           &    6 &  4.02e-08 &  1.07e-15 &      \\ 
     &           &    7 &  3.75e-08 &  1.15e-15 &      \\ 
     &           &    8 &  4.02e-08 &  1.07e-15 &      \\ 
     &           &    9 &  3.75e-08 &  1.15e-15 &      \\ 
     &           &   10 &  4.02e-08 &  1.07e-15 &      \\ 
 811 &  8.10e+03 &   10 &           &           & iters  \\ 
 \hdashline 
     &           &    1 &  5.47e-09 &  2.24e-08 &      \\ 
     &           &    2 &  5.47e-09 &  1.56e-16 &      \\ 
     &           &    3 &  5.46e-09 &  1.56e-16 &      \\ 
     &           &    4 &  7.22e-08 &  1.56e-16 &      \\ 
     &           &    5 &  4.44e-08 &  2.06e-15 &      \\ 
     &           &    6 &  5.49e-09 &  1.27e-15 &      \\ 
     &           &    7 &  5.48e-09 &  1.57e-16 &      \\ 
     &           &    8 &  5.48e-09 &  1.56e-16 &      \\ 
     &           &    9 &  5.47e-09 &  1.56e-16 &      \\ 
     &           &   10 &  5.46e-09 &  1.56e-16 &      \\ 
 812 &  8.11e+03 &   10 &           &           & iters  \\ 
 \hdashline 
     &           &    1 &  7.54e-08 &  4.60e-08 &      \\ 
     &           &    2 &  4.12e-08 &  2.15e-15 &      \\ 
     &           &    3 &  2.30e-09 &  1.18e-15 &      \\ 
     &           &    4 &  2.30e-09 &  6.57e-17 &      \\ 
     &           &    5 &  2.29e-09 &  6.56e-17 &      \\ 
     &           &    6 &  2.29e-09 &  6.55e-17 &      \\ 
     &           &    7 &  2.29e-09 &  6.54e-17 &      \\ 
     &           &    8 &  2.28e-09 &  6.53e-17 &      \\ 
     &           &    9 &  2.28e-09 &  6.52e-17 &      \\ 
     &           &   10 &  2.28e-09 &  6.51e-17 &      \\ 
 813 &  8.12e+03 &   10 &           &           & iters  \\ 
 \hdashline 
     &           &    1 &  4.32e-08 &  9.84e-09 &      \\ 
     &           &    2 &  4.30e-09 &  1.23e-15 &      \\ 
     &           &    3 &  4.29e-09 &  1.23e-16 &      \\ 
     &           &    4 &  4.28e-09 &  1.22e-16 &      \\ 
     &           &    5 &  4.28e-09 &  1.22e-16 &      \\ 
     &           &    6 &  4.27e-09 &  1.22e-16 &      \\ 
     &           &    7 &  4.27e-09 &  1.22e-16 &      \\ 
     &           &    8 &  4.26e-09 &  1.22e-16 &      \\ 
     &           &    9 &  4.25e-09 &  1.22e-16 &      \\ 
     &           &   10 &  7.34e-08 &  1.21e-16 &      \\ 
 814 &  8.13e+03 &   10 &           &           & iters  \\ 
 \hdashline 
     &           &    1 &  7.47e-08 &  9.89e-09 &      \\ 
     &           &    2 &  8.08e-08 &  2.13e-15 &      \\ 
     &           &    3 &  7.47e-08 &  2.31e-15 &      \\ 
     &           &    4 &  8.08e-08 &  2.13e-15 &      \\ 
     &           &    5 &  7.47e-08 &  2.31e-15 &      \\ 
     &           &    6 &  3.07e-09 &  2.13e-15 &      \\ 
     &           &    7 &  3.07e-09 &  8.77e-17 &      \\ 
     &           &    8 &  3.07e-09 &  8.76e-17 &      \\ 
     &           &    9 &  3.06e-09 &  8.75e-17 &      \\ 
     &           &   10 &  3.06e-09 &  8.74e-17 &      \\ 
 815 &  8.14e+03 &   10 &           &           & iters  \\ 
 \hdashline 
     &           &    1 &  9.21e-09 &  8.33e-09 &      \\ 
     &           &    2 &  9.19e-09 &  2.63e-16 &      \\ 
     &           &    3 &  9.18e-09 &  2.62e-16 &      \\ 
     &           &    4 &  6.85e-08 &  2.62e-16 &      \\ 
     &           &    5 &  4.81e-08 &  1.96e-15 &      \\ 
     &           &    6 &  9.20e-09 &  1.37e-15 &      \\ 
     &           &    7 &  9.18e-09 &  2.62e-16 &      \\ 
     &           &    8 &  9.17e-09 &  2.62e-16 &      \\ 
     &           &    9 &  6.85e-08 &  2.62e-16 &      \\ 
     &           &   10 &  9.26e-09 &  1.96e-15 &      \\ 
 816 &  8.15e+03 &   10 &           &           & iters  \\ 
 \hdashline 
     &           &    1 &  1.13e-09 &  3.96e-08 &      \\ 
     &           &    2 &  1.13e-09 &  3.24e-17 &      \\ 
     &           &    3 &  1.13e-09 &  3.23e-17 &      \\ 
     &           &    4 &  1.13e-09 &  3.23e-17 &      \\ 
     &           &    5 &  1.13e-09 &  3.23e-17 &      \\ 
     &           &    6 &  1.13e-09 &  3.22e-17 &      \\ 
     &           &    7 &  1.13e-09 &  3.22e-17 &      \\ 
     &           &    8 &  1.12e-09 &  3.21e-17 &      \\ 
     &           &    9 &  1.12e-09 &  3.21e-17 &      \\ 
     &           &   10 &  1.12e-09 &  3.20e-17 &      \\ 
 817 &  8.16e+03 &   10 &           &           & iters  \\ 
 \hdashline 
     &           &    1 &  2.99e-08 &  5.83e-08 &      \\ 
     &           &    2 &  2.99e-08 &  8.54e-16 &      \\ 
     &           &    3 &  4.79e-08 &  8.52e-16 &      \\ 
     &           &    4 &  2.99e-08 &  1.37e-15 &      \\ 
     &           &    5 &  2.99e-08 &  8.53e-16 &      \\ 
     &           &    6 &  4.79e-08 &  8.52e-16 &      \\ 
     &           &    7 &  2.99e-08 &  1.37e-15 &      \\ 
     &           &    8 &  2.98e-08 &  8.53e-16 &      \\ 
     &           &    9 &  4.79e-08 &  8.51e-16 &      \\ 
     &           &   10 &  2.99e-08 &  1.37e-15 &      \\ 
 818 &  8.17e+03 &   10 &           &           & iters  \\ 
 \hdashline 
     &           &    1 &  5.81e-09 &  5.18e-08 &      \\ 
     &           &    2 &  7.19e-08 &  1.66e-16 &      \\ 
     &           &    3 &  8.36e-08 &  2.05e-15 &      \\ 
     &           &    4 &  7.19e-08 &  2.39e-15 &      \\ 
     &           &    5 &  5.89e-09 &  2.05e-15 &      \\ 
     &           &    6 &  5.88e-09 &  1.68e-16 &      \\ 
     &           &    7 &  5.87e-09 &  1.68e-16 &      \\ 
     &           &    8 &  5.86e-09 &  1.68e-16 &      \\ 
     &           &    9 &  5.86e-09 &  1.67e-16 &      \\ 
     &           &   10 &  5.85e-09 &  1.67e-16 &      \\ 
 819 &  8.18e+03 &   10 &           &           & iters  \\ 
 \hdashline 
     &           &    1 &  1.30e-08 &  5.20e-09 &      \\ 
     &           &    2 &  1.29e-08 &  3.70e-16 &      \\ 
     &           &    3 &  1.29e-08 &  3.70e-16 &      \\ 
     &           &    4 &  6.48e-08 &  3.69e-16 &      \\ 
     &           &    5 &  1.30e-08 &  1.85e-15 &      \\ 
     &           &    6 &  1.30e-08 &  3.71e-16 &      \\ 
     &           &    7 &  1.30e-08 &  3.71e-16 &      \\ 
     &           &    8 &  1.29e-08 &  3.70e-16 &      \\ 
     &           &    9 &  1.29e-08 &  3.70e-16 &      \\ 
     &           &   10 &  6.48e-08 &  3.69e-16 &      \\ 
 820 &  8.19e+03 &   10 &           &           & iters  \\ 
 \hdashline 
     &           &    1 &  1.93e-08 &  5.49e-08 &      \\ 
     &           &    2 &  5.84e-08 &  5.51e-16 &      \\ 
     &           &    3 &  1.93e-08 &  1.67e-15 &      \\ 
     &           &    4 &  1.93e-08 &  5.52e-16 &      \\ 
     &           &    5 &  1.93e-08 &  5.51e-16 &      \\ 
     &           &    6 &  5.84e-08 &  5.51e-16 &      \\ 
     &           &    7 &  1.93e-08 &  1.67e-15 &      \\ 
     &           &    8 &  1.93e-08 &  5.52e-16 &      \\ 
     &           &    9 &  1.93e-08 &  5.51e-16 &      \\ 
     &           &   10 &  5.84e-08 &  5.51e-16 &      \\ 
 821 &  8.20e+03 &   10 &           &           & iters  \\ 
 \hdashline 
     &           &    1 &  1.86e-08 &  6.19e-08 &      \\ 
     &           &    2 &  5.91e-08 &  5.30e-16 &      \\ 
     &           &    3 &  1.86e-08 &  1.69e-15 &      \\ 
     &           &    4 &  1.86e-08 &  5.32e-16 &      \\ 
     &           &    5 &  1.86e-08 &  5.31e-16 &      \\ 
     &           &    6 &  5.91e-08 &  5.30e-16 &      \\ 
     &           &    7 &  1.86e-08 &  1.69e-15 &      \\ 
     &           &    8 &  1.86e-08 &  5.32e-16 &      \\ 
     &           &    9 &  1.86e-08 &  5.31e-16 &      \\ 
     &           &   10 &  5.91e-08 &  5.30e-16 &      \\ 
 822 &  8.21e+03 &   10 &           &           & iters  \\ 
 \hdashline 
     &           &    1 &  2.67e-09 &  5.42e-08 &      \\ 
     &           &    2 &  2.67e-09 &  7.63e-17 &      \\ 
     &           &    3 &  2.67e-09 &  7.62e-17 &      \\ 
     &           &    4 &  2.66e-09 &  7.61e-17 &      \\ 
     &           &    5 &  2.66e-09 &  7.60e-17 &      \\ 
     &           &    6 &  2.66e-09 &  7.59e-17 &      \\ 
     &           &    7 &  2.65e-09 &  7.58e-17 &      \\ 
     &           &    8 &  2.65e-09 &  7.57e-17 &      \\ 
     &           &    9 &  2.64e-09 &  7.56e-17 &      \\ 
     &           &   10 &  2.64e-09 &  7.55e-17 &      \\ 
 823 &  8.22e+03 &   10 &           &           & iters  \\ 
 \hdashline 
     &           &    1 &  4.88e-08 &  1.28e-08 &      \\ 
     &           &    2 &  2.90e-08 &  1.39e-15 &      \\ 
     &           &    3 &  2.90e-08 &  8.28e-16 &      \\ 
     &           &    4 &  4.88e-08 &  8.26e-16 &      \\ 
     &           &    5 &  2.90e-08 &  1.39e-15 &      \\ 
     &           &    6 &  2.89e-08 &  8.27e-16 &      \\ 
     &           &    7 &  4.88e-08 &  8.26e-16 &      \\ 
     &           &    8 &  2.90e-08 &  1.39e-15 &      \\ 
     &           &    9 &  4.88e-08 &  8.27e-16 &      \\ 
     &           &   10 &  2.90e-08 &  1.39e-15 &      \\ 
 824 &  8.23e+03 &   10 &           &           & iters  \\ 
 \hdashline 
     &           &    1 &  4.14e-08 &  9.42e-09 &      \\ 
     &           &    2 &  3.64e-08 &  1.18e-15 &      \\ 
     &           &    3 &  4.14e-08 &  1.04e-15 &      \\ 
     &           &    4 &  3.64e-08 &  1.18e-15 &      \\ 
     &           &    5 &  4.14e-08 &  1.04e-15 &      \\ 
     &           &    6 &  3.64e-08 &  1.18e-15 &      \\ 
     &           &    7 &  4.14e-08 &  1.04e-15 &      \\ 
     &           &    8 &  3.64e-08 &  1.18e-15 &      \\ 
     &           &    9 &  4.14e-08 &  1.04e-15 &      \\ 
     &           &   10 &  3.64e-08 &  1.18e-15 &      \\ 
 825 &  8.24e+03 &   10 &           &           & iters  \\ 
 \hdashline 
     &           &    1 &  1.62e-08 &  4.78e-09 &      \\ 
     &           &    2 &  2.26e-08 &  4.63e-16 &      \\ 
     &           &    3 &  1.62e-08 &  6.46e-16 &      \\ 
     &           &    4 &  2.26e-08 &  4.63e-16 &      \\ 
     &           &    5 &  1.62e-08 &  6.46e-16 &      \\ 
     &           &    6 &  1.62e-08 &  4.64e-16 &      \\ 
     &           &    7 &  2.26e-08 &  4.63e-16 &      \\ 
     &           &    8 &  1.62e-08 &  6.46e-16 &      \\ 
     &           &    9 &  2.26e-08 &  4.63e-16 &      \\ 
     &           &   10 &  1.62e-08 &  6.46e-16 &      \\ 
 826 &  8.25e+03 &   10 &           &           & iters  \\ 
 \hdashline 
     &           &    1 &  2.25e-09 &  9.11e-09 &      \\ 
     &           &    2 &  2.24e-09 &  6.41e-17 &      \\ 
     &           &    3 &  2.24e-09 &  6.40e-17 &      \\ 
     &           &    4 &  2.24e-09 &  6.40e-17 &      \\ 
     &           &    5 &  2.23e-09 &  6.39e-17 &      \\ 
     &           &    6 &  2.23e-09 &  6.38e-17 &      \\ 
     &           &    7 &  2.23e-09 &  6.37e-17 &      \\ 
     &           &    8 &  2.22e-09 &  6.36e-17 &      \\ 
     &           &    9 &  2.22e-09 &  6.35e-17 &      \\ 
     &           &   10 &  2.22e-09 &  6.34e-17 &      \\ 
 827 &  8.26e+03 &   10 &           &           & iters  \\ 
 \hdashline 
     &           &    1 &  1.52e-08 &  7.66e-09 &      \\ 
     &           &    2 &  1.52e-08 &  4.33e-16 &      \\ 
     &           &    3 &  1.51e-08 &  4.33e-16 &      \\ 
     &           &    4 &  1.51e-08 &  4.32e-16 &      \\ 
     &           &    5 &  6.26e-08 &  4.32e-16 &      \\ 
     &           &    6 &  1.52e-08 &  1.79e-15 &      \\ 
     &           &    7 &  1.52e-08 &  4.34e-16 &      \\ 
     &           &    8 &  1.51e-08 &  4.33e-16 &      \\ 
     &           &    9 &  1.51e-08 &  4.32e-16 &      \\ 
     &           &   10 &  6.26e-08 &  4.32e-16 &      \\ 
 828 &  8.27e+03 &   10 &           &           & iters  \\ 
 \hdashline 
     &           &    1 &  3.47e-08 &  1.35e-08 &      \\ 
     &           &    2 &  4.24e-09 &  9.89e-16 &      \\ 
     &           &    3 &  4.23e-09 &  1.21e-16 &      \\ 
     &           &    4 &  4.23e-09 &  1.21e-16 &      \\ 
     &           &    5 &  4.22e-09 &  1.21e-16 &      \\ 
     &           &    6 &  4.21e-09 &  1.20e-16 &      \\ 
     &           &    7 &  4.21e-09 &  1.20e-16 &      \\ 
     &           &    8 &  4.20e-09 &  1.20e-16 &      \\ 
     &           &    9 &  3.46e-08 &  1.20e-16 &      \\ 
     &           &   10 &  4.24e-09 &  9.89e-16 &      \\ 
 829 &  8.28e+03 &   10 &           &           & iters  \\ 
 \hdashline 
     &           &    1 &  3.65e-08 &  1.65e-08 &      \\ 
     &           &    2 &  4.13e-08 &  1.04e-15 &      \\ 
     &           &    3 &  3.65e-08 &  1.18e-15 &      \\ 
     &           &    4 &  4.13e-08 &  1.04e-15 &      \\ 
     &           &    5 &  3.65e-08 &  1.18e-15 &      \\ 
     &           &    6 &  4.13e-08 &  1.04e-15 &      \\ 
     &           &    7 &  3.65e-08 &  1.18e-15 &      \\ 
     &           &    8 &  3.64e-08 &  1.04e-15 &      \\ 
     &           &    9 &  4.13e-08 &  1.04e-15 &      \\ 
     &           &   10 &  3.64e-08 &  1.18e-15 &      \\ 
 830 &  8.29e+03 &   10 &           &           & iters  \\ 
 \hdashline 
     &           &    1 &  8.82e-08 &  1.70e-08 &      \\ 
     &           &    2 &  6.73e-08 &  2.52e-15 &      \\ 
     &           &    3 &  1.05e-08 &  1.92e-15 &      \\ 
     &           &    4 &  1.05e-08 &  2.99e-16 &      \\ 
     &           &    5 &  1.05e-08 &  2.99e-16 &      \\ 
     &           &    6 &  1.04e-08 &  2.98e-16 &      \\ 
     &           &    7 &  1.04e-08 &  2.98e-16 &      \\ 
     &           &    8 &  1.04e-08 &  2.98e-16 &      \\ 
     &           &    9 &  6.73e-08 &  2.97e-16 &      \\ 
     &           &   10 &  1.05e-08 &  1.92e-15 &      \\ 
 831 &  8.30e+03 &   10 &           &           & iters  \\ 
 \hdashline 
     &           &    1 &  1.59e-08 &  3.21e-08 &      \\ 
     &           &    2 &  1.59e-08 &  4.55e-16 &      \\ 
     &           &    3 &  6.18e-08 &  4.54e-16 &      \\ 
     &           &    4 &  1.60e-08 &  1.76e-15 &      \\ 
     &           &    5 &  1.60e-08 &  4.56e-16 &      \\ 
     &           &    6 &  1.59e-08 &  4.56e-16 &      \\ 
     &           &    7 &  6.18e-08 &  4.55e-16 &      \\ 
     &           &    8 &  1.60e-08 &  1.76e-15 &      \\ 
     &           &    9 &  1.60e-08 &  4.57e-16 &      \\ 
     &           &   10 &  1.60e-08 &  4.56e-16 &      \\ 
 832 &  8.31e+03 &   10 &           &           & iters  \\ 
 \hdashline 
     &           &    1 &  1.32e-08 &  1.39e-08 &      \\ 
     &           &    2 &  1.32e-08 &  3.77e-16 &      \\ 
     &           &    3 &  1.32e-08 &  3.76e-16 &      \\ 
     &           &    4 &  6.45e-08 &  3.76e-16 &      \\ 
     &           &    5 &  1.32e-08 &  1.84e-15 &      \\ 
     &           &    6 &  1.32e-08 &  3.78e-16 &      \\ 
     &           &    7 &  1.32e-08 &  3.77e-16 &      \\ 
     &           &    8 &  1.32e-08 &  3.77e-16 &      \\ 
     &           &    9 &  1.32e-08 &  3.76e-16 &      \\ 
     &           &   10 &  6.45e-08 &  3.76e-16 &      \\ 
 833 &  8.32e+03 &   10 &           &           & iters  \\ 
 \hdashline 
     &           &    1 &  2.24e-08 &  3.55e-08 &      \\ 
     &           &    2 &  5.54e-08 &  6.38e-16 &      \\ 
     &           &    3 &  2.24e-08 &  1.58e-15 &      \\ 
     &           &    4 &  2.24e-08 &  6.40e-16 &      \\ 
     &           &    5 &  2.23e-08 &  6.39e-16 &      \\ 
     &           &    6 &  5.54e-08 &  6.38e-16 &      \\ 
     &           &    7 &  2.24e-08 &  1.58e-15 &      \\ 
     &           &    8 &  2.24e-08 &  6.39e-16 &      \\ 
     &           &    9 &  5.54e-08 &  6.38e-16 &      \\ 
     &           &   10 &  2.24e-08 &  1.58e-15 &      \\ 
 834 &  8.33e+03 &   10 &           &           & iters  \\ 
 \hdashline 
     &           &    1 &  5.05e-08 &  2.64e-08 &      \\ 
     &           &    2 &  2.73e-08 &  1.44e-15 &      \\ 
     &           &    3 &  2.72e-08 &  7.78e-16 &      \\ 
     &           &    4 &  5.05e-08 &  7.77e-16 &      \\ 
     &           &    5 &  2.73e-08 &  1.44e-15 &      \\ 
     &           &    6 &  2.72e-08 &  7.78e-16 &      \\ 
     &           &    7 &  5.05e-08 &  7.77e-16 &      \\ 
     &           &    8 &  2.73e-08 &  1.44e-15 &      \\ 
     &           &    9 &  2.72e-08 &  7.78e-16 &      \\ 
     &           &   10 &  5.05e-08 &  7.77e-16 &      \\ 
 835 &  8.34e+03 &   10 &           &           & iters  \\ 
 \hdashline 
     &           &    1 &  7.90e-08 &  2.42e-08 &      \\ 
     &           &    2 &  7.64e-08 &  2.26e-15 &      \\ 
     &           &    3 &  7.90e-08 &  2.18e-15 &      \\ 
     &           &    4 &  7.65e-08 &  2.26e-15 &      \\ 
     &           &    5 &  1.35e-09 &  2.18e-15 &      \\ 
     &           &    6 &  1.34e-09 &  3.84e-17 &      \\ 
     &           &    7 &  1.34e-09 &  3.83e-17 &      \\ 
     &           &    8 &  1.34e-09 &  3.83e-17 &      \\ 
     &           &    9 &  1.34e-09 &  3.82e-17 &      \\ 
     &           &   10 &  1.34e-09 &  3.82e-17 &      \\ 
 836 &  8.35e+03 &   10 &           &           & iters  \\ 
 \hdashline 
     &           &    1 &  3.37e-08 &  1.30e-08 &      \\ 
     &           &    2 &  4.40e-08 &  9.63e-16 &      \\ 
     &           &    3 &  3.38e-08 &  1.26e-15 &      \\ 
     &           &    4 &  4.40e-08 &  9.64e-16 &      \\ 
     &           &    5 &  3.38e-08 &  1.26e-15 &      \\ 
     &           &    6 &  4.40e-08 &  9.64e-16 &      \\ 
     &           &    7 &  3.38e-08 &  1.25e-15 &      \\ 
     &           &    8 &  3.37e-08 &  9.64e-16 &      \\ 
     &           &    9 &  4.40e-08 &  9.63e-16 &      \\ 
     &           &   10 &  3.38e-08 &  1.26e-15 &      \\ 
 837 &  8.36e+03 &   10 &           &           & iters  \\ 
 \hdashline 
     &           &    1 &  1.92e-08 &  5.36e-08 &      \\ 
     &           &    2 &  1.92e-08 &  5.49e-16 &      \\ 
     &           &    3 &  1.97e-08 &  5.48e-16 &      \\ 
     &           &    4 &  1.92e-08 &  5.61e-16 &      \\ 
     &           &    5 &  1.97e-08 &  5.48e-16 &      \\ 
     &           &    6 &  1.92e-08 &  5.61e-16 &      \\ 
     &           &    7 &  1.97e-08 &  5.48e-16 &      \\ 
     &           &    8 &  1.92e-08 &  5.61e-16 &      \\ 
     &           &    9 &  1.97e-08 &  5.48e-16 &      \\ 
     &           &   10 &  1.92e-08 &  5.61e-16 &      \\ 
 838 &  8.37e+03 &   10 &           &           & iters  \\ 
 \hdashline 
     &           &    1 &  1.40e-08 &  6.74e-08 &      \\ 
     &           &    2 &  1.40e-08 &  3.99e-16 &      \\ 
     &           &    3 &  2.49e-08 &  3.98e-16 &      \\ 
     &           &    4 &  1.40e-08 &  7.11e-16 &      \\ 
     &           &    5 &  1.40e-08 &  3.99e-16 &      \\ 
     &           &    6 &  2.49e-08 &  3.98e-16 &      \\ 
     &           &    7 &  1.40e-08 &  7.11e-16 &      \\ 
     &           &    8 &  1.40e-08 &  3.99e-16 &      \\ 
     &           &    9 &  2.49e-08 &  3.98e-16 &      \\ 
     &           &   10 &  1.40e-08 &  7.11e-16 &      \\ 
 839 &  8.38e+03 &   10 &           &           & iters  \\ 
 \hdashline 
     &           &    1 &  2.48e-08 &  4.48e-08 &      \\ 
     &           &    2 &  2.48e-08 &  7.08e-16 &      \\ 
     &           &    3 &  5.30e-08 &  7.07e-16 &      \\ 
     &           &    4 &  2.48e-08 &  1.51e-15 &      \\ 
     &           &    5 &  2.48e-08 &  7.08e-16 &      \\ 
     &           &    6 &  5.29e-08 &  7.07e-16 &      \\ 
     &           &    7 &  2.48e-08 &  1.51e-15 &      \\ 
     &           &    8 &  2.48e-08 &  7.08e-16 &      \\ 
     &           &    9 &  5.29e-08 &  7.07e-16 &      \\ 
     &           &   10 &  2.48e-08 &  1.51e-15 &      \\ 
 840 &  8.39e+03 &   10 &           &           & iters  \\ 
 \hdashline 
     &           &    1 &  2.63e-08 &  2.86e-08 &      \\ 
     &           &    2 &  5.14e-08 &  7.51e-16 &      \\ 
     &           &    3 &  2.63e-08 &  1.47e-15 &      \\ 
     &           &    4 &  2.63e-08 &  7.52e-16 &      \\ 
     &           &    5 &  5.14e-08 &  7.51e-16 &      \\ 
     &           &    6 &  2.63e-08 &  1.47e-15 &      \\ 
     &           &    7 &  2.63e-08 &  7.52e-16 &      \\ 
     &           &    8 &  5.14e-08 &  7.51e-16 &      \\ 
     &           &    9 &  2.63e-08 &  1.47e-15 &      \\ 
     &           &   10 &  2.63e-08 &  7.52e-16 &      \\ 
 841 &  8.40e+03 &   10 &           &           & iters  \\ 
 \hdashline 
     &           &    1 &  2.43e-08 &  1.90e-08 &      \\ 
     &           &    2 &  2.43e-08 &  6.95e-16 &      \\ 
     &           &    3 &  5.34e-08 &  6.94e-16 &      \\ 
     &           &    4 &  2.43e-08 &  1.52e-15 &      \\ 
     &           &    5 &  2.43e-08 &  6.95e-16 &      \\ 
     &           &    6 &  5.34e-08 &  6.94e-16 &      \\ 
     &           &    7 &  2.43e-08 &  1.52e-15 &      \\ 
     &           &    8 &  2.43e-08 &  6.95e-16 &      \\ 
     &           &    9 &  2.43e-08 &  6.94e-16 &      \\ 
     &           &   10 &  5.34e-08 &  6.93e-16 &      \\ 
 842 &  8.41e+03 &   10 &           &           & iters  \\ 
 \hdashline 
     &           &    1 &  6.90e-08 &  8.04e-08 &      \\ 
     &           &    2 &  3.00e-08 &  1.97e-15 &      \\ 
     &           &    3 &  4.77e-08 &  8.56e-16 &      \\ 
     &           &    4 &  3.00e-08 &  1.36e-15 &      \\ 
     &           &    5 &  4.77e-08 &  8.57e-16 &      \\ 
     &           &    6 &  3.01e-08 &  1.36e-15 &      \\ 
     &           &    7 &  3.00e-08 &  8.58e-16 &      \\ 
     &           &    8 &  4.77e-08 &  8.57e-16 &      \\ 
     &           &    9 &  3.00e-08 &  1.36e-15 &      \\ 
     &           &   10 &  4.77e-08 &  8.57e-16 &      \\ 
 843 &  8.42e+03 &   10 &           &           & iters  \\ 
 \hdashline 
     &           &    1 &  5.53e-08 &  9.94e-08 &      \\ 
     &           &    2 &  2.24e-08 &  1.58e-15 &      \\ 
     &           &    3 &  2.24e-08 &  6.40e-16 &      \\ 
     &           &    4 &  2.24e-08 &  6.39e-16 &      \\ 
     &           &    5 &  5.54e-08 &  6.38e-16 &      \\ 
     &           &    6 &  2.24e-08 &  1.58e-15 &      \\ 
     &           &    7 &  2.24e-08 &  6.39e-16 &      \\ 
     &           &    8 &  5.54e-08 &  6.38e-16 &      \\ 
     &           &    9 &  2.24e-08 &  1.58e-15 &      \\ 
     &           &   10 &  2.24e-08 &  6.40e-16 &      \\ 
 844 &  8.43e+03 &   10 &           &           & iters  \\ 
 \hdashline 
     &           &    1 &  3.90e-08 &  6.90e-08 &      \\ 
     &           &    2 &  3.88e-08 &  1.11e-15 &      \\ 
     &           &    3 &  3.90e-08 &  1.11e-15 &      \\ 
     &           &    4 &  3.88e-08 &  1.11e-15 &      \\ 
     &           &    5 &  3.90e-08 &  1.11e-15 &      \\ 
     &           &    6 &  3.88e-08 &  1.11e-15 &      \\ 
     &           &    7 &  3.90e-08 &  1.11e-15 &      \\ 
     &           &    8 &  3.88e-08 &  1.11e-15 &      \\ 
     &           &    9 &  3.90e-08 &  1.11e-15 &      \\ 
     &           &   10 &  3.88e-08 &  1.11e-15 &      \\ 
 845 &  8.44e+03 &   10 &           &           & iters  \\ 
 \hdashline 
     &           &    1 &  3.31e-08 &  4.39e-08 &      \\ 
     &           &    2 &  4.47e-08 &  9.44e-16 &      \\ 
     &           &    3 &  3.31e-08 &  1.27e-15 &      \\ 
     &           &    4 &  4.46e-08 &  9.45e-16 &      \\ 
     &           &    5 &  3.31e-08 &  1.27e-15 &      \\ 
     &           &    6 &  4.46e-08 &  9.45e-16 &      \\ 
     &           &    7 &  3.31e-08 &  1.27e-15 &      \\ 
     &           &    8 &  3.31e-08 &  9.46e-16 &      \\ 
     &           &    9 &  4.47e-08 &  9.44e-16 &      \\ 
     &           &   10 &  3.31e-08 &  1.27e-15 &      \\ 
 846 &  8.45e+03 &   10 &           &           & iters  \\ 
 \hdashline 
     &           &    1 &  2.92e-08 &  7.62e-09 &      \\ 
     &           &    2 &  4.85e-08 &  8.34e-16 &      \\ 
     &           &    3 &  2.92e-08 &  1.38e-15 &      \\ 
     &           &    4 &  2.92e-08 &  8.34e-16 &      \\ 
     &           &    5 &  4.85e-08 &  8.33e-16 &      \\ 
     &           &    6 &  2.92e-08 &  1.39e-15 &      \\ 
     &           &    7 &  4.85e-08 &  8.34e-16 &      \\ 
     &           &    8 &  2.93e-08 &  1.38e-15 &      \\ 
     &           &    9 &  2.92e-08 &  8.35e-16 &      \\ 
     &           &   10 &  4.85e-08 &  8.34e-16 &      \\ 
 847 &  8.46e+03 &   10 &           &           & iters  \\ 
 \hdashline 
     &           &    1 &  1.27e-08 &  6.72e-08 &      \\ 
     &           &    2 &  1.27e-08 &  3.62e-16 &      \\ 
     &           &    3 &  1.26e-08 &  3.62e-16 &      \\ 
     &           &    4 &  6.51e-08 &  3.61e-16 &      \\ 
     &           &    5 &  1.27e-08 &  1.86e-15 &      \\ 
     &           &    6 &  1.27e-08 &  3.63e-16 &      \\ 
     &           &    7 &  1.27e-08 &  3.63e-16 &      \\ 
     &           &    8 &  1.27e-08 &  3.62e-16 &      \\ 
     &           &    9 &  1.27e-08 &  3.62e-16 &      \\ 
     &           &   10 &  1.26e-08 &  3.61e-16 &      \\ 
 848 &  8.47e+03 &   10 &           &           & iters  \\ 
 \hdashline 
     &           &    1 &  9.76e-08 &  1.14e-08 &      \\ 
     &           &    2 &  5.79e-08 &  2.79e-15 &      \\ 
     &           &    3 &  1.98e-08 &  1.65e-15 &      \\ 
     &           &    4 &  1.98e-08 &  5.66e-16 &      \\ 
     &           &    5 &  5.79e-08 &  5.65e-16 &      \\ 
     &           &    6 &  1.99e-08 &  1.65e-15 &      \\ 
     &           &    7 &  1.98e-08 &  5.67e-16 &      \\ 
     &           &    8 &  1.98e-08 &  5.66e-16 &      \\ 
     &           &    9 &  5.79e-08 &  5.65e-16 &      \\ 
     &           &   10 &  1.99e-08 &  1.65e-15 &      \\ 
 849 &  8.48e+03 &   10 &           &           & iters  \\ 
 \hdashline 
     &           &    1 &  2.22e-08 &  6.90e-08 &      \\ 
     &           &    2 &  2.21e-08 &  6.32e-16 &      \\ 
     &           &    3 &  5.56e-08 &  6.32e-16 &      \\ 
     &           &    4 &  2.22e-08 &  1.59e-15 &      \\ 
     &           &    5 &  2.21e-08 &  6.33e-16 &      \\ 
     &           &    6 &  2.21e-08 &  6.32e-16 &      \\ 
     &           &    7 &  5.56e-08 &  6.31e-16 &      \\ 
     &           &    8 &  2.22e-08 &  1.59e-15 &      \\ 
     &           &    9 &  2.21e-08 &  6.32e-16 &      \\ 
     &           &   10 &  5.56e-08 &  6.32e-16 &      \\ 
 850 &  8.49e+03 &   10 &           &           & iters  \\ 
 \hdashline 
     &           &    1 &  5.20e-09 &  5.13e-08 &      \\ 
     &           &    2 &  5.19e-09 &  1.48e-16 &      \\ 
     &           &    3 &  5.18e-09 &  1.48e-16 &      \\ 
     &           &    4 &  5.18e-09 &  1.48e-16 &      \\ 
     &           &    5 &  5.17e-09 &  1.48e-16 &      \\ 
     &           &    6 &  5.16e-09 &  1.48e-16 &      \\ 
     &           &    7 &  5.15e-09 &  1.47e-16 &      \\ 
     &           &    8 &  5.15e-09 &  1.47e-16 &      \\ 
     &           &    9 &  7.25e-08 &  1.47e-16 &      \\ 
     &           &   10 &  5.24e-09 &  2.07e-15 &      \\ 
 851 &  8.50e+03 &   10 &           &           & iters  \\ 
 \hdashline 
     &           &    1 &  3.31e-08 &  3.30e-08 &      \\ 
     &           &    2 &  4.46e-08 &  9.46e-16 &      \\ 
     &           &    3 &  3.32e-08 &  1.27e-15 &      \\ 
     &           &    4 &  3.31e-08 &  9.46e-16 &      \\ 
     &           &    5 &  4.46e-08 &  9.45e-16 &      \\ 
     &           &    6 &  3.31e-08 &  1.27e-15 &      \\ 
     &           &    7 &  4.46e-08 &  9.45e-16 &      \\ 
     &           &    8 &  3.31e-08 &  1.27e-15 &      \\ 
     &           &    9 &  4.46e-08 &  9.46e-16 &      \\ 
     &           &   10 &  3.32e-08 &  1.27e-15 &      \\ 
 852 &  8.51e+03 &   10 &           &           & iters  \\ 
 \hdashline 
     &           &    1 &  6.14e-08 &  5.67e-08 &      \\ 
     &           &    2 &  1.64e-08 &  1.75e-15 &      \\ 
     &           &    3 &  1.63e-08 &  4.67e-16 &      \\ 
     &           &    4 &  6.14e-08 &  4.66e-16 &      \\ 
     &           &    5 &  1.64e-08 &  1.75e-15 &      \\ 
     &           &    6 &  1.64e-08 &  4.68e-16 &      \\ 
     &           &    7 &  1.63e-08 &  4.67e-16 &      \\ 
     &           &    8 &  1.63e-08 &  4.67e-16 &      \\ 
     &           &    9 &  6.14e-08 &  4.66e-16 &      \\ 
     &           &   10 &  1.64e-08 &  1.75e-15 &      \\ 
 853 &  8.52e+03 &   10 &           &           & iters  \\ 
 \hdashline 
     &           &    1 &  1.00e-08 &  2.61e-08 &      \\ 
     &           &    2 &  6.77e-08 &  2.86e-16 &      \\ 
     &           &    3 &  1.01e-08 &  1.93e-15 &      \\ 
     &           &    4 &  1.01e-08 &  2.88e-16 &      \\ 
     &           &    5 &  1.01e-08 &  2.88e-16 &      \\ 
     &           &    6 &  1.01e-08 &  2.87e-16 &      \\ 
     &           &    7 &  1.00e-08 &  2.87e-16 &      \\ 
     &           &    8 &  1.00e-08 &  2.87e-16 &      \\ 
     &           &    9 &  1.00e-08 &  2.86e-16 &      \\ 
     &           &   10 &  6.77e-08 &  2.86e-16 &      \\ 
 854 &  8.53e+03 &   10 &           &           & iters  \\ 
 \hdashline 
     &           &    1 &  4.33e-08 &  1.35e-07 &      \\ 
     &           &    2 &  4.32e-08 &  1.24e-15 &      \\ 
     &           &    3 &  3.45e-08 &  1.23e-15 &      \\ 
     &           &    4 &  3.45e-08 &  9.85e-16 &      \\ 
     &           &    5 &  4.33e-08 &  9.84e-16 &      \\ 
     &           &    6 &  3.45e-08 &  1.24e-15 &      \\ 
     &           &    7 &  4.33e-08 &  9.84e-16 &      \\ 
     &           &    8 &  3.45e-08 &  1.23e-15 &      \\ 
     &           &    9 &  4.33e-08 &  9.84e-16 &      \\ 
     &           &   10 &  3.45e-08 &  1.23e-15 &      \\ 
 855 &  8.54e+03 &   10 &           &           & iters  \\ 
 \hdashline 
     &           &    1 &  1.57e-08 &  9.50e-08 &      \\ 
     &           &    2 &  1.57e-08 &  4.48e-16 &      \\ 
     &           &    3 &  1.56e-08 &  4.47e-16 &      \\ 
     &           &    4 &  1.56e-08 &  4.46e-16 &      \\ 
     &           &    5 &  1.56e-08 &  4.46e-16 &      \\ 
     &           &    6 &  1.56e-08 &  4.45e-16 &      \\ 
     &           &    7 &  1.55e-08 &  4.44e-16 &      \\ 
     &           &    8 &  6.22e-08 &  4.44e-16 &      \\ 
     &           &    9 &  1.56e-08 &  1.77e-15 &      \\ 
     &           &   10 &  1.56e-08 &  4.46e-16 &      \\ 
 856 &  8.55e+03 &   10 &           &           & iters  \\ 
 \hdashline 
     &           &    1 &  2.95e-08 &  7.79e-08 &      \\ 
     &           &    2 &  4.82e-08 &  8.42e-16 &      \\ 
     &           &    3 &  2.95e-08 &  1.38e-15 &      \\ 
     &           &    4 &  2.95e-08 &  8.43e-16 &      \\ 
     &           &    5 &  4.82e-08 &  8.42e-16 &      \\ 
     &           &    6 &  2.95e-08 &  1.38e-15 &      \\ 
     &           &    7 &  2.95e-08 &  8.42e-16 &      \\ 
     &           &    8 &  4.83e-08 &  8.41e-16 &      \\ 
     &           &    9 &  2.95e-08 &  1.38e-15 &      \\ 
     &           &   10 &  4.82e-08 &  8.42e-16 &      \\ 
 857 &  8.56e+03 &   10 &           &           & iters  \\ 
 \hdashline 
     &           &    1 &  9.38e-08 &  5.70e-08 &      \\ 
     &           &    2 &  2.29e-08 &  2.68e-15 &      \\ 
     &           &    3 &  2.28e-08 &  6.53e-16 &      \\ 
     &           &    4 &  1.60e-08 &  6.52e-16 &      \\ 
     &           &    5 &  1.60e-08 &  4.58e-16 &      \\ 
     &           &    6 &  2.28e-08 &  4.57e-16 &      \\ 
     &           &    7 &  1.60e-08 &  6.52e-16 &      \\ 
     &           &    8 &  2.28e-08 &  4.58e-16 &      \\ 
     &           &    9 &  1.60e-08 &  6.52e-16 &      \\ 
     &           &   10 &  1.60e-08 &  4.58e-16 &      \\ 
 858 &  8.57e+03 &   10 &           &           & iters  \\ 
 \hdashline 
     &           &    1 &  5.30e-08 &  4.98e-08 &      \\ 
     &           &    2 &  1.41e-08 &  1.51e-15 &      \\ 
     &           &    3 &  6.36e-08 &  4.03e-16 &      \\ 
     &           &    4 &  1.42e-08 &  1.82e-15 &      \\ 
     &           &    5 &  1.42e-08 &  4.05e-16 &      \\ 
     &           &    6 &  1.41e-08 &  4.04e-16 &      \\ 
     &           &    7 &  1.41e-08 &  4.03e-16 &      \\ 
     &           &    8 &  1.41e-08 &  4.03e-16 &      \\ 
     &           &    9 &  6.36e-08 &  4.02e-16 &      \\ 
     &           &   10 &  1.42e-08 &  1.82e-15 &      \\ 
 859 &  8.58e+03 &   10 &           &           & iters  \\ 
 \hdashline 
     &           &    1 &  1.81e-08 &  3.53e-08 &      \\ 
     &           &    2 &  1.81e-08 &  5.17e-16 &      \\ 
     &           &    3 &  1.81e-08 &  5.16e-16 &      \\ 
     &           &    4 &  5.96e-08 &  5.16e-16 &      \\ 
     &           &    5 &  1.81e-08 &  1.70e-15 &      \\ 
     &           &    6 &  1.81e-08 &  5.17e-16 &      \\ 
     &           &    7 &  1.81e-08 &  5.17e-16 &      \\ 
     &           &    8 &  5.96e-08 &  5.16e-16 &      \\ 
     &           &    9 &  1.81e-08 &  1.70e-15 &      \\ 
     &           &   10 &  1.81e-08 &  5.18e-16 &      \\ 
 860 &  8.59e+03 &   10 &           &           & iters  \\ 
 \hdashline 
     &           &    1 &  3.90e-08 &  1.62e-08 &      \\ 
     &           &    2 &  3.88e-08 &  1.11e-15 &      \\ 
     &           &    3 &  3.90e-08 &  1.11e-15 &      \\ 
     &           &    4 &  3.88e-08 &  1.11e-15 &      \\ 
     &           &    5 &  3.90e-08 &  1.11e-15 &      \\ 
     &           &    6 &  3.88e-08 &  1.11e-15 &      \\ 
     &           &    7 &  3.90e-08 &  1.11e-15 &      \\ 
     &           &    8 &  3.88e-08 &  1.11e-15 &      \\ 
     &           &    9 &  3.90e-08 &  1.11e-15 &      \\ 
     &           &   10 &  3.88e-08 &  1.11e-15 &      \\ 
 861 &  8.60e+03 &   10 &           &           & iters  \\ 
 \hdashline 
     &           &    1 &  1.19e-08 &  1.52e-08 &      \\ 
     &           &    2 &  1.19e-08 &  3.40e-16 &      \\ 
     &           &    3 &  1.19e-08 &  3.40e-16 &      \\ 
     &           &    4 &  1.19e-08 &  3.39e-16 &      \\ 
     &           &    5 &  6.58e-08 &  3.39e-16 &      \\ 
     &           &    6 &  1.20e-08 &  1.88e-15 &      \\ 
     &           &    7 &  1.19e-08 &  3.41e-16 &      \\ 
     &           &    8 &  1.19e-08 &  3.41e-16 &      \\ 
     &           &    9 &  1.19e-08 &  3.40e-16 &      \\ 
     &           &   10 &  1.19e-08 &  3.40e-16 &      \\ 
 862 &  8.61e+03 &   10 &           &           & iters  \\ 
 \hdashline 
     &           &    1 &  2.04e-08 &  2.71e-08 &      \\ 
     &           &    2 &  2.03e-08 &  5.81e-16 &      \\ 
     &           &    3 &  2.03e-08 &  5.80e-16 &      \\ 
     &           &    4 &  1.86e-08 &  5.79e-16 &      \\ 
     &           &    5 &  1.86e-08 &  5.30e-16 &      \\ 
     &           &    6 &  2.03e-08 &  5.30e-16 &      \\ 
     &           &    7 &  1.86e-08 &  5.80e-16 &      \\ 
     &           &    8 &  2.03e-08 &  5.30e-16 &      \\ 
     &           &    9 &  1.86e-08 &  5.80e-16 &      \\ 
     &           &   10 &  2.03e-08 &  5.30e-16 &      \\ 
 863 &  8.62e+03 &   10 &           &           & iters  \\ 
 \hdashline 
     &           &    1 &  3.95e-08 &  1.23e-08 &      \\ 
     &           &    2 &  3.83e-08 &  1.13e-15 &      \\ 
     &           &    3 &  3.82e-08 &  1.09e-15 &      \\ 
     &           &    4 &  3.95e-08 &  1.09e-15 &      \\ 
     &           &    5 &  3.82e-08 &  1.13e-15 &      \\ 
     &           &    6 &  3.95e-08 &  1.09e-15 &      \\ 
     &           &    7 &  3.82e-08 &  1.13e-15 &      \\ 
     &           &    8 &  3.95e-08 &  1.09e-15 &      \\ 
     &           &    9 &  3.82e-08 &  1.13e-15 &      \\ 
     &           &   10 &  3.95e-08 &  1.09e-15 &      \\ 
 864 &  8.63e+03 &   10 &           &           & iters  \\ 
 \hdashline 
     &           &    1 &  2.42e-08 &  6.74e-09 &      \\ 
     &           &    2 &  5.36e-08 &  6.89e-16 &      \\ 
     &           &    3 &  2.42e-08 &  1.53e-15 &      \\ 
     &           &    4 &  2.42e-08 &  6.91e-16 &      \\ 
     &           &    5 &  5.36e-08 &  6.90e-16 &      \\ 
     &           &    6 &  2.42e-08 &  1.53e-15 &      \\ 
     &           &    7 &  2.42e-08 &  6.91e-16 &      \\ 
     &           &    8 &  5.35e-08 &  6.90e-16 &      \\ 
     &           &    9 &  2.42e-08 &  1.53e-15 &      \\ 
     &           &   10 &  2.42e-08 &  6.91e-16 &      \\ 
 865 &  8.64e+03 &   10 &           &           & iters  \\ 
 \hdashline 
     &           &    1 &  3.72e-09 &  4.04e-09 &      \\ 
     &           &    2 &  3.71e-09 &  1.06e-16 &      \\ 
     &           &    3 &  7.40e-08 &  1.06e-16 &      \\ 
     &           &    4 &  8.15e-08 &  2.11e-15 &      \\ 
     &           &    5 &  7.40e-08 &  2.33e-15 &      \\ 
     &           &    6 &  8.15e-08 &  2.11e-15 &      \\ 
     &           &    7 &  7.40e-08 &  2.33e-15 &      \\ 
     &           &    8 &  3.79e-09 &  2.11e-15 &      \\ 
     &           &    9 &  3.79e-09 &  1.08e-16 &      \\ 
     &           &   10 &  3.78e-09 &  1.08e-16 &      \\ 
 866 &  8.65e+03 &   10 &           &           & iters  \\ 
 \hdashline 
     &           &    1 &  4.06e-08 &  2.44e-09 &      \\ 
     &           &    2 &  4.05e-08 &  1.16e-15 &      \\ 
     &           &    3 &  3.72e-08 &  1.16e-15 &      \\ 
     &           &    4 &  4.05e-08 &  1.06e-15 &      \\ 
     &           &    5 &  3.72e-08 &  1.16e-15 &      \\ 
     &           &    6 &  4.05e-08 &  1.06e-15 &      \\ 
     &           &    7 &  3.72e-08 &  1.16e-15 &      \\ 
     &           &    8 &  4.05e-08 &  1.06e-15 &      \\ 
     &           &    9 &  3.72e-08 &  1.16e-15 &      \\ 
     &           &   10 &  4.05e-08 &  1.06e-15 &      \\ 
 867 &  8.66e+03 &   10 &           &           & iters  \\ 
 \hdashline 
     &           &    1 &  4.47e-08 &  8.28e-09 &      \\ 
     &           &    2 &  4.46e-08 &  1.28e-15 &      \\ 
     &           &    3 &  3.31e-08 &  1.27e-15 &      \\ 
     &           &    4 &  4.46e-08 &  9.45e-16 &      \\ 
     &           &    5 &  3.31e-08 &  1.27e-15 &      \\ 
     &           &    6 &  3.31e-08 &  9.46e-16 &      \\ 
     &           &    7 &  4.46e-08 &  9.44e-16 &      \\ 
     &           &    8 &  3.31e-08 &  1.27e-15 &      \\ 
     &           &    9 &  4.46e-08 &  9.45e-16 &      \\ 
     &           &   10 &  3.31e-08 &  1.27e-15 &      \\ 
 868 &  8.67e+03 &   10 &           &           & iters  \\ 
 \hdashline 
     &           &    1 &  2.44e-08 &  4.05e-09 &      \\ 
     &           &    2 &  5.33e-08 &  6.97e-16 &      \\ 
     &           &    3 &  2.44e-08 &  1.52e-15 &      \\ 
     &           &    4 &  2.44e-08 &  6.98e-16 &      \\ 
     &           &    5 &  5.33e-08 &  6.97e-16 &      \\ 
     &           &    6 &  2.45e-08 &  1.52e-15 &      \\ 
     &           &    7 &  2.44e-08 &  6.98e-16 &      \\ 
     &           &    8 &  5.33e-08 &  6.97e-16 &      \\ 
     &           &    9 &  2.45e-08 &  1.52e-15 &      \\ 
     &           &   10 &  2.44e-08 &  6.98e-16 &      \\ 
 869 &  8.68e+03 &   10 &           &           & iters  \\ 
 \hdashline 
     &           &    1 &  5.70e-08 &  2.48e-08 &      \\ 
     &           &    2 &  2.08e-08 &  1.63e-15 &      \\ 
     &           &    3 &  2.08e-08 &  5.94e-16 &      \\ 
     &           &    4 &  5.69e-08 &  5.93e-16 &      \\ 
     &           &    5 &  2.08e-08 &  1.63e-15 &      \\ 
     &           &    6 &  2.08e-08 &  5.95e-16 &      \\ 
     &           &    7 &  2.08e-08 &  5.94e-16 &      \\ 
     &           &    8 &  5.69e-08 &  5.93e-16 &      \\ 
     &           &    9 &  2.08e-08 &  1.63e-15 &      \\ 
     &           &   10 &  2.08e-08 &  5.94e-16 &      \\ 
 870 &  8.69e+03 &   10 &           &           & iters  \\ 
 \hdashline 
     &           &    1 &  5.09e-08 &  3.28e-08 &      \\ 
     &           &    2 &  2.69e-08 &  1.45e-15 &      \\ 
     &           &    3 &  2.69e-08 &  7.68e-16 &      \\ 
     &           &    4 &  5.09e-08 &  7.66e-16 &      \\ 
     &           &    5 &  2.69e-08 &  1.45e-15 &      \\ 
     &           &    6 &  2.69e-08 &  7.67e-16 &      \\ 
     &           &    7 &  5.09e-08 &  7.66e-16 &      \\ 
     &           &    8 &  2.69e-08 &  1.45e-15 &      \\ 
     &           &    9 &  2.68e-08 &  7.67e-16 &      \\ 
     &           &   10 &  5.09e-08 &  7.66e-16 &      \\ 
 871 &  8.70e+03 &   10 &           &           & iters  \\ 
 \hdashline 
     &           &    1 &  5.54e-08 &  3.10e-09 &      \\ 
     &           &    2 &  2.24e-08 &  1.58e-15 &      \\ 
     &           &    3 &  2.24e-08 &  6.39e-16 &      \\ 
     &           &    4 &  2.23e-08 &  6.38e-16 &      \\ 
     &           &    5 &  5.54e-08 &  6.37e-16 &      \\ 
     &           &    6 &  2.24e-08 &  1.58e-15 &      \\ 
     &           &    7 &  2.23e-08 &  6.39e-16 &      \\ 
     &           &    8 &  5.54e-08 &  6.38e-16 &      \\ 
     &           &    9 &  2.24e-08 &  1.58e-15 &      \\ 
     &           &   10 &  2.24e-08 &  6.39e-16 &      \\ 
 872 &  8.71e+03 &   10 &           &           & iters  \\ 
 \hdashline 
     &           &    1 &  5.76e-08 &  2.70e-08 &      \\ 
     &           &    2 &  2.02e-08 &  1.64e-15 &      \\ 
     &           &    3 &  2.01e-08 &  5.75e-16 &      \\ 
     &           &    4 &  5.76e-08 &  5.74e-16 &      \\ 
     &           &    5 &  2.02e-08 &  1.64e-15 &      \\ 
     &           &    6 &  2.01e-08 &  5.76e-16 &      \\ 
     &           &    7 &  2.01e-08 &  5.75e-16 &      \\ 
     &           &    8 &  5.76e-08 &  5.74e-16 &      \\ 
     &           &    9 &  2.02e-08 &  1.64e-15 &      \\ 
     &           &   10 &  2.01e-08 &  5.76e-16 &      \\ 
 873 &  8.72e+03 &   10 &           &           & iters  \\ 
 \hdashline 
     &           &    1 &  4.87e-09 &  1.99e-08 &      \\ 
     &           &    2 &  4.87e-09 &  1.39e-16 &      \\ 
     &           &    3 &  4.86e-09 &  1.39e-16 &      \\ 
     &           &    4 &  4.85e-09 &  1.39e-16 &      \\ 
     &           &    5 &  4.85e-09 &  1.39e-16 &      \\ 
     &           &    6 &  4.84e-09 &  1.38e-16 &      \\ 
     &           &    7 &  4.83e-09 &  1.38e-16 &      \\ 
     &           &    8 &  7.29e-08 &  1.38e-16 &      \\ 
     &           &    9 &  8.26e-08 &  2.08e-15 &      \\ 
     &           &   10 &  7.29e-08 &  2.36e-15 &      \\ 
 874 &  8.73e+03 &   10 &           &           & iters  \\ 
 \hdashline 
     &           &    1 &  2.81e-09 &  2.46e-08 &      \\ 
     &           &    2 &  2.80e-09 &  8.01e-17 &      \\ 
     &           &    3 &  2.80e-09 &  8.00e-17 &      \\ 
     &           &    4 &  2.80e-09 &  7.99e-17 &      \\ 
     &           &    5 &  7.49e-08 &  7.98e-17 &      \\ 
     &           &    6 &  8.06e-08 &  2.14e-15 &      \\ 
     &           &    7 &  7.49e-08 &  2.30e-15 &      \\ 
     &           &    8 &  8.06e-08 &  2.14e-15 &      \\ 
     &           &    9 &  7.49e-08 &  2.30e-15 &      \\ 
     &           &   10 &  8.06e-08 &  2.14e-15 &      \\ 
 875 &  8.74e+03 &   10 &           &           & iters  \\ 
 \hdashline 
     &           &    1 &  7.61e-08 &  6.12e-08 &      \\ 
     &           &    2 &  4.06e-08 &  2.17e-15 &      \\ 
     &           &    3 &  1.66e-09 &  1.16e-15 &      \\ 
     &           &    4 &  1.65e-09 &  4.73e-17 &      \\ 
     &           &    5 &  1.65e-09 &  4.72e-17 &      \\ 
     &           &    6 &  1.65e-09 &  4.71e-17 &      \\ 
     &           &    7 &  1.65e-09 &  4.71e-17 &      \\ 
     &           &    8 &  1.64e-09 &  4.70e-17 &      \\ 
     &           &    9 &  1.64e-09 &  4.69e-17 &      \\ 
     &           &   10 &  1.64e-09 &  4.69e-17 &      \\ 
 876 &  8.75e+03 &   10 &           &           & iters  \\ 
 \hdashline 
     &           &    1 &  1.30e-08 &  2.54e-08 &      \\ 
     &           &    2 &  1.30e-08 &  3.72e-16 &      \\ 
     &           &    3 &  2.59e-08 &  3.71e-16 &      \\ 
     &           &    4 &  1.30e-08 &  7.38e-16 &      \\ 
     &           &    5 &  1.30e-08 &  3.72e-16 &      \\ 
     &           &    6 &  2.59e-08 &  3.71e-16 &      \\ 
     &           &    7 &  1.30e-08 &  7.38e-16 &      \\ 
     &           &    8 &  1.30e-08 &  3.72e-16 &      \\ 
     &           &    9 &  2.59e-08 &  3.71e-16 &      \\ 
     &           &   10 &  1.30e-08 &  7.38e-16 &      \\ 
 877 &  8.76e+03 &   10 &           &           & iters  \\ 
 \hdashline 
     &           &    1 &  5.54e-10 &  6.29e-09 &      \\ 
     &           &    2 &  5.53e-10 &  1.58e-17 &      \\ 
     &           &    3 &  5.52e-10 &  1.58e-17 &      \\ 
     &           &    4 &  5.51e-10 &  1.58e-17 &      \\ 
     &           &    5 &  5.51e-10 &  1.57e-17 &      \\ 
     &           &    6 &  5.50e-10 &  1.57e-17 &      \\ 
     &           &    7 &  5.49e-10 &  1.57e-17 &      \\ 
     &           &    8 &  5.48e-10 &  1.57e-17 &      \\ 
     &           &    9 &  5.48e-10 &  1.56e-17 &      \\ 
     &           &   10 &  5.47e-10 &  1.56e-17 &      \\ 
 878 &  8.77e+03 &   10 &           &           & iters  \\ 
 \hdashline 
     &           &    1 &  2.55e-08 &  1.54e-08 &      \\ 
     &           &    2 &  5.22e-08 &  7.28e-16 &      \\ 
     &           &    3 &  2.56e-08 &  1.49e-15 &      \\ 
     &           &    4 &  2.55e-08 &  7.30e-16 &      \\ 
     &           &    5 &  5.22e-08 &  7.28e-16 &      \\ 
     &           &    6 &  2.56e-08 &  1.49e-15 &      \\ 
     &           &    7 &  2.55e-08 &  7.30e-16 &      \\ 
     &           &    8 &  5.22e-08 &  7.29e-16 &      \\ 
     &           &    9 &  2.56e-08 &  1.49e-15 &      \\ 
     &           &   10 &  2.55e-08 &  7.30e-16 &      \\ 
 879 &  8.78e+03 &   10 &           &           & iters  \\ 
 \hdashline 
     &           &    1 &  5.46e-08 &  1.40e-08 &      \\ 
     &           &    2 &  2.32e-08 &  1.56e-15 &      \\ 
     &           &    3 &  2.31e-08 &  6.61e-16 &      \\ 
     &           &    4 &  5.46e-08 &  6.60e-16 &      \\ 
     &           &    5 &  2.32e-08 &  1.56e-15 &      \\ 
     &           &    6 &  2.31e-08 &  6.61e-16 &      \\ 
     &           &    7 &  2.31e-08 &  6.60e-16 &      \\ 
     &           &    8 &  5.46e-08 &  6.59e-16 &      \\ 
     &           &    9 &  2.31e-08 &  1.56e-15 &      \\ 
     &           &   10 &  2.31e-08 &  6.61e-16 &      \\ 
 880 &  8.79e+03 &   10 &           &           & iters  \\ 
 \hdashline 
     &           &    1 &  7.18e-09 &  3.78e-08 &      \\ 
     &           &    2 &  7.16e-09 &  2.05e-16 &      \\ 
     &           &    3 &  7.05e-08 &  2.04e-16 &      \\ 
     &           &    4 &  8.49e-08 &  2.01e-15 &      \\ 
     &           &    5 &  7.06e-08 &  2.42e-15 &      \\ 
     &           &    6 &  7.24e-09 &  2.01e-15 &      \\ 
     &           &    7 &  7.22e-09 &  2.06e-16 &      \\ 
     &           &    8 &  7.21e-09 &  2.06e-16 &      \\ 
     &           &    9 &  7.20e-09 &  2.06e-16 &      \\ 
     &           &   10 &  7.19e-09 &  2.06e-16 &      \\ 
 881 &  8.80e+03 &   10 &           &           & iters  \\ 
 \hdashline 
     &           &    1 &  1.32e-08 &  3.78e-08 &      \\ 
     &           &    2 &  2.57e-08 &  3.77e-16 &      \\ 
     &           &    3 &  1.32e-08 &  7.32e-16 &      \\ 
     &           &    4 &  1.32e-08 &  3.77e-16 &      \\ 
     &           &    5 &  2.57e-08 &  3.77e-16 &      \\ 
     &           &    6 &  1.32e-08 &  7.32e-16 &      \\ 
     &           &    7 &  1.32e-08 &  3.77e-16 &      \\ 
     &           &    8 &  2.57e-08 &  3.77e-16 &      \\ 
     &           &    9 &  1.32e-08 &  7.32e-16 &      \\ 
     &           &   10 &  1.32e-08 &  3.77e-16 &      \\ 
 882 &  8.81e+03 &   10 &           &           & iters  \\ 
 \hdashline 
     &           &    1 &  1.01e-08 &  5.13e-08 &      \\ 
     &           &    2 &  6.76e-08 &  2.87e-16 &      \\ 
     &           &    3 &  8.78e-08 &  1.93e-15 &      \\ 
     &           &    4 &  6.77e-08 &  2.51e-15 &      \\ 
     &           &    5 &  1.01e-08 &  1.93e-15 &      \\ 
     &           &    6 &  1.01e-08 &  2.89e-16 &      \\ 
     &           &    7 &  1.01e-08 &  2.88e-16 &      \\ 
     &           &    8 &  1.01e-08 &  2.88e-16 &      \\ 
     &           &    9 &  1.01e-08 &  2.88e-16 &      \\ 
     &           &   10 &  6.76e-08 &  2.87e-16 &      \\ 
 883 &  8.82e+03 &   10 &           &           & iters  \\ 
 \hdashline 
     &           &    1 &  3.58e-08 &  5.27e-08 &      \\ 
     &           &    2 &  3.58e-08 &  1.02e-15 &      \\ 
     &           &    3 &  4.20e-08 &  1.02e-15 &      \\ 
     &           &    4 &  3.58e-08 &  1.20e-15 &      \\ 
     &           &    5 &  4.19e-08 &  1.02e-15 &      \\ 
     &           &    6 &  3.58e-08 &  1.20e-15 &      \\ 
     &           &    7 &  3.58e-08 &  1.02e-15 &      \\ 
     &           &    8 &  4.20e-08 &  1.02e-15 &      \\ 
     &           &    9 &  3.58e-08 &  1.20e-15 &      \\ 
     &           &   10 &  4.20e-08 &  1.02e-15 &      \\ 
 884 &  8.83e+03 &   10 &           &           & iters  \\ 
 \hdashline 
     &           &    1 &  5.52e-09 &  1.06e-08 &      \\ 
     &           &    2 &  5.51e-09 &  1.57e-16 &      \\ 
     &           &    3 &  7.22e-08 &  1.57e-16 &      \\ 
     &           &    4 &  8.33e-08 &  2.06e-15 &      \\ 
     &           &    5 &  7.22e-08 &  2.38e-15 &      \\ 
     &           &    6 &  5.59e-09 &  2.06e-15 &      \\ 
     &           &    7 &  5.58e-09 &  1.59e-16 &      \\ 
     &           &    8 &  5.57e-09 &  1.59e-16 &      \\ 
     &           &    9 &  5.56e-09 &  1.59e-16 &      \\ 
     &           &   10 &  5.56e-09 &  1.59e-16 &      \\ 
 885 &  8.84e+03 &   10 &           &           & iters  \\ 
 \hdashline 
     &           &    1 &  3.52e-08 &  5.91e-08 &      \\ 
     &           &    2 &  4.25e-08 &  1.00e-15 &      \\ 
     &           &    3 &  3.52e-08 &  1.21e-15 &      \\ 
     &           &    4 &  4.25e-08 &  1.00e-15 &      \\ 
     &           &    5 &  3.52e-08 &  1.21e-15 &      \\ 
     &           &    6 &  4.25e-08 &  1.01e-15 &      \\ 
     &           &    7 &  3.52e-08 &  1.21e-15 &      \\ 
     &           &    8 &  4.25e-08 &  1.01e-15 &      \\ 
     &           &    9 &  3.52e-08 &  1.21e-15 &      \\ 
     &           &   10 &  3.52e-08 &  1.01e-15 &      \\ 
 886 &  8.85e+03 &   10 &           &           & iters  \\ 
 \hdashline 
     &           &    1 &  3.71e-08 &  1.31e-08 &      \\ 
     &           &    2 &  4.06e-08 &  1.06e-15 &      \\ 
     &           &    3 &  3.71e-08 &  1.16e-15 &      \\ 
     &           &    4 &  4.06e-08 &  1.06e-15 &      \\ 
     &           &    5 &  3.71e-08 &  1.16e-15 &      \\ 
     &           &    6 &  4.06e-08 &  1.06e-15 &      \\ 
     &           &    7 &  3.71e-08 &  1.16e-15 &      \\ 
     &           &    8 &  4.06e-08 &  1.06e-15 &      \\ 
     &           &    9 &  3.71e-08 &  1.16e-15 &      \\ 
     &           &   10 &  4.06e-08 &  1.06e-15 &      \\ 
 887 &  8.86e+03 &   10 &           &           & iters  \\ 
 \hdashline 
     &           &    1 &  1.76e-08 &  2.09e-08 &      \\ 
     &           &    2 &  6.01e-08 &  5.03e-16 &      \\ 
     &           &    3 &  1.77e-08 &  1.71e-15 &      \\ 
     &           &    4 &  1.77e-08 &  5.05e-16 &      \\ 
     &           &    5 &  1.76e-08 &  5.04e-16 &      \\ 
     &           &    6 &  6.01e-08 &  5.04e-16 &      \\ 
     &           &    7 &  1.77e-08 &  1.71e-15 &      \\ 
     &           &    8 &  1.77e-08 &  5.05e-16 &      \\ 
     &           &    9 &  1.77e-08 &  5.05e-16 &      \\ 
     &           &   10 &  6.01e-08 &  5.04e-16 &      \\ 
 888 &  8.87e+03 &   10 &           &           & iters  \\ 
 \hdashline 
     &           &    1 &  7.15e-08 &  4.25e-09 &      \\ 
     &           &    2 &  8.39e-08 &  2.04e-15 &      \\ 
     &           &    3 &  7.16e-08 &  2.40e-15 &      \\ 
     &           &    4 &  6.23e-09 &  2.04e-15 &      \\ 
     &           &    5 &  6.22e-09 &  1.78e-16 &      \\ 
     &           &    6 &  6.22e-09 &  1.78e-16 &      \\ 
     &           &    7 &  6.21e-09 &  1.77e-16 &      \\ 
     &           &    8 &  6.20e-09 &  1.77e-16 &      \\ 
     &           &    9 &  6.19e-09 &  1.77e-16 &      \\ 
     &           &   10 &  6.18e-09 &  1.77e-16 &      \\ 
 889 &  8.88e+03 &   10 &           &           & iters  \\ 
 \hdashline 
     &           &    1 &  1.83e-08 &  1.63e-08 &      \\ 
     &           &    2 &  1.82e-08 &  5.21e-16 &      \\ 
     &           &    3 &  1.82e-08 &  5.21e-16 &      \\ 
     &           &    4 &  5.95e-08 &  5.20e-16 &      \\ 
     &           &    5 &  1.83e-08 &  1.70e-15 &      \\ 
     &           &    6 &  1.82e-08 &  5.22e-16 &      \\ 
     &           &    7 &  1.82e-08 &  5.21e-16 &      \\ 
     &           &    8 &  5.95e-08 &  5.20e-16 &      \\ 
     &           &    9 &  1.83e-08 &  1.70e-15 &      \\ 
     &           &   10 &  1.83e-08 &  5.22e-16 &      \\ 
 890 &  8.89e+03 &   10 &           &           & iters  \\ 
 \hdashline 
     &           &    1 &  2.35e-08 &  1.95e-08 &      \\ 
     &           &    2 &  5.42e-08 &  6.71e-16 &      \\ 
     &           &    3 &  2.36e-08 &  1.55e-15 &      \\ 
     &           &    4 &  2.35e-08 &  6.72e-16 &      \\ 
     &           &    5 &  5.42e-08 &  6.71e-16 &      \\ 
     &           &    6 &  2.36e-08 &  1.55e-15 &      \\ 
     &           &    7 &  2.35e-08 &  6.73e-16 &      \\ 
     &           &    8 &  5.42e-08 &  6.72e-16 &      \\ 
     &           &    9 &  2.36e-08 &  1.55e-15 &      \\ 
     &           &   10 &  2.35e-08 &  6.73e-16 &      \\ 
 891 &  8.90e+03 &   10 &           &           & iters  \\ 
 \hdashline 
     &           &    1 &  5.38e-09 &  1.91e-08 &      \\ 
     &           &    2 &  5.37e-09 &  1.53e-16 &      \\ 
     &           &    3 &  5.36e-09 &  1.53e-16 &      \\ 
     &           &    4 &  5.35e-09 &  1.53e-16 &      \\ 
     &           &    5 &  5.35e-09 &  1.53e-16 &      \\ 
     &           &    6 &  5.34e-09 &  1.53e-16 &      \\ 
     &           &    7 &  5.33e-09 &  1.52e-16 &      \\ 
     &           &    8 &  5.32e-09 &  1.52e-16 &      \\ 
     &           &    9 &  5.32e-09 &  1.52e-16 &      \\ 
     &           &   10 &  5.31e-09 &  1.52e-16 &      \\ 
 892 &  8.91e+03 &   10 &           &           & iters  \\ 
 \hdashline 
     &           &    1 &  6.50e-08 &  2.09e-09 &      \\ 
     &           &    2 &  5.16e-08 &  1.86e-15 &      \\ 
     &           &    3 &  1.27e-08 &  1.47e-15 &      \\ 
     &           &    4 &  6.50e-08 &  3.62e-16 &      \\ 
     &           &    5 &  5.16e-08 &  1.86e-15 &      \\ 
     &           &    6 &  1.27e-08 &  1.47e-15 &      \\ 
     &           &    7 &  6.50e-08 &  3.62e-16 &      \\ 
     &           &    8 &  5.16e-08 &  1.86e-15 &      \\ 
     &           &    9 &  1.27e-08 &  1.47e-15 &      \\ 
     &           &   10 &  1.27e-08 &  3.62e-16 &      \\ 
 893 &  8.92e+03 &   10 &           &           & iters  \\ 
 \hdashline 
     &           &    1 &  4.23e-08 &  1.48e-08 &      \\ 
     &           &    2 &  3.54e-08 &  1.21e-15 &      \\ 
     &           &    3 &  4.23e-08 &  1.01e-15 &      \\ 
     &           &    4 &  3.54e-08 &  1.21e-15 &      \\ 
     &           &    5 &  4.23e-08 &  1.01e-15 &      \\ 
     &           &    6 &  3.54e-08 &  1.21e-15 &      \\ 
     &           &    7 &  4.23e-08 &  1.01e-15 &      \\ 
     &           &    8 &  3.55e-08 &  1.21e-15 &      \\ 
     &           &    9 &  4.23e-08 &  1.01e-15 &      \\ 
     &           &   10 &  3.55e-08 &  1.21e-15 &      \\ 
 894 &  8.93e+03 &   10 &           &           & iters  \\ 
 \hdashline 
     &           &    1 &  2.86e-08 &  3.33e-09 &      \\ 
     &           &    2 &  1.03e-08 &  8.15e-16 &      \\ 
     &           &    3 &  1.03e-08 &  2.95e-16 &      \\ 
     &           &    4 &  2.85e-08 &  2.94e-16 &      \\ 
     &           &    5 &  1.03e-08 &  8.15e-16 &      \\ 
     &           &    6 &  1.03e-08 &  2.95e-16 &      \\ 
     &           &    7 &  1.03e-08 &  2.95e-16 &      \\ 
     &           &    8 &  2.86e-08 &  2.94e-16 &      \\ 
     &           &    9 &  1.03e-08 &  8.15e-16 &      \\ 
     &           &   10 &  1.03e-08 &  2.95e-16 &      \\ 
 895 &  8.94e+03 &   10 &           &           & iters  \\ 
 \hdashline 
     &           &    1 &  3.23e-09 &  5.86e-09 &      \\ 
     &           &    2 &  3.23e-09 &  9.23e-17 &      \\ 
     &           &    3 &  3.22e-09 &  9.22e-17 &      \\ 
     &           &    4 &  3.22e-09 &  9.20e-17 &      \\ 
     &           &    5 &  3.22e-09 &  9.19e-17 &      \\ 
     &           &    6 &  3.21e-09 &  9.18e-17 &      \\ 
     &           &    7 &  3.21e-09 &  9.16e-17 &      \\ 
     &           &    8 &  3.56e-08 &  9.15e-17 &      \\ 
     &           &    9 &  3.25e-09 &  1.02e-15 &      \\ 
     &           &   10 &  3.25e-09 &  9.28e-17 &      \\ 
 896 &  8.95e+03 &   10 &           &           & iters  \\ 
 \hdashline 
     &           &    1 &  8.63e-09 &  2.51e-08 &      \\ 
     &           &    2 &  8.62e-09 &  2.46e-16 &      \\ 
     &           &    3 &  6.91e-08 &  2.46e-16 &      \\ 
     &           &    4 &  8.64e-08 &  1.97e-15 &      \\ 
     &           &    5 &  6.91e-08 &  2.47e-15 &      \\ 
     &           &    6 &  8.68e-09 &  1.97e-15 &      \\ 
     &           &    7 &  8.67e-09 &  2.48e-16 &      \\ 
     &           &    8 &  8.66e-09 &  2.47e-16 &      \\ 
     &           &    9 &  8.64e-09 &  2.47e-16 &      \\ 
     &           &   10 &  8.63e-09 &  2.47e-16 &      \\ 
 897 &  8.96e+03 &   10 &           &           & iters  \\ 
 \hdashline 
     &           &    1 &  6.75e-08 &  1.09e-08 &      \\ 
     &           &    2 &  1.03e-08 &  1.93e-15 &      \\ 
     &           &    3 &  1.03e-08 &  2.94e-16 &      \\ 
     &           &    4 &  1.03e-08 &  2.93e-16 &      \\ 
     &           &    5 &  1.02e-08 &  2.93e-16 &      \\ 
     &           &    6 &  1.02e-08 &  2.92e-16 &      \\ 
     &           &    7 &  1.02e-08 &  2.92e-16 &      \\ 
     &           &    8 &  6.75e-08 &  2.92e-16 &      \\ 
     &           &    9 &  8.80e-08 &  1.93e-15 &      \\ 
     &           &   10 &  6.75e-08 &  2.51e-15 &      \\ 
 898 &  8.97e+03 &   10 &           &           & iters  \\ 
 \hdashline 
     &           &    1 &  3.07e-08 &  3.80e-08 &      \\ 
     &           &    2 &  3.06e-08 &  8.76e-16 &      \\ 
     &           &    3 &  4.71e-08 &  8.74e-16 &      \\ 
     &           &    4 &  3.07e-08 &  1.34e-15 &      \\ 
     &           &    5 &  4.71e-08 &  8.75e-16 &      \\ 
     &           &    6 &  3.07e-08 &  1.34e-15 &      \\ 
     &           &    7 &  3.06e-08 &  8.76e-16 &      \\ 
     &           &    8 &  4.71e-08 &  8.75e-16 &      \\ 
     &           &    9 &  3.07e-08 &  1.34e-15 &      \\ 
     &           &   10 &  4.71e-08 &  8.75e-16 &      \\ 
 899 &  8.98e+03 &   10 &           &           & iters  \\ 
 \hdashline 
     &           &    1 &  9.63e-08 &  7.40e-08 &      \\ 
     &           &    2 &  5.92e-08 &  2.75e-15 &      \\ 
     &           &    3 &  1.86e-08 &  1.69e-15 &      \\ 
     &           &    4 &  1.86e-08 &  5.31e-16 &      \\ 
     &           &    5 &  5.91e-08 &  5.30e-16 &      \\ 
     &           &    6 &  1.86e-08 &  1.69e-15 &      \\ 
     &           &    7 &  1.86e-08 &  5.32e-16 &      \\ 
     &           &    8 &  1.86e-08 &  5.31e-16 &      \\ 
     &           &    9 &  5.91e-08 &  5.30e-16 &      \\ 
     &           &   10 &  1.86e-08 &  1.69e-15 &      \\ 
 900 &  8.99e+03 &   10 &           &           & iters  \\ 
 \hdashline 
     &           &    1 &  5.08e-08 &  3.86e-08 &      \\ 
     &           &    2 &  2.69e-08 &  1.45e-15 &      \\ 
     &           &    3 &  5.08e-08 &  7.69e-16 &      \\ 
     &           &    4 &  2.70e-08 &  1.45e-15 &      \\ 
     &           &    5 &  2.69e-08 &  7.70e-16 &      \\ 
     &           &    6 &  5.08e-08 &  7.68e-16 &      \\ 
     &           &    7 &  2.70e-08 &  1.45e-15 &      \\ 
     &           &    8 &  2.69e-08 &  7.69e-16 &      \\ 
     &           &    9 &  5.08e-08 &  7.68e-16 &      \\ 
     &           &   10 &  2.70e-08 &  1.45e-15 &      \\ 
 901 &  9.00e+03 &   10 &           &           & iters  \\ 
 \hdashline 
     &           &    1 &  7.21e-09 &  5.95e-08 &      \\ 
     &           &    2 &  7.20e-09 &  2.06e-16 &      \\ 
     &           &    3 &  7.19e-09 &  2.05e-16 &      \\ 
     &           &    4 &  7.18e-09 &  2.05e-16 &      \\ 
     &           &    5 &  7.17e-09 &  2.05e-16 &      \\ 
     &           &    6 &  7.16e-09 &  2.05e-16 &      \\ 
     &           &    7 &  7.15e-09 &  2.04e-16 &      \\ 
     &           &    8 &  7.06e-08 &  2.04e-16 &      \\ 
     &           &    9 &  8.49e-08 &  2.01e-15 &      \\ 
     &           &   10 &  7.06e-08 &  2.42e-15 &      \\ 
 902 &  9.01e+03 &   10 &           &           & iters  \\ 
 \hdashline 
     &           &    1 &  3.44e-08 &  7.43e-08 &      \\ 
     &           &    2 &  4.34e-08 &  9.81e-16 &      \\ 
     &           &    3 &  3.44e-08 &  1.24e-15 &      \\ 
     &           &    4 &  4.34e-08 &  9.81e-16 &      \\ 
     &           &    5 &  3.44e-08 &  1.24e-15 &      \\ 
     &           &    6 &  3.43e-08 &  9.81e-16 &      \\ 
     &           &    7 &  4.34e-08 &  9.80e-16 &      \\ 
     &           &    8 &  3.43e-08 &  1.24e-15 &      \\ 
     &           &    9 &  4.34e-08 &  9.80e-16 &      \\ 
     &           &   10 &  3.44e-08 &  1.24e-15 &      \\ 
 903 &  9.02e+03 &   10 &           &           & iters  \\ 
 \hdashline 
     &           &    1 &  9.16e-10 &  3.15e-09 &      \\ 
     &           &    2 &  9.14e-10 &  2.61e-17 &      \\ 
     &           &    3 &  9.13e-10 &  2.61e-17 &      \\ 
     &           &    4 &  9.12e-10 &  2.61e-17 &      \\ 
     &           &    5 &  9.10e-10 &  2.60e-17 &      \\ 
     &           &    6 &  9.09e-10 &  2.60e-17 &      \\ 
     &           &    7 &  9.08e-10 &  2.59e-17 &      \\ 
     &           &    8 &  9.07e-10 &  2.59e-17 &      \\ 
     &           &    9 &  9.05e-10 &  2.59e-17 &      \\ 
     &           &   10 &  9.04e-10 &  2.58e-17 &      \\ 
 904 &  9.03e+03 &   10 &           &           & iters  \\ 
 \hdashline 
     &           &    1 &  1.61e-08 &  4.07e-10 &      \\ 
     &           &    2 &  1.61e-08 &  4.61e-16 &      \\ 
     &           &    3 &  1.61e-08 &  4.60e-16 &      \\ 
     &           &    4 &  6.16e-08 &  4.59e-16 &      \\ 
     &           &    5 &  1.62e-08 &  1.76e-15 &      \\ 
     &           &    6 &  1.61e-08 &  4.61e-16 &      \\ 
     &           &    7 &  1.61e-08 &  4.60e-16 &      \\ 
     &           &    8 &  1.61e-08 &  4.60e-16 &      \\ 
     &           &    9 &  6.16e-08 &  4.59e-16 &      \\ 
     &           &   10 &  1.62e-08 &  1.76e-15 &      \\ 
 905 &  9.04e+03 &   10 &           &           & iters  \\ 
 \hdashline 
     &           &    1 &  8.41e-09 &  1.60e-08 &      \\ 
     &           &    2 &  8.39e-09 &  2.40e-16 &      \\ 
     &           &    3 &  8.38e-09 &  2.40e-16 &      \\ 
     &           &    4 &  6.93e-08 &  2.39e-16 &      \\ 
     &           &    5 &  8.62e-08 &  1.98e-15 &      \\ 
     &           &    6 &  6.93e-08 &  2.46e-15 &      \\ 
     &           &    7 &  8.45e-09 &  1.98e-15 &      \\ 
     &           &    8 &  8.43e-09 &  2.41e-16 &      \\ 
     &           &    9 &  8.42e-09 &  2.41e-16 &      \\ 
     &           &   10 &  8.41e-09 &  2.40e-16 &      \\ 
 906 &  9.05e+03 &   10 &           &           & iters  \\ 
 \hdashline 
     &           &    1 &  2.85e-08 &  3.72e-09 &      \\ 
     &           &    2 &  2.85e-08 &  8.14e-16 &      \\ 
     &           &    3 &  4.93e-08 &  8.13e-16 &      \\ 
     &           &    4 &  2.85e-08 &  1.41e-15 &      \\ 
     &           &    5 &  2.85e-08 &  8.14e-16 &      \\ 
     &           &    6 &  4.93e-08 &  8.12e-16 &      \\ 
     &           &    7 &  2.85e-08 &  1.41e-15 &      \\ 
     &           &    8 &  4.92e-08 &  8.13e-16 &      \\ 
     &           &    9 &  2.85e-08 &  1.41e-15 &      \\ 
     &           &   10 &  2.85e-08 &  8.14e-16 &      \\ 
 907 &  9.06e+03 &   10 &           &           & iters  \\ 
 \hdashline 
     &           &    1 &  9.51e-08 &  1.43e-09 &      \\ 
     &           &    2 &  6.04e-08 &  2.71e-15 &      \\ 
     &           &    3 &  1.73e-08 &  1.72e-15 &      \\ 
     &           &    4 &  1.73e-08 &  4.95e-16 &      \\ 
     &           &    5 &  1.73e-08 &  4.94e-16 &      \\ 
     &           &    6 &  6.04e-08 &  4.94e-16 &      \\ 
     &           &    7 &  1.74e-08 &  1.72e-15 &      \\ 
     &           &    8 &  1.73e-08 &  4.95e-16 &      \\ 
     &           &    9 &  1.73e-08 &  4.95e-16 &      \\ 
     &           &   10 &  6.04e-08 &  4.94e-16 &      \\ 
 908 &  9.07e+03 &   10 &           &           & iters  \\ 
 \hdashline 
     &           &    1 &  6.18e-08 &  1.01e-08 &      \\ 
     &           &    2 &  1.60e-08 &  1.76e-15 &      \\ 
     &           &    3 &  1.60e-08 &  4.57e-16 &      \\ 
     &           &    4 &  1.60e-08 &  4.56e-16 &      \\ 
     &           &    5 &  6.17e-08 &  4.56e-16 &      \\ 
     &           &    6 &  1.60e-08 &  1.76e-15 &      \\ 
     &           &    7 &  1.60e-08 &  4.58e-16 &      \\ 
     &           &    8 &  1.60e-08 &  4.57e-16 &      \\ 
     &           &    9 &  1.60e-08 &  4.56e-16 &      \\ 
     &           &   10 &  6.17e-08 &  4.56e-16 &      \\ 
 909 &  9.08e+03 &   10 &           &           & iters  \\ 
 \hdashline 
     &           &    1 &  1.70e-08 &  6.76e-09 &      \\ 
     &           &    2 &  1.70e-08 &  4.86e-16 &      \\ 
     &           &    3 &  1.70e-08 &  4.85e-16 &      \\ 
     &           &    4 &  6.07e-08 &  4.84e-16 &      \\ 
     &           &    5 &  1.70e-08 &  1.73e-15 &      \\ 
     &           &    6 &  1.70e-08 &  4.86e-16 &      \\ 
     &           &    7 &  1.70e-08 &  4.85e-16 &      \\ 
     &           &    8 &  1.70e-08 &  4.85e-16 &      \\ 
     &           &    9 &  6.08e-08 &  4.84e-16 &      \\ 
     &           &   10 &  1.70e-08 &  1.73e-15 &      \\ 
 910 &  9.09e+03 &   10 &           &           & iters  \\ 
 \hdashline 
     &           &    1 &  4.32e-08 &  1.03e-08 &      \\ 
     &           &    2 &  3.46e-08 &  1.23e-15 &      \\ 
     &           &    3 &  4.32e-08 &  9.87e-16 &      \\ 
     &           &    4 &  3.46e-08 &  1.23e-15 &      \\ 
     &           &    5 &  4.31e-08 &  9.87e-16 &      \\ 
     &           &    6 &  3.46e-08 &  1.23e-15 &      \\ 
     &           &    7 &  4.31e-08 &  9.88e-16 &      \\ 
     &           &    8 &  3.46e-08 &  1.23e-15 &      \\ 
     &           &    9 &  3.46e-08 &  9.88e-16 &      \\ 
     &           &   10 &  4.32e-08 &  9.87e-16 &      \\ 
 911 &  9.10e+03 &   10 &           &           & iters  \\ 
 \hdashline 
     &           &    1 &  1.06e-08 &  6.04e-09 &      \\ 
     &           &    2 &  1.06e-08 &  3.03e-16 &      \\ 
     &           &    3 &  1.06e-08 &  3.03e-16 &      \\ 
     &           &    4 &  1.06e-08 &  3.02e-16 &      \\ 
     &           &    5 &  1.06e-08 &  3.02e-16 &      \\ 
     &           &    6 &  1.05e-08 &  3.01e-16 &      \\ 
     &           &    7 &  6.72e-08 &  3.01e-16 &      \\ 
     &           &    8 &  1.06e-08 &  1.92e-15 &      \\ 
     &           &    9 &  1.06e-08 &  3.03e-16 &      \\ 
     &           &   10 &  1.06e-08 &  3.03e-16 &      \\ 
 912 &  9.11e+03 &   10 &           &           & iters  \\ 
 \hdashline 
     &           &    1 &  2.49e-08 &  3.88e-09 &      \\ 
     &           &    2 &  5.28e-08 &  7.12e-16 &      \\ 
     &           &    3 &  2.50e-08 &  1.51e-15 &      \\ 
     &           &    4 &  2.49e-08 &  7.13e-16 &      \\ 
     &           &    5 &  5.28e-08 &  7.12e-16 &      \\ 
     &           &    6 &  2.50e-08 &  1.51e-15 &      \\ 
     &           &    7 &  2.49e-08 &  7.13e-16 &      \\ 
     &           &    8 &  5.28e-08 &  7.12e-16 &      \\ 
     &           &    9 &  2.50e-08 &  1.51e-15 &      \\ 
     &           &   10 &  2.49e-08 &  7.13e-16 &      \\ 
 913 &  9.12e+03 &   10 &           &           & iters  \\ 
 \hdashline 
     &           &    1 &  1.78e-08 &  6.98e-09 &      \\ 
     &           &    2 &  1.78e-08 &  5.09e-16 &      \\ 
     &           &    3 &  1.78e-08 &  5.08e-16 &      \\ 
     &           &    4 &  5.99e-08 &  5.07e-16 &      \\ 
     &           &    5 &  1.78e-08 &  1.71e-15 &      \\ 
     &           &    6 &  1.78e-08 &  5.09e-16 &      \\ 
     &           &    7 &  1.78e-08 &  5.08e-16 &      \\ 
     &           &    8 &  1.78e-08 &  5.07e-16 &      \\ 
     &           &    9 &  6.00e-08 &  5.07e-16 &      \\ 
     &           &   10 &  1.78e-08 &  1.71e-15 &      \\ 
 914 &  9.13e+03 &   10 &           &           & iters  \\ 
 \hdashline 
     &           &    1 &  3.37e-08 &  4.44e-09 &      \\ 
     &           &    2 &  4.41e-08 &  9.61e-16 &      \\ 
     &           &    3 &  3.37e-08 &  1.26e-15 &      \\ 
     &           &    4 &  4.40e-08 &  9.62e-16 &      \\ 
     &           &    5 &  3.37e-08 &  1.26e-15 &      \\ 
     &           &    6 &  3.37e-08 &  9.62e-16 &      \\ 
     &           &    7 &  4.41e-08 &  9.61e-16 &      \\ 
     &           &    8 &  3.37e-08 &  1.26e-15 &      \\ 
     &           &    9 &  4.41e-08 &  9.61e-16 &      \\ 
     &           &   10 &  3.37e-08 &  1.26e-15 &      \\ 
 915 &  9.14e+03 &   10 &           &           & iters  \\ 
 \hdashline 
     &           &    1 &  3.10e-08 &  3.20e-08 &      \\ 
     &           &    2 &  4.67e-08 &  8.85e-16 &      \\ 
     &           &    3 &  3.10e-08 &  1.33e-15 &      \\ 
     &           &    4 &  3.10e-08 &  8.85e-16 &      \\ 
     &           &    5 &  4.68e-08 &  8.84e-16 &      \\ 
     &           &    6 &  3.10e-08 &  1.33e-15 &      \\ 
     &           &    7 &  4.67e-08 &  8.85e-16 &      \\ 
     &           &    8 &  3.10e-08 &  1.33e-15 &      \\ 
     &           &    9 &  3.10e-08 &  8.85e-16 &      \\ 
     &           &   10 &  4.68e-08 &  8.84e-16 &      \\ 
 916 &  9.15e+03 &   10 &           &           & iters  \\ 
 \hdashline 
     &           &    1 &  4.50e-11 &  4.54e-08 &      \\ 
     &           &    2 &  4.49e-11 &  1.28e-18 &      \\ 
     &           &    3 &  4.49e-11 &  1.28e-18 &      \\ 
     &           &    4 &  4.48e-11 &  1.28e-18 &      \\ 
     &           &    5 &  4.47e-11 &  1.28e-18 &      \\ 
     &           &    6 &  4.47e-11 &  1.28e-18 &      \\ 
     &           &    7 &  4.46e-11 &  1.28e-18 &      \\ 
     &           &    8 &  4.45e-11 &  1.27e-18 &      \\ 
     &           &    9 &  4.45e-11 &  1.27e-18 &      \\ 
     &           &   10 &  4.44e-11 &  1.27e-18 &      \\ 
 917 &  9.16e+03 &   10 &           &           & iters  \\ 
 \hdashline 
     &           &    1 &  3.10e-09 &  4.32e-08 &      \\ 
     &           &    2 &  3.09e-09 &  8.84e-17 &      \\ 
     &           &    3 &  3.09e-09 &  8.83e-17 &      \\ 
     &           &    4 &  3.09e-09 &  8.82e-17 &      \\ 
     &           &    5 &  3.08e-09 &  8.81e-17 &      \\ 
     &           &    6 &  3.08e-09 &  8.79e-17 &      \\ 
     &           &    7 &  3.07e-09 &  8.78e-17 &      \\ 
     &           &    8 &  3.07e-09 &  8.77e-17 &      \\ 
     &           &    9 &  3.06e-09 &  8.76e-17 &      \\ 
     &           &   10 &  3.06e-09 &  8.74e-17 &      \\ 
 918 &  9.17e+03 &   10 &           &           & iters  \\ 
 \hdashline 
     &           &    1 &  2.17e-09 &  4.53e-08 &      \\ 
     &           &    2 &  2.17e-09 &  6.19e-17 &      \\ 
     &           &    3 &  2.16e-09 &  6.18e-17 &      \\ 
     &           &    4 &  2.16e-09 &  6.17e-17 &      \\ 
     &           &    5 &  2.16e-09 &  6.16e-17 &      \\ 
     &           &    6 &  3.67e-08 &  6.16e-17 &      \\ 
     &           &    7 &  2.21e-09 &  1.05e-15 &      \\ 
     &           &    8 &  2.20e-09 &  6.30e-17 &      \\ 
     &           &    9 &  2.20e-09 &  6.29e-17 &      \\ 
     &           &   10 &  2.20e-09 &  6.28e-17 &      \\ 
 919 &  9.18e+03 &   10 &           &           & iters  \\ 
 \hdashline 
     &           &    1 &  2.43e-08 &  2.28e-08 &      \\ 
     &           &    2 &  1.46e-08 &  6.93e-16 &      \\ 
     &           &    3 &  1.46e-08 &  4.16e-16 &      \\ 
     &           &    4 &  2.43e-08 &  4.16e-16 &      \\ 
     &           &    5 &  1.46e-08 &  6.93e-16 &      \\ 
     &           &    6 &  2.43e-08 &  4.16e-16 &      \\ 
     &           &    7 &  1.46e-08 &  6.93e-16 &      \\ 
     &           &    8 &  1.46e-08 &  4.17e-16 &      \\ 
     &           &    9 &  2.43e-08 &  4.16e-16 &      \\ 
     &           &   10 &  1.46e-08 &  6.93e-16 &      \\ 
 920 &  9.19e+03 &   10 &           &           & iters  \\ 
 \hdashline 
     &           &    1 &  6.32e-09 &  2.73e-09 &      \\ 
     &           &    2 &  6.31e-09 &  1.80e-16 &      \\ 
     &           &    3 &  3.25e-08 &  1.80e-16 &      \\ 
     &           &    4 &  6.35e-09 &  9.29e-16 &      \\ 
     &           &    5 &  6.34e-09 &  1.81e-16 &      \\ 
     &           &    6 &  6.33e-09 &  1.81e-16 &      \\ 
     &           &    7 &  6.32e-09 &  1.81e-16 &      \\ 
     &           &    8 &  6.31e-09 &  1.80e-16 &      \\ 
     &           &    9 &  3.25e-08 &  1.80e-16 &      \\ 
     &           &   10 &  6.35e-09 &  9.29e-16 &      \\ 
 921 &  9.20e+03 &   10 &           &           & iters  \\ 
 \hdashline 
     &           &    1 &  2.29e-08 &  8.08e-09 &      \\ 
     &           &    2 &  5.49e-08 &  6.53e-16 &      \\ 
     &           &    3 &  2.29e-08 &  1.57e-15 &      \\ 
     &           &    4 &  2.29e-08 &  6.54e-16 &      \\ 
     &           &    5 &  2.28e-08 &  6.53e-16 &      \\ 
     &           &    6 &  5.49e-08 &  6.52e-16 &      \\ 
     &           &    7 &  2.29e-08 &  1.57e-15 &      \\ 
     &           &    8 &  2.29e-08 &  6.53e-16 &      \\ 
     &           &    9 &  5.49e-08 &  6.52e-16 &      \\ 
     &           &   10 &  2.29e-08 &  1.57e-15 &      \\ 
 922 &  9.21e+03 &   10 &           &           & iters  \\ 
 \hdashline 
     &           &    1 &  4.06e-08 &  9.41e-09 &      \\ 
     &           &    2 &  3.71e-08 &  1.16e-15 &      \\ 
     &           &    3 &  4.06e-08 &  1.06e-15 &      \\ 
     &           &    4 &  3.71e-08 &  1.16e-15 &      \\ 
     &           &    5 &  3.71e-08 &  1.06e-15 &      \\ 
     &           &    6 &  4.07e-08 &  1.06e-15 &      \\ 
     &           &    7 &  3.71e-08 &  1.16e-15 &      \\ 
     &           &    8 &  4.07e-08 &  1.06e-15 &      \\ 
     &           &    9 &  3.71e-08 &  1.16e-15 &      \\ 
     &           &   10 &  4.07e-08 &  1.06e-15 &      \\ 
 923 &  9.22e+03 &   10 &           &           & iters  \\ 
 \hdashline 
     &           &    1 &  2.09e-08 &  4.36e-08 &      \\ 
     &           &    2 &  5.68e-08 &  5.98e-16 &      \\ 
     &           &    3 &  2.10e-08 &  1.62e-15 &      \\ 
     &           &    4 &  2.10e-08 &  5.99e-16 &      \\ 
     &           &    5 &  2.09e-08 &  5.98e-16 &      \\ 
     &           &    6 &  5.68e-08 &  5.97e-16 &      \\ 
     &           &    7 &  2.10e-08 &  1.62e-15 &      \\ 
     &           &    8 &  2.09e-08 &  5.99e-16 &      \\ 
     &           &    9 &  5.68e-08 &  5.98e-16 &      \\ 
     &           &   10 &  2.10e-08 &  1.62e-15 &      \\ 
 924 &  9.23e+03 &   10 &           &           & iters  \\ 
 \hdashline 
     &           &    1 &  2.74e-09 &  5.08e-08 &      \\ 
     &           &    2 &  2.73e-09 &  7.81e-17 &      \\ 
     &           &    3 &  2.73e-09 &  7.79e-17 &      \\ 
     &           &    4 &  2.72e-09 &  7.78e-17 &      \\ 
     &           &    5 &  2.72e-09 &  7.77e-17 &      \\ 
     &           &    6 &  2.72e-09 &  7.76e-17 &      \\ 
     &           &    7 &  3.61e-08 &  7.75e-17 &      \\ 
     &           &    8 &  2.76e-09 &  1.03e-15 &      \\ 
     &           &    9 &  2.76e-09 &  7.89e-17 &      \\ 
     &           &   10 &  2.76e-09 &  7.88e-17 &      \\ 
 925 &  9.24e+03 &   10 &           &           & iters  \\ 
 \hdashline 
     &           &    1 &  7.45e-09 &  5.10e-08 &      \\ 
     &           &    2 &  7.44e-09 &  2.13e-16 &      \\ 
     &           &    3 &  7.03e-08 &  2.12e-16 &      \\ 
     &           &    4 &  8.52e-08 &  2.01e-15 &      \\ 
     &           &    5 &  7.03e-08 &  2.43e-15 &      \\ 
     &           &    6 &  7.51e-09 &  2.01e-15 &      \\ 
     &           &    7 &  7.50e-09 &  2.14e-16 &      \\ 
     &           &    8 &  7.49e-09 &  2.14e-16 &      \\ 
     &           &    9 &  7.48e-09 &  2.14e-16 &      \\ 
     &           &   10 &  7.46e-09 &  2.13e-16 &      \\ 
 926 &  9.25e+03 &   10 &           &           & iters  \\ 
 \hdashline 
     &           &    1 &  1.60e-08 &  4.90e-08 &      \\ 
     &           &    2 &  1.60e-08 &  4.57e-16 &      \\ 
     &           &    3 &  1.60e-08 &  4.57e-16 &      \\ 
     &           &    4 &  6.17e-08 &  4.56e-16 &      \\ 
     &           &    5 &  1.60e-08 &  1.76e-15 &      \\ 
     &           &    6 &  1.60e-08 &  4.58e-16 &      \\ 
     &           &    7 &  1.60e-08 &  4.57e-16 &      \\ 
     &           &    8 &  6.17e-08 &  4.57e-16 &      \\ 
     &           &    9 &  1.61e-08 &  1.76e-15 &      \\ 
     &           &   10 &  1.60e-08 &  4.58e-16 &      \\ 
 927 &  9.26e+03 &   10 &           &           & iters  \\ 
 \hdashline 
     &           &    1 &  2.80e-08 &  1.80e-08 &      \\ 
     &           &    2 &  4.97e-08 &  8.00e-16 &      \\ 
     &           &    3 &  2.80e-08 &  1.42e-15 &      \\ 
     &           &    4 &  4.97e-08 &  8.00e-16 &      \\ 
     &           &    5 &  2.81e-08 &  1.42e-15 &      \\ 
     &           &    6 &  2.80e-08 &  8.01e-16 &      \\ 
     &           &    7 &  4.97e-08 &  8.00e-16 &      \\ 
     &           &    8 &  2.81e-08 &  1.42e-15 &      \\ 
     &           &    9 &  2.80e-08 &  8.01e-16 &      \\ 
     &           &   10 &  4.97e-08 &  8.00e-16 &      \\ 
 928 &  9.27e+03 &   10 &           &           & iters  \\ 
 \hdashline 
     &           &    1 &  1.06e-08 &  1.44e-09 &      \\ 
     &           &    2 &  1.06e-08 &  3.02e-16 &      \\ 
     &           &    3 &  1.06e-08 &  3.02e-16 &      \\ 
     &           &    4 &  1.05e-08 &  3.01e-16 &      \\ 
     &           &    5 &  6.72e-08 &  3.01e-16 &      \\ 
     &           &    6 &  1.06e-08 &  1.92e-15 &      \\ 
     &           &    7 &  1.06e-08 &  3.03e-16 &      \\ 
     &           &    8 &  1.06e-08 &  3.03e-16 &      \\ 
     &           &    9 &  1.06e-08 &  3.02e-16 &      \\ 
     &           &   10 &  1.06e-08 &  3.02e-16 &      \\ 
 929 &  9.28e+03 &   10 &           &           & iters  \\ 
 \hdashline 
     &           &    1 &  1.26e-08 &  6.36e-09 &      \\ 
     &           &    2 &  1.26e-08 &  3.60e-16 &      \\ 
     &           &    3 &  6.51e-08 &  3.60e-16 &      \\ 
     &           &    4 &  1.27e-08 &  1.86e-15 &      \\ 
     &           &    5 &  1.27e-08 &  3.62e-16 &      \\ 
     &           &    6 &  1.26e-08 &  3.61e-16 &      \\ 
     &           &    7 &  1.26e-08 &  3.61e-16 &      \\ 
     &           &    8 &  1.26e-08 &  3.60e-16 &      \\ 
     &           &    9 &  6.51e-08 &  3.60e-16 &      \\ 
     &           &   10 &  1.27e-08 &  1.86e-15 &      \\ 
 930 &  9.29e+03 &   10 &           &           & iters  \\ 
 \hdashline 
     &           &    1 &  1.54e-08 &  1.23e-08 &      \\ 
     &           &    2 &  1.54e-08 &  4.39e-16 &      \\ 
     &           &    3 &  1.54e-08 &  4.39e-16 &      \\ 
     &           &    4 &  1.53e-08 &  4.38e-16 &      \\ 
     &           &    5 &  6.24e-08 &  4.38e-16 &      \\ 
     &           &    6 &  1.54e-08 &  1.78e-15 &      \\ 
     &           &    7 &  1.54e-08 &  4.39e-16 &      \\ 
     &           &    8 &  1.54e-08 &  4.39e-16 &      \\ 
     &           &    9 &  1.53e-08 &  4.38e-16 &      \\ 
     &           &   10 &  6.24e-08 &  4.38e-16 &      \\ 
 931 &  9.30e+03 &   10 &           &           & iters  \\ 
 \hdashline 
     &           &    1 &  5.39e-08 &  2.22e-09 &      \\ 
     &           &    2 &  5.38e-08 &  1.54e-15 &      \\ 
     &           &    3 &  2.39e-08 &  1.54e-15 &      \\ 
     &           &    4 &  2.39e-08 &  6.83e-16 &      \\ 
     &           &    5 &  5.38e-08 &  6.82e-16 &      \\ 
     &           &    6 &  2.39e-08 &  1.54e-15 &      \\ 
     &           &    7 &  2.39e-08 &  6.83e-16 &      \\ 
     &           &    8 &  2.39e-08 &  6.82e-16 &      \\ 
     &           &    9 &  5.39e-08 &  6.81e-16 &      \\ 
     &           &   10 &  2.39e-08 &  1.54e-15 &      \\ 
 932 &  9.31e+03 &   10 &           &           & iters  \\ 
 \hdashline 
     &           &    1 &  1.62e-10 &  1.17e-08 &      \\ 
 933 &  9.32e+03 &    1 &           &           & forces  \\ 
 \hdashline 
     &           &    1 &  5.32e-08 &  6.53e-09 &      \\ 
     &           &    2 &  2.46e-08 &  1.52e-15 &      \\ 
     &           &    3 &  2.45e-08 &  7.01e-16 &      \\ 
     &           &    4 &  5.32e-08 &  7.00e-16 &      \\ 
     &           &    5 &  2.46e-08 &  1.52e-15 &      \\ 
     &           &    6 &  2.45e-08 &  7.01e-16 &      \\ 
     &           &    7 &  5.32e-08 &  7.00e-16 &      \\ 
     &           &    8 &  2.46e-08 &  1.52e-15 &      \\ 
     &           &    9 &  2.45e-08 &  7.02e-16 &      \\ 
     &           &   10 &  5.32e-08 &  7.01e-16 &      \\ 
 934 &  9.33e+03 &   10 &           &           & iters  \\ 
 \hdashline 
     &           &    1 &  1.46e-08 &  2.94e-10 &      \\ 
     &           &    2 &  6.31e-08 &  4.17e-16 &      \\ 
     &           &    3 &  1.47e-08 &  1.80e-15 &      \\ 
     &           &    4 &  1.47e-08 &  4.19e-16 &      \\ 
     &           &    5 &  1.46e-08 &  4.18e-16 &      \\ 
     &           &    6 &  1.46e-08 &  4.18e-16 &      \\ 
     &           &    7 &  6.31e-08 &  4.17e-16 &      \\ 
     &           &    8 &  1.47e-08 &  1.80e-15 &      \\ 
     &           &    9 &  1.47e-08 &  4.19e-16 &      \\ 
     &           &   10 &  1.46e-08 &  4.19e-16 &      \\ 
 935 &  9.34e+03 &   10 &           &           & iters  \\ 
 \hdashline 
     &           &    1 &  5.13e-08 &  5.05e-09 &      \\ 
     &           &    2 &  2.65e-08 &  1.46e-15 &      \\ 
     &           &    3 &  5.12e-08 &  7.56e-16 &      \\ 
     &           &    4 &  2.65e-08 &  1.46e-15 &      \\ 
     &           &    5 &  2.65e-08 &  7.57e-16 &      \\ 
     &           &    6 &  5.12e-08 &  7.56e-16 &      \\ 
     &           &    7 &  2.65e-08 &  1.46e-15 &      \\ 
     &           &    8 &  2.65e-08 &  7.57e-16 &      \\ 
     &           &    9 &  5.12e-08 &  7.56e-16 &      \\ 
     &           &   10 &  2.65e-08 &  1.46e-15 &      \\ 
 936 &  9.35e+03 &   10 &           &           & iters  \\ 
 \hdashline 
     &           &    1 &  3.65e-08 &  1.18e-10 &      \\ 
     &           &    2 &  4.13e-08 &  1.04e-15 &      \\ 
     &           &    3 &  3.65e-08 &  1.18e-15 &      \\ 
     &           &    4 &  4.13e-08 &  1.04e-15 &      \\ 
     &           &    5 &  3.65e-08 &  1.18e-15 &      \\ 
     &           &    6 &  4.12e-08 &  1.04e-15 &      \\ 
     &           &    7 &  3.65e-08 &  1.18e-15 &      \\ 
     &           &    8 &  4.12e-08 &  1.04e-15 &      \\ 
     &           &    9 &  3.65e-08 &  1.18e-15 &      \\ 
     &           &   10 &  4.12e-08 &  1.04e-15 &      \\ 
 937 &  9.36e+03 &   10 &           &           & iters  \\ 
 \hdashline 
     &           &    1 &  2.97e-08 &  2.22e-09 &      \\ 
     &           &    2 &  2.96e-08 &  8.46e-16 &      \\ 
     &           &    3 &  2.96e-08 &  8.45e-16 &      \\ 
     &           &    4 &  4.82e-08 &  8.44e-16 &      \\ 
     &           &    5 &  2.96e-08 &  1.37e-15 &      \\ 
     &           &    6 &  2.96e-08 &  8.45e-16 &      \\ 
     &           &    7 &  4.82e-08 &  8.44e-16 &      \\ 
     &           &    8 &  2.96e-08 &  1.37e-15 &      \\ 
     &           &    9 &  4.81e-08 &  8.44e-16 &      \\ 
     &           &   10 &  2.96e-08 &  1.37e-15 &      \\ 
 938 &  9.37e+03 &   10 &           &           & iters  \\ 
 \hdashline 
     &           &    1 &  6.91e-09 &  2.49e-08 &      \\ 
     &           &    2 &  6.90e-09 &  1.97e-16 &      \\ 
     &           &    3 &  6.89e-09 &  1.97e-16 &      \\ 
     &           &    4 &  6.88e-09 &  1.97e-16 &      \\ 
     &           &    5 &  6.87e-09 &  1.96e-16 &      \\ 
     &           &    6 &  6.86e-09 &  1.96e-16 &      \\ 
     &           &    7 &  7.08e-08 &  1.96e-16 &      \\ 
     &           &    8 &  8.46e-08 &  2.02e-15 &      \\ 
     &           &    9 &  7.09e-08 &  2.42e-15 &      \\ 
     &           &   10 &  6.93e-09 &  2.02e-15 &      \\ 
 939 &  9.38e+03 &   10 &           &           & iters  \\ 
 \hdashline 
     &           &    1 &  1.75e-08 &  3.55e-08 &      \\ 
     &           &    2 &  1.74e-08 &  4.98e-16 &      \\ 
     &           &    3 &  6.03e-08 &  4.98e-16 &      \\ 
     &           &    4 &  1.75e-08 &  1.72e-15 &      \\ 
     &           &    5 &  1.75e-08 &  4.99e-16 &      \\ 
     &           &    6 &  1.74e-08 &  4.99e-16 &      \\ 
     &           &    7 &  6.03e-08 &  4.98e-16 &      \\ 
     &           &    8 &  1.75e-08 &  1.72e-15 &      \\ 
     &           &    9 &  1.75e-08 &  5.00e-16 &      \\ 
     &           &   10 &  1.75e-08 &  4.99e-16 &      \\ 
 940 &  9.39e+03 &   10 &           &           & iters  \\ 
 \hdashline 
     &           &    1 &  5.59e-08 &  2.34e-08 &      \\ 
     &           &    2 &  1.70e-08 &  1.60e-15 &      \\ 
     &           &    3 &  6.07e-08 &  4.85e-16 &      \\ 
     &           &    4 &  1.71e-08 &  1.73e-15 &      \\ 
     &           &    5 &  1.70e-08 &  4.87e-16 &      \\ 
     &           &    6 &  1.70e-08 &  4.86e-16 &      \\ 
     &           &    7 &  1.70e-08 &  4.86e-16 &      \\ 
     &           &    8 &  6.07e-08 &  4.85e-16 &      \\ 
     &           &    9 &  1.71e-08 &  1.73e-15 &      \\ 
     &           &   10 &  1.70e-08 &  4.87e-16 &      \\ 
 941 &  9.40e+03 &   10 &           &           & iters  \\ 
 \hdashline 
     &           &    1 &  5.21e-08 &  1.52e-08 &      \\ 
     &           &    2 &  2.56e-08 &  1.49e-15 &      \\ 
     &           &    3 &  2.56e-08 &  7.32e-16 &      \\ 
     &           &    4 &  5.21e-08 &  7.30e-16 &      \\ 
     &           &    5 &  2.56e-08 &  1.49e-15 &      \\ 
     &           &    6 &  2.56e-08 &  7.32e-16 &      \\ 
     &           &    7 &  2.56e-08 &  7.30e-16 &      \\ 
     &           &    8 &  5.22e-08 &  7.29e-16 &      \\ 
     &           &    9 &  2.56e-08 &  1.49e-15 &      \\ 
     &           &   10 &  2.56e-08 &  7.31e-16 &      \\ 
 942 &  9.41e+03 &   10 &           &           & iters  \\ 
 \hdashline 
     &           &    1 &  1.93e-08 &  1.86e-08 &      \\ 
     &           &    2 &  1.92e-08 &  5.50e-16 &      \\ 
     &           &    3 &  1.96e-08 &  5.49e-16 &      \\ 
     &           &    4 &  1.92e-08 &  5.60e-16 &      \\ 
     &           &    5 &  1.96e-08 &  5.49e-16 &      \\ 
     &           &    6 &  1.92e-08 &  5.60e-16 &      \\ 
     &           &    7 &  1.96e-08 &  5.49e-16 &      \\ 
     &           &    8 &  1.92e-08 &  5.60e-16 &      \\ 
     &           &    9 &  1.96e-08 &  5.49e-16 &      \\ 
     &           &   10 &  1.92e-08 &  5.60e-16 &      \\ 
 943 &  9.42e+03 &   10 &           &           & iters  \\ 
 \hdashline 
     &           &    1 &  5.85e-09 &  1.47e-08 &      \\ 
     &           &    2 &  5.84e-09 &  1.67e-16 &      \\ 
     &           &    3 &  5.83e-09 &  1.67e-16 &      \\ 
     &           &    4 &  5.82e-09 &  1.66e-16 &      \\ 
     &           &    5 &  7.19e-08 &  1.66e-16 &      \\ 
     &           &    6 &  8.36e-08 &  2.05e-15 &      \\ 
     &           &    7 &  7.19e-08 &  2.39e-15 &      \\ 
     &           &    8 &  5.90e-09 &  2.05e-15 &      \\ 
     &           &    9 &  5.89e-09 &  1.68e-16 &      \\ 
     &           &   10 &  5.89e-09 &  1.68e-16 &      \\ 
 944 &  9.43e+03 &   10 &           &           & iters  \\ 
 \hdashline 
     &           &    1 &  1.30e-09 &  6.59e-09 &      \\ 
     &           &    2 &  1.30e-09 &  3.72e-17 &      \\ 
     &           &    3 &  1.30e-09 &  3.72e-17 &      \\ 
     &           &    4 &  1.30e-09 &  3.71e-17 &      \\ 
     &           &    5 &  1.30e-09 &  3.71e-17 &      \\ 
     &           &    6 &  1.29e-09 &  3.70e-17 &      \\ 
     &           &    7 &  1.29e-09 &  3.70e-17 &      \\ 
     &           &    8 &  1.29e-09 &  3.69e-17 &      \\ 
     &           &    9 &  1.29e-09 &  3.68e-17 &      \\ 
     &           &   10 &  1.29e-09 &  3.68e-17 &      \\ 
 945 &  9.44e+03 &   10 &           &           & iters  \\ 
 \hdashline 
     &           &    1 &  2.99e-08 &  4.31e-10 &      \\ 
     &           &    2 &  4.78e-08 &  8.54e-16 &      \\ 
     &           &    3 &  2.99e-08 &  1.36e-15 &      \\ 
     &           &    4 &  2.99e-08 &  8.55e-16 &      \\ 
     &           &    5 &  4.78e-08 &  8.53e-16 &      \\ 
     &           &    6 &  2.99e-08 &  1.37e-15 &      \\ 
     &           &    7 &  4.78e-08 &  8.54e-16 &      \\ 
     &           &    8 &  3.00e-08 &  1.36e-15 &      \\ 
     &           &    9 &  2.99e-08 &  8.55e-16 &      \\ 
     &           &   10 &  4.78e-08 &  8.54e-16 &      \\ 
 946 &  9.45e+03 &   10 &           &           & iters  \\ 
 \hdashline 
     &           &    1 &  2.61e-08 &  1.59e-08 &      \\ 
     &           &    2 &  2.61e-08 &  7.46e-16 &      \\ 
     &           &    3 &  5.16e-08 &  7.45e-16 &      \\ 
     &           &    4 &  2.61e-08 &  1.47e-15 &      \\ 
     &           &    5 &  2.61e-08 &  7.46e-16 &      \\ 
     &           &    6 &  5.16e-08 &  7.45e-16 &      \\ 
     &           &    7 &  2.61e-08 &  1.47e-15 &      \\ 
     &           &    8 &  2.61e-08 &  7.46e-16 &      \\ 
     &           &    9 &  5.16e-08 &  7.45e-16 &      \\ 
     &           &   10 &  2.61e-08 &  1.47e-15 &      \\ 
 947 &  9.46e+03 &   10 &           &           & iters  \\ 
 \hdashline 
     &           &    1 &  1.14e-08 &  2.81e-08 &      \\ 
     &           &    2 &  1.14e-08 &  3.25e-16 &      \\ 
     &           &    3 &  1.14e-08 &  3.25e-16 &      \\ 
     &           &    4 &  1.13e-08 &  3.24e-16 &      \\ 
     &           &    5 &  6.64e-08 &  3.24e-16 &      \\ 
     &           &    6 &  5.03e-08 &  1.89e-15 &      \\ 
     &           &    7 &  1.14e-08 &  1.43e-15 &      \\ 
     &           &    8 &  1.13e-08 &  3.24e-16 &      \\ 
     &           &    9 &  6.64e-08 &  3.23e-16 &      \\ 
     &           &   10 &  1.14e-08 &  1.89e-15 &      \\ 
 948 &  9.47e+03 &   10 &           &           & iters  \\ 
 \hdashline 
     &           &    1 &  5.28e-08 &  2.48e-08 &      \\ 
     &           &    2 &  2.49e-08 &  1.51e-15 &      \\ 
     &           &    3 &  2.49e-08 &  7.11e-16 &      \\ 
     &           &    4 &  5.28e-08 &  7.10e-16 &      \\ 
     &           &    5 &  2.49e-08 &  1.51e-15 &      \\ 
     &           &    6 &  2.49e-08 &  7.12e-16 &      \\ 
     &           &    7 &  2.49e-08 &  7.11e-16 &      \\ 
     &           &    8 &  5.29e-08 &  7.10e-16 &      \\ 
     &           &    9 &  2.49e-08 &  1.51e-15 &      \\ 
     &           &   10 &  2.49e-08 &  7.11e-16 &      \\ 
 949 &  9.48e+03 &   10 &           &           & iters  \\ 
 \hdashline 
     &           &    1 &  5.91e-08 &  1.45e-08 &      \\ 
     &           &    2 &  1.87e-08 &  1.69e-15 &      \\ 
     &           &    3 &  1.87e-08 &  5.34e-16 &      \\ 
     &           &    4 &  1.86e-08 &  5.33e-16 &      \\ 
     &           &    5 &  1.86e-08 &  5.32e-16 &      \\ 
     &           &    6 &  5.91e-08 &  5.31e-16 &      \\ 
     &           &    7 &  1.87e-08 &  1.69e-15 &      \\ 
     &           &    8 &  1.87e-08 &  5.33e-16 &      \\ 
     &           &    9 &  1.86e-08 &  5.32e-16 &      \\ 
     &           &   10 &  5.91e-08 &  5.32e-16 &      \\ 
 950 &  9.49e+03 &   10 &           &           & iters  \\ 
 \hdashline 
     &           &    1 &  3.01e-08 &  1.07e-09 &      \\ 
     &           &    2 &  4.77e-08 &  8.58e-16 &      \\ 
     &           &    3 &  3.01e-08 &  1.36e-15 &      \\ 
     &           &    4 &  3.01e-08 &  8.59e-16 &      \\ 
     &           &    5 &  4.77e-08 &  8.58e-16 &      \\ 
     &           &    6 &  3.01e-08 &  1.36e-15 &      \\ 
     &           &    7 &  4.76e-08 &  8.59e-16 &      \\ 
     &           &    8 &  3.01e-08 &  1.36e-15 &      \\ 
     &           &    9 &  3.01e-08 &  8.59e-16 &      \\ 
     &           &   10 &  4.77e-08 &  8.58e-16 &      \\ 
 951 &  9.50e+03 &   10 &           &           & iters  \\ 
 \hdashline 
     &           &    1 &  2.61e-09 &  5.71e-09 &      \\ 
     &           &    2 &  2.61e-09 &  7.46e-17 &      \\ 
     &           &    3 &  2.61e-09 &  7.45e-17 &      \\ 
     &           &    4 &  2.60e-09 &  7.44e-17 &      \\ 
     &           &    5 &  2.60e-09 &  7.43e-17 &      \\ 
     &           &    6 &  2.60e-09 &  7.42e-17 &      \\ 
     &           &    7 &  2.59e-09 &  7.41e-17 &      \\ 
     &           &    8 &  2.59e-09 &  7.40e-17 &      \\ 
     &           &    9 &  2.58e-09 &  7.39e-17 &      \\ 
     &           &   10 &  7.51e-08 &  7.37e-17 &      \\ 
 952 &  9.51e+03 &   10 &           &           & iters  \\ 
 \hdashline 
     &           &    1 &  4.34e-08 &  3.39e-10 &      \\ 
     &           &    2 &  4.33e-08 &  1.24e-15 &      \\ 
     &           &    3 &  3.44e-08 &  1.24e-15 &      \\ 
     &           &    4 &  3.44e-08 &  9.82e-16 &      \\ 
     &           &    5 &  4.34e-08 &  9.81e-16 &      \\ 
     &           &    6 &  3.44e-08 &  1.24e-15 &      \\ 
     &           &    7 &  4.34e-08 &  9.81e-16 &      \\ 
     &           &    8 &  3.44e-08 &  1.24e-15 &      \\ 
     &           &    9 &  4.33e-08 &  9.81e-16 &      \\ 
     &           &   10 &  3.44e-08 &  1.24e-15 &      \\ 
 953 &  9.52e+03 &   10 &           &           & iters  \\ 
 \hdashline 
     &           &    1 &  1.33e-08 &  6.80e-09 &      \\ 
     &           &    2 &  6.44e-08 &  3.80e-16 &      \\ 
     &           &    3 &  1.34e-08 &  1.84e-15 &      \\ 
     &           &    4 &  1.34e-08 &  3.82e-16 &      \\ 
     &           &    5 &  1.33e-08 &  3.81e-16 &      \\ 
     &           &    6 &  1.33e-08 &  3.81e-16 &      \\ 
     &           &    7 &  1.33e-08 &  3.80e-16 &      \\ 
     &           &    8 &  6.44e-08 &  3.80e-16 &      \\ 
     &           &    9 &  1.34e-08 &  1.84e-15 &      \\ 
     &           &   10 &  1.34e-08 &  3.82e-16 &      \\ 
 954 &  9.53e+03 &   10 &           &           & iters  \\ 
 \hdashline 
     &           &    1 &  1.38e-08 &  1.06e-08 &      \\ 
     &           &    2 &  1.38e-08 &  3.94e-16 &      \\ 
     &           &    3 &  6.39e-08 &  3.93e-16 &      \\ 
     &           &    4 &  1.39e-08 &  1.82e-15 &      \\ 
     &           &    5 &  1.38e-08 &  3.95e-16 &      \\ 
     &           &    6 &  1.38e-08 &  3.95e-16 &      \\ 
     &           &    7 &  1.38e-08 &  3.94e-16 &      \\ 
     &           &    8 &  1.38e-08 &  3.94e-16 &      \\ 
     &           &    9 &  6.39e-08 &  3.93e-16 &      \\ 
     &           &   10 &  1.38e-08 &  1.82e-15 &      \\ 
 955 &  9.54e+03 &   10 &           &           & iters  \\ 
 \hdashline 
     &           &    1 &  3.06e-08 &  6.90e-09 &      \\ 
     &           &    2 &  4.71e-08 &  8.74e-16 &      \\ 
     &           &    3 &  3.06e-08 &  1.34e-15 &      \\ 
     &           &    4 &  3.06e-08 &  8.75e-16 &      \\ 
     &           &    5 &  4.71e-08 &  8.73e-16 &      \\ 
     &           &    6 &  3.06e-08 &  1.35e-15 &      \\ 
     &           &    7 &  4.71e-08 &  8.74e-16 &      \\ 
     &           &    8 &  3.06e-08 &  1.34e-15 &      \\ 
     &           &    9 &  3.06e-08 &  8.75e-16 &      \\ 
     &           &   10 &  4.71e-08 &  8.73e-16 &      \\ 
 956 &  9.55e+03 &   10 &           &           & iters  \\ 
 \hdashline 
     &           &    1 &  4.90e-08 &  1.19e-08 &      \\ 
     &           &    2 &  2.88e-08 &  1.40e-15 &      \\ 
     &           &    3 &  2.87e-08 &  8.21e-16 &      \\ 
     &           &    4 &  4.90e-08 &  8.20e-16 &      \\ 
     &           &    5 &  2.88e-08 &  1.40e-15 &      \\ 
     &           &    6 &  2.87e-08 &  8.21e-16 &      \\ 
     &           &    7 &  4.90e-08 &  8.20e-16 &      \\ 
     &           &    8 &  2.88e-08 &  1.40e-15 &      \\ 
     &           &    9 &  4.90e-08 &  8.21e-16 &      \\ 
     &           &   10 &  2.88e-08 &  1.40e-15 &      \\ 
 957 &  9.56e+03 &   10 &           &           & iters  \\ 
 \hdashline 
     &           &    1 &  6.57e-08 &  2.37e-08 &      \\ 
     &           &    2 &  1.21e-08 &  1.87e-15 &      \\ 
     &           &    3 &  1.21e-08 &  3.46e-16 &      \\ 
     &           &    4 &  1.21e-08 &  3.46e-16 &      \\ 
     &           &    5 &  1.21e-08 &  3.45e-16 &      \\ 
     &           &    6 &  1.21e-08 &  3.45e-16 &      \\ 
     &           &    7 &  6.56e-08 &  3.44e-16 &      \\ 
     &           &    8 &  1.21e-08 &  1.87e-15 &      \\ 
     &           &    9 &  1.21e-08 &  3.46e-16 &      \\ 
     &           &   10 &  1.21e-08 &  3.46e-16 &      \\ 
 958 &  9.57e+03 &   10 &           &           & iters  \\ 
 \hdashline 
     &           &    1 &  3.75e-08 &  1.43e-08 &      \\ 
     &           &    2 &  4.03e-08 &  1.07e-15 &      \\ 
     &           &    3 &  3.75e-08 &  1.15e-15 &      \\ 
     &           &    4 &  4.03e-08 &  1.07e-15 &      \\ 
     &           &    5 &  3.75e-08 &  1.15e-15 &      \\ 
     &           &    6 &  4.03e-08 &  1.07e-15 &      \\ 
     &           &    7 &  3.75e-08 &  1.15e-15 &      \\ 
     &           &    8 &  4.03e-08 &  1.07e-15 &      \\ 
     &           &    9 &  3.75e-08 &  1.15e-15 &      \\ 
     &           &   10 &  4.03e-08 &  1.07e-15 &      \\ 
 959 &  9.58e+03 &   10 &           &           & iters  \\ 
 \hdashline 
     &           &    1 &  1.61e-09 &  1.03e-08 &      \\ 
     &           &    2 &  1.61e-09 &  4.59e-17 &      \\ 
     &           &    3 &  1.60e-09 &  4.58e-17 &      \\ 
     &           &    4 &  1.60e-09 &  4.57e-17 &      \\ 
     &           &    5 &  1.60e-09 &  4.57e-17 &      \\ 
     &           &    6 &  1.60e-09 &  4.56e-17 &      \\ 
     &           &    7 &  1.59e-09 &  4.56e-17 &      \\ 
     &           &    8 &  1.59e-09 &  4.55e-17 &      \\ 
     &           &    9 &  1.59e-09 &  4.54e-17 &      \\ 
     &           &   10 &  1.59e-09 &  4.54e-17 &      \\ 
 960 &  9.59e+03 &   10 &           &           & iters  \\ 
 \hdashline 
     &           &    1 &  1.78e-08 &  1.69e-08 &      \\ 
     &           &    2 &  5.99e-08 &  5.08e-16 &      \\ 
     &           &    3 &  1.79e-08 &  1.71e-15 &      \\ 
     &           &    4 &  1.78e-08 &  5.10e-16 &      \\ 
     &           &    5 &  1.78e-08 &  5.09e-16 &      \\ 
     &           &    6 &  5.99e-08 &  5.08e-16 &      \\ 
     &           &    7 &  1.79e-08 &  1.71e-15 &      \\ 
     &           &    8 &  1.78e-08 &  5.10e-16 &      \\ 
     &           &    9 &  1.78e-08 &  5.09e-16 &      \\ 
     &           &   10 &  1.78e-08 &  5.08e-16 &      \\ 
 961 &  9.60e+03 &   10 &           &           & iters  \\ 
 \hdashline 
     &           &    1 &  4.04e-08 &  1.31e-08 &      \\ 
     &           &    2 &  3.73e-08 &  1.15e-15 &      \\ 
     &           &    3 &  4.04e-08 &  1.07e-15 &      \\ 
     &           &    4 &  3.73e-08 &  1.15e-15 &      \\ 
     &           &    5 &  4.04e-08 &  1.07e-15 &      \\ 
     &           &    6 &  3.74e-08 &  1.15e-15 &      \\ 
     &           &    7 &  4.04e-08 &  1.07e-15 &      \\ 
     &           &    8 &  3.74e-08 &  1.15e-15 &      \\ 
     &           &    9 &  4.04e-08 &  1.07e-15 &      \\ 
     &           &   10 &  3.74e-08 &  1.15e-15 &      \\ 
 962 &  9.61e+03 &   10 &           &           & iters  \\ 
 \hdashline 
     &           &    1 &  3.05e-08 &  3.37e-09 &      \\ 
     &           &    2 &  4.73e-08 &  8.70e-16 &      \\ 
     &           &    3 &  3.05e-08 &  1.35e-15 &      \\ 
     &           &    4 &  3.05e-08 &  8.70e-16 &      \\ 
     &           &    5 &  4.73e-08 &  8.69e-16 &      \\ 
     &           &    6 &  3.05e-08 &  1.35e-15 &      \\ 
     &           &    7 &  4.73e-08 &  8.70e-16 &      \\ 
     &           &    8 &  3.05e-08 &  1.35e-15 &      \\ 
     &           &    9 &  3.05e-08 &  8.70e-16 &      \\ 
     &           &   10 &  4.73e-08 &  8.69e-16 &      \\ 
 963 &  9.62e+03 &   10 &           &           & iters  \\ 
 \hdashline 
     &           &    1 &  1.09e-08 &  1.62e-10 &      \\ 
     &           &    2 &  1.08e-08 &  3.10e-16 &      \\ 
     &           &    3 &  1.08e-08 &  3.10e-16 &      \\ 
     &           &    4 &  1.08e-08 &  3.09e-16 &      \\ 
     &           &    5 &  1.08e-08 &  3.09e-16 &      \\ 
     &           &    6 &  1.08e-08 &  3.08e-16 &      \\ 
     &           &    7 &  6.69e-08 &  3.08e-16 &      \\ 
     &           &    8 &  1.09e-08 &  1.91e-15 &      \\ 
     &           &    9 &  1.09e-08 &  3.10e-16 &      \\ 
     &           &   10 &  1.08e-08 &  3.10e-16 &      \\ 
 964 &  9.63e+03 &   10 &           &           & iters  \\ 
 \hdashline 
     &           &    1 &  3.25e-08 &  4.64e-10 &      \\ 
     &           &    2 &  4.52e-08 &  9.27e-16 &      \\ 
     &           &    3 &  3.25e-08 &  1.29e-15 &      \\ 
     &           &    4 &  4.52e-08 &  9.28e-16 &      \\ 
     &           &    5 &  3.25e-08 &  1.29e-15 &      \\ 
     &           &    6 &  4.52e-08 &  9.28e-16 &      \\ 
     &           &    7 &  3.25e-08 &  1.29e-15 &      \\ 
     &           &    8 &  3.25e-08 &  9.29e-16 &      \\ 
     &           &    9 &  4.52e-08 &  9.28e-16 &      \\ 
     &           &   10 &  3.25e-08 &  1.29e-15 &      \\ 
 965 &  9.64e+03 &   10 &           &           & iters  \\ 
 \hdashline 
     &           &    1 &  3.21e-08 &  2.04e-10 &      \\ 
     &           &    2 &  6.76e-09 &  9.17e-16 &      \\ 
     &           &    3 &  6.75e-09 &  1.93e-16 &      \\ 
     &           &    4 &  6.74e-09 &  1.93e-16 &      \\ 
     &           &    5 &  6.73e-09 &  1.92e-16 &      \\ 
     &           &    6 &  3.21e-08 &  1.92e-16 &      \\ 
     &           &    7 &  6.77e-09 &  9.17e-16 &      \\ 
     &           &    8 &  6.76e-09 &  1.93e-16 &      \\ 
     &           &    9 &  6.75e-09 &  1.93e-16 &      \\ 
     &           &   10 &  6.74e-09 &  1.93e-16 &      \\ 
 966 &  9.65e+03 &   10 &           &           & iters  \\ 
 \hdashline 
     &           &    1 &  2.01e-08 &  1.96e-09 &      \\ 
     &           &    2 &  1.88e-08 &  5.73e-16 &      \\ 
     &           &    3 &  2.01e-08 &  5.37e-16 &      \\ 
     &           &    4 &  1.88e-08 &  5.73e-16 &      \\ 
     &           &    5 &  2.01e-08 &  5.37e-16 &      \\ 
     &           &    6 &  1.88e-08 &  5.73e-16 &      \\ 
     &           &    7 &  2.01e-08 &  5.37e-16 &      \\ 
     &           &    8 &  1.88e-08 &  5.73e-16 &      \\ 
     &           &    9 &  2.01e-08 &  5.37e-16 &      \\ 
     &           &   10 &  1.88e-08 &  5.73e-16 &      \\ 
 967 &  9.66e+03 &   10 &           &           & iters  \\ 
 \hdashline 
     &           &    1 &  1.98e-08 &  8.25e-09 &      \\ 
     &           &    2 &  5.80e-08 &  5.64e-16 &      \\ 
     &           &    3 &  1.98e-08 &  1.65e-15 &      \\ 
     &           &    4 &  1.98e-08 &  5.65e-16 &      \\ 
     &           &    5 &  1.98e-08 &  5.65e-16 &      \\ 
     &           &    6 &  5.80e-08 &  5.64e-16 &      \\ 
     &           &    7 &  1.98e-08 &  1.65e-15 &      \\ 
     &           &    8 &  1.98e-08 &  5.65e-16 &      \\ 
     &           &    9 &  1.98e-08 &  5.64e-16 &      \\ 
     &           &   10 &  5.80e-08 &  5.64e-16 &      \\ 
 968 &  9.67e+03 &   10 &           &           & iters  \\ 
 \hdashline 
     &           &    1 &  1.09e-08 &  1.44e-08 &      \\ 
     &           &    2 &  6.68e-08 &  3.12e-16 &      \\ 
     &           &    3 &  4.99e-08 &  1.91e-15 &      \\ 
     &           &    4 &  1.10e-08 &  1.42e-15 &      \\ 
     &           &    5 &  1.09e-08 &  3.13e-16 &      \\ 
     &           &    6 &  6.68e-08 &  3.12e-16 &      \\ 
     &           &    7 &  1.10e-08 &  1.91e-15 &      \\ 
     &           &    8 &  1.10e-08 &  3.14e-16 &      \\ 
     &           &    9 &  1.10e-08 &  3.14e-16 &      \\ 
     &           &   10 &  1.10e-08 &  3.13e-16 &      \\ 
 969 &  9.68e+03 &   10 &           &           & iters  \\ 
 \hdashline 
     &           &    1 &  2.20e-09 &  1.40e-08 &      \\ 
     &           &    2 &  2.19e-09 &  6.27e-17 &      \\ 
     &           &    3 &  3.67e-08 &  6.26e-17 &      \\ 
     &           &    4 &  2.24e-09 &  1.05e-15 &      \\ 
     &           &    5 &  2.24e-09 &  6.40e-17 &      \\ 
     &           &    6 &  2.24e-09 &  6.39e-17 &      \\ 
     &           &    7 &  2.23e-09 &  6.38e-17 &      \\ 
     &           &    8 &  2.23e-09 &  6.37e-17 &      \\ 
     &           &    9 &  2.23e-09 &  6.36e-17 &      \\ 
     &           &   10 &  2.22e-09 &  6.35e-17 &      \\ 
 970 &  9.69e+03 &   10 &           &           & iters  \\ 
 \hdashline 
     &           &    1 &  9.60e-09 &  1.28e-08 &      \\ 
     &           &    2 &  9.59e-09 &  2.74e-16 &      \\ 
     &           &    3 &  9.58e-09 &  2.74e-16 &      \\ 
     &           &    4 &  9.56e-09 &  2.73e-16 &      \\ 
     &           &    5 &  9.55e-09 &  2.73e-16 &      \\ 
     &           &    6 &  9.53e-09 &  2.73e-16 &      \\ 
     &           &    7 &  6.82e-08 &  2.72e-16 &      \\ 
     &           &    8 &  9.62e-09 &  1.95e-15 &      \\ 
     &           &    9 &  9.60e-09 &  2.75e-16 &      \\ 
     &           &   10 &  9.59e-09 &  2.74e-16 &      \\ 
 971 &  9.70e+03 &   10 &           &           & iters  \\ 
 \hdashline 
     &           &    1 &  5.11e-08 &  1.28e-08 &      \\ 
     &           &    2 &  2.67e-08 &  1.46e-15 &      \\ 
     &           &    3 &  5.11e-08 &  7.61e-16 &      \\ 
     &           &    4 &  2.67e-08 &  1.46e-15 &      \\ 
     &           &    5 &  2.67e-08 &  7.62e-16 &      \\ 
     &           &    6 &  5.11e-08 &  7.61e-16 &      \\ 
     &           &    7 &  2.67e-08 &  1.46e-15 &      \\ 
     &           &    8 &  2.67e-08 &  7.62e-16 &      \\ 
     &           &    9 &  5.11e-08 &  7.61e-16 &      \\ 
     &           &   10 &  2.67e-08 &  1.46e-15 &      \\ 
 972 &  9.71e+03 &   10 &           &           & iters  \\ 
 \hdashline 
     &           &    1 &  7.74e-08 &  8.29e-09 &      \\ 
     &           &    2 &  7.81e-08 &  2.21e-15 &      \\ 
     &           &    3 &  7.74e-08 &  2.23e-15 &      \\ 
     &           &    4 &  7.81e-08 &  2.21e-15 &      \\ 
     &           &    5 &  7.74e-08 &  2.23e-15 &      \\ 
     &           &    6 &  7.81e-08 &  2.21e-15 &      \\ 
     &           &    7 &  7.74e-08 &  2.23e-15 &      \\ 
     &           &    8 &  7.81e-08 &  2.21e-15 &      \\ 
     &           &    9 &  7.74e-08 &  2.23e-15 &      \\ 
     &           &   10 &  4.35e-10 &  2.21e-15 &      \\ 
 973 &  9.72e+03 &   10 &           &           & iters  \\ 
 \hdashline 
     &           &    1 &  4.82e-08 &  6.14e-09 &      \\ 
     &           &    2 &  2.96e-08 &  1.38e-15 &      \\ 
     &           &    3 &  2.95e-08 &  8.44e-16 &      \\ 
     &           &    4 &  4.82e-08 &  8.43e-16 &      \\ 
     &           &    5 &  2.96e-08 &  1.38e-15 &      \\ 
     &           &    6 &  4.82e-08 &  8.43e-16 &      \\ 
     &           &    7 &  2.96e-08 &  1.38e-15 &      \\ 
     &           &    8 &  2.95e-08 &  8.44e-16 &      \\ 
     &           &    9 &  4.82e-08 &  8.43e-16 &      \\ 
     &           &   10 &  2.96e-08 &  1.38e-15 &      \\ 
 974 &  9.73e+03 &   10 &           &           & iters  \\ 
 \hdashline 
     &           &    1 &  4.74e-09 &  6.52e-09 &      \\ 
     &           &    2 &  4.74e-09 &  1.35e-16 &      \\ 
     &           &    3 &  4.73e-09 &  1.35e-16 &      \\ 
     &           &    4 &  4.72e-09 &  1.35e-16 &      \\ 
     &           &    5 &  4.71e-09 &  1.35e-16 &      \\ 
     &           &    6 &  4.71e-09 &  1.35e-16 &      \\ 
     &           &    7 &  7.30e-08 &  1.34e-16 &      \\ 
     &           &    8 &  4.81e-09 &  2.08e-15 &      \\ 
     &           &    9 &  4.80e-09 &  1.37e-16 &      \\ 
     &           &   10 &  4.79e-09 &  1.37e-16 &      \\ 
 975 &  9.74e+03 &   10 &           &           & iters  \\ 
 \hdashline 
     &           &    1 &  2.84e-08 &  5.34e-09 &      \\ 
     &           &    2 &  4.93e-08 &  8.11e-16 &      \\ 
     &           &    3 &  2.84e-08 &  1.41e-15 &      \\ 
     &           &    4 &  2.84e-08 &  8.11e-16 &      \\ 
     &           &    5 &  4.93e-08 &  8.10e-16 &      \\ 
     &           &    6 &  2.84e-08 &  1.41e-15 &      \\ 
     &           &    7 &  2.84e-08 &  8.11e-16 &      \\ 
     &           &    8 &  4.94e-08 &  8.10e-16 &      \\ 
     &           &    9 &  2.84e-08 &  1.41e-15 &      \\ 
     &           &   10 &  4.93e-08 &  8.11e-16 &      \\ 
 976 &  9.75e+03 &   10 &           &           & iters  \\ 
 \hdashline 
     &           &    1 &  2.40e-11 &  3.81e-09 &      \\ 
 977 &  9.76e+03 &    1 &           &           & forces  \\ 
 \hdashline 
     &           &    1 &  5.38e-09 &  3.90e-10 &      \\ 
     &           &    2 &  5.38e-09 &  1.54e-16 &      \\ 
     &           &    3 &  5.37e-09 &  1.53e-16 &      \\ 
     &           &    4 &  5.36e-09 &  1.53e-16 &      \\ 
     &           &    5 &  5.35e-09 &  1.53e-16 &      \\ 
     &           &    6 &  5.34e-09 &  1.53e-16 &      \\ 
     &           &    7 &  7.24e-08 &  1.53e-16 &      \\ 
     &           &    8 &  5.44e-09 &  2.06e-15 &      \\ 
     &           &    9 &  5.43e-09 &  1.55e-16 &      \\ 
     &           &   10 &  5.42e-09 &  1.55e-16 &      \\ 
 978 &  9.77e+03 &   10 &           &           & iters  \\ 
 \hdashline 
     &           &    1 &  2.49e-08 &  9.71e-10 &      \\ 
     &           &    2 &  2.49e-08 &  7.11e-16 &      \\ 
     &           &    3 &  5.29e-08 &  7.10e-16 &      \\ 
     &           &    4 &  2.49e-08 &  1.51e-15 &      \\ 
     &           &    5 &  2.49e-08 &  7.11e-16 &      \\ 
     &           &    6 &  5.28e-08 &  7.10e-16 &      \\ 
     &           &    7 &  2.49e-08 &  1.51e-15 &      \\ 
     &           &    8 &  2.49e-08 &  7.11e-16 &      \\ 
     &           &    9 &  5.28e-08 &  7.10e-16 &      \\ 
     &           &   10 &  2.49e-08 &  1.51e-15 &      \\ 
 979 &  9.78e+03 &   10 &           &           & iters  \\ 
 \hdashline 
     &           &    1 &  7.48e-09 &  6.87e-09 &      \\ 
     &           &    2 &  7.47e-09 &  2.14e-16 &      \\ 
     &           &    3 &  7.46e-09 &  2.13e-16 &      \\ 
     &           &    4 &  7.45e-09 &  2.13e-16 &      \\ 
     &           &    5 &  7.44e-09 &  2.13e-16 &      \\ 
     &           &    6 &  7.43e-09 &  2.12e-16 &      \\ 
     &           &    7 &  7.42e-09 &  2.12e-16 &      \\ 
     &           &    8 &  7.03e-08 &  2.12e-16 &      \\ 
     &           &    9 &  8.52e-08 &  2.01e-15 &      \\ 
     &           &   10 &  7.03e-08 &  2.43e-15 &      \\ 
 980 &  9.79e+03 &   10 &           &           & iters  \\ 
 \hdashline 
     &           &    1 &  2.77e-08 &  3.57e-09 &      \\ 
     &           &    2 &  5.00e-08 &  7.90e-16 &      \\ 
     &           &    3 &  2.77e-08 &  1.43e-15 &      \\ 
     &           &    4 &  2.77e-08 &  7.91e-16 &      \\ 
     &           &    5 &  5.00e-08 &  7.90e-16 &      \\ 
     &           &    6 &  2.77e-08 &  1.43e-15 &      \\ 
     &           &    7 &  2.77e-08 &  7.91e-16 &      \\ 
     &           &    8 &  5.01e-08 &  7.90e-16 &      \\ 
     &           &    9 &  2.77e-08 &  1.43e-15 &      \\ 
     &           &   10 &  5.00e-08 &  7.91e-16 &      \\ 
 981 &  9.80e+03 &   10 &           &           & iters  \\ 
 \hdashline 
     &           &    1 &  3.46e-08 &  1.31e-08 &      \\ 
     &           &    2 &  3.45e-08 &  9.87e-16 &      \\ 
     &           &    3 &  4.32e-08 &  9.86e-16 &      \\ 
     &           &    4 &  3.45e-08 &  1.23e-15 &      \\ 
     &           &    5 &  4.32e-08 &  9.86e-16 &      \\ 
     &           &    6 &  3.46e-08 &  1.23e-15 &      \\ 
     &           &    7 &  3.45e-08 &  9.86e-16 &      \\ 
     &           &    8 &  4.32e-08 &  9.85e-16 &      \\ 
     &           &    9 &  3.45e-08 &  1.23e-15 &      \\ 
     &           &   10 &  4.32e-08 &  9.85e-16 &      \\ 
 982 &  9.81e+03 &   10 &           &           & iters  \\ 
 \hdashline 
     &           &    1 &  2.07e-08 &  2.58e-08 &      \\ 
     &           &    2 &  2.07e-08 &  5.91e-16 &      \\ 
     &           &    3 &  5.70e-08 &  5.90e-16 &      \\ 
     &           &    4 &  2.07e-08 &  1.63e-15 &      \\ 
     &           &    5 &  2.07e-08 &  5.92e-16 &      \\ 
     &           &    6 &  2.07e-08 &  5.91e-16 &      \\ 
     &           &    7 &  5.70e-08 &  5.90e-16 &      \\ 
     &           &    8 &  2.07e-08 &  1.63e-15 &      \\ 
     &           &    9 &  2.07e-08 &  5.92e-16 &      \\ 
     &           &   10 &  5.70e-08 &  5.91e-16 &      \\ 
 983 &  9.82e+03 &   10 &           &           & iters  \\ 
 \hdashline 
     &           &    1 &  5.91e-08 &  2.29e-08 &      \\ 
     &           &    2 &  2.02e-08 &  1.69e-15 &      \\ 
     &           &    3 &  5.75e-08 &  5.76e-16 &      \\ 
     &           &    4 &  2.02e-08 &  1.64e-15 &      \\ 
     &           &    5 &  2.02e-08 &  5.78e-16 &      \\ 
     &           &    6 &  2.02e-08 &  5.77e-16 &      \\ 
     &           &    7 &  5.75e-08 &  5.76e-16 &      \\ 
     &           &    8 &  2.02e-08 &  1.64e-15 &      \\ 
     &           &    9 &  2.02e-08 &  5.78e-16 &      \\ 
     &           &   10 &  2.02e-08 &  5.77e-16 &      \\ 
 984 &  9.83e+03 &   10 &           &           & iters  \\ 
 \hdashline 
     &           &    1 &  6.66e-08 &  1.68e-08 &      \\ 
     &           &    2 &  1.12e-08 &  1.90e-15 &      \\ 
     &           &    3 &  1.11e-08 &  3.19e-16 &      \\ 
     &           &    4 &  1.11e-08 &  3.18e-16 &      \\ 
     &           &    5 &  1.11e-08 &  3.18e-16 &      \\ 
     &           &    6 &  1.11e-08 &  3.17e-16 &      \\ 
     &           &    7 &  6.66e-08 &  3.17e-16 &      \\ 
     &           &    8 &  1.12e-08 &  1.90e-15 &      \\ 
     &           &    9 &  1.12e-08 &  3.19e-16 &      \\ 
     &           &   10 &  1.11e-08 &  3.19e-16 &      \\ 
 985 &  9.84e+03 &   10 &           &           & iters  \\ 
 \hdashline 
     &           &    1 &  8.72e-09 &  2.31e-08 &      \\ 
     &           &    2 &  8.71e-09 &  2.49e-16 &      \\ 
     &           &    3 &  8.69e-09 &  2.48e-16 &      \\ 
     &           &    4 &  8.68e-09 &  2.48e-16 &      \\ 
     &           &    5 &  8.67e-09 &  2.48e-16 &      \\ 
     &           &    6 &  6.90e-08 &  2.47e-16 &      \\ 
     &           &    7 &  4.76e-08 &  1.97e-15 &      \\ 
     &           &    8 &  8.69e-09 &  1.36e-15 &      \\ 
     &           &    9 &  8.67e-09 &  2.48e-16 &      \\ 
     &           &   10 &  8.66e-09 &  2.48e-16 &      \\ 
 986 &  9.85e+03 &   10 &           &           & iters  \\ 
 \hdashline 
     &           &    1 &  3.58e-08 &  3.37e-08 &      \\ 
     &           &    2 &  3.58e-08 &  1.02e-15 &      \\ 
     &           &    3 &  4.19e-08 &  1.02e-15 &      \\ 
     &           &    4 &  3.58e-08 &  1.20e-15 &      \\ 
     &           &    5 &  4.19e-08 &  1.02e-15 &      \\ 
     &           &    6 &  3.58e-08 &  1.20e-15 &      \\ 
     &           &    7 &  4.19e-08 &  1.02e-15 &      \\ 
     &           &    8 &  3.58e-08 &  1.20e-15 &      \\ 
     &           &    9 &  4.19e-08 &  1.02e-15 &      \\ 
     &           &   10 &  3.58e-08 &  1.20e-15 &      \\ 
 987 &  9.86e+03 &   10 &           &           & iters  \\ 
 \hdashline 
     &           &    1 &  3.28e-08 &  1.91e-08 &      \\ 
     &           &    2 &  4.49e-08 &  9.37e-16 &      \\ 
     &           &    3 &  3.28e-08 &  1.28e-15 &      \\ 
     &           &    4 &  4.49e-08 &  9.37e-16 &      \\ 
     &           &    5 &  3.29e-08 &  1.28e-15 &      \\ 
     &           &    6 &  3.28e-08 &  9.38e-16 &      \\ 
     &           &    7 &  4.49e-08 &  9.37e-16 &      \\ 
     &           &    8 &  3.28e-08 &  1.28e-15 &      \\ 
     &           &    9 &  4.49e-08 &  9.37e-16 &      \\ 
     &           &   10 &  3.29e-08 &  1.28e-15 &      \\ 
 988 &  9.87e+03 &   10 &           &           & iters  \\ 
 \hdashline 
     &           &    1 &  3.03e-08 &  1.44e-08 &      \\ 
     &           &    2 &  4.74e-08 &  8.65e-16 &      \\ 
     &           &    3 &  3.03e-08 &  1.35e-15 &      \\ 
     &           &    4 &  4.74e-08 &  8.65e-16 &      \\ 
     &           &    5 &  3.03e-08 &  1.35e-15 &      \\ 
     &           &    6 &  3.03e-08 &  8.66e-16 &      \\ 
     &           &    7 &  4.74e-08 &  8.65e-16 &      \\ 
     &           &    8 &  3.03e-08 &  1.35e-15 &      \\ 
     &           &    9 &  4.74e-08 &  8.65e-16 &      \\ 
     &           &   10 &  3.03e-08 &  1.35e-15 &      \\ 
 989 &  9.88e+03 &   10 &           &           & iters  \\ 
 \hdashline 
     &           &    1 &  2.43e-08 &  2.38e-08 &      \\ 
     &           &    2 &  2.42e-08 &  6.93e-16 &      \\ 
     &           &    3 &  5.35e-08 &  6.92e-16 &      \\ 
     &           &    4 &  2.43e-08 &  1.53e-15 &      \\ 
     &           &    5 &  2.42e-08 &  6.93e-16 &      \\ 
     &           &    6 &  5.35e-08 &  6.92e-16 &      \\ 
     &           &    7 &  2.43e-08 &  1.53e-15 &      \\ 
     &           &    8 &  2.42e-08 &  6.93e-16 &      \\ 
     &           &    9 &  5.35e-08 &  6.92e-16 &      \\ 
     &           &   10 &  2.43e-08 &  1.53e-15 &      \\ 
 990 &  9.89e+03 &   10 &           &           & iters  \\ 
 \hdashline 
     &           &    1 &  4.60e-08 &  1.67e-08 &      \\ 
     &           &    2 &  3.17e-08 &  1.31e-15 &      \\ 
     &           &    3 &  3.17e-08 &  9.05e-16 &      \\ 
     &           &    4 &  4.61e-08 &  9.04e-16 &      \\ 
     &           &    5 &  3.17e-08 &  1.31e-15 &      \\ 
     &           &    6 &  4.60e-08 &  9.04e-16 &      \\ 
     &           &    7 &  3.17e-08 &  1.31e-15 &      \\ 
     &           &    8 &  3.17e-08 &  9.05e-16 &      \\ 
     &           &    9 &  4.61e-08 &  9.04e-16 &      \\ 
     &           &   10 &  3.17e-08 &  1.31e-15 &      \\ 
 991 &  9.90e+03 &   10 &           &           & iters  \\ 
 \hdashline 
     &           &    1 &  3.71e-08 &  1.52e-08 &      \\ 
     &           &    2 &  4.07e-08 &  1.06e-15 &      \\ 
     &           &    3 &  3.71e-08 &  1.16e-15 &      \\ 
     &           &    4 &  4.06e-08 &  1.06e-15 &      \\ 
     &           &    5 &  3.71e-08 &  1.16e-15 &      \\ 
     &           &    6 &  4.06e-08 &  1.06e-15 &      \\ 
     &           &    7 &  3.71e-08 &  1.16e-15 &      \\ 
     &           &    8 &  4.06e-08 &  1.06e-15 &      \\ 
     &           &    9 &  3.71e-08 &  1.16e-15 &      \\ 
     &           &   10 &  4.06e-08 &  1.06e-15 &      \\ 
 992 &  9.91e+03 &   10 &           &           & iters  \\ 
 \hdashline 
     &           &    1 &  2.68e-08 &  1.10e-08 &      \\ 
     &           &    2 &  5.09e-08 &  7.66e-16 &      \\ 
     &           &    3 &  2.69e-08 &  1.45e-15 &      \\ 
     &           &    4 &  2.68e-08 &  7.67e-16 &      \\ 
     &           &    5 &  5.09e-08 &  7.65e-16 &      \\ 
     &           &    6 &  2.69e-08 &  1.45e-15 &      \\ 
     &           &    7 &  2.68e-08 &  7.66e-16 &      \\ 
     &           &    8 &  5.09e-08 &  7.65e-16 &      \\ 
     &           &    9 &  2.69e-08 &  1.45e-15 &      \\ 
     &           &   10 &  2.68e-08 &  7.66e-16 &      \\ 
 993 &  9.92e+03 &   10 &           &           & iters  \\ 
 \hdashline 
     &           &    1 &  2.10e-08 &  8.37e-09 &      \\ 
     &           &    2 &  2.10e-08 &  5.99e-16 &      \\ 
     &           &    3 &  5.68e-08 &  5.98e-16 &      \\ 
     &           &    4 &  2.10e-08 &  1.62e-15 &      \\ 
     &           &    5 &  2.10e-08 &  6.00e-16 &      \\ 
     &           &    6 &  2.10e-08 &  5.99e-16 &      \\ 
     &           &    7 &  5.68e-08 &  5.98e-16 &      \\ 
     &           &    8 &  2.10e-08 &  1.62e-15 &      \\ 
     &           &    9 &  2.10e-08 &  6.00e-16 &      \\ 
     &           &   10 &  2.09e-08 &  5.99e-16 &      \\ 
 994 &  9.93e+03 &   10 &           &           & iters  \\ 
 \hdashline 
     &           &    1 &  2.88e-08 &  1.60e-08 &      \\ 
     &           &    2 &  2.87e-08 &  8.22e-16 &      \\ 
     &           &    3 &  4.90e-08 &  8.20e-16 &      \\ 
     &           &    4 &  2.88e-08 &  1.40e-15 &      \\ 
     &           &    5 &  4.90e-08 &  8.21e-16 &      \\ 
     &           &    6 &  2.88e-08 &  1.40e-15 &      \\ 
     &           &    7 &  2.88e-08 &  8.22e-16 &      \\ 
     &           &    8 &  4.90e-08 &  8.21e-16 &      \\ 
     &           &    9 &  2.88e-08 &  1.40e-15 &      \\ 
     &           &   10 &  2.87e-08 &  8.22e-16 &      \\ 
 995 &  9.94e+03 &   10 &           &           & iters  \\ 
 \hdashline 
     &           &    1 &  1.51e-08 &  2.43e-08 &      \\ 
     &           &    2 &  6.26e-08 &  4.30e-16 &      \\ 
     &           &    3 &  1.51e-08 &  1.79e-15 &      \\ 
     &           &    4 &  1.51e-08 &  4.32e-16 &      \\ 
     &           &    5 &  1.51e-08 &  4.31e-16 &      \\ 
     &           &    6 &  1.51e-08 &  4.31e-16 &      \\ 
     &           &    7 &  6.26e-08 &  4.30e-16 &      \\ 
     &           &    8 &  1.51e-08 &  1.79e-15 &      \\ 
     &           &    9 &  1.51e-08 &  4.32e-16 &      \\ 
     &           &   10 &  1.51e-08 &  4.31e-16 &      \\ 
 996 &  9.95e+03 &   10 &           &           & iters  \\ 
 \hdashline 
     &           &    1 &  1.47e-08 &  1.71e-08 &      \\ 
     &           &    2 &  1.47e-08 &  4.20e-16 &      \\ 
     &           &    3 &  1.47e-08 &  4.20e-16 &      \\ 
     &           &    4 &  1.47e-08 &  4.19e-16 &      \\ 
     &           &    5 &  1.46e-08 &  4.19e-16 &      \\ 
     &           &    6 &  6.31e-08 &  4.18e-16 &      \\ 
     &           &    7 &  1.47e-08 &  1.80e-15 &      \\ 
     &           &    8 &  1.47e-08 &  4.20e-16 &      \\ 
     &           &    9 &  1.47e-08 &  4.19e-16 &      \\ 
     &           &   10 &  1.47e-08 &  4.19e-16 &      \\ 
 997 &  9.96e+03 &   10 &           &           & iters  \\ 
 \hdashline 
     &           &    1 &  9.22e-08 &  4.20e-09 &      \\ 
     &           &    2 &  6.34e-08 &  2.63e-15 &      \\ 
     &           &    3 &  1.44e-08 &  1.81e-15 &      \\ 
     &           &    4 &  1.44e-08 &  4.12e-16 &      \\ 
     &           &    5 &  1.44e-08 &  4.11e-16 &      \\ 
     &           &    6 &  6.33e-08 &  4.11e-16 &      \\ 
     &           &    7 &  1.45e-08 &  1.81e-15 &      \\ 
     &           &    8 &  1.44e-08 &  4.13e-16 &      \\ 
     &           &    9 &  1.44e-08 &  4.12e-16 &      \\ 
     &           &   10 &  1.44e-08 &  4.11e-16 &      \\ 
 998 &  9.97e+03 &   10 &           &           & iters  \\ 
 \hdashline 
     &           &    1 &  3.08e-08 &  1.53e-09 &      \\ 
     &           &    2 &  4.69e-08 &  8.79e-16 &      \\ 
     &           &    3 &  3.08e-08 &  1.34e-15 &      \\ 
     &           &    4 &  4.69e-08 &  8.79e-16 &      \\ 
     &           &    5 &  3.08e-08 &  1.34e-15 &      \\ 
     &           &    6 &  3.08e-08 &  8.80e-16 &      \\ 
     &           &    7 &  4.69e-08 &  8.79e-16 &      \\ 
     &           &    8 &  3.08e-08 &  1.34e-15 &      \\ 
     &           &    9 &  4.69e-08 &  8.79e-16 &      \\ 
     &           &   10 &  3.08e-08 &  1.34e-15 &      \\ 
 999 &  9.98e+03 &   10 &           &           & iters  \\ 
 \hdashline 
     &           &    1 &  1.28e-08 &  2.55e-09 &      \\ 
     &           &    2 &  1.27e-08 &  3.64e-16 &      \\ 
     &           &    3 &  1.27e-08 &  3.64e-16 &      \\ 
     &           &    4 &  1.27e-08 &  3.63e-16 &      \\ 
     &           &    5 &  1.27e-08 &  3.63e-16 &      \\ 
     &           &    6 &  1.27e-08 &  3.62e-16 &      \\ 
     &           &    7 &  6.50e-08 &  3.61e-16 &      \\ 
     &           &    8 &  1.27e-08 &  1.86e-15 &      \\ 
     &           &    9 &  1.27e-08 &  3.64e-16 &      \\ 
     &           &   10 &  1.27e-08 &  3.63e-16 &      \\ 
1000 &  9.99e+03 &   10 &           &           & iters  \\ 
 \hdashline 
     &           &    1 &  3.83e-08 &  5.85e-09 &      \\ 
     &           &    2 &  3.83e-08 &  1.09e-15 &      \\ 
     &           &    3 &  3.95e-08 &  1.09e-15 &      \\ 
     &           &    4 &  3.83e-08 &  1.13e-15 &      \\ 
     &           &    5 &  3.95e-08 &  1.09e-15 &      \\ 
     &           &    6 &  3.83e-08 &  1.13e-15 &      \\ 
     &           &    7 &  3.95e-08 &  1.09e-15 &      \\ 
     &           &    8 &  3.83e-08 &  1.13e-15 &      \\ 
     &           &    9 &  3.95e-08 &  1.09e-15 &      \\ 
     &           &   10 &  3.83e-08 &  1.13e-15 &      \\ 
1001 &  1.00e+04 &   10 &           &           & iters  \\ 
 \hdashline 
     &           &    1 &  9.35e-10 &  1.68e-08 &      \\ 
     &           &    2 &  9.34e-10 &  2.67e-17 &      \\ 
     &           &    3 &  9.32e-10 &  2.66e-17 &      \\ 
     &           &    4 &  9.31e-10 &  2.66e-17 &      \\ 
     &           &    5 &  9.30e-10 &  2.66e-17 &      \\ 
     &           &    6 &  9.28e-10 &  2.65e-17 &      \\ 
     &           &    7 &  9.27e-10 &  2.65e-17 &      \\ 
     &           &    8 &  9.26e-10 &  2.65e-17 &      \\ 
     &           &    9 &  9.24e-10 &  2.64e-17 &      \\ 
     &           &   10 &  9.23e-10 &  2.64e-17 &      \\ 
1002 &  1.00e+04 &   10 &           &           & iters  \\ 
 \hdashline 
     &           &    1 &  1.18e-08 &  1.56e-08 &      \\ 
     &           &    2 &  1.17e-08 &  3.36e-16 &      \\ 
     &           &    3 &  1.17e-08 &  3.35e-16 &      \\ 
     &           &    4 &  1.17e-08 &  3.35e-16 &      \\ 
     &           &    5 &  1.17e-08 &  3.34e-16 &      \\ 
     &           &    6 &  1.17e-08 &  3.34e-16 &      \\ 
     &           &    7 &  1.17e-08 &  3.33e-16 &      \\ 
     &           &    8 &  1.16e-08 &  3.33e-16 &      \\ 
     &           &    9 &  6.61e-08 &  3.32e-16 &      \\ 
     &           &   10 &  1.17e-08 &  1.89e-15 &      \\ 
1003 &  1.00e+04 &   10 &           &           & iters  \\ 
 \hdashline 
     &           &    1 &  4.18e-09 &  1.61e-09 &      \\ 
     &           &    2 &  4.18e-09 &  1.19e-16 &      \\ 
     &           &    3 &  4.17e-09 &  1.19e-16 &      \\ 
     &           &    4 &  4.16e-09 &  1.19e-16 &      \\ 
     &           &    5 &  4.16e-09 &  1.19e-16 &      \\ 
     &           &    6 &  4.15e-09 &  1.19e-16 &      \\ 
     &           &    7 &  4.15e-09 &  1.18e-16 &      \\ 
     &           &    8 &  4.14e-09 &  1.18e-16 &      \\ 
     &           &    9 &  4.13e-09 &  1.18e-16 &      \\ 
     &           &   10 &  4.13e-09 &  1.18e-16 &      \\ 
1004 &  1.00e+04 &   10 &           &           & iters  \\ 
 \hdashline 
     &           &    1 &  2.85e-08 &  8.08e-11 &      \\ 
     &           &    2 &  4.92e-08 &  8.14e-16 &      \\ 
     &           &    3 &  2.85e-08 &  1.40e-15 &      \\ 
     &           &    4 &  2.85e-08 &  8.15e-16 &      \\ 
     &           &    5 &  4.92e-08 &  8.14e-16 &      \\ 
     &           &    6 &  2.85e-08 &  1.40e-15 &      \\ 
     &           &    7 &  2.85e-08 &  8.14e-16 &      \\ 
     &           &    8 &  4.92e-08 &  8.13e-16 &      \\ 
     &           &    9 &  2.85e-08 &  1.41e-15 &      \\ 
     &           &   10 &  2.85e-08 &  8.14e-16 &      \\ 
1005 &  1.00e+04 &   10 &           &           & iters  \\ 
 \hdashline 
     &           &    1 &  1.38e-08 &  6.98e-09 &      \\ 
     &           &    2 &  1.38e-08 &  3.95e-16 &      \\ 
     &           &    3 &  6.39e-08 &  3.95e-16 &      \\ 
     &           &    4 &  1.39e-08 &  1.82e-15 &      \\ 
     &           &    5 &  1.39e-08 &  3.97e-16 &      \\ 
     &           &    6 &  1.39e-08 &  3.96e-16 &      \\ 
     &           &    7 &  1.38e-08 &  3.95e-16 &      \\ 
     &           &    8 &  6.39e-08 &  3.95e-16 &      \\ 
     &           &    9 &  1.39e-08 &  1.82e-15 &      \\ 
     &           &   10 &  1.39e-08 &  3.97e-16 &      \\ 
1006 &  1.00e+04 &   10 &           &           & iters  \\ 
 \hdashline 
     &           &    1 &  5.08e-09 &  1.01e-08 &      \\ 
     &           &    2 &  5.07e-09 &  1.45e-16 &      \\ 
     &           &    3 &  5.06e-09 &  1.45e-16 &      \\ 
     &           &    4 &  5.05e-09 &  1.44e-16 &      \\ 
     &           &    5 &  5.05e-09 &  1.44e-16 &      \\ 
     &           &    6 &  5.04e-09 &  1.44e-16 &      \\ 
     &           &    7 &  5.03e-09 &  1.44e-16 &      \\ 
     &           &    8 &  5.03e-09 &  1.44e-16 &      \\ 
     &           &    9 &  5.02e-09 &  1.43e-16 &      \\ 
     &           &   10 &  3.38e-08 &  1.43e-16 &      \\ 
1007 &  1.01e+04 &   10 &           &           & iters  \\ 
 \hdashline 
     &           &    1 &  4.83e-08 &  7.31e-09 &      \\ 
     &           &    2 &  2.95e-08 &  1.38e-15 &      \\ 
     &           &    3 &  2.94e-08 &  8.41e-16 &      \\ 
     &           &    4 &  4.83e-08 &  8.40e-16 &      \\ 
     &           &    5 &  2.95e-08 &  1.38e-15 &      \\ 
     &           &    6 &  2.94e-08 &  8.41e-16 &      \\ 
     &           &    7 &  4.83e-08 &  8.40e-16 &      \\ 
     &           &    8 &  2.94e-08 &  1.38e-15 &      \\ 
     &           &    9 &  4.83e-08 &  8.40e-16 &      \\ 
     &           &   10 &  2.95e-08 &  1.38e-15 &      \\ 
1008 &  1.01e+04 &   10 &           &           & iters  \\ 
 \hdashline 
     &           &    1 &  1.81e-08 &  1.36e-09 &      \\ 
     &           &    2 &  5.97e-08 &  5.15e-16 &      \\ 
     &           &    3 &  1.81e-08 &  1.70e-15 &      \\ 
     &           &    4 &  1.81e-08 &  5.17e-16 &      \\ 
     &           &    5 &  1.81e-08 &  5.16e-16 &      \\ 
     &           &    6 &  5.97e-08 &  5.15e-16 &      \\ 
     &           &    7 &  1.81e-08 &  1.70e-15 &      \\ 
     &           &    8 &  1.81e-08 &  5.17e-16 &      \\ 
     &           &    9 &  1.81e-08 &  5.16e-16 &      \\ 
     &           &   10 &  1.80e-08 &  5.16e-16 &      \\ 
1009 &  1.01e+04 &   10 &           &           & iters  \\ 
 \hdashline 
     &           &    1 &  3.79e-09 &  1.06e-09 &      \\ 
     &           &    2 &  3.78e-09 &  1.08e-16 &      \\ 
     &           &    3 &  3.78e-09 &  1.08e-16 &      \\ 
     &           &    4 &  3.77e-09 &  1.08e-16 &      \\ 
     &           &    5 &  3.77e-09 &  1.08e-16 &      \\ 
     &           &    6 &  3.76e-09 &  1.07e-16 &      \\ 
     &           &    7 &  3.76e-09 &  1.07e-16 &      \\ 
     &           &    8 &  3.75e-09 &  1.07e-16 &      \\ 
     &           &    9 &  3.75e-09 &  1.07e-16 &      \\ 
     &           &   10 &  3.74e-09 &  1.07e-16 &      \\ 
1010 &  1.01e+04 &   10 &           &           & iters  \\ 
 \hdashline 
     &           &    1 &  4.00e-09 &  5.34e-09 &      \\ 
     &           &    2 &  3.99e-09 &  1.14e-16 &      \\ 
     &           &    3 &  3.99e-09 &  1.14e-16 &      \\ 
     &           &    4 &  3.98e-09 &  1.14e-16 &      \\ 
     &           &    5 &  3.98e-09 &  1.14e-16 &      \\ 
     &           &    6 &  3.97e-09 &  1.13e-16 &      \\ 
     &           &    7 &  3.96e-09 &  1.13e-16 &      \\ 
     &           &    8 &  3.96e-09 &  1.13e-16 &      \\ 
     &           &    9 &  3.49e-08 &  1.13e-16 &      \\ 
     &           &   10 &  4.00e-09 &  9.96e-16 &      \\ 
1011 &  1.01e+04 &   10 &           &           & iters  \\ 
 \hdashline 
     &           &    1 &  1.48e-08 &  3.01e-09 &      \\ 
     &           &    2 &  1.48e-08 &  4.22e-16 &      \\ 
     &           &    3 &  6.29e-08 &  4.22e-16 &      \\ 
     &           &    4 &  1.48e-08 &  1.80e-15 &      \\ 
     &           &    5 &  1.48e-08 &  4.24e-16 &      \\ 
     &           &    6 &  1.48e-08 &  4.23e-16 &      \\ 
     &           &    7 &  1.48e-08 &  4.22e-16 &      \\ 
     &           &    8 &  6.29e-08 &  4.22e-16 &      \\ 
     &           &    9 &  1.48e-08 &  1.80e-15 &      \\ 
     &           &   10 &  1.48e-08 &  4.24e-16 &      \\ 
1012 &  1.01e+04 &   10 &           &           & iters  \\ 
 \hdashline 
     &           &    1 &  2.16e-09 &  1.22e-08 &      \\ 
     &           &    2 &  2.16e-09 &  6.17e-17 &      \\ 
     &           &    3 &  2.16e-09 &  6.16e-17 &      \\ 
     &           &    4 &  2.15e-09 &  6.15e-17 &      \\ 
     &           &    5 &  2.15e-09 &  6.14e-17 &      \\ 
     &           &    6 &  2.15e-09 &  6.13e-17 &      \\ 
     &           &    7 &  2.14e-09 &  6.13e-17 &      \\ 
     &           &    8 &  2.14e-09 &  6.12e-17 &      \\ 
     &           &    9 &  2.14e-09 &  6.11e-17 &      \\ 
     &           &   10 &  2.13e-09 &  6.10e-17 &      \\ 
1013 &  1.01e+04 &   10 &           &           & iters  \\ 
 \hdashline 
     &           &    1 &  4.53e-08 &  4.50e-09 &      \\ 
     &           &    2 &  3.24e-08 &  1.29e-15 &      \\ 
     &           &    3 &  3.24e-08 &  9.25e-16 &      \\ 
     &           &    4 &  4.54e-08 &  9.24e-16 &      \\ 
     &           &    5 &  3.24e-08 &  1.29e-15 &      \\ 
     &           &    6 &  4.54e-08 &  9.24e-16 &      \\ 
     &           &    7 &  3.24e-08 &  1.29e-15 &      \\ 
     &           &    8 &  4.53e-08 &  9.25e-16 &      \\ 
     &           &    9 &  3.24e-08 &  1.29e-15 &      \\ 
     &           &   10 &  3.24e-08 &  9.25e-16 &      \\ 
1014 &  1.01e+04 &   10 &           &           & iters  \\ 
 \hdashline 
     &           &    1 &  1.56e-08 &  1.05e-08 &      \\ 
     &           &    2 &  1.56e-08 &  4.45e-16 &      \\ 
     &           &    3 &  1.55e-08 &  4.44e-16 &      \\ 
     &           &    4 &  1.55e-08 &  4.44e-16 &      \\ 
     &           &    5 &  6.22e-08 &  4.43e-16 &      \\ 
     &           &    6 &  1.56e-08 &  1.77e-15 &      \\ 
     &           &    7 &  1.56e-08 &  4.45e-16 &      \\ 
     &           &    8 &  1.55e-08 &  4.44e-16 &      \\ 
     &           &    9 &  1.55e-08 &  4.44e-16 &      \\ 
     &           &   10 &  6.22e-08 &  4.43e-16 &      \\ 
1015 &  1.01e+04 &   10 &           &           & iters  \\ 
 \hdashline 
     &           &    1 &  3.69e-08 &  2.17e-08 &      \\ 
     &           &    2 &  3.68e-08 &  1.05e-15 &      \\ 
     &           &    3 &  4.09e-08 &  1.05e-15 &      \\ 
     &           &    4 &  3.68e-08 &  1.17e-15 &      \\ 
     &           &    5 &  4.09e-08 &  1.05e-15 &      \\ 
     &           &    6 &  3.68e-08 &  1.17e-15 &      \\ 
     &           &    7 &  4.09e-08 &  1.05e-15 &      \\ 
     &           &    8 &  3.68e-08 &  1.17e-15 &      \\ 
     &           &    9 &  4.09e-08 &  1.05e-15 &      \\ 
     &           &   10 &  3.68e-08 &  1.17e-15 &      \\ 
1016 &  1.02e+04 &   10 &           &           & iters  \\ 
 \hdashline 
     &           &    1 &  2.59e-08 &  4.99e-09 &      \\ 
     &           &    2 &  5.18e-08 &  7.39e-16 &      \\ 
     &           &    3 &  2.59e-08 &  1.48e-15 &      \\ 
     &           &    4 &  2.59e-08 &  7.40e-16 &      \\ 
     &           &    5 &  5.18e-08 &  7.39e-16 &      \\ 
     &           &    6 &  2.59e-08 &  1.48e-15 &      \\ 
     &           &    7 &  2.59e-08 &  7.40e-16 &      \\ 
     &           &    8 &  5.18e-08 &  7.39e-16 &      \\ 
     &           &    9 &  2.59e-08 &  1.48e-15 &      \\ 
     &           &   10 &  2.59e-08 &  7.40e-16 &      \\ 
1017 &  1.02e+04 &   10 &           &           & iters  \\ 
 \hdashline 
     &           &    1 &  3.85e-08 &  8.21e-09 &      \\ 
     &           &    2 &  3.85e-08 &  1.10e-15 &      \\ 
     &           &    3 &  3.93e-08 &  1.10e-15 &      \\ 
     &           &    4 &  3.85e-08 &  1.12e-15 &      \\ 
     &           &    5 &  3.93e-08 &  1.10e-15 &      \\ 
     &           &    6 &  3.85e-08 &  1.12e-15 &      \\ 
     &           &    7 &  3.93e-08 &  1.10e-15 &      \\ 
     &           &    8 &  3.85e-08 &  1.12e-15 &      \\ 
     &           &    9 &  3.93e-08 &  1.10e-15 &      \\ 
     &           &   10 &  3.85e-08 &  1.12e-15 &      \\ 
1018 &  1.02e+04 &   10 &           &           & iters  \\ 
 \hdashline 
     &           &    1 &  1.09e-08 &  1.64e-11 &      \\ 
     &           &    2 &  1.09e-08 &  3.12e-16 &      \\ 
     &           &    3 &  1.09e-08 &  3.11e-16 &      \\ 
     &           &    4 &  6.68e-08 &  3.11e-16 &      \\ 
     &           &    5 &  8.87e-08 &  1.91e-15 &      \\ 
     &           &    6 &  6.69e-08 &  2.53e-15 &      \\ 
     &           &    7 &  1.09e-08 &  1.91e-15 &      \\ 
     &           &    8 &  1.09e-08 &  3.12e-16 &      \\ 
     &           &    9 &  1.09e-08 &  3.12e-16 &      \\ 
     &           &   10 &  1.09e-08 &  3.11e-16 &      \\ 
1019 &  1.02e+04 &   10 &           &           & iters  \\ 
 \hdashline 
     &           &    1 &  3.39e-08 &  9.15e-09 &      \\ 
     &           &    2 &  4.39e-08 &  9.66e-16 &      \\ 
     &           &    3 &  3.39e-08 &  1.25e-15 &      \\ 
     &           &    4 &  4.39e-08 &  9.67e-16 &      \\ 
     &           &    5 &  3.39e-08 &  1.25e-15 &      \\ 
     &           &    6 &  4.39e-08 &  9.67e-16 &      \\ 
     &           &    7 &  3.39e-08 &  1.25e-15 &      \\ 
     &           &    8 &  3.38e-08 &  9.67e-16 &      \\ 
     &           &    9 &  4.39e-08 &  9.66e-16 &      \\ 
     &           &   10 &  3.39e-08 &  1.25e-15 &      \\ 
1020 &  1.02e+04 &   10 &           &           & iters  \\ 
 \hdashline 
     &           &    1 &  3.18e-08 &  9.19e-09 &      \\ 
     &           &    2 &  4.60e-08 &  9.06e-16 &      \\ 
     &           &    3 &  3.18e-08 &  1.31e-15 &      \\ 
     &           &    4 &  3.17e-08 &  9.07e-16 &      \\ 
     &           &    5 &  4.60e-08 &  9.06e-16 &      \\ 
     &           &    6 &  3.18e-08 &  1.31e-15 &      \\ 
     &           &    7 &  4.60e-08 &  9.06e-16 &      \\ 
     &           &    8 &  3.18e-08 &  1.31e-15 &      \\ 
     &           &    9 &  4.60e-08 &  9.07e-16 &      \\ 
     &           &   10 &  3.18e-08 &  1.31e-15 &      \\ 
1021 &  1.02e+04 &   10 &           &           & iters  \\ 
 \hdashline 
     &           &    1 &  3.18e-08 &  8.21e-10 &      \\ 
     &           &    2 &  4.59e-08 &  9.07e-16 &      \\ 
     &           &    3 &  3.18e-08 &  1.31e-15 &      \\ 
     &           &    4 &  4.59e-08 &  9.08e-16 &      \\ 
     &           &    5 &  3.18e-08 &  1.31e-15 &      \\ 
     &           &    6 &  3.18e-08 &  9.08e-16 &      \\ 
     &           &    7 &  4.60e-08 &  9.07e-16 &      \\ 
     &           &    8 &  3.18e-08 &  1.31e-15 &      \\ 
     &           &    9 &  4.59e-08 &  9.08e-16 &      \\ 
     &           &   10 &  3.18e-08 &  1.31e-15 &      \\ 
1022 &  1.02e+04 &   10 &           &           & iters  \\ 
 \hdashline 
     &           &    1 &  1.03e-08 &  6.83e-09 &      \\ 
     &           &    2 &  1.03e-08 &  2.93e-16 &      \\ 
     &           &    3 &  1.02e-08 &  2.93e-16 &      \\ 
     &           &    4 &  6.75e-08 &  2.92e-16 &      \\ 
     &           &    5 &  8.80e-08 &  1.93e-15 &      \\ 
     &           &    6 &  6.75e-08 &  2.51e-15 &      \\ 
     &           &    7 &  1.03e-08 &  1.93e-15 &      \\ 
     &           &    8 &  1.03e-08 &  2.94e-16 &      \\ 
     &           &    9 &  1.03e-08 &  2.93e-16 &      \\ 
     &           &   10 &  1.02e-08 &  2.93e-16 &      \\ 
1023 &  1.02e+04 &   10 &           &           & iters  \\ 
 \hdashline 
     &           &    1 &  7.76e-10 &  1.58e-08 &      \\ 
     &           &    2 &  7.74e-10 &  2.21e-17 &      \\ 
     &           &    3 &  7.73e-10 &  2.21e-17 &      \\ 
     &           &    4 &  7.72e-10 &  2.21e-17 &      \\ 
     &           &    5 &  7.71e-10 &  2.20e-17 &      \\ 
     &           &    6 &  7.70e-10 &  2.20e-17 &      \\ 
     &           &    7 &  7.69e-10 &  2.20e-17 &      \\ 
     &           &    8 &  7.68e-10 &  2.19e-17 &      \\ 
     &           &    9 &  7.67e-10 &  2.19e-17 &      \\ 
     &           &   10 &  7.66e-10 &  2.19e-17 &      \\ 
1024 &  1.02e+04 &   10 &           &           & iters  \\ 
 \hdashline 
     &           &    1 &  2.54e-08 &  1.95e-08 &      \\ 
     &           &    2 &  5.23e-08 &  7.25e-16 &      \\ 
     &           &    3 &  2.54e-08 &  1.49e-15 &      \\ 
     &           &    4 &  2.54e-08 &  7.26e-16 &      \\ 
     &           &    5 &  5.23e-08 &  7.25e-16 &      \\ 
     &           &    6 &  2.54e-08 &  1.49e-15 &      \\ 
     &           &    7 &  2.54e-08 &  7.26e-16 &      \\ 
     &           &    8 &  5.23e-08 &  7.25e-16 &      \\ 
     &           &    9 &  2.54e-08 &  1.49e-15 &      \\ 
     &           &   10 &  2.54e-08 &  7.26e-16 &      \\ 
1025 &  1.02e+04 &   10 &           &           & iters  \\ 
 \hdashline 
     &           &    1 &  2.06e-08 &  9.77e-09 &      \\ 
     &           &    2 &  1.83e-08 &  5.87e-16 &      \\ 
     &           &    3 &  2.06e-08 &  5.23e-16 &      \\ 
     &           &    4 &  1.83e-08 &  5.87e-16 &      \\ 
     &           &    5 &  2.06e-08 &  5.23e-16 &      \\ 
     &           &    6 &  1.83e-08 &  5.87e-16 &      \\ 
     &           &    7 &  2.06e-08 &  5.23e-16 &      \\ 
     &           &    8 &  1.83e-08 &  5.87e-16 &      \\ 
     &           &    9 &  2.06e-08 &  5.23e-16 &      \\ 
     &           &   10 &  1.83e-08 &  5.87e-16 &      \\ 
1026 &  1.02e+04 &   10 &           &           & iters  \\ 
 \hdashline 
     &           &    1 &  1.80e-08 &  1.57e-08 &      \\ 
     &           &    2 &  2.09e-08 &  5.13e-16 &      \\ 
     &           &    3 &  1.80e-08 &  5.97e-16 &      \\ 
     &           &    4 &  2.09e-08 &  5.13e-16 &      \\ 
     &           &    5 &  1.80e-08 &  5.97e-16 &      \\ 
     &           &    6 &  2.09e-08 &  5.13e-16 &      \\ 
     &           &    7 &  1.80e-08 &  5.97e-16 &      \\ 
     &           &    8 &  1.79e-08 &  5.13e-16 &      \\ 
     &           &    9 &  2.09e-08 &  5.12e-16 &      \\ 
     &           &   10 &  1.80e-08 &  5.97e-16 &      \\ 
1027 &  1.03e+04 &   10 &           &           & iters  \\ 
 \hdashline 
     &           &    1 &  3.82e-08 &  2.21e-08 &      \\ 
     &           &    2 &  3.96e-08 &  1.09e-15 &      \\ 
     &           &    3 &  3.82e-08 &  1.13e-15 &      \\ 
     &           &    4 &  3.96e-08 &  1.09e-15 &      \\ 
     &           &    5 &  3.82e-08 &  1.13e-15 &      \\ 
     &           &    6 &  3.96e-08 &  1.09e-15 &      \\ 
     &           &    7 &  3.82e-08 &  1.13e-15 &      \\ 
     &           &    8 &  3.96e-08 &  1.09e-15 &      \\ 
     &           &    9 &  3.82e-08 &  1.13e-15 &      \\ 
     &           &   10 &  3.96e-08 &  1.09e-15 &      \\ 
1028 &  1.03e+04 &   10 &           &           & iters  \\ 
 \hdashline 
     &           &    1 &  3.47e-08 &  8.85e-09 &      \\ 
     &           &    2 &  4.30e-08 &  9.91e-16 &      \\ 
     &           &    3 &  3.47e-08 &  1.23e-15 &      \\ 
     &           &    4 &  4.30e-08 &  9.91e-16 &      \\ 
     &           &    5 &  3.47e-08 &  1.23e-15 &      \\ 
     &           &    6 &  3.47e-08 &  9.91e-16 &      \\ 
     &           &    7 &  4.30e-08 &  9.90e-16 &      \\ 
     &           &    8 &  3.47e-08 &  1.23e-15 &      \\ 
     &           &    9 &  4.30e-08 &  9.90e-16 &      \\ 
     &           &   10 &  3.47e-08 &  1.23e-15 &      \\ 
1029 &  1.03e+04 &   10 &           &           & iters  \\ 
 \hdashline 
     &           &    1 &  4.04e-08 &  3.38e-09 &      \\ 
     &           &    2 &  3.73e-08 &  1.15e-15 &      \\ 
     &           &    3 &  4.04e-08 &  1.07e-15 &      \\ 
     &           &    4 &  3.73e-08 &  1.15e-15 &      \\ 
     &           &    5 &  4.04e-08 &  1.07e-15 &      \\ 
     &           &    6 &  3.73e-08 &  1.15e-15 &      \\ 
     &           &    7 &  4.04e-08 &  1.07e-15 &      \\ 
     &           &    8 &  3.73e-08 &  1.15e-15 &      \\ 
     &           &    9 &  4.04e-08 &  1.07e-15 &      \\ 
     &           &   10 &  3.73e-08 &  1.15e-15 &      \\ 
1030 &  1.03e+04 &   10 &           &           & iters  \\ 
 \hdashline 
     &           &    1 &  1.33e-08 &  1.28e-09 &      \\ 
     &           &    2 &  2.56e-08 &  3.79e-16 &      \\ 
     &           &    3 &  1.33e-08 &  7.30e-16 &      \\ 
     &           &    4 &  1.33e-08 &  3.79e-16 &      \\ 
     &           &    5 &  2.56e-08 &  3.79e-16 &      \\ 
     &           &    6 &  1.33e-08 &  7.30e-16 &      \\ 
     &           &    7 &  1.33e-08 &  3.79e-16 &      \\ 
     &           &    8 &  2.56e-08 &  3.79e-16 &      \\ 
     &           &    9 &  1.33e-08 &  7.30e-16 &      \\ 
     &           &   10 &  1.33e-08 &  3.79e-16 &      \\ 
1031 &  1.03e+04 &   10 &           &           & iters  \\ 
 \hdashline 
     &           &    1 &  3.04e-09 &  4.38e-09 &      \\ 
     &           &    2 &  3.58e-08 &  8.67e-17 &      \\ 
     &           &    3 &  3.09e-09 &  1.02e-15 &      \\ 
     &           &    4 &  3.08e-09 &  8.81e-17 &      \\ 
     &           &    5 &  3.08e-09 &  8.79e-17 &      \\ 
     &           &    6 &  3.07e-09 &  8.78e-17 &      \\ 
     &           &    7 &  3.07e-09 &  8.77e-17 &      \\ 
     &           &    8 &  3.06e-09 &  8.76e-17 &      \\ 
     &           &    9 &  3.06e-09 &  8.74e-17 &      \\ 
     &           &   10 &  3.05e-09 &  8.73e-17 &      \\ 
1032 &  1.03e+04 &   10 &           &           & iters  \\ 
 \hdashline 
     &           &    1 &  1.87e-08 &  5.94e-09 &      \\ 
     &           &    2 &  5.90e-08 &  5.34e-16 &      \\ 
     &           &    3 &  1.88e-08 &  1.68e-15 &      \\ 
     &           &    4 &  1.87e-08 &  5.36e-16 &      \\ 
     &           &    5 &  1.87e-08 &  5.35e-16 &      \\ 
     &           &    6 &  5.90e-08 &  5.34e-16 &      \\ 
     &           &    7 &  1.88e-08 &  1.68e-15 &      \\ 
     &           &    8 &  1.87e-08 &  5.36e-16 &      \\ 
     &           &    9 &  1.87e-08 &  5.35e-16 &      \\ 
     &           &   10 &  5.90e-08 &  5.34e-16 &      \\ 
1033 &  1.03e+04 &   10 &           &           & iters  \\ 
 \hdashline 
     &           &    1 &  5.70e-08 &  7.05e-09 &      \\ 
     &           &    2 &  2.08e-08 &  1.63e-15 &      \\ 
     &           &    3 &  2.07e-08 &  5.93e-16 &      \\ 
     &           &    4 &  5.70e-08 &  5.92e-16 &      \\ 
     &           &    5 &  2.08e-08 &  1.63e-15 &      \\ 
     &           &    6 &  2.08e-08 &  5.93e-16 &      \\ 
     &           &    7 &  2.07e-08 &  5.93e-16 &      \\ 
     &           &    8 &  5.70e-08 &  5.92e-16 &      \\ 
     &           &    9 &  2.08e-08 &  1.63e-15 &      \\ 
     &           &   10 &  2.08e-08 &  5.93e-16 &      \\ 
1034 &  1.03e+04 &   10 &           &           & iters  \\ 
 \hdashline 
     &           &    1 &  4.72e-09 &  8.18e-09 &      \\ 
     &           &    2 &  4.72e-09 &  1.35e-16 &      \\ 
     &           &    3 &  4.71e-09 &  1.35e-16 &      \\ 
     &           &    4 &  4.70e-09 &  1.34e-16 &      \\ 
     &           &    5 &  4.70e-09 &  1.34e-16 &      \\ 
     &           &    6 &  4.69e-09 &  1.34e-16 &      \\ 
     &           &    7 &  4.68e-09 &  1.34e-16 &      \\ 
     &           &    8 &  4.68e-09 &  1.34e-16 &      \\ 
     &           &    9 &  4.67e-09 &  1.33e-16 &      \\ 
     &           &   10 &  4.66e-09 &  1.33e-16 &      \\ 
1035 &  1.03e+04 &   10 &           &           & iters  \\ 
 \hdashline 
     &           &    1 &  6.56e-08 &  1.10e-08 &      \\ 
     &           &    2 &  1.22e-08 &  1.87e-15 &      \\ 
     &           &    3 &  1.22e-08 &  3.48e-16 &      \\ 
     &           &    4 &  1.21e-08 &  3.47e-16 &      \\ 
     &           &    5 &  1.21e-08 &  3.47e-16 &      \\ 
     &           &    6 &  1.21e-08 &  3.46e-16 &      \\ 
     &           &    7 &  6.56e-08 &  3.46e-16 &      \\ 
     &           &    8 &  5.10e-08 &  1.87e-15 &      \\ 
     &           &    9 &  1.21e-08 &  1.46e-15 &      \\ 
     &           &   10 &  1.21e-08 &  3.46e-16 &      \\ 
1036 &  1.04e+04 &   10 &           &           & iters  \\ 
 \hdashline 
     &           &    1 &  1.20e-08 &  1.07e-08 &      \\ 
     &           &    2 &  1.20e-08 &  3.43e-16 &      \\ 
     &           &    3 &  1.20e-08 &  3.42e-16 &      \\ 
     &           &    4 &  1.20e-08 &  3.42e-16 &      \\ 
     &           &    5 &  2.69e-08 &  3.41e-16 &      \\ 
     &           &    6 &  1.20e-08 &  7.68e-16 &      \\ 
     &           &    7 &  1.20e-08 &  3.42e-16 &      \\ 
     &           &    8 &  2.69e-08 &  3.41e-16 &      \\ 
     &           &    9 &  1.20e-08 &  7.68e-16 &      \\ 
     &           &   10 &  1.20e-08 &  3.42e-16 &      \\ 
1037 &  1.04e+04 &   10 &           &           & iters  \\ 
 \hdashline 
     &           &    1 &  2.37e-08 &  3.11e-09 &      \\ 
     &           &    2 &  5.41e-08 &  6.75e-16 &      \\ 
     &           &    3 &  2.37e-08 &  1.54e-15 &      \\ 
     &           &    4 &  2.37e-08 &  6.76e-16 &      \\ 
     &           &    5 &  5.41e-08 &  6.76e-16 &      \\ 
     &           &    6 &  2.37e-08 &  1.54e-15 &      \\ 
     &           &    7 &  2.37e-08 &  6.77e-16 &      \\ 
     &           &    8 &  5.40e-08 &  6.76e-16 &      \\ 
     &           &    9 &  2.37e-08 &  1.54e-15 &      \\ 
     &           &   10 &  2.37e-08 &  6.77e-16 &      \\ 
1038 &  1.04e+04 &   10 &           &           & iters  \\ 
 \hdashline 
     &           &    1 &  2.69e-08 &  6.80e-10 &      \\ 
     &           &    2 &  1.19e-08 &  7.69e-16 &      \\ 
     &           &    3 &  1.19e-08 &  3.41e-16 &      \\ 
     &           &    4 &  1.19e-08 &  3.40e-16 &      \\ 
     &           &    5 &  2.70e-08 &  3.40e-16 &      \\ 
     &           &    6 &  1.19e-08 &  7.69e-16 &      \\ 
     &           &    7 &  1.19e-08 &  3.40e-16 &      \\ 
     &           &    8 &  2.69e-08 &  3.40e-16 &      \\ 
     &           &    9 &  1.19e-08 &  7.69e-16 &      \\ 
     &           &   10 &  1.19e-08 &  3.41e-16 &      \\ 
1039 &  1.04e+04 &   10 &           &           & iters  \\ 
 \hdashline 
     &           &    1 &  1.84e-08 &  3.72e-09 &      \\ 
     &           &    2 &  2.05e-08 &  5.26e-16 &      \\ 
     &           &    3 &  1.84e-08 &  5.84e-16 &      \\ 
     &           &    4 &  2.05e-08 &  5.26e-16 &      \\ 
     &           &    5 &  1.84e-08 &  5.84e-16 &      \\ 
     &           &    6 &  2.05e-08 &  5.26e-16 &      \\ 
     &           &    7 &  1.84e-08 &  5.84e-16 &      \\ 
     &           &    8 &  1.84e-08 &  5.26e-16 &      \\ 
     &           &    9 &  2.05e-08 &  5.25e-16 &      \\ 
     &           &   10 &  1.84e-08 &  5.84e-16 &      \\ 
1040 &  1.04e+04 &   10 &           &           & iters  \\ 
 \hdashline 
     &           &    1 &  2.96e-08 &  6.65e-09 &      \\ 
     &           &    2 &  2.95e-08 &  8.44e-16 &      \\ 
     &           &    3 &  4.82e-08 &  8.42e-16 &      \\ 
     &           &    4 &  2.95e-08 &  1.38e-15 &      \\ 
     &           &    5 &  2.95e-08 &  8.43e-16 &      \\ 
     &           &    6 &  4.82e-08 &  8.42e-16 &      \\ 
     &           &    7 &  2.95e-08 &  1.38e-15 &      \\ 
     &           &    8 &  2.95e-08 &  8.43e-16 &      \\ 
     &           &    9 &  4.82e-08 &  8.42e-16 &      \\ 
     &           &   10 &  2.95e-08 &  1.38e-15 &      \\ 
1041 &  1.04e+04 &   10 &           &           & iters  \\ 
 \hdashline 
     &           &    1 &  7.33e-08 &  1.06e-08 &      \\ 
     &           &    2 &  4.53e-09 &  2.09e-15 &      \\ 
     &           &    3 &  4.53e-09 &  1.29e-16 &      \\ 
     &           &    4 &  4.52e-09 &  1.29e-16 &      \\ 
     &           &    5 &  4.51e-09 &  1.29e-16 &      \\ 
     &           &    6 &  4.51e-09 &  1.29e-16 &      \\ 
     &           &    7 &  4.50e-09 &  1.29e-16 &      \\ 
     &           &    8 &  4.49e-09 &  1.28e-16 &      \\ 
     &           &    9 &  4.49e-09 &  1.28e-16 &      \\ 
     &           &   10 &  4.48e-09 &  1.28e-16 &      \\ 
1042 &  1.04e+04 &   10 &           &           & iters  \\ 
 \hdashline 
     &           &    1 &  4.21e-10 &  7.01e-09 &      \\ 
     &           &    2 &  4.20e-10 &  1.20e-17 &      \\ 
     &           &    3 &  4.20e-10 &  1.20e-17 &      \\ 
     &           &    4 &  4.19e-10 &  1.20e-17 &      \\ 
     &           &    5 &  4.19e-10 &  1.20e-17 &      \\ 
     &           &    6 &  4.18e-10 &  1.19e-17 &      \\ 
     &           &    7 &  4.17e-10 &  1.19e-17 &      \\ 
     &           &    8 &  4.17e-10 &  1.19e-17 &      \\ 
     &           &    9 &  4.16e-10 &  1.19e-17 &      \\ 
     &           &   10 &  4.16e-10 &  1.19e-17 &      \\ 
1043 &  1.04e+04 &   10 &           &           & iters  \\ 
 \hdashline 
     &           &    1 &  2.37e-08 &  4.32e-09 &      \\ 
     &           &    2 &  2.37e-08 &  6.78e-16 &      \\ 
     &           &    3 &  2.37e-08 &  6.77e-16 &      \\ 
     &           &    4 &  5.40e-08 &  6.76e-16 &      \\ 
     &           &    5 &  2.37e-08 &  1.54e-15 &      \\ 
     &           &    6 &  2.37e-08 &  6.77e-16 &      \\ 
     &           &    7 &  5.40e-08 &  6.76e-16 &      \\ 
     &           &    8 &  2.37e-08 &  1.54e-15 &      \\ 
     &           &    9 &  2.37e-08 &  6.77e-16 &      \\ 
     &           &   10 &  5.40e-08 &  6.76e-16 &      \\ 
1044 &  1.04e+04 &   10 &           &           & iters  \\ 
 \hdashline 
     &           &    1 &  1.93e-08 &  9.08e-09 &      \\ 
     &           &    2 &  1.93e-08 &  5.51e-16 &      \\ 
     &           &    3 &  1.92e-08 &  5.50e-16 &      \\ 
     &           &    4 &  5.85e-08 &  5.49e-16 &      \\ 
     &           &    5 &  1.93e-08 &  1.67e-15 &      \\ 
     &           &    6 &  1.93e-08 &  5.51e-16 &      \\ 
     &           &    7 &  1.92e-08 &  5.50e-16 &      \\ 
     &           &    8 &  5.85e-08 &  5.49e-16 &      \\ 
     &           &    9 &  1.93e-08 &  1.67e-15 &      \\ 
     &           &   10 &  1.93e-08 &  5.51e-16 &      \\ 
1045 &  1.04e+04 &   10 &           &           & iters  \\ 
 \hdashline 
     &           &    1 &  3.62e-08 &  6.86e-09 &      \\ 
     &           &    2 &  3.62e-08 &  1.03e-15 &      \\ 
     &           &    3 &  4.16e-08 &  1.03e-15 &      \\ 
     &           &    4 &  3.62e-08 &  1.19e-15 &      \\ 
     &           &    5 &  4.16e-08 &  1.03e-15 &      \\ 
     &           &    6 &  3.62e-08 &  1.19e-15 &      \\ 
     &           &    7 &  4.16e-08 &  1.03e-15 &      \\ 
     &           &    8 &  3.62e-08 &  1.19e-15 &      \\ 
     &           &    9 &  4.15e-08 &  1.03e-15 &      \\ 
     &           &   10 &  3.62e-08 &  1.19e-15 &      \\ 
1046 &  1.04e+04 &   10 &           &           & iters  \\ 
 \hdashline 
     &           &    1 &  5.54e-08 &  2.83e-09 &      \\ 
     &           &    2 &  2.23e-08 &  1.58e-15 &      \\ 
     &           &    3 &  2.23e-08 &  6.38e-16 &      \\ 
     &           &    4 &  5.54e-08 &  6.37e-16 &      \\ 
     &           &    5 &  2.24e-08 &  1.58e-15 &      \\ 
     &           &    6 &  2.23e-08 &  6.38e-16 &      \\ 
     &           &    7 &  2.23e-08 &  6.37e-16 &      \\ 
     &           &    8 &  5.54e-08 &  6.36e-16 &      \\ 
     &           &    9 &  2.23e-08 &  1.58e-15 &      \\ 
     &           &   10 &  2.23e-08 &  6.38e-16 &      \\ 
1047 &  1.05e+04 &   10 &           &           & iters  \\ 
 \hdashline 
     &           &    1 &  1.77e-08 &  1.57e-10 &      \\ 
     &           &    2 &  1.76e-08 &  5.04e-16 &      \\ 
     &           &    3 &  6.01e-08 &  5.03e-16 &      \\ 
     &           &    4 &  1.77e-08 &  1.71e-15 &      \\ 
     &           &    5 &  1.77e-08 &  5.05e-16 &      \\ 
     &           &    6 &  1.76e-08 &  5.04e-16 &      \\ 
     &           &    7 &  1.76e-08 &  5.04e-16 &      \\ 
     &           &    8 &  6.01e-08 &  5.03e-16 &      \\ 
     &           &    9 &  1.77e-08 &  1.72e-15 &      \\ 
     &           &   10 &  1.77e-08 &  5.05e-16 &      \\ 
1048 &  1.05e+04 &   10 &           &           & iters  \\ 
 \hdashline 
     &           &    1 &  2.76e-08 &  3.67e-09 &      \\ 
     &           &    2 &  5.01e-08 &  7.88e-16 &      \\ 
     &           &    3 &  2.76e-08 &  1.43e-15 &      \\ 
     &           &    4 &  2.76e-08 &  7.89e-16 &      \\ 
     &           &    5 &  5.01e-08 &  7.88e-16 &      \\ 
     &           &    6 &  2.76e-08 &  1.43e-15 &      \\ 
     &           &    7 &  2.76e-08 &  7.89e-16 &      \\ 
     &           &    8 &  5.01e-08 &  7.88e-16 &      \\ 
     &           &    9 &  2.76e-08 &  1.43e-15 &      \\ 
     &           &   10 &  2.76e-08 &  7.89e-16 &      \\ 
1049 &  1.05e+04 &   10 &           &           & iters  \\ 
 \hdashline 
     &           &    1 &  4.70e-08 &  1.89e-09 &      \\ 
     &           &    2 &  3.08e-08 &  1.34e-15 &      \\ 
     &           &    3 &  4.70e-08 &  8.78e-16 &      \\ 
     &           &    4 &  3.08e-08 &  1.34e-15 &      \\ 
     &           &    5 &  4.69e-08 &  8.79e-16 &      \\ 
     &           &    6 &  3.08e-08 &  1.34e-15 &      \\ 
     &           &    7 &  3.08e-08 &  8.79e-16 &      \\ 
     &           &    8 &  4.70e-08 &  8.78e-16 &      \\ 
     &           &    9 &  3.08e-08 &  1.34e-15 &      \\ 
     &           &   10 &  4.69e-08 &  8.79e-16 &      \\ 
1050 &  1.05e+04 &   10 &           &           & iters  \\ 
 \hdashline 
     &           &    1 &  7.57e-08 &  1.78e-09 &      \\ 
     &           &    2 &  2.11e-09 &  2.16e-15 &      \\ 
     &           &    3 &  2.10e-09 &  6.01e-17 &      \\ 
     &           &    4 &  2.10e-09 &  6.00e-17 &      \\ 
     &           &    5 &  2.10e-09 &  5.99e-17 &      \\ 
     &           &    6 &  2.09e-09 &  5.99e-17 &      \\ 
     &           &    7 &  2.09e-09 &  5.98e-17 &      \\ 
     &           &    8 &  2.09e-09 &  5.97e-17 &      \\ 
     &           &    9 &  2.09e-09 &  5.96e-17 &      \\ 
     &           &   10 &  2.08e-09 &  5.95e-17 &      \\ 
1051 &  1.05e+04 &   10 &           &           & iters  \\ 
 \hdashline 
     &           &    1 &  1.44e-08 &  6.93e-09 &      \\ 
     &           &    2 &  1.44e-08 &  4.11e-16 &      \\ 
     &           &    3 &  1.43e-08 &  4.10e-16 &      \\ 
     &           &    4 &  1.43e-08 &  4.10e-16 &      \\ 
     &           &    5 &  1.43e-08 &  4.09e-16 &      \\ 
     &           &    6 &  6.34e-08 &  4.08e-16 &      \\ 
     &           &    7 &  1.44e-08 &  1.81e-15 &      \\ 
     &           &    8 &  1.44e-08 &  4.10e-16 &      \\ 
     &           &    9 &  1.43e-08 &  4.10e-16 &      \\ 
     &           &   10 &  1.43e-08 &  4.09e-16 &      \\ 
1052 &  1.05e+04 &   10 &           &           & iters  \\ 
 \hdashline 
     &           &    1 &  4.44e-08 &  8.22e-09 &      \\ 
     &           &    2 &  4.43e-08 &  1.27e-15 &      \\ 
     &           &    3 &  3.34e-08 &  1.27e-15 &      \\ 
     &           &    4 &  4.43e-08 &  9.54e-16 &      \\ 
     &           &    5 &  3.34e-08 &  1.26e-15 &      \\ 
     &           &    6 &  3.34e-08 &  9.54e-16 &      \\ 
     &           &    7 &  4.43e-08 &  9.53e-16 &      \\ 
     &           &    8 &  3.34e-08 &  1.27e-15 &      \\ 
     &           &    9 &  4.43e-08 &  9.54e-16 &      \\ 
     &           &   10 &  3.34e-08 &  1.27e-15 &      \\ 
1053 &  1.05e+04 &   10 &           &           & iters  \\ 
 \hdashline 
     &           &    1 &  9.57e-09 &  3.14e-09 &      \\ 
     &           &    2 &  6.81e-08 &  2.73e-16 &      \\ 
     &           &    3 &  8.73e-08 &  1.94e-15 &      \\ 
     &           &    4 &  6.82e-08 &  2.49e-15 &      \\ 
     &           &    5 &  9.62e-09 &  1.95e-15 &      \\ 
     &           &    6 &  9.61e-09 &  2.75e-16 &      \\ 
     &           &    7 &  9.60e-09 &  2.74e-16 &      \\ 
     &           &    8 &  9.58e-09 &  2.74e-16 &      \\ 
     &           &    9 &  9.57e-09 &  2.74e-16 &      \\ 
     &           &   10 &  6.81e-08 &  2.73e-16 &      \\ 
1054 &  1.05e+04 &   10 &           &           & iters  \\ 
 \hdashline 
     &           &    1 &  7.29e-08 &  7.78e-09 &      \\ 
     &           &    2 &  4.88e-09 &  2.08e-15 &      \\ 
     &           &    3 &  4.88e-09 &  1.39e-16 &      \\ 
     &           &    4 &  4.87e-09 &  1.39e-16 &      \\ 
     &           &    5 &  4.86e-09 &  1.39e-16 &      \\ 
     &           &    6 &  4.86e-09 &  1.39e-16 &      \\ 
     &           &    7 &  4.85e-09 &  1.39e-16 &      \\ 
     &           &    8 &  4.84e-09 &  1.38e-16 &      \\ 
     &           &    9 &  4.84e-09 &  1.38e-16 &      \\ 
     &           &   10 &  7.29e-08 &  1.38e-16 &      \\ 
1055 &  1.05e+04 &   10 &           &           & iters  \\ 
 \hdashline 
     &           &    1 &  5.79e-08 &  1.29e-08 &      \\ 
     &           &    2 &  1.99e-08 &  1.65e-15 &      \\ 
     &           &    3 &  5.78e-08 &  5.67e-16 &      \\ 
     &           &    4 &  1.99e-08 &  1.65e-15 &      \\ 
     &           &    5 &  1.99e-08 &  5.69e-16 &      \\ 
     &           &    6 &  1.99e-08 &  5.68e-16 &      \\ 
     &           &    7 &  5.79e-08 &  5.67e-16 &      \\ 
     &           &    8 &  1.99e-08 &  1.65e-15 &      \\ 
     &           &    9 &  1.99e-08 &  5.69e-16 &      \\ 
     &           &   10 &  1.99e-08 &  5.68e-16 &      \\ 
1056 &  1.06e+04 &   10 &           &           & iters  \\ 
 \hdashline 
     &           &    1 &  1.02e-08 &  4.60e-09 &      \\ 
     &           &    2 &  1.02e-08 &  2.91e-16 &      \\ 
     &           &    3 &  6.75e-08 &  2.90e-16 &      \\ 
     &           &    4 &  8.79e-08 &  1.93e-15 &      \\ 
     &           &    5 &  6.76e-08 &  2.51e-15 &      \\ 
     &           &    6 &  1.02e-08 &  1.93e-15 &      \\ 
     &           &    7 &  1.02e-08 &  2.92e-16 &      \\ 
     &           &    8 &  1.02e-08 &  2.92e-16 &      \\ 
     &           &    9 &  1.02e-08 &  2.91e-16 &      \\ 
     &           &   10 &  1.02e-08 &  2.91e-16 &      \\ 
1057 &  1.06e+04 &   10 &           &           & iters  \\ 
 \hdashline 
     &           &    1 &  7.80e-09 &  2.18e-08 &      \\ 
     &           &    2 &  7.79e-09 &  2.23e-16 &      \\ 
     &           &    3 &  7.77e-09 &  2.22e-16 &      \\ 
     &           &    4 &  6.99e-08 &  2.22e-16 &      \\ 
     &           &    5 &  8.56e-08 &  2.00e-15 &      \\ 
     &           &    6 &  6.99e-08 &  2.44e-15 &      \\ 
     &           &    7 &  7.84e-09 &  2.00e-15 &      \\ 
     &           &    8 &  7.83e-09 &  2.24e-16 &      \\ 
     &           &    9 &  7.82e-09 &  2.23e-16 &      \\ 
     &           &   10 &  7.81e-09 &  2.23e-16 &      \\ 
1058 &  1.06e+04 &   10 &           &           & iters  \\ 
 \hdashline 
     &           &    1 &  2.99e-08 &  1.68e-08 &      \\ 
     &           &    2 &  4.78e-08 &  8.54e-16 &      \\ 
     &           &    3 &  2.99e-08 &  1.36e-15 &      \\ 
     &           &    4 &  2.99e-08 &  8.54e-16 &      \\ 
     &           &    5 &  4.78e-08 &  8.53e-16 &      \\ 
     &           &    6 &  2.99e-08 &  1.37e-15 &      \\ 
     &           &    7 &  4.78e-08 &  8.54e-16 &      \\ 
     &           &    8 &  2.99e-08 &  1.36e-15 &      \\ 
     &           &    9 &  2.99e-08 &  8.55e-16 &      \\ 
     &           &   10 &  4.78e-08 &  8.53e-16 &      \\ 
1059 &  1.06e+04 &   10 &           &           & iters  \\ 
 \hdashline 
     &           &    1 &  3.23e-08 &  3.24e-09 &      \\ 
     &           &    2 &  6.63e-09 &  9.21e-16 &      \\ 
     &           &    3 &  6.62e-09 &  1.89e-16 &      \\ 
     &           &    4 &  6.62e-09 &  1.89e-16 &      \\ 
     &           &    5 &  3.22e-08 &  1.89e-16 &      \\ 
     &           &    6 &  6.65e-09 &  9.20e-16 &      \\ 
     &           &    7 &  6.64e-09 &  1.90e-16 &      \\ 
     &           &    8 &  6.63e-09 &  1.90e-16 &      \\ 
     &           &    9 &  6.62e-09 &  1.89e-16 &      \\ 
     &           &   10 &  6.61e-09 &  1.89e-16 &      \\ 
1060 &  1.06e+04 &   10 &           &           & iters  \\ 
 \hdashline 
     &           &    1 &  3.85e-11 &  1.45e-08 &      \\ 
     &           &    2 &  3.84e-11 &  1.10e-18 &      \\ 
     &           &    3 &  3.84e-11 &  1.10e-18 &      \\ 
     &           &    4 &  3.83e-11 &  1.10e-18 &      \\ 
     &           &    5 &  3.83e-11 &  1.09e-18 &      \\ 
     &           &    6 &  3.82e-11 &  1.09e-18 &      \\ 
     &           &    7 &  3.82e-11 &  1.09e-18 &      \\ 
     &           &    8 &  3.81e-11 &  1.09e-18 &      \\ 
     &           &    9 &  3.81e-11 &  1.09e-18 &      \\ 
     &           &   10 &  3.80e-11 &  1.09e-18 &      \\ 
1061 &  1.06e+04 &   10 &           &           & iters  \\ 
 \hdashline 
     &           &    1 &  2.15e-08 &  1.67e-08 &      \\ 
     &           &    2 &  2.15e-08 &  6.14e-16 &      \\ 
     &           &    3 &  5.62e-08 &  6.13e-16 &      \\ 
     &           &    4 &  2.15e-08 &  1.61e-15 &      \\ 
     &           &    5 &  2.15e-08 &  6.14e-16 &      \\ 
     &           &    6 &  5.62e-08 &  6.13e-16 &      \\ 
     &           &    7 &  2.15e-08 &  1.60e-15 &      \\ 
     &           &    8 &  2.15e-08 &  6.15e-16 &      \\ 
     &           &    9 &  2.15e-08 &  6.14e-16 &      \\ 
     &           &   10 &  5.62e-08 &  6.13e-16 &      \\ 
1062 &  1.06e+04 &   10 &           &           & iters  \\ 
 \hdashline 
     &           &    1 &  3.28e-08 &  1.21e-09 &      \\ 
     &           &    2 &  4.50e-08 &  9.36e-16 &      \\ 
     &           &    3 &  3.28e-08 &  1.28e-15 &      \\ 
     &           &    4 &  4.49e-08 &  9.36e-16 &      \\ 
     &           &    5 &  3.28e-08 &  1.28e-15 &      \\ 
     &           &    6 &  4.49e-08 &  9.37e-16 &      \\ 
     &           &    7 &  3.28e-08 &  1.28e-15 &      \\ 
     &           &    8 &  3.28e-08 &  9.37e-16 &      \\ 
     &           &    9 &  4.49e-08 &  9.36e-16 &      \\ 
     &           &   10 &  3.28e-08 &  1.28e-15 &      \\ 
1063 &  1.06e+04 &   10 &           &           & iters  \\ 
 \hdashline 
     &           &    1 &  1.95e-08 &  8.97e-09 &      \\ 
     &           &    2 &  1.95e-08 &  5.56e-16 &      \\ 
     &           &    3 &  5.82e-08 &  5.56e-16 &      \\ 
     &           &    4 &  1.95e-08 &  1.66e-15 &      \\ 
     &           &    5 &  1.95e-08 &  5.57e-16 &      \\ 
     &           &    6 &  1.95e-08 &  5.56e-16 &      \\ 
     &           &    7 &  5.82e-08 &  5.56e-16 &      \\ 
     &           &    8 &  1.95e-08 &  1.66e-15 &      \\ 
     &           &    9 &  1.95e-08 &  5.57e-16 &      \\ 
     &           &   10 &  1.95e-08 &  5.56e-16 &      \\ 
1064 &  1.06e+04 &   10 &           &           & iters  \\ 
 \hdashline 
     &           &    1 &  8.56e-08 &  1.75e-08 &      \\ 
     &           &    2 &  3.10e-08 &  2.44e-15 &      \\ 
     &           &    3 &  7.87e-09 &  8.85e-16 &      \\ 
     &           &    4 &  7.86e-09 &  2.25e-16 &      \\ 
     &           &    5 &  3.10e-08 &  2.24e-16 &      \\ 
     &           &    6 &  7.89e-09 &  8.85e-16 &      \\ 
     &           &    7 &  7.88e-09 &  2.25e-16 &      \\ 
     &           &    8 &  7.87e-09 &  2.25e-16 &      \\ 
     &           &    9 &  7.85e-09 &  2.24e-16 &      \\ 
     &           &   10 &  3.10e-08 &  2.24e-16 &      \\ 
1065 &  1.06e+04 &   10 &           &           & iters  \\ 
 \hdashline 
     &           &    1 &  4.38e-09 &  1.48e-08 &      \\ 
     &           &    2 &  4.38e-09 &  1.25e-16 &      \\ 
     &           &    3 &  4.37e-09 &  1.25e-16 &      \\ 
     &           &    4 &  4.36e-09 &  1.25e-16 &      \\ 
     &           &    5 &  4.36e-09 &  1.25e-16 &      \\ 
     &           &    6 &  4.35e-09 &  1.24e-16 &      \\ 
     &           &    7 &  4.34e-09 &  1.24e-16 &      \\ 
     &           &    8 &  3.45e-08 &  1.24e-16 &      \\ 
     &           &    9 &  4.39e-09 &  9.85e-16 &      \\ 
     &           &   10 &  4.38e-09 &  1.25e-16 &      \\ 
1066 &  1.06e+04 &   10 &           &           & iters  \\ 
 \hdashline 
     &           &    1 &  3.66e-08 &  7.71e-09 &      \\ 
     &           &    2 &  4.11e-08 &  1.04e-15 &      \\ 
     &           &    3 &  3.66e-08 &  1.17e-15 &      \\ 
     &           &    4 &  4.11e-08 &  1.05e-15 &      \\ 
     &           &    5 &  3.66e-08 &  1.17e-15 &      \\ 
     &           &    6 &  4.11e-08 &  1.05e-15 &      \\ 
     &           &    7 &  3.66e-08 &  1.17e-15 &      \\ 
     &           &    8 &  4.11e-08 &  1.05e-15 &      \\ 
     &           &    9 &  3.66e-08 &  1.17e-15 &      \\ 
     &           &   10 &  4.11e-08 &  1.05e-15 &      \\ 
1067 &  1.07e+04 &   10 &           &           & iters  \\ 
 \hdashline 
     &           &    1 &  5.95e-08 &  8.89e-09 &      \\ 
     &           &    2 &  1.82e-08 &  1.70e-15 &      \\ 
     &           &    3 &  1.82e-08 &  5.20e-16 &      \\ 
     &           &    4 &  1.82e-08 &  5.20e-16 &      \\ 
     &           &    5 &  5.95e-08 &  5.19e-16 &      \\ 
     &           &    6 &  1.82e-08 &  1.70e-15 &      \\ 
     &           &    7 &  1.82e-08 &  5.21e-16 &      \\ 
     &           &    8 &  1.82e-08 &  5.20e-16 &      \\ 
     &           &    9 &  5.95e-08 &  5.19e-16 &      \\ 
     &           &   10 &  1.83e-08 &  1.70e-15 &      \\ 
1068 &  1.07e+04 &   10 &           &           & iters  \\ 
 \hdashline 
     &           &    1 &  1.20e-09 &  4.05e-09 &      \\ 
     &           &    2 &  1.20e-09 &  3.42e-17 &      \\ 
     &           &    3 &  1.20e-09 &  3.42e-17 &      \\ 
     &           &    4 &  1.19e-09 &  3.41e-17 &      \\ 
     &           &    5 &  1.19e-09 &  3.41e-17 &      \\ 
     &           &    6 &  1.19e-09 &  3.40e-17 &      \\ 
     &           &    7 &  1.19e-09 &  3.40e-17 &      \\ 
     &           &    8 &  1.19e-09 &  3.39e-17 &      \\ 
     &           &    9 &  1.19e-09 &  3.39e-17 &      \\ 
     &           &   10 &  1.18e-09 &  3.38e-17 &      \\ 
1069 &  1.07e+04 &   10 &           &           & iters  \\ 
 \hdashline 
     &           &    1 &  3.95e-08 &  1.44e-08 &      \\ 
     &           &    2 &  3.95e-08 &  1.13e-15 &      \\ 
     &           &    3 &  3.83e-08 &  1.13e-15 &      \\ 
     &           &    4 &  3.95e-08 &  1.09e-15 &      \\ 
     &           &    5 &  3.83e-08 &  1.13e-15 &      \\ 
     &           &    6 &  3.82e-08 &  1.09e-15 &      \\ 
     &           &    7 &  3.95e-08 &  1.09e-15 &      \\ 
     &           &    8 &  3.82e-08 &  1.13e-15 &      \\ 
     &           &    9 &  3.95e-08 &  1.09e-15 &      \\ 
     &           &   10 &  3.82e-08 &  1.13e-15 &      \\ 
1070 &  1.07e+04 &   10 &           &           & iters  \\ 
 \hdashline 
     &           &    1 &  2.42e-08 &  4.80e-10 &      \\ 
     &           &    2 &  2.41e-08 &  6.89e-16 &      \\ 
     &           &    3 &  5.36e-08 &  6.88e-16 &      \\ 
     &           &    4 &  2.42e-08 &  1.53e-15 &      \\ 
     &           &    5 &  2.41e-08 &  6.90e-16 &      \\ 
     &           &    6 &  5.36e-08 &  6.89e-16 &      \\ 
     &           &    7 &  2.42e-08 &  1.53e-15 &      \\ 
     &           &    8 &  2.41e-08 &  6.90e-16 &      \\ 
     &           &    9 &  5.36e-08 &  6.89e-16 &      \\ 
     &           &   10 &  2.42e-08 &  1.53e-15 &      \\ 
1071 &  1.07e+04 &   10 &           &           & iters  \\ 
 \hdashline 
     &           &    1 &  1.93e-08 &  5.97e-09 &      \\ 
     &           &    2 &  5.84e-08 &  5.51e-16 &      \\ 
     &           &    3 &  1.94e-08 &  1.67e-15 &      \\ 
     &           &    4 &  1.93e-08 &  5.53e-16 &      \\ 
     &           &    5 &  1.93e-08 &  5.52e-16 &      \\ 
     &           &    6 &  5.84e-08 &  5.51e-16 &      \\ 
     &           &    7 &  1.94e-08 &  1.67e-15 &      \\ 
     &           &    8 &  1.93e-08 &  5.53e-16 &      \\ 
     &           &    9 &  1.93e-08 &  5.52e-16 &      \\ 
     &           &   10 &  5.84e-08 &  5.51e-16 &      \\ 
1072 &  1.07e+04 &   10 &           &           & iters  \\ 
 \hdashline 
     &           &    1 &  3.50e-08 &  4.83e-09 &      \\ 
     &           &    2 &  4.28e-08 &  9.98e-16 &      \\ 
     &           &    3 &  3.50e-08 &  1.22e-15 &      \\ 
     &           &    4 &  3.49e-08 &  9.98e-16 &      \\ 
     &           &    5 &  4.28e-08 &  9.97e-16 &      \\ 
     &           &    6 &  3.49e-08 &  1.22e-15 &      \\ 
     &           &    7 &  4.28e-08 &  9.97e-16 &      \\ 
     &           &    8 &  3.49e-08 &  1.22e-15 &      \\ 
     &           &    9 &  4.28e-08 &  9.97e-16 &      \\ 
     &           &   10 &  3.50e-08 &  1.22e-15 &      \\ 
1073 &  1.07e+04 &   10 &           &           & iters  \\ 
 \hdashline 
     &           &    1 &  2.75e-08 &  9.47e-09 &      \\ 
     &           &    2 &  2.75e-08 &  7.85e-16 &      \\ 
     &           &    3 &  5.03e-08 &  7.84e-16 &      \\ 
     &           &    4 &  2.75e-08 &  1.43e-15 &      \\ 
     &           &    5 &  2.75e-08 &  7.85e-16 &      \\ 
     &           &    6 &  5.03e-08 &  7.84e-16 &      \\ 
     &           &    7 &  2.75e-08 &  1.43e-15 &      \\ 
     &           &    8 &  2.75e-08 &  7.85e-16 &      \\ 
     &           &    9 &  5.03e-08 &  7.84e-16 &      \\ 
     &           &   10 &  2.75e-08 &  1.43e-15 &      \\ 
1074 &  1.07e+04 &   10 &           &           & iters  \\ 
 \hdashline 
     &           &    1 &  2.58e-09 &  4.95e-09 &      \\ 
     &           &    2 &  2.57e-09 &  7.35e-17 &      \\ 
     &           &    3 &  2.57e-09 &  7.34e-17 &      \\ 
     &           &    4 &  2.56e-09 &  7.33e-17 &      \\ 
     &           &    5 &  2.56e-09 &  7.32e-17 &      \\ 
     &           &    6 &  2.56e-09 &  7.31e-17 &      \\ 
     &           &    7 &  2.55e-09 &  7.30e-17 &      \\ 
     &           &    8 &  2.55e-09 &  7.29e-17 &      \\ 
     &           &    9 &  2.55e-09 &  7.28e-17 &      \\ 
     &           &   10 &  2.54e-09 &  7.27e-17 &      \\ 
1075 &  1.07e+04 &   10 &           &           & iters  \\ 
 \hdashline 
     &           &    1 &  4.89e-08 &  6.73e-09 &      \\ 
     &           &    2 &  2.89e-08 &  1.39e-15 &      \\ 
     &           &    3 &  2.89e-08 &  8.25e-16 &      \\ 
     &           &    4 &  4.89e-08 &  8.24e-16 &      \\ 
     &           &    5 &  2.89e-08 &  1.39e-15 &      \\ 
     &           &    6 &  2.89e-08 &  8.25e-16 &      \\ 
     &           &    7 &  4.89e-08 &  8.23e-16 &      \\ 
     &           &    8 &  2.89e-08 &  1.39e-15 &      \\ 
     &           &    9 &  4.88e-08 &  8.24e-16 &      \\ 
     &           &   10 &  2.89e-08 &  1.39e-15 &      \\ 
1076 &  1.08e+04 &   10 &           &           & iters  \\ 
 \hdashline 
     &           &    1 &  5.74e-08 &  9.84e-09 &      \\ 
     &           &    2 &  2.04e-08 &  1.64e-15 &      \\ 
     &           &    3 &  2.04e-08 &  5.82e-16 &      \\ 
     &           &    4 &  5.73e-08 &  5.82e-16 &      \\ 
     &           &    5 &  2.04e-08 &  1.64e-15 &      \\ 
     &           &    6 &  2.04e-08 &  5.83e-16 &      \\ 
     &           &    7 &  2.04e-08 &  5.82e-16 &      \\ 
     &           &    8 &  5.73e-08 &  5.81e-16 &      \\ 
     &           &    9 &  2.04e-08 &  1.64e-15 &      \\ 
     &           &   10 &  2.04e-08 &  5.83e-16 &      \\ 
1077 &  1.08e+04 &   10 &           &           & iters  \\ 
 \hdashline 
     &           &    1 &  2.37e-08 &  4.69e-09 &      \\ 
     &           &    2 &  5.40e-08 &  6.78e-16 &      \\ 
     &           &    3 &  2.38e-08 &  1.54e-15 &      \\ 
     &           &    4 &  2.38e-08 &  6.79e-16 &      \\ 
     &           &    5 &  5.40e-08 &  6.78e-16 &      \\ 
     &           &    6 &  2.38e-08 &  1.54e-15 &      \\ 
     &           &    7 &  2.38e-08 &  6.79e-16 &      \\ 
     &           &    8 &  2.37e-08 &  6.78e-16 &      \\ 
     &           &    9 &  5.40e-08 &  6.77e-16 &      \\ 
     &           &   10 &  2.38e-08 &  1.54e-15 &      \\ 
1078 &  1.08e+04 &   10 &           &           & iters  \\ 
 \hdashline 
     &           &    1 &  3.18e-08 &  4.42e-09 &      \\ 
     &           &    2 &  4.60e-08 &  9.07e-16 &      \\ 
     &           &    3 &  3.18e-08 &  1.31e-15 &      \\ 
     &           &    4 &  3.17e-08 &  9.07e-16 &      \\ 
     &           &    5 &  4.60e-08 &  9.06e-16 &      \\ 
     &           &    6 &  3.18e-08 &  1.31e-15 &      \\ 
     &           &    7 &  4.60e-08 &  9.07e-16 &      \\ 
     &           &    8 &  3.18e-08 &  1.31e-15 &      \\ 
     &           &    9 &  3.17e-08 &  9.07e-16 &      \\ 
     &           &   10 &  4.60e-08 &  9.06e-16 &      \\ 
1079 &  1.08e+04 &   10 &           &           & iters  \\ 
 \hdashline 
     &           &    1 &  3.33e-09 &  5.91e-09 &      \\ 
     &           &    2 &  3.33e-09 &  9.50e-17 &      \\ 
     &           &    3 &  3.32e-09 &  9.49e-17 &      \\ 
     &           &    4 &  3.32e-09 &  9.48e-17 &      \\ 
     &           &    5 &  3.31e-09 &  9.46e-17 &      \\ 
     &           &    6 &  3.31e-09 &  9.45e-17 &      \\ 
     &           &    7 &  7.44e-08 &  9.44e-17 &      \\ 
     &           &    8 &  8.11e-08 &  2.12e-15 &      \\ 
     &           &    9 &  7.44e-08 &  2.31e-15 &      \\ 
     &           &   10 &  8.11e-08 &  2.12e-15 &      \\ 
1080 &  1.08e+04 &   10 &           &           & iters  \\ 
 \hdashline 
     &           &    1 &  5.31e-08 &  4.24e-10 &      \\ 
     &           &    2 &  2.47e-08 &  1.52e-15 &      \\ 
     &           &    3 &  5.30e-08 &  7.04e-16 &      \\ 
     &           &    4 &  2.47e-08 &  1.51e-15 &      \\ 
     &           &    5 &  2.47e-08 &  7.06e-16 &      \\ 
     &           &    6 &  5.30e-08 &  7.05e-16 &      \\ 
     &           &    7 &  2.47e-08 &  1.51e-15 &      \\ 
     &           &    8 &  2.47e-08 &  7.06e-16 &      \\ 
     &           &    9 &  2.47e-08 &  7.05e-16 &      \\ 
     &           &   10 &  5.31e-08 &  7.04e-16 &      \\ 
1081 &  1.08e+04 &   10 &           &           & iters  \\ 
 \hdashline 
     &           &    1 &  1.01e-08 &  3.19e-09 &      \\ 
     &           &    2 &  1.01e-08 &  2.87e-16 &      \\ 
     &           &    3 &  1.00e-08 &  2.87e-16 &      \\ 
     &           &    4 &  6.77e-08 &  2.87e-16 &      \\ 
     &           &    5 &  1.01e-08 &  1.93e-15 &      \\ 
     &           &    6 &  1.01e-08 &  2.89e-16 &      \\ 
     &           &    7 &  1.01e-08 &  2.89e-16 &      \\ 
     &           &    8 &  1.01e-08 &  2.88e-16 &      \\ 
     &           &    9 &  1.01e-08 &  2.88e-16 &      \\ 
     &           &   10 &  1.01e-08 &  2.87e-16 &      \\ 
1082 &  1.08e+04 &   10 &           &           & iters  \\ 
 \hdashline 
     &           &    1 &  2.43e-08 &  3.11e-09 &      \\ 
     &           &    2 &  2.43e-08 &  6.93e-16 &      \\ 
     &           &    3 &  2.42e-08 &  6.92e-16 &      \\ 
     &           &    4 &  5.35e-08 &  6.91e-16 &      \\ 
     &           &    5 &  2.43e-08 &  1.53e-15 &      \\ 
     &           &    6 &  2.42e-08 &  6.92e-16 &      \\ 
     &           &    7 &  5.35e-08 &  6.91e-16 &      \\ 
     &           &    8 &  2.43e-08 &  1.53e-15 &      \\ 
     &           &    9 &  2.42e-08 &  6.93e-16 &      \\ 
     &           &   10 &  5.35e-08 &  6.92e-16 &      \\ 
1083 &  1.08e+04 &   10 &           &           & iters  \\ 
 \hdashline 
     &           &    1 &  2.87e-08 &  7.53e-09 &      \\ 
     &           &    2 &  4.90e-08 &  8.20e-16 &      \\ 
     &           &    3 &  2.88e-08 &  1.40e-15 &      \\ 
     &           &    4 &  2.87e-08 &  8.21e-16 &      \\ 
     &           &    5 &  4.90e-08 &  8.20e-16 &      \\ 
     &           &    6 &  2.88e-08 &  1.40e-15 &      \\ 
     &           &    7 &  2.87e-08 &  8.21e-16 &      \\ 
     &           &    8 &  4.90e-08 &  8.20e-16 &      \\ 
     &           &    9 &  2.88e-08 &  1.40e-15 &      \\ 
     &           &   10 &  4.90e-08 &  8.21e-16 &      \\ 
1084 &  1.08e+04 &   10 &           &           & iters  \\ 
 \hdashline 
     &           &    1 &  7.38e-08 &  9.41e-09 &      \\ 
     &           &    2 &  4.00e-09 &  2.11e-15 &      \\ 
     &           &    3 &  4.00e-09 &  1.14e-16 &      \\ 
     &           &    4 &  3.99e-09 &  1.14e-16 &      \\ 
     &           &    5 &  3.99e-09 &  1.14e-16 &      \\ 
     &           &    6 &  3.98e-09 &  1.14e-16 &      \\ 
     &           &    7 &  3.97e-09 &  1.14e-16 &      \\ 
     &           &    8 &  3.97e-09 &  1.13e-16 &      \\ 
     &           &    9 &  7.37e-08 &  1.13e-16 &      \\ 
     &           &   10 &  8.18e-08 &  2.10e-15 &      \\ 
1085 &  1.08e+04 &   10 &           &           & iters  \\ 
 \hdashline 
     &           &    1 &  9.78e-09 &  3.16e-08 &      \\ 
     &           &    2 &  9.76e-09 &  2.79e-16 &      \\ 
     &           &    3 &  9.75e-09 &  2.79e-16 &      \\ 
     &           &    4 &  9.73e-09 &  2.78e-16 &      \\ 
     &           &    5 &  9.72e-09 &  2.78e-16 &      \\ 
     &           &    6 &  6.80e-08 &  2.77e-16 &      \\ 
     &           &    7 &  8.75e-08 &  1.94e-15 &      \\ 
     &           &    8 &  6.80e-08 &  2.50e-15 &      \\ 
     &           &    9 &  9.78e-09 &  1.94e-15 &      \\ 
     &           &   10 &  9.76e-09 &  2.79e-16 &      \\ 
1086 &  1.08e+04 &   10 &           &           & iters  \\ 
 \hdashline 
     &           &    1 &  2.81e-08 &  2.92e-08 &      \\ 
     &           &    2 &  4.96e-08 &  8.02e-16 &      \\ 
     &           &    3 &  2.81e-08 &  1.42e-15 &      \\ 
     &           &    4 &  2.81e-08 &  8.02e-16 &      \\ 
     &           &    5 &  4.97e-08 &  8.01e-16 &      \\ 
     &           &    6 &  2.81e-08 &  1.42e-15 &      \\ 
     &           &    7 &  4.96e-08 &  8.02e-16 &      \\ 
     &           &    8 &  2.81e-08 &  1.42e-15 &      \\ 
     &           &    9 &  2.81e-08 &  8.03e-16 &      \\ 
     &           &   10 &  4.96e-08 &  8.02e-16 &      \\ 
1087 &  1.09e+04 &   10 &           &           & iters  \\ 
 \hdashline 
     &           &    1 &  5.47e-08 &  1.33e-08 &      \\ 
     &           &    2 &  2.31e-08 &  1.56e-15 &      \\ 
     &           &    3 &  2.30e-08 &  6.59e-16 &      \\ 
     &           &    4 &  2.30e-08 &  6.58e-16 &      \\ 
     &           &    5 &  5.47e-08 &  6.57e-16 &      \\ 
     &           &    6 &  2.31e-08 &  1.56e-15 &      \\ 
     &           &    7 &  2.30e-08 &  6.58e-16 &      \\ 
     &           &    8 &  5.47e-08 &  6.57e-16 &      \\ 
     &           &    9 &  2.31e-08 &  1.56e-15 &      \\ 
     &           &   10 &  2.30e-08 &  6.58e-16 &      \\ 
1088 &  1.09e+04 &   10 &           &           & iters  \\ 
 \hdashline 
     &           &    1 &  1.55e-08 &  5.19e-09 &      \\ 
     &           &    2 &  6.23e-08 &  4.41e-16 &      \\ 
     &           &    3 &  1.55e-08 &  1.78e-15 &      \\ 
     &           &    4 &  1.55e-08 &  4.43e-16 &      \\ 
     &           &    5 &  1.55e-08 &  4.42e-16 &      \\ 
     &           &    6 &  1.55e-08 &  4.42e-16 &      \\ 
     &           &    7 &  6.23e-08 &  4.41e-16 &      \\ 
     &           &    8 &  1.55e-08 &  1.78e-15 &      \\ 
     &           &    9 &  1.55e-08 &  4.43e-16 &      \\ 
     &           &   10 &  1.55e-08 &  4.42e-16 &      \\ 
1089 &  1.09e+04 &   10 &           &           & iters  \\ 
 \hdashline 
     &           &    1 &  1.57e-08 &  1.19e-08 &      \\ 
     &           &    2 &  6.20e-08 &  4.49e-16 &      \\ 
     &           &    3 &  1.58e-08 &  1.77e-15 &      \\ 
     &           &    4 &  1.58e-08 &  4.51e-16 &      \\ 
     &           &    5 &  1.57e-08 &  4.50e-16 &      \\ 
     &           &    6 &  1.57e-08 &  4.49e-16 &      \\ 
     &           &    7 &  6.20e-08 &  4.49e-16 &      \\ 
     &           &    8 &  1.58e-08 &  1.77e-15 &      \\ 
     &           &    9 &  1.58e-08 &  4.50e-16 &      \\ 
     &           &   10 &  1.57e-08 &  4.50e-16 &      \\ 
1090 &  1.09e+04 &   10 &           &           & iters  \\ 
 \hdashline 
     &           &    1 &  2.05e-08 &  1.84e-09 &      \\ 
     &           &    2 &  2.04e-08 &  5.84e-16 &      \\ 
     &           &    3 &  2.04e-08 &  5.83e-16 &      \\ 
     &           &    4 &  5.73e-08 &  5.83e-16 &      \\ 
     &           &    5 &  2.05e-08 &  1.64e-15 &      \\ 
     &           &    6 &  2.04e-08 &  5.84e-16 &      \\ 
     &           &    7 &  5.73e-08 &  5.83e-16 &      \\ 
     &           &    8 &  2.05e-08 &  1.63e-15 &      \\ 
     &           &    9 &  2.05e-08 &  5.85e-16 &      \\ 
     &           &   10 &  2.04e-08 &  5.84e-16 &      \\ 
1091 &  1.09e+04 &   10 &           &           & iters  \\ 
 \hdashline 
     &           &    1 &  4.45e-08 &  7.13e-09 &      \\ 
     &           &    2 &  3.33e-08 &  1.27e-15 &      \\ 
     &           &    3 &  4.45e-08 &  9.50e-16 &      \\ 
     &           &    4 &  3.33e-08 &  1.27e-15 &      \\ 
     &           &    5 &  3.32e-08 &  9.50e-16 &      \\ 
     &           &    6 &  4.45e-08 &  9.49e-16 &      \\ 
     &           &    7 &  3.33e-08 &  1.27e-15 &      \\ 
     &           &    8 &  4.45e-08 &  9.49e-16 &      \\ 
     &           &    9 &  3.33e-08 &  1.27e-15 &      \\ 
     &           &   10 &  4.45e-08 &  9.50e-16 &      \\ 
1092 &  1.09e+04 &   10 &           &           & iters  \\ 
 \hdashline 
     &           &    1 &  4.10e-08 &  4.80e-09 &      \\ 
     &           &    2 &  2.08e-09 &  1.17e-15 &      \\ 
     &           &    3 &  2.08e-09 &  5.95e-17 &      \\ 
     &           &    4 &  2.08e-09 &  5.94e-17 &      \\ 
     &           &    5 &  2.07e-09 &  5.93e-17 &      \\ 
     &           &    6 &  2.07e-09 &  5.92e-17 &      \\ 
     &           &    7 &  2.07e-09 &  5.91e-17 &      \\ 
     &           &    8 &  2.07e-09 &  5.90e-17 &      \\ 
     &           &    9 &  2.06e-09 &  5.90e-17 &      \\ 
     &           &   10 &  2.06e-09 &  5.89e-17 &      \\ 
1093 &  1.09e+04 &   10 &           &           & iters  \\ 
 \hdashline 
     &           &    1 &  4.36e-08 &  7.82e-09 &      \\ 
     &           &    2 &  4.70e-09 &  1.24e-15 &      \\ 
     &           &    3 &  4.69e-09 &  1.34e-16 &      \\ 
     &           &    4 &  4.68e-09 &  1.34e-16 &      \\ 
     &           &    5 &  4.68e-09 &  1.34e-16 &      \\ 
     &           &    6 &  4.67e-09 &  1.33e-16 &      \\ 
     &           &    7 &  4.66e-09 &  1.33e-16 &      \\ 
     &           &    8 &  7.30e-08 &  1.33e-16 &      \\ 
     &           &    9 &  4.36e-08 &  2.08e-15 &      \\ 
     &           &   10 &  4.70e-09 &  1.24e-15 &      \\ 
1094 &  1.09e+04 &   10 &           &           & iters  \\ 
 \hdashline 
     &           &    1 &  3.63e-08 &  7.48e-10 &      \\ 
     &           &    2 &  4.15e-08 &  1.04e-15 &      \\ 
     &           &    3 &  3.63e-08 &  1.18e-15 &      \\ 
     &           &    4 &  4.15e-08 &  1.04e-15 &      \\ 
     &           &    5 &  3.63e-08 &  1.18e-15 &      \\ 
     &           &    6 &  4.14e-08 &  1.04e-15 &      \\ 
     &           &    7 &  3.63e-08 &  1.18e-15 &      \\ 
     &           &    8 &  4.14e-08 &  1.04e-15 &      \\ 
     &           &    9 &  3.63e-08 &  1.18e-15 &      \\ 
     &           &   10 &  4.14e-08 &  1.04e-15 &      \\ 
1095 &  1.09e+04 &   10 &           &           & iters  \\ 
 \hdashline 
     &           &    1 &  2.36e-08 &  6.15e-09 &      \\ 
     &           &    2 &  2.36e-08 &  6.74e-16 &      \\ 
     &           &    3 &  1.53e-08 &  6.73e-16 &      \\ 
     &           &    4 &  2.36e-08 &  4.36e-16 &      \\ 
     &           &    5 &  1.53e-08 &  6.73e-16 &      \\ 
     &           &    6 &  2.36e-08 &  4.37e-16 &      \\ 
     &           &    7 &  1.53e-08 &  6.72e-16 &      \\ 
     &           &    8 &  1.53e-08 &  4.37e-16 &      \\ 
     &           &    9 &  2.36e-08 &  4.37e-16 &      \\ 
     &           &   10 &  1.53e-08 &  6.73e-16 &      \\ 
1096 &  1.10e+04 &   10 &           &           & iters  \\ 
 \hdashline 
     &           &    1 &  5.08e-08 &  7.06e-09 &      \\ 
     &           &    2 &  2.70e-08 &  1.45e-15 &      \\ 
     &           &    3 &  2.70e-08 &  7.70e-16 &      \\ 
     &           &    4 &  5.08e-08 &  7.69e-16 &      \\ 
     &           &    5 &  2.70e-08 &  1.45e-15 &      \\ 
     &           &    6 &  2.70e-08 &  7.70e-16 &      \\ 
     &           &    7 &  5.08e-08 &  7.69e-16 &      \\ 
     &           &    8 &  2.70e-08 &  1.45e-15 &      \\ 
     &           &    9 &  2.69e-08 &  7.70e-16 &      \\ 
     &           &   10 &  5.08e-08 &  7.69e-16 &      \\ 
1097 &  1.10e+04 &   10 &           &           & iters  \\ 
 \hdashline 
     &           &    1 &  6.16e-08 &  2.86e-09 &      \\ 
     &           &    2 &  1.62e-08 &  1.76e-15 &      \\ 
     &           &    3 &  1.62e-08 &  4.63e-16 &      \\ 
     &           &    4 &  1.62e-08 &  4.62e-16 &      \\ 
     &           &    5 &  1.62e-08 &  4.62e-16 &      \\ 
     &           &    6 &  6.16e-08 &  4.61e-16 &      \\ 
     &           &    7 &  1.62e-08 &  1.76e-15 &      \\ 
     &           &    8 &  1.62e-08 &  4.63e-16 &      \\ 
     &           &    9 &  1.62e-08 &  4.62e-16 &      \\ 
     &           &   10 &  1.62e-08 &  4.62e-16 &      \\ 
1098 &  1.10e+04 &   10 &           &           & iters  \\ 
 \hdashline 
     &           &    1 &  5.69e-08 &  6.48e-09 &      \\ 
     &           &    2 &  1.80e-08 &  1.63e-15 &      \\ 
     &           &    3 &  5.97e-08 &  5.14e-16 &      \\ 
     &           &    4 &  1.81e-08 &  1.70e-15 &      \\ 
     &           &    5 &  1.80e-08 &  5.16e-16 &      \\ 
     &           &    6 &  1.80e-08 &  5.15e-16 &      \\ 
     &           &    7 &  5.97e-08 &  5.14e-16 &      \\ 
     &           &    8 &  1.81e-08 &  1.70e-15 &      \\ 
     &           &    9 &  1.81e-08 &  5.16e-16 &      \\ 
     &           &   10 &  1.80e-08 &  5.15e-16 &      \\ 
1099 &  1.10e+04 &   10 &           &           & iters  \\ 
 \hdashline 
     &           &    1 &  4.38e-08 &  1.28e-08 &      \\ 
     &           &    2 &  3.39e-08 &  1.25e-15 &      \\ 
     &           &    3 &  4.38e-08 &  9.68e-16 &      \\ 
     &           &    4 &  3.39e-08 &  1.25e-15 &      \\ 
     &           &    5 &  3.39e-08 &  9.69e-16 &      \\ 
     &           &    6 &  4.38e-08 &  9.67e-16 &      \\ 
     &           &    7 &  3.39e-08 &  1.25e-15 &      \\ 
     &           &    8 &  4.38e-08 &  9.68e-16 &      \\ 
     &           &    9 &  3.39e-08 &  1.25e-15 &      \\ 
     &           &   10 &  4.38e-08 &  9.68e-16 &      \\ 
1100 &  1.10e+04 &   10 &           &           & iters  \\ 
 \hdashline 
     &           &    1 &  5.24e-09 &  9.47e-09 &      \\ 
     &           &    2 &  5.23e-09 &  1.50e-16 &      \\ 
     &           &    3 &  5.23e-09 &  1.49e-16 &      \\ 
     &           &    4 &  5.22e-09 &  1.49e-16 &      \\ 
     &           &    5 &  5.21e-09 &  1.49e-16 &      \\ 
     &           &    6 &  5.20e-09 &  1.49e-16 &      \\ 
     &           &    7 &  5.20e-09 &  1.49e-16 &      \\ 
     &           &    8 &  5.19e-09 &  1.48e-16 &      \\ 
     &           &    9 &  5.18e-09 &  1.48e-16 &      \\ 
     &           &   10 &  5.17e-09 &  1.48e-16 &      \\ 
1101 &  1.10e+04 &   10 &           &           & iters  \\ 
 \hdashline 
     &           &    1 &  6.40e-09 &  5.63e-10 &      \\ 
     &           &    2 &  6.39e-09 &  1.83e-16 &      \\ 
     &           &    3 &  6.38e-09 &  1.82e-16 &      \\ 
     &           &    4 &  7.13e-08 &  1.82e-16 &      \\ 
     &           &    5 &  4.53e-08 &  2.04e-15 &      \\ 
     &           &    6 &  6.41e-09 &  1.29e-15 &      \\ 
     &           &    7 &  6.40e-09 &  1.83e-16 &      \\ 
     &           &    8 &  6.39e-09 &  1.83e-16 &      \\ 
     &           &    9 &  6.38e-09 &  1.82e-16 &      \\ 
     &           &   10 &  7.13e-08 &  1.82e-16 &      \\ 
1102 &  1.10e+04 &   10 &           &           & iters  \\ 
 \hdashline 
     &           &    1 &  3.75e-08 &  4.33e-09 &      \\ 
     &           &    2 &  4.02e-08 &  1.07e-15 &      \\ 
     &           &    3 &  3.75e-08 &  1.15e-15 &      \\ 
     &           &    4 &  4.02e-08 &  1.07e-15 &      \\ 
     &           &    5 &  3.75e-08 &  1.15e-15 &      \\ 
     &           &    6 &  4.02e-08 &  1.07e-15 &      \\ 
     &           &    7 &  3.75e-08 &  1.15e-15 &      \\ 
     &           &    8 &  3.75e-08 &  1.07e-15 &      \\ 
     &           &    9 &  4.03e-08 &  1.07e-15 &      \\ 
     &           &   10 &  3.75e-08 &  1.15e-15 &      \\ 
1103 &  1.10e+04 &   10 &           &           & iters  \\ 
 \hdashline 
     &           &    1 &  5.71e-08 &  2.04e-09 &      \\ 
     &           &    2 &  2.07e-08 &  1.63e-15 &      \\ 
     &           &    3 &  2.07e-08 &  5.91e-16 &      \\ 
     &           &    4 &  2.07e-08 &  5.90e-16 &      \\ 
     &           &    5 &  5.71e-08 &  5.90e-16 &      \\ 
     &           &    6 &  2.07e-08 &  1.63e-15 &      \\ 
     &           &    7 &  2.07e-08 &  5.91e-16 &      \\ 
     &           &    8 &  2.06e-08 &  5.90e-16 &      \\ 
     &           &    9 &  5.71e-08 &  5.89e-16 &      \\ 
     &           &   10 &  2.07e-08 &  1.63e-15 &      \\ 
1104 &  1.10e+04 &   10 &           &           & iters  \\ 
 \hdashline 
     &           &    1 &  1.45e-08 &  5.00e-09 &      \\ 
     &           &    2 &  1.44e-08 &  4.13e-16 &      \\ 
     &           &    3 &  6.33e-08 &  4.12e-16 &      \\ 
     &           &    4 &  1.45e-08 &  1.81e-15 &      \\ 
     &           &    5 &  1.45e-08 &  4.14e-16 &      \\ 
     &           &    6 &  1.45e-08 &  4.14e-16 &      \\ 
     &           &    7 &  1.45e-08 &  4.13e-16 &      \\ 
     &           &    8 &  1.44e-08 &  4.12e-16 &      \\ 
     &           &    9 &  6.33e-08 &  4.12e-16 &      \\ 
     &           &   10 &  1.45e-08 &  1.81e-15 &      \\ 
1105 &  1.10e+04 &   10 &           &           & iters  \\ 
 \hdashline 
     &           &    1 &  1.72e-08 &  3.21e-09 &      \\ 
     &           &    2 &  2.17e-08 &  4.91e-16 &      \\ 
     &           &    3 &  1.72e-08 &  6.18e-16 &      \\ 
     &           &    4 &  2.17e-08 &  4.91e-16 &      \\ 
     &           &    5 &  1.72e-08 &  6.18e-16 &      \\ 
     &           &    6 &  1.72e-08 &  4.91e-16 &      \\ 
     &           &    7 &  2.17e-08 &  4.91e-16 &      \\ 
     &           &    8 &  1.72e-08 &  6.19e-16 &      \\ 
     &           &    9 &  2.17e-08 &  4.91e-16 &      \\ 
     &           &   10 &  1.72e-08 &  6.19e-16 &      \\ 
1106 &  1.10e+04 &   10 &           &           & iters  \\ 
 \hdashline 
     &           &    1 &  2.70e-08 &  4.99e-09 &      \\ 
     &           &    2 &  1.19e-08 &  7.70e-16 &      \\ 
     &           &    3 &  1.19e-08 &  3.39e-16 &      \\ 
     &           &    4 &  1.19e-08 &  3.39e-16 &      \\ 
     &           &    5 &  2.70e-08 &  3.39e-16 &      \\ 
     &           &    6 &  1.19e-08 &  7.71e-16 &      \\ 
     &           &    7 &  1.19e-08 &  3.39e-16 &      \\ 
     &           &    8 &  2.70e-08 &  3.39e-16 &      \\ 
     &           &    9 &  1.19e-08 &  7.70e-16 &      \\ 
     &           &   10 &  1.19e-08 &  3.39e-16 &      \\ 
1107 &  1.11e+04 &   10 &           &           & iters  \\ 
 \hdashline 
     &           &    1 &  2.74e-08 &  5.38e-09 &      \\ 
     &           &    2 &  2.73e-08 &  7.81e-16 &      \\ 
     &           &    3 &  5.04e-08 &  7.79e-16 &      \\ 
     &           &    4 &  2.73e-08 &  1.44e-15 &      \\ 
     &           &    5 &  2.73e-08 &  7.80e-16 &      \\ 
     &           &    6 &  5.04e-08 &  7.79e-16 &      \\ 
     &           &    7 &  2.73e-08 &  1.44e-15 &      \\ 
     &           &    8 &  5.04e-08 &  7.80e-16 &      \\ 
     &           &    9 &  2.74e-08 &  1.44e-15 &      \\ 
     &           &   10 &  2.73e-08 &  7.81e-16 &      \\ 
1108 &  1.11e+04 &   10 &           &           & iters  \\ 
 \hdashline 
     &           &    1 &  3.88e-08 &  8.98e-10 &      \\ 
     &           &    2 &  3.90e-08 &  1.11e-15 &      \\ 
     &           &    3 &  3.88e-08 &  1.11e-15 &      \\ 
     &           &    4 &  3.90e-08 &  1.11e-15 &      \\ 
     &           &    5 &  3.88e-08 &  1.11e-15 &      \\ 
     &           &    6 &  3.90e-08 &  1.11e-15 &      \\ 
     &           &    7 &  3.88e-08 &  1.11e-15 &      \\ 
     &           &    8 &  3.90e-08 &  1.11e-15 &      \\ 
     &           &    9 &  3.88e-08 &  1.11e-15 &      \\ 
     &           &   10 &  3.90e-08 &  1.11e-15 &      \\ 
1109 &  1.11e+04 &   10 &           &           & iters  \\ 
 \hdashline 
     &           &    1 &  5.39e-08 &  2.77e-10 &      \\ 
     &           &    2 &  2.38e-08 &  1.54e-15 &      \\ 
     &           &    3 &  2.38e-08 &  6.80e-16 &      \\ 
     &           &    4 &  2.38e-08 &  6.79e-16 &      \\ 
     &           &    5 &  5.40e-08 &  6.78e-16 &      \\ 
     &           &    6 &  2.38e-08 &  1.54e-15 &      \\ 
     &           &    7 &  2.38e-08 &  6.79e-16 &      \\ 
     &           &    8 &  5.40e-08 &  6.78e-16 &      \\ 
     &           &    9 &  2.38e-08 &  1.54e-15 &      \\ 
     &           &   10 &  2.38e-08 &  6.79e-16 &      \\ 
1110 &  1.11e+04 &   10 &           &           & iters  \\ 
 \hdashline 
     &           &    1 &  7.61e-08 &  3.05e-09 &      \\ 
     &           &    2 &  7.94e-08 &  2.17e-15 &      \\ 
     &           &    3 &  7.61e-08 &  2.27e-15 &      \\ 
     &           &    4 &  7.94e-08 &  2.17e-15 &      \\ 
     &           &    5 &  7.61e-08 &  2.27e-15 &      \\ 
     &           &    6 &  7.94e-08 &  2.17e-15 &      \\ 
     &           &    7 &  7.61e-08 &  2.27e-15 &      \\ 
     &           &    8 &  7.94e-08 &  2.17e-15 &      \\ 
     &           &    9 &  7.61e-08 &  2.27e-15 &      \\ 
     &           &   10 &  7.94e-08 &  2.17e-15 &      \\ 
1111 &  1.11e+04 &   10 &           &           & iters  \\ 
 \hdashline 
     &           &    1 &  8.59e-08 &  8.07e-09 &      \\ 
     &           &    2 &  6.96e-08 &  2.45e-15 &      \\ 
     &           &    3 &  8.18e-09 &  1.99e-15 &      \\ 
     &           &    4 &  8.17e-09 &  2.33e-16 &      \\ 
     &           &    5 &  8.15e-09 &  2.33e-16 &      \\ 
     &           &    6 &  8.14e-09 &  2.33e-16 &      \\ 
     &           &    7 &  8.13e-09 &  2.32e-16 &      \\ 
     &           &    8 &  8.12e-09 &  2.32e-16 &      \\ 
     &           &    9 &  8.11e-09 &  2.32e-16 &      \\ 
     &           &   10 &  6.96e-08 &  2.31e-16 &      \\ 
1112 &  1.11e+04 &   10 &           &           & iters  \\ 
 \hdashline 
     &           &    1 &  4.18e-08 &  9.39e-09 &      \\ 
     &           &    2 &  3.59e-08 &  1.19e-15 &      \\ 
     &           &    3 &  4.18e-08 &  1.03e-15 &      \\ 
     &           &    4 &  3.59e-08 &  1.19e-15 &      \\ 
     &           &    5 &  4.18e-08 &  1.03e-15 &      \\ 
     &           &    6 &  3.59e-08 &  1.19e-15 &      \\ 
     &           &    7 &  3.59e-08 &  1.03e-15 &      \\ 
     &           &    8 &  4.19e-08 &  1.02e-15 &      \\ 
     &           &    9 &  3.59e-08 &  1.19e-15 &      \\ 
     &           &   10 &  4.18e-08 &  1.02e-15 &      \\ 
1113 &  1.11e+04 &   10 &           &           & iters  \\ 
 \hdashline 
     &           &    1 &  3.92e-08 &  6.99e-09 &      \\ 
     &           &    2 &  3.86e-08 &  1.12e-15 &      \\ 
     &           &    3 &  3.92e-08 &  1.10e-15 &      \\ 
     &           &    4 &  3.86e-08 &  1.12e-15 &      \\ 
     &           &    5 &  3.92e-08 &  1.10e-15 &      \\ 
     &           &    6 &  3.86e-08 &  1.12e-15 &      \\ 
     &           &    7 &  3.92e-08 &  1.10e-15 &      \\ 
     &           &    8 &  3.86e-08 &  1.12e-15 &      \\ 
     &           &    9 &  3.92e-08 &  1.10e-15 &      \\ 
     &           &   10 &  3.86e-08 &  1.12e-15 &      \\ 
1114 &  1.11e+04 &   10 &           &           & iters  \\ 
 \hdashline 
     &           &    1 &  6.14e-08 &  5.75e-09 &      \\ 
     &           &    2 &  1.64e-08 &  1.75e-15 &      \\ 
     &           &    3 &  1.63e-08 &  4.67e-16 &      \\ 
     &           &    4 &  1.63e-08 &  4.67e-16 &      \\ 
     &           &    5 &  1.63e-08 &  4.66e-16 &      \\ 
     &           &    6 &  6.14e-08 &  4.65e-16 &      \\ 
     &           &    7 &  1.64e-08 &  1.75e-15 &      \\ 
     &           &    8 &  1.63e-08 &  4.67e-16 &      \\ 
     &           &    9 &  1.63e-08 &  4.66e-16 &      \\ 
     &           &   10 &  6.14e-08 &  4.66e-16 &      \\ 
1115 &  1.11e+04 &   10 &           &           & iters  \\ 
 \hdashline 
     &           &    1 &  6.74e-08 &  6.79e-09 &      \\ 
     &           &    2 &  1.04e-08 &  1.92e-15 &      \\ 
     &           &    3 &  1.04e-08 &  2.96e-16 &      \\ 
     &           &    4 &  1.03e-08 &  2.96e-16 &      \\ 
     &           &    5 &  1.03e-08 &  2.95e-16 &      \\ 
     &           &    6 &  1.03e-08 &  2.95e-16 &      \\ 
     &           &    7 &  1.03e-08 &  2.95e-16 &      \\ 
     &           &    8 &  6.74e-08 &  2.94e-16 &      \\ 
     &           &    9 &  1.04e-08 &  1.92e-15 &      \\ 
     &           &   10 &  1.04e-08 &  2.96e-16 &      \\ 
1116 &  1.12e+04 &   10 &           &           & iters  \\ 
 \hdashline 
     &           &    1 &  8.62e-09 &  6.74e-09 &      \\ 
     &           &    2 &  8.61e-09 &  2.46e-16 &      \\ 
     &           &    3 &  8.60e-09 &  2.46e-16 &      \\ 
     &           &    4 &  8.58e-09 &  2.45e-16 &      \\ 
     &           &    5 &  6.91e-08 &  2.45e-16 &      \\ 
     &           &    6 &  8.64e-08 &  1.97e-15 &      \\ 
     &           &    7 &  6.91e-08 &  2.46e-15 &      \\ 
     &           &    8 &  8.65e-09 &  1.97e-15 &      \\ 
     &           &    9 &  8.63e-09 &  2.47e-16 &      \\ 
     &           &   10 &  8.62e-09 &  2.46e-16 &      \\ 
1117 &  1.12e+04 &   10 &           &           & iters  \\ 
 \hdashline 
     &           &    1 &  3.02e-08 &  3.54e-09 &      \\ 
     &           &    2 &  3.02e-08 &  8.62e-16 &      \\ 
     &           &    3 &  4.76e-08 &  8.61e-16 &      \\ 
     &           &    4 &  3.02e-08 &  1.36e-15 &      \\ 
     &           &    5 &  4.75e-08 &  8.61e-16 &      \\ 
     &           &    6 &  3.02e-08 &  1.36e-15 &      \\ 
     &           &    7 &  3.02e-08 &  8.62e-16 &      \\ 
     &           &    8 &  4.76e-08 &  8.61e-16 &      \\ 
     &           &    9 &  3.02e-08 &  1.36e-15 &      \\ 
     &           &   10 &  4.75e-08 &  8.62e-16 &      \\ 
1118 &  1.12e+04 &   10 &           &           & iters  \\ 
 \hdashline 
     &           &    1 &  6.24e-08 &  5.90e-10 &      \\ 
     &           &    2 &  1.54e-08 &  1.78e-15 &      \\ 
     &           &    3 &  1.54e-08 &  4.40e-16 &      \\ 
     &           &    4 &  1.54e-08 &  4.39e-16 &      \\ 
     &           &    5 &  1.53e-08 &  4.39e-16 &      \\ 
     &           &    6 &  6.24e-08 &  4.38e-16 &      \\ 
     &           &    7 &  1.54e-08 &  1.78e-15 &      \\ 
     &           &    8 &  1.54e-08 &  4.40e-16 &      \\ 
     &           &    9 &  1.54e-08 &  4.39e-16 &      \\ 
     &           &   10 &  1.53e-08 &  4.39e-16 &      \\ 
1119 &  1.12e+04 &   10 &           &           & iters  \\ 
 \hdashline 
     &           &    1 &  2.96e-08 &  1.47e-09 &      \\ 
     &           &    2 &  4.82e-08 &  8.44e-16 &      \\ 
     &           &    3 &  2.96e-08 &  1.37e-15 &      \\ 
     &           &    4 &  4.81e-08 &  8.45e-16 &      \\ 
     &           &    5 &  2.96e-08 &  1.37e-15 &      \\ 
     &           &    6 &  2.96e-08 &  8.46e-16 &      \\ 
     &           &    7 &  4.81e-08 &  8.44e-16 &      \\ 
     &           &    8 &  2.96e-08 &  1.37e-15 &      \\ 
     &           &    9 &  2.96e-08 &  8.45e-16 &      \\ 
     &           &   10 &  4.82e-08 &  8.44e-16 &      \\ 
1120 &  1.12e+04 &   10 &           &           & iters  \\ 
 \hdashline 
     &           &    1 &  2.13e-08 &  7.34e-09 &      \\ 
     &           &    2 &  5.64e-08 &  6.08e-16 &      \\ 
     &           &    3 &  2.13e-08 &  1.61e-15 &      \\ 
     &           &    4 &  2.13e-08 &  6.09e-16 &      \\ 
     &           &    5 &  2.13e-08 &  6.08e-16 &      \\ 
     &           &    6 &  5.64e-08 &  6.07e-16 &      \\ 
     &           &    7 &  2.13e-08 &  1.61e-15 &      \\ 
     &           &    8 &  2.13e-08 &  6.09e-16 &      \\ 
     &           &    9 &  2.13e-08 &  6.08e-16 &      \\ 
     &           &   10 &  5.64e-08 &  6.07e-16 &      \\ 
1121 &  1.12e+04 &   10 &           &           & iters  \\ 
 \hdashline 
     &           &    1 &  4.50e-08 &  1.11e-08 &      \\ 
     &           &    2 &  3.28e-08 &  1.28e-15 &      \\ 
     &           &    3 &  4.50e-08 &  9.35e-16 &      \\ 
     &           &    4 &  3.28e-08 &  1.28e-15 &      \\ 
     &           &    5 &  4.50e-08 &  9.35e-16 &      \\ 
     &           &    6 &  3.28e-08 &  1.28e-15 &      \\ 
     &           &    7 &  3.27e-08 &  9.36e-16 &      \\ 
     &           &    8 &  4.50e-08 &  9.34e-16 &      \\ 
     &           &    9 &  3.28e-08 &  1.28e-15 &      \\ 
     &           &   10 &  4.50e-08 &  9.35e-16 &      \\ 
1122 &  1.12e+04 &   10 &           &           & iters  \\ 
 \hdashline 
     &           &    1 &  1.71e-08 &  6.79e-09 &      \\ 
     &           &    2 &  1.71e-08 &  4.89e-16 &      \\ 
     &           &    3 &  1.71e-08 &  4.88e-16 &      \\ 
     &           &    4 &  6.06e-08 &  4.87e-16 &      \\ 
     &           &    5 &  1.71e-08 &  1.73e-15 &      \\ 
     &           &    6 &  1.71e-08 &  4.89e-16 &      \\ 
     &           &    7 &  1.71e-08 &  4.89e-16 &      \\ 
     &           &    8 &  1.71e-08 &  4.88e-16 &      \\ 
     &           &    9 &  6.06e-08 &  4.87e-16 &      \\ 
     &           &   10 &  1.71e-08 &  1.73e-15 &      \\ 
1123 &  1.12e+04 &   10 &           &           & iters  \\ 
 \hdashline 
     &           &    1 &  1.14e-08 &  4.46e-09 &      \\ 
     &           &    2 &  1.14e-08 &  3.25e-16 &      \\ 
     &           &    3 &  1.14e-08 &  3.25e-16 &      \\ 
     &           &    4 &  1.14e-08 &  3.24e-16 &      \\ 
     &           &    5 &  1.13e-08 &  3.24e-16 &      \\ 
     &           &    6 &  6.64e-08 &  3.23e-16 &      \\ 
     &           &    7 &  8.91e-08 &  1.89e-15 &      \\ 
     &           &    8 &  6.64e-08 &  2.54e-15 &      \\ 
     &           &    9 &  1.14e-08 &  1.90e-15 &      \\ 
     &           &   10 &  1.14e-08 &  3.25e-16 &      \\ 
1124 &  1.12e+04 &   10 &           &           & iters  \\ 
 \hdashline 
     &           &    1 &  6.05e-08 &  4.73e-09 &      \\ 
     &           &    2 &  1.73e-08 &  1.73e-15 &      \\ 
     &           &    3 &  1.73e-08 &  4.94e-16 &      \\ 
     &           &    4 &  1.73e-08 &  4.94e-16 &      \\ 
     &           &    5 &  1.72e-08 &  4.93e-16 &      \\ 
     &           &    6 &  6.05e-08 &  4.92e-16 &      \\ 
     &           &    7 &  1.73e-08 &  1.73e-15 &      \\ 
     &           &    8 &  1.73e-08 &  4.94e-16 &      \\ 
     &           &    9 &  1.73e-08 &  4.93e-16 &      \\ 
     &           &   10 &  6.05e-08 &  4.92e-16 &      \\ 
1125 &  1.12e+04 &   10 &           &           & iters  \\ 
 \hdashline 
     &           &    1 &  5.90e-08 &  1.76e-09 &      \\ 
     &           &    2 &  2.01e-08 &  1.68e-15 &      \\ 
     &           &    3 &  5.76e-08 &  5.74e-16 &      \\ 
     &           &    4 &  2.02e-08 &  1.64e-15 &      \\ 
     &           &    5 &  2.01e-08 &  5.75e-16 &      \\ 
     &           &    6 &  2.01e-08 &  5.74e-16 &      \\ 
     &           &    7 &  5.76e-08 &  5.74e-16 &      \\ 
     &           &    8 &  2.02e-08 &  1.64e-15 &      \\ 
     &           &    9 &  2.01e-08 &  5.75e-16 &      \\ 
     &           &   10 &  2.01e-08 &  5.74e-16 &      \\ 
1126 &  1.12e+04 &   10 &           &           & iters  \\ 
 \hdashline 
     &           &    1 &  3.04e-08 &  1.98e-09 &      \\ 
     &           &    2 &  3.04e-08 &  8.68e-16 &      \\ 
     &           &    3 &  4.74e-08 &  8.67e-16 &      \\ 
     &           &    4 &  3.04e-08 &  1.35e-15 &      \\ 
     &           &    5 &  4.73e-08 &  8.67e-16 &      \\ 
     &           &    6 &  3.04e-08 &  1.35e-15 &      \\ 
     &           &    7 &  3.04e-08 &  8.68e-16 &      \\ 
     &           &    8 &  4.74e-08 &  8.67e-16 &      \\ 
     &           &    9 &  3.04e-08 &  1.35e-15 &      \\ 
     &           &   10 &  3.03e-08 &  8.67e-16 &      \\ 
1127 &  1.13e+04 &   10 &           &           & iters  \\ 
 \hdashline 
     &           &    1 &  1.53e-08 &  5.22e-09 &      \\ 
     &           &    2 &  1.53e-08 &  4.37e-16 &      \\ 
     &           &    3 &  1.53e-08 &  4.36e-16 &      \\ 
     &           &    4 &  1.52e-08 &  4.36e-16 &      \\ 
     &           &    5 &  6.25e-08 &  4.35e-16 &      \\ 
     &           &    6 &  1.53e-08 &  1.78e-15 &      \\ 
     &           &    7 &  1.53e-08 &  4.37e-16 &      \\ 
     &           &    8 &  1.53e-08 &  4.36e-16 &      \\ 
     &           &    9 &  1.52e-08 &  4.36e-16 &      \\ 
     &           &   10 &  1.52e-08 &  4.35e-16 &      \\ 
1128 &  1.13e+04 &   10 &           &           & iters  \\ 
 \hdashline 
     &           &    1 &  3.15e-08 &  6.13e-09 &      \\ 
     &           &    2 &  3.15e-08 &  8.99e-16 &      \\ 
     &           &    3 &  4.63e-08 &  8.98e-16 &      \\ 
     &           &    4 &  3.15e-08 &  1.32e-15 &      \\ 
     &           &    5 &  4.62e-08 &  8.99e-16 &      \\ 
     &           &    6 &  3.15e-08 &  1.32e-15 &      \\ 
     &           &    7 &  3.15e-08 &  8.99e-16 &      \\ 
     &           &    8 &  4.63e-08 &  8.98e-16 &      \\ 
     &           &    9 &  3.15e-08 &  1.32e-15 &      \\ 
     &           &   10 &  4.62e-08 &  8.99e-16 &      \\ 
1129 &  1.13e+04 &   10 &           &           & iters  \\ 
 \hdashline 
     &           &    1 &  4.25e-08 &  4.99e-09 &      \\ 
     &           &    2 &  3.64e-09 &  1.21e-15 &      \\ 
     &           &    3 &  3.64e-09 &  1.04e-16 &      \\ 
     &           &    4 &  3.63e-09 &  1.04e-16 &      \\ 
     &           &    5 &  3.63e-09 &  1.04e-16 &      \\ 
     &           &    6 &  3.62e-09 &  1.04e-16 &      \\ 
     &           &    7 &  3.62e-09 &  1.03e-16 &      \\ 
     &           &    8 &  3.61e-09 &  1.03e-16 &      \\ 
     &           &    9 &  3.61e-09 &  1.03e-16 &      \\ 
     &           &   10 &  3.60e-09 &  1.03e-16 &      \\ 
1130 &  1.13e+04 &   10 &           &           & iters  \\ 
 \hdashline 
     &           &    1 &  9.19e-09 &  3.61e-09 &      \\ 
     &           &    2 &  9.18e-09 &  2.62e-16 &      \\ 
     &           &    3 &  9.16e-09 &  2.62e-16 &      \\ 
     &           &    4 &  9.15e-09 &  2.62e-16 &      \\ 
     &           &    5 &  9.14e-09 &  2.61e-16 &      \\ 
     &           &    6 &  9.13e-09 &  2.61e-16 &      \\ 
     &           &    7 &  2.97e-08 &  2.60e-16 &      \\ 
     &           &    8 &  9.15e-09 &  8.49e-16 &      \\ 
     &           &    9 &  9.14e-09 &  2.61e-16 &      \\ 
     &           &   10 &  9.13e-09 &  2.61e-16 &      \\ 
1131 &  1.13e+04 &   10 &           &           & iters  \\ 
 \hdashline 
     &           &    1 &  1.89e-08 &  1.26e-09 &      \\ 
     &           &    2 &  1.89e-08 &  5.41e-16 &      \\ 
     &           &    3 &  1.99e-08 &  5.40e-16 &      \\ 
     &           &    4 &  1.89e-08 &  5.69e-16 &      \\ 
     &           &    5 &  1.99e-08 &  5.40e-16 &      \\ 
     &           &    6 &  1.89e-08 &  5.69e-16 &      \\ 
     &           &    7 &  1.99e-08 &  5.40e-16 &      \\ 
     &           &    8 &  1.89e-08 &  5.69e-16 &      \\ 
     &           &    9 &  1.99e-08 &  5.40e-16 &      \\ 
     &           &   10 &  1.89e-08 &  5.69e-16 &      \\ 
1132 &  1.13e+04 &   10 &           &           & iters  \\ 
 \hdashline 
     &           &    1 &  3.33e-09 &  2.35e-09 &      \\ 
     &           &    2 &  3.33e-09 &  9.52e-17 &      \\ 
     &           &    3 &  3.32e-09 &  9.50e-17 &      \\ 
     &           &    4 &  3.32e-09 &  9.49e-17 &      \\ 
     &           &    5 &  3.32e-09 &  9.48e-17 &      \\ 
     &           &    6 &  3.31e-09 &  9.46e-17 &      \\ 
     &           &    7 &  3.31e-09 &  9.45e-17 &      \\ 
     &           &    8 &  3.30e-09 &  9.43e-17 &      \\ 
     &           &    9 &  3.30e-09 &  9.42e-17 &      \\ 
     &           &   10 &  3.29e-09 &  9.41e-17 &      \\ 
1133 &  1.13e+04 &   10 &           &           & iters  \\ 
 \hdashline 
     &           &    1 &  7.85e-09 &  2.49e-09 &      \\ 
     &           &    2 &  7.84e-09 &  2.24e-16 &      \\ 
     &           &    3 &  6.99e-08 &  2.24e-16 &      \\ 
     &           &    4 &  4.68e-08 &  1.99e-15 &      \\ 
     &           &    5 &  7.86e-09 &  1.33e-15 &      \\ 
     &           &    6 &  7.85e-09 &  2.24e-16 &      \\ 
     &           &    7 &  7.84e-09 &  2.24e-16 &      \\ 
     &           &    8 &  6.99e-08 &  2.24e-16 &      \\ 
     &           &    9 &  4.68e-08 &  1.99e-15 &      \\ 
     &           &   10 &  7.86e-09 &  1.33e-15 &      \\ 
1134 &  1.13e+04 &   10 &           &           & iters  \\ 
 \hdashline 
     &           &    1 &  1.73e-08 &  1.19e-09 &      \\ 
     &           &    2 &  1.73e-08 &  4.95e-16 &      \\ 
     &           &    3 &  2.16e-08 &  4.94e-16 &      \\ 
     &           &    4 &  1.73e-08 &  6.15e-16 &      \\ 
     &           &    5 &  2.16e-08 &  4.94e-16 &      \\ 
     &           &    6 &  1.73e-08 &  6.15e-16 &      \\ 
     &           &    7 &  2.16e-08 &  4.94e-16 &      \\ 
     &           &    8 &  1.73e-08 &  6.15e-16 &      \\ 
     &           &    9 &  1.73e-08 &  4.94e-16 &      \\ 
     &           &   10 &  2.16e-08 &  4.94e-16 &      \\ 
1135 &  1.13e+04 &   10 &           &           & iters  \\ 
 \hdashline 
     &           &    1 &  4.23e-08 &  2.67e-09 &      \\ 
     &           &    2 &  3.55e-08 &  1.21e-15 &      \\ 
     &           &    3 &  3.54e-08 &  1.01e-15 &      \\ 
     &           &    4 &  4.23e-08 &  1.01e-15 &      \\ 
     &           &    5 &  3.54e-08 &  1.21e-15 &      \\ 
     &           &    6 &  4.23e-08 &  1.01e-15 &      \\ 
     &           &    7 &  3.54e-08 &  1.21e-15 &      \\ 
     &           &    8 &  4.23e-08 &  1.01e-15 &      \\ 
     &           &    9 &  3.55e-08 &  1.21e-15 &      \\ 
     &           &   10 &  4.23e-08 &  1.01e-15 &      \\ 
1136 &  1.14e+04 &   10 &           &           & iters  \\ 
 \hdashline 
     &           &    1 &  4.20e-09 &  3.29e-09 &      \\ 
     &           &    2 &  4.20e-09 &  1.20e-16 &      \\ 
     &           &    3 &  4.19e-09 &  1.20e-16 &      \\ 
     &           &    4 &  4.18e-09 &  1.20e-16 &      \\ 
     &           &    5 &  4.18e-09 &  1.19e-16 &      \\ 
     &           &    6 &  4.17e-09 &  1.19e-16 &      \\ 
     &           &    7 &  4.17e-09 &  1.19e-16 &      \\ 
     &           &    8 &  4.16e-09 &  1.19e-16 &      \\ 
     &           &    9 &  4.15e-09 &  1.19e-16 &      \\ 
     &           &   10 &  4.15e-09 &  1.19e-16 &      \\ 
1137 &  1.14e+04 &   10 &           &           & iters  \\ 
 \hdashline 
     &           &    1 &  1.92e-08 &  4.41e-09 &      \\ 
     &           &    2 &  5.85e-08 &  5.48e-16 &      \\ 
     &           &    3 &  1.93e-08 &  1.67e-15 &      \\ 
     &           &    4 &  1.92e-08 &  5.50e-16 &      \\ 
     &           &    5 &  1.92e-08 &  5.49e-16 &      \\ 
     &           &    6 &  5.85e-08 &  5.48e-16 &      \\ 
     &           &    7 &  1.93e-08 &  1.67e-15 &      \\ 
     &           &    8 &  1.92e-08 &  5.50e-16 &      \\ 
     &           &    9 &  1.92e-08 &  5.49e-16 &      \\ 
     &           &   10 &  1.92e-08 &  5.48e-16 &      \\ 
1138 &  1.14e+04 &   10 &           &           & iters  \\ 
 \hdashline 
     &           &    1 &  4.41e-08 &  6.75e-09 &      \\ 
     &           &    2 &  4.40e-08 &  1.26e-15 &      \\ 
     &           &    3 &  3.38e-08 &  1.26e-15 &      \\ 
     &           &    4 &  3.37e-08 &  9.63e-16 &      \\ 
     &           &    5 &  4.40e-08 &  9.62e-16 &      \\ 
     &           &    6 &  3.37e-08 &  1.26e-15 &      \\ 
     &           &    7 &  4.40e-08 &  9.63e-16 &      \\ 
     &           &    8 &  3.37e-08 &  1.26e-15 &      \\ 
     &           &    9 &  4.40e-08 &  9.63e-16 &      \\ 
     &           &   10 &  3.38e-08 &  1.26e-15 &      \\ 
1139 &  1.14e+04 &   10 &           &           & iters  \\ 
 \hdashline 
     &           &    1 &  4.80e-08 &  5.47e-09 &      \\ 
     &           &    2 &  2.98e-08 &  1.37e-15 &      \\ 
     &           &    3 &  4.80e-08 &  8.50e-16 &      \\ 
     &           &    4 &  2.98e-08 &  1.37e-15 &      \\ 
     &           &    5 &  2.97e-08 &  8.50e-16 &      \\ 
     &           &    6 &  4.80e-08 &  8.49e-16 &      \\ 
     &           &    7 &  2.98e-08 &  1.37e-15 &      \\ 
     &           &    8 &  2.97e-08 &  8.50e-16 &      \\ 
     &           &    9 &  4.80e-08 &  8.49e-16 &      \\ 
     &           &   10 &  2.98e-08 &  1.37e-15 &      \\ 
1140 &  1.14e+04 &   10 &           &           & iters  \\ 
 \hdashline 
     &           &    1 &  5.60e-09 &  3.71e-09 &      \\ 
     &           &    2 &  5.59e-09 &  1.60e-16 &      \\ 
     &           &    3 &  5.58e-09 &  1.59e-16 &      \\ 
     &           &    4 &  5.57e-09 &  1.59e-16 &      \\ 
     &           &    5 &  5.56e-09 &  1.59e-16 &      \\ 
     &           &    6 &  5.56e-09 &  1.59e-16 &      \\ 
     &           &    7 &  5.55e-09 &  1.59e-16 &      \\ 
     &           &    8 &  5.54e-09 &  1.58e-16 &      \\ 
     &           &    9 &  5.53e-09 &  1.58e-16 &      \\ 
     &           &   10 &  7.22e-08 &  1.58e-16 &      \\ 
1141 &  1.14e+04 &   10 &           &           & iters  \\ 
 \hdashline 
     &           &    1 &  3.09e-10 &  3.53e-09 &      \\ 
     &           &    2 &  3.08e-10 &  8.81e-18 &      \\ 
     &           &    3 &  3.08e-10 &  8.79e-18 &      \\ 
     &           &    4 &  3.07e-10 &  8.78e-18 &      \\ 
     &           &    5 &  3.07e-10 &  8.77e-18 &      \\ 
     &           &    6 &  3.06e-10 &  8.75e-18 &      \\ 
     &           &    7 &  3.06e-10 &  8.74e-18 &      \\ 
     &           &    8 &  3.05e-10 &  8.73e-18 &      \\ 
     &           &    9 &  3.05e-10 &  8.72e-18 &      \\ 
     &           &   10 &  3.05e-10 &  8.70e-18 &      \\ 
1142 &  1.14e+04 &   10 &           &           & iters  \\ 
 \hdashline 
     &           &    1 &  3.56e-08 &  6.83e-10 &      \\ 
     &           &    2 &  4.21e-08 &  1.02e-15 &      \\ 
     &           &    3 &  3.56e-08 &  1.20e-15 &      \\ 
     &           &    4 &  4.21e-08 &  1.02e-15 &      \\ 
     &           &    5 &  3.56e-08 &  1.20e-15 &      \\ 
     &           &    6 &  4.21e-08 &  1.02e-15 &      \\ 
     &           &    7 &  3.56e-08 &  1.20e-15 &      \\ 
     &           &    8 &  4.21e-08 &  1.02e-15 &      \\ 
     &           &    9 &  3.56e-08 &  1.20e-15 &      \\ 
     &           &   10 &  3.56e-08 &  1.02e-15 &      \\ 
1143 &  1.14e+04 &   10 &           &           & iters  \\ 
 \hdashline 
     &           &    1 &  8.89e-08 &  1.05e-09 &      \\ 
     &           &    2 &  6.66e-08 &  2.54e-15 &      \\ 
     &           &    3 &  1.12e-08 &  1.90e-15 &      \\ 
     &           &    4 &  1.12e-08 &  3.20e-16 &      \\ 
     &           &    5 &  1.12e-08 &  3.20e-16 &      \\ 
     &           &    6 &  1.12e-08 &  3.19e-16 &      \\ 
     &           &    7 &  1.11e-08 &  3.19e-16 &      \\ 
     &           &    8 &  6.66e-08 &  3.18e-16 &      \\ 
     &           &    9 &  8.89e-08 &  1.90e-15 &      \\ 
     &           &   10 &  6.66e-08 &  2.54e-15 &      \\ 
1144 &  1.14e+04 &   10 &           &           & iters  \\ 
 \hdashline 
     &           &    1 &  8.30e-08 &  1.32e-09 &      \\ 
     &           &    2 &  7.25e-08 &  2.37e-15 &      \\ 
     &           &    3 &  5.30e-09 &  2.07e-15 &      \\ 
     &           &    4 &  5.29e-09 &  1.51e-16 &      \\ 
     &           &    5 &  5.28e-09 &  1.51e-16 &      \\ 
     &           &    6 &  5.27e-09 &  1.51e-16 &      \\ 
     &           &    7 &  5.27e-09 &  1.50e-16 &      \\ 
     &           &    8 &  5.26e-09 &  1.50e-16 &      \\ 
     &           &    9 &  5.25e-09 &  1.50e-16 &      \\ 
     &           &   10 &  5.24e-09 &  1.50e-16 &      \\ 
1145 &  1.14e+04 &   10 &           &           & iters  \\ 
 \hdashline 
     &           &    1 &  2.55e-08 &  4.46e-09 &      \\ 
     &           &    2 &  5.22e-08 &  7.27e-16 &      \\ 
     &           &    3 &  2.55e-08 &  1.49e-15 &      \\ 
     &           &    4 &  2.55e-08 &  7.28e-16 &      \\ 
     &           &    5 &  5.22e-08 &  7.27e-16 &      \\ 
     &           &    6 &  2.55e-08 &  1.49e-15 &      \\ 
     &           &    7 &  2.55e-08 &  7.29e-16 &      \\ 
     &           &    8 &  5.22e-08 &  7.27e-16 &      \\ 
     &           &    9 &  2.55e-08 &  1.49e-15 &      \\ 
     &           &   10 &  2.55e-08 &  7.29e-16 &      \\ 
1146 &  1.14e+04 &   10 &           &           & iters  \\ 
 \hdashline 
     &           &    1 &  8.61e-08 &  8.21e-09 &      \\ 
     &           &    2 &  6.94e-08 &  2.46e-15 &      \\ 
     &           &    3 &  8.35e-09 &  1.98e-15 &      \\ 
     &           &    4 &  8.34e-09 &  2.38e-16 &      \\ 
     &           &    5 &  8.33e-09 &  2.38e-16 &      \\ 
     &           &    6 &  8.32e-09 &  2.38e-16 &      \\ 
     &           &    7 &  8.31e-09 &  2.37e-16 &      \\ 
     &           &    8 &  8.29e-09 &  2.37e-16 &      \\ 
     &           &    9 &  8.28e-09 &  2.37e-16 &      \\ 
     &           &   10 &  8.27e-09 &  2.36e-16 &      \\ 
1147 &  1.15e+04 &   10 &           &           & iters  \\ 
 \hdashline 
     &           &    1 &  4.19e-08 &  6.84e-09 &      \\ 
     &           &    2 &  3.58e-08 &  1.20e-15 &      \\ 
     &           &    3 &  3.58e-08 &  1.02e-15 &      \\ 
     &           &    4 &  4.20e-08 &  1.02e-15 &      \\ 
     &           &    5 &  3.58e-08 &  1.20e-15 &      \\ 
     &           &    6 &  4.20e-08 &  1.02e-15 &      \\ 
     &           &    7 &  3.58e-08 &  1.20e-15 &      \\ 
     &           &    8 &  4.20e-08 &  1.02e-15 &      \\ 
     &           &    9 &  3.58e-08 &  1.20e-15 &      \\ 
     &           &   10 &  4.19e-08 &  1.02e-15 &      \\ 
1148 &  1.15e+04 &   10 &           &           & iters  \\ 
 \hdashline 
     &           &    1 &  2.61e-09 &  1.11e-09 &      \\ 
     &           &    2 &  2.60e-09 &  7.44e-17 &      \\ 
     &           &    3 &  2.60e-09 &  7.42e-17 &      \\ 
     &           &    4 &  2.59e-09 &  7.41e-17 &      \\ 
     &           &    5 &  2.59e-09 &  7.40e-17 &      \\ 
     &           &    6 &  2.59e-09 &  7.39e-17 &      \\ 
     &           &    7 &  2.58e-09 &  7.38e-17 &      \\ 
     &           &    8 &  2.58e-09 &  7.37e-17 &      \\ 
     &           &    9 &  2.58e-09 &  7.36e-17 &      \\ 
     &           &   10 &  2.57e-09 &  7.35e-17 &      \\ 
1149 &  1.15e+04 &   10 &           &           & iters  \\ 
 \hdashline 
     &           &    1 &  7.01e-08 &  1.63e-09 &      \\ 
     &           &    2 &  7.68e-09 &  2.00e-15 &      \\ 
     &           &    3 &  7.67e-09 &  2.19e-16 &      \\ 
     &           &    4 &  7.66e-09 &  2.19e-16 &      \\ 
     &           &    5 &  7.65e-09 &  2.19e-16 &      \\ 
     &           &    6 &  7.63e-09 &  2.18e-16 &      \\ 
     &           &    7 &  7.62e-09 &  2.18e-16 &      \\ 
     &           &    8 &  7.61e-09 &  2.18e-16 &      \\ 
     &           &    9 &  7.01e-08 &  2.17e-16 &      \\ 
     &           &   10 &  8.54e-08 &  2.00e-15 &      \\ 
1150 &  1.15e+04 &   10 &           &           & iters  \\ 
 \hdashline 
     &           &    1 &  2.72e-08 &  3.20e-09 &      \\ 
     &           &    2 &  5.05e-08 &  7.77e-16 &      \\ 
     &           &    3 &  2.73e-08 &  1.44e-15 &      \\ 
     &           &    4 &  2.72e-08 &  7.78e-16 &      \\ 
     &           &    5 &  5.05e-08 &  7.77e-16 &      \\ 
     &           &    6 &  2.73e-08 &  1.44e-15 &      \\ 
     &           &    7 &  2.72e-08 &  7.78e-16 &      \\ 
     &           &    8 &  5.05e-08 &  7.77e-16 &      \\ 
     &           &    9 &  2.73e-08 &  1.44e-15 &      \\ 
     &           &   10 &  5.05e-08 &  7.78e-16 &      \\ 
1151 &  1.15e+04 &   10 &           &           & iters  \\ 
 \hdashline 
     &           &    1 &  2.40e-08 &  2.05e-09 &      \\ 
     &           &    2 &  2.40e-08 &  6.86e-16 &      \\ 
     &           &    3 &  2.40e-08 &  6.85e-16 &      \\ 
     &           &    4 &  5.38e-08 &  6.84e-16 &      \\ 
     &           &    5 &  2.40e-08 &  1.53e-15 &      \\ 
     &           &    6 &  2.40e-08 &  6.85e-16 &      \\ 
     &           &    7 &  5.37e-08 &  6.84e-16 &      \\ 
     &           &    8 &  2.40e-08 &  1.53e-15 &      \\ 
     &           &    9 &  2.40e-08 &  6.86e-16 &      \\ 
     &           &   10 &  5.37e-08 &  6.85e-16 &      \\ 
1152 &  1.15e+04 &   10 &           &           & iters  \\ 
 \hdashline 
     &           &    1 &  2.70e-08 &  1.45e-08 &      \\ 
     &           &    2 &  2.69e-08 &  7.70e-16 &      \\ 
     &           &    3 &  5.08e-08 &  7.69e-16 &      \\ 
     &           &    4 &  2.70e-08 &  1.45e-15 &      \\ 
     &           &    5 &  5.08e-08 &  7.70e-16 &      \\ 
     &           &    6 &  2.70e-08 &  1.45e-15 &      \\ 
     &           &    7 &  2.70e-08 &  7.71e-16 &      \\ 
     &           &    8 &  5.08e-08 &  7.70e-16 &      \\ 
     &           &    9 &  2.70e-08 &  1.45e-15 &      \\ 
     &           &   10 &  2.70e-08 &  7.71e-16 &      \\ 
1153 &  1.15e+04 &   10 &           &           & iters  \\ 
 \hdashline 
     &           &    1 &  4.95e-08 &  1.77e-08 &      \\ 
     &           &    2 &  1.06e-08 &  1.41e-15 &      \\ 
     &           &    3 &  1.06e-08 &  3.03e-16 &      \\ 
     &           &    4 &  1.06e-08 &  3.03e-16 &      \\ 
     &           &    5 &  6.71e-08 &  3.02e-16 &      \\ 
     &           &    6 &  4.95e-08 &  1.92e-15 &      \\ 
     &           &    7 &  1.06e-08 &  1.41e-15 &      \\ 
     &           &    8 &  1.06e-08 &  3.03e-16 &      \\ 
     &           &    9 &  6.71e-08 &  3.02e-16 &      \\ 
     &           &   10 &  1.07e-08 &  1.92e-15 &      \\ 
1154 &  1.15e+04 &   10 &           &           & iters  \\ 
 \hdashline 
     &           &    1 &  1.41e-08 &  8.65e-09 &      \\ 
     &           &    2 &  1.41e-08 &  4.03e-16 &      \\ 
     &           &    3 &  1.41e-08 &  4.02e-16 &      \\ 
     &           &    4 &  6.36e-08 &  4.02e-16 &      \\ 
     &           &    5 &  1.42e-08 &  1.82e-15 &      \\ 
     &           &    6 &  1.41e-08 &  4.04e-16 &      \\ 
     &           &    7 &  1.41e-08 &  4.03e-16 &      \\ 
     &           &    8 &  1.41e-08 &  4.03e-16 &      \\ 
     &           &    9 &  1.41e-08 &  4.02e-16 &      \\ 
     &           &   10 &  6.36e-08 &  4.02e-16 &      \\ 
1155 &  1.15e+04 &   10 &           &           & iters  \\ 
 \hdashline 
     &           &    1 &  5.91e-08 &  5.79e-09 &      \\ 
     &           &    2 &  1.87e-08 &  1.69e-15 &      \\ 
     &           &    3 &  1.87e-08 &  5.34e-16 &      \\ 
     &           &    4 &  1.86e-08 &  5.33e-16 &      \\ 
     &           &    5 &  5.91e-08 &  5.32e-16 &      \\ 
     &           &    6 &  1.87e-08 &  1.69e-15 &      \\ 
     &           &    7 &  1.87e-08 &  5.34e-16 &      \\ 
     &           &    8 &  1.86e-08 &  5.33e-16 &      \\ 
     &           &    9 &  5.91e-08 &  5.32e-16 &      \\ 
     &           &   10 &  1.87e-08 &  1.69e-15 &      \\ 
1156 &  1.16e+04 &   10 &           &           & iters  \\ 
 \hdashline 
     &           &    1 &  3.11e-08 &  5.54e-09 &      \\ 
     &           &    2 &  3.11e-08 &  8.89e-16 &      \\ 
     &           &    3 &  4.66e-08 &  8.87e-16 &      \\ 
     &           &    4 &  3.11e-08 &  1.33e-15 &      \\ 
     &           &    5 &  4.66e-08 &  8.88e-16 &      \\ 
     &           &    6 &  3.11e-08 &  1.33e-15 &      \\ 
     &           &    7 &  3.11e-08 &  8.89e-16 &      \\ 
     &           &    8 &  4.66e-08 &  8.87e-16 &      \\ 
     &           &    9 &  3.11e-08 &  1.33e-15 &      \\ 
     &           &   10 &  4.66e-08 &  8.88e-16 &      \\ 
1157 &  1.16e+04 &   10 &           &           & iters  \\ 
 \hdashline 
     &           &    1 &  2.93e-09 &  3.68e-09 &      \\ 
     &           &    2 &  2.92e-09 &  8.35e-17 &      \\ 
     &           &    3 &  2.92e-09 &  8.34e-17 &      \\ 
     &           &    4 &  2.91e-09 &  8.33e-17 &      \\ 
     &           &    5 &  2.91e-09 &  8.32e-17 &      \\ 
     &           &    6 &  2.91e-09 &  8.30e-17 &      \\ 
     &           &    7 &  2.90e-09 &  8.29e-17 &      \\ 
     &           &    8 &  7.48e-08 &  8.28e-17 &      \\ 
     &           &    9 &  4.18e-08 &  2.13e-15 &      \\ 
     &           &   10 &  2.94e-09 &  1.19e-15 &      \\ 
1158 &  1.16e+04 &   10 &           &           & iters  \\ 
 \hdashline 
     &           &    1 &  5.33e-08 &  7.01e-09 &      \\ 
     &           &    2 &  2.45e-08 &  1.52e-15 &      \\ 
     &           &    3 &  2.45e-08 &  6.99e-16 &      \\ 
     &           &    4 &  2.44e-08 &  6.98e-16 &      \\ 
     &           &    5 &  5.33e-08 &  6.97e-16 &      \\ 
     &           &    6 &  2.45e-08 &  1.52e-15 &      \\ 
     &           &    7 &  2.44e-08 &  6.98e-16 &      \\ 
     &           &    8 &  5.33e-08 &  6.97e-16 &      \\ 
     &           &    9 &  2.45e-08 &  1.52e-15 &      \\ 
     &           &   10 &  2.44e-08 &  6.98e-16 &      \\ 
1159 &  1.16e+04 &   10 &           &           & iters  \\ 
 \hdashline 
     &           &    1 &  7.75e-09 &  3.15e-09 &      \\ 
     &           &    2 &  7.74e-09 &  2.21e-16 &      \\ 
     &           &    3 &  7.73e-09 &  2.21e-16 &      \\ 
     &           &    4 &  7.72e-09 &  2.21e-16 &      \\ 
     &           &    5 &  7.71e-09 &  2.20e-16 &      \\ 
     &           &    6 &  3.11e-08 &  2.20e-16 &      \\ 
     &           &    7 &  7.74e-09 &  8.89e-16 &      \\ 
     &           &    8 &  7.73e-09 &  2.21e-16 &      \\ 
     &           &    9 &  7.72e-09 &  2.21e-16 &      \\ 
     &           &   10 &  7.71e-09 &  2.20e-16 &      \\ 
1160 &  1.16e+04 &   10 &           &           & iters  \\ 
 \hdashline 
     &           &    1 &  1.91e-08 &  1.62e-09 &      \\ 
     &           &    2 &  5.87e-08 &  5.44e-16 &      \\ 
     &           &    3 &  1.91e-08 &  1.67e-15 &      \\ 
     &           &    4 &  1.91e-08 &  5.45e-16 &      \\ 
     &           &    5 &  1.91e-08 &  5.45e-16 &      \\ 
     &           &    6 &  5.87e-08 &  5.44e-16 &      \\ 
     &           &    7 &  1.91e-08 &  1.67e-15 &      \\ 
     &           &    8 &  1.91e-08 &  5.45e-16 &      \\ 
     &           &    9 &  1.91e-08 &  5.45e-16 &      \\ 
     &           &   10 &  5.87e-08 &  5.44e-16 &      \\ 
1161 &  1.16e+04 &   10 &           &           & iters  \\ 
 \hdashline 
     &           &    1 &  5.48e-08 &  1.92e-09 &      \\ 
     &           &    2 &  2.29e-08 &  1.57e-15 &      \\ 
     &           &    3 &  2.29e-08 &  6.54e-16 &      \\ 
     &           &    4 &  2.29e-08 &  6.53e-16 &      \\ 
     &           &    5 &  5.49e-08 &  6.52e-16 &      \\ 
     &           &    6 &  2.29e-08 &  1.57e-15 &      \\ 
     &           &    7 &  2.29e-08 &  6.54e-16 &      \\ 
     &           &    8 &  5.49e-08 &  6.53e-16 &      \\ 
     &           &    9 &  2.29e-08 &  1.57e-15 &      \\ 
     &           &   10 &  2.29e-08 &  6.54e-16 &      \\ 
1162 &  1.16e+04 &   10 &           &           & iters  \\ 
 \hdashline 
     &           &    1 &  4.58e-08 &  5.18e-09 &      \\ 
     &           &    2 &  3.19e-08 &  1.31e-15 &      \\ 
     &           &    3 &  3.19e-08 &  9.12e-16 &      \\ 
     &           &    4 &  4.58e-08 &  9.10e-16 &      \\ 
     &           &    5 &  3.19e-08 &  1.31e-15 &      \\ 
     &           &    6 &  4.58e-08 &  9.11e-16 &      \\ 
     &           &    7 &  3.19e-08 &  1.31e-15 &      \\ 
     &           &    8 &  3.19e-08 &  9.12e-16 &      \\ 
     &           &    9 &  4.58e-08 &  9.10e-16 &      \\ 
     &           &   10 &  3.19e-08 &  1.31e-15 &      \\ 
1163 &  1.16e+04 &   10 &           &           & iters  \\ 
 \hdashline 
     &           &    1 &  1.22e-08 &  7.72e-09 &      \\ 
     &           &    2 &  1.22e-08 &  3.49e-16 &      \\ 
     &           &    3 &  1.22e-08 &  3.49e-16 &      \\ 
     &           &    4 &  1.22e-08 &  3.48e-16 &      \\ 
     &           &    5 &  1.22e-08 &  3.48e-16 &      \\ 
     &           &    6 &  1.22e-08 &  3.47e-16 &      \\ 
     &           &    7 &  1.21e-08 &  3.47e-16 &      \\ 
     &           &    8 &  6.56e-08 &  3.46e-16 &      \\ 
     &           &    9 &  1.22e-08 &  1.87e-15 &      \\ 
     &           &   10 &  1.22e-08 &  3.49e-16 &      \\ 
1164 &  1.16e+04 &   10 &           &           & iters  \\ 
 \hdashline 
     &           &    1 &  2.77e-09 &  4.90e-09 &      \\ 
     &           &    2 &  2.77e-09 &  7.91e-17 &      \\ 
     &           &    3 &  2.77e-09 &  7.90e-17 &      \\ 
     &           &    4 &  2.76e-09 &  7.89e-17 &      \\ 
     &           &    5 &  2.76e-09 &  7.88e-17 &      \\ 
     &           &    6 &  2.75e-09 &  7.87e-17 &      \\ 
     &           &    7 &  2.75e-09 &  7.86e-17 &      \\ 
     &           &    8 &  2.75e-09 &  7.85e-17 &      \\ 
     &           &    9 &  2.74e-09 &  7.84e-17 &      \\ 
     &           &   10 &  2.74e-09 &  7.82e-17 &      \\ 
1165 &  1.16e+04 &   10 &           &           & iters  \\ 
 \hdashline 
     &           &    1 &  9.03e-09 &  1.82e-10 &      \\ 
     &           &    2 &  9.02e-09 &  2.58e-16 &      \\ 
     &           &    3 &  6.87e-08 &  2.57e-16 &      \\ 
     &           &    4 &  8.68e-08 &  1.96e-15 &      \\ 
     &           &    5 &  6.87e-08 &  2.48e-15 &      \\ 
     &           &    6 &  9.08e-09 &  1.96e-15 &      \\ 
     &           &    7 &  9.06e-09 &  2.59e-16 &      \\ 
     &           &    8 &  9.05e-09 &  2.59e-16 &      \\ 
     &           &    9 &  9.04e-09 &  2.58e-16 &      \\ 
     &           &   10 &  9.03e-09 &  2.58e-16 &      \\ 
1166 &  1.16e+04 &   10 &           &           & iters  \\ 
 \hdashline 
     &           &    1 &  1.12e-08 &  4.77e-09 &      \\ 
     &           &    2 &  1.12e-08 &  3.20e-16 &      \\ 
     &           &    3 &  6.65e-08 &  3.20e-16 &      \\ 
     &           &    4 &  1.13e-08 &  1.90e-15 &      \\ 
     &           &    5 &  1.13e-08 &  3.22e-16 &      \\ 
     &           &    6 &  1.12e-08 &  3.21e-16 &      \\ 
     &           &    7 &  1.12e-08 &  3.21e-16 &      \\ 
     &           &    8 &  1.12e-08 &  3.21e-16 &      \\ 
     &           &    9 &  1.12e-08 &  3.20e-16 &      \\ 
     &           &   10 &  6.65e-08 &  3.20e-16 &      \\ 
1167 &  1.17e+04 &   10 &           &           & iters  \\ 
 \hdashline 
     &           &    1 &  2.66e-09 &  5.48e-09 &      \\ 
     &           &    2 &  2.66e-09 &  7.61e-17 &      \\ 
     &           &    3 &  2.66e-09 &  7.59e-17 &      \\ 
     &           &    4 &  2.65e-09 &  7.58e-17 &      \\ 
     &           &    5 &  2.65e-09 &  7.57e-17 &      \\ 
     &           &    6 &  2.65e-09 &  7.56e-17 &      \\ 
     &           &    7 &  2.64e-09 &  7.55e-17 &      \\ 
     &           &    8 &  2.64e-09 &  7.54e-17 &      \\ 
     &           &    9 &  2.63e-09 &  7.53e-17 &      \\ 
     &           &   10 &  2.63e-09 &  7.52e-17 &      \\ 
1168 &  1.17e+04 &   10 &           &           & iters  \\ 
 \hdashline 
     &           &    1 &  2.46e-08 &  2.82e-09 &      \\ 
     &           &    2 &  5.32e-08 &  7.01e-16 &      \\ 
     &           &    3 &  2.46e-08 &  1.52e-15 &      \\ 
     &           &    4 &  2.46e-08 &  7.02e-16 &      \\ 
     &           &    5 &  5.32e-08 &  7.01e-16 &      \\ 
     &           &    6 &  2.46e-08 &  1.52e-15 &      \\ 
     &           &    7 &  2.46e-08 &  7.02e-16 &      \\ 
     &           &    8 &  5.32e-08 &  7.01e-16 &      \\ 
     &           &    9 &  2.46e-08 &  1.52e-15 &      \\ 
     &           &   10 &  2.46e-08 &  7.02e-16 &      \\ 
1169 &  1.17e+04 &   10 &           &           & iters  \\ 
 \hdashline 
     &           &    1 &  6.10e-09 &  4.08e-09 &      \\ 
     &           &    2 &  6.09e-09 &  1.74e-16 &      \\ 
     &           &    3 &  6.08e-09 &  1.74e-16 &      \\ 
     &           &    4 &  6.07e-09 &  1.74e-16 &      \\ 
     &           &    5 &  6.07e-09 &  1.73e-16 &      \\ 
     &           &    6 &  6.06e-09 &  1.73e-16 &      \\ 
     &           &    7 &  6.05e-09 &  1.73e-16 &      \\ 
     &           &    8 &  7.16e-08 &  1.73e-16 &      \\ 
     &           &    9 &  4.50e-08 &  2.04e-15 &      \\ 
     &           &   10 &  6.08e-09 &  1.28e-15 &      \\ 
1170 &  1.17e+04 &   10 &           &           & iters  \\ 
 \hdashline 
     &           &    1 &  5.87e-09 &  3.03e-09 &      \\ 
     &           &    2 &  5.86e-09 &  1.67e-16 &      \\ 
     &           &    3 &  5.85e-09 &  1.67e-16 &      \\ 
     &           &    4 &  5.84e-09 &  1.67e-16 &      \\ 
     &           &    5 &  5.83e-09 &  1.67e-16 &      \\ 
     &           &    6 &  5.83e-09 &  1.67e-16 &      \\ 
     &           &    7 &  5.82e-09 &  1.66e-16 &      \\ 
     &           &    8 &  7.19e-08 &  1.66e-16 &      \\ 
     &           &    9 &  8.36e-08 &  2.05e-15 &      \\ 
     &           &   10 &  7.19e-08 &  2.39e-15 &      \\ 
1171 &  1.17e+04 &   10 &           &           & iters  \\ 
 \hdashline 
     &           &    1 &  4.32e-08 &  1.01e-09 &      \\ 
     &           &    2 &  4.33e-09 &  1.23e-15 &      \\ 
     &           &    3 &  4.33e-09 &  1.24e-16 &      \\ 
     &           &    4 &  4.32e-09 &  1.24e-16 &      \\ 
     &           &    5 &  4.32e-09 &  1.23e-16 &      \\ 
     &           &    6 &  4.31e-09 &  1.23e-16 &      \\ 
     &           &    7 &  4.30e-09 &  1.23e-16 &      \\ 
     &           &    8 &  4.30e-09 &  1.23e-16 &      \\ 
     &           &    9 &  4.29e-09 &  1.23e-16 &      \\ 
     &           &   10 &  7.34e-08 &  1.22e-16 &      \\ 
1172 &  1.17e+04 &   10 &           &           & iters  \\ 
 \hdashline 
     &           &    1 &  6.55e-10 &  2.65e-10 &      \\ 
     &           &    2 &  7.70e-08 &  1.87e-17 &      \\ 
     &           &    3 &  7.85e-08 &  2.20e-15 &      \\ 
     &           &    4 &  7.70e-08 &  2.24e-15 &      \\ 
     &           &    5 &  7.85e-08 &  2.20e-15 &      \\ 
     &           &    6 &  7.70e-08 &  2.24e-15 &      \\ 
     &           &    7 &  7.84e-08 &  2.20e-15 &      \\ 
     &           &    8 &  7.70e-08 &  2.24e-15 &      \\ 
     &           &    9 &  7.84e-08 &  2.20e-15 &      \\ 
     &           &   10 &  7.70e-08 &  2.24e-15 &      \\ 
1173 &  1.17e+04 &   10 &           &           & iters  \\ 
 \hdashline 
     &           &    1 &  3.61e-08 &  1.31e-09 &      \\ 
     &           &    2 &  4.16e-08 &  1.03e-15 &      \\ 
     &           &    3 &  3.61e-08 &  1.19e-15 &      \\ 
     &           &    4 &  3.61e-08 &  1.03e-15 &      \\ 
     &           &    5 &  4.17e-08 &  1.03e-15 &      \\ 
     &           &    6 &  3.61e-08 &  1.19e-15 &      \\ 
     &           &    7 &  4.17e-08 &  1.03e-15 &      \\ 
     &           &    8 &  3.61e-08 &  1.19e-15 &      \\ 
     &           &    9 &  4.16e-08 &  1.03e-15 &      \\ 
     &           &   10 &  3.61e-08 &  1.19e-15 &      \\ 
1174 &  1.17e+04 &   10 &           &           & iters  \\ 
 \hdashline 
     &           &    1 &  3.01e-08 &  5.60e-09 &      \\ 
     &           &    2 &  4.77e-08 &  8.58e-16 &      \\ 
     &           &    3 &  3.01e-08 &  1.36e-15 &      \\ 
     &           &    4 &  4.76e-08 &  8.59e-16 &      \\ 
     &           &    5 &  3.01e-08 &  1.36e-15 &      \\ 
     &           &    6 &  3.01e-08 &  8.59e-16 &      \\ 
     &           &    7 &  4.77e-08 &  8.58e-16 &      \\ 
     &           &    8 &  3.01e-08 &  1.36e-15 &      \\ 
     &           &    9 &  3.00e-08 &  8.59e-16 &      \\ 
     &           &   10 &  4.77e-08 &  8.58e-16 &      \\ 
1175 &  1.17e+04 &   10 &           &           & iters  \\ 
 \hdashline 
     &           &    1 &  1.27e-08 &  4.20e-09 &      \\ 
     &           &    2 &  1.27e-08 &  3.63e-16 &      \\ 
     &           &    3 &  1.27e-08 &  3.63e-16 &      \\ 
     &           &    4 &  1.27e-08 &  3.62e-16 &      \\ 
     &           &    5 &  1.27e-08 &  3.62e-16 &      \\ 
     &           &    6 &  1.26e-08 &  3.61e-16 &      \\ 
     &           &    7 &  6.51e-08 &  3.61e-16 &      \\ 
     &           &    8 &  1.27e-08 &  1.86e-15 &      \\ 
     &           &    9 &  1.27e-08 &  3.63e-16 &      \\ 
     &           &   10 &  1.27e-08 &  3.62e-16 &      \\ 
1176 &  1.18e+04 &   10 &           &           & iters  \\ 
 \hdashline 
     &           &    1 &  3.51e-08 &  1.03e-09 &      \\ 
     &           &    2 &  3.51e-08 &  1.00e-15 &      \\ 
     &           &    3 &  4.26e-08 &  1.00e-15 &      \\ 
     &           &    4 &  3.51e-08 &  1.22e-15 &      \\ 
     &           &    5 &  4.26e-08 &  1.00e-15 &      \\ 
     &           &    6 &  3.51e-08 &  1.22e-15 &      \\ 
     &           &    7 &  4.26e-08 &  1.00e-15 &      \\ 
     &           &    8 &  3.51e-08 &  1.22e-15 &      \\ 
     &           &    9 &  4.26e-08 &  1.00e-15 &      \\ 
     &           &   10 &  3.51e-08 &  1.22e-15 &      \\ 
1177 &  1.18e+04 &   10 &           &           & iters  \\ 
 \hdashline 
     &           &    1 &  1.46e-08 &  2.50e-09 &      \\ 
     &           &    2 &  6.31e-08 &  4.16e-16 &      \\ 
     &           &    3 &  1.47e-08 &  1.80e-15 &      \\ 
     &           &    4 &  1.46e-08 &  4.18e-16 &      \\ 
     &           &    5 &  1.46e-08 &  4.18e-16 &      \\ 
     &           &    6 &  1.46e-08 &  4.17e-16 &      \\ 
     &           &    7 &  6.31e-08 &  4.17e-16 &      \\ 
     &           &    8 &  1.47e-08 &  1.80e-15 &      \\ 
     &           &    9 &  1.46e-08 &  4.19e-16 &      \\ 
     &           &   10 &  1.46e-08 &  4.18e-16 &      \\ 
1178 &  1.18e+04 &   10 &           &           & iters  \\ 
 \hdashline 
     &           &    1 &  2.75e-08 &  1.51e-09 &      \\ 
     &           &    2 &  2.75e-08 &  7.85e-16 &      \\ 
     &           &    3 &  5.02e-08 &  7.84e-16 &      \\ 
     &           &    4 &  2.75e-08 &  1.43e-15 &      \\ 
     &           &    5 &  2.75e-08 &  7.85e-16 &      \\ 
     &           &    6 &  5.03e-08 &  7.84e-16 &      \\ 
     &           &    7 &  2.75e-08 &  1.43e-15 &      \\ 
     &           &    8 &  5.02e-08 &  7.85e-16 &      \\ 
     &           &    9 &  2.75e-08 &  1.43e-15 &      \\ 
     &           &   10 &  2.75e-08 &  7.86e-16 &      \\ 
1179 &  1.18e+04 &   10 &           &           & iters  \\ 
 \hdashline 
     &           &    1 &  1.49e-08 &  6.46e-09 &      \\ 
     &           &    2 &  1.49e-08 &  4.25e-16 &      \\ 
     &           &    3 &  1.48e-08 &  4.24e-16 &      \\ 
     &           &    4 &  6.29e-08 &  4.23e-16 &      \\ 
     &           &    5 &  1.49e-08 &  1.79e-15 &      \\ 
     &           &    6 &  1.49e-08 &  4.25e-16 &      \\ 
     &           &    7 &  1.49e-08 &  4.25e-16 &      \\ 
     &           &    8 &  1.48e-08 &  4.24e-16 &      \\ 
     &           &    9 &  6.29e-08 &  4.24e-16 &      \\ 
     &           &   10 &  9.26e-08 &  1.79e-15 &      \\ 
1180 &  1.18e+04 &   10 &           &           & iters  \\ 
 \hdashline 
     &           &    1 &  7.54e-09 &  8.81e-09 &      \\ 
     &           &    2 &  7.53e-09 &  2.15e-16 &      \\ 
     &           &    3 &  7.51e-09 &  2.15e-16 &      \\ 
     &           &    4 &  7.50e-09 &  2.14e-16 &      \\ 
     &           &    5 &  7.49e-09 &  2.14e-16 &      \\ 
     &           &    6 &  7.48e-09 &  2.14e-16 &      \\ 
     &           &    7 &  7.47e-09 &  2.14e-16 &      \\ 
     &           &    8 &  7.46e-09 &  2.13e-16 &      \\ 
     &           &    9 &  7.02e-08 &  2.13e-16 &      \\ 
     &           &   10 &  8.52e-08 &  2.00e-15 &      \\ 
1181 &  1.18e+04 &   10 &           &           & iters  \\ 
 \hdashline 
     &           &    1 &  4.11e-08 &  8.82e-09 &      \\ 
     &           &    2 &  3.66e-08 &  1.17e-15 &      \\ 
     &           &    3 &  4.11e-08 &  1.05e-15 &      \\ 
     &           &    4 &  3.66e-08 &  1.17e-15 &      \\ 
     &           &    5 &  4.11e-08 &  1.05e-15 &      \\ 
     &           &    6 &  3.67e-08 &  1.17e-15 &      \\ 
     &           &    7 &  4.11e-08 &  1.05e-15 &      \\ 
     &           &    8 &  3.67e-08 &  1.17e-15 &      \\ 
     &           &    9 &  4.11e-08 &  1.05e-15 &      \\ 
     &           &   10 &  3.67e-08 &  1.17e-15 &      \\ 
1182 &  1.18e+04 &   10 &           &           & iters  \\ 
 \hdashline 
     &           &    1 &  8.34e-09 &  3.15e-10 &      \\ 
     &           &    2 &  8.33e-09 &  2.38e-16 &      \\ 
     &           &    3 &  8.32e-09 &  2.38e-16 &      \\ 
     &           &    4 &  6.94e-08 &  2.37e-16 &      \\ 
     &           &    5 &  8.40e-09 &  1.98e-15 &      \\ 
     &           &    6 &  8.39e-09 &  2.40e-16 &      \\ 
     &           &    7 &  8.38e-09 &  2.40e-16 &      \\ 
     &           &    8 &  8.37e-09 &  2.39e-16 &      \\ 
     &           &    9 &  8.36e-09 &  2.39e-16 &      \\ 
     &           &   10 &  8.34e-09 &  2.38e-16 &      \\ 
1183 &  1.18e+04 &   10 &           &           & iters  \\ 
 \hdashline 
     &           &    1 &  2.16e-08 &  1.31e-08 &      \\ 
     &           &    2 &  2.15e-08 &  6.16e-16 &      \\ 
     &           &    3 &  2.15e-08 &  6.15e-16 &      \\ 
     &           &    4 &  2.15e-08 &  6.14e-16 &      \\ 
     &           &    5 &  5.62e-08 &  6.13e-16 &      \\ 
     &           &    6 &  2.15e-08 &  1.60e-15 &      \\ 
     &           &    7 &  2.15e-08 &  6.15e-16 &      \\ 
     &           &    8 &  5.62e-08 &  6.14e-16 &      \\ 
     &           &    9 &  2.16e-08 &  1.60e-15 &      \\ 
     &           &   10 &  2.15e-08 &  6.15e-16 &      \\ 
1184 &  1.18e+04 &   10 &           &           & iters  \\ 
 \hdashline 
     &           &    1 &  3.23e-08 &  9.20e-09 &      \\ 
     &           &    2 &  4.55e-08 &  9.21e-16 &      \\ 
     &           &    3 &  3.23e-08 &  1.30e-15 &      \\ 
     &           &    4 &  3.22e-08 &  9.21e-16 &      \\ 
     &           &    5 &  4.55e-08 &  9.20e-16 &      \\ 
     &           &    6 &  3.23e-08 &  1.30e-15 &      \\ 
     &           &    7 &  4.55e-08 &  9.21e-16 &      \\ 
     &           &    8 &  3.23e-08 &  1.30e-15 &      \\ 
     &           &    9 &  3.22e-08 &  9.21e-16 &      \\ 
     &           &   10 &  4.55e-08 &  9.20e-16 &      \\ 
1185 &  1.18e+04 &   10 &           &           & iters  \\ 
 \hdashline 
     &           &    1 &  2.68e-08 &  1.38e-09 &      \\ 
     &           &    2 &  2.68e-08 &  7.66e-16 &      \\ 
     &           &    3 &  5.09e-08 &  7.65e-16 &      \\ 
     &           &    4 &  2.68e-08 &  1.45e-15 &      \\ 
     &           &    5 &  2.68e-08 &  7.66e-16 &      \\ 
     &           &    6 &  5.09e-08 &  7.65e-16 &      \\ 
     &           &    7 &  2.68e-08 &  1.45e-15 &      \\ 
     &           &    8 &  2.68e-08 &  7.65e-16 &      \\ 
     &           &    9 &  5.09e-08 &  7.64e-16 &      \\ 
     &           &   10 &  2.68e-08 &  1.45e-15 &      \\ 
1186 &  1.18e+04 &   10 &           &           & iters  \\ 
 \hdashline 
     &           &    1 &  1.18e-08 &  4.70e-09 &      \\ 
     &           &    2 &  1.18e-08 &  3.36e-16 &      \\ 
     &           &    3 &  1.17e-08 &  3.36e-16 &      \\ 
     &           &    4 &  1.17e-08 &  3.35e-16 &      \\ 
     &           &    5 &  1.17e-08 &  3.35e-16 &      \\ 
     &           &    6 &  6.60e-08 &  3.34e-16 &      \\ 
     &           &    7 &  1.18e-08 &  1.88e-15 &      \\ 
     &           &    8 &  1.18e-08 &  3.36e-16 &      \\ 
     &           &    9 &  1.18e-08 &  3.36e-16 &      \\ 
     &           &   10 &  1.17e-08 &  3.35e-16 &      \\ 
1187 &  1.19e+04 &   10 &           &           & iters  \\ 
 \hdashline 
     &           &    1 &  8.20e-09 &  6.93e-09 &      \\ 
     &           &    2 &  8.18e-09 &  2.34e-16 &      \\ 
     &           &    3 &  6.95e-08 &  2.34e-16 &      \\ 
     &           &    4 &  4.71e-08 &  1.98e-15 &      \\ 
     &           &    5 &  8.20e-09 &  1.34e-15 &      \\ 
     &           &    6 &  8.19e-09 &  2.34e-16 &      \\ 
     &           &    7 &  8.18e-09 &  2.34e-16 &      \\ 
     &           &    8 &  6.95e-08 &  2.33e-16 &      \\ 
     &           &    9 &  4.71e-08 &  1.98e-15 &      \\ 
     &           &   10 &  8.20e-09 &  1.34e-15 &      \\ 
1188 &  1.19e+04 &   10 &           &           & iters  \\ 
 \hdashline 
     &           &    1 &  2.50e-08 &  1.97e-09 &      \\ 
     &           &    2 &  2.50e-08 &  7.14e-16 &      \\ 
     &           &    3 &  1.39e-08 &  7.13e-16 &      \\ 
     &           &    4 &  1.39e-08 &  3.97e-16 &      \\ 
     &           &    5 &  2.50e-08 &  3.96e-16 &      \\ 
     &           &    6 &  1.39e-08 &  7.13e-16 &      \\ 
     &           &    7 &  1.39e-08 &  3.97e-16 &      \\ 
     &           &    8 &  2.50e-08 &  3.96e-16 &      \\ 
     &           &    9 &  1.39e-08 &  7.13e-16 &      \\ 
     &           &   10 &  1.39e-08 &  3.97e-16 &      \\ 
1189 &  1.19e+04 &   10 &           &           & iters  \\ 
 \hdashline 
     &           &    1 &  5.24e-09 &  3.51e-09 &      \\ 
     &           &    2 &  5.24e-09 &  1.50e-16 &      \\ 
     &           &    3 &  5.23e-09 &  1.49e-16 &      \\ 
     &           &    4 &  5.22e-09 &  1.49e-16 &      \\ 
     &           &    5 &  7.25e-08 &  1.49e-16 &      \\ 
     &           &    6 &  4.42e-08 &  2.07e-15 &      \\ 
     &           &    7 &  5.25e-09 &  1.26e-15 &      \\ 
     &           &    8 &  5.25e-09 &  1.50e-16 &      \\ 
     &           &    9 &  5.24e-09 &  1.50e-16 &      \\ 
     &           &   10 &  5.23e-09 &  1.50e-16 &      \\ 
1190 &  1.19e+04 &   10 &           &           & iters  \\ 
 \hdashline 
     &           &    1 &  3.65e-08 &  1.18e-09 &      \\ 
     &           &    2 &  2.38e-09 &  1.04e-15 &      \\ 
     &           &    3 &  2.38e-09 &  6.80e-17 &      \\ 
     &           &    4 &  2.38e-09 &  6.79e-17 &      \\ 
     &           &    5 &  2.37e-09 &  6.78e-17 &      \\ 
     &           &    6 &  2.37e-09 &  6.77e-17 &      \\ 
     &           &    7 &  2.37e-09 &  6.76e-17 &      \\ 
     &           &    8 &  2.36e-09 &  6.75e-17 &      \\ 
     &           &    9 &  2.36e-09 &  6.74e-17 &      \\ 
     &           &   10 &  2.36e-09 &  6.73e-17 &      \\ 
1191 &  1.19e+04 &   10 &           &           & iters  \\ 
 \hdashline 
     &           &    1 &  2.90e-08 &  1.34e-09 &      \\ 
     &           &    2 &  2.89e-08 &  8.27e-16 &      \\ 
     &           &    3 &  4.88e-08 &  8.26e-16 &      \\ 
     &           &    4 &  2.90e-08 &  1.39e-15 &      \\ 
     &           &    5 &  4.88e-08 &  8.26e-16 &      \\ 
     &           &    6 &  2.90e-08 &  1.39e-15 &      \\ 
     &           &    7 &  2.89e-08 &  8.27e-16 &      \\ 
     &           &    8 &  4.88e-08 &  8.26e-16 &      \\ 
     &           &    9 &  2.90e-08 &  1.39e-15 &      \\ 
     &           &   10 &  2.89e-08 &  8.27e-16 &      \\ 
1192 &  1.19e+04 &   10 &           &           & iters  \\ 
 \hdashline 
     &           &    1 &  1.97e-08 &  2.45e-09 &      \\ 
     &           &    2 &  1.97e-08 &  5.62e-16 &      \\ 
     &           &    3 &  1.96e-08 &  5.61e-16 &      \\ 
     &           &    4 &  1.92e-08 &  5.60e-16 &      \\ 
     &           &    5 &  1.96e-08 &  5.49e-16 &      \\ 
     &           &    6 &  1.93e-08 &  5.60e-16 &      \\ 
     &           &    7 &  1.96e-08 &  5.49e-16 &      \\ 
     &           &    8 &  1.93e-08 &  5.60e-16 &      \\ 
     &           &    9 &  1.96e-08 &  5.49e-16 &      \\ 
     &           &   10 &  1.93e-08 &  5.60e-16 &      \\ 
1193 &  1.19e+04 &   10 &           &           & iters  \\ 
 \hdashline 
     &           &    1 &  2.18e-08 &  3.84e-09 &      \\ 
     &           &    2 &  5.60e-08 &  6.21e-16 &      \\ 
     &           &    3 &  2.18e-08 &  1.60e-15 &      \\ 
     &           &    4 &  2.18e-08 &  6.23e-16 &      \\ 
     &           &    5 &  2.18e-08 &  6.22e-16 &      \\ 
     &           &    6 &  5.60e-08 &  6.21e-16 &      \\ 
     &           &    7 &  2.18e-08 &  1.60e-15 &      \\ 
     &           &    8 &  2.18e-08 &  6.22e-16 &      \\ 
     &           &    9 &  2.17e-08 &  6.21e-16 &      \\ 
     &           &   10 &  5.60e-08 &  6.20e-16 &      \\ 
1194 &  1.19e+04 &   10 &           &           & iters  \\ 
 \hdashline 
     &           &    1 &  3.47e-08 &  9.83e-11 &      \\ 
     &           &    2 &  4.20e-09 &  9.90e-16 &      \\ 
     &           &    3 &  4.19e-09 &  1.20e-16 &      \\ 
     &           &    4 &  4.18e-09 &  1.20e-16 &      \\ 
     &           &    5 &  4.18e-09 &  1.19e-16 &      \\ 
     &           &    6 &  4.17e-09 &  1.19e-16 &      \\ 
     &           &    7 &  4.17e-09 &  1.19e-16 &      \\ 
     &           &    8 &  4.16e-09 &  1.19e-16 &      \\ 
     &           &    9 &  3.47e-08 &  1.19e-16 &      \\ 
     &           &   10 &  4.20e-09 &  9.90e-16 &      \\ 
1195 &  1.19e+04 &   10 &           &           & iters  \\ 
 \hdashline 
     &           &    1 &  1.34e-08 &  4.01e-09 &      \\ 
     &           &    2 &  1.34e-08 &  3.82e-16 &      \\ 
     &           &    3 &  2.55e-08 &  3.82e-16 &      \\ 
     &           &    4 &  1.34e-08 &  7.27e-16 &      \\ 
     &           &    5 &  1.34e-08 &  3.82e-16 &      \\ 
     &           &    6 &  2.55e-08 &  3.82e-16 &      \\ 
     &           &    7 &  1.34e-08 &  7.28e-16 &      \\ 
     &           &    8 &  1.34e-08 &  3.82e-16 &      \\ 
     &           &    9 &  2.55e-08 &  3.82e-16 &      \\ 
     &           &   10 &  1.34e-08 &  7.28e-16 &      \\ 
1196 &  1.20e+04 &   10 &           &           & iters  \\ 
 \hdashline 
     &           &    1 &  1.54e-08 &  2.62e-09 &      \\ 
     &           &    2 &  1.54e-08 &  4.39e-16 &      \\ 
     &           &    3 &  6.24e-08 &  4.38e-16 &      \\ 
     &           &    4 &  1.54e-08 &  1.78e-15 &      \\ 
     &           &    5 &  1.54e-08 &  4.40e-16 &      \\ 
     &           &    6 &  1.54e-08 &  4.40e-16 &      \\ 
     &           &    7 &  1.54e-08 &  4.39e-16 &      \\ 
     &           &    8 &  6.24e-08 &  4.38e-16 &      \\ 
     &           &    9 &  1.54e-08 &  1.78e-15 &      \\ 
     &           &   10 &  1.54e-08 &  4.40e-16 &      \\ 
1197 &  1.20e+04 &   10 &           &           & iters  \\ 
 \hdashline 
     &           &    1 &  2.23e-08 &  7.45e-10 &      \\ 
     &           &    2 &  2.22e-08 &  6.35e-16 &      \\ 
     &           &    3 &  5.55e-08 &  6.34e-16 &      \\ 
     &           &    4 &  2.23e-08 &  1.58e-15 &      \\ 
     &           &    5 &  2.22e-08 &  6.35e-16 &      \\ 
     &           &    6 &  5.55e-08 &  6.35e-16 &      \\ 
     &           &    7 &  2.23e-08 &  1.58e-15 &      \\ 
     &           &    8 &  2.23e-08 &  6.36e-16 &      \\ 
     &           &    9 &  2.22e-08 &  6.35e-16 &      \\ 
     &           &   10 &  5.55e-08 &  6.34e-16 &      \\ 
1198 &  1.20e+04 &   10 &           &           & iters  \\ 
 \hdashline 
     &           &    1 &  2.71e-08 &  4.74e-10 &      \\ 
     &           &    2 &  1.18e-08 &  7.72e-16 &      \\ 
     &           &    3 &  1.18e-08 &  3.37e-16 &      \\ 
     &           &    4 &  1.18e-08 &  3.37e-16 &      \\ 
     &           &    5 &  2.71e-08 &  3.36e-16 &      \\ 
     &           &    6 &  1.18e-08 &  7.73e-16 &      \\ 
     &           &    7 &  1.18e-08 &  3.37e-16 &      \\ 
     &           &    8 &  2.71e-08 &  3.36e-16 &      \\ 
     &           &    9 &  1.18e-08 &  7.73e-16 &      \\ 
     &           &   10 &  1.18e-08 &  3.37e-16 &      \\ 
1199 &  1.20e+04 &   10 &           &           & iters  \\ 
 \hdashline 
     &           &    1 &  1.04e-08 &  4.74e-12 &      \\ 
     &           &    2 &  1.04e-08 &  2.96e-16 &      \\ 
     &           &    3 &  1.04e-08 &  2.96e-16 &      \\ 
     &           &    4 &  1.03e-08 &  2.96e-16 &      \\ 
     &           &    5 &  1.03e-08 &  2.95e-16 &      \\ 
     &           &    6 &  6.74e-08 &  2.95e-16 &      \\ 
     &           &    7 &  1.04e-08 &  1.92e-15 &      \\ 
     &           &    8 &  1.04e-08 &  2.97e-16 &      \\ 
     &           &    9 &  1.04e-08 &  2.97e-16 &      \\ 
     &           &   10 &  1.04e-08 &  2.96e-16 &      \\ 
1200 &  1.20e+04 &   10 &           &           & iters  \\ 
 \hdashline 
     &           &    1 &  8.92e-10 &  2.12e-09 &      \\ 
     &           &    2 &  8.91e-10 &  2.55e-17 &      \\ 
     &           &    3 &  8.90e-10 &  2.54e-17 &      \\ 
     &           &    4 &  8.88e-10 &  2.54e-17 &      \\ 
     &           &    5 &  8.87e-10 &  2.54e-17 &      \\ 
     &           &    6 &  8.86e-10 &  2.53e-17 &      \\ 
     &           &    7 &  8.85e-10 &  2.53e-17 &      \\ 
     &           &    8 &  8.83e-10 &  2.52e-17 &      \\ 
     &           &    9 &  8.82e-10 &  2.52e-17 &      \\ 
     &           &   10 &  8.81e-10 &  2.52e-17 &      \\ 
1201 &  1.20e+04 &   10 &           &           & iters  \\ 
 \hdashline 
     &           &    1 &  2.34e-09 &  3.30e-09 &      \\ 
     &           &    2 &  2.34e-09 &  6.69e-17 &      \\ 
     &           &    3 &  2.34e-09 &  6.68e-17 &      \\ 
     &           &    4 &  2.33e-09 &  6.67e-17 &      \\ 
     &           &    5 &  2.33e-09 &  6.66e-17 &      \\ 
     &           &    6 &  2.33e-09 &  6.65e-17 &      \\ 
     &           &    7 &  2.32e-09 &  6.64e-17 &      \\ 
     &           &    8 &  2.32e-09 &  6.63e-17 &      \\ 
     &           &    9 &  2.32e-09 &  6.62e-17 &      \\ 
     &           &   10 &  2.31e-09 &  6.61e-17 &      \\ 
1202 &  1.20e+04 &   10 &           &           & iters  \\ 
 \hdashline 
     &           &    1 &  4.75e-08 &  3.29e-09 &      \\ 
     &           &    2 &  3.03e-08 &  1.36e-15 &      \\ 
     &           &    3 &  3.02e-08 &  8.64e-16 &      \\ 
     &           &    4 &  4.75e-08 &  8.63e-16 &      \\ 
     &           &    5 &  3.03e-08 &  1.36e-15 &      \\ 
     &           &    6 &  3.02e-08 &  8.63e-16 &      \\ 
     &           &    7 &  4.75e-08 &  8.62e-16 &      \\ 
     &           &    8 &  3.02e-08 &  1.36e-15 &      \\ 
     &           &    9 &  4.75e-08 &  8.63e-16 &      \\ 
     &           &   10 &  3.03e-08 &  1.36e-15 &      \\ 
1203 &  1.20e+04 &   10 &           &           & iters  \\ 
 \hdashline 
     &           &    1 &  2.29e-08 &  4.03e-09 &      \\ 
     &           &    2 &  2.29e-08 &  6.53e-16 &      \\ 
     &           &    3 &  5.49e-08 &  6.52e-16 &      \\ 
     &           &    4 &  2.29e-08 &  1.57e-15 &      \\ 
     &           &    5 &  2.29e-08 &  6.54e-16 &      \\ 
     &           &    6 &  5.48e-08 &  6.53e-16 &      \\ 
     &           &    7 &  2.29e-08 &  1.57e-15 &      \\ 
     &           &    8 &  2.29e-08 &  6.54e-16 &      \\ 
     &           &    9 &  2.29e-08 &  6.53e-16 &      \\ 
     &           &   10 &  5.49e-08 &  6.52e-16 &      \\ 
1204 &  1.20e+04 &   10 &           &           & iters  \\ 
 \hdashline 
     &           &    1 &  3.58e-08 &  2.17e-09 &      \\ 
     &           &    2 &  4.20e-08 &  1.02e-15 &      \\ 
     &           &    3 &  3.58e-08 &  1.20e-15 &      \\ 
     &           &    4 &  4.20e-08 &  1.02e-15 &      \\ 
     &           &    5 &  3.58e-08 &  1.20e-15 &      \\ 
     &           &    6 &  4.19e-08 &  1.02e-15 &      \\ 
     &           &    7 &  3.58e-08 &  1.20e-15 &      \\ 
     &           &    8 &  4.19e-08 &  1.02e-15 &      \\ 
     &           &    9 &  3.58e-08 &  1.20e-15 &      \\ 
     &           &   10 &  3.58e-08 &  1.02e-15 &      \\ 
1205 &  1.20e+04 &   10 &           &           & iters  \\ 
 \hdashline 
     &           &    1 &  3.39e-08 &  1.97e-09 &      \\ 
     &           &    2 &  3.38e-08 &  9.67e-16 &      \\ 
     &           &    3 &  4.39e-08 &  9.66e-16 &      \\ 
     &           &    4 &  3.38e-08 &  1.25e-15 &      \\ 
     &           &    5 &  4.39e-08 &  9.66e-16 &      \\ 
     &           &    6 &  3.39e-08 &  1.25e-15 &      \\ 
     &           &    7 &  4.39e-08 &  9.66e-16 &      \\ 
     &           &    8 &  3.39e-08 &  1.25e-15 &      \\ 
     &           &    9 &  3.38e-08 &  9.67e-16 &      \\ 
     &           &   10 &  4.39e-08 &  9.65e-16 &      \\ 
1206 &  1.20e+04 &   10 &           &           & iters  \\ 
 \hdashline 
     &           &    1 &  9.39e-09 &  1.22e-09 &      \\ 
     &           &    2 &  9.38e-09 &  2.68e-16 &      \\ 
     &           &    3 &  9.37e-09 &  2.68e-16 &      \\ 
     &           &    4 &  9.35e-09 &  2.67e-16 &      \\ 
     &           &    5 &  9.34e-09 &  2.67e-16 &      \\ 
     &           &    6 &  6.84e-08 &  2.67e-16 &      \\ 
     &           &    7 &  9.42e-09 &  1.95e-15 &      \\ 
     &           &    8 &  9.41e-09 &  2.69e-16 &      \\ 
     &           &    9 &  9.40e-09 &  2.69e-16 &      \\ 
     &           &   10 &  9.38e-09 &  2.68e-16 &      \\ 
1207 &  1.21e+04 &   10 &           &           & iters  \\ 
 \hdashline 
     &           &    1 &  2.81e-08 &  1.61e-09 &      \\ 
     &           &    2 &  4.96e-08 &  8.02e-16 &      \\ 
     &           &    3 &  2.81e-08 &  1.42e-15 &      \\ 
     &           &    4 &  2.81e-08 &  8.03e-16 &      \\ 
     &           &    5 &  4.96e-08 &  8.02e-16 &      \\ 
     &           &    6 &  2.81e-08 &  1.42e-15 &      \\ 
     &           &    7 &  2.81e-08 &  8.02e-16 &      \\ 
     &           &    8 &  4.97e-08 &  8.01e-16 &      \\ 
     &           &    9 &  2.81e-08 &  1.42e-15 &      \\ 
     &           &   10 &  2.81e-08 &  8.02e-16 &      \\ 
1208 &  1.21e+04 &   10 &           &           & iters  \\ 
 \hdashline 
     &           &    1 &  9.33e-08 &  2.20e-09 &      \\ 
     &           &    2 &  6.22e-08 &  2.66e-15 &      \\ 
     &           &    3 &  1.56e-08 &  1.78e-15 &      \\ 
     &           &    4 &  1.55e-08 &  4.44e-16 &      \\ 
     &           &    5 &  1.55e-08 &  4.43e-16 &      \\ 
     &           &    6 &  6.22e-08 &  4.43e-16 &      \\ 
     &           &    7 &  1.56e-08 &  1.78e-15 &      \\ 
     &           &    8 &  1.56e-08 &  4.45e-16 &      \\ 
     &           &    9 &  1.55e-08 &  4.44e-16 &      \\ 
     &           &   10 &  1.55e-08 &  4.43e-16 &      \\ 
1209 &  1.21e+04 &   10 &           &           & iters  \\ 
 \hdashline 
     &           &    1 &  4.59e-09 &  5.87e-09 &      \\ 
     &           &    2 &  4.58e-09 &  1.31e-16 &      \\ 
     &           &    3 &  4.58e-09 &  1.31e-16 &      \\ 
     &           &    4 &  7.31e-08 &  1.31e-16 &      \\ 
     &           &    5 &  4.35e-08 &  2.09e-15 &      \\ 
     &           &    6 &  4.61e-09 &  1.24e-15 &      \\ 
     &           &    7 &  4.61e-09 &  1.32e-16 &      \\ 
     &           &    8 &  4.60e-09 &  1.31e-16 &      \\ 
     &           &    9 &  4.59e-09 &  1.31e-16 &      \\ 
     &           &   10 &  4.59e-09 &  1.31e-16 &      \\ 
1210 &  1.21e+04 &   10 &           &           & iters  \\ 
 \hdashline 
     &           &    1 &  6.69e-08 &  2.25e-09 &      \\ 
     &           &    2 &  1.09e-08 &  1.91e-15 &      \\ 
     &           &    3 &  1.09e-08 &  3.10e-16 &      \\ 
     &           &    4 &  1.08e-08 &  3.10e-16 &      \\ 
     &           &    5 &  1.08e-08 &  3.09e-16 &      \\ 
     &           &    6 &  1.08e-08 &  3.09e-16 &      \\ 
     &           &    7 &  1.08e-08 &  3.08e-16 &      \\ 
     &           &    8 &  6.69e-08 &  3.08e-16 &      \\ 
     &           &    9 &  1.09e-08 &  1.91e-15 &      \\ 
     &           &   10 &  1.09e-08 &  3.10e-16 &      \\ 
1211 &  1.21e+04 &   10 &           &           & iters  \\ 
 \hdashline 
     &           &    1 &  4.07e-08 &  1.58e-09 &      \\ 
     &           &    2 &  3.70e-08 &  1.16e-15 &      \\ 
     &           &    3 &  4.07e-08 &  1.06e-15 &      \\ 
     &           &    4 &  3.70e-08 &  1.16e-15 &      \\ 
     &           &    5 &  4.07e-08 &  1.06e-15 &      \\ 
     &           &    6 &  3.70e-08 &  1.16e-15 &      \\ 
     &           &    7 &  4.07e-08 &  1.06e-15 &      \\ 
     &           &    8 &  3.70e-08 &  1.16e-15 &      \\ 
     &           &    9 &  3.70e-08 &  1.06e-15 &      \\ 
     &           &   10 &  4.08e-08 &  1.06e-15 &      \\ 
1212 &  1.21e+04 &   10 &           &           & iters  \\ 
 \hdashline 
     &           &    1 &  1.92e-08 &  6.65e-10 &      \\ 
     &           &    2 &  5.85e-08 &  5.47e-16 &      \\ 
     &           &    3 &  1.92e-08 &  1.67e-15 &      \\ 
     &           &    4 &  1.92e-08 &  5.49e-16 &      \\ 
     &           &    5 &  1.92e-08 &  5.48e-16 &      \\ 
     &           &    6 &  5.85e-08 &  5.47e-16 &      \\ 
     &           &    7 &  1.92e-08 &  1.67e-15 &      \\ 
     &           &    8 &  1.92e-08 &  5.49e-16 &      \\ 
     &           &    9 &  1.92e-08 &  5.48e-16 &      \\ 
     &           &   10 &  5.85e-08 &  5.47e-16 &      \\ 
1213 &  1.21e+04 &   10 &           &           & iters  \\ 
 \hdashline 
     &           &    1 &  2.09e-08 &  1.77e-09 &      \\ 
     &           &    2 &  2.09e-08 &  5.97e-16 &      \\ 
     &           &    3 &  1.80e-08 &  5.97e-16 &      \\ 
     &           &    4 &  2.09e-08 &  5.13e-16 &      \\ 
     &           &    5 &  1.80e-08 &  5.96e-16 &      \\ 
     &           &    6 &  2.09e-08 &  5.13e-16 &      \\ 
     &           &    7 &  1.80e-08 &  5.96e-16 &      \\ 
     &           &    8 &  2.09e-08 &  5.13e-16 &      \\ 
     &           &    9 &  1.80e-08 &  5.96e-16 &      \\ 
     &           &   10 &  2.09e-08 &  5.13e-16 &      \\ 
1214 &  1.21e+04 &   10 &           &           & iters  \\ 
 \hdashline 
     &           &    1 &  3.85e-08 &  1.84e-10 &      \\ 
     &           &    2 &  3.92e-08 &  1.10e-15 &      \\ 
     &           &    3 &  3.85e-08 &  1.12e-15 &      \\ 
     &           &    4 &  3.92e-08 &  1.10e-15 &      \\ 
     &           &    5 &  3.85e-08 &  1.12e-15 &      \\ 
     &           &    6 &  3.92e-08 &  1.10e-15 &      \\ 
     &           &    7 &  3.85e-08 &  1.12e-15 &      \\ 
     &           &    8 &  3.92e-08 &  1.10e-15 &      \\ 
     &           &    9 &  3.85e-08 &  1.12e-15 &      \\ 
     &           &   10 &  3.92e-08 &  1.10e-15 &      \\ 
1215 &  1.21e+04 &   10 &           &           & iters  \\ 
 \hdashline 
     &           &    1 &  6.65e-08 &  2.85e-09 &      \\ 
     &           &    2 &  1.13e-08 &  1.90e-15 &      \\ 
     &           &    3 &  1.13e-08 &  3.22e-16 &      \\ 
     &           &    4 &  1.13e-08 &  3.22e-16 &      \\ 
     &           &    5 &  1.12e-08 &  3.21e-16 &      \\ 
     &           &    6 &  1.12e-08 &  3.21e-16 &      \\ 
     &           &    7 &  1.12e-08 &  3.20e-16 &      \\ 
     &           &    8 &  6.65e-08 &  3.20e-16 &      \\ 
     &           &    9 &  1.13e-08 &  1.90e-15 &      \\ 
     &           &   10 &  1.13e-08 &  3.22e-16 &      \\ 
1216 &  1.22e+04 &   10 &           &           & iters  \\ 
 \hdashline 
     &           &    1 &  2.32e-08 &  2.40e-09 &      \\ 
     &           &    2 &  2.32e-08 &  6.62e-16 &      \\ 
     &           &    3 &  2.31e-08 &  6.61e-16 &      \\ 
     &           &    4 &  5.46e-08 &  6.61e-16 &      \\ 
     &           &    5 &  2.32e-08 &  1.56e-15 &      \\ 
     &           &    6 &  2.32e-08 &  6.62e-16 &      \\ 
     &           &    7 &  5.46e-08 &  6.61e-16 &      \\ 
     &           &    8 &  2.32e-08 &  1.56e-15 &      \\ 
     &           &    9 &  2.32e-08 &  6.62e-16 &      \\ 
     &           &   10 &  2.31e-08 &  6.61e-16 &      \\ 
1217 &  1.22e+04 &   10 &           &           & iters  \\ 
 \hdashline 
     &           &    1 &  7.18e-08 &  1.60e-09 &      \\ 
     &           &    2 &  6.02e-09 &  2.05e-15 &      \\ 
     &           &    3 &  6.01e-09 &  1.72e-16 &      \\ 
     &           &    4 &  6.00e-09 &  1.71e-16 &      \\ 
     &           &    5 &  5.99e-09 &  1.71e-16 &      \\ 
     &           &    6 &  7.17e-08 &  1.71e-16 &      \\ 
     &           &    7 &  8.38e-08 &  2.05e-15 &      \\ 
     &           &    8 &  7.17e-08 &  2.39e-15 &      \\ 
     &           &    9 &  6.07e-09 &  2.05e-15 &      \\ 
     &           &   10 &  6.06e-09 &  1.73e-16 &      \\ 
1218 &  1.22e+04 &   10 &           &           & iters  \\ 
 \hdashline 
     &           &    1 &  8.43e-09 &  3.46e-10 &      \\ 
     &           &    2 &  8.42e-09 &  2.41e-16 &      \\ 
     &           &    3 &  8.41e-09 &  2.40e-16 &      \\ 
     &           &    4 &  8.40e-09 &  2.40e-16 &      \\ 
     &           &    5 &  8.39e-09 &  2.40e-16 &      \\ 
     &           &    6 &  8.37e-09 &  2.39e-16 &      \\ 
     &           &    7 &  6.93e-08 &  2.39e-16 &      \\ 
     &           &    8 &  8.46e-09 &  1.98e-15 &      \\ 
     &           &    9 &  8.45e-09 &  2.41e-16 &      \\ 
     &           &   10 &  8.44e-09 &  2.41e-16 &      \\ 
1219 &  1.22e+04 &   10 &           &           & iters  \\ 
 \hdashline 
     &           &    1 &  7.81e-09 &  1.28e-09 &      \\ 
     &           &    2 &  7.80e-09 &  2.23e-16 &      \\ 
     &           &    3 &  7.79e-09 &  2.23e-16 &      \\ 
     &           &    4 &  7.77e-09 &  2.22e-16 &      \\ 
     &           &    5 &  6.99e-08 &  2.22e-16 &      \\ 
     &           &    6 &  7.86e-09 &  2.00e-15 &      \\ 
     &           &    7 &  7.85e-09 &  2.24e-16 &      \\ 
     &           &    8 &  7.84e-09 &  2.24e-16 &      \\ 
     &           &    9 &  7.83e-09 &  2.24e-16 &      \\ 
     &           &   10 &  7.82e-09 &  2.23e-16 &      \\ 
1220 &  1.22e+04 &   10 &           &           & iters  \\ 
 \hdashline 
     &           &    1 &  3.96e-08 &  7.51e-11 &      \\ 
     &           &    2 &  3.81e-08 &  1.13e-15 &      \\ 
     &           &    3 &  3.96e-08 &  1.09e-15 &      \\ 
     &           &    4 &  3.81e-08 &  1.13e-15 &      \\ 
     &           &    5 &  3.96e-08 &  1.09e-15 &      \\ 
     &           &    6 &  3.81e-08 &  1.13e-15 &      \\ 
     &           &    7 &  3.96e-08 &  1.09e-15 &      \\ 
     &           &    8 &  3.81e-08 &  1.13e-15 &      \\ 
     &           &    9 &  3.96e-08 &  1.09e-15 &      \\ 
     &           &   10 &  3.81e-08 &  1.13e-15 &      \\ 
1221 &  1.22e+04 &   10 &           &           & iters  \\ 
 \hdashline 
     &           &    1 &  2.89e-08 &  2.91e-10 &      \\ 
     &           &    2 &  2.89e-08 &  8.26e-16 &      \\ 
     &           &    3 &  4.88e-08 &  8.25e-16 &      \\ 
     &           &    4 &  2.89e-08 &  1.39e-15 &      \\ 
     &           &    5 &  2.89e-08 &  8.25e-16 &      \\ 
     &           &    6 &  4.89e-08 &  8.24e-16 &      \\ 
     &           &    7 &  2.89e-08 &  1.39e-15 &      \\ 
     &           &    8 &  4.88e-08 &  8.25e-16 &      \\ 
     &           &    9 &  2.89e-08 &  1.39e-15 &      \\ 
     &           &   10 &  2.89e-08 &  8.26e-16 &      \\ 
1222 &  1.22e+04 &   10 &           &           & iters  \\ 
 \hdashline 
     &           &    1 &  3.77e-08 &  1.01e-09 &      \\ 
     &           &    2 &  4.01e-08 &  1.08e-15 &      \\ 
     &           &    3 &  3.77e-08 &  1.14e-15 &      \\ 
     &           &    4 &  4.00e-08 &  1.08e-15 &      \\ 
     &           &    5 &  3.77e-08 &  1.14e-15 &      \\ 
     &           &    6 &  4.00e-08 &  1.08e-15 &      \\ 
     &           &    7 &  3.77e-08 &  1.14e-15 &      \\ 
     &           &    8 &  4.00e-08 &  1.08e-15 &      \\ 
     &           &    9 &  3.77e-08 &  1.14e-15 &      \\ 
     &           &   10 &  4.00e-08 &  1.08e-15 &      \\ 
1223 &  1.22e+04 &   10 &           &           & iters  \\ 
 \hdashline 
     &           &    1 &  6.03e-08 &  4.09e-10 &      \\ 
     &           &    2 &  1.75e-08 &  1.72e-15 &      \\ 
     &           &    3 &  1.74e-08 &  4.98e-16 &      \\ 
     &           &    4 &  1.74e-08 &  4.98e-16 &      \\ 
     &           &    5 &  6.03e-08 &  4.97e-16 &      \\ 
     &           &    6 &  1.75e-08 &  1.72e-15 &      \\ 
     &           &    7 &  1.75e-08 &  4.99e-16 &      \\ 
     &           &    8 &  1.74e-08 &  4.98e-16 &      \\ 
     &           &    9 &  6.03e-08 &  4.97e-16 &      \\ 
     &           &   10 &  1.75e-08 &  1.72e-15 &      \\ 
1224 &  1.22e+04 &   10 &           &           & iters  \\ 
 \hdashline 
     &           &    1 &  1.15e-08 &  1.51e-09 &      \\ 
     &           &    2 &  6.62e-08 &  3.27e-16 &      \\ 
     &           &    3 &  1.15e-08 &  1.89e-15 &      \\ 
     &           &    4 &  1.15e-08 &  3.30e-16 &      \\ 
     &           &    5 &  1.15e-08 &  3.29e-16 &      \\ 
     &           &    6 &  1.15e-08 &  3.29e-16 &      \\ 
     &           &    7 &  1.15e-08 &  3.28e-16 &      \\ 
     &           &    8 &  1.15e-08 &  3.28e-16 &      \\ 
     &           &    9 &  6.62e-08 &  3.27e-16 &      \\ 
     &           &   10 &  1.15e-08 &  1.89e-15 &      \\ 
1225 &  1.22e+04 &   10 &           &           & iters  \\ 
 \hdashline 
     &           &    1 &  1.92e-08 &  7.87e-10 &      \\ 
     &           &    2 &  1.91e-08 &  5.47e-16 &      \\ 
     &           &    3 &  1.91e-08 &  5.46e-16 &      \\ 
     &           &    4 &  5.86e-08 &  5.45e-16 &      \\ 
     &           &    5 &  1.92e-08 &  1.67e-15 &      \\ 
     &           &    6 &  1.91e-08 &  5.47e-16 &      \\ 
     &           &    7 &  1.91e-08 &  5.46e-16 &      \\ 
     &           &    8 &  5.86e-08 &  5.45e-16 &      \\ 
     &           &    9 &  1.92e-08 &  1.67e-15 &      \\ 
     &           &   10 &  1.91e-08 &  5.47e-16 &      \\ 
1226 &  1.22e+04 &   10 &           &           & iters  \\ 
 \hdashline 
     &           &    1 &  2.17e-08 &  8.66e-10 &      \\ 
     &           &    2 &  2.16e-08 &  6.18e-16 &      \\ 
     &           &    3 &  2.16e-08 &  6.17e-16 &      \\ 
     &           &    4 &  5.61e-08 &  6.16e-16 &      \\ 
     &           &    5 &  2.16e-08 &  1.60e-15 &      \\ 
     &           &    6 &  2.16e-08 &  6.18e-16 &      \\ 
     &           &    7 &  5.61e-08 &  6.17e-16 &      \\ 
     &           &    8 &  2.17e-08 &  1.60e-15 &      \\ 
     &           &    9 &  2.16e-08 &  6.18e-16 &      \\ 
     &           &   10 &  2.16e-08 &  6.17e-16 &      \\ 
1227 &  1.23e+04 &   10 &           &           & iters  \\ 
 \hdashline 
     &           &    1 &  5.66e-08 &  1.27e-09 &      \\ 
     &           &    2 &  2.12e-08 &  1.61e-15 &      \\ 
     &           &    3 &  2.12e-08 &  6.05e-16 &      \\ 
     &           &    4 &  5.66e-08 &  6.04e-16 &      \\ 
     &           &    5 &  2.12e-08 &  1.61e-15 &      \\ 
     &           &    6 &  2.12e-08 &  6.06e-16 &      \\ 
     &           &    7 &  2.12e-08 &  6.05e-16 &      \\ 
     &           &    8 &  5.66e-08 &  6.04e-16 &      \\ 
     &           &    9 &  2.12e-08 &  1.61e-15 &      \\ 
     &           &   10 &  2.12e-08 &  6.05e-16 &      \\ 
1228 &  1.23e+04 &   10 &           &           & iters  \\ 
 \hdashline 
     &           &    1 &  4.16e-08 &  1.20e-09 &      \\ 
     &           &    2 &  3.62e-08 &  1.19e-15 &      \\ 
     &           &    3 &  3.61e-08 &  1.03e-15 &      \\ 
     &           &    4 &  4.16e-08 &  1.03e-15 &      \\ 
     &           &    5 &  3.61e-08 &  1.19e-15 &      \\ 
     &           &    6 &  4.16e-08 &  1.03e-15 &      \\ 
     &           &    7 &  3.62e-08 &  1.19e-15 &      \\ 
     &           &    8 &  4.16e-08 &  1.03e-15 &      \\ 
     &           &    9 &  3.62e-08 &  1.19e-15 &      \\ 
     &           &   10 &  4.16e-08 &  1.03e-15 &      \\ 
1229 &  1.23e+04 &   10 &           &           & iters  \\ 
 \hdashline 
     &           &    1 &  3.75e-08 &  1.32e-09 &      \\ 
     &           &    2 &  4.03e-08 &  1.07e-15 &      \\ 
     &           &    3 &  3.75e-08 &  1.15e-15 &      \\ 
     &           &    4 &  4.03e-08 &  1.07e-15 &      \\ 
     &           &    5 &  3.75e-08 &  1.15e-15 &      \\ 
     &           &    6 &  4.03e-08 &  1.07e-15 &      \\ 
     &           &    7 &  3.75e-08 &  1.15e-15 &      \\ 
     &           &    8 &  4.03e-08 &  1.07e-15 &      \\ 
     &           &    9 &  3.75e-08 &  1.15e-15 &      \\ 
     &           &   10 &  3.74e-08 &  1.07e-15 &      \\ 
1230 &  1.23e+04 &   10 &           &           & iters  \\ 
 \hdashline 
     &           &    1 &  1.07e-07 &  3.36e-11 &      \\ 
     &           &    2 &  4.84e-08 &  3.06e-15 &      \\ 
     &           &    3 &  2.94e-08 &  1.38e-15 &      \\ 
     &           &    4 &  4.83e-08 &  8.39e-16 &      \\ 
     &           &    5 &  2.94e-08 &  1.38e-15 &      \\ 
     &           &    6 &  2.94e-08 &  8.40e-16 &      \\ 
     &           &    7 &  4.84e-08 &  8.38e-16 &      \\ 
     &           &    8 &  2.94e-08 &  1.38e-15 &      \\ 
     &           &    9 &  4.83e-08 &  8.39e-16 &      \\ 
     &           &   10 &  2.94e-08 &  1.38e-15 &      \\ 
1231 &  1.23e+04 &   10 &           &           & iters  \\ 
 \hdashline 
     &           &    1 &  5.95e-08 &  1.62e-09 &      \\ 
     &           &    2 &  1.83e-08 &  1.70e-15 &      \\ 
     &           &    3 &  1.83e-08 &  5.22e-16 &      \\ 
     &           &    4 &  1.82e-08 &  5.22e-16 &      \\ 
     &           &    5 &  1.82e-08 &  5.21e-16 &      \\ 
     &           &    6 &  5.95e-08 &  5.20e-16 &      \\ 
     &           &    7 &  1.83e-08 &  1.70e-15 &      \\ 
     &           &    8 &  1.83e-08 &  5.22e-16 &      \\ 
     &           &    9 &  1.82e-08 &  5.21e-16 &      \\ 
     &           &   10 &  5.95e-08 &  5.20e-16 &      \\ 
1232 &  1.23e+04 &   10 &           &           & iters  \\ 
 \hdashline 
     &           &    1 &  2.54e-08 &  1.14e-09 &      \\ 
     &           &    2 &  2.54e-08 &  7.26e-16 &      \\ 
     &           &    3 &  2.54e-08 &  7.25e-16 &      \\ 
     &           &    4 &  5.24e-08 &  7.24e-16 &      \\ 
     &           &    5 &  2.54e-08 &  1.49e-15 &      \\ 
     &           &    6 &  2.54e-08 &  7.25e-16 &      \\ 
     &           &    7 &  5.24e-08 &  7.24e-16 &      \\ 
     &           &    8 &  2.54e-08 &  1.49e-15 &      \\ 
     &           &    9 &  2.54e-08 &  7.25e-16 &      \\ 
     &           &   10 &  5.24e-08 &  7.24e-16 &      \\ 
1233 &  1.23e+04 &   10 &           &           & iters  \\ 
 \hdashline 
     &           &    1 &  5.10e-08 &  1.83e-09 &      \\ 
     &           &    2 &  2.67e-08 &  1.46e-15 &      \\ 
     &           &    3 &  5.10e-08 &  7.63e-16 &      \\ 
     &           &    4 &  2.68e-08 &  1.46e-15 &      \\ 
     &           &    5 &  2.67e-08 &  7.64e-16 &      \\ 
     &           &    6 &  5.10e-08 &  7.63e-16 &      \\ 
     &           &    7 &  2.68e-08 &  1.46e-15 &      \\ 
     &           &    8 &  2.67e-08 &  7.64e-16 &      \\ 
     &           &    9 &  5.10e-08 &  7.63e-16 &      \\ 
     &           &   10 &  2.68e-08 &  1.46e-15 &      \\ 
1234 &  1.23e+04 &   10 &           &           & iters  \\ 
 \hdashline 
     &           &    1 &  1.12e-08 &  1.33e-09 &      \\ 
     &           &    2 &  1.12e-08 &  3.20e-16 &      \\ 
     &           &    3 &  1.12e-08 &  3.19e-16 &      \\ 
     &           &    4 &  1.11e-08 &  3.19e-16 &      \\ 
     &           &    5 &  1.11e-08 &  3.18e-16 &      \\ 
     &           &    6 &  6.66e-08 &  3.18e-16 &      \\ 
     &           &    7 &  1.12e-08 &  1.90e-15 &      \\ 
     &           &    8 &  1.12e-08 &  3.20e-16 &      \\ 
     &           &    9 &  1.12e-08 &  3.20e-16 &      \\ 
     &           &   10 &  1.12e-08 &  3.19e-16 &      \\ 
1235 &  1.23e+04 &   10 &           &           & iters  \\ 
 \hdashline 
     &           &    1 &  5.75e-08 &  1.81e-09 &      \\ 
     &           &    2 &  2.03e-08 &  1.64e-15 &      \\ 
     &           &    3 &  2.03e-08 &  5.80e-16 &      \\ 
     &           &    4 &  5.74e-08 &  5.79e-16 &      \\ 
     &           &    5 &  2.03e-08 &  1.64e-15 &      \\ 
     &           &    6 &  2.03e-08 &  5.80e-16 &      \\ 
     &           &    7 &  2.03e-08 &  5.79e-16 &      \\ 
     &           &    8 &  5.74e-08 &  5.79e-16 &      \\ 
     &           &    9 &  2.03e-08 &  1.64e-15 &      \\ 
     &           &   10 &  2.03e-08 &  5.80e-16 &      \\ 
1236 &  1.24e+04 &   10 &           &           & iters  \\ 
 \hdashline 
     &           &    1 &  7.95e-09 &  1.08e-09 &      \\ 
     &           &    2 &  7.94e-09 &  2.27e-16 &      \\ 
     &           &    3 &  7.92e-09 &  2.26e-16 &      \\ 
     &           &    4 &  7.91e-09 &  2.26e-16 &      \\ 
     &           &    5 &  7.90e-09 &  2.26e-16 &      \\ 
     &           &    6 &  7.89e-09 &  2.26e-16 &      \\ 
     &           &    7 &  6.98e-08 &  2.25e-16 &      \\ 
     &           &    8 &  7.98e-09 &  1.99e-15 &      \\ 
     &           &    9 &  7.97e-09 &  2.28e-16 &      \\ 
     &           &   10 &  7.96e-09 &  2.27e-16 &      \\ 
1237 &  1.24e+04 &   10 &           &           & iters  \\ 
 \hdashline 
     &           &    1 &  2.02e-08 &  1.27e-09 &      \\ 
     &           &    2 &  5.75e-08 &  5.78e-16 &      \\ 
     &           &    3 &  2.03e-08 &  1.64e-15 &      \\ 
     &           &    4 &  2.03e-08 &  5.79e-16 &      \\ 
     &           &    5 &  2.02e-08 &  5.79e-16 &      \\ 
     &           &    6 &  5.75e-08 &  5.78e-16 &      \\ 
     &           &    7 &  2.03e-08 &  1.64e-15 &      \\ 
     &           &    8 &  2.03e-08 &  5.79e-16 &      \\ 
     &           &    9 &  2.02e-08 &  5.78e-16 &      \\ 
     &           &   10 &  5.75e-08 &  5.78e-16 &      \\ 
1238 &  1.24e+04 &   10 &           &           & iters  \\ 
 \hdashline 
     &           &    1 &  1.66e-09 &  1.71e-09 &      \\ 
     &           &    2 &  1.66e-09 &  4.75e-17 &      \\ 
     &           &    3 &  1.66e-09 &  4.74e-17 &      \\ 
     &           &    4 &  1.66e-09 &  4.73e-17 &      \\ 
     &           &    5 &  1.65e-09 &  4.73e-17 &      \\ 
     &           &    6 &  1.65e-09 &  4.72e-17 &      \\ 
     &           &    7 &  1.65e-09 &  4.71e-17 &      \\ 
     &           &    8 &  1.65e-09 &  4.71e-17 &      \\ 
     &           &    9 &  1.64e-09 &  4.70e-17 &      \\ 
     &           &   10 &  1.64e-09 &  4.69e-17 &      \\ 
1239 &  1.24e+04 &   10 &           &           & iters  \\ 
 \hdashline 
     &           &    1 &  6.53e-09 &  3.28e-09 &      \\ 
     &           &    2 &  6.53e-09 &  1.87e-16 &      \\ 
     &           &    3 &  3.23e-08 &  1.86e-16 &      \\ 
     &           &    4 &  6.56e-09 &  9.23e-16 &      \\ 
     &           &    5 &  6.55e-09 &  1.87e-16 &      \\ 
     &           &    6 &  6.54e-09 &  1.87e-16 &      \\ 
     &           &    7 &  6.53e-09 &  1.87e-16 &      \\ 
     &           &    8 &  6.52e-09 &  1.86e-16 &      \\ 
     &           &    9 &  3.23e-08 &  1.86e-16 &      \\ 
     &           &   10 &  6.56e-09 &  9.23e-16 &      \\ 
1240 &  1.24e+04 &   10 &           &           & iters  \\ 
 \hdashline 
     &           &    1 &  2.12e-08 &  4.78e-09 &      \\ 
     &           &    2 &  5.66e-08 &  6.04e-16 &      \\ 
     &           &    3 &  2.12e-08 &  1.61e-15 &      \\ 
     &           &    4 &  2.12e-08 &  6.05e-16 &      \\ 
     &           &    5 &  2.12e-08 &  6.05e-16 &      \\ 
     &           &    6 &  5.66e-08 &  6.04e-16 &      \\ 
     &           &    7 &  2.12e-08 &  1.61e-15 &      \\ 
     &           &    8 &  2.12e-08 &  6.05e-16 &      \\ 
     &           &    9 &  5.65e-08 &  6.04e-16 &      \\ 
     &           &   10 &  2.12e-08 &  1.61e-15 &      \\ 
1241 &  1.24e+04 &   10 &           &           & iters  \\ 
 \hdashline 
     &           &    1 &  9.01e-09 &  3.20e-09 &      \\ 
     &           &    2 &  9.00e-09 &  2.57e-16 &      \\ 
     &           &    3 &  8.99e-09 &  2.57e-16 &      \\ 
     &           &    4 &  6.87e-08 &  2.56e-16 &      \\ 
     &           &    5 &  4.79e-08 &  1.96e-15 &      \\ 
     &           &    6 &  9.00e-09 &  1.37e-15 &      \\ 
     &           &    7 &  8.99e-09 &  2.57e-16 &      \\ 
     &           &    8 &  6.87e-08 &  2.57e-16 &      \\ 
     &           &    9 &  4.79e-08 &  1.96e-15 &      \\ 
     &           &   10 &  9.01e-09 &  1.37e-15 &      \\ 
1242 &  1.24e+04 &   10 &           &           & iters  \\ 
 \hdashline 
     &           &    1 &  9.77e-09 &  1.02e-09 &      \\ 
     &           &    2 &  9.76e-09 &  2.79e-16 &      \\ 
     &           &    3 &  9.75e-09 &  2.79e-16 &      \\ 
     &           &    4 &  9.73e-09 &  2.78e-16 &      \\ 
     &           &    5 &  9.72e-09 &  2.78e-16 &      \\ 
     &           &    6 &  9.70e-09 &  2.77e-16 &      \\ 
     &           &    7 &  6.80e-08 &  2.77e-16 &      \\ 
     &           &    8 &  4.86e-08 &  1.94e-15 &      \\ 
     &           &    9 &  9.72e-09 &  1.39e-15 &      \\ 
     &           &   10 &  9.70e-09 &  2.77e-16 &      \\ 
1243 &  1.24e+04 &   10 &           &           & iters  \\ 
 \hdashline 
     &           &    1 &  1.57e-08 &  1.46e-09 &      \\ 
     &           &    2 &  1.57e-08 &  4.49e-16 &      \\ 
     &           &    3 &  1.57e-08 &  4.48e-16 &      \\ 
     &           &    4 &  1.56e-08 &  4.47e-16 &      \\ 
     &           &    5 &  6.21e-08 &  4.47e-16 &      \\ 
     &           &    6 &  1.57e-08 &  1.77e-15 &      \\ 
     &           &    7 &  1.57e-08 &  4.49e-16 &      \\ 
     &           &    8 &  1.57e-08 &  4.48e-16 &      \\ 
     &           &    9 &  1.56e-08 &  4.47e-16 &      \\ 
     &           &   10 &  6.21e-08 &  4.47e-16 &      \\ 
1244 &  1.24e+04 &   10 &           &           & iters  \\ 
 \hdashline 
     &           &    1 &  3.52e-08 &  2.73e-09 &      \\ 
     &           &    2 &  3.68e-09 &  1.01e-15 &      \\ 
     &           &    3 &  3.68e-09 &  1.05e-16 &      \\ 
     &           &    4 &  3.67e-09 &  1.05e-16 &      \\ 
     &           &    5 &  3.52e-08 &  1.05e-16 &      \\ 
     &           &    6 &  3.72e-09 &  1.00e-15 &      \\ 
     &           &    7 &  3.71e-09 &  1.06e-16 &      \\ 
     &           &    8 &  3.70e-09 &  1.06e-16 &      \\ 
     &           &    9 &  3.70e-09 &  1.06e-16 &      \\ 
     &           &   10 &  3.69e-09 &  1.06e-16 &      \\ 
1245 &  1.24e+04 &   10 &           &           & iters  \\ 
 \hdashline 
     &           &    1 &  3.12e-08 &  2.59e-09 &      \\ 
     &           &    2 &  4.65e-08 &  8.92e-16 &      \\ 
     &           &    3 &  3.13e-08 &  1.33e-15 &      \\ 
     &           &    4 &  4.65e-08 &  8.92e-16 &      \\ 
     &           &    5 &  3.13e-08 &  1.33e-15 &      \\ 
     &           &    6 &  3.12e-08 &  8.93e-16 &      \\ 
     &           &    7 &  4.65e-08 &  8.92e-16 &      \\ 
     &           &    8 &  3.13e-08 &  1.33e-15 &      \\ 
     &           &    9 &  4.65e-08 &  8.92e-16 &      \\ 
     &           &   10 &  3.13e-08 &  1.33e-15 &      \\ 
1246 &  1.24e+04 &   10 &           &           & iters  \\ 
 \hdashline 
     &           &    1 &  1.85e-08 &  4.12e-10 &      \\ 
     &           &    2 &  1.85e-08 &  5.29e-16 &      \\ 
     &           &    3 &  5.92e-08 &  5.28e-16 &      \\ 
     &           &    4 &  1.86e-08 &  1.69e-15 &      \\ 
     &           &    5 &  1.85e-08 &  5.29e-16 &      \\ 
     &           &    6 &  1.85e-08 &  5.29e-16 &      \\ 
     &           &    7 &  5.92e-08 &  5.28e-16 &      \\ 
     &           &    8 &  1.86e-08 &  1.69e-15 &      \\ 
     &           &    9 &  1.85e-08 &  5.30e-16 &      \\ 
     &           &   10 &  1.85e-08 &  5.29e-16 &      \\ 
1247 &  1.25e+04 &   10 &           &           & iters  \\ 
 \hdashline 
     &           &    1 &  3.67e-08 &  4.71e-09 &      \\ 
     &           &    2 &  2.23e-09 &  1.05e-15 &      \\ 
     &           &    3 &  2.23e-09 &  6.37e-17 &      \\ 
     &           &    4 &  2.22e-09 &  6.36e-17 &      \\ 
     &           &    5 &  2.22e-09 &  6.35e-17 &      \\ 
     &           &    6 &  2.22e-09 &  6.34e-17 &      \\ 
     &           &    7 &  2.22e-09 &  6.33e-17 &      \\ 
     &           &    8 &  2.21e-09 &  6.32e-17 &      \\ 
     &           &    9 &  2.21e-09 &  6.31e-17 &      \\ 
     &           &   10 &  2.21e-09 &  6.30e-17 &      \\ 
1248 &  1.25e+04 &   10 &           &           & iters  \\ 
 \hdashline 
     &           &    1 &  1.85e-08 &  8.16e-09 &      \\ 
     &           &    2 &  1.85e-08 &  5.28e-16 &      \\ 
     &           &    3 &  5.92e-08 &  5.28e-16 &      \\ 
     &           &    4 &  1.85e-08 &  1.69e-15 &      \\ 
     &           &    5 &  1.85e-08 &  5.29e-16 &      \\ 
     &           &    6 &  1.85e-08 &  5.29e-16 &      \\ 
     &           &    7 &  1.85e-08 &  5.28e-16 &      \\ 
     &           &    8 &  5.92e-08 &  5.27e-16 &      \\ 
     &           &    9 &  1.85e-08 &  1.69e-15 &      \\ 
     &           &   10 &  1.85e-08 &  5.29e-16 &      \\ 
1249 &  1.25e+04 &   10 &           &           & iters  \\ 
 \hdashline 
     &           &    1 &  6.86e-09 &  4.60e-09 &      \\ 
     &           &    2 &  6.85e-09 &  1.96e-16 &      \\ 
     &           &    3 &  6.84e-09 &  1.95e-16 &      \\ 
     &           &    4 &  6.83e-09 &  1.95e-16 &      \\ 
     &           &    5 &  6.82e-09 &  1.95e-16 &      \\ 
     &           &    6 &  6.81e-09 &  1.95e-16 &      \\ 
     &           &    7 &  6.80e-09 &  1.94e-16 &      \\ 
     &           &    8 &  7.09e-08 &  1.94e-16 &      \\ 
     &           &    9 &  8.46e-08 &  2.02e-15 &      \\ 
     &           &   10 &  7.09e-08 &  2.41e-15 &      \\ 
1250 &  1.25e+04 &   10 &           &           & iters  \\ 
 \hdashline 
     &           &    1 &  2.43e-08 &  9.95e-10 &      \\ 
     &           &    2 &  2.42e-08 &  6.93e-16 &      \\ 
     &           &    3 &  2.42e-08 &  6.92e-16 &      \\ 
     &           &    4 &  5.35e-08 &  6.91e-16 &      \\ 
     &           &    5 &  2.42e-08 &  1.53e-15 &      \\ 
     &           &    6 &  2.42e-08 &  6.92e-16 &      \\ 
     &           &    7 &  2.42e-08 &  6.91e-16 &      \\ 
     &           &    8 &  5.36e-08 &  6.90e-16 &      \\ 
     &           &    9 &  2.42e-08 &  1.53e-15 &      \\ 
     &           &   10 &  2.42e-08 &  6.91e-16 &      \\ 
1251 &  1.25e+04 &   10 &           &           & iters  \\ 
 \hdashline 
     &           &    1 &  4.87e-08 &  2.31e-09 &      \\ 
     &           &    2 &  4.86e-08 &  1.39e-15 &      \\ 
     &           &    3 &  2.91e-08 &  1.39e-15 &      \\ 
     &           &    4 &  2.91e-08 &  8.31e-16 &      \\ 
     &           &    5 &  4.86e-08 &  8.30e-16 &      \\ 
     &           &    6 &  2.91e-08 &  1.39e-15 &      \\ 
     &           &    7 &  2.91e-08 &  8.31e-16 &      \\ 
     &           &    8 &  4.87e-08 &  8.30e-16 &      \\ 
     &           &    9 &  2.91e-08 &  1.39e-15 &      \\ 
     &           &   10 &  4.86e-08 &  8.31e-16 &      \\ 
1252 &  1.25e+04 &   10 &           &           & iters  \\ 
 \hdashline 
     &           &    1 &  1.08e-08 &  7.09e-10 &      \\ 
     &           &    2 &  1.08e-08 &  3.09e-16 &      \\ 
     &           &    3 &  1.08e-08 &  3.09e-16 &      \\ 
     &           &    4 &  1.08e-08 &  3.08e-16 &      \\ 
     &           &    5 &  1.08e-08 &  3.08e-16 &      \\ 
     &           &    6 &  2.81e-08 &  3.07e-16 &      \\ 
     &           &    7 &  1.08e-08 &  8.02e-16 &      \\ 
     &           &    8 &  1.08e-08 &  3.08e-16 &      \\ 
     &           &    9 &  2.81e-08 &  3.08e-16 &      \\ 
     &           &   10 &  1.08e-08 &  8.01e-16 &      \\ 
1253 &  1.25e+04 &   10 &           &           & iters  \\ 
 \hdashline 
     &           &    1 &  1.97e-08 &  1.94e-10 &      \\ 
     &           &    2 &  5.81e-08 &  5.61e-16 &      \\ 
     &           &    3 &  1.97e-08 &  1.66e-15 &      \\ 
     &           &    4 &  1.97e-08 &  5.62e-16 &      \\ 
     &           &    5 &  1.97e-08 &  5.62e-16 &      \\ 
     &           &    6 &  5.81e-08 &  5.61e-16 &      \\ 
     &           &    7 &  1.97e-08 &  1.66e-15 &      \\ 
     &           &    8 &  1.97e-08 &  5.62e-16 &      \\ 
     &           &    9 &  1.96e-08 &  5.62e-16 &      \\ 
     &           &   10 &  5.81e-08 &  5.61e-16 &      \\ 
1254 &  1.25e+04 &   10 &           &           & iters  \\ 
 \hdashline 
     &           &    1 &  6.06e-08 &  1.41e-09 &      \\ 
     &           &    2 &  1.72e-08 &  1.73e-15 &      \\ 
     &           &    3 &  1.72e-08 &  4.91e-16 &      \\ 
     &           &    4 &  1.72e-08 &  4.90e-16 &      \\ 
     &           &    5 &  1.71e-08 &  4.90e-16 &      \\ 
     &           &    6 &  6.06e-08 &  4.89e-16 &      \\ 
     &           &    7 &  1.72e-08 &  1.73e-15 &      \\ 
     &           &    8 &  1.72e-08 &  4.91e-16 &      \\ 
     &           &    9 &  1.71e-08 &  4.90e-16 &      \\ 
     &           &   10 &  6.06e-08 &  4.89e-16 &      \\ 
1255 &  1.25e+04 &   10 &           &           & iters  \\ 
 \hdashline 
     &           &    1 &  7.33e-08 &  2.67e-09 &      \\ 
     &           &    2 &  8.22e-08 &  2.09e-15 &      \\ 
     &           &    3 &  7.33e-08 &  2.35e-15 &      \\ 
     &           &    4 &  4.50e-09 &  2.09e-15 &      \\ 
     &           &    5 &  4.49e-09 &  1.28e-16 &      \\ 
     &           &    6 &  4.49e-09 &  1.28e-16 &      \\ 
     &           &    7 &  4.48e-09 &  1.28e-16 &      \\ 
     &           &    8 &  4.47e-09 &  1.28e-16 &      \\ 
     &           &    9 &  4.47e-09 &  1.28e-16 &      \\ 
     &           &   10 &  4.46e-09 &  1.27e-16 &      \\ 
1256 &  1.26e+04 &   10 &           &           & iters  \\ 
 \hdashline 
     &           &    1 &  2.89e-09 &  1.73e-09 &      \\ 
     &           &    2 &  2.88e-09 &  8.24e-17 &      \\ 
     &           &    3 &  2.88e-09 &  8.23e-17 &      \\ 
     &           &    4 &  2.88e-09 &  8.22e-17 &      \\ 
     &           &    5 &  2.87e-09 &  8.21e-17 &      \\ 
     &           &    6 &  2.87e-09 &  8.19e-17 &      \\ 
     &           &    7 &  2.86e-09 &  8.18e-17 &      \\ 
     &           &    8 &  2.86e-09 &  8.17e-17 &      \\ 
     &           &    9 &  2.85e-09 &  8.16e-17 &      \\ 
     &           &   10 &  2.85e-09 &  8.15e-17 &      \\ 
1257 &  1.26e+04 &   10 &           &           & iters  \\ 
 \hdashline 
     &           &    1 &  2.00e-08 &  1.44e-09 &      \\ 
     &           &    2 &  2.00e-08 &  5.72e-16 &      \\ 
     &           &    3 &  5.77e-08 &  5.71e-16 &      \\ 
     &           &    4 &  2.01e-08 &  1.65e-15 &      \\ 
     &           &    5 &  2.00e-08 &  5.73e-16 &      \\ 
     &           &    6 &  2.00e-08 &  5.72e-16 &      \\ 
     &           &    7 &  5.77e-08 &  5.71e-16 &      \\ 
     &           &    8 &  2.01e-08 &  1.65e-15 &      \\ 
     &           &    9 &  2.00e-08 &  5.72e-16 &      \\ 
     &           &   10 &  2.00e-08 &  5.72e-16 &      \\ 
1258 &  1.26e+04 &   10 &           &           & iters  \\ 
 \hdashline 
     &           &    1 &  2.60e-08 &  1.65e-09 &      \\ 
     &           &    2 &  5.18e-08 &  7.41e-16 &      \\ 
     &           &    3 &  2.60e-08 &  1.48e-15 &      \\ 
     &           &    4 &  2.60e-08 &  7.42e-16 &      \\ 
     &           &    5 &  5.18e-08 &  7.41e-16 &      \\ 
     &           &    6 &  2.60e-08 &  1.48e-15 &      \\ 
     &           &    7 &  2.60e-08 &  7.42e-16 &      \\ 
     &           &    8 &  5.18e-08 &  7.41e-16 &      \\ 
     &           &    9 &  2.60e-08 &  1.48e-15 &      \\ 
     &           &   10 &  2.60e-08 &  7.42e-16 &      \\ 
1259 &  1.26e+04 &   10 &           &           & iters  \\ 
 \hdashline 
     &           &    1 &  3.03e-08 &  2.50e-10 &      \\ 
     &           &    2 &  4.74e-08 &  8.65e-16 &      \\ 
     &           &    3 &  3.03e-08 &  1.35e-15 &      \\ 
     &           &    4 &  4.74e-08 &  8.65e-16 &      \\ 
     &           &    5 &  3.03e-08 &  1.35e-15 &      \\ 
     &           &    6 &  3.03e-08 &  8.66e-16 &      \\ 
     &           &    7 &  4.74e-08 &  8.65e-16 &      \\ 
     &           &    8 &  3.03e-08 &  1.35e-15 &      \\ 
     &           &    9 &  3.03e-08 &  8.66e-16 &      \\ 
     &           &   10 &  4.74e-08 &  8.64e-16 &      \\ 
1260 &  1.26e+04 &   10 &           &           & iters  \\ 
 \hdashline 
     &           &    1 &  2.67e-08 &  1.08e-09 &      \\ 
     &           &    2 &  5.10e-08 &  7.62e-16 &      \\ 
     &           &    3 &  2.67e-08 &  1.46e-15 &      \\ 
     &           &    4 &  2.67e-08 &  7.63e-16 &      \\ 
     &           &    5 &  5.10e-08 &  7.62e-16 &      \\ 
     &           &    6 &  2.67e-08 &  1.46e-15 &      \\ 
     &           &    7 &  2.67e-08 &  7.63e-16 &      \\ 
     &           &    8 &  5.10e-08 &  7.61e-16 &      \\ 
     &           &    9 &  2.67e-08 &  1.46e-15 &      \\ 
     &           &   10 &  5.10e-08 &  7.62e-16 &      \\ 
1261 &  1.26e+04 &   10 &           &           & iters  \\ 
 \hdashline 
     &           &    1 &  6.97e-08 &  3.36e-09 &      \\ 
     &           &    2 &  4.69e-08 &  1.99e-15 &      \\ 
     &           &    3 &  7.98e-09 &  1.34e-15 &      \\ 
     &           &    4 &  7.96e-09 &  2.28e-16 &      \\ 
     &           &    5 &  6.97e-08 &  2.27e-16 &      \\ 
     &           &    6 &  4.69e-08 &  1.99e-15 &      \\ 
     &           &    7 &  7.99e-09 &  1.34e-15 &      \\ 
     &           &    8 &  7.97e-09 &  2.28e-16 &      \\ 
     &           &    9 &  7.96e-09 &  2.28e-16 &      \\ 
     &           &   10 &  6.97e-08 &  2.27e-16 &      \\ 
1262 &  1.26e+04 &   10 &           &           & iters  \\ 
 \hdashline 
     &           &    1 &  5.95e-08 &  4.89e-09 &      \\ 
     &           &    2 &  1.83e-08 &  1.70e-15 &      \\ 
     &           &    3 &  1.83e-08 &  5.22e-16 &      \\ 
     &           &    4 &  1.83e-08 &  5.22e-16 &      \\ 
     &           &    5 &  5.95e-08 &  5.21e-16 &      \\ 
     &           &    6 &  1.83e-08 &  1.70e-15 &      \\ 
     &           &    7 &  1.83e-08 &  5.23e-16 &      \\ 
     &           &    8 &  1.83e-08 &  5.22e-16 &      \\ 
     &           &    9 &  5.95e-08 &  5.21e-16 &      \\ 
     &           &   10 &  1.83e-08 &  1.70e-15 &      \\ 
1263 &  1.26e+04 &   10 &           &           & iters  \\ 
 \hdashline 
     &           &    1 &  2.45e-09 &  4.92e-09 &      \\ 
     &           &    2 &  2.45e-09 &  6.99e-17 &      \\ 
     &           &    3 &  2.44e-09 &  6.98e-17 &      \\ 
     &           &    4 &  2.44e-09 &  6.97e-17 &      \\ 
     &           &    5 &  2.44e-09 &  6.96e-17 &      \\ 
     &           &    6 &  2.43e-09 &  6.95e-17 &      \\ 
     &           &    7 &  2.43e-09 &  6.94e-17 &      \\ 
     &           &    8 &  2.43e-09 &  6.93e-17 &      \\ 
     &           &    9 &  2.42e-09 &  6.92e-17 &      \\ 
     &           &   10 &  2.42e-09 &  6.91e-17 &      \\ 
1264 &  1.26e+04 &   10 &           &           & iters  \\ 
 \hdashline 
     &           &    1 &  1.02e-08 &  1.98e-09 &      \\ 
     &           &    2 &  1.02e-08 &  2.90e-16 &      \\ 
     &           &    3 &  1.01e-08 &  2.90e-16 &      \\ 
     &           &    4 &  1.01e-08 &  2.89e-16 &      \\ 
     &           &    5 &  6.76e-08 &  2.89e-16 &      \\ 
     &           &    6 &  1.02e-08 &  1.93e-15 &      \\ 
     &           &    7 &  1.02e-08 &  2.91e-16 &      \\ 
     &           &    8 &  1.02e-08 &  2.91e-16 &      \\ 
     &           &    9 &  1.02e-08 &  2.91e-16 &      \\ 
     &           &   10 &  1.01e-08 &  2.90e-16 &      \\ 
1265 &  1.26e+04 &   10 &           &           & iters  \\ 
 \hdashline 
     &           &    1 &  3.16e-08 &  3.39e-10 &      \\ 
     &           &    2 &  4.62e-08 &  9.01e-16 &      \\ 
     &           &    3 &  3.16e-08 &  1.32e-15 &      \\ 
     &           &    4 &  4.61e-08 &  9.02e-16 &      \\ 
     &           &    5 &  3.16e-08 &  1.32e-15 &      \\ 
     &           &    6 &  3.16e-08 &  9.02e-16 &      \\ 
     &           &    7 &  4.62e-08 &  9.01e-16 &      \\ 
     &           &    8 &  3.16e-08 &  1.32e-15 &      \\ 
     &           &    9 &  4.61e-08 &  9.02e-16 &      \\ 
     &           &   10 &  3.16e-08 &  1.32e-15 &      \\ 
1266 &  1.26e+04 &   10 &           &           & iters  \\ 
 \hdashline 
     &           &    1 &  6.61e-08 &  1.35e-09 &      \\ 
     &           &    2 &  1.17e-08 &  1.89e-15 &      \\ 
     &           &    3 &  1.17e-08 &  3.34e-16 &      \\ 
     &           &    4 &  1.17e-08 &  3.33e-16 &      \\ 
     &           &    5 &  1.16e-08 &  3.33e-16 &      \\ 
     &           &    6 &  1.16e-08 &  3.32e-16 &      \\ 
     &           &    7 &  6.61e-08 &  3.32e-16 &      \\ 
     &           &    8 &  1.17e-08 &  1.89e-15 &      \\ 
     &           &    9 &  1.17e-08 &  3.34e-16 &      \\ 
     &           &   10 &  1.17e-08 &  3.33e-16 &      \\ 
1267 &  1.27e+04 &   10 &           &           & iters  \\ 
 \hdashline 
     &           &    1 &  1.08e-08 &  1.62e-09 &      \\ 
     &           &    2 &  6.69e-08 &  3.07e-16 &      \\ 
     &           &    3 &  1.08e-08 &  1.91e-15 &      \\ 
     &           &    4 &  1.08e-08 &  3.09e-16 &      \\ 
     &           &    5 &  1.08e-08 &  3.09e-16 &      \\ 
     &           &    6 &  1.08e-08 &  3.09e-16 &      \\ 
     &           &    7 &  1.08e-08 &  3.08e-16 &      \\ 
     &           &    8 &  1.08e-08 &  3.08e-16 &      \\ 
     &           &    9 &  6.69e-08 &  3.07e-16 &      \\ 
     &           &   10 &  8.85e-08 &  1.91e-15 &      \\ 
1268 &  1.27e+04 &   10 &           &           & iters  \\ 
 \hdashline 
     &           &    1 &  6.10e-08 &  3.42e-10 &      \\ 
     &           &    2 &  1.68e-08 &  1.74e-15 &      \\ 
     &           &    3 &  6.09e-08 &  4.79e-16 &      \\ 
     &           &    4 &  1.69e-08 &  1.74e-15 &      \\ 
     &           &    5 &  1.68e-08 &  4.81e-16 &      \\ 
     &           &    6 &  1.68e-08 &  4.80e-16 &      \\ 
     &           &    7 &  1.68e-08 &  4.80e-16 &      \\ 
     &           &    8 &  6.09e-08 &  4.79e-16 &      \\ 
     &           &    9 &  1.68e-08 &  1.74e-15 &      \\ 
     &           &   10 &  1.68e-08 &  4.81e-16 &      \\ 
1269 &  1.27e+04 &   10 &           &           & iters  \\ 
 \hdashline 
     &           &    1 &  1.15e-08 &  1.00e-09 &      \\ 
     &           &    2 &  1.15e-08 &  3.28e-16 &      \\ 
     &           &    3 &  1.14e-08 &  3.27e-16 &      \\ 
     &           &    4 &  6.63e-08 &  3.27e-16 &      \\ 
     &           &    5 &  1.15e-08 &  1.89e-15 &      \\ 
     &           &    6 &  1.15e-08 &  3.29e-16 &      \\ 
     &           &    7 &  1.15e-08 &  3.28e-16 &      \\ 
     &           &    8 &  1.15e-08 &  3.28e-16 &      \\ 
     &           &    9 &  1.15e-08 &  3.27e-16 &      \\ 
     &           &   10 &  1.14e-08 &  3.27e-16 &      \\ 
1270 &  1.27e+04 &   10 &           &           & iters  \\ 
 \hdashline 
     &           &    1 &  3.87e-08 &  1.70e-09 &      \\ 
     &           &    2 &  3.86e-08 &  1.10e-15 &      \\ 
     &           &    3 &  3.91e-08 &  1.10e-15 &      \\ 
     &           &    4 &  3.86e-08 &  1.12e-15 &      \\ 
     &           &    5 &  3.91e-08 &  1.10e-15 &      \\ 
     &           &    6 &  3.86e-08 &  1.12e-15 &      \\ 
     &           &    7 &  3.91e-08 &  1.10e-15 &      \\ 
     &           &    8 &  3.86e-08 &  1.12e-15 &      \\ 
     &           &    9 &  3.91e-08 &  1.10e-15 &      \\ 
     &           &   10 &  3.86e-08 &  1.12e-15 &      \\ 
1271 &  1.27e+04 &   10 &           &           & iters  \\ 
 \hdashline 
     &           &    1 &  4.73e-09 &  7.07e-10 &      \\ 
     &           &    2 &  4.73e-09 &  1.35e-16 &      \\ 
     &           &    3 &  4.72e-09 &  1.35e-16 &      \\ 
     &           &    4 &  4.71e-09 &  1.35e-16 &      \\ 
     &           &    5 &  4.71e-09 &  1.35e-16 &      \\ 
     &           &    6 &  4.70e-09 &  1.34e-16 &      \\ 
     &           &    7 &  4.69e-09 &  1.34e-16 &      \\ 
     &           &    8 &  7.30e-08 &  1.34e-16 &      \\ 
     &           &    9 &  8.25e-08 &  2.08e-15 &      \\ 
     &           &   10 &  7.30e-08 &  2.35e-15 &      \\ 
1272 &  1.27e+04 &   10 &           &           & iters  \\ 
 \hdashline 
     &           &    1 &  2.30e-08 &  2.49e-10 &      \\ 
     &           &    2 &  2.30e-08 &  6.56e-16 &      \\ 
     &           &    3 &  5.48e-08 &  6.55e-16 &      \\ 
     &           &    4 &  2.30e-08 &  1.56e-15 &      \\ 
     &           &    5 &  2.30e-08 &  6.56e-16 &      \\ 
     &           &    6 &  2.29e-08 &  6.55e-16 &      \\ 
     &           &    7 &  5.48e-08 &  6.54e-16 &      \\ 
     &           &    8 &  2.30e-08 &  1.56e-15 &      \\ 
     &           &    9 &  2.29e-08 &  6.56e-16 &      \\ 
     &           &   10 &  5.48e-08 &  6.55e-16 &      \\ 
1273 &  1.27e+04 &   10 &           &           & iters  \\ 
 \hdashline 
     &           &    1 &  5.06e-08 &  9.90e-11 &      \\ 
     &           &    2 &  2.72e-08 &  1.44e-15 &      \\ 
     &           &    3 &  5.06e-08 &  7.75e-16 &      \\ 
     &           &    4 &  2.72e-08 &  1.44e-15 &      \\ 
     &           &    5 &  2.71e-08 &  7.76e-16 &      \\ 
     &           &    6 &  5.06e-08 &  7.75e-16 &      \\ 
     &           &    7 &  2.72e-08 &  1.44e-15 &      \\ 
     &           &    8 &  2.71e-08 &  7.76e-16 &      \\ 
     &           &    9 &  5.06e-08 &  7.75e-16 &      \\ 
     &           &   10 &  2.72e-08 &  1.44e-15 &      \\ 
1274 &  1.27e+04 &   10 &           &           & iters  \\ 
 \hdashline 
     &           &    1 &  2.81e-08 &  2.46e-09 &      \\ 
     &           &    2 &  2.81e-08 &  8.02e-16 &      \\ 
     &           &    3 &  4.97e-08 &  8.01e-16 &      \\ 
     &           &    4 &  2.81e-08 &  1.42e-15 &      \\ 
     &           &    5 &  4.96e-08 &  8.02e-16 &      \\ 
     &           &    6 &  2.81e-08 &  1.42e-15 &      \\ 
     &           &    7 &  2.81e-08 &  8.03e-16 &      \\ 
     &           &    8 &  4.96e-08 &  8.02e-16 &      \\ 
     &           &    9 &  2.81e-08 &  1.42e-15 &      \\ 
     &           &   10 &  2.81e-08 &  8.03e-16 &      \\ 
1275 &  1.27e+04 &   10 &           &           & iters  \\ 
 \hdashline 
     &           &    1 &  1.60e-08 &  7.63e-09 &      \\ 
     &           &    2 &  1.59e-08 &  4.56e-16 &      \\ 
     &           &    3 &  1.59e-08 &  4.55e-16 &      \\ 
     &           &    4 &  2.29e-08 &  4.55e-16 &      \\ 
     &           &    5 &  1.59e-08 &  6.55e-16 &      \\ 
     &           &    6 &  2.29e-08 &  4.55e-16 &      \\ 
     &           &    7 &  1.59e-08 &  6.54e-16 &      \\ 
     &           &    8 &  2.29e-08 &  4.55e-16 &      \\ 
     &           &    9 &  1.60e-08 &  6.54e-16 &      \\ 
     &           &   10 &  1.59e-08 &  4.55e-16 &      \\ 
1276 &  1.28e+04 &   10 &           &           & iters  \\ 
 \hdashline 
     &           &    1 &  3.41e-08 &  8.74e-09 &      \\ 
     &           &    2 &  4.36e-08 &  9.73e-16 &      \\ 
     &           &    3 &  3.41e-08 &  1.25e-15 &      \\ 
     &           &    4 &  4.36e-08 &  9.73e-16 &      \\ 
     &           &    5 &  3.41e-08 &  1.25e-15 &      \\ 
     &           &    6 &  4.36e-08 &  9.74e-16 &      \\ 
     &           &    7 &  3.41e-08 &  1.24e-15 &      \\ 
     &           &    8 &  3.41e-08 &  9.74e-16 &      \\ 
     &           &    9 &  4.37e-08 &  9.73e-16 &      \\ 
     &           &   10 &  3.41e-08 &  1.25e-15 &      \\ 
1277 &  1.28e+04 &   10 &           &           & iters  \\ 
 \hdashline 
     &           &    1 &  8.30e-09 &  2.88e-09 &      \\ 
     &           &    2 &  8.29e-09 &  2.37e-16 &      \\ 
     &           &    3 &  8.28e-09 &  2.37e-16 &      \\ 
     &           &    4 &  8.26e-09 &  2.36e-16 &      \\ 
     &           &    5 &  8.25e-09 &  2.36e-16 &      \\ 
     &           &    6 &  3.06e-08 &  2.35e-16 &      \\ 
     &           &    7 &  8.28e-09 &  8.73e-16 &      \\ 
     &           &    8 &  8.27e-09 &  2.36e-16 &      \\ 
     &           &    9 &  8.26e-09 &  2.36e-16 &      \\ 
     &           &   10 &  8.25e-09 &  2.36e-16 &      \\ 
1278 &  1.28e+04 &   10 &           &           & iters  \\ 
 \hdashline 
     &           &    1 &  6.67e-09 &  2.27e-09 &      \\ 
     &           &    2 &  6.66e-09 &  1.90e-16 &      \\ 
     &           &    3 &  6.65e-09 &  1.90e-16 &      \\ 
     &           &    4 &  6.64e-09 &  1.90e-16 &      \\ 
     &           &    5 &  6.63e-09 &  1.90e-16 &      \\ 
     &           &    6 &  6.62e-09 &  1.89e-16 &      \\ 
     &           &    7 &  6.61e-09 &  1.89e-16 &      \\ 
     &           &    8 &  6.60e-09 &  1.89e-16 &      \\ 
     &           &    9 &  6.59e-09 &  1.88e-16 &      \\ 
     &           &   10 &  7.11e-08 &  1.88e-16 &      \\ 
1279 &  1.28e+04 &   10 &           &           & iters  \\ 
 \hdashline 
     &           &    1 &  2.15e-08 &  1.76e-09 &      \\ 
     &           &    2 &  2.15e-08 &  6.14e-16 &      \\ 
     &           &    3 &  2.15e-08 &  6.13e-16 &      \\ 
     &           &    4 &  5.63e-08 &  6.12e-16 &      \\ 
     &           &    5 &  2.15e-08 &  1.61e-15 &      \\ 
     &           &    6 &  2.15e-08 &  6.14e-16 &      \\ 
     &           &    7 &  5.62e-08 &  6.13e-16 &      \\ 
     &           &    8 &  2.15e-08 &  1.61e-15 &      \\ 
     &           &    9 &  2.15e-08 &  6.14e-16 &      \\ 
     &           &   10 &  2.15e-08 &  6.14e-16 &      \\ 
1280 &  1.28e+04 &   10 &           &           & iters  \\ 
 \hdashline 
     &           &    1 &  6.44e-09 &  4.23e-10 &      \\ 
     &           &    2 &  6.43e-09 &  1.84e-16 &      \\ 
     &           &    3 &  6.42e-09 &  1.84e-16 &      \\ 
     &           &    4 &  7.13e-08 &  1.83e-16 &      \\ 
     &           &    5 &  4.54e-08 &  2.03e-15 &      \\ 
     &           &    6 &  6.45e-09 &  1.29e-15 &      \\ 
     &           &    7 &  6.44e-09 &  1.84e-16 &      \\ 
     &           &    8 &  6.43e-09 &  1.84e-16 &      \\ 
     &           &    9 &  6.42e-09 &  1.84e-16 &      \\ 
     &           &   10 &  7.13e-08 &  1.83e-16 &      \\ 
1281 &  1.28e+04 &   10 &           &           & iters  \\ 
 \hdashline 
     &           &    1 &  3.87e-08 &  2.93e-09 &      \\ 
     &           &    2 &  3.91e-08 &  1.10e-15 &      \\ 
     &           &    3 &  3.87e-08 &  1.12e-15 &      \\ 
     &           &    4 &  3.91e-08 &  1.10e-15 &      \\ 
     &           &    5 &  3.87e-08 &  1.12e-15 &      \\ 
     &           &    6 &  3.91e-08 &  1.10e-15 &      \\ 
     &           &    7 &  3.87e-08 &  1.12e-15 &      \\ 
     &           &    8 &  3.91e-08 &  1.10e-15 &      \\ 
     &           &    9 &  3.87e-08 &  1.12e-15 &      \\ 
     &           &   10 &  3.91e-08 &  1.10e-15 &      \\ 
1282 &  1.28e+04 &   10 &           &           & iters  \\ 
 \hdashline 
     &           &    1 &  2.60e-08 &  3.11e-09 &      \\ 
     &           &    2 &  1.29e-08 &  7.42e-16 &      \\ 
     &           &    3 &  2.60e-08 &  3.68e-16 &      \\ 
     &           &    4 &  1.29e-08 &  7.41e-16 &      \\ 
     &           &    5 &  1.29e-08 &  3.68e-16 &      \\ 
     &           &    6 &  2.60e-08 &  3.68e-16 &      \\ 
     &           &    7 &  1.29e-08 &  7.41e-16 &      \\ 
     &           &    8 &  1.29e-08 &  3.68e-16 &      \\ 
     &           &    9 &  2.60e-08 &  3.68e-16 &      \\ 
     &           &   10 &  1.29e-08 &  7.41e-16 &      \\ 
1283 &  1.28e+04 &   10 &           &           & iters  \\ 
 \hdashline 
     &           &    1 &  4.51e-09 &  2.20e-09 &      \\ 
     &           &    2 &  4.50e-09 &  1.29e-16 &      \\ 
     &           &    3 &  4.50e-09 &  1.29e-16 &      \\ 
     &           &    4 &  4.49e-09 &  1.28e-16 &      \\ 
     &           &    5 &  4.48e-09 &  1.28e-16 &      \\ 
     &           &    6 &  4.48e-09 &  1.28e-16 &      \\ 
     &           &    7 &  4.47e-09 &  1.28e-16 &      \\ 
     &           &    8 &  7.32e-08 &  1.28e-16 &      \\ 
     &           &    9 &  8.23e-08 &  2.09e-15 &      \\ 
     &           &   10 &  7.32e-08 &  2.35e-15 &      \\ 
1284 &  1.28e+04 &   10 &           &           & iters  \\ 
 \hdashline 
     &           &    1 &  3.79e-08 &  5.99e-09 &      \\ 
     &           &    2 &  3.98e-08 &  1.08e-15 &      \\ 
     &           &    3 &  3.79e-08 &  1.14e-15 &      \\ 
     &           &    4 &  3.98e-08 &  1.08e-15 &      \\ 
     &           &    5 &  3.80e-08 &  1.14e-15 &      \\ 
     &           &    6 &  3.98e-08 &  1.08e-15 &      \\ 
     &           &    7 &  3.80e-08 &  1.14e-15 &      \\ 
     &           &    8 &  3.98e-08 &  1.08e-15 &      \\ 
     &           &    9 &  3.80e-08 &  1.14e-15 &      \\ 
     &           &   10 &  3.98e-08 &  1.08e-15 &      \\ 
1285 &  1.28e+04 &   10 &           &           & iters  \\ 
 \hdashline 
     &           &    1 &  7.90e-08 &  2.67e-09 &      \\ 
     &           &    2 &  3.76e-08 &  2.26e-15 &      \\ 
     &           &    3 &  1.30e-09 &  1.07e-15 &      \\ 
     &           &    4 &  1.30e-09 &  3.71e-17 &      \\ 
     &           &    5 &  1.30e-09 &  3.71e-17 &      \\ 
     &           &    6 &  1.30e-09 &  3.70e-17 &      \\ 
     &           &    7 &  1.29e-09 &  3.70e-17 &      \\ 
     &           &    8 &  1.29e-09 &  3.69e-17 &      \\ 
     &           &    9 &  1.29e-09 &  3.69e-17 &      \\ 
     &           &   10 &  1.29e-09 &  3.68e-17 &      \\ 
1286 &  1.28e+04 &   10 &           &           & iters  \\ 
 \hdashline 
     &           &    1 &  1.21e-08 &  4.96e-11 &      \\ 
     &           &    2 &  1.20e-08 &  3.44e-16 &      \\ 
     &           &    3 &  1.20e-08 &  3.44e-16 &      \\ 
     &           &    4 &  1.20e-08 &  3.43e-16 &      \\ 
     &           &    5 &  6.57e-08 &  3.43e-16 &      \\ 
     &           &    6 &  1.21e-08 &  1.88e-15 &      \\ 
     &           &    7 &  1.21e-08 &  3.45e-16 &      \\ 
     &           &    8 &  1.20e-08 &  3.44e-16 &      \\ 
     &           &    9 &  1.20e-08 &  3.44e-16 &      \\ 
     &           &   10 &  1.20e-08 &  3.43e-16 &      \\ 
1287 &  1.29e+04 &   10 &           &           & iters  \\ 
 \hdashline 
     &           &    1 &  2.02e-08 &  5.53e-10 &      \\ 
     &           &    2 &  2.02e-08 &  5.76e-16 &      \\ 
     &           &    3 &  1.87e-08 &  5.75e-16 &      \\ 
     &           &    4 &  2.02e-08 &  5.34e-16 &      \\ 
     &           &    5 &  1.87e-08 &  5.75e-16 &      \\ 
     &           &    6 &  2.02e-08 &  5.34e-16 &      \\ 
     &           &    7 &  1.87e-08 &  5.75e-16 &      \\ 
     &           &    8 &  2.02e-08 &  5.34e-16 &      \\ 
     &           &    9 &  1.87e-08 &  5.75e-16 &      \\ 
     &           &   10 &  2.02e-08 &  5.34e-16 &      \\ 
1288 &  1.29e+04 &   10 &           &           & iters  \\ 
 \hdashline 
     &           &    1 &  6.68e-09 &  3.87e-10 &      \\ 
     &           &    2 &  6.67e-09 &  1.91e-16 &      \\ 
     &           &    3 &  6.66e-09 &  1.90e-16 &      \\ 
     &           &    4 &  3.22e-08 &  1.90e-16 &      \\ 
     &           &    5 &  6.70e-09 &  9.19e-16 &      \\ 
     &           &    6 &  6.69e-09 &  1.91e-16 &      \\ 
     &           &    7 &  6.68e-09 &  1.91e-16 &      \\ 
     &           &    8 &  6.67e-09 &  1.91e-16 &      \\ 
     &           &    9 &  6.66e-09 &  1.90e-16 &      \\ 
     &           &   10 &  3.22e-08 &  1.90e-16 &      \\ 
1289 &  1.29e+04 &   10 &           &           & iters  \\ 
 \hdashline 
     &           &    1 &  4.04e-08 &  3.95e-10 &      \\ 
     &           &    2 &  3.73e-08 &  1.15e-15 &      \\ 
     &           &    3 &  4.04e-08 &  1.07e-15 &      \\ 
     &           &    4 &  3.73e-08 &  1.15e-15 &      \\ 
     &           &    5 &  4.04e-08 &  1.07e-15 &      \\ 
     &           &    6 &  3.73e-08 &  1.15e-15 &      \\ 
     &           &    7 &  4.04e-08 &  1.07e-15 &      \\ 
     &           &    8 &  3.73e-08 &  1.15e-15 &      \\ 
     &           &    9 &  4.04e-08 &  1.07e-15 &      \\ 
     &           &   10 &  3.73e-08 &  1.15e-15 &      \\ 
1290 &  1.29e+04 &   10 &           &           & iters  \\ 
 \hdashline 
     &           &    1 &  1.32e-08 &  1.08e-09 &      \\ 
     &           &    2 &  1.31e-08 &  3.75e-16 &      \\ 
     &           &    3 &  6.46e-08 &  3.75e-16 &      \\ 
     &           &    4 &  9.09e-08 &  1.84e-15 &      \\ 
     &           &    5 &  6.46e-08 &  2.59e-15 &      \\ 
     &           &    6 &  1.32e-08 &  1.84e-15 &      \\ 
     &           &    7 &  1.32e-08 &  3.76e-16 &      \\ 
     &           &    8 &  1.31e-08 &  3.75e-16 &      \\ 
     &           &    9 &  6.46e-08 &  3.75e-16 &      \\ 
     &           &   10 &  1.32e-08 &  1.84e-15 &      \\ 
1291 &  1.29e+04 &   10 &           &           & iters  \\ 
 \hdashline 
     &           &    1 &  2.57e-08 &  8.64e-10 &      \\ 
     &           &    2 &  2.57e-08 &  7.33e-16 &      \\ 
     &           &    3 &  5.21e-08 &  7.32e-16 &      \\ 
     &           &    4 &  2.57e-08 &  1.49e-15 &      \\ 
     &           &    5 &  2.57e-08 &  7.33e-16 &      \\ 
     &           &    6 &  5.21e-08 &  7.32e-16 &      \\ 
     &           &    7 &  2.57e-08 &  1.49e-15 &      \\ 
     &           &    8 &  2.57e-08 &  7.33e-16 &      \\ 
     &           &    9 &  5.21e-08 &  7.32e-16 &      \\ 
     &           &   10 &  2.57e-08 &  1.49e-15 &      \\ 
1292 &  1.29e+04 &   10 &           &           & iters  \\ 
 \hdashline 
     &           &    1 &  1.99e-09 &  5.73e-10 &      \\ 
     &           &    2 &  1.99e-09 &  5.68e-17 &      \\ 
     &           &    3 &  1.99e-09 &  5.68e-17 &      \\ 
     &           &    4 &  1.98e-09 &  5.67e-17 &      \\ 
     &           &    5 &  1.98e-09 &  5.66e-17 &      \\ 
     &           &    6 &  1.98e-09 &  5.65e-17 &      \\ 
     &           &    7 &  1.97e-09 &  5.64e-17 &      \\ 
     &           &    8 &  1.97e-09 &  5.64e-17 &      \\ 
     &           &    9 &  1.97e-09 &  5.63e-17 &      \\ 
     &           &   10 &  1.97e-09 &  5.62e-17 &      \\ 
1293 &  1.29e+04 &   10 &           &           & iters  \\ 
 \hdashline 
     &           &    1 &  3.12e-08 &  5.11e-10 &      \\ 
     &           &    2 &  4.66e-08 &  8.89e-16 &      \\ 
     &           &    3 &  3.12e-08 &  1.33e-15 &      \\ 
     &           &    4 &  3.11e-08 &  8.90e-16 &      \\ 
     &           &    5 &  4.66e-08 &  8.89e-16 &      \\ 
     &           &    6 &  3.12e-08 &  1.33e-15 &      \\ 
     &           &    7 &  4.66e-08 &  8.89e-16 &      \\ 
     &           &    8 &  3.12e-08 &  1.33e-15 &      \\ 
     &           &    9 &  3.11e-08 &  8.90e-16 &      \\ 
     &           &   10 &  4.66e-08 &  8.89e-16 &      \\ 
1294 &  1.29e+04 &   10 &           &           & iters  \\ 
 \hdashline 
     &           &    1 &  5.27e-09 &  5.07e-10 &      \\ 
     &           &    2 &  7.24e-08 &  1.50e-16 &      \\ 
     &           &    3 &  8.31e-08 &  2.07e-15 &      \\ 
     &           &    4 &  7.24e-08 &  2.37e-15 &      \\ 
     &           &    5 &  5.35e-09 &  2.07e-15 &      \\ 
     &           &    6 &  5.34e-09 &  1.53e-16 &      \\ 
     &           &    7 &  5.33e-09 &  1.52e-16 &      \\ 
     &           &    8 &  5.33e-09 &  1.52e-16 &      \\ 
     &           &    9 &  5.32e-09 &  1.52e-16 &      \\ 
     &           &   10 &  5.31e-09 &  1.52e-16 &      \\ 
1295 &  1.29e+04 &   10 &           &           & iters  \\ 
 \hdashline 
     &           &    1 &  1.86e-08 &  5.97e-10 &      \\ 
     &           &    2 &  2.03e-08 &  5.30e-16 &      \\ 
     &           &    3 &  1.86e-08 &  5.80e-16 &      \\ 
     &           &    4 &  2.03e-08 &  5.30e-16 &      \\ 
     &           &    5 &  1.86e-08 &  5.80e-16 &      \\ 
     &           &    6 &  2.03e-08 &  5.30e-16 &      \\ 
     &           &    7 &  1.86e-08 &  5.79e-16 &      \\ 
     &           &    8 &  2.03e-08 &  5.30e-16 &      \\ 
     &           &    9 &  1.86e-08 &  5.79e-16 &      \\ 
     &           &   10 &  2.03e-08 &  5.30e-16 &      \\ 
1296 &  1.30e+04 &   10 &           &           & iters  \\ 
 \hdashline 
     &           &    1 &  2.31e-08 &  3.87e-10 &      \\ 
     &           &    2 &  1.58e-08 &  6.58e-16 &      \\ 
     &           &    3 &  1.58e-08 &  4.51e-16 &      \\ 
     &           &    4 &  2.31e-08 &  4.51e-16 &      \\ 
     &           &    5 &  1.58e-08 &  6.58e-16 &      \\ 
     &           &    6 &  2.31e-08 &  4.51e-16 &      \\ 
     &           &    7 &  1.58e-08 &  6.58e-16 &      \\ 
     &           &    8 &  1.58e-08 &  4.51e-16 &      \\ 
     &           &    9 &  2.31e-08 &  4.51e-16 &      \\ 
     &           &   10 &  1.58e-08 &  6.59e-16 &      \\ 
1297 &  1.30e+04 &   10 &           &           & iters  \\ 
 \hdashline 
     &           &    1 &  1.27e-08 &  9.06e-10 &      \\ 
     &           &    2 &  1.27e-08 &  3.63e-16 &      \\ 
     &           &    3 &  1.27e-08 &  3.62e-16 &      \\ 
     &           &    4 &  6.50e-08 &  3.62e-16 &      \\ 
     &           &    5 &  1.28e-08 &  1.86e-15 &      \\ 
     &           &    6 &  1.27e-08 &  3.64e-16 &      \\ 
     &           &    7 &  1.27e-08 &  3.63e-16 &      \\ 
     &           &    8 &  1.27e-08 &  3.63e-16 &      \\ 
     &           &    9 &  1.27e-08 &  3.62e-16 &      \\ 
     &           &   10 &  6.50e-08 &  3.62e-16 &      \\ 
1298 &  1.30e+04 &   10 &           &           & iters  \\ 
 \hdashline 
     &           &    1 &  3.13e-08 &  5.18e-10 &      \\ 
     &           &    2 &  3.12e-08 &  8.93e-16 &      \\ 
     &           &    3 &  7.66e-09 &  8.91e-16 &      \\ 
     &           &    4 &  7.65e-09 &  2.19e-16 &      \\ 
     &           &    5 &  7.64e-09 &  2.18e-16 &      \\ 
     &           &    6 &  7.62e-09 &  2.18e-16 &      \\ 
     &           &    7 &  3.12e-08 &  2.18e-16 &      \\ 
     &           &    8 &  7.66e-09 &  8.91e-16 &      \\ 
     &           &    9 &  7.65e-09 &  2.19e-16 &      \\ 
     &           &   10 &  7.64e-09 &  2.18e-16 &      \\ 
1299 &  1.30e+04 &   10 &           &           & iters  \\ 
 \hdashline 
     &           &    1 &  1.34e-09 &  1.17e-09 &      \\ 
     &           &    2 &  1.33e-09 &  3.81e-17 &      \\ 
     &           &    3 &  1.33e-09 &  3.81e-17 &      \\ 
     &           &    4 &  1.33e-09 &  3.80e-17 &      \\ 
     &           &    5 &  1.33e-09 &  3.80e-17 &      \\ 
     &           &    6 &  1.33e-09 &  3.79e-17 &      \\ 
     &           &    7 &  1.32e-09 &  3.78e-17 &      \\ 
     &           &    8 &  1.32e-09 &  3.78e-17 &      \\ 
     &           &    9 &  1.32e-09 &  3.77e-17 &      \\ 
     &           &   10 &  1.32e-09 &  3.77e-17 &      \\ 
1300 &  1.30e+04 &   10 &           &           & iters  \\ 
 \hdashline 
     &           &    1 &  4.27e-09 &  3.21e-09 &      \\ 
     &           &    2 &  4.26e-09 &  1.22e-16 &      \\ 
     &           &    3 &  4.26e-09 &  1.22e-16 &      \\ 
     &           &    4 &  4.25e-09 &  1.22e-16 &      \\ 
     &           &    5 &  4.25e-09 &  1.21e-16 &      \\ 
     &           &    6 &  4.24e-09 &  1.21e-16 &      \\ 
     &           &    7 &  4.23e-09 &  1.21e-16 &      \\ 
     &           &    8 &  4.23e-09 &  1.21e-16 &      \\ 
     &           &    9 &  3.46e-08 &  1.21e-16 &      \\ 
     &           &   10 &  4.27e-09 &  9.88e-16 &      \\ 
1301 &  1.30e+04 &   10 &           &           & iters  \\ 
 \hdashline 
     &           &    1 &  2.23e-08 &  5.14e-09 &      \\ 
     &           &    2 &  1.66e-08 &  6.35e-16 &      \\ 
     &           &    3 &  1.66e-08 &  4.74e-16 &      \\ 
     &           &    4 &  2.23e-08 &  4.73e-16 &      \\ 
     &           &    5 &  1.66e-08 &  6.36e-16 &      \\ 
     &           &    6 &  2.23e-08 &  4.74e-16 &      \\ 
     &           &    7 &  1.66e-08 &  6.36e-16 &      \\ 
     &           &    8 &  2.23e-08 &  4.74e-16 &      \\ 
     &           &    9 &  1.66e-08 &  6.35e-16 &      \\ 
     &           &   10 &  1.66e-08 &  4.74e-16 &      \\ 
1302 &  1.30e+04 &   10 &           &           & iters  \\ 
 \hdashline 
     &           &    1 &  2.08e-08 &  8.52e-10 &      \\ 
     &           &    2 &  2.08e-08 &  5.94e-16 &      \\ 
     &           &    3 &  1.81e-08 &  5.93e-16 &      \\ 
     &           &    4 &  2.08e-08 &  5.16e-16 &      \\ 
     &           &    5 &  1.81e-08 &  5.93e-16 &      \\ 
     &           &    6 &  1.81e-08 &  5.17e-16 &      \\ 
     &           &    7 &  2.08e-08 &  5.16e-16 &      \\ 
     &           &    8 &  1.81e-08 &  5.94e-16 &      \\ 
     &           &    9 &  2.08e-08 &  5.16e-16 &      \\ 
     &           &   10 &  1.81e-08 &  5.93e-16 &      \\ 
1303 &  1.30e+04 &   10 &           &           & iters  \\ 
 \hdashline 
     &           &    1 &  1.02e-08 &  7.93e-09 &      \\ 
     &           &    2 &  2.87e-08 &  2.91e-16 &      \\ 
     &           &    3 &  1.02e-08 &  8.19e-16 &      \\ 
     &           &    4 &  1.02e-08 &  2.91e-16 &      \\ 
     &           &    5 &  1.02e-08 &  2.91e-16 &      \\ 
     &           &    6 &  2.87e-08 &  2.90e-16 &      \\ 
     &           &    7 &  1.02e-08 &  8.19e-16 &      \\ 
     &           &    8 &  1.02e-08 &  2.91e-16 &      \\ 
     &           &    9 &  1.02e-08 &  2.91e-16 &      \\ 
     &           &   10 &  2.87e-08 &  2.90e-16 &      \\ 
1304 &  1.30e+04 &   10 &           &           & iters  \\ 
 \hdashline 
     &           &    1 &  2.75e-08 &  5.41e-09 &      \\ 
     &           &    2 &  5.02e-08 &  7.86e-16 &      \\ 
     &           &    3 &  2.76e-08 &  1.43e-15 &      \\ 
     &           &    4 &  2.75e-08 &  7.87e-16 &      \\ 
     &           &    5 &  5.02e-08 &  7.86e-16 &      \\ 
     &           &    6 &  2.76e-08 &  1.43e-15 &      \\ 
     &           &    7 &  2.75e-08 &  7.87e-16 &      \\ 
     &           &    8 &  5.02e-08 &  7.86e-16 &      \\ 
     &           &    9 &  2.76e-08 &  1.43e-15 &      \\ 
     &           &   10 &  2.75e-08 &  7.86e-16 &      \\ 
1305 &  1.30e+04 &   10 &           &           & iters  \\ 
 \hdashline 
     &           &    1 &  2.90e-08 &  1.14e-09 &      \\ 
     &           &    2 &  2.90e-08 &  8.28e-16 &      \\ 
     &           &    3 &  4.88e-08 &  8.26e-16 &      \\ 
     &           &    4 &  2.90e-08 &  1.39e-15 &      \\ 
     &           &    5 &  4.87e-08 &  8.27e-16 &      \\ 
     &           &    6 &  2.90e-08 &  1.39e-15 &      \\ 
     &           &    7 &  2.90e-08 &  8.28e-16 &      \\ 
     &           &    8 &  4.88e-08 &  8.27e-16 &      \\ 
     &           &    9 &  2.90e-08 &  1.39e-15 &      \\ 
     &           &   10 &  2.90e-08 &  8.28e-16 &      \\ 
1306 &  1.30e+04 &   10 &           &           & iters  \\ 
 \hdashline 
     &           &    1 &  9.66e-09 &  4.54e-09 &      \\ 
     &           &    2 &  9.65e-09 &  2.76e-16 &      \\ 
     &           &    3 &  9.64e-09 &  2.75e-16 &      \\ 
     &           &    4 &  6.81e-08 &  2.75e-16 &      \\ 
     &           &    5 &  8.74e-08 &  1.94e-15 &      \\ 
     &           &    6 &  6.81e-08 &  2.49e-15 &      \\ 
     &           &    7 &  9.69e-09 &  1.94e-15 &      \\ 
     &           &    8 &  9.68e-09 &  2.77e-16 &      \\ 
     &           &    9 &  9.66e-09 &  2.76e-16 &      \\ 
     &           &   10 &  9.65e-09 &  2.76e-16 &      \\ 
1307 &  1.31e+04 &   10 &           &           & iters  \\ 
 \hdashline 
     &           &    1 &  1.83e-08 &  1.61e-09 &      \\ 
     &           &    2 &  5.94e-08 &  5.23e-16 &      \\ 
     &           &    3 &  1.84e-08 &  1.69e-15 &      \\ 
     &           &    4 &  1.84e-08 &  5.25e-16 &      \\ 
     &           &    5 &  1.83e-08 &  5.24e-16 &      \\ 
     &           &    6 &  5.94e-08 &  5.24e-16 &      \\ 
     &           &    7 &  1.84e-08 &  1.69e-15 &      \\ 
     &           &    8 &  1.84e-08 &  5.25e-16 &      \\ 
     &           &    9 &  1.84e-08 &  5.24e-16 &      \\ 
     &           &   10 &  5.94e-08 &  5.24e-16 &      \\ 
1308 &  1.31e+04 &   10 &           &           & iters  \\ 
 \hdashline 
     &           &    1 &  4.14e-08 &  2.36e-09 &      \\ 
     &           &    2 &  4.13e-08 &  1.18e-15 &      \\ 
     &           &    3 &  3.65e-08 &  1.18e-15 &      \\ 
     &           &    4 &  4.13e-08 &  1.04e-15 &      \\ 
     &           &    5 &  3.65e-08 &  1.18e-15 &      \\ 
     &           &    6 &  4.13e-08 &  1.04e-15 &      \\ 
     &           &    7 &  3.65e-08 &  1.18e-15 &      \\ 
     &           &    8 &  4.13e-08 &  1.04e-15 &      \\ 
     &           &    9 &  3.65e-08 &  1.18e-15 &      \\ 
     &           &   10 &  3.64e-08 &  1.04e-15 &      \\ 
1309 &  1.31e+04 &   10 &           &           & iters  \\ 
 \hdashline 
     &           &    1 &  3.20e-09 &  5.38e-10 &      \\ 
     &           &    2 &  3.20e-09 &  9.13e-17 &      \\ 
     &           &    3 &  3.19e-09 &  9.12e-17 &      \\ 
     &           &    4 &  3.19e-09 &  9.11e-17 &      \\ 
     &           &    5 &  3.18e-09 &  9.09e-17 &      \\ 
     &           &    6 &  3.18e-09 &  9.08e-17 &      \\ 
     &           &    7 &  3.17e-09 &  9.07e-17 &      \\ 
     &           &    8 &  3.17e-09 &  9.05e-17 &      \\ 
     &           &    9 &  3.16e-09 &  9.04e-17 &      \\ 
     &           &   10 &  3.16e-09 &  9.03e-17 &      \\ 
1310 &  1.31e+04 &   10 &           &           & iters  \\ 
 \hdashline 
     &           &    1 &  6.75e-08 &  1.45e-10 &      \\ 
     &           &    2 &  1.03e-08 &  1.93e-15 &      \\ 
     &           &    3 &  1.03e-08 &  2.93e-16 &      \\ 
     &           &    4 &  1.02e-08 &  2.93e-16 &      \\ 
     &           &    5 &  1.02e-08 &  2.92e-16 &      \\ 
     &           &    6 &  1.02e-08 &  2.92e-16 &      \\ 
     &           &    7 &  1.02e-08 &  2.91e-16 &      \\ 
     &           &    8 &  6.75e-08 &  2.91e-16 &      \\ 
     &           &    9 &  1.03e-08 &  1.93e-15 &      \\ 
     &           &   10 &  1.03e-08 &  2.93e-16 &      \\ 
1311 &  1.31e+04 &   10 &           &           & iters  \\ 
 \hdashline 
     &           &    1 &  5.25e-08 &  2.68e-09 &      \\ 
     &           &    2 &  2.53e-08 &  1.50e-15 &      \\ 
     &           &    3 &  5.24e-08 &  7.22e-16 &      \\ 
     &           &    4 &  2.53e-08 &  1.50e-15 &      \\ 
     &           &    5 &  2.53e-08 &  7.23e-16 &      \\ 
     &           &    6 &  5.24e-08 &  7.22e-16 &      \\ 
     &           &    7 &  2.53e-08 &  1.50e-15 &      \\ 
     &           &    8 &  2.53e-08 &  7.23e-16 &      \\ 
     &           &    9 &  5.24e-08 &  7.22e-16 &      \\ 
     &           &   10 &  2.54e-08 &  1.50e-15 &      \\ 
1312 &  1.31e+04 &   10 &           &           & iters  \\ 
 \hdashline 
     &           &    1 &  3.80e-08 &  2.77e-09 &      \\ 
     &           &    2 &  3.97e-08 &  1.09e-15 &      \\ 
     &           &    3 &  3.80e-08 &  1.13e-15 &      \\ 
     &           &    4 &  3.97e-08 &  1.09e-15 &      \\ 
     &           &    5 &  3.80e-08 &  1.13e-15 &      \\ 
     &           &    6 &  3.97e-08 &  1.09e-15 &      \\ 
     &           &    7 &  3.80e-08 &  1.13e-15 &      \\ 
     &           &    8 &  3.97e-08 &  1.09e-15 &      \\ 
     &           &    9 &  3.80e-08 &  1.13e-15 &      \\ 
     &           &   10 &  3.97e-08 &  1.09e-15 &      \\ 
1313 &  1.31e+04 &   10 &           &           & iters  \\ 
 \hdashline 
     &           &    1 &  4.24e-08 &  2.50e-09 &      \\ 
     &           &    2 &  3.53e-08 &  1.21e-15 &      \\ 
     &           &    3 &  4.24e-08 &  1.01e-15 &      \\ 
     &           &    4 &  3.53e-08 &  1.21e-15 &      \\ 
     &           &    5 &  4.24e-08 &  1.01e-15 &      \\ 
     &           &    6 &  3.53e-08 &  1.21e-15 &      \\ 
     &           &    7 &  4.24e-08 &  1.01e-15 &      \\ 
     &           &    8 &  3.53e-08 &  1.21e-15 &      \\ 
     &           &    9 &  4.24e-08 &  1.01e-15 &      \\ 
     &           &   10 &  3.54e-08 &  1.21e-15 &      \\ 
1314 &  1.31e+04 &   10 &           &           & iters  \\ 
 \hdashline 
     &           &    1 &  2.51e-08 &  8.26e-10 &      \\ 
     &           &    2 &  2.51e-08 &  7.17e-16 &      \\ 
     &           &    3 &  5.26e-08 &  7.16e-16 &      \\ 
     &           &    4 &  2.51e-08 &  1.50e-15 &      \\ 
     &           &    5 &  2.51e-08 &  7.17e-16 &      \\ 
     &           &    6 &  5.26e-08 &  7.16e-16 &      \\ 
     &           &    7 &  2.51e-08 &  1.50e-15 &      \\ 
     &           &    8 &  2.51e-08 &  7.17e-16 &      \\ 
     &           &    9 &  5.26e-08 &  7.16e-16 &      \\ 
     &           &   10 &  2.51e-08 &  1.50e-15 &      \\ 
1315 &  1.31e+04 &   10 &           &           & iters  \\ 
 \hdashline 
     &           &    1 &  1.60e-08 &  1.67e-10 &      \\ 
     &           &    2 &  1.60e-08 &  4.58e-16 &      \\ 
     &           &    3 &  1.60e-08 &  4.57e-16 &      \\ 
     &           &    4 &  6.17e-08 &  4.57e-16 &      \\ 
     &           &    5 &  1.61e-08 &  1.76e-15 &      \\ 
     &           &    6 &  1.60e-08 &  4.59e-16 &      \\ 
     &           &    7 &  1.60e-08 &  4.58e-16 &      \\ 
     &           &    8 &  6.17e-08 &  4.57e-16 &      \\ 
     &           &    9 &  1.61e-08 &  1.76e-15 &      \\ 
     &           &   10 &  1.61e-08 &  4.59e-16 &      \\ 
1316 &  1.32e+04 &   10 &           &           & iters  \\ 
 \hdashline 
     &           &    1 &  9.33e-09 &  2.31e-09 &      \\ 
     &           &    2 &  2.95e-08 &  2.66e-16 &      \\ 
     &           &    3 &  9.36e-09 &  8.43e-16 &      \\ 
     &           &    4 &  9.34e-09 &  2.67e-16 &      \\ 
     &           &    5 &  9.33e-09 &  2.67e-16 &      \\ 
     &           &    6 &  2.95e-08 &  2.66e-16 &      \\ 
     &           &    7 &  9.36e-09 &  8.43e-16 &      \\ 
     &           &    8 &  9.34e-09 &  2.67e-16 &      \\ 
     &           &    9 &  9.33e-09 &  2.67e-16 &      \\ 
     &           &   10 &  2.95e-08 &  2.66e-16 &      \\ 
1317 &  1.32e+04 &   10 &           &           & iters  \\ 
 \hdashline 
     &           &    1 &  4.62e-08 &  4.16e-09 &      \\ 
     &           &    2 &  3.15e-08 &  1.32e-15 &      \\ 
     &           &    3 &  4.62e-08 &  9.00e-16 &      \\ 
     &           &    4 &  3.15e-08 &  1.32e-15 &      \\ 
     &           &    5 &  3.15e-08 &  9.00e-16 &      \\ 
     &           &    6 &  4.62e-08 &  8.99e-16 &      \\ 
     &           &    7 &  3.15e-08 &  1.32e-15 &      \\ 
     &           &    8 &  4.62e-08 &  9.00e-16 &      \\ 
     &           &    9 &  3.15e-08 &  1.32e-15 &      \\ 
     &           &   10 &  3.15e-08 &  9.00e-16 &      \\ 
1318 &  1.32e+04 &   10 &           &           & iters  \\ 
 \hdashline 
     &           &    1 &  1.49e-08 &  5.58e-09 &      \\ 
     &           &    2 &  1.49e-08 &  4.25e-16 &      \\ 
     &           &    3 &  1.48e-08 &  4.24e-16 &      \\ 
     &           &    4 &  1.48e-08 &  4.23e-16 &      \\ 
     &           &    5 &  6.29e-08 &  4.23e-16 &      \\ 
     &           &    6 &  1.49e-08 &  1.79e-15 &      \\ 
     &           &    7 &  1.49e-08 &  4.25e-16 &      \\ 
     &           &    8 &  1.48e-08 &  4.24e-16 &      \\ 
     &           &    9 &  1.48e-08 &  4.24e-16 &      \\ 
     &           &   10 &  1.48e-08 &  4.23e-16 &      \\ 
1319 &  1.32e+04 &   10 &           &           & iters  \\ 
 \hdashline 
     &           &    1 &  4.22e-08 &  3.74e-09 &      \\ 
     &           &    2 &  3.28e-09 &  1.20e-15 &      \\ 
     &           &    3 &  3.28e-09 &  9.37e-17 &      \\ 
     &           &    4 &  3.27e-09 &  9.36e-17 &      \\ 
     &           &    5 &  3.27e-09 &  9.34e-17 &      \\ 
     &           &    6 &  3.26e-09 &  9.33e-17 &      \\ 
     &           &    7 &  3.26e-09 &  9.32e-17 &      \\ 
     &           &    8 &  3.26e-09 &  9.30e-17 &      \\ 
     &           &    9 &  3.25e-09 &  9.29e-17 &      \\ 
     &           &   10 &  3.25e-09 &  9.28e-17 &      \\ 
1320 &  1.32e+04 &   10 &           &           & iters  \\ 
 \hdashline 
     &           &    1 &  3.44e-08 &  4.78e-10 &      \\ 
     &           &    2 &  4.47e-09 &  9.82e-16 &      \\ 
     &           &    3 &  4.46e-09 &  1.28e-16 &      \\ 
     &           &    4 &  4.46e-09 &  1.27e-16 &      \\ 
     &           &    5 &  4.45e-09 &  1.27e-16 &      \\ 
     &           &    6 &  4.45e-09 &  1.27e-16 &      \\ 
     &           &    7 &  4.44e-09 &  1.27e-16 &      \\ 
     &           &    8 &  3.44e-08 &  1.27e-16 &      \\ 
     &           &    9 &  4.48e-09 &  9.82e-16 &      \\ 
     &           &   10 &  4.48e-09 &  1.28e-16 &      \\ 
1321 &  1.32e+04 &   10 &           &           & iters  \\ 
 \hdashline 
     &           &    1 &  1.06e-08 &  2.21e-09 &      \\ 
     &           &    2 &  1.06e-08 &  3.02e-16 &      \\ 
     &           &    3 &  1.05e-08 &  3.01e-16 &      \\ 
     &           &    4 &  1.05e-08 &  3.01e-16 &      \\ 
     &           &    5 &  1.05e-08 &  3.01e-16 &      \\ 
     &           &    6 &  6.72e-08 &  3.00e-16 &      \\ 
     &           &    7 &  1.06e-08 &  1.92e-15 &      \\ 
     &           &    8 &  1.06e-08 &  3.02e-16 &      \\ 
     &           &    9 &  1.06e-08 &  3.02e-16 &      \\ 
     &           &   10 &  1.06e-08 &  3.02e-16 &      \\ 
1322 &  1.32e+04 &   10 &           &           & iters  \\ 
 \hdashline 
     &           &    1 &  3.84e-08 &  2.82e-09 &      \\ 
     &           &    2 &  3.93e-08 &  1.10e-15 &      \\ 
     &           &    3 &  3.84e-08 &  1.12e-15 &      \\ 
     &           &    4 &  3.93e-08 &  1.10e-15 &      \\ 
     &           &    5 &  3.84e-08 &  1.12e-15 &      \\ 
     &           &    6 &  3.93e-08 &  1.10e-15 &      \\ 
     &           &    7 &  3.84e-08 &  1.12e-15 &      \\ 
     &           &    8 &  3.93e-08 &  1.10e-15 &      \\ 
     &           &    9 &  3.84e-08 &  1.12e-15 &      \\ 
     &           &   10 &  3.93e-08 &  1.10e-15 &      \\ 
1323 &  1.32e+04 &   10 &           &           & iters  \\ 
 \hdashline 
     &           &    1 &  5.67e-08 &  2.76e-09 &      \\ 
     &           &    2 &  2.10e-08 &  1.62e-15 &      \\ 
     &           &    3 &  2.10e-08 &  6.00e-16 &      \\ 
     &           &    4 &  2.10e-08 &  6.00e-16 &      \\ 
     &           &    5 &  5.67e-08 &  5.99e-16 &      \\ 
     &           &    6 &  2.10e-08 &  1.62e-15 &      \\ 
     &           &    7 &  2.10e-08 &  6.00e-16 &      \\ 
     &           &    8 &  2.10e-08 &  5.99e-16 &      \\ 
     &           &    9 &  5.67e-08 &  5.98e-16 &      \\ 
     &           &   10 &  2.10e-08 &  1.62e-15 &      \\ 
1324 &  1.32e+04 &   10 &           &           & iters  \\ 
 \hdashline 
     &           &    1 &  1.39e-08 &  7.42e-10 &      \\ 
     &           &    2 &  2.49e-08 &  3.97e-16 &      \\ 
     &           &    3 &  1.39e-08 &  7.12e-16 &      \\ 
     &           &    4 &  1.39e-08 &  3.98e-16 &      \\ 
     &           &    5 &  2.49e-08 &  3.97e-16 &      \\ 
     &           &    6 &  1.39e-08 &  7.12e-16 &      \\ 
     &           &    7 &  2.49e-08 &  3.98e-16 &      \\ 
     &           &    8 &  1.39e-08 &  7.12e-16 &      \\ 
     &           &    9 &  1.39e-08 &  3.98e-16 &      \\ 
     &           &   10 &  2.49e-08 &  3.97e-16 &      \\ 
1325 &  1.32e+04 &   10 &           &           & iters  \\ 
 \hdashline 
     &           &    1 &  3.70e-08 &  6.79e-10 &      \\ 
     &           &    2 &  4.08e-08 &  1.06e-15 &      \\ 
     &           &    3 &  3.70e-08 &  1.16e-15 &      \\ 
     &           &    4 &  4.08e-08 &  1.06e-15 &      \\ 
     &           &    5 &  3.70e-08 &  1.16e-15 &      \\ 
     &           &    6 &  4.08e-08 &  1.06e-15 &      \\ 
     &           &    7 &  3.70e-08 &  1.16e-15 &      \\ 
     &           &    8 &  4.07e-08 &  1.06e-15 &      \\ 
     &           &    9 &  3.70e-08 &  1.16e-15 &      \\ 
     &           &   10 &  4.07e-08 &  1.06e-15 &      \\ 
1326 &  1.32e+04 &   10 &           &           & iters  \\ 
 \hdashline 
     &           &    1 &  4.26e-08 &  4.53e-10 &      \\ 
     &           &    2 &  3.51e-08 &  1.22e-15 &      \\ 
     &           &    3 &  4.26e-08 &  1.00e-15 &      \\ 
     &           &    4 &  3.52e-08 &  1.22e-15 &      \\ 
     &           &    5 &  4.26e-08 &  1.00e-15 &      \\ 
     &           &    6 &  3.52e-08 &  1.22e-15 &      \\ 
     &           &    7 &  4.26e-08 &  1.00e-15 &      \\ 
     &           &    8 &  3.52e-08 &  1.22e-15 &      \\ 
     &           &    9 &  4.26e-08 &  1.00e-15 &      \\ 
     &           &   10 &  3.52e-08 &  1.21e-15 &      \\ 
1327 &  1.33e+04 &   10 &           &           & iters  \\ 
 \hdashline 
     &           &    1 &  7.10e-09 &  1.67e-09 &      \\ 
     &           &    2 &  7.09e-09 &  2.03e-16 &      \\ 
     &           &    3 &  7.07e-09 &  2.02e-16 &      \\ 
     &           &    4 &  7.06e-09 &  2.02e-16 &      \\ 
     &           &    5 &  7.05e-09 &  2.02e-16 &      \\ 
     &           &    6 &  7.04e-09 &  2.01e-16 &      \\ 
     &           &    7 &  7.07e-08 &  2.01e-16 &      \\ 
     &           &    8 &  4.60e-08 &  2.02e-15 &      \\ 
     &           &    9 &  7.07e-09 &  1.31e-15 &      \\ 
     &           &   10 &  7.06e-09 &  2.02e-16 &      \\ 
1328 &  1.33e+04 &   10 &           &           & iters  \\ 
 \hdashline 
     &           &    1 &  9.34e-09 &  1.24e-09 &      \\ 
     &           &    2 &  9.33e-09 &  2.67e-16 &      \\ 
     &           &    3 &  9.31e-09 &  2.66e-16 &      \\ 
     &           &    4 &  9.30e-09 &  2.66e-16 &      \\ 
     &           &    5 &  9.29e-09 &  2.65e-16 &      \\ 
     &           &    6 &  9.27e-09 &  2.65e-16 &      \\ 
     &           &    7 &  9.26e-09 &  2.65e-16 &      \\ 
     &           &    8 &  6.84e-08 &  2.64e-16 &      \\ 
     &           &    9 &  4.82e-08 &  1.95e-15 &      \\ 
     &           &   10 &  9.27e-09 &  1.38e-15 &      \\ 
1329 &  1.33e+04 &   10 &           &           & iters  \\ 
 \hdashline 
     &           &    1 &  1.28e-08 &  1.17e-09 &      \\ 
     &           &    2 &  6.49e-08 &  3.66e-16 &      \\ 
     &           &    3 &  1.29e-08 &  1.85e-15 &      \\ 
     &           &    4 &  1.29e-08 &  3.68e-16 &      \\ 
     &           &    5 &  1.29e-08 &  3.67e-16 &      \\ 
     &           &    6 &  1.28e-08 &  3.67e-16 &      \\ 
     &           &    7 &  1.28e-08 &  3.66e-16 &      \\ 
     &           &    8 &  6.49e-08 &  3.66e-16 &      \\ 
     &           &    9 &  1.29e-08 &  1.85e-15 &      \\ 
     &           &   10 &  1.29e-08 &  3.68e-16 &      \\ 
1330 &  1.33e+04 &   10 &           &           & iters  \\ 
 \hdashline 
     &           &    1 &  4.62e-09 &  1.37e-09 &      \\ 
     &           &    2 &  4.61e-09 &  1.32e-16 &      \\ 
     &           &    3 &  3.42e-08 &  1.32e-16 &      \\ 
     &           &    4 &  4.66e-09 &  9.77e-16 &      \\ 
     &           &    5 &  4.65e-09 &  1.33e-16 &      \\ 
     &           &    6 &  4.64e-09 &  1.33e-16 &      \\ 
     &           &    7 &  4.64e-09 &  1.32e-16 &      \\ 
     &           &    8 &  4.63e-09 &  1.32e-16 &      \\ 
     &           &    9 &  4.62e-09 &  1.32e-16 &      \\ 
     &           &   10 &  4.62e-09 &  1.32e-16 &      \\ 
1331 &  1.33e+04 &   10 &           &           & iters  \\ 
 \hdashline 
     &           &    1 &  2.30e-08 &  2.39e-11 &      \\ 
     &           &    2 &  5.47e-08 &  6.56e-16 &      \\ 
     &           &    3 &  2.30e-08 &  1.56e-15 &      \\ 
     &           &    4 &  2.30e-08 &  6.57e-16 &      \\ 
     &           &    5 &  2.30e-08 &  6.56e-16 &      \\ 
     &           &    6 &  5.48e-08 &  6.55e-16 &      \\ 
     &           &    7 &  2.30e-08 &  1.56e-15 &      \\ 
     &           &    8 &  2.30e-08 &  6.57e-16 &      \\ 
     &           &    9 &  5.47e-08 &  6.56e-16 &      \\ 
     &           &   10 &  2.30e-08 &  1.56e-15 &      \\ 
1332 &  1.33e+04 &   10 &           &           & iters  \\ 
 \hdashline 
     &           &    1 &  2.26e-08 &  1.17e-09 &      \\ 
     &           &    2 &  2.26e-08 &  6.46e-16 &      \\ 
     &           &    3 &  5.51e-08 &  6.45e-16 &      \\ 
     &           &    4 &  2.26e-08 &  1.57e-15 &      \\ 
     &           &    5 &  2.26e-08 &  6.46e-16 &      \\ 
     &           &    6 &  5.51e-08 &  6.45e-16 &      \\ 
     &           &    7 &  2.27e-08 &  1.57e-15 &      \\ 
     &           &    8 &  2.26e-08 &  6.47e-16 &      \\ 
     &           &    9 &  5.51e-08 &  6.46e-16 &      \\ 
     &           &   10 &  2.27e-08 &  1.57e-15 &      \\ 
1333 &  1.33e+04 &   10 &           &           & iters  \\ 
 \hdashline 
     &           &    1 &  3.53e-08 &  4.72e-10 &      \\ 
     &           &    2 &  3.57e-09 &  1.01e-15 &      \\ 
     &           &    3 &  3.56e-09 &  1.02e-16 &      \\ 
     &           &    4 &  3.56e-09 &  1.02e-16 &      \\ 
     &           &    5 &  3.55e-09 &  1.02e-16 &      \\ 
     &           &    6 &  3.55e-09 &  1.01e-16 &      \\ 
     &           &    7 &  3.54e-09 &  1.01e-16 &      \\ 
     &           &    8 &  3.54e-09 &  1.01e-16 &      \\ 
     &           &    9 &  3.53e-09 &  1.01e-16 &      \\ 
     &           &   10 &  3.53e-09 &  1.01e-16 &      \\ 
1334 &  1.33e+04 &   10 &           &           & iters  \\ 
 \hdashline 
     &           &    1 &  4.23e-09 &  2.38e-09 &      \\ 
     &           &    2 &  4.23e-09 &  1.21e-16 &      \\ 
     &           &    3 &  4.22e-09 &  1.21e-16 &      \\ 
     &           &    4 &  4.21e-09 &  1.20e-16 &      \\ 
     &           &    5 &  4.21e-09 &  1.20e-16 &      \\ 
     &           &    6 &  4.20e-09 &  1.20e-16 &      \\ 
     &           &    7 &  7.35e-08 &  1.20e-16 &      \\ 
     &           &    8 &  4.31e-08 &  2.10e-15 &      \\ 
     &           &    9 &  4.24e-09 &  1.23e-15 &      \\ 
     &           &   10 &  4.23e-09 &  1.21e-16 &      \\ 
1335 &  1.33e+04 &   10 &           &           & iters  \\ 
 \hdashline 
     &           &    1 &  6.75e-08 &  1.96e-10 &      \\ 
     &           &    2 &  8.80e-08 &  1.93e-15 &      \\ 
     &           &    3 &  6.75e-08 &  2.51e-15 &      \\ 
     &           &    4 &  1.03e-08 &  1.93e-15 &      \\ 
     &           &    5 &  1.03e-08 &  2.94e-16 &      \\ 
     &           &    6 &  1.03e-08 &  2.93e-16 &      \\ 
     &           &    7 &  1.03e-08 &  2.93e-16 &      \\ 
     &           &    8 &  1.02e-08 &  2.93e-16 &      \\ 
     &           &    9 &  6.75e-08 &  2.92e-16 &      \\ 
     &           &   10 &  1.03e-08 &  1.93e-15 &      \\ 
1336 &  1.34e+04 &   10 &           &           & iters  \\ 
 \hdashline 
     &           &    1 &  7.18e-09 &  1.30e-09 &      \\ 
     &           &    2 &  7.17e-09 &  2.05e-16 &      \\ 
     &           &    3 &  7.05e-08 &  2.05e-16 &      \\ 
     &           &    4 &  4.61e-08 &  2.01e-15 &      \\ 
     &           &    5 &  7.20e-09 &  1.32e-15 &      \\ 
     &           &    6 &  7.19e-09 &  2.05e-16 &      \\ 
     &           &    7 &  7.18e-09 &  2.05e-16 &      \\ 
     &           &    8 &  7.05e-08 &  2.05e-16 &      \\ 
     &           &    9 &  4.61e-08 &  2.01e-15 &      \\ 
     &           &   10 &  7.20e-09 &  1.32e-15 &      \\ 
1337 &  1.34e+04 &   10 &           &           & iters  \\ 
 \hdashline 
     &           &    1 &  3.32e-08 &  1.00e-09 &      \\ 
     &           &    2 &  5.68e-09 &  9.48e-16 &      \\ 
     &           &    3 &  5.67e-09 &  1.62e-16 &      \\ 
     &           &    4 &  5.66e-09 &  1.62e-16 &      \\ 
     &           &    5 &  5.65e-09 &  1.62e-16 &      \\ 
     &           &    6 &  3.32e-08 &  1.61e-16 &      \\ 
     &           &    7 &  5.69e-09 &  9.48e-16 &      \\ 
     &           &    8 &  5.69e-09 &  1.62e-16 &      \\ 
     &           &    9 &  5.68e-09 &  1.62e-16 &      \\ 
     &           &   10 &  5.67e-09 &  1.62e-16 &      \\ 
1338 &  1.34e+04 &   10 &           &           & iters  \\ 
 \hdashline 
     &           &    1 &  1.77e-08 &  3.69e-10 &      \\ 
     &           &    2 &  6.00e-08 &  5.07e-16 &      \\ 
     &           &    3 &  1.78e-08 &  1.71e-15 &      \\ 
     &           &    4 &  1.78e-08 &  5.08e-16 &      \\ 
     &           &    5 &  1.78e-08 &  5.08e-16 &      \\ 
     &           &    6 &  1.77e-08 &  5.07e-16 &      \\ 
     &           &    7 &  6.00e-08 &  5.06e-16 &      \\ 
     &           &    8 &  1.78e-08 &  1.71e-15 &      \\ 
     &           &    9 &  1.78e-08 &  5.08e-16 &      \\ 
     &           &   10 &  1.77e-08 &  5.07e-16 &      \\ 
1339 &  1.34e+04 &   10 &           &           & iters  \\ 
 \hdashline 
     &           &    1 &  9.28e-09 &  1.83e-09 &      \\ 
     &           &    2 &  9.27e-09 &  2.65e-16 &      \\ 
     &           &    3 &  9.26e-09 &  2.65e-16 &      \\ 
     &           &    4 &  9.24e-09 &  2.64e-16 &      \\ 
     &           &    5 &  9.23e-09 &  2.64e-16 &      \\ 
     &           &    6 &  9.22e-09 &  2.63e-16 &      \\ 
     &           &    7 &  6.85e-08 &  2.63e-16 &      \\ 
     &           &    8 &  9.30e-09 &  1.95e-15 &      \\ 
     &           &    9 &  9.29e-09 &  2.65e-16 &      \\ 
     &           &   10 &  9.27e-09 &  2.65e-16 &      \\ 
1340 &  1.34e+04 &   10 &           &           & iters  \\ 
 \hdashline 
     &           &    1 &  9.93e-09 &  1.43e-09 &      \\ 
     &           &    2 &  2.89e-08 &  2.83e-16 &      \\ 
     &           &    3 &  9.96e-09 &  8.26e-16 &      \\ 
     &           &    4 &  9.94e-09 &  2.84e-16 &      \\ 
     &           &    5 &  9.93e-09 &  2.84e-16 &      \\ 
     &           &    6 &  2.89e-08 &  2.83e-16 &      \\ 
     &           &    7 &  9.96e-09 &  8.26e-16 &      \\ 
     &           &    8 &  9.94e-09 &  2.84e-16 &      \\ 
     &           &    9 &  9.93e-09 &  2.84e-16 &      \\ 
     &           &   10 &  2.89e-08 &  2.83e-16 &      \\ 
1341 &  1.34e+04 &   10 &           &           & iters  \\ 
 \hdashline 
     &           &    1 &  4.45e-08 &  1.93e-09 &      \\ 
     &           &    2 &  3.33e-08 &  1.27e-15 &      \\ 
     &           &    3 &  4.44e-08 &  9.50e-16 &      \\ 
     &           &    4 &  3.33e-08 &  1.27e-15 &      \\ 
     &           &    5 &  4.44e-08 &  9.51e-16 &      \\ 
     &           &    6 &  3.33e-08 &  1.27e-15 &      \\ 
     &           &    7 &  4.44e-08 &  9.51e-16 &      \\ 
     &           &    8 &  3.33e-08 &  1.27e-15 &      \\ 
     &           &    9 &  3.33e-08 &  9.52e-16 &      \\ 
     &           &   10 &  4.44e-08 &  9.50e-16 &      \\ 
1342 &  1.34e+04 &   10 &           &           & iters  \\ 
 \hdashline 
     &           &    1 &  1.17e-08 &  7.37e-10 &      \\ 
     &           &    2 &  1.17e-08 &  3.34e-16 &      \\ 
     &           &    3 &  1.17e-08 &  3.33e-16 &      \\ 
     &           &    4 &  1.16e-08 &  3.33e-16 &      \\ 
     &           &    5 &  1.16e-08 &  3.32e-16 &      \\ 
     &           &    6 &  2.72e-08 &  3.32e-16 &      \\ 
     &           &    7 &  1.16e-08 &  7.77e-16 &      \\ 
     &           &    8 &  1.16e-08 &  3.32e-16 &      \\ 
     &           &    9 &  2.72e-08 &  3.32e-16 &      \\ 
     &           &   10 &  1.17e-08 &  7.77e-16 &      \\ 
1343 &  1.34e+04 &   10 &           &           & iters  \\ 
 \hdashline 
     &           &    1 &  4.10e-08 &  1.30e-09 &      \\ 
     &           &    2 &  3.67e-08 &  1.17e-15 &      \\ 
     &           &    3 &  3.67e-08 &  1.05e-15 &      \\ 
     &           &    4 &  4.11e-08 &  1.05e-15 &      \\ 
     &           &    5 &  3.67e-08 &  1.17e-15 &      \\ 
     &           &    6 &  4.11e-08 &  1.05e-15 &      \\ 
     &           &    7 &  3.67e-08 &  1.17e-15 &      \\ 
     &           &    8 &  4.11e-08 &  1.05e-15 &      \\ 
     &           &    9 &  3.67e-08 &  1.17e-15 &      \\ 
     &           &   10 &  4.11e-08 &  1.05e-15 &      \\ 
1344 &  1.34e+04 &   10 &           &           & iters  \\ 
 \hdashline 
     &           &    1 &  2.38e-08 &  5.68e-10 &      \\ 
     &           &    2 &  2.37e-08 &  6.79e-16 &      \\ 
     &           &    3 &  5.40e-08 &  6.78e-16 &      \\ 
     &           &    4 &  2.38e-08 &  1.54e-15 &      \\ 
     &           &    5 &  2.37e-08 &  6.79e-16 &      \\ 
     &           &    6 &  2.37e-08 &  6.78e-16 &      \\ 
     &           &    7 &  5.40e-08 &  6.77e-16 &      \\ 
     &           &    8 &  2.38e-08 &  1.54e-15 &      \\ 
     &           &    9 &  2.37e-08 &  6.78e-16 &      \\ 
     &           &   10 &  5.40e-08 &  6.77e-16 &      \\ 
1345 &  1.34e+04 &   10 &           &           & iters  \\ 
 \hdashline 
     &           &    1 &  1.94e-08 &  1.53e-09 &      \\ 
     &           &    2 &  1.94e-08 &  5.53e-16 &      \\ 
     &           &    3 &  5.84e-08 &  5.53e-16 &      \\ 
     &           &    4 &  1.94e-08 &  1.67e-15 &      \\ 
     &           &    5 &  1.94e-08 &  5.54e-16 &      \\ 
     &           &    6 &  1.94e-08 &  5.53e-16 &      \\ 
     &           &    7 &  5.84e-08 &  5.53e-16 &      \\ 
     &           &    8 &  1.94e-08 &  1.67e-15 &      \\ 
     &           &    9 &  1.94e-08 &  5.54e-16 &      \\ 
     &           &   10 &  1.94e-08 &  5.53e-16 &      \\ 
1346 &  1.34e+04 &   10 &           &           & iters  \\ 
 \hdashline 
     &           &    1 &  6.12e-08 &  2.19e-09 &      \\ 
     &           &    2 &  1.65e-08 &  1.75e-15 &      \\ 
     &           &    3 &  1.65e-08 &  4.72e-16 &      \\ 
     &           &    4 &  1.65e-08 &  4.71e-16 &      \\ 
     &           &    5 &  1.65e-08 &  4.71e-16 &      \\ 
     &           &    6 &  6.12e-08 &  4.70e-16 &      \\ 
     &           &    7 &  1.65e-08 &  1.75e-15 &      \\ 
     &           &    8 &  1.65e-08 &  4.72e-16 &      \\ 
     &           &    9 &  1.65e-08 &  4.71e-16 &      \\ 
     &           &   10 &  1.65e-08 &  4.71e-16 &      \\ 
1347 &  1.35e+04 &   10 &           &           & iters  \\ 
 \hdashline 
     &           &    1 &  6.69e-08 &  9.58e-10 &      \\ 
     &           &    2 &  1.09e-08 &  1.91e-15 &      \\ 
     &           &    3 &  1.09e-08 &  3.12e-16 &      \\ 
     &           &    4 &  1.09e-08 &  3.11e-16 &      \\ 
     &           &    5 &  1.09e-08 &  3.11e-16 &      \\ 
     &           &    6 &  1.09e-08 &  3.10e-16 &      \\ 
     &           &    7 &  6.68e-08 &  3.10e-16 &      \\ 
     &           &    8 &  8.86e-08 &  1.91e-15 &      \\ 
     &           &    9 &  6.69e-08 &  2.53e-15 &      \\ 
     &           &   10 &  1.09e-08 &  1.91e-15 &      \\ 
1348 &  1.35e+04 &   10 &           &           & iters  \\ 
 \hdashline 
     &           &    1 &  1.80e-08 &  4.31e-10 &      \\ 
     &           &    2 &  5.97e-08 &  5.14e-16 &      \\ 
     &           &    3 &  1.81e-08 &  1.70e-15 &      \\ 
     &           &    4 &  1.80e-08 &  5.15e-16 &      \\ 
     &           &    5 &  1.80e-08 &  5.15e-16 &      \\ 
     &           &    6 &  5.97e-08 &  5.14e-16 &      \\ 
     &           &    7 &  1.81e-08 &  1.70e-15 &      \\ 
     &           &    8 &  1.80e-08 &  5.16e-16 &      \\ 
     &           &    9 &  1.80e-08 &  5.15e-16 &      \\ 
     &           &   10 &  1.80e-08 &  5.14e-16 &      \\ 
1349 &  1.35e+04 &   10 &           &           & iters  \\ 
 \hdashline 
     &           &    1 &  4.15e-08 &  6.38e-11 &      \\ 
     &           &    2 &  2.64e-09 &  1.19e-15 &      \\ 
     &           &    3 &  2.64e-09 &  7.53e-17 &      \\ 
     &           &    4 &  2.63e-09 &  7.52e-17 &      \\ 
     &           &    5 &  2.63e-09 &  7.51e-17 &      \\ 
     &           &    6 &  2.62e-09 &  7.50e-17 &      \\ 
     &           &    7 &  2.62e-09 &  7.49e-17 &      \\ 
     &           &    8 &  2.62e-09 &  7.48e-17 &      \\ 
     &           &    9 &  2.61e-09 &  7.47e-17 &      \\ 
     &           &   10 &  2.61e-09 &  7.46e-17 &      \\ 
1350 &  1.35e+04 &   10 &           &           & iters  \\ 
 \hdashline 
     &           &    1 &  3.13e-08 &  1.12e-09 &      \\ 
     &           &    2 &  3.12e-08 &  8.92e-16 &      \\ 
     &           &    3 &  4.65e-08 &  8.91e-16 &      \\ 
     &           &    4 &  3.12e-08 &  1.33e-15 &      \\ 
     &           &    5 &  4.65e-08 &  8.92e-16 &      \\ 
     &           &    6 &  3.13e-08 &  1.33e-15 &      \\ 
     &           &    7 &  3.12e-08 &  8.92e-16 &      \\ 
     &           &    8 &  4.65e-08 &  8.91e-16 &      \\ 
     &           &    9 &  3.12e-08 &  1.33e-15 &      \\ 
     &           &   10 &  4.65e-08 &  8.92e-16 &      \\ 
1351 &  1.35e+04 &   10 &           &           & iters  \\ 
 \hdashline 
     &           &    1 &  1.84e-08 &  1.16e-09 &      \\ 
     &           &    2 &  1.84e-08 &  5.26e-16 &      \\ 
     &           &    3 &  2.05e-08 &  5.25e-16 &      \\ 
     &           &    4 &  1.84e-08 &  5.85e-16 &      \\ 
     &           &    5 &  2.05e-08 &  5.25e-16 &      \\ 
     &           &    6 &  1.84e-08 &  5.84e-16 &      \\ 
     &           &    7 &  2.05e-08 &  5.25e-16 &      \\ 
     &           &    8 &  1.84e-08 &  5.84e-16 &      \\ 
     &           &    9 &  2.05e-08 &  5.25e-16 &      \\ 
     &           &   10 &  1.84e-08 &  5.84e-16 &      \\ 
1352 &  1.35e+04 &   10 &           &           & iters  \\ 
 \hdashline 
     &           &    1 &  3.95e-10 &  1.20e-09 &      \\ 
     &           &    2 &  3.95e-10 &  1.13e-17 &      \\ 
     &           &    3 &  3.94e-10 &  1.13e-17 &      \\ 
     &           &    4 &  3.94e-10 &  1.13e-17 &      \\ 
     &           &    5 &  3.93e-10 &  1.12e-17 &      \\ 
     &           &    6 &  3.93e-10 &  1.12e-17 &      \\ 
     &           &    7 &  3.92e-10 &  1.12e-17 &      \\ 
     &           &    8 &  3.92e-10 &  1.12e-17 &      \\ 
     &           &    9 &  3.91e-10 &  1.12e-17 &      \\ 
     &           &   10 &  3.90e-10 &  1.12e-17 &      \\ 
1353 &  1.35e+04 &   10 &           &           & iters  \\ 
 \hdashline 
     &           &    1 &  2.13e-08 &  1.37e-09 &      \\ 
     &           &    2 &  5.64e-08 &  6.09e-16 &      \\ 
     &           &    3 &  2.14e-08 &  1.61e-15 &      \\ 
     &           &    4 &  2.14e-08 &  6.10e-16 &      \\ 
     &           &    5 &  2.13e-08 &  6.09e-16 &      \\ 
     &           &    6 &  5.64e-08 &  6.09e-16 &      \\ 
     &           &    7 &  2.14e-08 &  1.61e-15 &      \\ 
     &           &    8 &  2.13e-08 &  6.10e-16 &      \\ 
     &           &    9 &  5.64e-08 &  6.09e-16 &      \\ 
     &           &   10 &  2.14e-08 &  1.61e-15 &      \\ 
1354 &  1.35e+04 &   10 &           &           & iters  \\ 
 \hdashline 
     &           &    1 &  1.54e-08 &  6.07e-10 &      \\ 
     &           &    2 &  1.53e-08 &  4.38e-16 &      \\ 
     &           &    3 &  1.53e-08 &  4.38e-16 &      \\ 
     &           &    4 &  6.24e-08 &  4.37e-16 &      \\ 
     &           &    5 &  1.54e-08 &  1.78e-15 &      \\ 
     &           &    6 &  1.54e-08 &  4.39e-16 &      \\ 
     &           &    7 &  1.53e-08 &  4.38e-16 &      \\ 
     &           &    8 &  1.53e-08 &  4.38e-16 &      \\ 
     &           &    9 &  6.24e-08 &  4.37e-16 &      \\ 
     &           &   10 &  1.54e-08 &  1.78e-15 &      \\ 
1355 &  1.35e+04 &   10 &           &           & iters  \\ 
 \hdashline 
     &           &    1 &  6.13e-10 &  7.21e-10 &      \\ 
     &           &    2 &  6.12e-10 &  1.75e-17 &      \\ 
     &           &    3 &  6.11e-10 &  1.75e-17 &      \\ 
     &           &    4 &  6.10e-10 &  1.74e-17 &      \\ 
     &           &    5 &  6.09e-10 &  1.74e-17 &      \\ 
     &           &    6 &  6.08e-10 &  1.74e-17 &      \\ 
     &           &    7 &  6.07e-10 &  1.74e-17 &      \\ 
     &           &    8 &  6.06e-10 &  1.73e-17 &      \\ 
     &           &    9 &  6.06e-10 &  1.73e-17 &      \\ 
     &           &   10 &  6.05e-10 &  1.73e-17 &      \\ 
1356 &  1.36e+04 &   10 &           &           & iters  \\ 
 \hdashline 
     &           &    1 &  3.51e-08 &  2.74e-09 &      \\ 
     &           &    2 &  3.81e-09 &  1.00e-15 &      \\ 
     &           &    3 &  3.81e-09 &  1.09e-16 &      \\ 
     &           &    4 &  3.80e-09 &  1.09e-16 &      \\ 
     &           &    5 &  3.80e-09 &  1.09e-16 &      \\ 
     &           &    6 &  3.79e-09 &  1.08e-16 &      \\ 
     &           &    7 &  3.79e-09 &  1.08e-16 &      \\ 
     &           &    8 &  3.78e-09 &  1.08e-16 &      \\ 
     &           &    9 &  3.51e-08 &  1.08e-16 &      \\ 
     &           &   10 &  3.83e-09 &  1.00e-15 &      \\ 
1357 &  1.36e+04 &   10 &           &           & iters  \\ 
 \hdashline 
     &           &    1 &  1.15e-09 &  2.73e-09 &      \\ 
     &           &    2 &  1.15e-09 &  3.29e-17 &      \\ 
     &           &    3 &  1.15e-09 &  3.29e-17 &      \\ 
     &           &    4 &  1.15e-09 &  3.28e-17 &      \\ 
     &           &    5 &  1.15e-09 &  3.28e-17 &      \\ 
     &           &    6 &  1.15e-09 &  3.27e-17 &      \\ 
     &           &    7 &  1.14e-09 &  3.27e-17 &      \\ 
     &           &    8 &  1.14e-09 &  3.26e-17 &      \\ 
     &           &    9 &  1.14e-09 &  3.26e-17 &      \\ 
     &           &   10 &  1.14e-09 &  3.26e-17 &      \\ 
1358 &  1.36e+04 &   10 &           &           & iters  \\ 
 \hdashline 
     &           &    1 &  3.77e-08 &  1.56e-10 &      \\ 
     &           &    2 &  1.17e-09 &  1.08e-15 &      \\ 
     &           &    3 &  1.17e-09 &  3.35e-17 &      \\ 
     &           &    4 &  1.17e-09 &  3.34e-17 &      \\ 
     &           &    5 &  1.17e-09 &  3.34e-17 &      \\ 
     &           &    6 &  1.17e-09 &  3.33e-17 &      \\ 
     &           &    7 &  1.16e-09 &  3.33e-17 &      \\ 
     &           &    8 &  1.16e-09 &  3.32e-17 &      \\ 
     &           &    9 &  1.16e-09 &  3.32e-17 &      \\ 
     &           &   10 &  1.16e-09 &  3.31e-17 &      \\ 
1359 &  1.36e+04 &   10 &           &           & iters  \\ 
 \hdashline 
     &           &    1 &  2.58e-08 &  2.22e-09 &      \\ 
     &           &    2 &  5.20e-08 &  7.36e-16 &      \\ 
     &           &    3 &  2.58e-08 &  1.48e-15 &      \\ 
     &           &    4 &  2.58e-08 &  7.37e-16 &      \\ 
     &           &    5 &  5.20e-08 &  7.36e-16 &      \\ 
     &           &    6 &  2.58e-08 &  1.48e-15 &      \\ 
     &           &    7 &  2.58e-08 &  7.37e-16 &      \\ 
     &           &    8 &  5.20e-08 &  7.36e-16 &      \\ 
     &           &    9 &  2.58e-08 &  1.48e-15 &      \\ 
     &           &   10 &  2.58e-08 &  7.37e-16 &      \\ 
1360 &  1.36e+04 &   10 &           &           & iters  \\ 
 \hdashline 
     &           &    1 &  1.29e-08 &  2.71e-09 &      \\ 
     &           &    2 &  1.29e-08 &  3.68e-16 &      \\ 
     &           &    3 &  1.29e-08 &  3.68e-16 &      \\ 
     &           &    4 &  1.29e-08 &  3.67e-16 &      \\ 
     &           &    5 &  6.49e-08 &  3.67e-16 &      \\ 
     &           &    6 &  1.29e-08 &  1.85e-15 &      \\ 
     &           &    7 &  1.29e-08 &  3.69e-16 &      \\ 
     &           &    8 &  1.29e-08 &  3.68e-16 &      \\ 
     &           &    9 &  1.29e-08 &  3.68e-16 &      \\ 
     &           &   10 &  1.29e-08 &  3.67e-16 &      \\ 
1361 &  1.36e+04 &   10 &           &           & iters  \\ 
 \hdashline 
     &           &    1 &  3.27e-08 &  2.61e-09 &      \\ 
     &           &    2 &  3.26e-08 &  9.33e-16 &      \\ 
     &           &    3 &  4.51e-08 &  9.31e-16 &      \\ 
     &           &    4 &  3.27e-08 &  1.29e-15 &      \\ 
     &           &    5 &  4.51e-08 &  9.32e-16 &      \\ 
     &           &    6 &  3.27e-08 &  1.29e-15 &      \\ 
     &           &    7 &  3.26e-08 &  9.32e-16 &      \\ 
     &           &    8 &  4.51e-08 &  9.31e-16 &      \\ 
     &           &    9 &  3.26e-08 &  1.29e-15 &      \\ 
     &           &   10 &  4.51e-08 &  9.32e-16 &      \\ 
1362 &  1.36e+04 &   10 &           &           & iters  \\ 
 \hdashline 
     &           &    1 &  1.51e-08 &  2.49e-09 &      \\ 
     &           &    2 &  1.51e-08 &  4.32e-16 &      \\ 
     &           &    3 &  1.51e-08 &  4.32e-16 &      \\ 
     &           &    4 &  6.26e-08 &  4.31e-16 &      \\ 
     &           &    5 &  9.29e-08 &  1.79e-15 &      \\ 
     &           &    6 &  6.27e-08 &  2.65e-15 &      \\ 
     &           &    7 &  1.51e-08 &  1.79e-15 &      \\ 
     &           &    8 &  1.51e-08 &  4.32e-16 &      \\ 
     &           &    9 &  1.51e-08 &  4.31e-16 &      \\ 
     &           &   10 &  6.26e-08 &  4.30e-16 &      \\ 
1363 &  1.36e+04 &   10 &           &           & iters  \\ 
 \hdashline 
     &           &    1 &  6.07e-08 &  4.78e-10 &      \\ 
     &           &    2 &  1.71e-08 &  1.73e-15 &      \\ 
     &           &    3 &  1.71e-08 &  4.88e-16 &      \\ 
     &           &    4 &  1.70e-08 &  4.87e-16 &      \\ 
     &           &    5 &  1.70e-08 &  4.86e-16 &      \\ 
     &           &    6 &  6.07e-08 &  4.86e-16 &      \\ 
     &           &    7 &  1.71e-08 &  1.73e-15 &      \\ 
     &           &    8 &  1.71e-08 &  4.87e-16 &      \\ 
     &           &    9 &  1.70e-08 &  4.87e-16 &      \\ 
     &           &   10 &  6.07e-08 &  4.86e-16 &      \\ 
1364 &  1.36e+04 &   10 &           &           & iters  \\ 
 \hdashline 
     &           &    1 &  3.54e-08 &  5.20e-10 &      \\ 
     &           &    2 &  3.53e-08 &  1.01e-15 &      \\ 
     &           &    3 &  4.24e-08 &  1.01e-15 &      \\ 
     &           &    4 &  3.53e-08 &  1.21e-15 &      \\ 
     &           &    5 &  4.24e-08 &  1.01e-15 &      \\ 
     &           &    6 &  3.54e-08 &  1.21e-15 &      \\ 
     &           &    7 &  4.24e-08 &  1.01e-15 &      \\ 
     &           &    8 &  3.54e-08 &  1.21e-15 &      \\ 
     &           &    9 &  3.53e-08 &  1.01e-15 &      \\ 
     &           &   10 &  4.24e-08 &  1.01e-15 &      \\ 
1365 &  1.36e+04 &   10 &           &           & iters  \\ 
 \hdashline 
     &           &    1 &  1.19e-08 &  1.31e-09 &      \\ 
     &           &    2 &  6.58e-08 &  3.41e-16 &      \\ 
     &           &    3 &  1.20e-08 &  1.88e-15 &      \\ 
     &           &    4 &  1.20e-08 &  3.43e-16 &      \\ 
     &           &    5 &  1.20e-08 &  3.42e-16 &      \\ 
     &           &    6 &  1.20e-08 &  3.42e-16 &      \\ 
     &           &    7 &  1.19e-08 &  3.41e-16 &      \\ 
     &           &    8 &  6.58e-08 &  3.41e-16 &      \\ 
     &           &    9 &  8.97e-08 &  1.88e-15 &      \\ 
     &           &   10 &  6.58e-08 &  2.56e-15 &      \\ 
1366 &  1.36e+04 &   10 &           &           & iters  \\ 
 \hdashline 
     &           &    1 &  8.01e-08 &  2.61e-10 &      \\ 
     &           &    2 &  7.54e-08 &  2.29e-15 &      \\ 
     &           &    3 &  2.40e-09 &  2.15e-15 &      \\ 
     &           &    4 &  2.39e-09 &  6.84e-17 &      \\ 
     &           &    5 &  2.39e-09 &  6.83e-17 &      \\ 
     &           &    6 &  2.39e-09 &  6.82e-17 &      \\ 
     &           &    7 &  2.38e-09 &  6.81e-17 &      \\ 
     &           &    8 &  2.38e-09 &  6.80e-17 &      \\ 
     &           &    9 &  2.38e-09 &  6.79e-17 &      \\ 
     &           &   10 &  2.37e-09 &  6.78e-17 &      \\ 
1367 &  1.37e+04 &   10 &           &           & iters  \\ 
 \hdashline 
     &           &    1 &  8.47e-09 &  1.14e-09 &      \\ 
     &           &    2 &  8.45e-09 &  2.42e-16 &      \\ 
     &           &    3 &  8.44e-09 &  2.41e-16 &      \\ 
     &           &    4 &  8.43e-09 &  2.41e-16 &      \\ 
     &           &    5 &  8.42e-09 &  2.41e-16 &      \\ 
     &           &    6 &  6.93e-08 &  2.40e-16 &      \\ 
     &           &    7 &  8.50e-09 &  1.98e-15 &      \\ 
     &           &    8 &  8.49e-09 &  2.43e-16 &      \\ 
     &           &    9 &  8.48e-09 &  2.42e-16 &      \\ 
     &           &   10 &  8.47e-09 &  2.42e-16 &      \\ 
1368 &  1.37e+04 &   10 &           &           & iters  \\ 
 \hdashline 
     &           &    1 &  6.30e-08 &  7.57e-10 &      \\ 
     &           &    2 &  1.48e-08 &  1.80e-15 &      \\ 
     &           &    3 &  1.48e-08 &  4.22e-16 &      \\ 
     &           &    4 &  1.47e-08 &  4.21e-16 &      \\ 
     &           &    5 &  1.47e-08 &  4.21e-16 &      \\ 
     &           &    6 &  6.30e-08 &  4.20e-16 &      \\ 
     &           &    7 &  1.48e-08 &  1.80e-15 &      \\ 
     &           &    8 &  1.48e-08 &  4.22e-16 &      \\ 
     &           &    9 &  1.47e-08 &  4.22e-16 &      \\ 
     &           &   10 &  1.47e-08 &  4.21e-16 &      \\ 
1369 &  1.37e+04 &   10 &           &           & iters  \\ 
 \hdashline 
     &           &    1 &  2.75e-08 &  2.55e-09 &      \\ 
     &           &    2 &  5.02e-08 &  7.85e-16 &      \\ 
     &           &    3 &  2.75e-08 &  1.43e-15 &      \\ 
     &           &    4 &  2.75e-08 &  7.86e-16 &      \\ 
     &           &    5 &  5.02e-08 &  7.85e-16 &      \\ 
     &           &    6 &  2.75e-08 &  1.43e-15 &      \\ 
     &           &    7 &  2.75e-08 &  7.85e-16 &      \\ 
     &           &    8 &  5.02e-08 &  7.84e-16 &      \\ 
     &           &    9 &  2.75e-08 &  1.43e-15 &      \\ 
     &           &   10 &  2.75e-08 &  7.85e-16 &      \\ 
1370 &  1.37e+04 &   10 &           &           & iters  \\ 
 \hdashline 
     &           &    1 &  7.64e-09 &  2.81e-09 &      \\ 
     &           &    2 &  7.01e-08 &  2.18e-16 &      \\ 
     &           &    3 &  4.66e-08 &  2.00e-15 &      \\ 
     &           &    4 &  7.66e-09 &  1.33e-15 &      \\ 
     &           &    5 &  7.65e-09 &  2.19e-16 &      \\ 
     &           &    6 &  7.64e-09 &  2.18e-16 &      \\ 
     &           &    7 &  7.01e-08 &  2.18e-16 &      \\ 
     &           &    8 &  4.66e-08 &  2.00e-15 &      \\ 
     &           &    9 &  7.66e-09 &  1.33e-15 &      \\ 
     &           &   10 &  7.65e-09 &  2.19e-16 &      \\ 
1371 &  1.37e+04 &   10 &           &           & iters  \\ 
 \hdashline 
     &           &    1 &  3.51e-08 &  1.10e-09 &      \\ 
     &           &    2 &  4.27e-08 &  1.00e-15 &      \\ 
     &           &    3 &  3.51e-08 &  1.22e-15 &      \\ 
     &           &    4 &  4.27e-08 &  1.00e-15 &      \\ 
     &           &    5 &  3.51e-08 &  1.22e-15 &      \\ 
     &           &    6 &  3.50e-08 &  1.00e-15 &      \\ 
     &           &    7 &  4.27e-08 &  1.00e-15 &      \\ 
     &           &    8 &  3.50e-08 &  1.22e-15 &      \\ 
     &           &    9 &  4.27e-08 &  1.00e-15 &      \\ 
     &           &   10 &  3.50e-08 &  1.22e-15 &      \\ 
1372 &  1.37e+04 &   10 &           &           & iters  \\ 
 \hdashline 
     &           &    1 &  3.23e-08 &  2.88e-09 &      \\ 
     &           &    2 &  4.55e-08 &  9.20e-16 &      \\ 
     &           &    3 &  3.23e-08 &  1.30e-15 &      \\ 
     &           &    4 &  4.55e-08 &  9.21e-16 &      \\ 
     &           &    5 &  3.23e-08 &  1.30e-15 &      \\ 
     &           &    6 &  3.22e-08 &  9.22e-16 &      \\ 
     &           &    7 &  4.55e-08 &  9.20e-16 &      \\ 
     &           &    8 &  3.23e-08 &  1.30e-15 &      \\ 
     &           &    9 &  4.55e-08 &  9.21e-16 &      \\ 
     &           &   10 &  3.23e-08 &  1.30e-15 &      \\ 
1373 &  1.37e+04 &   10 &           &           & iters  \\ 
 \hdashline 
     &           &    1 &  9.43e-09 &  1.78e-09 &      \\ 
     &           &    2 &  9.41e-09 &  2.69e-16 &      \\ 
     &           &    3 &  9.40e-09 &  2.69e-16 &      \\ 
     &           &    4 &  9.39e-09 &  2.68e-16 &      \\ 
     &           &    5 &  9.37e-09 &  2.68e-16 &      \\ 
     &           &    6 &  9.36e-09 &  2.68e-16 &      \\ 
     &           &    7 &  9.35e-09 &  2.67e-16 &      \\ 
     &           &    8 &  9.33e-09 &  2.67e-16 &      \\ 
     &           &    9 &  9.32e-09 &  2.66e-16 &      \\ 
     &           &   10 &  6.84e-08 &  2.66e-16 &      \\ 
1374 &  1.37e+04 &   10 &           &           & iters  \\ 
 \hdashline 
     &           &    1 &  4.71e-10 &  1.45e-09 &      \\ 
     &           &    2 &  4.71e-10 &  1.35e-17 &      \\ 
     &           &    3 &  4.70e-10 &  1.34e-17 &      \\ 
     &           &    4 &  4.69e-10 &  1.34e-17 &      \\ 
     &           &    5 &  4.69e-10 &  1.34e-17 &      \\ 
     &           &    6 &  4.68e-10 &  1.34e-17 &      \\ 
     &           &    7 &  4.67e-10 &  1.34e-17 &      \\ 
     &           &    8 &  4.67e-10 &  1.33e-17 &      \\ 
     &           &    9 &  4.66e-10 &  1.33e-17 &      \\ 
     &           &   10 &  4.65e-10 &  1.33e-17 &      \\ 
1375 &  1.37e+04 &   10 &           &           & iters  \\ 
 \hdashline 
     &           &    1 &  3.37e-08 &  6.56e-10 &      \\ 
     &           &    2 &  4.40e-08 &  9.63e-16 &      \\ 
     &           &    3 &  3.38e-08 &  1.26e-15 &      \\ 
     &           &    4 &  4.40e-08 &  9.64e-16 &      \\ 
     &           &    5 &  3.38e-08 &  1.26e-15 &      \\ 
     &           &    6 &  4.40e-08 &  9.64e-16 &      \\ 
     &           &    7 &  3.38e-08 &  1.25e-15 &      \\ 
     &           &    8 &  3.37e-08 &  9.64e-16 &      \\ 
     &           &    9 &  4.40e-08 &  9.63e-16 &      \\ 
     &           &   10 &  3.38e-08 &  1.26e-15 &      \\ 
1376 &  1.38e+04 &   10 &           &           & iters  \\ 
 \hdashline 
     &           &    1 &  2.12e-08 &  4.88e-10 &      \\ 
     &           &    2 &  5.65e-08 &  6.06e-16 &      \\ 
     &           &    3 &  2.13e-08 &  1.61e-15 &      \\ 
     &           &    4 &  2.13e-08 &  6.08e-16 &      \\ 
     &           &    5 &  2.12e-08 &  6.07e-16 &      \\ 
     &           &    6 &  5.65e-08 &  6.06e-16 &      \\ 
     &           &    7 &  2.13e-08 &  1.61e-15 &      \\ 
     &           &    8 &  2.12e-08 &  6.07e-16 &      \\ 
     &           &    9 &  5.65e-08 &  6.06e-16 &      \\ 
     &           &   10 &  2.13e-08 &  1.61e-15 &      \\ 
1377 &  1.38e+04 &   10 &           &           & iters  \\ 
 \hdashline 
     &           &    1 &  5.72e-10 &  2.13e-09 &      \\ 
     &           &    2 &  5.71e-10 &  1.63e-17 &      \\ 
     &           &    3 &  5.70e-10 &  1.63e-17 &      \\ 
     &           &    4 &  5.69e-10 &  1.63e-17 &      \\ 
     &           &    5 &  5.69e-10 &  1.62e-17 &      \\ 
     &           &    6 &  5.68e-10 &  1.62e-17 &      \\ 
     &           &    7 &  5.67e-10 &  1.62e-17 &      \\ 
     &           &    8 &  5.66e-10 &  1.62e-17 &      \\ 
     &           &    9 &  5.65e-10 &  1.62e-17 &      \\ 
     &           &   10 &  5.64e-10 &  1.61e-17 &      \\ 
1378 &  1.38e+04 &   10 &           &           & iters  \\ 
 \hdashline 
     &           &    1 &  4.11e-09 &  2.10e-09 &      \\ 
     &           &    2 &  4.10e-09 &  1.17e-16 &      \\ 
     &           &    3 &  4.09e-09 &  1.17e-16 &      \\ 
     &           &    4 &  4.09e-09 &  1.17e-16 &      \\ 
     &           &    5 &  4.08e-09 &  1.17e-16 &      \\ 
     &           &    6 &  4.08e-09 &  1.16e-16 &      \\ 
     &           &    7 &  4.07e-09 &  1.16e-16 &      \\ 
     &           &    8 &  7.36e-08 &  1.16e-16 &      \\ 
     &           &    9 &  8.19e-08 &  2.10e-15 &      \\ 
     &           &   10 &  7.36e-08 &  2.34e-15 &      \\ 
1379 &  1.38e+04 &   10 &           &           & iters  \\ 
 \hdashline 
     &           &    1 &  2.06e-08 &  1.53e-10 &      \\ 
     &           &    2 &  2.06e-08 &  5.89e-16 &      \\ 
     &           &    3 &  2.06e-08 &  5.88e-16 &      \\ 
     &           &    4 &  5.71e-08 &  5.88e-16 &      \\ 
     &           &    5 &  2.06e-08 &  1.63e-15 &      \\ 
     &           &    6 &  2.06e-08 &  5.89e-16 &      \\ 
     &           &    7 &  2.06e-08 &  5.88e-16 &      \\ 
     &           &    8 &  5.71e-08 &  5.87e-16 &      \\ 
     &           &    9 &  2.06e-08 &  1.63e-15 &      \\ 
     &           &   10 &  2.06e-08 &  5.89e-16 &      \\ 
1380 &  1.38e+04 &   10 &           &           & iters  \\ 
 \hdashline 
     &           &    1 &  1.42e-08 &  1.42e-09 &      \\ 
     &           &    2 &  1.42e-08 &  4.06e-16 &      \\ 
     &           &    3 &  1.42e-08 &  4.06e-16 &      \\ 
     &           &    4 &  1.42e-08 &  4.05e-16 &      \\ 
     &           &    5 &  1.42e-08 &  4.05e-16 &      \\ 
     &           &    6 &  6.36e-08 &  4.04e-16 &      \\ 
     &           &    7 &  1.42e-08 &  1.81e-15 &      \\ 
     &           &    8 &  1.42e-08 &  4.06e-16 &      \\ 
     &           &    9 &  1.42e-08 &  4.05e-16 &      \\ 
     &           &   10 &  1.42e-08 &  4.05e-16 &      \\ 
1381 &  1.38e+04 &   10 &           &           & iters  \\ 
 \hdashline 
     &           &    1 &  5.01e-08 &  3.07e-09 &      \\ 
     &           &    2 &  2.76e-08 &  1.43e-15 &      \\ 
     &           &    3 &  2.76e-08 &  7.89e-16 &      \\ 
     &           &    4 &  5.01e-08 &  7.88e-16 &      \\ 
     &           &    5 &  2.76e-08 &  1.43e-15 &      \\ 
     &           &    6 &  2.76e-08 &  7.88e-16 &      \\ 
     &           &    7 &  5.01e-08 &  7.87e-16 &      \\ 
     &           &    8 &  2.76e-08 &  1.43e-15 &      \\ 
     &           &    9 &  5.01e-08 &  7.88e-16 &      \\ 
     &           &   10 &  2.77e-08 &  1.43e-15 &      \\ 
1382 &  1.38e+04 &   10 &           &           & iters  \\ 
 \hdashline 
     &           &    1 &  1.45e-08 &  2.35e-09 &      \\ 
     &           &    2 &  1.45e-08 &  4.15e-16 &      \\ 
     &           &    3 &  6.32e-08 &  4.14e-16 &      \\ 
     &           &    4 &  1.46e-08 &  1.80e-15 &      \\ 
     &           &    5 &  1.46e-08 &  4.16e-16 &      \\ 
     &           &    6 &  1.45e-08 &  4.16e-16 &      \\ 
     &           &    7 &  1.45e-08 &  4.15e-16 &      \\ 
     &           &    8 &  6.32e-08 &  4.14e-16 &      \\ 
     &           &    9 &  1.46e-08 &  1.80e-15 &      \\ 
     &           &   10 &  1.46e-08 &  4.16e-16 &      \\ 
1383 &  1.38e+04 &   10 &           &           & iters  \\ 
 \hdashline 
     &           &    1 &  4.83e-08 &  6.18e-10 &      \\ 
     &           &    2 &  2.95e-08 &  1.38e-15 &      \\ 
     &           &    3 &  2.95e-08 &  8.42e-16 &      \\ 
     &           &    4 &  4.83e-08 &  8.41e-16 &      \\ 
     &           &    5 &  2.95e-08 &  1.38e-15 &      \\ 
     &           &    6 &  2.94e-08 &  8.41e-16 &      \\ 
     &           &    7 &  4.83e-08 &  8.40e-16 &      \\ 
     &           &    8 &  2.95e-08 &  1.38e-15 &      \\ 
     &           &    9 &  4.83e-08 &  8.41e-16 &      \\ 
     &           &   10 &  2.95e-08 &  1.38e-15 &      \\ 
1384 &  1.38e+04 &   10 &           &           & iters  \\ 
 \hdashline 
     &           &    1 &  3.78e-08 &  3.61e-09 &      \\ 
     &           &    2 &  3.99e-08 &  1.08e-15 &      \\ 
     &           &    3 &  3.78e-08 &  1.14e-15 &      \\ 
     &           &    4 &  3.99e-08 &  1.08e-15 &      \\ 
     &           &    5 &  3.78e-08 &  1.14e-15 &      \\ 
     &           &    6 &  3.99e-08 &  1.08e-15 &      \\ 
     &           &    7 &  3.79e-08 &  1.14e-15 &      \\ 
     &           &    8 &  3.99e-08 &  1.08e-15 &      \\ 
     &           &    9 &  3.79e-08 &  1.14e-15 &      \\ 
     &           &   10 &  3.99e-08 &  1.08e-15 &      \\ 
1385 &  1.38e+04 &   10 &           &           & iters  \\ 
 \hdashline 
     &           &    1 &  1.10e-08 &  5.96e-09 &      \\ 
     &           &    2 &  1.10e-08 &  3.15e-16 &      \\ 
     &           &    3 &  1.10e-08 &  3.14e-16 &      \\ 
     &           &    4 &  1.10e-08 &  3.14e-16 &      \\ 
     &           &    5 &  6.67e-08 &  3.13e-16 &      \\ 
     &           &    6 &  8.87e-08 &  1.90e-15 &      \\ 
     &           &    7 &  6.68e-08 &  2.53e-15 &      \\ 
     &           &    8 &  1.10e-08 &  1.91e-15 &      \\ 
     &           &    9 &  1.10e-08 &  3.15e-16 &      \\ 
     &           &   10 &  1.10e-08 &  3.14e-16 &      \\ 
1386 &  1.38e+04 &   10 &           &           & iters  \\ 
 \hdashline 
     &           &    1 &  8.67e-09 &  2.26e-09 &      \\ 
     &           &    2 &  8.66e-09 &  2.48e-16 &      \\ 
     &           &    3 &  6.90e-08 &  2.47e-16 &      \\ 
     &           &    4 &  8.64e-08 &  1.97e-15 &      \\ 
     &           &    5 &  6.91e-08 &  2.47e-15 &      \\ 
     &           &    6 &  8.72e-09 &  1.97e-15 &      \\ 
     &           &    7 &  8.71e-09 &  2.49e-16 &      \\ 
     &           &    8 &  8.70e-09 &  2.49e-16 &      \\ 
     &           &    9 &  8.68e-09 &  2.48e-16 &      \\ 
     &           &   10 &  8.67e-09 &  2.48e-16 &      \\ 
1387 &  1.39e+04 &   10 &           &           & iters  \\ 
 \hdashline 
     &           &    1 &  3.33e-08 &  2.41e-09 &      \\ 
     &           &    2 &  5.58e-09 &  9.51e-16 &      \\ 
     &           &    3 &  5.57e-09 &  1.59e-16 &      \\ 
     &           &    4 &  5.57e-09 &  1.59e-16 &      \\ 
     &           &    5 &  5.56e-09 &  1.59e-16 &      \\ 
     &           &    6 &  5.55e-09 &  1.59e-16 &      \\ 
     &           &    7 &  3.33e-08 &  1.58e-16 &      \\ 
     &           &    8 &  5.59e-09 &  9.50e-16 &      \\ 
     &           &    9 &  5.58e-09 &  1.60e-16 &      \\ 
     &           &   10 &  5.57e-09 &  1.59e-16 &      \\ 
1388 &  1.39e+04 &   10 &           &           & iters  \\ 
 \hdashline 
     &           &    1 &  1.44e-08 &  3.03e-09 &      \\ 
     &           &    2 &  2.45e-08 &  4.10e-16 &      \\ 
     &           &    3 &  1.44e-08 &  6.99e-16 &      \\ 
     &           &    4 &  1.44e-08 &  4.10e-16 &      \\ 
     &           &    5 &  2.45e-08 &  4.10e-16 &      \\ 
     &           &    6 &  1.44e-08 &  6.99e-16 &      \\ 
     &           &    7 &  2.45e-08 &  4.10e-16 &      \\ 
     &           &    8 &  1.44e-08 &  6.99e-16 &      \\ 
     &           &    9 &  1.44e-08 &  4.11e-16 &      \\ 
     &           &   10 &  2.45e-08 &  4.10e-16 &      \\ 
1389 &  1.39e+04 &   10 &           &           & iters  \\ 
 \hdashline 
     &           &    1 &  1.75e-08 &  6.77e-10 &      \\ 
     &           &    2 &  1.75e-08 &  4.99e-16 &      \\ 
     &           &    3 &  6.02e-08 &  4.98e-16 &      \\ 
     &           &    4 &  1.75e-08 &  1.72e-15 &      \\ 
     &           &    5 &  1.75e-08 &  5.00e-16 &      \\ 
     &           &    6 &  1.75e-08 &  4.99e-16 &      \\ 
     &           &    7 &  6.02e-08 &  4.99e-16 &      \\ 
     &           &    8 &  1.75e-08 &  1.72e-15 &      \\ 
     &           &    9 &  1.75e-08 &  5.00e-16 &      \\ 
     &           &   10 &  1.75e-08 &  5.00e-16 &      \\ 
1390 &  1.39e+04 &   10 &           &           & iters  \\ 
 \hdashline 
     &           &    1 &  3.40e-08 &  3.73e-10 &      \\ 
     &           &    2 &  4.38e-08 &  9.69e-16 &      \\ 
     &           &    3 &  3.40e-08 &  1.25e-15 &      \\ 
     &           &    4 &  4.38e-08 &  9.70e-16 &      \\ 
     &           &    5 &  3.40e-08 &  1.25e-15 &      \\ 
     &           &    6 &  4.37e-08 &  9.70e-16 &      \\ 
     &           &    7 &  3.40e-08 &  1.25e-15 &      \\ 
     &           &    8 &  3.40e-08 &  9.71e-16 &      \\ 
     &           &    9 &  4.38e-08 &  9.69e-16 &      \\ 
     &           &   10 &  3.40e-08 &  1.25e-15 &      \\ 
1391 &  1.39e+04 &   10 &           &           & iters  \\ 
 \hdashline 
     &           &    1 &  8.08e-09 &  1.53e-09 &      \\ 
     &           &    2 &  8.07e-09 &  2.31e-16 &      \\ 
     &           &    3 &  8.06e-09 &  2.30e-16 &      \\ 
     &           &    4 &  8.04e-09 &  2.30e-16 &      \\ 
     &           &    5 &  8.03e-09 &  2.30e-16 &      \\ 
     &           &    6 &  3.08e-08 &  2.29e-16 &      \\ 
     &           &    7 &  8.07e-09 &  8.80e-16 &      \\ 
     &           &    8 &  8.05e-09 &  2.30e-16 &      \\ 
     &           &    9 &  8.04e-09 &  2.30e-16 &      \\ 
     &           &   10 &  8.03e-09 &  2.30e-16 &      \\ 
1392 &  1.39e+04 &   10 &           &           & iters  \\ 
 \hdashline 
     &           &    1 &  2.66e-08 &  4.11e-09 &      \\ 
     &           &    2 &  5.11e-08 &  7.60e-16 &      \\ 
     &           &    3 &  2.67e-08 &  1.46e-15 &      \\ 
     &           &    4 &  2.66e-08 &  7.61e-16 &      \\ 
     &           &    5 &  5.11e-08 &  7.60e-16 &      \\ 
     &           &    6 &  2.66e-08 &  1.46e-15 &      \\ 
     &           &    7 &  2.66e-08 &  7.61e-16 &      \\ 
     &           &    8 &  5.11e-08 &  7.59e-16 &      \\ 
     &           &    9 &  2.66e-08 &  1.46e-15 &      \\ 
     &           &   10 &  2.66e-08 &  7.60e-16 &      \\ 
1393 &  1.39e+04 &   10 &           &           & iters  \\ 
 \hdashline 
     &           &    1 &  4.73e-08 &  3.04e-09 &      \\ 
     &           &    2 &  3.04e-08 &  1.35e-15 &      \\ 
     &           &    3 &  4.73e-08 &  8.68e-16 &      \\ 
     &           &    4 &  3.04e-08 &  1.35e-15 &      \\ 
     &           &    5 &  4.73e-08 &  8.69e-16 &      \\ 
     &           &    6 &  3.05e-08 &  1.35e-15 &      \\ 
     &           &    7 &  3.04e-08 &  8.69e-16 &      \\ 
     &           &    8 &  4.73e-08 &  8.68e-16 &      \\ 
     &           &    9 &  3.04e-08 &  1.35e-15 &      \\ 
     &           &   10 &  4.73e-08 &  8.69e-16 &      \\ 
1394 &  1.39e+04 &   10 &           &           & iters  \\ 
 \hdashline 
     &           &    1 &  3.39e-09 &  5.15e-12 &      \\ 
     &           &    2 &  3.38e-09 &  9.66e-17 &      \\ 
     &           &    3 &  3.38e-09 &  9.65e-17 &      \\ 
     &           &    4 &  3.37e-09 &  9.64e-17 &      \\ 
     &           &    5 &  3.37e-09 &  9.62e-17 &      \\ 
     &           &    6 &  3.36e-09 &  9.61e-17 &      \\ 
     &           &    7 &  3.36e-09 &  9.60e-17 &      \\ 
     &           &    8 &  3.35e-09 &  9.58e-17 &      \\ 
     &           &    9 &  3.35e-09 &  9.57e-17 &      \\ 
     &           &   10 &  3.34e-09 &  9.55e-17 &      \\ 
1395 &  1.39e+04 &   10 &           &           & iters  \\ 
 \hdashline 
     &           &    1 &  4.63e-08 &  2.25e-09 &      \\ 
     &           &    2 &  3.14e-08 &  1.32e-15 &      \\ 
     &           &    3 &  4.63e-08 &  8.97e-16 &      \\ 
     &           &    4 &  3.15e-08 &  1.32e-15 &      \\ 
     &           &    5 &  3.14e-08 &  8.98e-16 &      \\ 
     &           &    6 &  4.63e-08 &  8.96e-16 &      \\ 
     &           &    7 &  3.14e-08 &  1.32e-15 &      \\ 
     &           &    8 &  4.63e-08 &  8.97e-16 &      \\ 
     &           &    9 &  3.14e-08 &  1.32e-15 &      \\ 
     &           &   10 &  3.14e-08 &  8.98e-16 &      \\ 
1396 &  1.40e+04 &   10 &           &           & iters  \\ 
 \hdashline 
     &           &    1 &  3.68e-08 &  3.16e-09 &      \\ 
     &           &    2 &  4.09e-08 &  1.05e-15 &      \\ 
     &           &    3 &  3.68e-08 &  1.17e-15 &      \\ 
     &           &    4 &  4.09e-08 &  1.05e-15 &      \\ 
     &           &    5 &  3.68e-08 &  1.17e-15 &      \\ 
     &           &    6 &  4.09e-08 &  1.05e-15 &      \\ 
     &           &    7 &  3.68e-08 &  1.17e-15 &      \\ 
     &           &    8 &  4.09e-08 &  1.05e-15 &      \\ 
     &           &    9 &  3.68e-08 &  1.17e-15 &      \\ 
     &           &   10 &  4.09e-08 &  1.05e-15 &      \\ 
1397 &  1.40e+04 &   10 &           &           & iters  \\ 
 \hdashline 
     &           &    1 &  4.84e-08 &  5.26e-11 &      \\ 
     &           &    2 &  2.93e-08 &  1.38e-15 &      \\ 
     &           &    3 &  4.84e-08 &  8.37e-16 &      \\ 
     &           &    4 &  2.93e-08 &  1.38e-15 &      \\ 
     &           &    5 &  2.93e-08 &  8.37e-16 &      \\ 
     &           &    6 &  4.84e-08 &  8.36e-16 &      \\ 
     &           &    7 &  2.93e-08 &  1.38e-15 &      \\ 
     &           &    8 &  2.93e-08 &  8.37e-16 &      \\ 
     &           &    9 &  4.84e-08 &  8.36e-16 &      \\ 
     &           &   10 &  2.93e-08 &  1.38e-15 &      \\ 
1398 &  1.40e+04 &   10 &           &           & iters  \\ 
 \hdashline 
     &           &    1 &  2.72e-08 &  2.38e-09 &      \\ 
     &           &    2 &  2.71e-08 &  7.75e-16 &      \\ 
     &           &    3 &  5.06e-08 &  7.74e-16 &      \\ 
     &           &    4 &  2.72e-08 &  1.44e-15 &      \\ 
     &           &    5 &  5.06e-08 &  7.75e-16 &      \\ 
     &           &    6 &  2.72e-08 &  1.44e-15 &      \\ 
     &           &    7 &  2.72e-08 &  7.76e-16 &      \\ 
     &           &    8 &  5.06e-08 &  7.75e-16 &      \\ 
     &           &    9 &  2.72e-08 &  1.44e-15 &      \\ 
     &           &   10 &  2.71e-08 &  7.76e-16 &      \\ 
1399 &  1.40e+04 &   10 &           &           & iters  \\ 
 \hdashline 
     &           &    1 &  2.26e-09 &  2.15e-09 &      \\ 
     &           &    2 &  7.54e-08 &  6.46e-17 &      \\ 
     &           &    3 &  8.01e-08 &  2.15e-15 &      \\ 
     &           &    4 &  7.54e-08 &  2.28e-15 &      \\ 
     &           &    5 &  8.00e-08 &  2.15e-15 &      \\ 
     &           &    6 &  7.54e-08 &  2.28e-15 &      \\ 
     &           &    7 &  8.00e-08 &  2.15e-15 &      \\ 
     &           &    8 &  7.54e-08 &  2.28e-15 &      \\ 
     &           &    9 &  2.35e-09 &  2.15e-15 &      \\ 
     &           &   10 &  2.34e-09 &  6.70e-17 &      \\ 
1400 &  1.40e+04 &   10 &           &           & iters  \\ 
 \hdashline 
     &           &    1 &  1.82e-08 &  2.73e-09 &      \\ 
     &           &    2 &  1.82e-08 &  5.20e-16 &      \\ 
     &           &    3 &  1.82e-08 &  5.19e-16 &      \\ 
     &           &    4 &  5.96e-08 &  5.18e-16 &      \\ 
     &           &    5 &  1.82e-08 &  1.70e-15 &      \\ 
     &           &    6 &  1.82e-08 &  5.20e-16 &      \\ 
     &           &    7 &  1.82e-08 &  5.19e-16 &      \\ 
     &           &    8 &  5.95e-08 &  5.18e-16 &      \\ 
     &           &    9 &  1.82e-08 &  1.70e-15 &      \\ 
     &           &   10 &  1.82e-08 &  5.20e-16 &      \\ 
1401 &  1.40e+04 &   10 &           &           & iters  \\ 
 \hdashline 
     &           &    1 &  2.41e-08 &  2.46e-09 &      \\ 
     &           &    2 &  5.36e-08 &  6.87e-16 &      \\ 
     &           &    3 &  2.41e-08 &  1.53e-15 &      \\ 
     &           &    4 &  2.41e-08 &  6.89e-16 &      \\ 
     &           &    5 &  2.41e-08 &  6.88e-16 &      \\ 
     &           &    6 &  5.37e-08 &  6.87e-16 &      \\ 
     &           &    7 &  2.41e-08 &  1.53e-15 &      \\ 
     &           &    8 &  2.41e-08 &  6.88e-16 &      \\ 
     &           &    9 &  5.37e-08 &  6.87e-16 &      \\ 
     &           &   10 &  2.41e-08 &  1.53e-15 &      \\ 
1402 &  1.40e+04 &   10 &           &           & iters  \\ 
 \hdashline 
     &           &    1 &  3.31e-08 &  2.82e-10 &      \\ 
     &           &    2 &  4.46e-08 &  9.44e-16 &      \\ 
     &           &    3 &  3.31e-08 &  1.27e-15 &      \\ 
     &           &    4 &  4.46e-08 &  9.45e-16 &      \\ 
     &           &    5 &  3.31e-08 &  1.27e-15 &      \\ 
     &           &    6 &  4.46e-08 &  9.45e-16 &      \\ 
     &           &    7 &  3.31e-08 &  1.27e-15 &      \\ 
     &           &    8 &  3.31e-08 &  9.46e-16 &      \\ 
     &           &    9 &  4.46e-08 &  9.45e-16 &      \\ 
     &           &   10 &  3.31e-08 &  1.27e-15 &      \\ 
1403 &  1.40e+04 &   10 &           &           & iters  \\ 
 \hdashline 
     &           &    1 &  5.03e-08 &  6.27e-11 &      \\ 
     &           &    2 &  2.74e-08 &  1.44e-15 &      \\ 
     &           &    3 &  5.03e-08 &  7.83e-16 &      \\ 
     &           &    4 &  2.75e-08 &  1.44e-15 &      \\ 
     &           &    5 &  2.74e-08 &  7.84e-16 &      \\ 
     &           &    6 &  5.03e-08 &  7.83e-16 &      \\ 
     &           &    7 &  2.75e-08 &  1.44e-15 &      \\ 
     &           &    8 &  2.74e-08 &  7.84e-16 &      \\ 
     &           &    9 &  5.03e-08 &  7.83e-16 &      \\ 
     &           &   10 &  2.75e-08 &  1.44e-15 &      \\ 
1404 &  1.40e+04 &   10 &           &           & iters  \\ 
 \hdashline 
     &           &    1 &  2.33e-08 &  2.37e-09 &      \\ 
     &           &    2 &  5.44e-08 &  6.65e-16 &      \\ 
     &           &    3 &  2.34e-08 &  1.55e-15 &      \\ 
     &           &    4 &  2.33e-08 &  6.67e-16 &      \\ 
     &           &    5 &  2.33e-08 &  6.66e-16 &      \\ 
     &           &    6 &  5.44e-08 &  6.65e-16 &      \\ 
     &           &    7 &  2.33e-08 &  1.55e-15 &      \\ 
     &           &    8 &  2.33e-08 &  6.66e-16 &      \\ 
     &           &    9 &  5.44e-08 &  6.65e-16 &      \\ 
     &           &   10 &  2.33e-08 &  1.55e-15 &      \\ 
1405 &  1.40e+04 &   10 &           &           & iters  \\ 
 \hdashline 
     &           &    1 &  1.93e-08 &  1.78e-09 &      \\ 
     &           &    2 &  1.92e-08 &  5.50e-16 &      \\ 
     &           &    3 &  1.92e-08 &  5.49e-16 &      \\ 
     &           &    4 &  5.85e-08 &  5.48e-16 &      \\ 
     &           &    5 &  1.93e-08 &  1.67e-15 &      \\ 
     &           &    6 &  1.92e-08 &  5.50e-16 &      \\ 
     &           &    7 &  1.92e-08 &  5.49e-16 &      \\ 
     &           &    8 &  5.85e-08 &  5.48e-16 &      \\ 
     &           &    9 &  1.93e-08 &  1.67e-15 &      \\ 
     &           &   10 &  1.92e-08 &  5.50e-16 &      \\ 
1406 &  1.40e+04 &   10 &           &           & iters  \\ 
 \hdashline 
     &           &    1 &  3.61e-08 &  7.65e-10 &      \\ 
     &           &    2 &  4.17e-08 &  1.03e-15 &      \\ 
     &           &    3 &  3.61e-08 &  1.19e-15 &      \\ 
     &           &    4 &  4.17e-08 &  1.03e-15 &      \\ 
     &           &    5 &  3.61e-08 &  1.19e-15 &      \\ 
     &           &    6 &  4.17e-08 &  1.03e-15 &      \\ 
     &           &    7 &  3.61e-08 &  1.19e-15 &      \\ 
     &           &    8 &  3.60e-08 &  1.03e-15 &      \\ 
     &           &    9 &  4.17e-08 &  1.03e-15 &      \\ 
     &           &   10 &  3.60e-08 &  1.19e-15 &      \\ 
1407 &  1.41e+04 &   10 &           &           & iters  \\ 
 \hdashline 
     &           &    1 &  4.49e-08 &  1.36e-10 &      \\ 
     &           &    2 &  3.28e-08 &  1.28e-15 &      \\ 
     &           &    3 &  3.28e-08 &  9.37e-16 &      \\ 
     &           &    4 &  4.49e-08 &  9.36e-16 &      \\ 
     &           &    5 &  3.28e-08 &  1.28e-15 &      \\ 
     &           &    6 &  4.49e-08 &  9.36e-16 &      \\ 
     &           &    7 &  3.28e-08 &  1.28e-15 &      \\ 
     &           &    8 &  4.49e-08 &  9.37e-16 &      \\ 
     &           &    9 &  3.28e-08 &  1.28e-15 &      \\ 
     &           &   10 &  3.28e-08 &  9.37e-16 &      \\ 
1408 &  1.41e+04 &   10 &           &           & iters  \\ 
 \hdashline 
     &           &    1 &  9.97e-08 &  8.17e-10 &      \\ 
     &           &    2 &  5.58e-08 &  2.84e-15 &      \\ 
     &           &    3 &  2.19e-08 &  1.59e-15 &      \\ 
     &           &    4 &  5.58e-08 &  6.26e-16 &      \\ 
     &           &    5 &  2.20e-08 &  1.59e-15 &      \\ 
     &           &    6 &  2.19e-08 &  6.27e-16 &      \\ 
     &           &    7 &  2.19e-08 &  6.26e-16 &      \\ 
     &           &    8 &  5.58e-08 &  6.25e-16 &      \\ 
     &           &    9 &  2.20e-08 &  1.59e-15 &      \\ 
     &           &   10 &  2.19e-08 &  6.27e-16 &      \\ 
1409 &  1.41e+04 &   10 &           &           & iters  \\ 
 \hdashline 
     &           &    1 &  7.23e-10 &  2.43e-10 &      \\ 
     &           &    2 &  7.22e-10 &  2.06e-17 &      \\ 
     &           &    3 &  7.21e-10 &  2.06e-17 &      \\ 
     &           &    4 &  7.19e-10 &  2.06e-17 &      \\ 
     &           &    5 &  7.18e-10 &  2.05e-17 &      \\ 
     &           &    6 &  7.17e-10 &  2.05e-17 &      \\ 
     &           &    7 &  7.16e-10 &  2.05e-17 &      \\ 
     &           &    8 &  7.15e-10 &  2.04e-17 &      \\ 
     &           &    9 &  7.14e-10 &  2.04e-17 &      \\ 
     &           &   10 &  7.13e-10 &  2.04e-17 &      \\ 
1410 &  1.41e+04 &   10 &           &           & iters  \\ 
 \hdashline 
     &           &    1 &  2.18e-09 &  1.68e-09 &      \\ 
     &           &    2 &  2.18e-09 &  6.22e-17 &      \\ 
     &           &    3 &  2.17e-09 &  6.21e-17 &      \\ 
     &           &    4 &  2.17e-09 &  6.20e-17 &      \\ 
     &           &    5 &  2.17e-09 &  6.20e-17 &      \\ 
     &           &    6 &  2.16e-09 &  6.19e-17 &      \\ 
     &           &    7 &  2.16e-09 &  6.18e-17 &      \\ 
     &           &    8 &  2.16e-09 &  6.17e-17 &      \\ 
     &           &    9 &  2.16e-09 &  6.16e-17 &      \\ 
     &           &   10 &  2.15e-09 &  6.15e-17 &      \\ 
1411 &  1.41e+04 &   10 &           &           & iters  \\ 
 \hdashline 
     &           &    1 &  2.60e-08 &  4.13e-09 &      \\ 
     &           &    2 &  5.17e-08 &  7.43e-16 &      \\ 
     &           &    3 &  2.61e-08 &  1.48e-15 &      \\ 
     &           &    4 &  2.60e-08 &  7.44e-16 &      \\ 
     &           &    5 &  5.17e-08 &  7.43e-16 &      \\ 
     &           &    6 &  2.61e-08 &  1.48e-15 &      \\ 
     &           &    7 &  2.60e-08 &  7.44e-16 &      \\ 
     &           &    8 &  5.17e-08 &  7.43e-16 &      \\ 
     &           &    9 &  2.61e-08 &  1.48e-15 &      \\ 
     &           &   10 &  2.60e-08 &  7.44e-16 &      \\ 
1412 &  1.41e+04 &   10 &           &           & iters  \\ 
 \hdashline 
     &           &    1 &  7.35e-09 &  5.12e-09 &      \\ 
     &           &    2 &  7.34e-09 &  2.10e-16 &      \\ 
     &           &    3 &  7.33e-09 &  2.10e-16 &      \\ 
     &           &    4 &  7.32e-09 &  2.09e-16 &      \\ 
     &           &    5 &  7.31e-09 &  2.09e-16 &      \\ 
     &           &    6 &  7.30e-09 &  2.09e-16 &      \\ 
     &           &    7 &  7.29e-09 &  2.08e-16 &      \\ 
     &           &    8 &  7.28e-09 &  2.08e-16 &      \\ 
     &           &    9 &  7.27e-09 &  2.08e-16 &      \\ 
     &           &   10 &  7.26e-09 &  2.08e-16 &      \\ 
1413 &  1.41e+04 &   10 &           &           & iters  \\ 
 \hdashline 
     &           &    1 &  3.70e-08 &  6.78e-10 &      \\ 
     &           &    2 &  3.69e-08 &  1.06e-15 &      \\ 
     &           &    3 &  4.08e-08 &  1.05e-15 &      \\ 
     &           &    4 &  3.69e-08 &  1.16e-15 &      \\ 
     &           &    5 &  4.08e-08 &  1.05e-15 &      \\ 
     &           &    6 &  3.69e-08 &  1.16e-15 &      \\ 
     &           &    7 &  4.08e-08 &  1.05e-15 &      \\ 
     &           &    8 &  3.70e-08 &  1.16e-15 &      \\ 
     &           &    9 &  4.08e-08 &  1.05e-15 &      \\ 
     &           &   10 &  3.70e-08 &  1.16e-15 &      \\ 
1414 &  1.41e+04 &   10 &           &           & iters  \\ 
 \hdashline 
     &           &    1 &  8.89e-09 &  1.51e-09 &      \\ 
     &           &    2 &  8.88e-09 &  2.54e-16 &      \\ 
     &           &    3 &  8.87e-09 &  2.53e-16 &      \\ 
     &           &    4 &  8.85e-09 &  2.53e-16 &      \\ 
     &           &    5 &  8.84e-09 &  2.53e-16 &      \\ 
     &           &    6 &  8.83e-09 &  2.52e-16 &      \\ 
     &           &    7 &  8.82e-09 &  2.52e-16 &      \\ 
     &           &    8 &  8.80e-09 &  2.52e-16 &      \\ 
     &           &    9 &  8.79e-09 &  2.51e-16 &      \\ 
     &           &   10 &  8.78e-09 &  2.51e-16 &      \\ 
1415 &  1.41e+04 &   10 &           &           & iters  \\ 
 \hdashline 
     &           &    1 &  8.70e-08 &  2.68e-09 &      \\ 
     &           &    2 &  6.85e-08 &  2.48e-15 &      \\ 
     &           &    3 &  9.30e-09 &  1.95e-15 &      \\ 
     &           &    4 &  9.28e-09 &  2.65e-16 &      \\ 
     &           &    5 &  9.27e-09 &  2.65e-16 &      \\ 
     &           &    6 &  9.26e-09 &  2.65e-16 &      \\ 
     &           &    7 &  9.24e-09 &  2.64e-16 &      \\ 
     &           &    8 &  6.85e-08 &  2.64e-16 &      \\ 
     &           &    9 &  8.70e-08 &  1.95e-15 &      \\ 
     &           &   10 &  6.85e-08 &  2.48e-15 &      \\ 
1416 &  1.42e+04 &   10 &           &           & iters  \\ 
 \hdashline 
     &           &    1 &  4.55e-08 &  3.67e-09 &      \\ 
     &           &    2 &  3.22e-08 &  1.30e-15 &      \\ 
     &           &    3 &  4.55e-08 &  9.19e-16 &      \\ 
     &           &    4 &  3.22e-08 &  1.30e-15 &      \\ 
     &           &    5 &  4.55e-08 &  9.20e-16 &      \\ 
     &           &    6 &  3.22e-08 &  1.30e-15 &      \\ 
     &           &    7 &  3.22e-08 &  9.20e-16 &      \\ 
     &           &    8 &  4.55e-08 &  9.19e-16 &      \\ 
     &           &    9 &  3.22e-08 &  1.30e-15 &      \\ 
     &           &   10 &  4.55e-08 &  9.20e-16 &      \\ 
1417 &  1.42e+04 &   10 &           &           & iters  \\ 
 \hdashline 
     &           &    1 &  4.79e-08 &  1.27e-09 &      \\ 
     &           &    2 &  2.98e-08 &  1.37e-15 &      \\ 
     &           &    3 &  2.98e-08 &  8.52e-16 &      \\ 
     &           &    4 &  4.79e-08 &  8.50e-16 &      \\ 
     &           &    5 &  2.98e-08 &  1.37e-15 &      \\ 
     &           &    6 &  4.79e-08 &  8.51e-16 &      \\ 
     &           &    7 &  2.98e-08 &  1.37e-15 &      \\ 
     &           &    8 &  2.98e-08 &  8.52e-16 &      \\ 
     &           &    9 &  4.79e-08 &  8.51e-16 &      \\ 
     &           &   10 &  2.98e-08 &  1.37e-15 &      \\ 
1418 &  1.42e+04 &   10 &           &           & iters  \\ 
 \hdashline 
     &           &    1 &  1.02e-08 &  2.77e-10 &      \\ 
     &           &    2 &  2.86e-08 &  2.92e-16 &      \\ 
     &           &    3 &  1.03e-08 &  8.17e-16 &      \\ 
     &           &    4 &  1.03e-08 &  2.93e-16 &      \\ 
     &           &    5 &  2.86e-08 &  2.93e-16 &      \\ 
     &           &    6 &  1.03e-08 &  8.16e-16 &      \\ 
     &           &    7 &  1.03e-08 &  2.93e-16 &      \\ 
     &           &    8 &  1.03e-08 &  2.93e-16 &      \\ 
     &           &    9 &  2.86e-08 &  2.93e-16 &      \\ 
     &           &   10 &  1.03e-08 &  8.16e-16 &      \\ 
1419 &  1.42e+04 &   10 &           &           & iters  \\ 
 \hdashline 
     &           &    1 &  3.64e-08 &  3.47e-09 &      \\ 
     &           &    2 &  4.13e-08 &  1.04e-15 &      \\ 
     &           &    3 &  3.64e-08 &  1.18e-15 &      \\ 
     &           &    4 &  4.13e-08 &  1.04e-15 &      \\ 
     &           &    5 &  3.64e-08 &  1.18e-15 &      \\ 
     &           &    6 &  4.13e-08 &  1.04e-15 &      \\ 
     &           &    7 &  3.64e-08 &  1.18e-15 &      \\ 
     &           &    8 &  4.13e-08 &  1.04e-15 &      \\ 
     &           &    9 &  3.64e-08 &  1.18e-15 &      \\ 
     &           &   10 &  4.13e-08 &  1.04e-15 &      \\ 
1420 &  1.42e+04 &   10 &           &           & iters  \\ 
 \hdashline 
     &           &    1 &  3.29e-08 &  6.62e-09 &      \\ 
     &           &    2 &  4.49e-08 &  9.38e-16 &      \\ 
     &           &    3 &  3.29e-08 &  1.28e-15 &      \\ 
     &           &    4 &  4.48e-08 &  9.39e-16 &      \\ 
     &           &    5 &  3.29e-08 &  1.28e-15 &      \\ 
     &           &    6 &  3.29e-08 &  9.39e-16 &      \\ 
     &           &    7 &  4.49e-08 &  9.38e-16 &      \\ 
     &           &    8 &  3.29e-08 &  1.28e-15 &      \\ 
     &           &    9 &  4.49e-08 &  9.38e-16 &      \\ 
     &           &   10 &  3.29e-08 &  1.28e-15 &      \\ 
1421 &  1.42e+04 &   10 &           &           & iters  \\ 
 \hdashline 
     &           &    1 &  4.00e-08 &  2.73e-09 &      \\ 
     &           &    2 &  3.78e-08 &  1.14e-15 &      \\ 
     &           &    3 &  4.00e-08 &  1.08e-15 &      \\ 
     &           &    4 &  3.78e-08 &  1.14e-15 &      \\ 
     &           &    5 &  4.00e-08 &  1.08e-15 &      \\ 
     &           &    6 &  3.78e-08 &  1.14e-15 &      \\ 
     &           &    7 &  4.00e-08 &  1.08e-15 &      \\ 
     &           &    8 &  3.78e-08 &  1.14e-15 &      \\ 
     &           &    9 &  4.00e-08 &  1.08e-15 &      \\ 
     &           &   10 &  3.78e-08 &  1.14e-15 &      \\ 
1422 &  1.42e+04 &   10 &           &           & iters  \\ 
 \hdashline 
     &           &    1 &  4.04e-08 &  8.47e-10 &      \\ 
     &           &    2 &  3.74e-08 &  1.15e-15 &      \\ 
     &           &    3 &  4.04e-08 &  1.07e-15 &      \\ 
     &           &    4 &  3.74e-08 &  1.15e-15 &      \\ 
     &           &    5 &  4.04e-08 &  1.07e-15 &      \\ 
     &           &    6 &  3.74e-08 &  1.15e-15 &      \\ 
     &           &    7 &  3.73e-08 &  1.07e-15 &      \\ 
     &           &    8 &  4.04e-08 &  1.06e-15 &      \\ 
     &           &    9 &  3.73e-08 &  1.15e-15 &      \\ 
     &           &   10 &  4.04e-08 &  1.07e-15 &      \\ 
1423 &  1.42e+04 &   10 &           &           & iters  \\ 
 \hdashline 
     &           &    1 &  3.01e-08 &  1.29e-09 &      \\ 
     &           &    2 &  4.76e-08 &  8.60e-16 &      \\ 
     &           &    3 &  3.02e-08 &  1.36e-15 &      \\ 
     &           &    4 &  3.01e-08 &  8.61e-16 &      \\ 
     &           &    5 &  4.76e-08 &  8.60e-16 &      \\ 
     &           &    6 &  3.01e-08 &  1.36e-15 &      \\ 
     &           &    7 &  4.76e-08 &  8.60e-16 &      \\ 
     &           &    8 &  3.02e-08 &  1.36e-15 &      \\ 
     &           &    9 &  3.01e-08 &  8.61e-16 &      \\ 
     &           &   10 &  4.76e-08 &  8.60e-16 &      \\ 
1424 &  1.42e+04 &   10 &           &           & iters  \\ 
 \hdashline 
     &           &    1 &  5.34e-08 &  1.05e-09 &      \\ 
     &           &    2 &  2.44e-08 &  1.52e-15 &      \\ 
     &           &    3 &  2.44e-08 &  6.96e-16 &      \\ 
     &           &    4 &  2.43e-08 &  6.95e-16 &      \\ 
     &           &    5 &  5.34e-08 &  6.94e-16 &      \\ 
     &           &    6 &  2.44e-08 &  1.52e-15 &      \\ 
     &           &    7 &  2.43e-08 &  6.95e-16 &      \\ 
     &           &    8 &  5.34e-08 &  6.94e-16 &      \\ 
     &           &    9 &  2.44e-08 &  1.52e-15 &      \\ 
     &           &   10 &  2.43e-08 &  6.96e-16 &      \\ 
1425 &  1.42e+04 &   10 &           &           & iters  \\ 
 \hdashline 
     &           &    1 &  5.52e-08 &  1.10e-09 &      \\ 
     &           &    2 &  2.25e-08 &  1.58e-15 &      \\ 
     &           &    3 &  2.25e-08 &  6.43e-16 &      \\ 
     &           &    4 &  5.52e-08 &  6.42e-16 &      \\ 
     &           &    5 &  2.26e-08 &  1.58e-15 &      \\ 
     &           &    6 &  2.25e-08 &  6.44e-16 &      \\ 
     &           &    7 &  5.52e-08 &  6.43e-16 &      \\ 
     &           &    8 &  2.26e-08 &  1.58e-15 &      \\ 
     &           &    9 &  2.25e-08 &  6.44e-16 &      \\ 
     &           &   10 &  2.25e-08 &  6.43e-16 &      \\ 
1426 &  1.42e+04 &   10 &           &           & iters  \\ 
 \hdashline 
     &           &    1 &  5.67e-08 &  1.70e-09 &      \\ 
     &           &    2 &  2.11e-08 &  1.62e-15 &      \\ 
     &           &    3 &  2.11e-08 &  6.02e-16 &      \\ 
     &           &    4 &  5.66e-08 &  6.01e-16 &      \\ 
     &           &    5 &  2.11e-08 &  1.62e-15 &      \\ 
     &           &    6 &  2.11e-08 &  6.03e-16 &      \\ 
     &           &    7 &  5.66e-08 &  6.02e-16 &      \\ 
     &           &    8 &  2.11e-08 &  1.62e-15 &      \\ 
     &           &    9 &  2.11e-08 &  6.03e-16 &      \\ 
     &           &   10 &  2.11e-08 &  6.03e-16 &      \\ 
1427 &  1.43e+04 &   10 &           &           & iters  \\ 
 \hdashline 
     &           &    1 &  1.37e-09 &  1.21e-09 &      \\ 
     &           &    2 &  1.37e-09 &  3.91e-17 &      \\ 
     &           &    3 &  1.37e-09 &  3.91e-17 &      \\ 
     &           &    4 &  1.37e-09 &  3.90e-17 &      \\ 
     &           &    5 &  1.36e-09 &  3.90e-17 &      \\ 
     &           &    6 &  1.36e-09 &  3.89e-17 &      \\ 
     &           &    7 &  1.36e-09 &  3.89e-17 &      \\ 
     &           &    8 &  1.36e-09 &  3.88e-17 &      \\ 
     &           &    9 &  1.36e-09 &  3.87e-17 &      \\ 
     &           &   10 &  1.35e-09 &  3.87e-17 &      \\ 
1428 &  1.43e+04 &   10 &           &           & iters  \\ 
 \hdashline 
     &           &    1 &  3.54e-08 &  2.15e-09 &      \\ 
     &           &    2 &  3.54e-08 &  1.01e-15 &      \\ 
     &           &    3 &  4.24e-08 &  1.01e-15 &      \\ 
     &           &    4 &  3.54e-08 &  1.21e-15 &      \\ 
     &           &    5 &  4.24e-08 &  1.01e-15 &      \\ 
     &           &    6 &  3.54e-08 &  1.21e-15 &      \\ 
     &           &    7 &  4.24e-08 &  1.01e-15 &      \\ 
     &           &    8 &  3.54e-08 &  1.21e-15 &      \\ 
     &           &    9 &  4.23e-08 &  1.01e-15 &      \\ 
     &           &   10 &  3.54e-08 &  1.21e-15 &      \\ 
1429 &  1.43e+04 &   10 &           &           & iters  \\ 
 \hdashline 
     &           &    1 &  7.25e-08 &  8.87e-10 &      \\ 
     &           &    2 &  8.30e-08 &  2.07e-15 &      \\ 
     &           &    3 &  7.25e-08 &  2.37e-15 &      \\ 
     &           &    4 &  5.33e-09 &  2.07e-15 &      \\ 
     &           &    5 &  5.32e-09 &  1.52e-16 &      \\ 
     &           &    6 &  5.31e-09 &  1.52e-16 &      \\ 
     &           &    7 &  5.30e-09 &  1.52e-16 &      \\ 
     &           &    8 &  5.30e-09 &  1.51e-16 &      \\ 
     &           &    9 &  5.29e-09 &  1.51e-16 &      \\ 
     &           &   10 &  5.28e-09 &  1.51e-16 &      \\ 
1430 &  1.43e+04 &   10 &           &           & iters  \\ 
 \hdashline 
     &           &    1 &  9.05e-09 &  5.47e-09 &      \\ 
     &           &    2 &  9.04e-09 &  2.58e-16 &      \\ 
     &           &    3 &  6.87e-08 &  2.58e-16 &      \\ 
     &           &    4 &  8.68e-08 &  1.96e-15 &      \\ 
     &           &    5 &  6.87e-08 &  2.48e-15 &      \\ 
     &           &    6 &  9.09e-09 &  1.96e-15 &      \\ 
     &           &    7 &  9.08e-09 &  2.60e-16 &      \\ 
     &           &    8 &  9.07e-09 &  2.59e-16 &      \\ 
     &           &    9 &  9.06e-09 &  2.59e-16 &      \\ 
     &           &   10 &  9.04e-09 &  2.58e-16 &      \\ 
1431 &  1.43e+04 &   10 &           &           & iters  \\ 
 \hdashline 
     &           &    1 &  2.33e-08 &  1.26e-09 &      \\ 
     &           &    2 &  2.33e-08 &  6.65e-16 &      \\ 
     &           &    3 &  5.45e-08 &  6.64e-16 &      \\ 
     &           &    4 &  2.33e-08 &  1.55e-15 &      \\ 
     &           &    5 &  2.33e-08 &  6.65e-16 &      \\ 
     &           &    6 &  2.32e-08 &  6.64e-16 &      \\ 
     &           &    7 &  5.45e-08 &  6.63e-16 &      \\ 
     &           &    8 &  2.33e-08 &  1.56e-15 &      \\ 
     &           &    9 &  2.32e-08 &  6.64e-16 &      \\ 
     &           &   10 &  5.45e-08 &  6.64e-16 &      \\ 
1432 &  1.43e+04 &   10 &           &           & iters  \\ 
 \hdashline 
     &           &    1 &  3.77e-08 &  4.90e-09 &      \\ 
     &           &    2 &  4.00e-08 &  1.08e-15 &      \\ 
     &           &    3 &  3.77e-08 &  1.14e-15 &      \\ 
     &           &    4 &  4.00e-08 &  1.08e-15 &      \\ 
     &           &    5 &  3.77e-08 &  1.14e-15 &      \\ 
     &           &    6 &  4.00e-08 &  1.08e-15 &      \\ 
     &           &    7 &  3.77e-08 &  1.14e-15 &      \\ 
     &           &    8 &  4.00e-08 &  1.08e-15 &      \\ 
     &           &    9 &  3.77e-08 &  1.14e-15 &      \\ 
     &           &   10 &  4.00e-08 &  1.08e-15 &      \\ 
1433 &  1.43e+04 &   10 &           &           & iters  \\ 
 \hdashline 
     &           &    1 &  1.23e-08 &  3.98e-09 &      \\ 
     &           &    2 &  1.23e-08 &  3.51e-16 &      \\ 
     &           &    3 &  1.23e-08 &  3.51e-16 &      \\ 
     &           &    4 &  6.54e-08 &  3.50e-16 &      \\ 
     &           &    5 &  1.23e-08 &  1.87e-15 &      \\ 
     &           &    6 &  1.23e-08 &  3.52e-16 &      \\ 
     &           &    7 &  1.23e-08 &  3.52e-16 &      \\ 
     &           &    8 &  1.23e-08 &  3.51e-16 &      \\ 
     &           &    9 &  1.23e-08 &  3.51e-16 &      \\ 
     &           &   10 &  6.54e-08 &  3.50e-16 &      \\ 
1434 &  1.43e+04 &   10 &           &           & iters  \\ 
 \hdashline 
     &           &    1 &  2.60e-08 &  1.50e-09 &      \\ 
     &           &    2 &  5.17e-08 &  7.43e-16 &      \\ 
     &           &    3 &  2.61e-08 &  1.47e-15 &      \\ 
     &           &    4 &  2.60e-08 &  7.44e-16 &      \\ 
     &           &    5 &  5.17e-08 &  7.43e-16 &      \\ 
     &           &    6 &  2.61e-08 &  1.47e-15 &      \\ 
     &           &    7 &  2.60e-08 &  7.44e-16 &      \\ 
     &           &    8 &  5.17e-08 &  7.43e-16 &      \\ 
     &           &    9 &  2.61e-08 &  1.48e-15 &      \\ 
     &           &   10 &  2.60e-08 &  7.44e-16 &      \\ 
1435 &  1.43e+04 &   10 &           &           & iters  \\ 
 \hdashline 
     &           &    1 &  4.32e-08 &  8.52e-10 &      \\ 
     &           &    2 &  3.45e-08 &  1.23e-15 &      \\ 
     &           &    3 &  4.32e-08 &  9.86e-16 &      \\ 
     &           &    4 &  3.45e-08 &  1.23e-15 &      \\ 
     &           &    5 &  4.32e-08 &  9.86e-16 &      \\ 
     &           &    6 &  3.46e-08 &  1.23e-15 &      \\ 
     &           &    7 &  4.32e-08 &  9.86e-16 &      \\ 
     &           &    8 &  3.46e-08 &  1.23e-15 &      \\ 
     &           &    9 &  4.32e-08 &  9.87e-16 &      \\ 
     &           &   10 &  3.46e-08 &  1.23e-15 &      \\ 
1436 &  1.44e+04 &   10 &           &           & iters  \\ 
 \hdashline 
     &           &    1 &  1.43e-08 &  3.95e-09 &      \\ 
     &           &    2 &  6.34e-08 &  4.09e-16 &      \\ 
     &           &    3 &  1.44e-08 &  1.81e-15 &      \\ 
     &           &    4 &  1.44e-08 &  4.11e-16 &      \\ 
     &           &    5 &  1.43e-08 &  4.10e-16 &      \\ 
     &           &    6 &  1.43e-08 &  4.09e-16 &      \\ 
     &           &    7 &  1.43e-08 &  4.09e-16 &      \\ 
     &           &    8 &  6.34e-08 &  4.08e-16 &      \\ 
     &           &    9 &  1.44e-08 &  1.81e-15 &      \\ 
     &           &   10 &  1.44e-08 &  4.10e-16 &      \\ 
1437 &  1.44e+04 &   10 &           &           & iters  \\ 
 \hdashline 
     &           &    1 &  1.32e-09 &  4.25e-09 &      \\ 
     &           &    2 &  1.32e-09 &  3.77e-17 &      \\ 
     &           &    3 &  1.32e-09 &  3.77e-17 &      \\ 
     &           &    4 &  1.32e-09 &  3.76e-17 &      \\ 
     &           &    5 &  1.31e-09 &  3.75e-17 &      \\ 
     &           &    6 &  1.31e-09 &  3.75e-17 &      \\ 
     &           &    7 &  7.64e-08 &  3.74e-17 &      \\ 
     &           &    8 &  7.91e-08 &  2.18e-15 &      \\ 
     &           &    9 &  7.64e-08 &  2.26e-15 &      \\ 
     &           &   10 &  7.91e-08 &  2.18e-15 &      \\ 
1438 &  1.44e+04 &   10 &           &           & iters  \\ 
 \hdashline 
     &           &    1 &  4.21e-08 &  2.13e-09 &      \\ 
     &           &    2 &  4.21e-08 &  1.20e-15 &      \\ 
     &           &    3 &  3.57e-08 &  1.20e-15 &      \\ 
     &           &    4 &  4.21e-08 &  1.02e-15 &      \\ 
     &           &    5 &  3.57e-08 &  1.20e-15 &      \\ 
     &           &    6 &  4.21e-08 &  1.02e-15 &      \\ 
     &           &    7 &  3.57e-08 &  1.20e-15 &      \\ 
     &           &    8 &  3.56e-08 &  1.02e-15 &      \\ 
     &           &    9 &  4.21e-08 &  1.02e-15 &      \\ 
     &           &   10 &  3.56e-08 &  1.20e-15 &      \\ 
1439 &  1.44e+04 &   10 &           &           & iters  \\ 
 \hdashline 
     &           &    1 &  3.11e-08 &  2.12e-09 &      \\ 
     &           &    2 &  4.67e-08 &  8.87e-16 &      \\ 
     &           &    3 &  3.11e-08 &  1.33e-15 &      \\ 
     &           &    4 &  3.10e-08 &  8.87e-16 &      \\ 
     &           &    5 &  4.67e-08 &  8.86e-16 &      \\ 
     &           &    6 &  3.11e-08 &  1.33e-15 &      \\ 
     &           &    7 &  4.67e-08 &  8.87e-16 &      \\ 
     &           &    8 &  3.11e-08 &  1.33e-15 &      \\ 
     &           &    9 &  3.10e-08 &  8.87e-16 &      \\ 
     &           &   10 &  4.67e-08 &  8.86e-16 &      \\ 
1440 &  1.44e+04 &   10 &           &           & iters  \\ 
 \hdashline 
     &           &    1 &  4.09e-08 &  2.52e-09 &      \\ 
     &           &    2 &  3.68e-08 &  1.17e-15 &      \\ 
     &           &    3 &  4.09e-08 &  1.05e-15 &      \\ 
     &           &    4 &  3.69e-08 &  1.17e-15 &      \\ 
     &           &    5 &  4.09e-08 &  1.05e-15 &      \\ 
     &           &    6 &  3.69e-08 &  1.17e-15 &      \\ 
     &           &    7 &  3.68e-08 &  1.05e-15 &      \\ 
     &           &    8 &  4.09e-08 &  1.05e-15 &      \\ 
     &           &    9 &  3.68e-08 &  1.17e-15 &      \\ 
     &           &   10 &  4.09e-08 &  1.05e-15 &      \\ 
1441 &  1.44e+04 &   10 &           &           & iters  \\ 
 \hdashline 
     &           &    1 &  4.57e-08 &  9.82e-11 &      \\ 
     &           &    2 &  3.20e-08 &  1.30e-15 &      \\ 
     &           &    3 &  4.57e-08 &  9.14e-16 &      \\ 
     &           &    4 &  3.21e-08 &  1.30e-15 &      \\ 
     &           &    5 &  3.20e-08 &  9.15e-16 &      \\ 
     &           &    6 &  4.57e-08 &  9.14e-16 &      \\ 
     &           &    7 &  3.20e-08 &  1.30e-15 &      \\ 
     &           &    8 &  4.57e-08 &  9.14e-16 &      \\ 
     &           &    9 &  3.21e-08 &  1.30e-15 &      \\ 
     &           &   10 &  3.20e-08 &  9.15e-16 &      \\ 
1442 &  1.44e+04 &   10 &           &           & iters  \\ 
 \hdashline 
     &           &    1 &  9.90e-10 &  1.80e-09 &      \\ 
     &           &    2 &  9.89e-10 &  2.83e-17 &      \\ 
     &           &    3 &  9.88e-10 &  2.82e-17 &      \\ 
     &           &    4 &  9.86e-10 &  2.82e-17 &      \\ 
     &           &    5 &  9.85e-10 &  2.81e-17 &      \\ 
     &           &    6 &  9.83e-10 &  2.81e-17 &      \\ 
     &           &    7 &  9.82e-10 &  2.81e-17 &      \\ 
     &           &    8 &  9.81e-10 &  2.80e-17 &      \\ 
     &           &    9 &  9.79e-10 &  2.80e-17 &      \\ 
     &           &   10 &  9.78e-10 &  2.79e-17 &      \\ 
1443 &  1.44e+04 &   10 &           &           & iters  \\ 
 \hdashline 
     &           &    1 &  3.65e-09 &  7.55e-10 &      \\ 
     &           &    2 &  3.64e-09 &  1.04e-16 &      \\ 
     &           &    3 &  3.64e-09 &  1.04e-16 &      \\ 
     &           &    4 &  3.63e-09 &  1.04e-16 &      \\ 
     &           &    5 &  3.63e-09 &  1.04e-16 &      \\ 
     &           &    6 &  3.62e-09 &  1.03e-16 &      \\ 
     &           &    7 &  3.62e-09 &  1.03e-16 &      \\ 
     &           &    8 &  3.61e-09 &  1.03e-16 &      \\ 
     &           &    9 &  3.60e-09 &  1.03e-16 &      \\ 
     &           &   10 &  3.60e-09 &  1.03e-16 &      \\ 
1444 &  1.44e+04 &   10 &           &           & iters  \\ 
 \hdashline 
     &           &    1 &  1.42e-08 &  4.19e-10 &      \\ 
     &           &    2 &  1.42e-08 &  4.06e-16 &      \\ 
     &           &    3 &  1.42e-08 &  4.06e-16 &      \\ 
     &           &    4 &  1.42e-08 &  4.05e-16 &      \\ 
     &           &    5 &  6.35e-08 &  4.05e-16 &      \\ 
     &           &    6 &  1.42e-08 &  1.81e-15 &      \\ 
     &           &    7 &  1.42e-08 &  4.07e-16 &      \\ 
     &           &    8 &  1.42e-08 &  4.06e-16 &      \\ 
     &           &    9 &  1.42e-08 &  4.06e-16 &      \\ 
     &           &   10 &  1.42e-08 &  4.05e-16 &      \\ 
1445 &  1.44e+04 &   10 &           &           & iters  \\ 
 \hdashline 
     &           &    1 &  3.56e-08 &  4.55e-11 &      \\ 
     &           &    2 &  3.55e-08 &  1.02e-15 &      \\ 
     &           &    3 &  4.22e-08 &  1.01e-15 &      \\ 
     &           &    4 &  3.56e-08 &  1.20e-15 &      \\ 
     &           &    5 &  4.22e-08 &  1.01e-15 &      \\ 
     &           &    6 &  3.56e-08 &  1.20e-15 &      \\ 
     &           &    7 &  4.22e-08 &  1.02e-15 &      \\ 
     &           &    8 &  3.56e-08 &  1.20e-15 &      \\ 
     &           &    9 &  4.22e-08 &  1.02e-15 &      \\ 
     &           &   10 &  3.56e-08 &  1.20e-15 &      \\ 
1446 &  1.44e+04 &   10 &           &           & iters  \\ 
 \hdashline 
     &           &    1 &  7.10e-10 &  2.37e-09 &      \\ 
     &           &    2 &  7.09e-10 &  2.03e-17 &      \\ 
     &           &    3 &  7.08e-10 &  2.02e-17 &      \\ 
     &           &    4 &  7.07e-10 &  2.02e-17 &      \\ 
     &           &    5 &  7.06e-10 &  2.02e-17 &      \\ 
     &           &    6 &  7.05e-10 &  2.01e-17 &      \\ 
     &           &    7 &  7.04e-10 &  2.01e-17 &      \\ 
     &           &    8 &  7.03e-10 &  2.01e-17 &      \\ 
     &           &    9 &  7.02e-10 &  2.01e-17 &      \\ 
     &           &   10 &  7.01e-10 &  2.00e-17 &      \\ 
1447 &  1.45e+04 &   10 &           &           & iters  \\ 
 \hdashline 
     &           &    1 &  2.12e-08 &  3.60e-09 &      \\ 
     &           &    2 &  5.66e-08 &  6.04e-16 &      \\ 
     &           &    3 &  2.12e-08 &  1.61e-15 &      \\ 
     &           &    4 &  2.12e-08 &  6.05e-16 &      \\ 
     &           &    5 &  5.65e-08 &  6.04e-16 &      \\ 
     &           &    6 &  2.12e-08 &  1.61e-15 &      \\ 
     &           &    7 &  2.12e-08 &  6.06e-16 &      \\ 
     &           &    8 &  2.12e-08 &  6.05e-16 &      \\ 
     &           &    9 &  5.66e-08 &  6.04e-16 &      \\ 
     &           &   10 &  2.12e-08 &  1.61e-15 &      \\ 
1448 &  1.45e+04 &   10 &           &           & iters  \\ 
 \hdashline 
     &           &    1 &  3.20e-08 &  2.15e-09 &      \\ 
     &           &    2 &  3.20e-08 &  9.14e-16 &      \\ 
     &           &    3 &  4.58e-08 &  9.13e-16 &      \\ 
     &           &    4 &  3.20e-08 &  1.31e-15 &      \\ 
     &           &    5 &  3.20e-08 &  9.13e-16 &      \\ 
     &           &    6 &  4.58e-08 &  9.12e-16 &      \\ 
     &           &    7 &  3.20e-08 &  1.31e-15 &      \\ 
     &           &    8 &  4.58e-08 &  9.13e-16 &      \\ 
     &           &    9 &  3.20e-08 &  1.31e-15 &      \\ 
     &           &   10 &  3.20e-08 &  9.13e-16 &      \\ 
1449 &  1.45e+04 &   10 &           &           & iters  \\ 
 \hdashline 
     &           &    1 &  2.45e-08 &  6.77e-10 &      \\ 
     &           &    2 &  2.44e-08 &  6.99e-16 &      \\ 
     &           &    3 &  2.44e-08 &  6.98e-16 &      \\ 
     &           &    4 &  2.44e-08 &  6.97e-16 &      \\ 
     &           &    5 &  5.34e-08 &  6.96e-16 &      \\ 
     &           &    6 &  2.44e-08 &  1.52e-15 &      \\ 
     &           &    7 &  2.44e-08 &  6.97e-16 &      \\ 
     &           &    8 &  2.43e-08 &  6.96e-16 &      \\ 
     &           &    9 &  5.34e-08 &  6.95e-16 &      \\ 
     &           &   10 &  2.44e-08 &  1.52e-15 &      \\ 
1450 &  1.45e+04 &   10 &           &           & iters  \\ 
 \hdashline 
     &           &    1 &  6.11e-08 &  3.88e-09 &      \\ 
     &           &    2 &  1.67e-08 &  1.74e-15 &      \\ 
     &           &    3 &  1.66e-08 &  4.76e-16 &      \\ 
     &           &    4 &  6.11e-08 &  4.75e-16 &      \\ 
     &           &    5 &  1.67e-08 &  1.74e-15 &      \\ 
     &           &    6 &  1.67e-08 &  4.77e-16 &      \\ 
     &           &    7 &  1.67e-08 &  4.76e-16 &      \\ 
     &           &    8 &  1.66e-08 &  4.75e-16 &      \\ 
     &           &    9 &  6.11e-08 &  4.75e-16 &      \\ 
     &           &   10 &  1.67e-08 &  1.74e-15 &      \\ 
1451 &  1.45e+04 &   10 &           &           & iters  \\ 
 \hdashline 
     &           &    1 &  2.86e-08 &  3.63e-09 &      \\ 
     &           &    2 &  2.85e-08 &  8.16e-16 &      \\ 
     &           &    3 &  4.92e-08 &  8.14e-16 &      \\ 
     &           &    4 &  2.86e-08 &  1.40e-15 &      \\ 
     &           &    5 &  2.85e-08 &  8.15e-16 &      \\ 
     &           &    6 &  4.92e-08 &  8.14e-16 &      \\ 
     &           &    7 &  2.86e-08 &  1.40e-15 &      \\ 
     &           &    8 &  2.85e-08 &  8.15e-16 &      \\ 
     &           &    9 &  4.92e-08 &  8.14e-16 &      \\ 
     &           &   10 &  2.85e-08 &  1.40e-15 &      \\ 
1452 &  1.45e+04 &   10 &           &           & iters  \\ 
 \hdashline 
     &           &    1 &  3.78e-08 &  1.30e-10 &      \\ 
     &           &    2 &  4.00e-08 &  1.08e-15 &      \\ 
     &           &    3 &  3.78e-08 &  1.14e-15 &      \\ 
     &           &    4 &  4.00e-08 &  1.08e-15 &      \\ 
     &           &    5 &  3.78e-08 &  1.14e-15 &      \\ 
     &           &    6 &  4.00e-08 &  1.08e-15 &      \\ 
     &           &    7 &  3.78e-08 &  1.14e-15 &      \\ 
     &           &    8 &  3.99e-08 &  1.08e-15 &      \\ 
     &           &    9 &  3.78e-08 &  1.14e-15 &      \\ 
     &           &   10 &  3.99e-08 &  1.08e-15 &      \\ 
1453 &  1.45e+04 &   10 &           &           & iters  \\ 
 \hdashline 
     &           &    1 &  3.45e-08 &  1.58e-09 &      \\ 
     &           &    2 &  4.32e-08 &  9.85e-16 &      \\ 
     &           &    3 &  3.45e-08 &  1.23e-15 &      \\ 
     &           &    4 &  4.32e-08 &  9.85e-16 &      \\ 
     &           &    5 &  3.45e-08 &  1.23e-15 &      \\ 
     &           &    6 &  4.32e-08 &  9.85e-16 &      \\ 
     &           &    7 &  3.45e-08 &  1.23e-15 &      \\ 
     &           &    8 &  4.32e-08 &  9.86e-16 &      \\ 
     &           &    9 &  3.46e-08 &  1.23e-15 &      \\ 
     &           &   10 &  3.45e-08 &  9.86e-16 &      \\ 
1454 &  1.45e+04 &   10 &           &           & iters  \\ 
 \hdashline 
     &           &    1 &  9.69e-09 &  4.29e-10 &      \\ 
     &           &    2 &  9.68e-09 &  2.77e-16 &      \\ 
     &           &    3 &  9.66e-09 &  2.76e-16 &      \\ 
     &           &    4 &  9.65e-09 &  2.76e-16 &      \\ 
     &           &    5 &  9.64e-09 &  2.75e-16 &      \\ 
     &           &    6 &  9.62e-09 &  2.75e-16 &      \\ 
     &           &    7 &  9.61e-09 &  2.75e-16 &      \\ 
     &           &    8 &  6.81e-08 &  2.74e-16 &      \\ 
     &           &    9 &  9.69e-09 &  1.94e-15 &      \\ 
     &           &   10 &  9.68e-09 &  2.77e-16 &      \\ 
1455 &  1.45e+04 &   10 &           &           & iters  \\ 
 \hdashline 
     &           &    1 &  2.14e-09 &  1.55e-09 &      \\ 
     &           &    2 &  2.14e-09 &  6.11e-17 &      \\ 
     &           &    3 &  2.14e-09 &  6.11e-17 &      \\ 
     &           &    4 &  2.13e-09 &  6.10e-17 &      \\ 
     &           &    5 &  2.13e-09 &  6.09e-17 &      \\ 
     &           &    6 &  3.67e-08 &  6.08e-17 &      \\ 
     &           &    7 &  2.18e-09 &  1.05e-15 &      \\ 
     &           &    8 &  2.18e-09 &  6.22e-17 &      \\ 
     &           &    9 &  2.17e-09 &  6.21e-17 &      \\ 
     &           &   10 &  2.17e-09 &  6.20e-17 &      \\ 
1456 &  1.46e+04 &   10 &           &           & iters  \\ 
 \hdashline 
     &           &    1 &  2.31e-08 &  2.78e-10 &      \\ 
     &           &    2 &  5.46e-08 &  6.59e-16 &      \\ 
     &           &    3 &  2.31e-08 &  1.56e-15 &      \\ 
     &           &    4 &  2.31e-08 &  6.60e-16 &      \\ 
     &           &    5 &  2.31e-08 &  6.59e-16 &      \\ 
     &           &    6 &  5.47e-08 &  6.58e-16 &      \\ 
     &           &    7 &  2.31e-08 &  1.56e-15 &      \\ 
     &           &    8 &  2.31e-08 &  6.59e-16 &      \\ 
     &           &    9 &  5.47e-08 &  6.58e-16 &      \\ 
     &           &   10 &  2.31e-08 &  1.56e-15 &      \\ 
1457 &  1.46e+04 &   10 &           &           & iters  \\ 
 \hdashline 
     &           &    1 &  1.87e-09 &  8.01e-10 &      \\ 
     &           &    2 &  1.87e-09 &  5.34e-17 &      \\ 
     &           &    3 &  1.87e-09 &  5.34e-17 &      \\ 
     &           &    4 &  1.86e-09 &  5.33e-17 &      \\ 
     &           &    5 &  1.86e-09 &  5.32e-17 &      \\ 
     &           &    6 &  1.86e-09 &  5.31e-17 &      \\ 
     &           &    7 &  1.86e-09 &  5.31e-17 &      \\ 
     &           &    8 &  1.85e-09 &  5.30e-17 &      \\ 
     &           &    9 &  1.85e-09 &  5.29e-17 &      \\ 
     &           &   10 &  1.85e-09 &  5.28e-17 &      \\ 
1458 &  1.46e+04 &   10 &           &           & iters  \\ 
 \hdashline 
     &           &    1 &  7.99e-09 &  1.01e-10 &      \\ 
     &           &    2 &  7.98e-09 &  2.28e-16 &      \\ 
     &           &    3 &  7.97e-09 &  2.28e-16 &      \\ 
     &           &    4 &  7.96e-09 &  2.27e-16 &      \\ 
     &           &    5 &  7.95e-09 &  2.27e-16 &      \\ 
     &           &    6 &  7.93e-09 &  2.27e-16 &      \\ 
     &           &    7 &  7.92e-09 &  2.26e-16 &      \\ 
     &           &    8 &  6.98e-08 &  2.26e-16 &      \\ 
     &           &    9 &  8.57e-08 &  1.99e-15 &      \\ 
     &           &   10 &  6.98e-08 &  2.45e-15 &      \\ 
1459 &  1.46e+04 &   10 &           &           & iters  \\ 
 \hdashline 
     &           &    1 &  8.19e-10 &  1.18e-09 &      \\ 
     &           &    2 &  8.18e-10 &  2.34e-17 &      \\ 
     &           &    3 &  8.17e-10 &  2.33e-17 &      \\ 
     &           &    4 &  8.15e-10 &  2.33e-17 &      \\ 
     &           &    5 &  8.14e-10 &  2.33e-17 &      \\ 
     &           &    6 &  8.13e-10 &  2.32e-17 &      \\ 
     &           &    7 &  8.12e-10 &  2.32e-17 &      \\ 
     &           &    8 &  8.11e-10 &  2.32e-17 &      \\ 
     &           &    9 &  8.10e-10 &  2.31e-17 &      \\ 
     &           &   10 &  8.08e-10 &  2.31e-17 &      \\ 
1460 &  1.46e+04 &   10 &           &           & iters  \\ 
 \hdashline 
     &           &    1 &  3.21e-08 &  5.31e-10 &      \\ 
     &           &    2 &  6.76e-09 &  9.17e-16 &      \\ 
     &           &    3 &  6.75e-09 &  1.93e-16 &      \\ 
     &           &    4 &  6.74e-09 &  1.93e-16 &      \\ 
     &           &    5 &  6.73e-09 &  1.92e-16 &      \\ 
     &           &    6 &  3.21e-08 &  1.92e-16 &      \\ 
     &           &    7 &  6.77e-09 &  9.17e-16 &      \\ 
     &           &    8 &  6.76e-09 &  1.93e-16 &      \\ 
     &           &    9 &  6.75e-09 &  1.93e-16 &      \\ 
     &           &   10 &  6.74e-09 &  1.93e-16 &      \\ 
1461 &  1.46e+04 &   10 &           &           & iters  \\ 
 \hdashline 
     &           &    1 &  2.52e-08 &  6.10e-10 &      \\ 
     &           &    2 &  5.25e-08 &  7.20e-16 &      \\ 
     &           &    3 &  2.53e-08 &  1.50e-15 &      \\ 
     &           &    4 &  2.52e-08 &  7.21e-16 &      \\ 
     &           &    5 &  5.25e-08 &  7.20e-16 &      \\ 
     &           &    6 &  2.53e-08 &  1.50e-15 &      \\ 
     &           &    7 &  2.52e-08 &  7.21e-16 &      \\ 
     &           &    8 &  5.25e-08 &  7.20e-16 &      \\ 
     &           &    9 &  2.53e-08 &  1.50e-15 &      \\ 
     &           &   10 &  2.52e-08 &  7.21e-16 &      \\ 
1462 &  1.46e+04 &   10 &           &           & iters  \\ 
 \hdashline 
     &           &    1 &  2.83e-08 &  8.53e-10 &      \\ 
     &           &    2 &  2.83e-08 &  8.09e-16 &      \\ 
     &           &    3 &  4.94e-08 &  8.08e-16 &      \\ 
     &           &    4 &  2.83e-08 &  1.41e-15 &      \\ 
     &           &    5 &  4.94e-08 &  8.09e-16 &      \\ 
     &           &    6 &  2.84e-08 &  1.41e-15 &      \\ 
     &           &    7 &  2.83e-08 &  8.10e-16 &      \\ 
     &           &    8 &  4.94e-08 &  8.08e-16 &      \\ 
     &           &    9 &  2.84e-08 &  1.41e-15 &      \\ 
     &           &   10 &  2.83e-08 &  8.09e-16 &      \\ 
1463 &  1.46e+04 &   10 &           &           & iters  \\ 
 \hdashline 
     &           &    1 &  5.79e-08 &  4.01e-10 &      \\ 
     &           &    2 &  1.98e-08 &  1.65e-15 &      \\ 
     &           &    3 &  1.98e-08 &  5.66e-16 &      \\ 
     &           &    4 &  5.79e-08 &  5.66e-16 &      \\ 
     &           &    5 &  1.99e-08 &  1.65e-15 &      \\ 
     &           &    6 &  1.98e-08 &  5.67e-16 &      \\ 
     &           &    7 &  1.98e-08 &  5.66e-16 &      \\ 
     &           &    8 &  5.79e-08 &  5.66e-16 &      \\ 
     &           &    9 &  1.99e-08 &  1.65e-15 &      \\ 
     &           &   10 &  1.98e-08 &  5.67e-16 &      \\ 
1464 &  1.46e+04 &   10 &           &           & iters  \\ 
 \hdashline 
     &           &    1 &  4.26e-09 &  7.01e-10 &      \\ 
     &           &    2 &  4.26e-09 &  1.22e-16 &      \\ 
     &           &    3 &  4.25e-09 &  1.22e-16 &      \\ 
     &           &    4 &  4.25e-09 &  1.21e-16 &      \\ 
     &           &    5 &  4.24e-09 &  1.21e-16 &      \\ 
     &           &    6 &  4.23e-09 &  1.21e-16 &      \\ 
     &           &    7 &  4.23e-09 &  1.21e-16 &      \\ 
     &           &    8 &  4.22e-09 &  1.21e-16 &      \\ 
     &           &    9 &  4.22e-09 &  1.20e-16 &      \\ 
     &           &   10 &  4.21e-09 &  1.20e-16 &      \\ 
1465 &  1.46e+04 &   10 &           &           & iters  \\ 
 \hdashline 
     &           &    1 &  2.25e-08 &  1.80e-09 &      \\ 
     &           &    2 &  2.25e-08 &  6.43e-16 &      \\ 
     &           &    3 &  2.25e-08 &  6.42e-16 &      \\ 
     &           &    4 &  2.24e-08 &  6.41e-16 &      \\ 
     &           &    5 &  5.53e-08 &  6.40e-16 &      \\ 
     &           &    6 &  2.25e-08 &  1.58e-15 &      \\ 
     &           &    7 &  2.24e-08 &  6.41e-16 &      \\ 
     &           &    8 &  5.53e-08 &  6.40e-16 &      \\ 
     &           &    9 &  2.25e-08 &  1.58e-15 &      \\ 
     &           &   10 &  2.24e-08 &  6.42e-16 &      \\ 
1466 &  1.46e+04 &   10 &           &           & iters  \\ 
 \hdashline 
     &           &    1 &  5.49e-08 &  6.91e-10 &      \\ 
     &           &    2 &  2.28e-08 &  1.57e-15 &      \\ 
     &           &    3 &  5.49e-08 &  6.52e-16 &      \\ 
     &           &    4 &  2.29e-08 &  1.57e-15 &      \\ 
     &           &    5 &  2.29e-08 &  6.53e-16 &      \\ 
     &           &    6 &  5.49e-08 &  6.53e-16 &      \\ 
     &           &    7 &  2.29e-08 &  1.57e-15 &      \\ 
     &           &    8 &  2.29e-08 &  6.54e-16 &      \\ 
     &           &    9 &  2.28e-08 &  6.53e-16 &      \\ 
     &           &   10 &  5.49e-08 &  6.52e-16 &      \\ 
1467 &  1.47e+04 &   10 &           &           & iters  \\ 
 \hdashline 
     &           &    1 &  1.97e-08 &  1.26e-09 &      \\ 
     &           &    2 &  1.97e-08 &  5.63e-16 &      \\ 
     &           &    3 &  1.97e-08 &  5.63e-16 &      \\ 
     &           &    4 &  1.97e-08 &  5.62e-16 &      \\ 
     &           &    5 &  1.96e-08 &  5.61e-16 &      \\ 
     &           &    6 &  5.81e-08 &  5.60e-16 &      \\ 
     &           &    7 &  1.97e-08 &  1.66e-15 &      \\ 
     &           &    8 &  1.97e-08 &  5.62e-16 &      \\ 
     &           &    9 &  1.96e-08 &  5.61e-16 &      \\ 
     &           &   10 &  5.81e-08 &  5.60e-16 &      \\ 
1468 &  1.47e+04 &   10 &           &           & iters  \\ 
 \hdashline 
     &           &    1 &  1.12e-08 &  7.19e-11 &      \\ 
     &           &    2 &  1.12e-08 &  3.19e-16 &      \\ 
     &           &    3 &  1.11e-08 &  3.19e-16 &      \\ 
     &           &    4 &  6.66e-08 &  3.18e-16 &      \\ 
     &           &    5 &  8.89e-08 &  1.90e-15 &      \\ 
     &           &    6 &  6.66e-08 &  2.54e-15 &      \\ 
     &           &    7 &  1.12e-08 &  1.90e-15 &      \\ 
     &           &    8 &  1.12e-08 &  3.20e-16 &      \\ 
     &           &    9 &  1.12e-08 &  3.19e-16 &      \\ 
     &           &   10 &  1.11e-08 &  3.19e-16 &      \\ 
1469 &  1.47e+04 &   10 &           &           & iters  \\ 
 \hdashline 
     &           &    1 &  1.60e-08 &  2.13e-10 &      \\ 
     &           &    2 &  1.60e-08 &  4.58e-16 &      \\ 
     &           &    3 &  1.60e-08 &  4.57e-16 &      \\ 
     &           &    4 &  1.60e-08 &  4.57e-16 &      \\ 
     &           &    5 &  1.59e-08 &  4.56e-16 &      \\ 
     &           &    6 &  1.59e-08 &  4.55e-16 &      \\ 
     &           &    7 &  6.18e-08 &  4.55e-16 &      \\ 
     &           &    8 &  1.60e-08 &  1.76e-15 &      \\ 
     &           &    9 &  1.60e-08 &  4.56e-16 &      \\ 
     &           &   10 &  1.59e-08 &  4.56e-16 &      \\ 
1470 &  1.47e+04 &   10 &           &           & iters  \\ 
 \hdashline 
     &           &    1 &  7.66e-08 &  2.21e-09 &      \\ 
     &           &    2 &  7.89e-08 &  2.19e-15 &      \\ 
     &           &    3 &  7.66e-08 &  2.25e-15 &      \\ 
     &           &    4 &  1.16e-09 &  2.19e-15 &      \\ 
     &           &    5 &  1.16e-09 &  3.32e-17 &      \\ 
     &           &    6 &  1.16e-09 &  3.31e-17 &      \\ 
     &           &    7 &  1.16e-09 &  3.31e-17 &      \\ 
     &           &    8 &  1.16e-09 &  3.30e-17 &      \\ 
     &           &    9 &  1.15e-09 &  3.30e-17 &      \\ 
     &           &   10 &  1.15e-09 &  3.29e-17 &      \\ 
1471 &  1.47e+04 &   10 &           &           & iters  \\ 
 \hdashline 
     &           &    1 &  3.24e-08 &  1.67e-09 &      \\ 
     &           &    2 &  6.46e-09 &  9.26e-16 &      \\ 
     &           &    3 &  6.45e-09 &  1.84e-16 &      \\ 
     &           &    4 &  6.44e-09 &  1.84e-16 &      \\ 
     &           &    5 &  6.43e-09 &  1.84e-16 &      \\ 
     &           &    6 &  6.42e-09 &  1.83e-16 &      \\ 
     &           &    7 &  3.24e-08 &  1.83e-16 &      \\ 
     &           &    8 &  6.46e-09 &  9.26e-16 &      \\ 
     &           &    9 &  6.45e-09 &  1.84e-16 &      \\ 
     &           &   10 &  6.44e-09 &  1.84e-16 &      \\ 
1472 &  1.47e+04 &   10 &           &           & iters  \\ 
 \hdashline 
     &           &    1 &  7.07e-09 &  4.79e-10 &      \\ 
     &           &    2 &  7.06e-09 &  2.02e-16 &      \\ 
     &           &    3 &  7.05e-09 &  2.02e-16 &      \\ 
     &           &    4 &  7.04e-09 &  2.01e-16 &      \\ 
     &           &    5 &  7.03e-09 &  2.01e-16 &      \\ 
     &           &    6 &  7.02e-09 &  2.01e-16 &      \\ 
     &           &    7 &  7.01e-09 &  2.00e-16 &      \\ 
     &           &    8 &  7.07e-08 &  2.00e-16 &      \\ 
     &           &    9 &  4.59e-08 &  2.02e-15 &      \\ 
     &           &   10 &  7.04e-09 &  1.31e-15 &      \\ 
1473 &  1.47e+04 &   10 &           &           & iters  \\ 
 \hdashline 
     &           &    1 &  1.88e-08 &  3.22e-09 &      \\ 
     &           &    2 &  1.88e-08 &  5.36e-16 &      \\ 
     &           &    3 &  5.90e-08 &  5.35e-16 &      \\ 
     &           &    4 &  1.88e-08 &  1.68e-15 &      \\ 
     &           &    5 &  1.88e-08 &  5.37e-16 &      \\ 
     &           &    6 &  1.88e-08 &  5.36e-16 &      \\ 
     &           &    7 &  5.89e-08 &  5.36e-16 &      \\ 
     &           &    8 &  1.88e-08 &  1.68e-15 &      \\ 
     &           &    9 &  1.88e-08 &  5.37e-16 &      \\ 
     &           &   10 &  1.88e-08 &  5.36e-16 &      \\ 
1474 &  1.47e+04 &   10 &           &           & iters  \\ 
 \hdashline 
     &           &    1 &  1.01e-07 &  3.84e-09 &      \\ 
     &           &    2 &  5.42e-08 &  2.89e-15 &      \\ 
     &           &    3 &  2.36e-08 &  1.55e-15 &      \\ 
     &           &    4 &  2.35e-08 &  6.72e-16 &      \\ 
     &           &    5 &  5.42e-08 &  6.71e-16 &      \\ 
     &           &    6 &  2.36e-08 &  1.55e-15 &      \\ 
     &           &    7 &  2.35e-08 &  6.73e-16 &      \\ 
     &           &    8 &  5.42e-08 &  6.72e-16 &      \\ 
     &           &    9 &  2.36e-08 &  1.55e-15 &      \\ 
     &           &   10 &  2.35e-08 &  6.73e-16 &      \\ 
1475 &  1.47e+04 &   10 &           &           & iters  \\ 
 \hdashline 
     &           &    1 &  1.29e-08 &  2.00e-11 &      \\ 
     &           &    2 &  1.29e-08 &  3.67e-16 &      \\ 
     &           &    3 &  6.49e-08 &  3.67e-16 &      \\ 
     &           &    4 &  1.29e-08 &  1.85e-15 &      \\ 
     &           &    5 &  1.29e-08 &  3.69e-16 &      \\ 
     &           &    6 &  1.29e-08 &  3.68e-16 &      \\ 
     &           &    7 &  1.29e-08 &  3.68e-16 &      \\ 
     &           &    8 &  1.29e-08 &  3.67e-16 &      \\ 
     &           &    9 &  6.49e-08 &  3.67e-16 &      \\ 
     &           &   10 &  1.29e-08 &  1.85e-15 &      \\ 
1476 &  1.48e+04 &   10 &           &           & iters  \\ 
 \hdashline 
     &           &    1 &  8.34e-10 &  1.31e-09 &      \\ 
     &           &    2 &  8.33e-10 &  2.38e-17 &      \\ 
     &           &    3 &  8.32e-10 &  2.38e-17 &      \\ 
     &           &    4 &  8.31e-10 &  2.37e-17 &      \\ 
     &           &    5 &  8.30e-10 &  2.37e-17 &      \\ 
     &           &    6 &  8.28e-10 &  2.37e-17 &      \\ 
     &           &    7 &  8.27e-10 &  2.36e-17 &      \\ 
     &           &    8 &  8.26e-10 &  2.36e-17 &      \\ 
     &           &    9 &  8.25e-10 &  2.36e-17 &      \\ 
     &           &   10 &  8.24e-10 &  2.35e-17 &      \\ 
1477 &  1.48e+04 &   10 &           &           & iters  \\ 
 \hdashline 
     &           &    1 &  2.33e-08 &  1.10e-09 &      \\ 
     &           &    2 &  5.44e-08 &  6.64e-16 &      \\ 
     &           &    3 &  2.33e-08 &  1.55e-15 &      \\ 
     &           &    4 &  2.33e-08 &  6.66e-16 &      \\ 
     &           &    5 &  5.44e-08 &  6.65e-16 &      \\ 
     &           &    6 &  2.33e-08 &  1.55e-15 &      \\ 
     &           &    7 &  2.33e-08 &  6.66e-16 &      \\ 
     &           &    8 &  2.33e-08 &  6.65e-16 &      \\ 
     &           &    9 &  5.45e-08 &  6.64e-16 &      \\ 
     &           &   10 &  2.33e-08 &  1.55e-15 &      \\ 
1478 &  1.48e+04 &   10 &           &           & iters  \\ 
 \hdashline 
     &           &    1 &  2.26e-08 &  1.95e-09 &      \\ 
     &           &    2 &  2.25e-08 &  6.44e-16 &      \\ 
     &           &    3 &  5.52e-08 &  6.43e-16 &      \\ 
     &           &    4 &  2.26e-08 &  1.58e-15 &      \\ 
     &           &    5 &  2.25e-08 &  6.44e-16 &      \\ 
     &           &    6 &  5.52e-08 &  6.44e-16 &      \\ 
     &           &    7 &  2.26e-08 &  1.57e-15 &      \\ 
     &           &    8 &  2.26e-08 &  6.45e-16 &      \\ 
     &           &    9 &  2.25e-08 &  6.44e-16 &      \\ 
     &           &   10 &  5.52e-08 &  6.43e-16 &      \\ 
1479 &  1.48e+04 &   10 &           &           & iters  \\ 
 \hdashline 
     &           &    1 &  3.23e-08 &  1.04e-10 &      \\ 
     &           &    2 &  4.54e-08 &  9.23e-16 &      \\ 
     &           &    3 &  3.23e-08 &  1.30e-15 &      \\ 
     &           &    4 &  3.23e-08 &  9.23e-16 &      \\ 
     &           &    5 &  4.54e-08 &  9.22e-16 &      \\ 
     &           &    6 &  3.23e-08 &  1.30e-15 &      \\ 
     &           &    7 &  4.54e-08 &  9.22e-16 &      \\ 
     &           &    8 &  3.23e-08 &  1.30e-15 &      \\ 
     &           &    9 &  3.23e-08 &  9.23e-16 &      \\ 
     &           &   10 &  4.54e-08 &  9.22e-16 &      \\ 
1480 &  1.48e+04 &   10 &           &           & iters  \\ 
 \hdashline 
     &           &    1 &  2.65e-08 &  6.06e-10 &      \\ 
     &           &    2 &  5.13e-08 &  7.55e-16 &      \\ 
     &           &    3 &  2.65e-08 &  1.46e-15 &      \\ 
     &           &    4 &  2.65e-08 &  7.56e-16 &      \\ 
     &           &    5 &  5.13e-08 &  7.55e-16 &      \\ 
     &           &    6 &  2.65e-08 &  1.46e-15 &      \\ 
     &           &    7 &  2.65e-08 &  7.56e-16 &      \\ 
     &           &    8 &  5.13e-08 &  7.55e-16 &      \\ 
     &           &    9 &  2.65e-08 &  1.46e-15 &      \\ 
     &           &   10 &  2.64e-08 &  7.56e-16 &      \\ 
1481 &  1.48e+04 &   10 &           &           & iters  \\ 
 \hdashline 
     &           &    1 &  3.22e-08 &  1.26e-09 &      \\ 
     &           &    2 &  4.56e-08 &  9.18e-16 &      \\ 
     &           &    3 &  3.22e-08 &  1.30e-15 &      \\ 
     &           &    4 &  4.55e-08 &  9.19e-16 &      \\ 
     &           &    5 &  3.22e-08 &  1.30e-15 &      \\ 
     &           &    6 &  3.22e-08 &  9.19e-16 &      \\ 
     &           &    7 &  4.56e-08 &  9.18e-16 &      \\ 
     &           &    8 &  3.22e-08 &  1.30e-15 &      \\ 
     &           &    9 &  4.55e-08 &  9.19e-16 &      \\ 
     &           &   10 &  3.22e-08 &  1.30e-15 &      \\ 
1482 &  1.48e+04 &   10 &           &           & iters  \\ 
 \hdashline 
     &           &    1 &  5.95e-09 &  1.39e-09 &      \\ 
     &           &    2 &  5.94e-09 &  1.70e-16 &      \\ 
     &           &    3 &  5.93e-09 &  1.70e-16 &      \\ 
     &           &    4 &  5.92e-09 &  1.69e-16 &      \\ 
     &           &    5 &  5.92e-09 &  1.69e-16 &      \\ 
     &           &    6 &  5.91e-09 &  1.69e-16 &      \\ 
     &           &    7 &  5.90e-09 &  1.69e-16 &      \\ 
     &           &    8 &  5.89e-09 &  1.68e-16 &      \\ 
     &           &    9 &  5.88e-09 &  1.68e-16 &      \\ 
     &           &   10 &  7.18e-08 &  1.68e-16 &      \\ 
1483 &  1.48e+04 &   10 &           &           & iters  \\ 
 \hdashline 
     &           &    1 &  7.81e-08 &  1.21e-09 &      \\ 
     &           &    2 &  7.74e-08 &  2.23e-15 &      \\ 
     &           &    3 &  7.81e-08 &  2.21e-15 &      \\ 
     &           &    4 &  7.74e-08 &  2.23e-15 &      \\ 
     &           &    5 &  7.81e-08 &  2.21e-15 &      \\ 
     &           &    6 &  7.74e-08 &  2.23e-15 &      \\ 
     &           &    7 &  7.81e-08 &  2.21e-15 &      \\ 
     &           &    8 &  7.74e-08 &  2.23e-15 &      \\ 
     &           &    9 &  7.81e-08 &  2.21e-15 &      \\ 
     &           &   10 &  7.74e-08 &  2.23e-15 &      \\ 
1484 &  1.48e+04 &   10 &           &           & iters  \\ 
 \hdashline 
     &           &    1 &  5.40e-08 &  8.91e-10 &      \\ 
     &           &    2 &  2.37e-08 &  1.54e-15 &      \\ 
     &           &    3 &  2.37e-08 &  6.77e-16 &      \\ 
     &           &    4 &  5.40e-08 &  6.76e-16 &      \\ 
     &           &    5 &  2.37e-08 &  1.54e-15 &      \\ 
     &           &    6 &  2.37e-08 &  6.77e-16 &      \\ 
     &           &    7 &  5.40e-08 &  6.76e-16 &      \\ 
     &           &    8 &  2.37e-08 &  1.54e-15 &      \\ 
     &           &    9 &  2.37e-08 &  6.78e-16 &      \\ 
     &           &   10 &  2.37e-08 &  6.77e-16 &      \\ 
1485 &  1.48e+04 &   10 &           &           & iters  \\ 
 \hdashline 
     &           &    1 &  3.19e-08 &  1.40e-09 &      \\ 
     &           &    2 &  6.97e-09 &  9.11e-16 &      \\ 
     &           &    3 &  6.96e-09 &  1.99e-16 &      \\ 
     &           &    4 &  3.19e-08 &  1.99e-16 &      \\ 
     &           &    5 &  7.00e-09 &  9.10e-16 &      \\ 
     &           &    6 &  6.99e-09 &  2.00e-16 &      \\ 
     &           &    7 &  6.98e-09 &  1.99e-16 &      \\ 
     &           &    8 &  6.97e-09 &  1.99e-16 &      \\ 
     &           &    9 &  6.96e-09 &  1.99e-16 &      \\ 
     &           &   10 &  3.19e-08 &  1.99e-16 &      \\ 
1486 &  1.48e+04 &   10 &           &           & iters  \\ 
 \hdashline 
     &           &    1 &  2.32e-09 &  2.27e-09 &      \\ 
     &           &    2 &  2.32e-09 &  6.62e-17 &      \\ 
     &           &    3 &  2.31e-09 &  6.61e-17 &      \\ 
     &           &    4 &  2.31e-09 &  6.60e-17 &      \\ 
     &           &    5 &  2.31e-09 &  6.59e-17 &      \\ 
     &           &    6 &  2.30e-09 &  6.58e-17 &      \\ 
     &           &    7 &  2.30e-09 &  6.57e-17 &      \\ 
     &           &    8 &  2.30e-09 &  6.56e-17 &      \\ 
     &           &    9 &  2.29e-09 &  6.55e-17 &      \\ 
     &           &   10 &  2.29e-09 &  6.54e-17 &      \\ 
1487 &  1.49e+04 &   10 &           &           & iters  \\ 
 \hdashline 
     &           &    1 &  4.86e-08 &  5.66e-10 &      \\ 
     &           &    2 &  2.92e-08 &  1.39e-15 &      \\ 
     &           &    3 &  2.91e-08 &  8.32e-16 &      \\ 
     &           &    4 &  4.86e-08 &  8.31e-16 &      \\ 
     &           &    5 &  2.91e-08 &  1.39e-15 &      \\ 
     &           &    6 &  4.86e-08 &  8.32e-16 &      \\ 
     &           &    7 &  2.92e-08 &  1.39e-15 &      \\ 
     &           &    8 &  2.91e-08 &  8.32e-16 &      \\ 
     &           &    9 &  4.86e-08 &  8.31e-16 &      \\ 
     &           &   10 &  2.92e-08 &  1.39e-15 &      \\ 
1488 &  1.49e+04 &   10 &           &           & iters  \\ 
 \hdashline 
     &           &    1 &  2.20e-08 &  5.28e-10 &      \\ 
     &           &    2 &  5.57e-08 &  6.28e-16 &      \\ 
     &           &    3 &  2.21e-08 &  1.59e-15 &      \\ 
     &           &    4 &  2.20e-08 &  6.30e-16 &      \\ 
     &           &    5 &  2.20e-08 &  6.29e-16 &      \\ 
     &           &    6 &  5.57e-08 &  6.28e-16 &      \\ 
     &           &    7 &  2.21e-08 &  1.59e-15 &      \\ 
     &           &    8 &  2.20e-08 &  6.29e-16 &      \\ 
     &           &    9 &  5.57e-08 &  6.29e-16 &      \\ 
     &           &   10 &  2.21e-08 &  1.59e-15 &      \\ 
1489 &  1.49e+04 &   10 &           &           & iters  \\ 
 \hdashline 
     &           &    1 &  3.04e-08 &  8.34e-10 &      \\ 
     &           &    2 &  4.73e-08 &  8.69e-16 &      \\ 
     &           &    3 &  3.05e-08 &  1.35e-15 &      \\ 
     &           &    4 &  4.73e-08 &  8.70e-16 &      \\ 
     &           &    5 &  3.05e-08 &  1.35e-15 &      \\ 
     &           &    6 &  3.04e-08 &  8.70e-16 &      \\ 
     &           &    7 &  4.73e-08 &  8.69e-16 &      \\ 
     &           &    8 &  3.05e-08 &  1.35e-15 &      \\ 
     &           &    9 &  4.73e-08 &  8.70e-16 &      \\ 
     &           &   10 &  3.05e-08 &  1.35e-15 &      \\ 
1490 &  1.49e+04 &   10 &           &           & iters  \\ 
 \hdashline 
     &           &    1 &  2.36e-09 &  1.42e-09 &      \\ 
     &           &    2 &  2.36e-09 &  6.75e-17 &      \\ 
     &           &    3 &  2.36e-09 &  6.74e-17 &      \\ 
     &           &    4 &  2.35e-09 &  6.73e-17 &      \\ 
     &           &    5 &  2.35e-09 &  6.72e-17 &      \\ 
     &           &    6 &  2.35e-09 &  6.71e-17 &      \\ 
     &           &    7 &  2.34e-09 &  6.70e-17 &      \\ 
     &           &    8 &  2.34e-09 &  6.69e-17 &      \\ 
     &           &    9 &  2.34e-09 &  6.68e-17 &      \\ 
     &           &   10 &  2.33e-09 &  6.67e-17 &      \\ 
1491 &  1.49e+04 &   10 &           &           & iters  \\ 
 \hdashline 
     &           &    1 &  1.11e-08 &  1.27e-09 &      \\ 
     &           &    2 &  1.10e-08 &  3.16e-16 &      \\ 
     &           &    3 &  1.10e-08 &  3.15e-16 &      \\ 
     &           &    4 &  1.10e-08 &  3.15e-16 &      \\ 
     &           &    5 &  1.10e-08 &  3.14e-16 &      \\ 
     &           &    6 &  1.10e-08 &  3.14e-16 &      \\ 
     &           &    7 &  1.10e-08 &  3.13e-16 &      \\ 
     &           &    8 &  1.09e-08 &  3.13e-16 &      \\ 
     &           &    9 &  6.68e-08 &  3.12e-16 &      \\ 
     &           &   10 &  1.10e-08 &  1.91e-15 &      \\ 
1492 &  1.49e+04 &   10 &           &           & iters  \\ 
 \hdashline 
     &           &    1 &  1.68e-08 &  2.48e-09 &      \\ 
     &           &    2 &  1.68e-08 &  4.79e-16 &      \\ 
     &           &    3 &  1.67e-08 &  4.79e-16 &      \\ 
     &           &    4 &  6.10e-08 &  4.78e-16 &      \\ 
     &           &    5 &  1.68e-08 &  1.74e-15 &      \\ 
     &           &    6 &  1.68e-08 &  4.80e-16 &      \\ 
     &           &    7 &  1.68e-08 &  4.79e-16 &      \\ 
     &           &    8 &  1.67e-08 &  4.78e-16 &      \\ 
     &           &    9 &  6.10e-08 &  4.78e-16 &      \\ 
     &           &   10 &  1.68e-08 &  1.74e-15 &      \\ 
1493 &  1.49e+04 &   10 &           &           & iters  \\ 
 \hdashline 
     &           &    1 &  2.14e-08 &  1.03e-09 &      \\ 
     &           &    2 &  5.63e-08 &  6.10e-16 &      \\ 
     &           &    3 &  2.14e-08 &  1.61e-15 &      \\ 
     &           &    4 &  2.14e-08 &  6.12e-16 &      \\ 
     &           &    5 &  2.14e-08 &  6.11e-16 &      \\ 
     &           &    6 &  5.64e-08 &  6.10e-16 &      \\ 
     &           &    7 &  2.14e-08 &  1.61e-15 &      \\ 
     &           &    8 &  2.14e-08 &  6.11e-16 &      \\ 
     &           &    9 &  5.63e-08 &  6.10e-16 &      \\ 
     &           &   10 &  2.14e-08 &  1.61e-15 &      \\ 
1494 &  1.49e+04 &   10 &           &           & iters  \\ 
 \hdashline 
     &           &    1 &  9.25e-08 &  8.67e-10 &      \\ 
     &           &    2 &  6.30e-08 &  2.64e-15 &      \\ 
     &           &    3 &  1.48e-08 &  1.80e-15 &      \\ 
     &           &    4 &  1.47e-08 &  4.21e-16 &      \\ 
     &           &    5 &  1.47e-08 &  4.21e-16 &      \\ 
     &           &    6 &  6.30e-08 &  4.20e-16 &      \\ 
     &           &    7 &  1.48e-08 &  1.80e-15 &      \\ 
     &           &    8 &  1.48e-08 &  4.22e-16 &      \\ 
     &           &    9 &  1.47e-08 &  4.21e-16 &      \\ 
     &           &   10 &  1.47e-08 &  4.21e-16 &      \\ 
1495 &  1.49e+04 &   10 &           &           & iters  \\ 
 \hdashline 
     &           &    1 &  3.13e-08 &  1.48e-10 &      \\ 
     &           &    2 &  3.12e-08 &  8.92e-16 &      \\ 
     &           &    3 &  3.12e-08 &  8.91e-16 &      \\ 
     &           &    4 &  4.66e-08 &  8.89e-16 &      \\ 
     &           &    5 &  3.12e-08 &  1.33e-15 &      \\ 
     &           &    6 &  4.65e-08 &  8.90e-16 &      \\ 
     &           &    7 &  3.12e-08 &  1.33e-15 &      \\ 
     &           &    8 &  3.12e-08 &  8.91e-16 &      \\ 
     &           &    9 &  4.66e-08 &  8.89e-16 &      \\ 
     &           &   10 &  3.12e-08 &  1.33e-15 &      \\ 
1496 &  1.50e+04 &   10 &           &           & iters  \\ 
 \hdashline 
     &           &    1 &  1.42e-08 &  2.61e-10 &      \\ 
     &           &    2 &  1.42e-08 &  4.06e-16 &      \\ 
     &           &    3 &  1.42e-08 &  4.05e-16 &      \\ 
     &           &    4 &  1.42e-08 &  4.05e-16 &      \\ 
     &           &    5 &  6.35e-08 &  4.04e-16 &      \\ 
     &           &    6 &  1.42e-08 &  1.81e-15 &      \\ 
     &           &    7 &  1.42e-08 &  4.06e-16 &      \\ 
     &           &    8 &  1.42e-08 &  4.06e-16 &      \\ 
     &           &    9 &  1.42e-08 &  4.05e-16 &      \\ 
     &           &   10 &  6.35e-08 &  4.04e-16 &      \\ 
1497 &  1.50e+04 &   10 &           &           & iters  \\ 
 \hdashline 
     &           &    1 &  4.07e-08 &  1.94e-09 &      \\ 
     &           &    2 &  3.71e-08 &  1.16e-15 &      \\ 
     &           &    3 &  4.06e-08 &  1.06e-15 &      \\ 
     &           &    4 &  3.71e-08 &  1.16e-15 &      \\ 
     &           &    5 &  4.06e-08 &  1.06e-15 &      \\ 
     &           &    6 &  3.71e-08 &  1.16e-15 &      \\ 
     &           &    7 &  4.06e-08 &  1.06e-15 &      \\ 
     &           &    8 &  3.71e-08 &  1.16e-15 &      \\ 
     &           &    9 &  4.06e-08 &  1.06e-15 &      \\ 
     &           &   10 &  3.71e-08 &  1.16e-15 &      \\ 
1498 &  1.50e+04 &   10 &           &           & iters  \\ 
 \hdashline 
     &           &    1 &  3.41e-09 &  2.08e-09 &      \\ 
     &           &    2 &  3.40e-09 &  9.72e-17 &      \\ 
     &           &    3 &  7.43e-08 &  9.71e-17 &      \\ 
     &           &    4 &  8.12e-08 &  2.12e-15 &      \\ 
     &           &    5 &  7.43e-08 &  2.32e-15 &      \\ 
     &           &    6 &  8.12e-08 &  2.12e-15 &      \\ 
     &           &    7 &  7.43e-08 &  2.32e-15 &      \\ 
     &           &    8 &  3.48e-09 &  2.12e-15 &      \\ 
     &           &    9 &  3.48e-09 &  9.94e-17 &      \\ 
     &           &   10 &  3.47e-09 &  9.93e-17 &      \\ 
1499 &  1.50e+04 &   10 &           &           & iters  \\ 
 \hdashline 
     &           &    1 &  1.79e-08 &  8.84e-10 &      \\ 
     &           &    2 &  1.79e-08 &  5.11e-16 &      \\ 
     &           &    3 &  1.79e-08 &  5.11e-16 &      \\ 
     &           &    4 &  5.98e-08 &  5.10e-16 &      \\ 
     &           &    5 &  1.79e-08 &  1.71e-15 &      \\ 
     &           &    6 &  1.79e-08 &  5.12e-16 &      \\ 
     &           &    7 &  1.79e-08 &  5.11e-16 &      \\ 
     &           &    8 &  1.78e-08 &  5.10e-16 &      \\ 
     &           &    9 &  5.99e-08 &  5.09e-16 &      \\ 
     &           &   10 &  1.79e-08 &  1.71e-15 &      \\ 
1500 &  1.50e+04 &   10 &           &           & iters  \\ 
 \hdashline 
     &           &    1 &  1.92e-09 &  2.03e-10 &      \\ 
     &           &    2 &  1.92e-09 &  5.49e-17 &      \\ 
     &           &    3 &  1.92e-09 &  5.48e-17 &      \\ 
     &           &    4 &  1.92e-09 &  5.47e-17 &      \\ 
     &           &    5 &  1.91e-09 &  5.47e-17 &      \\ 
     &           &    6 &  1.91e-09 &  5.46e-17 &      \\ 
     &           &    7 &  1.91e-09 &  5.45e-17 &      \\ 
     &           &    8 &  1.90e-09 &  5.44e-17 &      \\ 
     &           &    9 &  1.90e-09 &  5.43e-17 &      \\ 
     &           &   10 &  1.90e-09 &  5.43e-17 &      \\ 
1501 &  1.50e+04 &   10 &           &           & iters  \\ 
 \hdashline 
     &           &    1 &  2.06e-09 &  1.32e-09 &      \\ 
     &           &    2 &  2.05e-09 &  5.87e-17 &      \\ 
     &           &    3 &  2.05e-09 &  5.86e-17 &      \\ 
     &           &    4 &  2.05e-09 &  5.85e-17 &      \\ 
     &           &    5 &  2.04e-09 &  5.84e-17 &      \\ 
     &           &    6 &  2.04e-09 &  5.83e-17 &      \\ 
     &           &    7 &  2.04e-09 &  5.82e-17 &      \\ 
     &           &    8 &  2.03e-09 &  5.82e-17 &      \\ 
     &           &    9 &  2.03e-09 &  5.81e-17 &      \\ 
     &           &   10 &  2.03e-09 &  5.80e-17 &      \\ 
1502 &  1.50e+04 &   10 &           &           & iters  \\ 
 \hdashline 
     &           &    1 &  4.57e-08 &  7.49e-10 &      \\ 
     &           &    2 &  3.20e-08 &  1.31e-15 &      \\ 
     &           &    3 &  4.57e-08 &  9.14e-16 &      \\ 
     &           &    4 &  3.20e-08 &  1.30e-15 &      \\ 
     &           &    5 &  3.20e-08 &  9.14e-16 &      \\ 
     &           &    6 &  4.57e-08 &  9.13e-16 &      \\ 
     &           &    7 &  3.20e-08 &  1.31e-15 &      \\ 
     &           &    8 &  4.57e-08 &  9.14e-16 &      \\ 
     &           &    9 &  3.20e-08 &  1.30e-15 &      \\ 
     &           &   10 &  3.20e-08 &  9.14e-16 &      \\ 
1503 &  1.50e+04 &   10 &           &           & iters  \\ 
 \hdashline 
     &           &    1 &  6.46e-08 &  7.30e-11 &      \\ 
     &           &    2 &  1.32e-08 &  1.84e-15 &      \\ 
     &           &    3 &  1.32e-08 &  3.77e-16 &      \\ 
     &           &    4 &  1.32e-08 &  3.77e-16 &      \\ 
     &           &    5 &  6.45e-08 &  3.76e-16 &      \\ 
     &           &    6 &  9.09e-08 &  1.84e-15 &      \\ 
     &           &    7 &  6.46e-08 &  2.60e-15 &      \\ 
     &           &    8 &  1.32e-08 &  1.84e-15 &      \\ 
     &           &    9 &  1.32e-08 &  3.77e-16 &      \\ 
     &           &   10 &  1.32e-08 &  3.76e-16 &      \\ 
1504 &  1.50e+04 &   10 &           &           & iters  \\ 
 \hdashline 
     &           &    1 &  2.35e-08 &  3.65e-10 &      \\ 
     &           &    2 &  2.35e-08 &  6.70e-16 &      \\ 
     &           &    3 &  5.43e-08 &  6.69e-16 &      \\ 
     &           &    4 &  2.35e-08 &  1.55e-15 &      \\ 
     &           &    5 &  2.35e-08 &  6.71e-16 &      \\ 
     &           &    6 &  5.43e-08 &  6.70e-16 &      \\ 
     &           &    7 &  2.35e-08 &  1.55e-15 &      \\ 
     &           &    8 &  2.35e-08 &  6.71e-16 &      \\ 
     &           &    9 &  5.42e-08 &  6.70e-16 &      \\ 
     &           &   10 &  2.35e-08 &  1.55e-15 &      \\ 
1505 &  1.50e+04 &   10 &           &           & iters  \\ 
 \hdashline 
     &           &    1 &  5.82e-09 &  6.81e-10 &      \\ 
     &           &    2 &  5.81e-09 &  1.66e-16 &      \\ 
     &           &    3 &  5.80e-09 &  1.66e-16 &      \\ 
     &           &    4 &  7.19e-08 &  1.66e-16 &      \\ 
     &           &    5 &  8.36e-08 &  2.05e-15 &      \\ 
     &           &    6 &  7.19e-08 &  2.39e-15 &      \\ 
     &           &    7 &  5.88e-09 &  2.05e-15 &      \\ 
     &           &    8 &  5.87e-09 &  1.68e-16 &      \\ 
     &           &    9 &  5.87e-09 &  1.68e-16 &      \\ 
     &           &   10 &  5.86e-09 &  1.67e-16 &      \\ 
1506 &  1.50e+04 &   10 &           &           & iters  \\ 
 \hdashline 
     &           &    1 &  5.37e-08 &  3.30e-09 &      \\ 
     &           &    2 &  2.41e-08 &  1.53e-15 &      \\ 
     &           &    3 &  2.40e-08 &  6.87e-16 &      \\ 
     &           &    4 &  5.37e-08 &  6.86e-16 &      \\ 
     &           &    5 &  2.41e-08 &  1.53e-15 &      \\ 
     &           &    6 &  2.40e-08 &  6.87e-16 &      \\ 
     &           &    7 &  5.37e-08 &  6.86e-16 &      \\ 
     &           &    8 &  2.41e-08 &  1.53e-15 &      \\ 
     &           &    9 &  2.40e-08 &  6.87e-16 &      \\ 
     &           &   10 &  2.40e-08 &  6.86e-16 &      \\ 
1507 &  1.51e+04 &   10 &           &           & iters  \\ 
 \hdashline 
     &           &    1 &  6.17e-08 &  1.28e-09 &      \\ 
     &           &    2 &  1.61e-08 &  1.76e-15 &      \\ 
     &           &    3 &  1.61e-08 &  4.59e-16 &      \\ 
     &           &    4 &  1.60e-08 &  4.58e-16 &      \\ 
     &           &    5 &  6.17e-08 &  4.57e-16 &      \\ 
     &           &    6 &  1.61e-08 &  1.76e-15 &      \\ 
     &           &    7 &  1.61e-08 &  4.59e-16 &      \\ 
     &           &    8 &  1.60e-08 &  4.59e-16 &      \\ 
     &           &    9 &  1.60e-08 &  4.58e-16 &      \\ 
     &           &   10 &  6.17e-08 &  4.57e-16 &      \\ 
1508 &  1.51e+04 &   10 &           &           & iters  \\ 
 \hdashline 
     &           &    1 &  6.66e-09 &  2.23e-09 &      \\ 
     &           &    2 &  6.65e-09 &  1.90e-16 &      \\ 
     &           &    3 &  6.64e-09 &  1.90e-16 &      \\ 
     &           &    4 &  6.63e-09 &  1.89e-16 &      \\ 
     &           &    5 &  6.62e-09 &  1.89e-16 &      \\ 
     &           &    6 &  6.61e-09 &  1.89e-16 &      \\ 
     &           &    7 &  6.60e-09 &  1.89e-16 &      \\ 
     &           &    8 &  6.59e-09 &  1.88e-16 &      \\ 
     &           &    9 &  6.58e-09 &  1.88e-16 &      \\ 
     &           &   10 &  6.57e-09 &  1.88e-16 &      \\ 
1509 &  1.51e+04 &   10 &           &           & iters  \\ 
 \hdashline 
     &           &    1 &  2.73e-08 &  1.67e-10 &      \\ 
     &           &    2 &  5.04e-08 &  7.81e-16 &      \\ 
     &           &    3 &  2.74e-08 &  1.44e-15 &      \\ 
     &           &    4 &  2.73e-08 &  7.82e-16 &      \\ 
     &           &    5 &  5.04e-08 &  7.80e-16 &      \\ 
     &           &    6 &  2.74e-08 &  1.44e-15 &      \\ 
     &           &    7 &  2.73e-08 &  7.81e-16 &      \\ 
     &           &    8 &  5.04e-08 &  7.80e-16 &      \\ 
     &           &    9 &  2.74e-08 &  1.44e-15 &      \\ 
     &           &   10 &  2.73e-08 &  7.81e-16 &      \\ 
1510 &  1.51e+04 &   10 &           &           & iters  \\ 
 \hdashline 
     &           &    1 &  7.73e-09 &  1.04e-09 &      \\ 
     &           &    2 &  7.00e-08 &  2.21e-16 &      \\ 
     &           &    3 &  8.55e-08 &  2.00e-15 &      \\ 
     &           &    4 &  7.00e-08 &  2.44e-15 &      \\ 
     &           &    5 &  7.80e-09 &  2.00e-15 &      \\ 
     &           &    6 &  7.79e-09 &  2.23e-16 &      \\ 
     &           &    7 &  7.78e-09 &  2.22e-16 &      \\ 
     &           &    8 &  7.76e-09 &  2.22e-16 &      \\ 
     &           &    9 &  7.75e-09 &  2.22e-16 &      \\ 
     &           &   10 &  7.74e-09 &  2.21e-16 &      \\ 
1511 &  1.51e+04 &   10 &           &           & iters  \\ 
 \hdashline 
     &           &    1 &  4.70e-09 &  2.74e-10 &      \\ 
     &           &    2 &  4.69e-09 &  1.34e-16 &      \\ 
     &           &    3 &  4.68e-09 &  1.34e-16 &      \\ 
     &           &    4 &  4.68e-09 &  1.34e-16 &      \\ 
     &           &    5 &  4.67e-09 &  1.34e-16 &      \\ 
     &           &    6 &  4.66e-09 &  1.33e-16 &      \\ 
     &           &    7 &  4.66e-09 &  1.33e-16 &      \\ 
     &           &    8 &  4.65e-09 &  1.33e-16 &      \\ 
     &           &    9 &  3.42e-08 &  1.33e-16 &      \\ 
     &           &   10 &  4.69e-09 &  9.76e-16 &      \\ 
1512 &  1.51e+04 &   10 &           &           & iters  \\ 
 \hdashline 
     &           &    1 &  6.94e-09 &  2.27e-11 &      \\ 
     &           &    2 &  6.93e-09 &  1.98e-16 &      \\ 
     &           &    3 &  6.92e-09 &  1.98e-16 &      \\ 
     &           &    4 &  6.91e-09 &  1.98e-16 &      \\ 
     &           &    5 &  6.90e-09 &  1.97e-16 &      \\ 
     &           &    6 &  7.08e-08 &  1.97e-16 &      \\ 
     &           &    7 &  8.47e-08 &  2.02e-15 &      \\ 
     &           &    8 &  7.08e-08 &  2.42e-15 &      \\ 
     &           &    9 &  6.98e-09 &  2.02e-15 &      \\ 
     &           &   10 &  6.97e-09 &  1.99e-16 &      \\ 
1513 &  1.51e+04 &   10 &           &           & iters  \\ 
 \hdashline 
     &           &    1 &  4.55e-08 &  3.14e-10 &      \\ 
     &           &    2 &  3.23e-08 &  1.30e-15 &      \\ 
     &           &    3 &  3.22e-08 &  9.21e-16 &      \\ 
     &           &    4 &  4.55e-08 &  9.19e-16 &      \\ 
     &           &    5 &  3.22e-08 &  1.30e-15 &      \\ 
     &           &    6 &  4.55e-08 &  9.20e-16 &      \\ 
     &           &    7 &  3.22e-08 &  1.30e-15 &      \\ 
     &           &    8 &  3.22e-08 &  9.20e-16 &      \\ 
     &           &    9 &  4.55e-08 &  9.19e-16 &      \\ 
     &           &   10 &  3.22e-08 &  1.30e-15 &      \\ 
1514 &  1.51e+04 &   10 &           &           & iters  \\ 
 \hdashline 
     &           &    1 &  2.14e-08 &  2.29e-10 &      \\ 
     &           &    2 &  2.14e-08 &  6.11e-16 &      \\ 
     &           &    3 &  2.14e-08 &  6.10e-16 &      \\ 
     &           &    4 &  5.64e-08 &  6.09e-16 &      \\ 
     &           &    5 &  2.14e-08 &  1.61e-15 &      \\ 
     &           &    6 &  2.14e-08 &  6.11e-16 &      \\ 
     &           &    7 &  2.13e-08 &  6.10e-16 &      \\ 
     &           &    8 &  5.64e-08 &  6.09e-16 &      \\ 
     &           &    9 &  2.14e-08 &  1.61e-15 &      \\ 
     &           &   10 &  2.14e-08 &  6.10e-16 &      \\ 
1515 &  1.51e+04 &   10 &           &           & iters  \\ 
 \hdashline 
     &           &    1 &  1.81e-08 &  6.64e-10 &      \\ 
     &           &    2 &  1.80e-08 &  5.16e-16 &      \\ 
     &           &    3 &  1.80e-08 &  5.15e-16 &      \\ 
     &           &    4 &  1.80e-08 &  5.14e-16 &      \\ 
     &           &    5 &  1.80e-08 &  5.14e-16 &      \\ 
     &           &    6 &  5.97e-08 &  5.13e-16 &      \\ 
     &           &    7 &  1.80e-08 &  1.71e-15 &      \\ 
     &           &    8 &  1.80e-08 &  5.15e-16 &      \\ 
     &           &    9 &  1.80e-08 &  5.14e-16 &      \\ 
     &           &   10 &  5.97e-08 &  5.13e-16 &      \\ 
1516 &  1.52e+04 &   10 &           &           & iters  \\ 
 \hdashline 
     &           &    1 &  3.03e-08 &  6.87e-10 &      \\ 
     &           &    2 &  4.74e-08 &  8.65e-16 &      \\ 
     &           &    3 &  3.03e-08 &  1.35e-15 &      \\ 
     &           &    4 &  3.03e-08 &  8.65e-16 &      \\ 
     &           &    5 &  4.75e-08 &  8.64e-16 &      \\ 
     &           &    6 &  3.03e-08 &  1.35e-15 &      \\ 
     &           &    7 &  4.74e-08 &  8.65e-16 &      \\ 
     &           &    8 &  3.03e-08 &  1.35e-15 &      \\ 
     &           &    9 &  3.03e-08 &  8.65e-16 &      \\ 
     &           &   10 &  4.75e-08 &  8.64e-16 &      \\ 
1517 &  1.52e+04 &   10 &           &           & iters  \\ 
 \hdashline 
     &           &    1 &  1.18e-08 &  7.10e-10 &      \\ 
     &           &    2 &  6.59e-08 &  3.38e-16 &      \\ 
     &           &    3 &  1.19e-08 &  1.88e-15 &      \\ 
     &           &    4 &  1.19e-08 &  3.40e-16 &      \\ 
     &           &    5 &  1.19e-08 &  3.39e-16 &      \\ 
     &           &    6 &  1.19e-08 &  3.39e-16 &      \\ 
     &           &    7 &  1.18e-08 &  3.38e-16 &      \\ 
     &           &    8 &  1.18e-08 &  3.38e-16 &      \\ 
     &           &    9 &  6.59e-08 &  3.37e-16 &      \\ 
     &           &   10 &  1.19e-08 &  1.88e-15 &      \\ 
1518 &  1.52e+04 &   10 &           &           & iters  \\ 
 \hdashline 
     &           &    1 &  6.58e-09 &  3.47e-10 &      \\ 
     &           &    2 &  6.57e-09 &  1.88e-16 &      \\ 
     &           &    3 &  6.56e-09 &  1.87e-16 &      \\ 
     &           &    4 &  7.11e-08 &  1.87e-16 &      \\ 
     &           &    5 &  8.43e-08 &  2.03e-15 &      \\ 
     &           &    6 &  7.12e-08 &  2.41e-15 &      \\ 
     &           &    7 &  6.63e-09 &  2.03e-15 &      \\ 
     &           &    8 &  6.62e-09 &  1.89e-16 &      \\ 
     &           &    9 &  6.61e-09 &  1.89e-16 &      \\ 
     &           &   10 &  6.61e-09 &  1.89e-16 &      \\ 
1519 &  1.52e+04 &   10 &           &           & iters  \\ 
 \hdashline 
     &           &    1 &  5.08e-09 &  8.53e-10 &      \\ 
     &           &    2 &  5.08e-09 &  1.45e-16 &      \\ 
     &           &    3 &  5.07e-09 &  1.45e-16 &      \\ 
     &           &    4 &  5.06e-09 &  1.45e-16 &      \\ 
     &           &    5 &  5.06e-09 &  1.44e-16 &      \\ 
     &           &    6 &  5.05e-09 &  1.44e-16 &      \\ 
     &           &    7 &  5.04e-09 &  1.44e-16 &      \\ 
     &           &    8 &  5.03e-09 &  1.44e-16 &      \\ 
     &           &    9 &  5.03e-09 &  1.44e-16 &      \\ 
     &           &   10 &  5.02e-09 &  1.43e-16 &      \\ 
1520 &  1.52e+04 &   10 &           &           & iters  \\ 
 \hdashline 
     &           &    1 &  1.90e-08 &  1.24e-09 &      \\ 
     &           &    2 &  1.90e-08 &  5.44e-16 &      \\ 
     &           &    3 &  1.99e-08 &  5.43e-16 &      \\ 
     &           &    4 &  1.90e-08 &  5.67e-16 &      \\ 
     &           &    5 &  1.99e-08 &  5.43e-16 &      \\ 
     &           &    6 &  1.90e-08 &  5.67e-16 &      \\ 
     &           &    7 &  1.98e-08 &  5.43e-16 &      \\ 
     &           &    8 &  1.90e-08 &  5.67e-16 &      \\ 
     &           &    9 &  1.98e-08 &  5.43e-16 &      \\ 
     &           &   10 &  1.90e-08 &  5.66e-16 &      \\ 
1521 &  1.52e+04 &   10 &           &           & iters  \\ 
 \hdashline 
     &           &    1 &  1.55e-08 &  8.52e-11 &      \\ 
     &           &    2 &  1.54e-08 &  4.41e-16 &      \\ 
     &           &    3 &  6.23e-08 &  4.41e-16 &      \\ 
     &           &    4 &  1.55e-08 &  1.78e-15 &      \\ 
     &           &    5 &  1.55e-08 &  4.43e-16 &      \\ 
     &           &    6 &  1.55e-08 &  4.42e-16 &      \\ 
     &           &    7 &  1.54e-08 &  4.41e-16 &      \\ 
     &           &    8 &  6.23e-08 &  4.41e-16 &      \\ 
     &           &    9 &  1.55e-08 &  1.78e-15 &      \\ 
     &           &   10 &  1.55e-08 &  4.43e-16 &      \\ 
1522 &  1.52e+04 &   10 &           &           & iters  \\ 
 \hdashline 
     &           &    1 &  3.49e-08 &  6.92e-10 &      \\ 
     &           &    2 &  4.28e-08 &  9.97e-16 &      \\ 
     &           &    3 &  3.49e-08 &  1.22e-15 &      \\ 
     &           &    4 &  3.49e-08 &  9.97e-16 &      \\ 
     &           &    5 &  4.29e-08 &  9.96e-16 &      \\ 
     &           &    6 &  3.49e-08 &  1.22e-15 &      \\ 
     &           &    7 &  4.28e-08 &  9.96e-16 &      \\ 
     &           &    8 &  3.49e-08 &  1.22e-15 &      \\ 
     &           &    9 &  4.28e-08 &  9.96e-16 &      \\ 
     &           &   10 &  3.49e-08 &  1.22e-15 &      \\ 
1523 &  1.52e+04 &   10 &           &           & iters  \\ 
 \hdashline 
     &           &    1 &  2.82e-08 &  3.18e-10 &      \\ 
     &           &    2 &  2.82e-08 &  8.05e-16 &      \\ 
     &           &    3 &  2.81e-08 &  8.04e-16 &      \\ 
     &           &    4 &  4.96e-08 &  8.03e-16 &      \\ 
     &           &    5 &  2.82e-08 &  1.42e-15 &      \\ 
     &           &    6 &  2.81e-08 &  8.04e-16 &      \\ 
     &           &    7 &  4.96e-08 &  8.03e-16 &      \\ 
     &           &    8 &  2.82e-08 &  1.42e-15 &      \\ 
     &           &    9 &  4.96e-08 &  8.03e-16 &      \\ 
     &           &   10 &  2.82e-08 &  1.41e-15 &      \\ 
1524 &  1.52e+04 &   10 &           &           & iters  \\ 
 \hdashline 
     &           &    1 &  2.76e-08 &  1.22e-09 &      \\ 
     &           &    2 &  5.01e-08 &  7.87e-16 &      \\ 
     &           &    3 &  2.76e-08 &  1.43e-15 &      \\ 
     &           &    4 &  2.76e-08 &  7.88e-16 &      \\ 
     &           &    5 &  5.02e-08 &  7.87e-16 &      \\ 
     &           &    6 &  2.76e-08 &  1.43e-15 &      \\ 
     &           &    7 &  2.76e-08 &  7.88e-16 &      \\ 
     &           &    8 &  5.02e-08 &  7.87e-16 &      \\ 
     &           &    9 &  2.76e-08 &  1.43e-15 &      \\ 
     &           &   10 &  2.76e-08 &  7.88e-16 &      \\ 
1525 &  1.52e+04 &   10 &           &           & iters  \\ 
 \hdashline 
     &           &    1 &  4.99e-08 &  4.55e-10 &      \\ 
     &           &    2 &  2.79e-08 &  1.42e-15 &      \\ 
     &           &    3 &  2.78e-08 &  7.95e-16 &      \\ 
     &           &    4 &  4.99e-08 &  7.94e-16 &      \\ 
     &           &    5 &  2.78e-08 &  1.42e-15 &      \\ 
     &           &    6 &  4.99e-08 &  7.95e-16 &      \\ 
     &           &    7 &  2.79e-08 &  1.42e-15 &      \\ 
     &           &    8 &  2.78e-08 &  7.96e-16 &      \\ 
     &           &    9 &  4.99e-08 &  7.94e-16 &      \\ 
     &           &   10 &  2.79e-08 &  1.42e-15 &      \\ 
1526 &  1.52e+04 &   10 &           &           & iters  \\ 
 \hdashline 
     &           &    1 &  1.09e-10 &  4.55e-10 &      \\ 
     &           &    2 &  1.09e-10 &  3.10e-18 &      \\ 
     &           &    3 &  1.08e-10 &  3.10e-18 &      \\ 
     &           &    4 &  1.08e-10 &  3.09e-18 &      \\ 
     &           &    5 &  1.08e-10 &  3.09e-18 &      \\ 
     &           &    6 &  1.08e-10 &  3.08e-18 &      \\ 
     &           &    7 &  1.08e-10 &  3.08e-18 &      \\ 
     &           &    8 &  1.08e-10 &  3.08e-18 &      \\ 
     &           &    9 &  1.07e-10 &  3.07e-18 &      \\ 
     &           &   10 &  1.07e-10 &  3.07e-18 &      \\ 
1527 &  1.53e+04 &   10 &           &           & iters  \\ 
 \hdashline 
     &           &    1 &  8.01e-09 &  9.32e-11 &      \\ 
     &           &    2 &  8.00e-09 &  2.29e-16 &      \\ 
     &           &    3 &  7.99e-09 &  2.28e-16 &      \\ 
     &           &    4 &  7.98e-09 &  2.28e-16 &      \\ 
     &           &    5 &  6.97e-08 &  2.28e-16 &      \\ 
     &           &    6 &  8.58e-08 &  1.99e-15 &      \\ 
     &           &    7 &  6.97e-08 &  2.45e-15 &      \\ 
     &           &    8 &  8.04e-09 &  1.99e-15 &      \\ 
     &           &    9 &  8.03e-09 &  2.30e-16 &      \\ 
     &           &   10 &  8.02e-09 &  2.29e-16 &      \\ 
1528 &  1.53e+04 &   10 &           &           & iters  \\ 
 \hdashline 
     &           &    1 &  2.53e-08 &  7.69e-10 &      \\ 
     &           &    2 &  5.24e-08 &  7.22e-16 &      \\ 
     &           &    3 &  2.54e-08 &  1.50e-15 &      \\ 
     &           &    4 &  2.53e-08 &  7.24e-16 &      \\ 
     &           &    5 &  5.24e-08 &  7.23e-16 &      \\ 
     &           &    6 &  2.54e-08 &  1.50e-15 &      \\ 
     &           &    7 &  2.53e-08 &  7.24e-16 &      \\ 
     &           &    8 &  5.24e-08 &  7.23e-16 &      \\ 
     &           &    9 &  2.54e-08 &  1.50e-15 &      \\ 
     &           &   10 &  2.53e-08 &  7.24e-16 &      \\ 
1529 &  1.53e+04 &   10 &           &           & iters  \\ 
 \hdashline 
     &           &    1 &  2.50e-08 &  7.29e-10 &      \\ 
     &           &    2 &  2.50e-08 &  7.14e-16 &      \\ 
     &           &    3 &  5.27e-08 &  7.13e-16 &      \\ 
     &           &    4 &  2.50e-08 &  1.51e-15 &      \\ 
     &           &    5 &  2.50e-08 &  7.14e-16 &      \\ 
     &           &    6 &  5.27e-08 &  7.13e-16 &      \\ 
     &           &    7 &  2.50e-08 &  1.51e-15 &      \\ 
     &           &    8 &  2.50e-08 &  7.14e-16 &      \\ 
     &           &    9 &  5.27e-08 &  7.13e-16 &      \\ 
     &           &   10 &  2.50e-08 &  1.51e-15 &      \\ 
1530 &  1.53e+04 &   10 &           &           & iters  \\ 
 \hdashline 
     &           &    1 &  2.37e-08 &  4.05e-10 &      \\ 
     &           &    2 &  1.52e-08 &  6.76e-16 &      \\ 
     &           &    3 &  1.52e-08 &  4.34e-16 &      \\ 
     &           &    4 &  2.37e-08 &  4.33e-16 &      \\ 
     &           &    5 &  1.52e-08 &  6.76e-16 &      \\ 
     &           &    6 &  2.37e-08 &  4.34e-16 &      \\ 
     &           &    7 &  1.52e-08 &  6.75e-16 &      \\ 
     &           &    8 &  1.52e-08 &  4.34e-16 &      \\ 
     &           &    9 &  2.37e-08 &  4.33e-16 &      \\ 
     &           &   10 &  1.52e-08 &  6.76e-16 &      \\ 
1531 &  1.53e+04 &   10 &           &           & iters  \\ 
 \hdashline 
     &           &    1 &  2.07e-08 &  2.64e-11 &      \\ 
     &           &    2 &  5.70e-08 &  5.91e-16 &      \\ 
     &           &    3 &  2.08e-08 &  1.63e-15 &      \\ 
     &           &    4 &  2.07e-08 &  5.93e-16 &      \\ 
     &           &    5 &  2.07e-08 &  5.92e-16 &      \\ 
     &           &    6 &  5.70e-08 &  5.91e-16 &      \\ 
     &           &    7 &  2.08e-08 &  1.63e-15 &      \\ 
     &           &    8 &  2.07e-08 &  5.93e-16 &      \\ 
     &           &    9 &  2.07e-08 &  5.92e-16 &      \\ 
     &           &   10 &  5.70e-08 &  5.91e-16 &      \\ 
1532 &  1.53e+04 &   10 &           &           & iters  \\ 
 \hdashline 
     &           &    1 &  5.59e-08 &  4.64e-10 &      \\ 
     &           &    2 &  2.19e-08 &  1.60e-15 &      \\ 
     &           &    3 &  2.18e-08 &  6.24e-16 &      \\ 
     &           &    4 &  2.18e-08 &  6.23e-16 &      \\ 
     &           &    5 &  5.59e-08 &  6.22e-16 &      \\ 
     &           &    6 &  2.18e-08 &  1.60e-15 &      \\ 
     &           &    7 &  2.18e-08 &  6.23e-16 &      \\ 
     &           &    8 &  5.59e-08 &  6.22e-16 &      \\ 
     &           &    9 &  2.19e-08 &  1.60e-15 &      \\ 
     &           &   10 &  2.18e-08 &  6.24e-16 &      \\ 
1533 &  1.53e+04 &   10 &           &           & iters  \\ 
 \hdashline 
     &           &    1 &  6.29e-09 &  1.22e-10 &      \\ 
     &           &    2 &  6.28e-09 &  1.79e-16 &      \\ 
     &           &    3 &  6.27e-09 &  1.79e-16 &      \\ 
     &           &    4 &  6.26e-09 &  1.79e-16 &      \\ 
     &           &    5 &  6.25e-09 &  1.79e-16 &      \\ 
     &           &    6 &  6.24e-09 &  1.78e-16 &      \\ 
     &           &    7 &  7.15e-08 &  1.78e-16 &      \\ 
     &           &    8 &  4.52e-08 &  2.04e-15 &      \\ 
     &           &    9 &  6.27e-09 &  1.29e-15 &      \\ 
     &           &   10 &  6.26e-09 &  1.79e-16 &      \\ 
1534 &  1.53e+04 &   10 &           &           & iters  \\ 
 \hdashline 
     &           &    1 &  2.08e-08 &  1.56e-10 &      \\ 
     &           &    2 &  2.08e-08 &  5.94e-16 &      \\ 
     &           &    3 &  1.81e-08 &  5.94e-16 &      \\ 
     &           &    4 &  2.08e-08 &  5.16e-16 &      \\ 
     &           &    5 &  1.81e-08 &  5.93e-16 &      \\ 
     &           &    6 &  2.08e-08 &  5.16e-16 &      \\ 
     &           &    7 &  1.81e-08 &  5.93e-16 &      \\ 
     &           &    8 &  2.08e-08 &  5.16e-16 &      \\ 
     &           &    9 &  1.81e-08 &  5.93e-16 &      \\ 
     &           &   10 &  2.08e-08 &  5.16e-16 &      \\ 
1535 &  1.53e+04 &   10 &           &           & iters  \\ 
 \hdashline 
     &           &    1 &  2.48e-09 &  2.67e-10 &      \\ 
     &           &    2 &  2.48e-09 &  7.09e-17 &      \\ 
     &           &    3 &  2.48e-09 &  7.08e-17 &      \\ 
     &           &    4 &  2.47e-09 &  7.07e-17 &      \\ 
     &           &    5 &  7.52e-08 &  7.06e-17 &      \\ 
     &           &    6 &  8.03e-08 &  2.15e-15 &      \\ 
     &           &    7 &  7.52e-08 &  2.29e-15 &      \\ 
     &           &    8 &  8.03e-08 &  2.15e-15 &      \\ 
     &           &    9 &  7.52e-08 &  2.29e-15 &      \\ 
     &           &   10 &  8.03e-08 &  2.15e-15 &      \\ 
1536 &  1.54e+04 &   10 &           &           & iters  \\ 
 \hdashline 
     &           &    1 &  5.71e-08 &  3.55e-10 &      \\ 
     &           &    2 &  1.82e-08 &  1.63e-15 &      \\ 
     &           &    3 &  5.95e-08 &  5.19e-16 &      \\ 
     &           &    4 &  1.82e-08 &  1.70e-15 &      \\ 
     &           &    5 &  1.82e-08 &  5.20e-16 &      \\ 
     &           &    6 &  1.82e-08 &  5.20e-16 &      \\ 
     &           &    7 &  1.82e-08 &  5.19e-16 &      \\ 
     &           &    8 &  5.96e-08 &  5.18e-16 &      \\ 
     &           &    9 &  1.82e-08 &  1.70e-15 &      \\ 
     &           &   10 &  1.82e-08 &  5.20e-16 &      \\ 
1537 &  1.54e+04 &   10 &           &           & iters  \\ 
 \hdashline 
     &           &    1 &  3.24e-08 &  1.63e-10 &      \\ 
     &           &    2 &  6.51e-09 &  9.24e-16 &      \\ 
     &           &    3 &  6.50e-09 &  1.86e-16 &      \\ 
     &           &    4 &  6.49e-09 &  1.85e-16 &      \\ 
     &           &    5 &  6.48e-09 &  1.85e-16 &      \\ 
     &           &    6 &  6.47e-09 &  1.85e-16 &      \\ 
     &           &    7 &  3.24e-08 &  1.85e-16 &      \\ 
     &           &    8 &  6.51e-09 &  9.24e-16 &      \\ 
     &           &    9 &  6.50e-09 &  1.86e-16 &      \\ 
     &           &   10 &  6.49e-09 &  1.85e-16 &      \\ 
1538 &  1.54e+04 &   10 &           &           & iters  \\ 
 \hdashline 
     &           &    1 &  1.15e-08 &  3.33e-10 &      \\ 
     &           &    2 &  1.15e-08 &  3.29e-16 &      \\ 
     &           &    3 &  1.15e-08 &  3.29e-16 &      \\ 
     &           &    4 &  6.62e-08 &  3.28e-16 &      \\ 
     &           &    5 &  1.16e-08 &  1.89e-15 &      \\ 
     &           &    6 &  1.16e-08 &  3.31e-16 &      \\ 
     &           &    7 &  1.16e-08 &  3.30e-16 &      \\ 
     &           &    8 &  1.15e-08 &  3.30e-16 &      \\ 
     &           &    9 &  1.15e-08 &  3.29e-16 &      \\ 
     &           &   10 &  1.15e-08 &  3.29e-16 &      \\ 
1539 &  1.54e+04 &   10 &           &           & iters  \\ 
 \hdashline 
     &           &    1 &  1.18e-08 &  2.04e-10 &      \\ 
     &           &    2 &  1.18e-08 &  3.37e-16 &      \\ 
     &           &    3 &  1.18e-08 &  3.37e-16 &      \\ 
     &           &    4 &  1.18e-08 &  3.36e-16 &      \\ 
     &           &    5 &  1.18e-08 &  3.36e-16 &      \\ 
     &           &    6 &  6.59e-08 &  3.36e-16 &      \\ 
     &           &    7 &  1.18e-08 &  1.88e-15 &      \\ 
     &           &    8 &  1.18e-08 &  3.38e-16 &      \\ 
     &           &    9 &  1.18e-08 &  3.37e-16 &      \\ 
     &           &   10 &  1.18e-08 &  3.37e-16 &      \\ 
1540 &  1.54e+04 &   10 &           &           & iters  \\ 
 \hdashline 
     &           &    1 &  1.78e-08 &  2.90e-09 &      \\ 
     &           &    2 &  5.99e-08 &  5.08e-16 &      \\ 
     &           &    3 &  1.79e-08 &  1.71e-15 &      \\ 
     &           &    4 &  1.78e-08 &  5.10e-16 &      \\ 
     &           &    5 &  1.78e-08 &  5.09e-16 &      \\ 
     &           &    6 &  5.99e-08 &  5.08e-16 &      \\ 
     &           &    7 &  1.79e-08 &  1.71e-15 &      \\ 
     &           &    8 &  1.79e-08 &  5.10e-16 &      \\ 
     &           &    9 &  1.78e-08 &  5.09e-16 &      \\ 
     &           &   10 &  1.78e-08 &  5.09e-16 &      \\ 
1541 &  1.54e+04 &   10 &           &           & iters  \\ 
 \hdashline 
     &           &    1 &  1.32e-08 &  3.68e-09 &      \\ 
     &           &    2 &  1.32e-08 &  3.77e-16 &      \\ 
     &           &    3 &  1.32e-08 &  3.77e-16 &      \\ 
     &           &    4 &  1.32e-08 &  3.76e-16 &      \\ 
     &           &    5 &  1.31e-08 &  3.76e-16 &      \\ 
     &           &    6 &  6.46e-08 &  3.75e-16 &      \\ 
     &           &    7 &  1.32e-08 &  1.84e-15 &      \\ 
     &           &    8 &  1.32e-08 &  3.77e-16 &      \\ 
     &           &    9 &  1.32e-08 &  3.77e-16 &      \\ 
     &           &   10 &  1.32e-08 &  3.76e-16 &      \\ 
1542 &  1.54e+04 &   10 &           &           & iters  \\ 
 \hdashline 
     &           &    1 &  4.31e-08 &  1.34e-09 &      \\ 
     &           &    2 &  3.46e-08 &  1.23e-15 &      \\ 
     &           &    3 &  4.31e-08 &  9.88e-16 &      \\ 
     &           &    4 &  3.46e-08 &  1.23e-15 &      \\ 
     &           &    5 &  3.46e-08 &  9.88e-16 &      \\ 
     &           &    6 &  4.32e-08 &  9.86e-16 &      \\ 
     &           &    7 &  3.46e-08 &  1.23e-15 &      \\ 
     &           &    8 &  4.32e-08 &  9.87e-16 &      \\ 
     &           &    9 &  3.46e-08 &  1.23e-15 &      \\ 
     &           &   10 &  4.31e-08 &  9.87e-16 &      \\ 
1543 &  1.54e+04 &   10 &           &           & iters  \\ 
 \hdashline 
     &           &    1 &  1.78e-08 &  4.84e-10 &      \\ 
     &           &    2 &  1.78e-08 &  5.07e-16 &      \\ 
     &           &    3 &  1.77e-08 &  5.07e-16 &      \\ 
     &           &    4 &  6.00e-08 &  5.06e-16 &      \\ 
     &           &    5 &  1.78e-08 &  1.71e-15 &      \\ 
     &           &    6 &  1.78e-08 &  5.08e-16 &      \\ 
     &           &    7 &  1.77e-08 &  5.07e-16 &      \\ 
     &           &    8 &  6.00e-08 &  5.06e-16 &      \\ 
     &           &    9 &  1.78e-08 &  1.71e-15 &      \\ 
     &           &   10 &  1.78e-08 &  5.08e-16 &      \\ 
1544 &  1.54e+04 &   10 &           &           & iters  \\ 
 \hdashline 
     &           &    1 &  5.19e-08 &  1.41e-09 &      \\ 
     &           &    2 &  2.59e-08 &  1.48e-15 &      \\ 
     &           &    3 &  5.18e-08 &  7.39e-16 &      \\ 
     &           &    4 &  2.59e-08 &  1.48e-15 &      \\ 
     &           &    5 &  2.59e-08 &  7.40e-16 &      \\ 
     &           &    6 &  5.18e-08 &  7.39e-16 &      \\ 
     &           &    7 &  2.59e-08 &  1.48e-15 &      \\ 
     &           &    8 &  2.59e-08 &  7.40e-16 &      \\ 
     &           &    9 &  5.18e-08 &  7.39e-16 &      \\ 
     &           &   10 &  2.59e-08 &  1.48e-15 &      \\ 
1545 &  1.54e+04 &   10 &           &           & iters  \\ 
 \hdashline 
     &           &    1 &  1.27e-09 &  2.46e-09 &      \\ 
     &           &    2 &  1.27e-09 &  3.62e-17 &      \\ 
     &           &    3 &  1.27e-09 &  3.62e-17 &      \\ 
     &           &    4 &  1.26e-09 &  3.61e-17 &      \\ 
     &           &    5 &  1.26e-09 &  3.61e-17 &      \\ 
     &           &    6 &  1.26e-09 &  3.60e-17 &      \\ 
     &           &    7 &  1.26e-09 &  3.60e-17 &      \\ 
     &           &    8 &  1.26e-09 &  3.59e-17 &      \\ 
     &           &    9 &  1.25e-09 &  3.59e-17 &      \\ 
     &           &   10 &  1.25e-09 &  3.58e-17 &      \\ 
1546 &  1.54e+04 &   10 &           &           & iters  \\ 
 \hdashline 
     &           &    1 &  3.21e-08 &  2.52e-09 &      \\ 
     &           &    2 &  6.75e-09 &  9.17e-16 &      \\ 
     &           &    3 &  6.74e-09 &  1.93e-16 &      \\ 
     &           &    4 &  6.73e-09 &  1.92e-16 &      \\ 
     &           &    5 &  6.72e-09 &  1.92e-16 &      \\ 
     &           &    6 &  3.21e-08 &  1.92e-16 &      \\ 
     &           &    7 &  6.76e-09 &  9.17e-16 &      \\ 
     &           &    8 &  6.75e-09 &  1.93e-16 &      \\ 
     &           &    9 &  6.74e-09 &  1.93e-16 &      \\ 
     &           &   10 &  6.73e-09 &  1.92e-16 &      \\ 
1547 &  1.55e+04 &   10 &           &           & iters  \\ 
 \hdashline 
     &           &    1 &  4.25e-08 &  2.04e-09 &      \\ 
     &           &    2 &  3.52e-08 &  1.21e-15 &      \\ 
     &           &    3 &  4.25e-08 &  1.01e-15 &      \\ 
     &           &    4 &  3.52e-08 &  1.21e-15 &      \\ 
     &           &    5 &  3.52e-08 &  1.01e-15 &      \\ 
     &           &    6 &  4.26e-08 &  1.00e-15 &      \\ 
     &           &    7 &  3.52e-08 &  1.21e-15 &      \\ 
     &           &    8 &  4.25e-08 &  1.00e-15 &      \\ 
     &           &    9 &  3.52e-08 &  1.21e-15 &      \\ 
     &           &   10 &  4.25e-08 &  1.00e-15 &      \\ 
1548 &  1.55e+04 &   10 &           &           & iters  \\ 
 \hdashline 
     &           &    1 &  3.22e-08 &  1.77e-09 &      \\ 
     &           &    2 &  3.21e-08 &  9.19e-16 &      \\ 
     &           &    3 &  4.56e-08 &  9.18e-16 &      \\ 
     &           &    4 &  3.22e-08 &  1.30e-15 &      \\ 
     &           &    5 &  4.56e-08 &  9.18e-16 &      \\ 
     &           &    6 &  3.22e-08 &  1.30e-15 &      \\ 
     &           &    7 &  4.55e-08 &  9.19e-16 &      \\ 
     &           &    8 &  3.22e-08 &  1.30e-15 &      \\ 
     &           &    9 &  3.22e-08 &  9.19e-16 &      \\ 
     &           &   10 &  4.56e-08 &  9.18e-16 &      \\ 
1549 &  1.55e+04 &   10 &           &           & iters  \\ 
 \hdashline 
     &           &    1 &  4.06e-09 &  1.85e-09 &      \\ 
     &           &    2 &  4.06e-09 &  1.16e-16 &      \\ 
     &           &    3 &  4.05e-09 &  1.16e-16 &      \\ 
     &           &    4 &  7.36e-08 &  1.16e-16 &      \\ 
     &           &    5 &  8.18e-08 &  2.10e-15 &      \\ 
     &           &    6 &  7.37e-08 &  2.34e-15 &      \\ 
     &           &    7 &  8.18e-08 &  2.10e-15 &      \\ 
     &           &    8 &  7.37e-08 &  2.34e-15 &      \\ 
     &           &    9 &  4.13e-09 &  2.10e-15 &      \\ 
     &           &   10 &  4.12e-09 &  1.18e-16 &      \\ 
1550 &  1.55e+04 &   10 &           &           & iters  \\ 
 \hdashline 
     &           &    1 &  1.13e-08 &  2.85e-09 &      \\ 
     &           &    2 &  1.13e-08 &  3.24e-16 &      \\ 
     &           &    3 &  1.13e-08 &  3.23e-16 &      \\ 
     &           &    4 &  1.13e-08 &  3.23e-16 &      \\ 
     &           &    5 &  1.13e-08 &  3.22e-16 &      \\ 
     &           &    6 &  6.64e-08 &  3.22e-16 &      \\ 
     &           &    7 &  5.02e-08 &  1.90e-15 &      \\ 
     &           &    8 &  1.13e-08 &  1.43e-15 &      \\ 
     &           &    9 &  1.13e-08 &  3.22e-16 &      \\ 
     &           &   10 &  6.64e-08 &  3.22e-16 &      \\ 
1551 &  1.55e+04 &   10 &           &           & iters  \\ 
 \hdashline 
     &           &    1 &  3.37e-08 &  2.60e-09 &      \\ 
     &           &    2 &  5.22e-09 &  9.61e-16 &      \\ 
     &           &    3 &  5.22e-09 &  1.49e-16 &      \\ 
     &           &    4 &  5.21e-09 &  1.49e-16 &      \\ 
     &           &    5 &  5.20e-09 &  1.49e-16 &      \\ 
     &           &    6 &  5.19e-09 &  1.48e-16 &      \\ 
     &           &    7 &  5.19e-09 &  1.48e-16 &      \\ 
     &           &    8 &  3.37e-08 &  1.48e-16 &      \\ 
     &           &    9 &  5.23e-09 &  9.61e-16 &      \\ 
     &           &   10 &  5.22e-09 &  1.49e-16 &      \\ 
1552 &  1.55e+04 &   10 &           &           & iters  \\ 
 \hdashline 
     &           &    1 &  1.43e-08 &  7.12e-10 &      \\ 
     &           &    2 &  1.43e-08 &  4.08e-16 &      \\ 
     &           &    3 &  1.42e-08 &  4.07e-16 &      \\ 
     &           &    4 &  6.35e-08 &  4.06e-16 &      \\ 
     &           &    5 &  1.43e-08 &  1.81e-15 &      \\ 
     &           &    6 &  1.43e-08 &  4.08e-16 &      \\ 
     &           &    7 &  1.43e-08 &  4.08e-16 &      \\ 
     &           &    8 &  1.43e-08 &  4.07e-16 &      \\ 
     &           &    9 &  6.35e-08 &  4.07e-16 &      \\ 
     &           &   10 &  1.43e-08 &  1.81e-15 &      \\ 
1553 &  1.55e+04 &   10 &           &           & iters  \\ 
 \hdashline 
     &           &    1 &  1.91e-08 &  3.57e-09 &      \\ 
     &           &    2 &  1.91e-08 &  5.45e-16 &      \\ 
     &           &    3 &  5.86e-08 &  5.45e-16 &      \\ 
     &           &    4 &  1.91e-08 &  1.67e-15 &      \\ 
     &           &    5 &  1.91e-08 &  5.46e-16 &      \\ 
     &           &    6 &  1.91e-08 &  5.45e-16 &      \\ 
     &           &    7 &  1.91e-08 &  5.45e-16 &      \\ 
     &           &    8 &  5.87e-08 &  5.44e-16 &      \\ 
     &           &    9 &  1.91e-08 &  1.67e-15 &      \\ 
     &           &   10 &  1.91e-08 &  5.46e-16 &      \\ 
1554 &  1.55e+04 &   10 &           &           & iters  \\ 
 \hdashline 
     &           &    1 &  4.86e-09 &  2.42e-09 &      \\ 
     &           &    2 &  4.85e-09 &  1.39e-16 &      \\ 
     &           &    3 &  4.85e-09 &  1.39e-16 &      \\ 
     &           &    4 &  4.84e-09 &  1.38e-16 &      \\ 
     &           &    5 &  4.83e-09 &  1.38e-16 &      \\ 
     &           &    6 &  4.83e-09 &  1.38e-16 &      \\ 
     &           &    7 &  4.82e-09 &  1.38e-16 &      \\ 
     &           &    8 &  4.81e-09 &  1.38e-16 &      \\ 
     &           &    9 &  4.81e-09 &  1.37e-16 &      \\ 
     &           &   10 &  4.80e-09 &  1.37e-16 &      \\ 
1555 &  1.55e+04 &   10 &           &           & iters  \\ 
 \hdashline 
     &           &    1 &  7.03e-10 &  7.25e-10 &      \\ 
     &           &    2 &  3.81e-08 &  2.01e-17 &      \\ 
     &           &    3 &  7.57e-10 &  1.09e-15 &      \\ 
     &           &    4 &  7.56e-10 &  2.16e-17 &      \\ 
     &           &    5 &  7.55e-10 &  2.16e-17 &      \\ 
     &           &    6 &  7.54e-10 &  2.15e-17 &      \\ 
     &           &    7 &  7.53e-10 &  2.15e-17 &      \\ 
     &           &    8 &  7.52e-10 &  2.15e-17 &      \\ 
     &           &    9 &  7.51e-10 &  2.15e-17 &      \\ 
     &           &   10 &  7.49e-10 &  2.14e-17 &      \\ 
1556 &  1.56e+04 &   10 &           &           & iters  \\ 
 \hdashline 
     &           &    1 &  3.96e-08 &  1.36e-10 &      \\ 
     &           &    2 &  3.81e-08 &  1.13e-15 &      \\ 
     &           &    3 &  3.81e-08 &  1.09e-15 &      \\ 
     &           &    4 &  3.97e-08 &  1.09e-15 &      \\ 
     &           &    5 &  3.81e-08 &  1.13e-15 &      \\ 
     &           &    6 &  3.97e-08 &  1.09e-15 &      \\ 
     &           &    7 &  3.81e-08 &  1.13e-15 &      \\ 
     &           &    8 &  3.97e-08 &  1.09e-15 &      \\ 
     &           &    9 &  3.81e-08 &  1.13e-15 &      \\ 
     &           &   10 &  3.97e-08 &  1.09e-15 &      \\ 
1557 &  1.56e+04 &   10 &           &           & iters  \\ 
 \hdashline 
     &           &    1 &  2.21e-08 &  3.59e-10 &      \\ 
     &           &    2 &  1.68e-08 &  6.30e-16 &      \\ 
     &           &    3 &  1.68e-08 &  4.79e-16 &      \\ 
     &           &    4 &  2.21e-08 &  4.78e-16 &      \\ 
     &           &    5 &  1.68e-08 &  6.31e-16 &      \\ 
     &           &    6 &  2.21e-08 &  4.79e-16 &      \\ 
     &           &    7 &  1.68e-08 &  6.31e-16 &      \\ 
     &           &    8 &  2.21e-08 &  4.79e-16 &      \\ 
     &           &    9 &  1.68e-08 &  6.30e-16 &      \\ 
     &           &   10 &  1.68e-08 &  4.79e-16 &      \\ 
1558 &  1.56e+04 &   10 &           &           & iters  \\ 
 \hdashline 
     &           &    1 &  4.79e-10 &  2.98e-10 &      \\ 
     &           &    2 &  4.78e-10 &  1.37e-17 &      \\ 
     &           &    3 &  4.77e-10 &  1.36e-17 &      \\ 
     &           &    4 &  4.77e-10 &  1.36e-17 &      \\ 
     &           &    5 &  4.76e-10 &  1.36e-17 &      \\ 
     &           &    6 &  4.75e-10 &  1.36e-17 &      \\ 
     &           &    7 &  4.74e-10 &  1.36e-17 &      \\ 
     &           &    8 &  4.74e-10 &  1.35e-17 &      \\ 
     &           &    9 &  4.73e-10 &  1.35e-17 &      \\ 
     &           &   10 &  4.72e-10 &  1.35e-17 &      \\ 
1559 &  1.56e+04 &   10 &           &           & iters  \\ 
 \hdashline 
     &           &    1 &  1.94e-08 &  3.53e-10 &      \\ 
     &           &    2 &  1.95e-08 &  5.53e-16 &      \\ 
     &           &    3 &  1.94e-08 &  5.57e-16 &      \\ 
     &           &    4 &  1.95e-08 &  5.53e-16 &      \\ 
     &           &    5 &  1.94e-08 &  5.57e-16 &      \\ 
     &           &    6 &  1.95e-08 &  5.53e-16 &      \\ 
     &           &    7 &  1.94e-08 &  5.57e-16 &      \\ 
     &           &    8 &  1.95e-08 &  5.53e-16 &      \\ 
     &           &    9 &  1.94e-08 &  5.57e-16 &      \\ 
     &           &   10 &  1.95e-08 &  5.53e-16 &      \\ 
1560 &  1.56e+04 &   10 &           &           & iters  \\ 
 \hdashline 
     &           &    1 &  1.49e-09 &  6.69e-10 &      \\ 
     &           &    2 &  1.49e-09 &  4.25e-17 &      \\ 
     &           &    3 &  1.49e-09 &  4.25e-17 &      \\ 
     &           &    4 &  1.48e-09 &  4.24e-17 &      \\ 
     &           &    5 &  1.48e-09 &  4.24e-17 &      \\ 
     &           &    6 &  1.48e-09 &  4.23e-17 &      \\ 
     &           &    7 &  1.48e-09 &  4.22e-17 &      \\ 
     &           &    8 &  1.48e-09 &  4.22e-17 &      \\ 
     &           &    9 &  1.47e-09 &  4.21e-17 &      \\ 
     &           &   10 &  1.47e-09 &  4.21e-17 &      \\ 
1561 &  1.56e+04 &   10 &           &           & iters  \\ 
 \hdashline 
     &           &    1 &  2.82e-08 &  1.87e-09 &      \\ 
     &           &    2 &  4.95e-08 &  8.06e-16 &      \\ 
     &           &    3 &  2.83e-08 &  1.41e-15 &      \\ 
     &           &    4 &  2.82e-08 &  8.07e-16 &      \\ 
     &           &    5 &  4.95e-08 &  8.06e-16 &      \\ 
     &           &    6 &  2.83e-08 &  1.41e-15 &      \\ 
     &           &    7 &  4.95e-08 &  8.07e-16 &      \\ 
     &           &    8 &  2.83e-08 &  1.41e-15 &      \\ 
     &           &    9 &  2.83e-08 &  8.08e-16 &      \\ 
     &           &   10 &  4.95e-08 &  8.06e-16 &      \\ 
1562 &  1.56e+04 &   10 &           &           & iters  \\ 
 \hdashline 
     &           &    1 &  1.60e-08 &  3.83e-10 &      \\ 
     &           &    2 &  1.60e-08 &  4.57e-16 &      \\ 
     &           &    3 &  1.60e-08 &  4.56e-16 &      \\ 
     &           &    4 &  2.29e-08 &  4.56e-16 &      \\ 
     &           &    5 &  1.60e-08 &  6.54e-16 &      \\ 
     &           &    6 &  1.60e-08 &  4.56e-16 &      \\ 
     &           &    7 &  2.29e-08 &  4.55e-16 &      \\ 
     &           &    8 &  1.60e-08 &  6.54e-16 &      \\ 
     &           &    9 &  2.29e-08 &  4.56e-16 &      \\ 
     &           &   10 &  1.60e-08 &  6.54e-16 &      \\ 
1563 &  1.56e+04 &   10 &           &           & iters  \\ 
 \hdashline 
     &           &    1 &  2.84e-08 &  1.15e-09 &      \\ 
     &           &    2 &  4.94e-08 &  8.10e-16 &      \\ 
     &           &    3 &  2.84e-08 &  1.41e-15 &      \\ 
     &           &    4 &  4.93e-08 &  8.11e-16 &      \\ 
     &           &    5 &  2.84e-08 &  1.41e-15 &      \\ 
     &           &    6 &  2.84e-08 &  8.11e-16 &      \\ 
     &           &    7 &  4.93e-08 &  8.10e-16 &      \\ 
     &           &    8 &  2.84e-08 &  1.41e-15 &      \\ 
     &           &    9 &  2.84e-08 &  8.11e-16 &      \\ 
     &           &   10 &  4.93e-08 &  8.10e-16 &      \\ 
1564 &  1.56e+04 &   10 &           &           & iters  \\ 
 \hdashline 
     &           &    1 &  1.51e-08 &  4.01e-10 &      \\ 
     &           &    2 &  6.26e-08 &  4.31e-16 &      \\ 
     &           &    3 &  1.52e-08 &  1.79e-15 &      \\ 
     &           &    4 &  1.51e-08 &  4.33e-16 &      \\ 
     &           &    5 &  1.51e-08 &  4.32e-16 &      \\ 
     &           &    6 &  1.51e-08 &  4.31e-16 &      \\ 
     &           &    7 &  6.26e-08 &  4.31e-16 &      \\ 
     &           &    8 &  1.52e-08 &  1.79e-15 &      \\ 
     &           &    9 &  1.51e-08 &  4.33e-16 &      \\ 
     &           &   10 &  1.51e-08 &  4.32e-16 &      \\ 
1565 &  1.56e+04 &   10 &           &           & iters  \\ 
 \hdashline 
     &           &    1 &  2.67e-08 &  9.18e-10 &      \\ 
     &           &    2 &  5.10e-08 &  7.62e-16 &      \\ 
     &           &    3 &  2.67e-08 &  1.46e-15 &      \\ 
     &           &    4 &  2.67e-08 &  7.63e-16 &      \\ 
     &           &    5 &  5.10e-08 &  7.62e-16 &      \\ 
     &           &    6 &  2.67e-08 &  1.46e-15 &      \\ 
     &           &    7 &  5.10e-08 &  7.63e-16 &      \\ 
     &           &    8 &  2.68e-08 &  1.46e-15 &      \\ 
     &           &    9 &  2.67e-08 &  7.64e-16 &      \\ 
     &           &   10 &  5.10e-08 &  7.63e-16 &      \\ 
1566 &  1.56e+04 &   10 &           &           & iters  \\ 
 \hdashline 
     &           &    1 &  4.16e-08 &  1.80e-09 &      \\ 
     &           &    2 &  3.61e-08 &  1.19e-15 &      \\ 
     &           &    3 &  4.16e-08 &  1.03e-15 &      \\ 
     &           &    4 &  3.62e-08 &  1.19e-15 &      \\ 
     &           &    5 &  4.16e-08 &  1.03e-15 &      \\ 
     &           &    6 &  3.62e-08 &  1.19e-15 &      \\ 
     &           &    7 &  4.16e-08 &  1.03e-15 &      \\ 
     &           &    8 &  3.62e-08 &  1.19e-15 &      \\ 
     &           &    9 &  3.61e-08 &  1.03e-15 &      \\ 
     &           &   10 &  4.16e-08 &  1.03e-15 &      \\ 
1567 &  1.57e+04 &   10 &           &           & iters  \\ 
 \hdashline 
     &           &    1 &  3.58e-08 &  1.08e-09 &      \\ 
     &           &    2 &  4.19e-08 &  1.02e-15 &      \\ 
     &           &    3 &  3.58e-08 &  1.20e-15 &      \\ 
     &           &    4 &  4.19e-08 &  1.02e-15 &      \\ 
     &           &    5 &  3.59e-08 &  1.20e-15 &      \\ 
     &           &    6 &  4.19e-08 &  1.02e-15 &      \\ 
     &           &    7 &  3.59e-08 &  1.20e-15 &      \\ 
     &           &    8 &  4.19e-08 &  1.02e-15 &      \\ 
     &           &    9 &  3.59e-08 &  1.20e-15 &      \\ 
     &           &   10 &  4.19e-08 &  1.02e-15 &      \\ 
1568 &  1.57e+04 &   10 &           &           & iters  \\ 
 \hdashline 
     &           &    1 &  1.63e-08 &  2.03e-10 &      \\ 
     &           &    2 &  1.63e-08 &  4.66e-16 &      \\ 
     &           &    3 &  1.63e-08 &  4.65e-16 &      \\ 
     &           &    4 &  1.63e-08 &  4.64e-16 &      \\ 
     &           &    5 &  6.15e-08 &  4.64e-16 &      \\ 
     &           &    6 &  1.63e-08 &  1.75e-15 &      \\ 
     &           &    7 &  1.63e-08 &  4.66e-16 &      \\ 
     &           &    8 &  1.63e-08 &  4.65e-16 &      \\ 
     &           &    9 &  1.62e-08 &  4.64e-16 &      \\ 
     &           &   10 &  6.15e-08 &  4.64e-16 &      \\ 
1569 &  1.57e+04 &   10 &           &           & iters  \\ 
 \hdashline 
     &           &    1 &  4.96e-08 &  7.32e-10 &      \\ 
     &           &    2 &  2.81e-08 &  1.42e-15 &      \\ 
     &           &    3 &  2.81e-08 &  8.03e-16 &      \\ 
     &           &    4 &  4.96e-08 &  8.02e-16 &      \\ 
     &           &    5 &  2.81e-08 &  1.42e-15 &      \\ 
     &           &    6 &  4.96e-08 &  8.03e-16 &      \\ 
     &           &    7 &  2.82e-08 &  1.42e-15 &      \\ 
     &           &    8 &  2.81e-08 &  8.04e-16 &      \\ 
     &           &    9 &  4.96e-08 &  8.02e-16 &      \\ 
     &           &   10 &  2.81e-08 &  1.42e-15 &      \\ 
1570 &  1.57e+04 &   10 &           &           & iters  \\ 
 \hdashline 
     &           &    1 &  3.00e-08 &  2.92e-10 &      \\ 
     &           &    2 &  4.78e-08 &  8.55e-16 &      \\ 
     &           &    3 &  3.00e-08 &  1.36e-15 &      \\ 
     &           &    4 &  4.77e-08 &  8.56e-16 &      \\ 
     &           &    5 &  3.00e-08 &  1.36e-15 &      \\ 
     &           &    6 &  3.00e-08 &  8.57e-16 &      \\ 
     &           &    7 &  4.77e-08 &  8.56e-16 &      \\ 
     &           &    8 &  3.00e-08 &  1.36e-15 &      \\ 
     &           &    9 &  3.00e-08 &  8.56e-16 &      \\ 
     &           &   10 &  4.78e-08 &  8.55e-16 &      \\ 
1571 &  1.57e+04 &   10 &           &           & iters  \\ 
 \hdashline 
     &           &    1 &  9.03e-09 &  2.61e-09 &      \\ 
     &           &    2 &  9.02e-09 &  2.58e-16 &      \\ 
     &           &    3 &  9.01e-09 &  2.57e-16 &      \\ 
     &           &    4 &  9.00e-09 &  2.57e-16 &      \\ 
     &           &    5 &  8.98e-09 &  2.57e-16 &      \\ 
     &           &    6 &  6.87e-08 &  2.56e-16 &      \\ 
     &           &    7 &  8.68e-08 &  1.96e-15 &      \\ 
     &           &    8 &  6.87e-08 &  2.48e-15 &      \\ 
     &           &    9 &  9.04e-09 &  1.96e-15 &      \\ 
     &           &   10 &  9.03e-09 &  2.58e-16 &      \\ 
1572 &  1.57e+04 &   10 &           &           & iters  \\ 
 \hdashline 
     &           &    1 &  1.55e-08 &  2.51e-09 &      \\ 
     &           &    2 &  1.55e-08 &  4.43e-16 &      \\ 
     &           &    3 &  1.55e-08 &  4.42e-16 &      \\ 
     &           &    4 &  6.22e-08 &  4.42e-16 &      \\ 
     &           &    5 &  1.55e-08 &  1.78e-15 &      \\ 
     &           &    6 &  1.55e-08 &  4.44e-16 &      \\ 
     &           &    7 &  1.55e-08 &  4.43e-16 &      \\ 
     &           &    8 &  1.55e-08 &  4.42e-16 &      \\ 
     &           &    9 &  6.22e-08 &  4.42e-16 &      \\ 
     &           &   10 &  1.55e-08 &  1.78e-15 &      \\ 
1573 &  1.57e+04 &   10 &           &           & iters  \\ 
 \hdashline 
     &           &    1 &  1.64e-08 &  1.92e-09 &      \\ 
     &           &    2 &  2.25e-08 &  4.68e-16 &      \\ 
     &           &    3 &  1.64e-08 &  6.42e-16 &      \\ 
     &           &    4 &  1.64e-08 &  4.68e-16 &      \\ 
     &           &    5 &  2.25e-08 &  4.67e-16 &      \\ 
     &           &    6 &  1.64e-08 &  6.42e-16 &      \\ 
     &           &    7 &  2.25e-08 &  4.68e-16 &      \\ 
     &           &    8 &  1.64e-08 &  6.42e-16 &      \\ 
     &           &    9 &  2.25e-08 &  4.68e-16 &      \\ 
     &           &   10 &  1.64e-08 &  6.42e-16 &      \\ 
1574 &  1.57e+04 &   10 &           &           & iters  \\ 
 \hdashline 
     &           &    1 &  2.71e-08 &  3.41e-09 &      \\ 
     &           &    2 &  5.07e-08 &  7.72e-16 &      \\ 
     &           &    3 &  2.71e-08 &  1.45e-15 &      \\ 
     &           &    4 &  2.70e-08 &  7.73e-16 &      \\ 
     &           &    5 &  5.07e-08 &  7.72e-16 &      \\ 
     &           &    6 &  2.71e-08 &  1.45e-15 &      \\ 
     &           &    7 &  2.70e-08 &  7.73e-16 &      \\ 
     &           &    8 &  5.07e-08 &  7.72e-16 &      \\ 
     &           &    9 &  2.71e-08 &  1.45e-15 &      \\ 
     &           &   10 &  2.70e-08 &  7.73e-16 &      \\ 
1575 &  1.57e+04 &   10 &           &           & iters  \\ 
 \hdashline 
     &           &    1 &  2.85e-08 &  3.39e-09 &      \\ 
     &           &    2 &  2.84e-08 &  8.12e-16 &      \\ 
     &           &    3 &  1.05e-08 &  8.11e-16 &      \\ 
     &           &    4 &  1.04e-08 &  2.99e-16 &      \\ 
     &           &    5 &  1.04e-08 &  2.98e-16 &      \\ 
     &           &    6 &  2.84e-08 &  2.98e-16 &      \\ 
     &           &    7 &  1.05e-08 &  8.11e-16 &      \\ 
     &           &    8 &  1.04e-08 &  2.99e-16 &      \\ 
     &           &    9 &  1.04e-08 &  2.98e-16 &      \\ 
     &           &   10 &  2.84e-08 &  2.98e-16 &      \\ 
1576 &  1.58e+04 &   10 &           &           & iters  \\ 
 \hdashline 
     &           &    1 &  3.99e-08 &  1.78e-09 &      \\ 
     &           &    2 &  1.01e-09 &  1.14e-15 &      \\ 
     &           &    3 &  1.01e-09 &  2.88e-17 &      \\ 
     &           &    4 &  1.01e-09 &  2.88e-17 &      \\ 
     &           &    5 &  1.01e-09 &  2.87e-17 &      \\ 
     &           &    6 &  1.00e-09 &  2.87e-17 &      \\ 
     &           &    7 &  1.00e-09 &  2.87e-17 &      \\ 
     &           &    8 &  1.00e-09 &  2.86e-17 &      \\ 
     &           &    9 &  1.00e-09 &  2.86e-17 &      \\ 
     &           &   10 &  9.98e-10 &  2.85e-17 &      \\ 
1577 &  1.58e+04 &   10 &           &           & iters  \\ 
 \hdashline 
     &           &    1 &  3.65e-08 &  5.54e-10 &      \\ 
     &           &    2 &  3.64e-08 &  1.04e-15 &      \\ 
     &           &    3 &  4.13e-08 &  1.04e-15 &      \\ 
     &           &    4 &  3.65e-08 &  1.18e-15 &      \\ 
     &           &    5 &  4.13e-08 &  1.04e-15 &      \\ 
     &           &    6 &  3.65e-08 &  1.18e-15 &      \\ 
     &           &    7 &  4.13e-08 &  1.04e-15 &      \\ 
     &           &    8 &  3.65e-08 &  1.18e-15 &      \\ 
     &           &    9 &  4.13e-08 &  1.04e-15 &      \\ 
     &           &   10 &  3.65e-08 &  1.18e-15 &      \\ 
1578 &  1.58e+04 &   10 &           &           & iters  \\ 
 \hdashline 
     &           &    1 &  5.68e-08 &  2.88e-09 &      \\ 
     &           &    2 &  2.09e-08 &  1.62e-15 &      \\ 
     &           &    3 &  2.09e-08 &  5.97e-16 &      \\ 
     &           &    4 &  2.09e-08 &  5.96e-16 &      \\ 
     &           &    5 &  5.69e-08 &  5.96e-16 &      \\ 
     &           &    6 &  2.09e-08 &  1.62e-15 &      \\ 
     &           &    7 &  2.09e-08 &  5.97e-16 &      \\ 
     &           &    8 &  5.68e-08 &  5.96e-16 &      \\ 
     &           &    9 &  2.09e-08 &  1.62e-15 &      \\ 
     &           &   10 &  2.09e-08 &  5.98e-16 &      \\ 
1579 &  1.58e+04 &   10 &           &           & iters  \\ 
 \hdashline 
     &           &    1 &  2.76e-08 &  4.68e-09 &      \\ 
     &           &    2 &  5.01e-08 &  7.88e-16 &      \\ 
     &           &    3 &  2.77e-08 &  1.43e-15 &      \\ 
     &           &    4 &  2.76e-08 &  7.89e-16 &      \\ 
     &           &    5 &  5.01e-08 &  7.88e-16 &      \\ 
     &           &    6 &  2.77e-08 &  1.43e-15 &      \\ 
     &           &    7 &  2.76e-08 &  7.89e-16 &      \\ 
     &           &    8 &  5.01e-08 &  7.88e-16 &      \\ 
     &           &    9 &  2.76e-08 &  1.43e-15 &      \\ 
     &           &   10 &  2.76e-08 &  7.89e-16 &      \\ 
1580 &  1.58e+04 &   10 &           &           & iters  \\ 
 \hdashline 
     &           &    1 &  4.55e-08 &  4.36e-09 &      \\ 
     &           &    2 &  3.22e-08 &  1.30e-15 &      \\ 
     &           &    3 &  4.55e-08 &  9.20e-16 &      \\ 
     &           &    4 &  3.23e-08 &  1.30e-15 &      \\ 
     &           &    5 &  4.55e-08 &  9.20e-16 &      \\ 
     &           &    6 &  3.23e-08 &  1.30e-15 &      \\ 
     &           &    7 &  3.22e-08 &  9.21e-16 &      \\ 
     &           &    8 &  4.55e-08 &  9.20e-16 &      \\ 
     &           &    9 &  3.22e-08 &  1.30e-15 &      \\ 
     &           &   10 &  4.55e-08 &  9.20e-16 &      \\ 
1581 &  1.58e+04 &   10 &           &           & iters  \\ 
 \hdashline 
     &           &    1 &  2.93e-08 &  2.17e-09 &      \\ 
     &           &    2 &  4.85e-08 &  8.35e-16 &      \\ 
     &           &    3 &  2.93e-08 &  1.38e-15 &      \\ 
     &           &    4 &  2.93e-08 &  8.36e-16 &      \\ 
     &           &    5 &  4.85e-08 &  8.35e-16 &      \\ 
     &           &    6 &  2.93e-08 &  1.38e-15 &      \\ 
     &           &    7 &  2.92e-08 &  8.36e-16 &      \\ 
     &           &    8 &  4.85e-08 &  8.35e-16 &      \\ 
     &           &    9 &  2.93e-08 &  1.38e-15 &      \\ 
     &           &   10 &  4.85e-08 &  8.35e-16 &      \\ 
1582 &  1.58e+04 &   10 &           &           & iters  \\ 
 \hdashline 
     &           &    1 &  5.10e-08 &  1.27e-09 &      \\ 
     &           &    2 &  2.67e-08 &  1.46e-15 &      \\ 
     &           &    3 &  5.10e-08 &  7.62e-16 &      \\ 
     &           &    4 &  2.67e-08 &  1.46e-15 &      \\ 
     &           &    5 &  2.67e-08 &  7.63e-16 &      \\ 
     &           &    6 &  5.10e-08 &  7.62e-16 &      \\ 
     &           &    7 &  2.67e-08 &  1.46e-15 &      \\ 
     &           &    8 &  2.67e-08 &  7.63e-16 &      \\ 
     &           &    9 &  5.10e-08 &  7.62e-16 &      \\ 
     &           &   10 &  2.67e-08 &  1.46e-15 &      \\ 
1583 &  1.58e+04 &   10 &           &           & iters  \\ 
 \hdashline 
     &           &    1 &  1.21e-08 &  2.78e-10 &      \\ 
     &           &    2 &  1.20e-08 &  3.44e-16 &      \\ 
     &           &    3 &  1.20e-08 &  3.44e-16 &      \\ 
     &           &    4 &  6.57e-08 &  3.43e-16 &      \\ 
     &           &    5 &  8.98e-08 &  1.87e-15 &      \\ 
     &           &    6 &  6.57e-08 &  2.56e-15 &      \\ 
     &           &    7 &  1.21e-08 &  1.88e-15 &      \\ 
     &           &    8 &  1.21e-08 &  3.44e-16 &      \\ 
     &           &    9 &  1.20e-08 &  3.44e-16 &      \\ 
     &           &   10 &  1.20e-08 &  3.43e-16 &      \\ 
1584 &  1.58e+04 &   10 &           &           & iters  \\ 
 \hdashline 
     &           &    1 &  7.77e-08 &  3.40e-10 &      \\ 
     &           &    2 &  1.26e-10 &  2.22e-15 &      \\ 
     &           &    3 &  1.26e-10 &  3.60e-18 &      \\ 
     &           &    4 &  1.26e-10 &  3.59e-18 &      \\ 
     &           &    5 &  1.25e-10 &  3.59e-18 &      \\ 
     &           &    6 &  1.25e-10 &  3.58e-18 &      \\ 
     &           &    7 &  1.25e-10 &  3.58e-18 &      \\ 
     &           &    8 &  1.25e-10 &  3.57e-18 &      \\ 
     &           &    9 &  1.25e-10 &  3.57e-18 &      \\ 
     &           &   10 &  1.25e-10 &  3.56e-18 &      \\ 
1585 &  1.58e+04 &   10 &           &           & iters  \\ 
 \hdashline 
     &           &    1 &  5.84e-08 &  5.37e-11 &      \\ 
     &           &    2 &  1.94e-08 &  1.67e-15 &      \\ 
     &           &    3 &  1.94e-08 &  5.54e-16 &      \\ 
     &           &    4 &  5.83e-08 &  5.54e-16 &      \\ 
     &           &    5 &  1.94e-08 &  1.66e-15 &      \\ 
     &           &    6 &  1.94e-08 &  5.55e-16 &      \\ 
     &           &    7 &  1.94e-08 &  5.54e-16 &      \\ 
     &           &    8 &  5.83e-08 &  5.54e-16 &      \\ 
     &           &    9 &  1.94e-08 &  1.66e-15 &      \\ 
     &           &   10 &  1.94e-08 &  5.55e-16 &      \\ 
1586 &  1.58e+04 &   10 &           &           & iters  \\ 
 \hdashline 
     &           &    1 &  9.28e-09 &  2.96e-11 &      \\ 
     &           &    2 &  6.84e-08 &  2.65e-16 &      \\ 
     &           &    3 &  9.37e-09 &  1.95e-15 &      \\ 
     &           &    4 &  9.35e-09 &  2.67e-16 &      \\ 
     &           &    5 &  9.34e-09 &  2.67e-16 &      \\ 
     &           &    6 &  9.33e-09 &  2.67e-16 &      \\ 
     &           &    7 &  9.31e-09 &  2.66e-16 &      \\ 
     &           &    8 &  9.30e-09 &  2.66e-16 &      \\ 
     &           &    9 &  9.29e-09 &  2.65e-16 &      \\ 
     &           &   10 &  6.84e-08 &  2.65e-16 &      \\ 
1587 &  1.59e+04 &   10 &           &           & iters  \\ 
 \hdashline 
     &           &    1 &  2.24e-09 &  7.82e-10 &      \\ 
     &           &    2 &  2.24e-09 &  6.40e-17 &      \\ 
     &           &    3 &  2.24e-09 &  6.39e-17 &      \\ 
     &           &    4 &  2.23e-09 &  6.38e-17 &      \\ 
     &           &    5 &  2.23e-09 &  6.37e-17 &      \\ 
     &           &    6 &  2.23e-09 &  6.36e-17 &      \\ 
     &           &    7 &  2.22e-09 &  6.35e-17 &      \\ 
     &           &    8 &  2.22e-09 &  6.34e-17 &      \\ 
     &           &    9 &  2.22e-09 &  6.33e-17 &      \\ 
     &           &   10 &  2.21e-09 &  6.33e-17 &      \\ 
1588 &  1.59e+04 &   10 &           &           & iters  \\ 
 \hdashline 
     &           &    1 &  3.22e-08 &  1.82e-09 &      \\ 
     &           &    2 &  3.22e-08 &  9.20e-16 &      \\ 
     &           &    3 &  4.55e-08 &  9.19e-16 &      \\ 
     &           &    4 &  3.22e-08 &  1.30e-15 &      \\ 
     &           &    5 &  4.55e-08 &  9.20e-16 &      \\ 
     &           &    6 &  3.22e-08 &  1.30e-15 &      \\ 
     &           &    7 &  3.22e-08 &  9.20e-16 &      \\ 
     &           &    8 &  4.55e-08 &  9.19e-16 &      \\ 
     &           &    9 &  3.22e-08 &  1.30e-15 &      \\ 
     &           &   10 &  4.55e-08 &  9.19e-16 &      \\ 
1589 &  1.59e+04 &   10 &           &           & iters  \\ 
 \hdashline 
     &           &    1 &  2.04e-08 &  1.72e-09 &      \\ 
     &           &    2 &  1.85e-08 &  5.81e-16 &      \\ 
     &           &    3 &  2.04e-08 &  5.29e-16 &      \\ 
     &           &    4 &  1.85e-08 &  5.81e-16 &      \\ 
     &           &    5 &  2.03e-08 &  5.29e-16 &      \\ 
     &           &    6 &  1.85e-08 &  5.81e-16 &      \\ 
     &           &    7 &  2.03e-08 &  5.29e-16 &      \\ 
     &           &    8 &  1.85e-08 &  5.81e-16 &      \\ 
     &           &    9 &  2.03e-08 &  5.29e-16 &      \\ 
     &           &   10 &  1.85e-08 &  5.81e-16 &      \\ 
1590 &  1.59e+04 &   10 &           &           & iters  \\ 
 \hdashline 
     &           &    1 &  3.13e-08 &  1.25e-09 &      \\ 
     &           &    2 &  4.64e-08 &  8.94e-16 &      \\ 
     &           &    3 &  3.13e-08 &  1.32e-15 &      \\ 
     &           &    4 &  3.13e-08 &  8.94e-16 &      \\ 
     &           &    5 &  4.64e-08 &  8.93e-16 &      \\ 
     &           &    6 &  3.13e-08 &  1.33e-15 &      \\ 
     &           &    7 &  4.64e-08 &  8.94e-16 &      \\ 
     &           &    8 &  3.13e-08 &  1.33e-15 &      \\ 
     &           &    9 &  3.13e-08 &  8.94e-16 &      \\ 
     &           &   10 &  4.64e-08 &  8.93e-16 &      \\ 
1591 &  1.59e+04 &   10 &           &           & iters  \\ 
 \hdashline 
     &           &    1 &  2.04e-08 &  1.72e-10 &      \\ 
     &           &    2 &  2.03e-08 &  5.82e-16 &      \\ 
     &           &    3 &  5.74e-08 &  5.81e-16 &      \\ 
     &           &    4 &  2.04e-08 &  1.64e-15 &      \\ 
     &           &    5 &  2.04e-08 &  5.82e-16 &      \\ 
     &           &    6 &  2.03e-08 &  5.81e-16 &      \\ 
     &           &    7 &  5.74e-08 &  5.81e-16 &      \\ 
     &           &    8 &  2.04e-08 &  1.64e-15 &      \\ 
     &           &    9 &  2.04e-08 &  5.82e-16 &      \\ 
     &           &   10 &  2.03e-08 &  5.81e-16 &      \\ 
1592 &  1.59e+04 &   10 &           &           & iters  \\ 
 \hdashline 
     &           &    1 &  1.40e-08 &  1.69e-09 &      \\ 
     &           &    2 &  1.40e-08 &  4.00e-16 &      \\ 
     &           &    3 &  1.40e-08 &  4.00e-16 &      \\ 
     &           &    4 &  1.40e-08 &  3.99e-16 &      \\ 
     &           &    5 &  6.37e-08 &  3.99e-16 &      \\ 
     &           &    6 &  1.40e-08 &  1.82e-15 &      \\ 
     &           &    7 &  1.40e-08 &  4.01e-16 &      \\ 
     &           &    8 &  1.40e-08 &  4.00e-16 &      \\ 
     &           &    9 &  1.40e-08 &  3.99e-16 &      \\ 
     &           &   10 &  1.40e-08 &  3.99e-16 &      \\ 
1593 &  1.59e+04 &   10 &           &           & iters  \\ 
 \hdashline 
     &           &    1 &  8.67e-09 &  8.17e-10 &      \\ 
     &           &    2 &  8.66e-09 &  2.47e-16 &      \\ 
     &           &    3 &  8.65e-09 &  2.47e-16 &      \\ 
     &           &    4 &  8.63e-09 &  2.47e-16 &      \\ 
     &           &    5 &  8.62e-09 &  2.46e-16 &      \\ 
     &           &    6 &  8.61e-09 &  2.46e-16 &      \\ 
     &           &    7 &  6.91e-08 &  2.46e-16 &      \\ 
     &           &    8 &  8.64e-08 &  1.97e-15 &      \\ 
     &           &    9 &  6.91e-08 &  2.47e-15 &      \\ 
     &           &   10 &  8.67e-09 &  1.97e-15 &      \\ 
1594 &  1.59e+04 &   10 &           &           & iters  \\ 
 \hdashline 
     &           &    1 &  3.65e-08 &  2.19e-10 &      \\ 
     &           &    2 &  4.12e-08 &  1.04e-15 &      \\ 
     &           &    3 &  3.65e-08 &  1.18e-15 &      \\ 
     &           &    4 &  4.12e-08 &  1.04e-15 &      \\ 
     &           &    5 &  3.65e-08 &  1.18e-15 &      \\ 
     &           &    6 &  4.12e-08 &  1.04e-15 &      \\ 
     &           &    7 &  3.65e-08 &  1.18e-15 &      \\ 
     &           &    8 &  4.12e-08 &  1.04e-15 &      \\ 
     &           &    9 &  3.66e-08 &  1.18e-15 &      \\ 
     &           &   10 &  4.12e-08 &  1.04e-15 &      \\ 
1595 &  1.59e+04 &   10 &           &           & iters  \\ 
 \hdashline 
     &           &    1 &  3.42e-08 &  8.11e-10 &      \\ 
     &           &    2 &  4.67e-09 &  9.77e-16 &      \\ 
     &           &    3 &  4.66e-09 &  1.33e-16 &      \\ 
     &           &    4 &  4.65e-09 &  1.33e-16 &      \\ 
     &           &    5 &  3.42e-08 &  1.33e-16 &      \\ 
     &           &    6 &  4.70e-09 &  9.76e-16 &      \\ 
     &           &    7 &  4.69e-09 &  1.34e-16 &      \\ 
     &           &    8 &  4.68e-09 &  1.34e-16 &      \\ 
     &           &    9 &  4.67e-09 &  1.34e-16 &      \\ 
     &           &   10 &  4.67e-09 &  1.33e-16 &      \\ 
1596 &  1.60e+04 &   10 &           &           & iters  \\ 
 \hdashline 
     &           &    1 &  3.40e-08 &  8.03e-10 &      \\ 
     &           &    2 &  4.37e-08 &  9.71e-16 &      \\ 
     &           &    3 &  3.40e-08 &  1.25e-15 &      \\ 
     &           &    4 &  4.37e-08 &  9.71e-16 &      \\ 
     &           &    5 &  3.40e-08 &  1.25e-15 &      \\ 
     &           &    6 &  4.37e-08 &  9.72e-16 &      \\ 
     &           &    7 &  3.41e-08 &  1.25e-15 &      \\ 
     &           &    8 &  3.40e-08 &  9.72e-16 &      \\ 
     &           &    9 &  4.37e-08 &  9.71e-16 &      \\ 
     &           &   10 &  3.40e-08 &  1.25e-15 &      \\ 
1597 &  1.60e+04 &   10 &           &           & iters  \\ 
 \hdashline 
     &           &    1 &  2.02e-08 &  2.17e-10 &      \\ 
     &           &    2 &  5.76e-08 &  5.76e-16 &      \\ 
     &           &    3 &  2.02e-08 &  1.64e-15 &      \\ 
     &           &    4 &  2.02e-08 &  5.77e-16 &      \\ 
     &           &    5 &  2.02e-08 &  5.76e-16 &      \\ 
     &           &    6 &  5.76e-08 &  5.75e-16 &      \\ 
     &           &    7 &  2.02e-08 &  1.64e-15 &      \\ 
     &           &    8 &  2.02e-08 &  5.77e-16 &      \\ 
     &           &    9 &  2.02e-08 &  5.76e-16 &      \\ 
     &           &   10 &  5.76e-08 &  5.75e-16 &      \\ 
1598 &  1.60e+04 &   10 &           &           & iters  \\ 
 \hdashline 
     &           &    1 &  2.50e-08 &  1.62e-09 &      \\ 
     &           &    2 &  2.50e-08 &  7.14e-16 &      \\ 
     &           &    3 &  5.27e-08 &  7.13e-16 &      \\ 
     &           &    4 &  2.50e-08 &  1.51e-15 &      \\ 
     &           &    5 &  2.50e-08 &  7.14e-16 &      \\ 
     &           &    6 &  5.27e-08 &  7.13e-16 &      \\ 
     &           &    7 &  2.50e-08 &  1.50e-15 &      \\ 
     &           &    8 &  2.50e-08 &  7.14e-16 &      \\ 
     &           &    9 &  5.27e-08 &  7.13e-16 &      \\ 
     &           &   10 &  2.50e-08 &  1.50e-15 &      \\ 
1599 &  1.60e+04 &   10 &           &           & iters  \\ 
 \hdashline 
     &           &    1 &  1.06e-08 &  1.65e-09 &      \\ 
     &           &    2 &  1.06e-08 &  3.02e-16 &      \\ 
     &           &    3 &  1.06e-08 &  3.02e-16 &      \\ 
     &           &    4 &  1.06e-08 &  3.02e-16 &      \\ 
     &           &    5 &  6.72e-08 &  3.01e-16 &      \\ 
     &           &    6 &  4.95e-08 &  1.92e-15 &      \\ 
     &           &    7 &  1.06e-08 &  1.41e-15 &      \\ 
     &           &    8 &  1.05e-08 &  3.01e-16 &      \\ 
     &           &    9 &  6.72e-08 &  3.01e-16 &      \\ 
     &           &   10 &  4.95e-08 &  1.92e-15 &      \\ 
1600 &  1.60e+04 &   10 &           &           & iters  \\ 
 \hdashline 
     &           &    1 &  1.54e-08 &  1.49e-09 &      \\ 
     &           &    2 &  1.54e-08 &  4.39e-16 &      \\ 
     &           &    3 &  1.53e-08 &  4.39e-16 &      \\ 
     &           &    4 &  2.35e-08 &  4.38e-16 &      \\ 
     &           &    5 &  1.54e-08 &  6.71e-16 &      \\ 
     &           &    6 &  2.35e-08 &  4.38e-16 &      \\ 
     &           &    7 &  1.54e-08 &  6.71e-16 &      \\ 
     &           &    8 &  1.53e-08 &  4.39e-16 &      \\ 
     &           &    9 &  2.35e-08 &  4.38e-16 &      \\ 
     &           &   10 &  1.54e-08 &  6.71e-16 &      \\ 
1601 &  1.60e+04 &   10 &           &           & iters  \\ 
 \hdashline 
     &           &    1 &  8.47e-09 &  6.82e-10 &      \\ 
     &           &    2 &  8.45e-09 &  2.42e-16 &      \\ 
     &           &    3 &  8.44e-09 &  2.41e-16 &      \\ 
     &           &    4 &  8.43e-09 &  2.41e-16 &      \\ 
     &           &    5 &  3.04e-08 &  2.41e-16 &      \\ 
     &           &    6 &  8.46e-09 &  8.68e-16 &      \\ 
     &           &    7 &  8.45e-09 &  2.41e-16 &      \\ 
     &           &    8 &  8.44e-09 &  2.41e-16 &      \\ 
     &           &    9 &  8.42e-09 &  2.41e-16 &      \\ 
     &           &   10 &  3.04e-08 &  2.40e-16 &      \\ 
1602 &  1.60e+04 &   10 &           &           & iters  \\ 
 \hdashline 
     &           &    1 &  1.81e-08 &  2.52e-09 &      \\ 
     &           &    2 &  1.80e-08 &  5.15e-16 &      \\ 
     &           &    3 &  5.97e-08 &  5.14e-16 &      \\ 
     &           &    4 &  1.81e-08 &  1.70e-15 &      \\ 
     &           &    5 &  1.81e-08 &  5.16e-16 &      \\ 
     &           &    6 &  1.80e-08 &  5.15e-16 &      \\ 
     &           &    7 &  1.80e-08 &  5.15e-16 &      \\ 
     &           &    8 &  5.97e-08 &  5.14e-16 &      \\ 
     &           &    9 &  1.81e-08 &  1.70e-15 &      \\ 
     &           &   10 &  1.80e-08 &  5.16e-16 &      \\ 
1603 &  1.60e+04 &   10 &           &           & iters  \\ 
 \hdashline 
     &           &    1 &  6.22e-09 &  3.36e-09 &      \\ 
     &           &    2 &  6.21e-09 &  1.77e-16 &      \\ 
     &           &    3 &  6.20e-09 &  1.77e-16 &      \\ 
     &           &    4 &  6.19e-09 &  1.77e-16 &      \\ 
     &           &    5 &  6.18e-09 &  1.77e-16 &      \\ 
     &           &    6 &  7.15e-08 &  1.76e-16 &      \\ 
     &           &    7 &  8.40e-08 &  2.04e-15 &      \\ 
     &           &    8 &  7.15e-08 &  2.40e-15 &      \\ 
     &           &    9 &  6.26e-09 &  2.04e-15 &      \\ 
     &           &   10 &  6.25e-09 &  1.79e-16 &      \\ 
1604 &  1.60e+04 &   10 &           &           & iters  \\ 
 \hdashline 
     &           &    1 &  1.22e-08 &  1.01e-09 &      \\ 
     &           &    2 &  1.22e-08 &  3.48e-16 &      \\ 
     &           &    3 &  1.22e-08 &  3.48e-16 &      \\ 
     &           &    4 &  1.21e-08 &  3.47e-16 &      \\ 
     &           &    5 &  1.21e-08 &  3.47e-16 &      \\ 
     &           &    6 &  6.56e-08 &  3.46e-16 &      \\ 
     &           &    7 &  1.22e-08 &  1.87e-15 &      \\ 
     &           &    8 &  1.22e-08 &  3.48e-16 &      \\ 
     &           &    9 &  1.22e-08 &  3.48e-16 &      \\ 
     &           &   10 &  1.22e-08 &  3.47e-16 &      \\ 
1605 &  1.60e+04 &   10 &           &           & iters  \\ 
 \hdashline 
     &           &    1 &  2.26e-09 &  3.20e-10 &      \\ 
     &           &    2 &  2.25e-09 &  6.44e-17 &      \\ 
     &           &    3 &  2.25e-09 &  6.43e-17 &      \\ 
     &           &    4 &  2.25e-09 &  6.42e-17 &      \\ 
     &           &    5 &  2.24e-09 &  6.41e-17 &      \\ 
     &           &    6 &  2.24e-09 &  6.40e-17 &      \\ 
     &           &    7 &  2.24e-09 &  6.40e-17 &      \\ 
     &           &    8 &  2.23e-09 &  6.39e-17 &      \\ 
     &           &    9 &  2.23e-09 &  6.38e-17 &      \\ 
     &           &   10 &  2.23e-09 &  6.37e-17 &      \\ 
1606 &  1.60e+04 &   10 &           &           & iters  \\ 
 \hdashline 
     &           &    1 &  1.40e-08 &  1.82e-10 &      \\ 
     &           &    2 &  6.37e-08 &  4.00e-16 &      \\ 
     &           &    3 &  1.41e-08 &  1.82e-15 &      \\ 
     &           &    4 &  1.41e-08 &  4.02e-16 &      \\ 
     &           &    5 &  1.41e-08 &  4.02e-16 &      \\ 
     &           &    6 &  1.40e-08 &  4.01e-16 &      \\ 
     &           &    7 &  6.37e-08 &  4.01e-16 &      \\ 
     &           &    8 &  9.18e-08 &  1.82e-15 &      \\ 
     &           &    9 &  6.37e-08 &  2.62e-15 &      \\ 
     &           &   10 &  1.41e-08 &  1.82e-15 &      \\ 
1607 &  1.61e+04 &   10 &           &           & iters  \\ 
 \hdashline 
     &           &    1 &  6.39e-08 &  5.96e-10 &      \\ 
     &           &    2 &  9.16e-08 &  1.82e-15 &      \\ 
     &           &    3 &  6.39e-08 &  2.61e-15 &      \\ 
     &           &    4 &  1.39e-08 &  1.82e-15 &      \\ 
     &           &    5 &  1.39e-08 &  3.96e-16 &      \\ 
     &           &    6 &  1.38e-08 &  3.96e-16 &      \\ 
     &           &    7 &  6.39e-08 &  3.95e-16 &      \\ 
     &           &    8 &  1.39e-08 &  1.82e-15 &      \\ 
     &           &    9 &  1.39e-08 &  3.97e-16 &      \\ 
     &           &   10 &  1.39e-08 &  3.96e-16 &      \\ 
1608 &  1.61e+04 &   10 &           &           & iters  \\ 
 \hdashline 
     &           &    1 &  4.38e-08 &  4.48e-10 &      \\ 
     &           &    2 &  3.39e-08 &  1.25e-15 &      \\ 
     &           &    3 &  4.38e-08 &  9.69e-16 &      \\ 
     &           &    4 &  3.40e-08 &  1.25e-15 &      \\ 
     &           &    5 &  4.38e-08 &  9.69e-16 &      \\ 
     &           &    6 &  3.40e-08 &  1.25e-15 &      \\ 
     &           &    7 &  4.38e-08 &  9.70e-16 &      \\ 
     &           &    8 &  3.40e-08 &  1.25e-15 &      \\ 
     &           &    9 &  3.39e-08 &  9.70e-16 &      \\ 
     &           &   10 &  4.38e-08 &  9.69e-16 &      \\ 
1609 &  1.61e+04 &   10 &           &           & iters  \\ 
 \hdashline 
     &           &    1 &  7.45e-08 &  1.05e-09 &      \\ 
     &           &    2 &  8.10e-08 &  2.13e-15 &      \\ 
     &           &    3 &  7.45e-08 &  2.31e-15 &      \\ 
     &           &    4 &  3.32e-09 &  2.13e-15 &      \\ 
     &           &    5 &  3.31e-09 &  9.46e-17 &      \\ 
     &           &    6 &  3.31e-09 &  9.45e-17 &      \\ 
     &           &    7 &  3.30e-09 &  9.44e-17 &      \\ 
     &           &    8 &  3.30e-09 &  9.42e-17 &      \\ 
     &           &    9 &  3.29e-09 &  9.41e-17 &      \\ 
     &           &   10 &  3.29e-09 &  9.40e-17 &      \\ 
1610 &  1.61e+04 &   10 &           &           & iters  \\ 
 \hdashline 
     &           &    1 &  1.99e-08 &  6.40e-10 &      \\ 
     &           &    2 &  5.78e-08 &  5.69e-16 &      \\ 
     &           &    3 &  2.00e-08 &  1.65e-15 &      \\ 
     &           &    4 &  2.00e-08 &  5.71e-16 &      \\ 
     &           &    5 &  1.99e-08 &  5.70e-16 &      \\ 
     &           &    6 &  5.78e-08 &  5.69e-16 &      \\ 
     &           &    7 &  2.00e-08 &  1.65e-15 &      \\ 
     &           &    8 &  2.00e-08 &  5.71e-16 &      \\ 
     &           &    9 &  1.99e-08 &  5.70e-16 &      \\ 
     &           &   10 &  5.78e-08 &  5.69e-16 &      \\ 
1611 &  1.61e+04 &   10 &           &           & iters  \\ 
 \hdashline 
     &           &    1 &  1.98e-08 &  1.25e-09 &      \\ 
     &           &    2 &  1.97e-08 &  5.64e-16 &      \\ 
     &           &    3 &  1.97e-08 &  5.63e-16 &      \\ 
     &           &    4 &  5.80e-08 &  5.62e-16 &      \\ 
     &           &    5 &  1.97e-08 &  1.66e-15 &      \\ 
     &           &    6 &  1.97e-08 &  5.64e-16 &      \\ 
     &           &    7 &  1.97e-08 &  5.63e-16 &      \\ 
     &           &    8 &  5.80e-08 &  5.62e-16 &      \\ 
     &           &    9 &  1.97e-08 &  1.66e-15 &      \\ 
     &           &   10 &  1.97e-08 &  5.64e-16 &      \\ 
1612 &  1.61e+04 &   10 &           &           & iters  \\ 
 \hdashline 
     &           &    1 &  9.55e-09 &  5.08e-10 &      \\ 
     &           &    2 &  9.53e-09 &  2.72e-16 &      \\ 
     &           &    3 &  9.52e-09 &  2.72e-16 &      \\ 
     &           &    4 &  9.51e-09 &  2.72e-16 &      \\ 
     &           &    5 &  9.49e-09 &  2.71e-16 &      \\ 
     &           &    6 &  9.48e-09 &  2.71e-16 &      \\ 
     &           &    7 &  9.47e-09 &  2.71e-16 &      \\ 
     &           &    8 &  9.45e-09 &  2.70e-16 &      \\ 
     &           &    9 &  6.83e-08 &  2.70e-16 &      \\ 
     &           &   10 &  9.54e-09 &  1.95e-15 &      \\ 
1613 &  1.61e+04 &   10 &           &           & iters  \\ 
 \hdashline 
     &           &    1 &  2.05e-08 &  7.40e-10 &      \\ 
     &           &    2 &  1.83e-08 &  5.86e-16 &      \\ 
     &           &    3 &  2.05e-08 &  5.23e-16 &      \\ 
     &           &    4 &  1.83e-08 &  5.86e-16 &      \\ 
     &           &    5 &  1.83e-08 &  5.23e-16 &      \\ 
     &           &    6 &  2.06e-08 &  5.22e-16 &      \\ 
     &           &    7 &  1.83e-08 &  5.87e-16 &      \\ 
     &           &    8 &  2.06e-08 &  5.23e-16 &      \\ 
     &           &    9 &  1.83e-08 &  5.87e-16 &      \\ 
     &           &   10 &  2.06e-08 &  5.23e-16 &      \\ 
1614 &  1.61e+04 &   10 &           &           & iters  \\ 
 \hdashline 
     &           &    1 &  6.26e-09 &  1.27e-09 &      \\ 
     &           &    2 &  6.25e-09 &  1.79e-16 &      \\ 
     &           &    3 &  6.24e-09 &  1.78e-16 &      \\ 
     &           &    4 &  6.23e-09 &  1.78e-16 &      \\ 
     &           &    5 &  6.22e-09 &  1.78e-16 &      \\ 
     &           &    6 &  6.21e-09 &  1.78e-16 &      \\ 
     &           &    7 &  7.15e-08 &  1.77e-16 &      \\ 
     &           &    8 &  8.40e-08 &  2.04e-15 &      \\ 
     &           &    9 &  7.15e-08 &  2.40e-15 &      \\ 
     &           &   10 &  6.29e-09 &  2.04e-15 &      \\ 
1615 &  1.61e+04 &   10 &           &           & iters  \\ 
 \hdashline 
     &           &    1 &  1.62e-08 &  7.62e-10 &      \\ 
     &           &    2 &  1.62e-08 &  4.62e-16 &      \\ 
     &           &    3 &  1.61e-08 &  4.61e-16 &      \\ 
     &           &    4 &  2.27e-08 &  4.60e-16 &      \\ 
     &           &    5 &  1.61e-08 &  6.49e-16 &      \\ 
     &           &    6 &  2.27e-08 &  4.61e-16 &      \\ 
     &           &    7 &  1.62e-08 &  6.49e-16 &      \\ 
     &           &    8 &  2.27e-08 &  4.61e-16 &      \\ 
     &           &    9 &  1.62e-08 &  6.48e-16 &      \\ 
     &           &   10 &  1.61e-08 &  4.61e-16 &      \\ 
1616 &  1.62e+04 &   10 &           &           & iters  \\ 
 \hdashline 
     &           &    1 &  2.11e-08 &  3.33e-10 &      \\ 
     &           &    2 &  5.66e-08 &  6.02e-16 &      \\ 
     &           &    3 &  2.12e-08 &  1.62e-15 &      \\ 
     &           &    4 &  2.11e-08 &  6.04e-16 &      \\ 
     &           &    5 &  2.11e-08 &  6.03e-16 &      \\ 
     &           &    6 &  5.66e-08 &  6.02e-16 &      \\ 
     &           &    7 &  2.11e-08 &  1.62e-15 &      \\ 
     &           &    8 &  2.11e-08 &  6.04e-16 &      \\ 
     &           &    9 &  5.66e-08 &  6.03e-16 &      \\ 
     &           &   10 &  2.12e-08 &  1.62e-15 &      \\ 
1617 &  1.62e+04 &   10 &           &           & iters  \\ 
 \hdashline 
     &           &    1 &  1.24e-08 &  1.06e-09 &      \\ 
     &           &    2 &  1.24e-08 &  3.54e-16 &      \\ 
     &           &    3 &  1.24e-08 &  3.53e-16 &      \\ 
     &           &    4 &  1.23e-08 &  3.53e-16 &      \\ 
     &           &    5 &  6.54e-08 &  3.52e-16 &      \\ 
     &           &    6 &  1.24e-08 &  1.87e-15 &      \\ 
     &           &    7 &  1.24e-08 &  3.55e-16 &      \\ 
     &           &    8 &  1.24e-08 &  3.54e-16 &      \\ 
     &           &    9 &  1.24e-08 &  3.54e-16 &      \\ 
     &           &   10 &  1.24e-08 &  3.53e-16 &      \\ 
1618 &  1.62e+04 &   10 &           &           & iters  \\ 
 \hdashline 
     &           &    1 &  1.22e-09 &  1.26e-09 &      \\ 
     &           &    2 &  1.22e-09 &  3.47e-17 &      \\ 
     &           &    3 &  1.21e-09 &  3.47e-17 &      \\ 
     &           &    4 &  7.65e-08 &  3.46e-17 &      \\ 
     &           &    5 &  7.90e-08 &  2.18e-15 &      \\ 
     &           &    6 &  7.65e-08 &  2.25e-15 &      \\ 
     &           &    7 &  7.90e-08 &  2.18e-15 &      \\ 
     &           &    8 &  7.65e-08 &  2.25e-15 &      \\ 
     &           &    9 &  7.90e-08 &  2.18e-15 &      \\ 
     &           &   10 &  7.65e-08 &  2.25e-15 &      \\ 
1619 &  1.62e+04 &   10 &           &           & iters  \\ 
 \hdashline 
     &           &    1 &  3.13e-08 &  1.44e-09 &      \\ 
     &           &    2 &  3.13e-08 &  8.94e-16 &      \\ 
     &           &    3 &  4.64e-08 &  8.93e-16 &      \\ 
     &           &    4 &  3.13e-08 &  1.33e-15 &      \\ 
     &           &    5 &  3.13e-08 &  8.94e-16 &      \\ 
     &           &    6 &  4.65e-08 &  8.92e-16 &      \\ 
     &           &    7 &  3.13e-08 &  1.33e-15 &      \\ 
     &           &    8 &  4.64e-08 &  8.93e-16 &      \\ 
     &           &    9 &  3.13e-08 &  1.33e-15 &      \\ 
     &           &   10 &  3.13e-08 &  8.93e-16 &      \\ 
1620 &  1.62e+04 &   10 &           &           & iters  \\ 
 \hdashline 
     &           &    1 &  6.79e-08 &  1.90e-09 &      \\ 
     &           &    2 &  9.90e-09 &  1.94e-15 &      \\ 
     &           &    3 &  9.89e-09 &  2.83e-16 &      \\ 
     &           &    4 &  9.88e-09 &  2.82e-16 &      \\ 
     &           &    5 &  9.86e-09 &  2.82e-16 &      \\ 
     &           &    6 &  9.85e-09 &  2.81e-16 &      \\ 
     &           &    7 &  9.83e-09 &  2.81e-16 &      \\ 
     &           &    8 &  9.82e-09 &  2.81e-16 &      \\ 
     &           &    9 &  6.79e-08 &  2.80e-16 &      \\ 
     &           &   10 &  9.90e-09 &  1.94e-15 &      \\ 
1621 &  1.62e+04 &   10 &           &           & iters  \\ 
 \hdashline 
     &           &    1 &  3.33e-08 &  1.84e-09 &      \\ 
     &           &    2 &  4.44e-08 &  9.50e-16 &      \\ 
     &           &    3 &  3.33e-08 &  1.27e-15 &      \\ 
     &           &    4 &  4.44e-08 &  9.51e-16 &      \\ 
     &           &    5 &  3.33e-08 &  1.27e-15 &      \\ 
     &           &    6 &  3.33e-08 &  9.51e-16 &      \\ 
     &           &    7 &  4.44e-08 &  9.50e-16 &      \\ 
     &           &    8 &  3.33e-08 &  1.27e-15 &      \\ 
     &           &    9 &  4.44e-08 &  9.50e-16 &      \\ 
     &           &   10 &  3.33e-08 &  1.27e-15 &      \\ 
1622 &  1.62e+04 &   10 &           &           & iters  \\ 
 \hdashline 
     &           &    1 &  1.23e-08 &  1.40e-09 &      \\ 
     &           &    2 &  1.23e-08 &  3.52e-16 &      \\ 
     &           &    3 &  1.23e-08 &  3.51e-16 &      \\ 
     &           &    4 &  1.23e-08 &  3.51e-16 &      \\ 
     &           &    5 &  1.23e-08 &  3.50e-16 &      \\ 
     &           &    6 &  1.22e-08 &  3.50e-16 &      \\ 
     &           &    7 &  6.55e-08 &  3.49e-16 &      \\ 
     &           &    8 &  1.23e-08 &  1.87e-15 &      \\ 
     &           &    9 &  1.23e-08 &  3.52e-16 &      \\ 
     &           &   10 &  1.23e-08 &  3.51e-16 &      \\ 
1623 &  1.62e+04 &   10 &           &           & iters  \\ 
 \hdashline 
     &           &    1 &  5.20e-08 &  9.98e-10 &      \\ 
     &           &    2 &  2.58e-08 &  1.48e-15 &      \\ 
     &           &    3 &  5.19e-08 &  7.36e-16 &      \\ 
     &           &    4 &  2.58e-08 &  1.48e-15 &      \\ 
     &           &    5 &  2.58e-08 &  7.37e-16 &      \\ 
     &           &    6 &  5.19e-08 &  7.36e-16 &      \\ 
     &           &    7 &  2.58e-08 &  1.48e-15 &      \\ 
     &           &    8 &  2.58e-08 &  7.37e-16 &      \\ 
     &           &    9 &  5.19e-08 &  7.36e-16 &      \\ 
     &           &   10 &  2.58e-08 &  1.48e-15 &      \\ 
1624 &  1.62e+04 &   10 &           &           & iters  \\ 
 \hdashline 
     &           &    1 &  1.05e-08 &  3.17e-10 &      \\ 
     &           &    2 &  1.05e-08 &  3.00e-16 &      \\ 
     &           &    3 &  1.05e-08 &  2.99e-16 &      \\ 
     &           &    4 &  1.05e-08 &  2.99e-16 &      \\ 
     &           &    5 &  1.04e-08 &  2.98e-16 &      \\ 
     &           &    6 &  6.73e-08 &  2.98e-16 &      \\ 
     &           &    7 &  8.82e-08 &  1.92e-15 &      \\ 
     &           &    8 &  6.73e-08 &  2.52e-15 &      \\ 
     &           &    9 &  1.05e-08 &  1.92e-15 &      \\ 
     &           &   10 &  1.05e-08 &  3.00e-16 &      \\ 
1625 &  1.62e+04 &   10 &           &           & iters  \\ 
 \hdashline 
     &           &    1 &  1.39e-08 &  1.99e-10 &      \\ 
     &           &    2 &  1.39e-08 &  3.96e-16 &      \\ 
     &           &    3 &  6.39e-08 &  3.95e-16 &      \\ 
     &           &    4 &  1.39e-08 &  1.82e-15 &      \\ 
     &           &    5 &  1.39e-08 &  3.97e-16 &      \\ 
     &           &    6 &  1.39e-08 &  3.97e-16 &      \\ 
     &           &    7 &  1.39e-08 &  3.96e-16 &      \\ 
     &           &    8 &  1.38e-08 &  3.96e-16 &      \\ 
     &           &    9 &  6.39e-08 &  3.95e-16 &      \\ 
     &           &   10 &  1.39e-08 &  1.82e-15 &      \\ 
1626 &  1.62e+04 &   10 &           &           & iters  \\ 
 \hdashline 
     &           &    1 &  4.75e-08 &  4.66e-10 &      \\ 
     &           &    2 &  3.03e-08 &  1.36e-15 &      \\ 
     &           &    3 &  3.02e-08 &  8.64e-16 &      \\ 
     &           &    4 &  4.75e-08 &  8.62e-16 &      \\ 
     &           &    5 &  3.02e-08 &  1.36e-15 &      \\ 
     &           &    6 &  3.02e-08 &  8.63e-16 &      \\ 
     &           &    7 &  4.75e-08 &  8.62e-16 &      \\ 
     &           &    8 &  3.02e-08 &  1.36e-15 &      \\ 
     &           &    9 &  4.75e-08 &  8.63e-16 &      \\ 
     &           &   10 &  3.02e-08 &  1.36e-15 &      \\ 
1627 &  1.63e+04 &   10 &           &           & iters  \\ 
 \hdashline 
     &           &    1 &  6.17e-09 &  1.44e-10 &      \\ 
     &           &    2 &  6.16e-09 &  1.76e-16 &      \\ 
     &           &    3 &  6.15e-09 &  1.76e-16 &      \\ 
     &           &    4 &  6.14e-09 &  1.75e-16 &      \\ 
     &           &    5 &  6.13e-09 &  1.75e-16 &      \\ 
     &           &    6 &  6.12e-09 &  1.75e-16 &      \\ 
     &           &    7 &  6.11e-09 &  1.75e-16 &      \\ 
     &           &    8 &  6.11e-09 &  1.74e-16 &      \\ 
     &           &    9 &  7.16e-08 &  1.74e-16 &      \\ 
     &           &   10 &  6.20e-09 &  2.04e-15 &      \\ 
1628 &  1.63e+04 &   10 &           &           & iters  \\ 
 \hdashline 
     &           &    1 &  2.50e-08 &  5.70e-10 &      \\ 
     &           &    2 &  2.50e-08 &  7.13e-16 &      \\ 
     &           &    3 &  5.28e-08 &  7.12e-16 &      \\ 
     &           &    4 &  2.50e-08 &  1.51e-15 &      \\ 
     &           &    5 &  2.50e-08 &  7.13e-16 &      \\ 
     &           &    6 &  5.28e-08 &  7.12e-16 &      \\ 
     &           &    7 &  2.50e-08 &  1.51e-15 &      \\ 
     &           &    8 &  2.50e-08 &  7.14e-16 &      \\ 
     &           &    9 &  5.28e-08 &  7.13e-16 &      \\ 
     &           &   10 &  2.50e-08 &  1.51e-15 &      \\ 
1629 &  1.63e+04 &   10 &           &           & iters  \\ 
 \hdashline 
     &           &    1 &  1.14e-08 &  9.97e-10 &      \\ 
     &           &    2 &  1.14e-08 &  3.25e-16 &      \\ 
     &           &    3 &  1.14e-08 &  3.25e-16 &      \\ 
     &           &    4 &  1.13e-08 &  3.24e-16 &      \\ 
     &           &    5 &  1.13e-08 &  3.24e-16 &      \\ 
     &           &    6 &  1.13e-08 &  3.23e-16 &      \\ 
     &           &    7 &  6.64e-08 &  3.23e-16 &      \\ 
     &           &    8 &  1.14e-08 &  1.90e-15 &      \\ 
     &           &    9 &  1.14e-08 &  3.25e-16 &      \\ 
     &           &   10 &  1.14e-08 &  3.24e-16 &      \\ 
1630 &  1.63e+04 &   10 &           &           & iters  \\ 
 \hdashline 
     &           &    1 &  9.39e-11 &  2.28e-09 &      \\ 
     &           &    2 &  9.38e-11 &  2.68e-18 &      \\ 
     &           &    3 &  9.36e-11 &  2.68e-18 &      \\ 
     &           &    4 &  9.35e-11 &  2.67e-18 &      \\ 
     &           &    5 &  9.34e-11 &  2.67e-18 &      \\ 
     &           &    6 &  9.32e-11 &  2.66e-18 &      \\ 
     &           &    7 &  9.31e-11 &  2.66e-18 &      \\ 
     &           &    8 &  9.30e-11 &  2.66e-18 &      \\ 
     &           &    9 &  9.28e-11 &  2.65e-18 &      \\ 
     &           &   10 &  9.27e-11 &  2.65e-18 &      \\ 
1631 &  1.63e+04 &   10 &           &           & iters  \\ 
 \hdashline 
     &           &    1 &  3.29e-08 &  4.09e-09 &      \\ 
     &           &    2 &  3.28e-08 &  9.39e-16 &      \\ 
     &           &    3 &  4.49e-08 &  9.37e-16 &      \\ 
     &           &    4 &  3.29e-08 &  1.28e-15 &      \\ 
     &           &    5 &  3.28e-08 &  9.38e-16 &      \\ 
     &           &    6 &  4.49e-08 &  9.36e-16 &      \\ 
     &           &    7 &  3.28e-08 &  1.28e-15 &      \\ 
     &           &    8 &  4.49e-08 &  9.37e-16 &      \\ 
     &           &    9 &  3.28e-08 &  1.28e-15 &      \\ 
     &           &   10 &  4.49e-08 &  9.37e-16 &      \\ 
1632 &  1.63e+04 &   10 &           &           & iters  \\ 
 \hdashline 
     &           &    1 &  1.24e-08 &  3.89e-09 &      \\ 
     &           &    2 &  6.53e-08 &  3.53e-16 &      \\ 
     &           &    3 &  9.01e-08 &  1.86e-15 &      \\ 
     &           &    4 &  6.54e-08 &  2.57e-15 &      \\ 
     &           &    5 &  1.24e-08 &  1.87e-15 &      \\ 
     &           &    6 &  1.24e-08 &  3.54e-16 &      \\ 
     &           &    7 &  1.24e-08 &  3.54e-16 &      \\ 
     &           &    8 &  1.24e-08 &  3.53e-16 &      \\ 
     &           &    9 &  6.53e-08 &  3.53e-16 &      \\ 
     &           &   10 &  1.24e-08 &  1.87e-15 &      \\ 
1633 &  1.63e+04 &   10 &           &           & iters  \\ 
 \hdashline 
     &           &    1 &  8.58e-10 &  3.27e-09 &      \\ 
     &           &    2 &  8.57e-10 &  2.45e-17 &      \\ 
     &           &    3 &  8.55e-10 &  2.45e-17 &      \\ 
     &           &    4 &  8.54e-10 &  2.44e-17 &      \\ 
     &           &    5 &  8.53e-10 &  2.44e-17 &      \\ 
     &           &    6 &  8.52e-10 &  2.43e-17 &      \\ 
     &           &    7 &  8.51e-10 &  2.43e-17 &      \\ 
     &           &    8 &  8.49e-10 &  2.43e-17 &      \\ 
     &           &    9 &  8.48e-10 &  2.42e-17 &      \\ 
     &           &   10 &  8.47e-10 &  2.42e-17 &      \\ 
1634 &  1.63e+04 &   10 &           &           & iters  \\ 
 \hdashline 
     &           &    1 &  1.54e-08 &  2.27e-09 &      \\ 
     &           &    2 &  6.23e-08 &  4.40e-16 &      \\ 
     &           &    3 &  1.55e-08 &  1.78e-15 &      \\ 
     &           &    4 &  1.55e-08 &  4.42e-16 &      \\ 
     &           &    5 &  1.54e-08 &  4.41e-16 &      \\ 
     &           &    6 &  1.54e-08 &  4.41e-16 &      \\ 
     &           &    7 &  6.23e-08 &  4.40e-16 &      \\ 
     &           &    8 &  1.55e-08 &  1.78e-15 &      \\ 
     &           &    9 &  1.55e-08 &  4.42e-16 &      \\ 
     &           &   10 &  1.54e-08 &  4.41e-16 &      \\ 
1635 &  1.63e+04 &   10 &           &           & iters  \\ 
 \hdashline 
     &           &    1 &  2.57e-08 &  1.11e-11 &      \\ 
     &           &    2 &  1.32e-08 &  7.34e-16 &      \\ 
     &           &    3 &  1.31e-08 &  3.76e-16 &      \\ 
     &           &    4 &  2.57e-08 &  3.75e-16 &      \\ 
     &           &    5 &  1.32e-08 &  7.34e-16 &      \\ 
     &           &    6 &  1.31e-08 &  3.76e-16 &      \\ 
     &           &    7 &  2.57e-08 &  3.75e-16 &      \\ 
     &           &    8 &  1.32e-08 &  7.34e-16 &      \\ 
     &           &    9 &  1.31e-08 &  3.76e-16 &      \\ 
     &           &   10 &  2.57e-08 &  3.75e-16 &      \\ 
1636 &  1.64e+04 &   10 &           &           & iters  \\ 
 \hdashline 
     &           &    1 &  1.49e-08 &  1.06e-09 &      \\ 
     &           &    2 &  6.28e-08 &  4.26e-16 &      \\ 
     &           &    3 &  1.50e-08 &  1.79e-15 &      \\ 
     &           &    4 &  1.50e-08 &  4.28e-16 &      \\ 
     &           &    5 &  1.49e-08 &  4.27e-16 &      \\ 
     &           &    6 &  1.49e-08 &  4.26e-16 &      \\ 
     &           &    7 &  6.28e-08 &  4.26e-16 &      \\ 
     &           &    8 &  1.50e-08 &  1.79e-15 &      \\ 
     &           &    9 &  1.50e-08 &  4.28e-16 &      \\ 
     &           &   10 &  1.49e-08 &  4.27e-16 &      \\ 
1637 &  1.64e+04 &   10 &           &           & iters  \\ 
 \hdashline 
     &           &    1 &  1.50e-08 &  7.84e-10 &      \\ 
     &           &    2 &  2.39e-08 &  4.28e-16 &      \\ 
     &           &    3 &  1.50e-08 &  6.81e-16 &      \\ 
     &           &    4 &  2.39e-08 &  4.28e-16 &      \\ 
     &           &    5 &  1.50e-08 &  6.81e-16 &      \\ 
     &           &    6 &  1.50e-08 &  4.29e-16 &      \\ 
     &           &    7 &  2.39e-08 &  4.28e-16 &      \\ 
     &           &    8 &  1.50e-08 &  6.81e-16 &      \\ 
     &           &    9 &  1.50e-08 &  4.28e-16 &      \\ 
     &           &   10 &  2.39e-08 &  4.28e-16 &      \\ 
1638 &  1.64e+04 &   10 &           &           & iters  \\ 
 \hdashline 
     &           &    1 &  1.06e-08 &  9.12e-10 &      \\ 
     &           &    2 &  1.06e-08 &  3.03e-16 &      \\ 
     &           &    3 &  1.06e-08 &  3.02e-16 &      \\ 
     &           &    4 &  6.71e-08 &  3.02e-16 &      \\ 
     &           &    5 &  1.07e-08 &  1.92e-15 &      \\ 
     &           &    6 &  1.06e-08 &  3.04e-16 &      \\ 
     &           &    7 &  1.06e-08 &  3.04e-16 &      \\ 
     &           &    8 &  1.06e-08 &  3.03e-16 &      \\ 
     &           &    9 &  1.06e-08 &  3.03e-16 &      \\ 
     &           &   10 &  1.06e-08 &  3.03e-16 &      \\ 
1639 &  1.64e+04 &   10 &           &           & iters  \\ 
 \hdashline 
     &           &    1 &  3.87e-08 &  2.11e-09 &      \\ 
     &           &    2 &  3.91e-08 &  1.10e-15 &      \\ 
     &           &    3 &  3.87e-08 &  1.12e-15 &      \\ 
     &           &    4 &  3.91e-08 &  1.10e-15 &      \\ 
     &           &    5 &  3.87e-08 &  1.12e-15 &      \\ 
     &           &    6 &  3.91e-08 &  1.10e-15 &      \\ 
     &           &    7 &  3.87e-08 &  1.12e-15 &      \\ 
     &           &    8 &  3.91e-08 &  1.10e-15 &      \\ 
     &           &    9 &  3.87e-08 &  1.12e-15 &      \\ 
     &           &   10 &  3.91e-08 &  1.10e-15 &      \\ 
1640 &  1.64e+04 &   10 &           &           & iters  \\ 
 \hdashline 
     &           &    1 &  5.32e-08 &  2.80e-09 &      \\ 
     &           &    2 &  2.45e-08 &  1.52e-15 &      \\ 
     &           &    3 &  2.45e-08 &  7.00e-16 &      \\ 
     &           &    4 &  5.32e-08 &  6.99e-16 &      \\ 
     &           &    5 &  2.45e-08 &  1.52e-15 &      \\ 
     &           &    6 &  2.45e-08 &  7.00e-16 &      \\ 
     &           &    7 &  5.32e-08 &  6.99e-16 &      \\ 
     &           &    8 &  2.45e-08 &  1.52e-15 &      \\ 
     &           &    9 &  2.45e-08 &  7.01e-16 &      \\ 
     &           &   10 &  5.32e-08 &  7.00e-16 &      \\ 
1641 &  1.64e+04 &   10 &           &           & iters  \\ 
 \hdashline 
     &           &    1 &  4.39e-08 &  2.76e-09 &      \\ 
     &           &    2 &  3.39e-08 &  1.25e-15 &      \\ 
     &           &    3 &  3.38e-08 &  9.67e-16 &      \\ 
     &           &    4 &  4.39e-08 &  9.65e-16 &      \\ 
     &           &    5 &  3.38e-08 &  1.25e-15 &      \\ 
     &           &    6 &  4.39e-08 &  9.66e-16 &      \\ 
     &           &    7 &  3.39e-08 &  1.25e-15 &      \\ 
     &           &    8 &  4.39e-08 &  9.66e-16 &      \\ 
     &           &    9 &  3.39e-08 &  1.25e-15 &      \\ 
     &           &   10 &  4.39e-08 &  9.67e-16 &      \\ 
1642 &  1.64e+04 &   10 &           &           & iters  \\ 
 \hdashline 
     &           &    1 &  4.14e-08 &  1.51e-09 &      \\ 
     &           &    2 &  3.64e-08 &  1.18e-15 &      \\ 
     &           &    3 &  4.13e-08 &  1.04e-15 &      \\ 
     &           &    4 &  3.64e-08 &  1.18e-15 &      \\ 
     &           &    5 &  4.13e-08 &  1.04e-15 &      \\ 
     &           &    6 &  3.64e-08 &  1.18e-15 &      \\ 
     &           &    7 &  4.13e-08 &  1.04e-15 &      \\ 
     &           &    8 &  3.64e-08 &  1.18e-15 &      \\ 
     &           &    9 &  4.13e-08 &  1.04e-15 &      \\ 
     &           &   10 &  3.64e-08 &  1.18e-15 &      \\ 
1643 &  1.64e+04 &   10 &           &           & iters  \\ 
 \hdashline 
     &           &    1 &  1.92e-08 &  2.15e-09 &      \\ 
     &           &    2 &  1.92e-08 &  5.49e-16 &      \\ 
     &           &    3 &  1.92e-08 &  5.48e-16 &      \\ 
     &           &    4 &  5.85e-08 &  5.48e-16 &      \\ 
     &           &    5 &  1.92e-08 &  1.67e-15 &      \\ 
     &           &    6 &  1.92e-08 &  5.49e-16 &      \\ 
     &           &    7 &  1.92e-08 &  5.49e-16 &      \\ 
     &           &    8 &  5.85e-08 &  5.48e-16 &      \\ 
     &           &    9 &  1.92e-08 &  1.67e-15 &      \\ 
     &           &   10 &  1.92e-08 &  5.49e-16 &      \\ 
1644 &  1.64e+04 &   10 &           &           & iters  \\ 
 \hdashline 
     &           &    1 &  2.74e-08 &  2.63e-09 &      \\ 
     &           &    2 &  2.73e-08 &  7.81e-16 &      \\ 
     &           &    3 &  5.04e-08 &  7.80e-16 &      \\ 
     &           &    4 &  2.74e-08 &  1.44e-15 &      \\ 
     &           &    5 &  2.73e-08 &  7.81e-16 &      \\ 
     &           &    6 &  5.04e-08 &  7.80e-16 &      \\ 
     &           &    7 &  2.74e-08 &  1.44e-15 &      \\ 
     &           &    8 &  2.73e-08 &  7.81e-16 &      \\ 
     &           &    9 &  5.04e-08 &  7.80e-16 &      \\ 
     &           &   10 &  2.73e-08 &  1.44e-15 &      \\ 
1645 &  1.64e+04 &   10 &           &           & iters  \\ 
 \hdashline 
     &           &    1 &  2.21e-08 &  1.75e-10 &      \\ 
     &           &    2 &  1.68e-08 &  6.30e-16 &      \\ 
     &           &    3 &  2.21e-08 &  4.80e-16 &      \\ 
     &           &    4 &  1.68e-08 &  6.30e-16 &      \\ 
     &           &    5 &  1.68e-08 &  4.80e-16 &      \\ 
     &           &    6 &  2.21e-08 &  4.79e-16 &      \\ 
     &           &    7 &  1.68e-08 &  6.30e-16 &      \\ 
     &           &    8 &  2.21e-08 &  4.79e-16 &      \\ 
     &           &    9 &  1.68e-08 &  6.30e-16 &      \\ 
     &           &   10 &  2.21e-08 &  4.80e-16 &      \\ 
1646 &  1.64e+04 &   10 &           &           & iters  \\ 
 \hdashline 
     &           &    1 &  1.51e-08 &  1.08e-09 &      \\ 
     &           &    2 &  1.51e-08 &  4.31e-16 &      \\ 
     &           &    3 &  6.26e-08 &  4.30e-16 &      \\ 
     &           &    4 &  1.51e-08 &  1.79e-15 &      \\ 
     &           &    5 &  1.51e-08 &  4.32e-16 &      \\ 
     &           &    6 &  1.51e-08 &  4.32e-16 &      \\ 
     &           &    7 &  1.51e-08 &  4.31e-16 &      \\ 
     &           &    8 &  6.26e-08 &  4.30e-16 &      \\ 
     &           &    9 &  1.51e-08 &  1.79e-15 &      \\ 
     &           &   10 &  1.51e-08 &  4.32e-16 &      \\ 
1647 &  1.65e+04 &   10 &           &           & iters  \\ 
 \hdashline 
     &           &    1 &  2.28e-08 &  1.55e-09 &      \\ 
     &           &    2 &  2.28e-08 &  6.51e-16 &      \\ 
     &           &    3 &  5.49e-08 &  6.51e-16 &      \\ 
     &           &    4 &  2.28e-08 &  1.57e-15 &      \\ 
     &           &    5 &  2.28e-08 &  6.52e-16 &      \\ 
     &           &    6 &  5.49e-08 &  6.51e-16 &      \\ 
     &           &    7 &  2.29e-08 &  1.57e-15 &      \\ 
     &           &    8 &  2.28e-08 &  6.52e-16 &      \\ 
     &           &    9 &  2.28e-08 &  6.51e-16 &      \\ 
     &           &   10 &  5.49e-08 &  6.50e-16 &      \\ 
1648 &  1.65e+04 &   10 &           &           & iters  \\ 
 \hdashline 
     &           &    1 &  2.69e-08 &  8.92e-10 &      \\ 
     &           &    2 &  1.20e-08 &  7.69e-16 &      \\ 
     &           &    3 &  2.69e-08 &  3.41e-16 &      \\ 
     &           &    4 &  1.20e-08 &  7.68e-16 &      \\ 
     &           &    5 &  1.20e-08 &  3.42e-16 &      \\ 
     &           &    6 &  1.19e-08 &  3.41e-16 &      \\ 
     &           &    7 &  2.69e-08 &  3.41e-16 &      \\ 
     &           &    8 &  1.20e-08 &  7.68e-16 &      \\ 
     &           &    9 &  1.19e-08 &  3.41e-16 &      \\ 
     &           &   10 &  2.69e-08 &  3.41e-16 &      \\ 
1649 &  1.65e+04 &   10 &           &           & iters  \\ 
 \hdashline 
     &           &    1 &  4.60e-08 &  2.78e-09 &      \\ 
     &           &    2 &  3.17e-08 &  1.31e-15 &      \\ 
     &           &    3 &  4.60e-08 &  9.05e-16 &      \\ 
     &           &    4 &  3.17e-08 &  1.31e-15 &      \\ 
     &           &    5 &  3.17e-08 &  9.06e-16 &      \\ 
     &           &    6 &  4.60e-08 &  9.05e-16 &      \\ 
     &           &    7 &  3.17e-08 &  1.31e-15 &      \\ 
     &           &    8 &  4.60e-08 &  9.05e-16 &      \\ 
     &           &    9 &  3.17e-08 &  1.31e-15 &      \\ 
     &           &   10 &  3.17e-08 &  9.06e-16 &      \\ 
1650 &  1.65e+04 &   10 &           &           & iters  \\ 
 \hdashline 
     &           &    1 &  3.38e-08 &  4.37e-09 &      \\ 
     &           &    2 &  3.38e-08 &  9.66e-16 &      \\ 
     &           &    3 &  4.39e-08 &  9.64e-16 &      \\ 
     &           &    4 &  3.38e-08 &  1.25e-15 &      \\ 
     &           &    5 &  4.39e-08 &  9.65e-16 &      \\ 
     &           &    6 &  3.38e-08 &  1.25e-15 &      \\ 
     &           &    7 &  3.38e-08 &  9.65e-16 &      \\ 
     &           &    8 &  4.40e-08 &  9.64e-16 &      \\ 
     &           &    9 &  3.38e-08 &  1.25e-15 &      \\ 
     &           &   10 &  4.40e-08 &  9.64e-16 &      \\ 
1651 &  1.65e+04 &   10 &           &           & iters  \\ 
 \hdashline 
     &           &    1 &  6.37e-09 &  3.95e-09 &      \\ 
     &           &    2 &  6.36e-09 &  1.82e-16 &      \\ 
     &           &    3 &  6.35e-09 &  1.82e-16 &      \\ 
     &           &    4 &  6.34e-09 &  1.81e-16 &      \\ 
     &           &    5 &  6.33e-09 &  1.81e-16 &      \\ 
     &           &    6 &  7.14e-08 &  1.81e-16 &      \\ 
     &           &    7 &  8.41e-08 &  2.04e-15 &      \\ 
     &           &    8 &  7.14e-08 &  2.40e-15 &      \\ 
     &           &    9 &  6.41e-09 &  2.04e-15 &      \\ 
     &           &   10 &  6.40e-09 &  1.83e-16 &      \\ 
1652 &  1.65e+04 &   10 &           &           & iters  \\ 
 \hdashline 
     &           &    1 &  5.12e-09 &  6.43e-10 &      \\ 
     &           &    2 &  5.11e-09 &  1.46e-16 &      \\ 
     &           &    3 &  7.26e-08 &  1.46e-16 &      \\ 
     &           &    4 &  8.29e-08 &  2.07e-15 &      \\ 
     &           &    5 &  7.26e-08 &  2.37e-15 &      \\ 
     &           &    6 &  5.19e-09 &  2.07e-15 &      \\ 
     &           &    7 &  5.18e-09 &  1.48e-16 &      \\ 
     &           &    8 &  5.18e-09 &  1.48e-16 &      \\ 
     &           &    9 &  5.17e-09 &  1.48e-16 &      \\ 
     &           &   10 &  5.16e-09 &  1.48e-16 &      \\ 
1653 &  1.65e+04 &   10 &           &           & iters  \\ 
 \hdashline 
     &           &    1 &  1.39e-08 &  1.55e-09 &      \\ 
     &           &    2 &  1.39e-08 &  3.97e-16 &      \\ 
     &           &    3 &  1.39e-08 &  3.97e-16 &      \\ 
     &           &    4 &  1.39e-08 &  3.96e-16 &      \\ 
     &           &    5 &  6.38e-08 &  3.96e-16 &      \\ 
     &           &    6 &  1.39e-08 &  1.82e-15 &      \\ 
     &           &    7 &  1.39e-08 &  3.98e-16 &      \\ 
     &           &    8 &  1.39e-08 &  3.97e-16 &      \\ 
     &           &    9 &  1.39e-08 &  3.97e-16 &      \\ 
     &           &   10 &  6.38e-08 &  3.96e-16 &      \\ 
1654 &  1.65e+04 &   10 &           &           & iters  \\ 
 \hdashline 
     &           &    1 &  1.69e-09 &  6.58e-10 &      \\ 
     &           &    2 &  1.68e-09 &  4.81e-17 &      \\ 
     &           &    3 &  1.68e-09 &  4.81e-17 &      \\ 
     &           &    4 &  1.68e-09 &  4.80e-17 &      \\ 
     &           &    5 &  1.68e-09 &  4.79e-17 &      \\ 
     &           &    6 &  1.67e-09 &  4.79e-17 &      \\ 
     &           &    7 &  1.67e-09 &  4.78e-17 &      \\ 
     &           &    8 &  1.67e-09 &  4.77e-17 &      \\ 
     &           &    9 &  1.67e-09 &  4.77e-17 &      \\ 
     &           &   10 &  1.66e-09 &  4.76e-17 &      \\ 
1655 &  1.65e+04 &   10 &           &           & iters  \\ 
 \hdashline 
     &           &    1 &  1.05e-08 &  1.03e-09 &      \\ 
     &           &    2 &  1.05e-08 &  3.01e-16 &      \\ 
     &           &    3 &  1.05e-08 &  3.00e-16 &      \\ 
     &           &    4 &  1.05e-08 &  3.00e-16 &      \\ 
     &           &    5 &  1.05e-08 &  2.99e-16 &      \\ 
     &           &    6 &  1.05e-08 &  2.99e-16 &      \\ 
     &           &    7 &  6.72e-08 &  2.99e-16 &      \\ 
     &           &    8 &  1.05e-08 &  1.92e-15 &      \\ 
     &           &    9 &  1.05e-08 &  3.01e-16 &      \\ 
     &           &   10 &  1.05e-08 &  3.00e-16 &      \\ 
1656 &  1.66e+04 &   10 &           &           & iters  \\ 
 \hdashline 
     &           &    1 &  3.55e-08 &  4.63e-10 &      \\ 
     &           &    2 &  4.22e-08 &  1.01e-15 &      \\ 
     &           &    3 &  3.55e-08 &  1.20e-15 &      \\ 
     &           &    4 &  4.22e-08 &  1.01e-15 &      \\ 
     &           &    5 &  3.56e-08 &  1.20e-15 &      \\ 
     &           &    6 &  4.22e-08 &  1.01e-15 &      \\ 
     &           &    7 &  3.56e-08 &  1.20e-15 &      \\ 
     &           &    8 &  3.55e-08 &  1.02e-15 &      \\ 
     &           &    9 &  4.22e-08 &  1.01e-15 &      \\ 
     &           &   10 &  3.55e-08 &  1.21e-15 &      \\ 
1657 &  1.66e+04 &   10 &           &           & iters  \\ 
 \hdashline 
     &           &    1 &  4.65e-08 &  1.36e-09 &      \\ 
     &           &    2 &  3.12e-08 &  1.33e-15 &      \\ 
     &           &    3 &  3.12e-08 &  8.91e-16 &      \\ 
     &           &    4 &  4.65e-08 &  8.90e-16 &      \\ 
     &           &    5 &  3.12e-08 &  1.33e-15 &      \\ 
     &           &    6 &  4.65e-08 &  8.91e-16 &      \\ 
     &           &    7 &  3.12e-08 &  1.33e-15 &      \\ 
     &           &    8 &  3.12e-08 &  8.91e-16 &      \\ 
     &           &    9 &  4.65e-08 &  8.90e-16 &      \\ 
     &           &   10 &  3.12e-08 &  1.33e-15 &      \\ 
1658 &  1.66e+04 &   10 &           &           & iters  \\ 
 \hdashline 
     &           &    1 &  2.63e-08 &  3.08e-09 &      \\ 
     &           &    2 &  2.63e-08 &  7.51e-16 &      \\ 
     &           &    3 &  5.14e-08 &  7.50e-16 &      \\ 
     &           &    4 &  2.63e-08 &  1.47e-15 &      \\ 
     &           &    5 &  2.63e-08 &  7.51e-16 &      \\ 
     &           &    6 &  5.14e-08 &  7.50e-16 &      \\ 
     &           &    7 &  2.63e-08 &  1.47e-15 &      \\ 
     &           &    8 &  2.63e-08 &  7.51e-16 &      \\ 
     &           &    9 &  5.14e-08 &  7.50e-16 &      \\ 
     &           &   10 &  2.63e-08 &  1.47e-15 &      \\ 
1659 &  1.66e+04 &   10 &           &           & iters  \\ 
 \hdashline 
     &           &    1 &  3.57e-09 &  2.65e-09 &      \\ 
     &           &    2 &  3.57e-09 &  1.02e-16 &      \\ 
     &           &    3 &  3.56e-09 &  1.02e-16 &      \\ 
     &           &    4 &  3.56e-09 &  1.02e-16 &      \\ 
     &           &    5 &  3.55e-09 &  1.02e-16 &      \\ 
     &           &    6 &  7.41e-08 &  1.01e-16 &      \\ 
     &           &    7 &  8.13e-08 &  2.12e-15 &      \\ 
     &           &    8 &  7.42e-08 &  2.32e-15 &      \\ 
     &           &    9 &  8.13e-08 &  2.12e-15 &      \\ 
     &           &   10 &  7.42e-08 &  2.32e-15 &      \\ 
1660 &  1.66e+04 &   10 &           &           & iters  \\ 
 \hdashline 
     &           &    1 &  3.10e-08 &  4.88e-10 &      \\ 
     &           &    2 &  3.10e-08 &  8.85e-16 &      \\ 
     &           &    3 &  4.68e-08 &  8.84e-16 &      \\ 
     &           &    4 &  3.10e-08 &  1.33e-15 &      \\ 
     &           &    5 &  3.09e-08 &  8.84e-16 &      \\ 
     &           &    6 &  4.68e-08 &  8.83e-16 &      \\ 
     &           &    7 &  3.10e-08 &  1.34e-15 &      \\ 
     &           &    8 &  4.68e-08 &  8.84e-16 &      \\ 
     &           &    9 &  3.10e-08 &  1.33e-15 &      \\ 
     &           &   10 &  3.09e-08 &  8.84e-16 &      \\ 
1661 &  1.66e+04 &   10 &           &           & iters  \\ 
 \hdashline 
     &           &    1 &  1.59e-08 &  2.83e-10 &      \\ 
     &           &    2 &  1.59e-08 &  4.55e-16 &      \\ 
     &           &    3 &  6.18e-08 &  4.54e-16 &      \\ 
     &           &    4 &  1.60e-08 &  1.76e-15 &      \\ 
     &           &    5 &  1.60e-08 &  4.56e-16 &      \\ 
     &           &    6 &  1.59e-08 &  4.55e-16 &      \\ 
     &           &    7 &  6.18e-08 &  4.55e-16 &      \\ 
     &           &    8 &  1.60e-08 &  1.76e-15 &      \\ 
     &           &    9 &  1.60e-08 &  4.57e-16 &      \\ 
     &           &   10 &  1.60e-08 &  4.56e-16 &      \\ 
1662 &  1.66e+04 &   10 &           &           & iters  \\ 
 \hdashline 
     &           &    1 &  4.51e-08 &  9.72e-10 &      \\ 
     &           &    2 &  4.50e-08 &  1.29e-15 &      \\ 
     &           &    3 &  3.27e-08 &  1.29e-15 &      \\ 
     &           &    4 &  3.27e-08 &  9.34e-16 &      \\ 
     &           &    5 &  4.51e-08 &  9.32e-16 &      \\ 
     &           &    6 &  3.27e-08 &  1.29e-15 &      \\ 
     &           &    7 &  4.51e-08 &  9.33e-16 &      \\ 
     &           &    8 &  3.27e-08 &  1.29e-15 &      \\ 
     &           &    9 &  4.50e-08 &  9.33e-16 &      \\ 
     &           &   10 &  3.27e-08 &  1.29e-15 &      \\ 
1663 &  1.66e+04 &   10 &           &           & iters  \\ 
 \hdashline 
     &           &    1 &  6.85e-09 &  1.70e-09 &      \\ 
     &           &    2 &  6.84e-09 &  1.96e-16 &      \\ 
     &           &    3 &  7.09e-08 &  1.95e-16 &      \\ 
     &           &    4 &  8.46e-08 &  2.02e-15 &      \\ 
     &           &    5 &  7.09e-08 &  2.42e-15 &      \\ 
     &           &    6 &  6.91e-09 &  2.02e-15 &      \\ 
     &           &    7 &  6.90e-09 &  1.97e-16 &      \\ 
     &           &    8 &  6.89e-09 &  1.97e-16 &      \\ 
     &           &    9 &  6.88e-09 &  1.97e-16 &      \\ 
     &           &   10 &  6.87e-09 &  1.96e-16 &      \\ 
1664 &  1.66e+04 &   10 &           &           & iters  \\ 
 \hdashline 
     &           &    1 &  4.34e-08 &  7.06e-10 &      \\ 
     &           &    2 &  3.44e-08 &  1.24e-15 &      \\ 
     &           &    3 &  4.33e-08 &  9.82e-16 &      \\ 
     &           &    4 &  3.44e-08 &  1.24e-15 &      \\ 
     &           &    5 &  4.33e-08 &  9.82e-16 &      \\ 
     &           &    6 &  3.44e-08 &  1.24e-15 &      \\ 
     &           &    7 &  4.33e-08 &  9.82e-16 &      \\ 
     &           &    8 &  3.44e-08 &  1.24e-15 &      \\ 
     &           &    9 &  4.33e-08 &  9.83e-16 &      \\ 
     &           &   10 &  3.44e-08 &  1.24e-15 &      \\ 
1665 &  1.66e+04 &   10 &           &           & iters  \\ 
 \hdashline 
     &           &    1 &  6.33e-08 &  5.82e-10 &      \\ 
     &           &    2 &  1.45e-08 &  1.81e-15 &      \\ 
     &           &    3 &  1.45e-08 &  4.14e-16 &      \\ 
     &           &    4 &  1.45e-08 &  4.13e-16 &      \\ 
     &           &    5 &  1.44e-08 &  4.13e-16 &      \\ 
     &           &    6 &  6.33e-08 &  4.12e-16 &      \\ 
     &           &    7 &  1.45e-08 &  1.81e-15 &      \\ 
     &           &    8 &  1.45e-08 &  4.14e-16 &      \\ 
     &           &    9 &  1.45e-08 &  4.14e-16 &      \\ 
     &           &   10 &  1.44e-08 &  4.13e-16 &      \\ 
1666 &  1.66e+04 &   10 &           &           & iters  \\ 
 \hdashline 
     &           &    1 &  3.39e-09 &  1.53e-10 &      \\ 
     &           &    2 &  3.39e-09 &  9.68e-17 &      \\ 
     &           &    3 &  3.38e-09 &  9.66e-17 &      \\ 
     &           &    4 &  3.38e-09 &  9.65e-17 &      \\ 
     &           &    5 &  3.37e-09 &  9.64e-17 &      \\ 
     &           &    6 &  3.37e-09 &  9.62e-17 &      \\ 
     &           &    7 &  3.36e-09 &  9.61e-17 &      \\ 
     &           &    8 &  3.36e-09 &  9.59e-17 &      \\ 
     &           &    9 &  3.35e-09 &  9.58e-17 &      \\ 
     &           &   10 &  7.43e-08 &  9.57e-17 &      \\ 
1667 &  1.67e+04 &   10 &           &           & iters  \\ 
 \hdashline 
     &           &    1 &  8.38e-09 &  7.26e-10 &      \\ 
     &           &    2 &  8.37e-09 &  2.39e-16 &      \\ 
     &           &    3 &  8.36e-09 &  2.39e-16 &      \\ 
     &           &    4 &  8.35e-09 &  2.39e-16 &      \\ 
     &           &    5 &  8.34e-09 &  2.38e-16 &      \\ 
     &           &    6 &  6.94e-08 &  2.38e-16 &      \\ 
     &           &    7 &  8.61e-08 &  1.98e-15 &      \\ 
     &           &    8 &  6.94e-08 &  2.46e-15 &      \\ 
     &           &    9 &  8.40e-09 &  1.98e-15 &      \\ 
     &           &   10 &  8.39e-09 &  2.40e-16 &      \\ 
1668 &  1.67e+04 &   10 &           &           & iters  \\ 
 \hdashline 
     &           &    1 &  5.11e-08 &  9.02e-10 &      \\ 
     &           &    2 &  2.67e-08 &  1.46e-15 &      \\ 
     &           &    3 &  5.10e-08 &  7.61e-16 &      \\ 
     &           &    4 &  2.67e-08 &  1.46e-15 &      \\ 
     &           &    5 &  2.67e-08 &  7.62e-16 &      \\ 
     &           &    6 &  5.11e-08 &  7.61e-16 &      \\ 
     &           &    7 &  2.67e-08 &  1.46e-15 &      \\ 
     &           &    8 &  2.67e-08 &  7.62e-16 &      \\ 
     &           &    9 &  5.11e-08 &  7.61e-16 &      \\ 
     &           &   10 &  2.67e-08 &  1.46e-15 &      \\ 
1669 &  1.67e+04 &   10 &           &           & iters  \\ 
 \hdashline 
     &           &    1 &  3.93e-08 &  1.77e-09 &      \\ 
     &           &    2 &  3.93e-08 &  1.12e-15 &      \\ 
     &           &    3 &  3.85e-08 &  1.12e-15 &      \\ 
     &           &    4 &  3.93e-08 &  1.10e-15 &      \\ 
     &           &    5 &  3.85e-08 &  1.12e-15 &      \\ 
     &           &    6 &  3.93e-08 &  1.10e-15 &      \\ 
     &           &    7 &  3.85e-08 &  1.12e-15 &      \\ 
     &           &    8 &  3.93e-08 &  1.10e-15 &      \\ 
     &           &    9 &  3.85e-08 &  1.12e-15 &      \\ 
     &           &   10 &  3.93e-08 &  1.10e-15 &      \\ 
1670 &  1.67e+04 &   10 &           &           & iters  \\ 
 \hdashline 
     &           &    1 &  8.72e-08 &  4.23e-10 &      \\ 
     &           &    2 &  6.83e-08 &  2.49e-15 &      \\ 
     &           &    3 &  8.72e-08 &  1.95e-15 &      \\ 
     &           &    4 &  6.83e-08 &  2.49e-15 &      \\ 
     &           &    5 &  9.47e-09 &  1.95e-15 &      \\ 
     &           &    6 &  9.46e-09 &  2.70e-16 &      \\ 
     &           &    7 &  9.45e-09 &  2.70e-16 &      \\ 
     &           &    8 &  9.43e-09 &  2.70e-16 &      \\ 
     &           &    9 &  9.42e-09 &  2.69e-16 &      \\ 
     &           &   10 &  6.83e-08 &  2.69e-16 &      \\ 
1671 &  1.67e+04 &   10 &           &           & iters  \\ 
 \hdashline 
     &           &    1 &  9.20e-09 &  7.18e-10 &      \\ 
     &           &    2 &  6.85e-08 &  2.62e-16 &      \\ 
     &           &    3 &  9.28e-09 &  1.96e-15 &      \\ 
     &           &    4 &  9.27e-09 &  2.65e-16 &      \\ 
     &           &    5 &  9.26e-09 &  2.65e-16 &      \\ 
     &           &    6 &  9.24e-09 &  2.64e-16 &      \\ 
     &           &    7 &  9.23e-09 &  2.64e-16 &      \\ 
     &           &    8 &  9.22e-09 &  2.63e-16 &      \\ 
     &           &    9 &  9.20e-09 &  2.63e-16 &      \\ 
     &           &   10 &  9.19e-09 &  2.63e-16 &      \\ 
1672 &  1.67e+04 &   10 &           &           & iters  \\ 
 \hdashline 
     &           &    1 &  1.17e-07 &  3.12e-10 &      \\ 
     &           &    2 &  3.90e-08 &  3.33e-15 &      \\ 
     &           &    3 &  3.90e-08 &  1.11e-15 &      \\ 
     &           &    4 &  3.88e-08 &  1.11e-15 &      \\ 
     &           &    5 &  3.90e-08 &  1.11e-15 &      \\ 
     &           &    6 &  3.88e-08 &  1.11e-15 &      \\ 
     &           &    7 &  3.90e-08 &  1.11e-15 &      \\ 
     &           &    8 &  3.88e-08 &  1.11e-15 &      \\ 
     &           &    9 &  3.90e-08 &  1.11e-15 &      \\ 
     &           &   10 &  3.88e-08 &  1.11e-15 &      \\ 
1673 &  1.67e+04 &   10 &           &           & iters  \\ 
 \hdashline 
     &           &    1 &  2.51e-08 &  3.91e-10 &      \\ 
     &           &    2 &  5.26e-08 &  7.17e-16 &      \\ 
     &           &    3 &  2.52e-08 &  1.50e-15 &      \\ 
     &           &    4 &  2.51e-08 &  7.18e-16 &      \\ 
     &           &    5 &  5.26e-08 &  7.17e-16 &      \\ 
     &           &    6 &  2.52e-08 &  1.50e-15 &      \\ 
     &           &    7 &  2.51e-08 &  7.18e-16 &      \\ 
     &           &    8 &  5.26e-08 &  7.17e-16 &      \\ 
     &           &    9 &  2.52e-08 &  1.50e-15 &      \\ 
     &           &   10 &  2.51e-08 &  7.18e-16 &      \\ 
1674 &  1.67e+04 &   10 &           &           & iters  \\ 
 \hdashline 
     &           &    1 &  3.83e-09 &  2.43e-10 &      \\ 
     &           &    2 &  3.82e-09 &  1.09e-16 &      \\ 
     &           &    3 &  3.82e-09 &  1.09e-16 &      \\ 
     &           &    4 &  3.81e-09 &  1.09e-16 &      \\ 
     &           &    5 &  3.81e-09 &  1.09e-16 &      \\ 
     &           &    6 &  3.80e-09 &  1.09e-16 &      \\ 
     &           &    7 &  3.80e-09 &  1.09e-16 &      \\ 
     &           &    8 &  3.79e-09 &  1.08e-16 &      \\ 
     &           &    9 &  3.79e-09 &  1.08e-16 &      \\ 
     &           &   10 &  3.78e-09 &  1.08e-16 &      \\ 
1675 &  1.67e+04 &   10 &           &           & iters  \\ 
 \hdashline 
     &           &    1 &  6.49e-09 &  1.32e-09 &      \\ 
     &           &    2 &  7.12e-08 &  1.85e-16 &      \\ 
     &           &    3 &  8.43e-08 &  2.03e-15 &      \\ 
     &           &    4 &  7.12e-08 &  2.41e-15 &      \\ 
     &           &    5 &  6.57e-09 &  2.03e-15 &      \\ 
     &           &    6 &  6.56e-09 &  1.87e-16 &      \\ 
     &           &    7 &  6.55e-09 &  1.87e-16 &      \\ 
     &           &    8 &  6.54e-09 &  1.87e-16 &      \\ 
     &           &    9 &  6.53e-09 &  1.87e-16 &      \\ 
     &           &   10 &  6.52e-09 &  1.86e-16 &      \\ 
1676 &  1.68e+04 &   10 &           &           & iters  \\ 
 \hdashline 
     &           &    1 &  2.94e-09 &  1.21e-09 &      \\ 
     &           &    2 &  2.93e-09 &  8.39e-17 &      \\ 
     &           &    3 &  2.93e-09 &  8.37e-17 &      \\ 
     &           &    4 &  2.93e-09 &  8.36e-17 &      \\ 
     &           &    5 &  2.92e-09 &  8.35e-17 &      \\ 
     &           &    6 &  2.92e-09 &  8.34e-17 &      \\ 
     &           &    7 &  2.91e-09 &  8.33e-17 &      \\ 
     &           &    8 &  2.91e-09 &  8.31e-17 &      \\ 
     &           &    9 &  2.90e-09 &  8.30e-17 &      \\ 
     &           &   10 &  2.90e-09 &  8.29e-17 &      \\ 
1677 &  1.68e+04 &   10 &           &           & iters  \\ 
 \hdashline 
     &           &    1 &  6.68e-09 &  5.54e-11 &      \\ 
     &           &    2 &  6.67e-09 &  1.91e-16 &      \\ 
     &           &    3 &  6.66e-09 &  1.90e-16 &      \\ 
     &           &    4 &  6.65e-09 &  1.90e-16 &      \\ 
     &           &    5 &  6.64e-09 &  1.90e-16 &      \\ 
     &           &    6 &  6.63e-09 &  1.90e-16 &      \\ 
     &           &    7 &  7.11e-08 &  1.89e-16 &      \\ 
     &           &    8 &  8.44e-08 &  2.03e-15 &      \\ 
     &           &    9 &  7.11e-08 &  2.41e-15 &      \\ 
     &           &   10 &  6.70e-09 &  2.03e-15 &      \\ 
1678 &  1.68e+04 &   10 &           &           & iters  \\ 
 \hdashline 
     &           &    1 &  2.72e-08 &  1.51e-09 &      \\ 
     &           &    2 &  5.05e-08 &  7.77e-16 &      \\ 
     &           &    3 &  2.73e-08 &  1.44e-15 &      \\ 
     &           &    4 &  2.72e-08 &  7.78e-16 &      \\ 
     &           &    5 &  5.05e-08 &  7.77e-16 &      \\ 
     &           &    6 &  2.72e-08 &  1.44e-15 &      \\ 
     &           &    7 &  2.72e-08 &  7.78e-16 &      \\ 
     &           &    8 &  5.05e-08 &  7.76e-16 &      \\ 
     &           &    9 &  2.72e-08 &  1.44e-15 &      \\ 
     &           &   10 &  2.72e-08 &  7.77e-16 &      \\ 
1679 &  1.68e+04 &   10 &           &           & iters  \\ 
 \hdashline 
     &           &    1 &  2.26e-08 &  1.12e-09 &      \\ 
     &           &    2 &  2.26e-08 &  6.46e-16 &      \\ 
     &           &    3 &  5.51e-08 &  6.45e-16 &      \\ 
     &           &    4 &  2.27e-08 &  1.57e-15 &      \\ 
     &           &    5 &  2.26e-08 &  6.47e-16 &      \\ 
     &           &    6 &  5.51e-08 &  6.46e-16 &      \\ 
     &           &    7 &  2.27e-08 &  1.57e-15 &      \\ 
     &           &    8 &  2.26e-08 &  6.47e-16 &      \\ 
     &           &    9 &  5.51e-08 &  6.46e-16 &      \\ 
     &           &   10 &  2.27e-08 &  1.57e-15 &      \\ 
1680 &  1.68e+04 &   10 &           &           & iters  \\ 
 \hdashline 
     &           &    1 &  1.34e-08 &  1.39e-10 &      \\ 
     &           &    2 &  1.34e-08 &  3.83e-16 &      \\ 
     &           &    3 &  1.34e-08 &  3.83e-16 &      \\ 
     &           &    4 &  1.34e-08 &  3.82e-16 &      \\ 
     &           &    5 &  6.43e-08 &  3.82e-16 &      \\ 
     &           &    6 &  1.34e-08 &  1.84e-15 &      \\ 
     &           &    7 &  1.34e-08 &  3.84e-16 &      \\ 
     &           &    8 &  1.34e-08 &  3.83e-16 &      \\ 
     &           &    9 &  1.34e-08 &  3.83e-16 &      \\ 
     &           &   10 &  6.43e-08 &  3.82e-16 &      \\ 
1681 &  1.68e+04 &   10 &           &           & iters  \\ 
 \hdashline 
     &           &    1 &  1.03e-07 &  5.26e-11 &      \\ 
     &           &    2 &  5.22e-08 &  2.95e-15 &      \\ 
     &           &    3 &  2.55e-08 &  1.49e-15 &      \\ 
     &           &    4 &  2.55e-08 &  7.29e-16 &      \\ 
     &           &    5 &  5.22e-08 &  7.28e-16 &      \\ 
     &           &    6 &  2.55e-08 &  1.49e-15 &      \\ 
     &           &    7 &  2.55e-08 &  7.29e-16 &      \\ 
     &           &    8 &  5.22e-08 &  7.28e-16 &      \\ 
     &           &    9 &  2.55e-08 &  1.49e-15 &      \\ 
     &           &   10 &  2.55e-08 &  7.29e-16 &      \\ 
1682 &  1.68e+04 &   10 &           &           & iters  \\ 
 \hdashline 
     &           &    1 &  1.96e-08 &  6.29e-10 &      \\ 
     &           &    2 &  1.96e-08 &  5.59e-16 &      \\ 
     &           &    3 &  1.95e-08 &  5.58e-16 &      \\ 
     &           &    4 &  5.82e-08 &  5.57e-16 &      \\ 
     &           &    5 &  1.96e-08 &  1.66e-15 &      \\ 
     &           &    6 &  1.96e-08 &  5.59e-16 &      \\ 
     &           &    7 &  1.95e-08 &  5.58e-16 &      \\ 
     &           &    8 &  5.82e-08 &  5.57e-16 &      \\ 
     &           &    9 &  1.96e-08 &  1.66e-15 &      \\ 
     &           &   10 &  1.96e-08 &  5.59e-16 &      \\ 
1683 &  1.68e+04 &   10 &           &           & iters  \\ 
 \hdashline 
     &           &    1 &  8.02e-09 &  7.87e-10 &      \\ 
     &           &    2 &  8.01e-09 &  2.29e-16 &      \\ 
     &           &    3 &  8.00e-09 &  2.29e-16 &      \\ 
     &           &    4 &  7.99e-09 &  2.28e-16 &      \\ 
     &           &    5 &  7.98e-09 &  2.28e-16 &      \\ 
     &           &    6 &  6.97e-08 &  2.28e-16 &      \\ 
     &           &    7 &  8.58e-08 &  1.99e-15 &      \\ 
     &           &    8 &  6.97e-08 &  2.45e-15 &      \\ 
     &           &    9 &  8.04e-09 &  1.99e-15 &      \\ 
     &           &   10 &  8.03e-09 &  2.30e-16 &      \\ 
1684 &  1.68e+04 &   10 &           &           & iters  \\ 
 \hdashline 
     &           &    1 &  2.10e-08 &  8.08e-10 &      \\ 
     &           &    2 &  2.10e-08 &  6.00e-16 &      \\ 
     &           &    3 &  5.67e-08 &  5.99e-16 &      \\ 
     &           &    4 &  2.10e-08 &  1.62e-15 &      \\ 
     &           &    5 &  2.10e-08 &  6.01e-16 &      \\ 
     &           &    6 &  2.10e-08 &  6.00e-16 &      \\ 
     &           &    7 &  5.67e-08 &  5.99e-16 &      \\ 
     &           &    8 &  2.10e-08 &  1.62e-15 &      \\ 
     &           &    9 &  2.10e-08 &  6.00e-16 &      \\ 
     &           &   10 &  5.67e-08 &  5.99e-16 &      \\ 
1685 &  1.68e+04 &   10 &           &           & iters  \\ 
 \hdashline 
     &           &    1 &  2.62e-08 &  1.25e-09 &      \\ 
     &           &    2 &  1.26e-08 &  7.49e-16 &      \\ 
     &           &    3 &  2.62e-08 &  3.61e-16 &      \\ 
     &           &    4 &  1.27e-08 &  7.48e-16 &      \\ 
     &           &    5 &  1.26e-08 &  3.61e-16 &      \\ 
     &           &    6 &  2.62e-08 &  3.61e-16 &      \\ 
     &           &    7 &  1.27e-08 &  7.48e-16 &      \\ 
     &           &    8 &  1.26e-08 &  3.61e-16 &      \\ 
     &           &    9 &  2.62e-08 &  3.61e-16 &      \\ 
     &           &   10 &  1.27e-08 &  7.48e-16 &      \\ 
1686 &  1.68e+04 &   10 &           &           & iters  \\ 
 \hdashline 
     &           &    1 &  5.67e-09 &  1.67e-09 &      \\ 
     &           &    2 &  5.66e-09 &  1.62e-16 &      \\ 
     &           &    3 &  5.65e-09 &  1.61e-16 &      \\ 
     &           &    4 &  5.64e-09 &  1.61e-16 &      \\ 
     &           &    5 &  3.32e-08 &  1.61e-16 &      \\ 
     &           &    6 &  5.68e-09 &  9.48e-16 &      \\ 
     &           &    7 &  5.67e-09 &  1.62e-16 &      \\ 
     &           &    8 &  5.66e-09 &  1.62e-16 &      \\ 
     &           &    9 &  5.66e-09 &  1.62e-16 &      \\ 
     &           &   10 &  5.65e-09 &  1.61e-16 &      \\ 
1687 &  1.69e+04 &   10 &           &           & iters  \\ 
 \hdashline 
     &           &    1 &  1.00e-08 &  9.16e-10 &      \\ 
     &           &    2 &  1.00e-08 &  2.86e-16 &      \\ 
     &           &    3 &  9.99e-09 &  2.86e-16 &      \\ 
     &           &    4 &  2.89e-08 &  2.85e-16 &      \\ 
     &           &    5 &  1.00e-08 &  8.24e-16 &      \\ 
     &           &    6 &  1.00e-08 &  2.86e-16 &      \\ 
     &           &    7 &  2.89e-08 &  2.86e-16 &      \\ 
     &           &    8 &  1.00e-08 &  8.23e-16 &      \\ 
     &           &    9 &  1.00e-08 &  2.86e-16 &      \\ 
     &           &   10 &  1.00e-08 &  2.86e-16 &      \\ 
1688 &  1.69e+04 &   10 &           &           & iters  \\ 
 \hdashline 
     &           &    1 &  3.46e-08 &  5.70e-10 &      \\ 
     &           &    2 &  4.32e-08 &  9.86e-16 &      \\ 
     &           &    3 &  3.46e-08 &  1.23e-15 &      \\ 
     &           &    4 &  4.32e-08 &  9.87e-16 &      \\ 
     &           &    5 &  3.46e-08 &  1.23e-15 &      \\ 
     &           &    6 &  4.32e-08 &  9.87e-16 &      \\ 
     &           &    7 &  3.46e-08 &  1.23e-15 &      \\ 
     &           &    8 &  4.31e-08 &  9.87e-16 &      \\ 
     &           &    9 &  3.46e-08 &  1.23e-15 &      \\ 
     &           &   10 &  3.46e-08 &  9.88e-16 &      \\ 
1689 &  1.69e+04 &   10 &           &           & iters  \\ 
 \hdashline 
     &           &    1 &  2.77e-08 &  8.11e-10 &      \\ 
     &           &    2 &  2.77e-08 &  7.91e-16 &      \\ 
     &           &    3 &  5.01e-08 &  7.90e-16 &      \\ 
     &           &    4 &  2.77e-08 &  1.43e-15 &      \\ 
     &           &    5 &  2.77e-08 &  7.91e-16 &      \\ 
     &           &    6 &  5.01e-08 &  7.90e-16 &      \\ 
     &           &    7 &  2.77e-08 &  1.43e-15 &      \\ 
     &           &    8 &  2.77e-08 &  7.91e-16 &      \\ 
     &           &    9 &  5.01e-08 &  7.90e-16 &      \\ 
     &           &   10 &  2.77e-08 &  1.43e-15 &      \\ 
1690 &  1.69e+04 &   10 &           &           & iters  \\ 
 \hdashline 
     &           &    1 &  1.48e-08 &  4.27e-10 &      \\ 
     &           &    2 &  1.48e-08 &  4.22e-16 &      \\ 
     &           &    3 &  6.29e-08 &  4.22e-16 &      \\ 
     &           &    4 &  1.48e-08 &  1.80e-15 &      \\ 
     &           &    5 &  1.48e-08 &  4.24e-16 &      \\ 
     &           &    6 &  1.48e-08 &  4.23e-16 &      \\ 
     &           &    7 &  1.48e-08 &  4.22e-16 &      \\ 
     &           &    8 &  6.29e-08 &  4.22e-16 &      \\ 
     &           &    9 &  1.49e-08 &  1.80e-15 &      \\ 
     &           &   10 &  1.48e-08 &  4.24e-16 &      \\ 
1691 &  1.69e+04 &   10 &           &           & iters  \\ 
 \hdashline 
     &           &    1 &  7.22e-09 &  1.27e-09 &      \\ 
     &           &    2 &  7.21e-09 &  2.06e-16 &      \\ 
     &           &    3 &  7.20e-09 &  2.06e-16 &      \\ 
     &           &    4 &  7.19e-09 &  2.06e-16 &      \\ 
     &           &    5 &  3.17e-08 &  2.05e-16 &      \\ 
     &           &    6 &  7.23e-09 &  9.04e-16 &      \\ 
     &           &    7 &  7.22e-09 &  2.06e-16 &      \\ 
     &           &    8 &  7.20e-09 &  2.06e-16 &      \\ 
     &           &    9 &  7.19e-09 &  2.06e-16 &      \\ 
     &           &   10 &  3.17e-08 &  2.05e-16 &      \\ 
1692 &  1.69e+04 &   10 &           &           & iters  \\ 
 \hdashline 
     &           &    1 &  6.88e-10 &  2.42e-11 &      \\ 
     &           &    2 &  6.87e-10 &  1.96e-17 &      \\ 
     &           &    3 &  6.86e-10 &  1.96e-17 &      \\ 
     &           &    4 &  6.85e-10 &  1.96e-17 &      \\ 
     &           &    5 &  6.84e-10 &  1.95e-17 &      \\ 
     &           &    6 &  6.83e-10 &  1.95e-17 &      \\ 
     &           &    7 &  6.82e-10 &  1.95e-17 &      \\ 
     &           &    8 &  6.81e-10 &  1.95e-17 &      \\ 
     &           &    9 &  6.80e-10 &  1.94e-17 &      \\ 
     &           &   10 &  6.79e-10 &  1.94e-17 &      \\ 
1693 &  1.69e+04 &   10 &           &           & iters  \\ 
 \hdashline 
     &           &    1 &  3.08e-09 &  5.86e-10 &      \\ 
     &           &    2 &  3.07e-09 &  8.78e-17 &      \\ 
     &           &    3 &  3.07e-09 &  8.77e-17 &      \\ 
     &           &    4 &  3.06e-09 &  8.76e-17 &      \\ 
     &           &    5 &  3.06e-09 &  8.74e-17 &      \\ 
     &           &    6 &  3.06e-09 &  8.73e-17 &      \\ 
     &           &    7 &  3.05e-09 &  8.72e-17 &      \\ 
     &           &    8 &  3.05e-09 &  8.71e-17 &      \\ 
     &           &    9 &  3.04e-09 &  8.69e-17 &      \\ 
     &           &   10 &  3.04e-09 &  8.68e-17 &      \\ 
1694 &  1.69e+04 &   10 &           &           & iters  \\ 
 \hdashline 
     &           &    1 &  4.57e-08 &  6.27e-11 &      \\ 
     &           &    2 &  3.20e-08 &  1.31e-15 &      \\ 
     &           &    3 &  3.20e-08 &  9.14e-16 &      \\ 
     &           &    4 &  4.58e-08 &  9.12e-16 &      \\ 
     &           &    5 &  3.20e-08 &  1.31e-15 &      \\ 
     &           &    6 &  4.58e-08 &  9.13e-16 &      \\ 
     &           &    7 &  3.20e-08 &  1.31e-15 &      \\ 
     &           &    8 &  3.20e-08 &  9.13e-16 &      \\ 
     &           &    9 &  4.58e-08 &  9.12e-16 &      \\ 
     &           &   10 &  3.20e-08 &  1.31e-15 &      \\ 
1695 &  1.69e+04 &   10 &           &           & iters  \\ 
 \hdashline 
     &           &    1 &  1.42e-08 &  2.78e-10 &      \\ 
     &           &    2 &  6.35e-08 &  4.05e-16 &      \\ 
     &           &    3 &  1.43e-08 &  1.81e-15 &      \\ 
     &           &    4 &  1.43e-08 &  4.07e-16 &      \\ 
     &           &    5 &  1.42e-08 &  4.07e-16 &      \\ 
     &           &    6 &  1.42e-08 &  4.06e-16 &      \\ 
     &           &    7 &  6.35e-08 &  4.06e-16 &      \\ 
     &           &    8 &  1.43e-08 &  1.81e-15 &      \\ 
     &           &    9 &  1.43e-08 &  4.08e-16 &      \\ 
     &           &   10 &  1.42e-08 &  4.07e-16 &      \\ 
1696 &  1.70e+04 &   10 &           &           & iters  \\ 
 \hdashline 
     &           &    1 &  5.08e-08 &  3.39e-10 &      \\ 
     &           &    2 &  2.69e-08 &  1.45e-15 &      \\ 
     &           &    3 &  2.69e-08 &  7.69e-16 &      \\ 
     &           &    4 &  5.08e-08 &  7.67e-16 &      \\ 
     &           &    5 &  2.69e-08 &  1.45e-15 &      \\ 
     &           &    6 &  2.69e-08 &  7.68e-16 &      \\ 
     &           &    7 &  5.08e-08 &  7.67e-16 &      \\ 
     &           &    8 &  2.69e-08 &  1.45e-15 &      \\ 
     &           &    9 &  2.69e-08 &  7.68e-16 &      \\ 
     &           &   10 &  5.08e-08 &  7.67e-16 &      \\ 
1697 &  1.70e+04 &   10 &           &           & iters  \\ 
 \hdashline 
     &           &    1 &  4.67e-08 &  4.34e-10 &      \\ 
     &           &    2 &  3.11e-08 &  1.33e-15 &      \\ 
     &           &    3 &  4.66e-08 &  8.88e-16 &      \\ 
     &           &    4 &  3.11e-08 &  1.33e-15 &      \\ 
     &           &    5 &  3.11e-08 &  8.88e-16 &      \\ 
     &           &    6 &  4.67e-08 &  8.87e-16 &      \\ 
     &           &    7 &  3.11e-08 &  1.33e-15 &      \\ 
     &           &    8 &  4.66e-08 &  8.88e-16 &      \\ 
     &           &    9 &  3.11e-08 &  1.33e-15 &      \\ 
     &           &   10 &  3.11e-08 &  8.88e-16 &      \\ 
1698 &  1.70e+04 &   10 &           &           & iters  \\ 
 \hdashline 
     &           &    1 &  2.45e-08 &  7.57e-10 &      \\ 
     &           &    2 &  1.43e-08 &  7.01e-16 &      \\ 
     &           &    3 &  2.45e-08 &  4.09e-16 &      \\ 
     &           &    4 &  1.43e-08 &  7.00e-16 &      \\ 
     &           &    5 &  1.43e-08 &  4.10e-16 &      \\ 
     &           &    6 &  2.45e-08 &  4.09e-16 &      \\ 
     &           &    7 &  1.43e-08 &  7.00e-16 &      \\ 
     &           &    8 &  1.43e-08 &  4.09e-16 &      \\ 
     &           &    9 &  2.45e-08 &  4.09e-16 &      \\ 
     &           &   10 &  1.43e-08 &  7.00e-16 &      \\ 
1699 &  1.70e+04 &   10 &           &           & iters  \\ 
 \hdashline 
     &           &    1 &  2.11e-09 &  2.28e-12 &      \\ 
     &           &    2 &  2.11e-09 &  6.02e-17 &      \\ 
     &           &    3 &  3.67e-08 &  6.01e-17 &      \\ 
     &           &    4 &  2.16e-09 &  1.05e-15 &      \\ 
     &           &    5 &  2.15e-09 &  6.15e-17 &      \\ 
     &           &    6 &  2.15e-09 &  6.14e-17 &      \\ 
     &           &    7 &  2.15e-09 &  6.13e-17 &      \\ 
     &           &    8 &  2.14e-09 &  6.12e-17 &      \\ 
     &           &    9 &  2.14e-09 &  6.12e-17 &      \\ 
     &           &   10 &  2.14e-09 &  6.11e-17 &      \\ 
1700 &  1.70e+04 &   10 &           &           & iters  \\ 
 \hdashline 
     &           &    1 &  2.08e-08 &  1.03e-09 &      \\ 
     &           &    2 &  1.81e-08 &  5.93e-16 &      \\ 
     &           &    3 &  2.08e-08 &  5.17e-16 &      \\ 
     &           &    4 &  1.81e-08 &  5.92e-16 &      \\ 
     &           &    5 &  2.08e-08 &  5.17e-16 &      \\ 
     &           &    6 &  1.81e-08 &  5.92e-16 &      \\ 
     &           &    7 &  2.08e-08 &  5.17e-16 &      \\ 
     &           &    8 &  1.81e-08 &  5.92e-16 &      \\ 
     &           &    9 &  1.81e-08 &  5.17e-16 &      \\ 
     &           &   10 &  2.08e-08 &  5.16e-16 &      \\ 
1701 &  1.70e+04 &   10 &           &           & iters  \\ 
 \hdashline 
     &           &    1 &  2.61e-09 &  1.27e-09 &      \\ 
     &           &    2 &  2.60e-09 &  7.44e-17 &      \\ 
     &           &    3 &  2.60e-09 &  7.43e-17 &      \\ 
     &           &    4 &  2.59e-09 &  7.42e-17 &      \\ 
     &           &    5 &  2.59e-09 &  7.40e-17 &      \\ 
     &           &    6 &  2.59e-09 &  7.39e-17 &      \\ 
     &           &    7 &  2.58e-09 &  7.38e-17 &      \\ 
     &           &    8 &  2.58e-09 &  7.37e-17 &      \\ 
     &           &    9 &  2.58e-09 &  7.36e-17 &      \\ 
     &           &   10 &  2.57e-09 &  7.35e-17 &      \\ 
1702 &  1.70e+04 &   10 &           &           & iters  \\ 
 \hdashline 
     &           &    1 &  2.28e-08 &  1.06e-10 &      \\ 
     &           &    2 &  2.27e-08 &  6.50e-16 &      \\ 
     &           &    3 &  5.50e-08 &  6.49e-16 &      \\ 
     &           &    4 &  2.28e-08 &  1.57e-15 &      \\ 
     &           &    5 &  2.27e-08 &  6.50e-16 &      \\ 
     &           &    6 &  5.50e-08 &  6.49e-16 &      \\ 
     &           &    7 &  2.28e-08 &  1.57e-15 &      \\ 
     &           &    8 &  2.28e-08 &  6.50e-16 &      \\ 
     &           &    9 &  2.27e-08 &  6.49e-16 &      \\ 
     &           &   10 &  5.50e-08 &  6.49e-16 &      \\ 
1703 &  1.70e+04 &   10 &           &           & iters  \\ 
 \hdashline 
     &           &    1 &  1.92e-09 &  2.75e-09 &      \\ 
     &           &    2 &  1.92e-09 &  5.49e-17 &      \\ 
     &           &    3 &  1.92e-09 &  5.48e-17 &      \\ 
     &           &    4 &  1.92e-09 &  5.48e-17 &      \\ 
     &           &    5 &  1.91e-09 &  5.47e-17 &      \\ 
     &           &    6 &  1.91e-09 &  5.46e-17 &      \\ 
     &           &    7 &  1.91e-09 &  5.45e-17 &      \\ 
     &           &    8 &  1.91e-09 &  5.45e-17 &      \\ 
     &           &    9 &  1.90e-09 &  5.44e-17 &      \\ 
     &           &   10 &  1.90e-09 &  5.43e-17 &      \\ 
1704 &  1.70e+04 &   10 &           &           & iters  \\ 
 \hdashline 
     &           &    1 &  2.57e-08 &  4.19e-09 &      \\ 
     &           &    2 &  2.57e-08 &  7.34e-16 &      \\ 
     &           &    3 &  5.20e-08 &  7.33e-16 &      \\ 
     &           &    4 &  2.57e-08 &  1.49e-15 &      \\ 
     &           &    5 &  2.57e-08 &  7.34e-16 &      \\ 
     &           &    6 &  5.20e-08 &  7.33e-16 &      \\ 
     &           &    7 &  2.57e-08 &  1.49e-15 &      \\ 
     &           &    8 &  2.57e-08 &  7.34e-16 &      \\ 
     &           &    9 &  5.20e-08 &  7.33e-16 &      \\ 
     &           &   10 &  2.57e-08 &  1.49e-15 &      \\ 
1705 &  1.70e+04 &   10 &           &           & iters  \\ 
 \hdashline 
     &           &    1 &  1.90e-08 &  3.91e-09 &      \\ 
     &           &    2 &  1.90e-08 &  5.43e-16 &      \\ 
     &           &    3 &  5.87e-08 &  5.42e-16 &      \\ 
     &           &    4 &  1.91e-08 &  1.68e-15 &      \\ 
     &           &    5 &  1.90e-08 &  5.44e-16 &      \\ 
     &           &    6 &  1.90e-08 &  5.43e-16 &      \\ 
     &           &    7 &  5.87e-08 &  5.42e-16 &      \\ 
     &           &    8 &  1.91e-08 &  1.68e-15 &      \\ 
     &           &    9 &  1.90e-08 &  5.44e-16 &      \\ 
     &           &   10 &  1.90e-08 &  5.43e-16 &      \\ 
1706 &  1.70e+04 &   10 &           &           & iters  \\ 
 \hdashline 
     &           &    1 &  4.44e-08 &  2.23e-09 &      \\ 
     &           &    2 &  3.33e-08 &  1.27e-15 &      \\ 
     &           &    3 &  3.33e-08 &  9.51e-16 &      \\ 
     &           &    4 &  4.45e-08 &  9.49e-16 &      \\ 
     &           &    5 &  3.33e-08 &  1.27e-15 &      \\ 
     &           &    6 &  4.45e-08 &  9.50e-16 &      \\ 
     &           &    7 &  3.33e-08 &  1.27e-15 &      \\ 
     &           &    8 &  4.44e-08 &  9.50e-16 &      \\ 
     &           &    9 &  3.33e-08 &  1.27e-15 &      \\ 
     &           &   10 &  3.33e-08 &  9.51e-16 &      \\ 
1707 &  1.71e+04 &   10 &           &           & iters  \\ 
 \hdashline 
     &           &    1 &  6.61e-08 &  7.04e-10 &      \\ 
     &           &    2 &  1.17e-08 &  1.89e-15 &      \\ 
     &           &    3 &  1.17e-08 &  3.33e-16 &      \\ 
     &           &    4 &  1.16e-08 &  3.33e-16 &      \\ 
     &           &    5 &  1.16e-08 &  3.32e-16 &      \\ 
     &           &    6 &  1.16e-08 &  3.32e-16 &      \\ 
     &           &    7 &  6.61e-08 &  3.31e-16 &      \\ 
     &           &    8 &  1.17e-08 &  1.89e-15 &      \\ 
     &           &    9 &  1.17e-08 &  3.34e-16 &      \\ 
     &           &   10 &  1.17e-08 &  3.33e-16 &      \\ 
1708 &  1.71e+04 &   10 &           &           & iters  \\ 
 \hdashline 
     &           &    1 &  3.57e-08 &  1.79e-09 &      \\ 
     &           &    2 &  4.20e-08 &  1.02e-15 &      \\ 
     &           &    3 &  3.57e-08 &  1.20e-15 &      \\ 
     &           &    4 &  4.20e-08 &  1.02e-15 &      \\ 
     &           &    5 &  3.57e-08 &  1.20e-15 &      \\ 
     &           &    6 &  3.57e-08 &  1.02e-15 &      \\ 
     &           &    7 &  4.20e-08 &  1.02e-15 &      \\ 
     &           &    8 &  3.57e-08 &  1.20e-15 &      \\ 
     &           &    9 &  4.20e-08 &  1.02e-15 &      \\ 
     &           &   10 &  3.57e-08 &  1.20e-15 &      \\ 
1709 &  1.71e+04 &   10 &           &           & iters  \\ 
 \hdashline 
     &           &    1 &  6.96e-08 &  7.52e-10 &      \\ 
     &           &    2 &  8.18e-09 &  1.99e-15 &      \\ 
     &           &    3 &  8.17e-09 &  2.34e-16 &      \\ 
     &           &    4 &  8.16e-09 &  2.33e-16 &      \\ 
     &           &    5 &  8.15e-09 &  2.33e-16 &      \\ 
     &           &    6 &  8.14e-09 &  2.33e-16 &      \\ 
     &           &    7 &  8.13e-09 &  2.32e-16 &      \\ 
     &           &    8 &  6.96e-08 &  2.32e-16 &      \\ 
     &           &    9 &  8.59e-08 &  1.99e-15 &      \\ 
     &           &   10 &  6.96e-08 &  2.45e-15 &      \\ 
1710 &  1.71e+04 &   10 &           &           & iters  \\ 
 \hdashline 
     &           &    1 &  2.50e-08 &  1.36e-10 &      \\ 
     &           &    2 &  2.50e-08 &  7.14e-16 &      \\ 
     &           &    3 &  5.27e-08 &  7.13e-16 &      \\ 
     &           &    4 &  2.50e-08 &  1.51e-15 &      \\ 
     &           &    5 &  2.50e-08 &  7.14e-16 &      \\ 
     &           &    6 &  5.27e-08 &  7.13e-16 &      \\ 
     &           &    7 &  2.50e-08 &  1.51e-15 &      \\ 
     &           &    8 &  2.50e-08 &  7.14e-16 &      \\ 
     &           &    9 &  5.27e-08 &  7.13e-16 &      \\ 
     &           &   10 &  2.50e-08 &  1.50e-15 &      \\ 
1711 &  1.71e+04 &   10 &           &           & iters  \\ 
 \hdashline 
     &           &    1 &  1.89e-08 &  9.18e-10 &      \\ 
     &           &    2 &  5.89e-08 &  5.38e-16 &      \\ 
     &           &    3 &  1.89e-08 &  1.68e-15 &      \\ 
     &           &    4 &  1.89e-08 &  5.40e-16 &      \\ 
     &           &    5 &  1.89e-08 &  5.39e-16 &      \\ 
     &           &    6 &  5.89e-08 &  5.38e-16 &      \\ 
     &           &    7 &  1.89e-08 &  1.68e-15 &      \\ 
     &           &    8 &  1.89e-08 &  5.40e-16 &      \\ 
     &           &    9 &  1.89e-08 &  5.39e-16 &      \\ 
     &           &   10 &  5.89e-08 &  5.38e-16 &      \\ 
1712 &  1.71e+04 &   10 &           &           & iters  \\ 
 \hdashline 
     &           &    1 &  6.93e-08 &  2.48e-09 &      \\ 
     &           &    2 &  8.51e-09 &  1.98e-15 &      \\ 
     &           &    3 &  8.50e-09 &  2.43e-16 &      \\ 
     &           &    4 &  8.48e-09 &  2.42e-16 &      \\ 
     &           &    5 &  8.47e-09 &  2.42e-16 &      \\ 
     &           &    6 &  8.46e-09 &  2.42e-16 &      \\ 
     &           &    7 &  8.45e-09 &  2.41e-16 &      \\ 
     &           &    8 &  8.43e-09 &  2.41e-16 &      \\ 
     &           &    9 &  6.93e-08 &  2.41e-16 &      \\ 
     &           &   10 &  8.52e-09 &  1.98e-15 &      \\ 
1713 &  1.71e+04 &   10 &           &           & iters  \\ 
 \hdashline 
     &           &    1 &  7.47e-09 &  2.93e-09 &      \\ 
     &           &    2 &  7.46e-09 &  2.13e-16 &      \\ 
     &           &    3 &  7.45e-09 &  2.13e-16 &      \\ 
     &           &    4 &  7.43e-09 &  2.13e-16 &      \\ 
     &           &    5 &  7.42e-09 &  2.12e-16 &      \\ 
     &           &    6 &  7.41e-09 &  2.12e-16 &      \\ 
     &           &    7 &  7.03e-08 &  2.12e-16 &      \\ 
     &           &    8 &  8.52e-08 &  2.01e-15 &      \\ 
     &           &    9 &  7.03e-08 &  2.43e-15 &      \\ 
     &           &   10 &  7.48e-09 &  2.01e-15 &      \\ 
1714 &  1.71e+04 &   10 &           &           & iters  \\ 
 \hdashline 
     &           &    1 &  3.29e-09 &  1.17e-09 &      \\ 
     &           &    2 &  3.29e-09 &  9.39e-17 &      \\ 
     &           &    3 &  3.28e-09 &  9.38e-17 &      \\ 
     &           &    4 &  3.28e-09 &  9.37e-17 &      \\ 
     &           &    5 &  3.27e-09 &  9.35e-17 &      \\ 
     &           &    6 &  3.27e-09 &  9.34e-17 &      \\ 
     &           &    7 &  7.44e-08 &  9.33e-17 &      \\ 
     &           &    8 &  4.22e-08 &  2.12e-15 &      \\ 
     &           &    9 &  3.31e-09 &  1.20e-15 &      \\ 
     &           &   10 &  3.30e-09 &  9.45e-17 &      \\ 
1715 &  1.71e+04 &   10 &           &           & iters  \\ 
 \hdashline 
     &           &    1 &  9.82e-09 &  9.38e-10 &      \\ 
     &           &    2 &  9.80e-09 &  2.80e-16 &      \\ 
     &           &    3 &  9.79e-09 &  2.80e-16 &      \\ 
     &           &    4 &  9.78e-09 &  2.79e-16 &      \\ 
     &           &    5 &  6.79e-08 &  2.79e-16 &      \\ 
     &           &    6 &  8.75e-08 &  1.94e-15 &      \\ 
     &           &    7 &  6.80e-08 &  2.50e-15 &      \\ 
     &           &    8 &  9.83e-09 &  1.94e-15 &      \\ 
     &           &    9 &  9.82e-09 &  2.81e-16 &      \\ 
     &           &   10 &  9.80e-09 &  2.80e-16 &      \\ 
1716 &  1.72e+04 &   10 &           &           & iters  \\ 
 \hdashline 
     &           &    1 &  3.90e-08 &  1.74e-09 &      \\ 
     &           &    2 &  3.88e-08 &  1.11e-15 &      \\ 
     &           &    3 &  3.90e-08 &  1.11e-15 &      \\ 
     &           &    4 &  3.88e-08 &  1.11e-15 &      \\ 
     &           &    5 &  3.90e-08 &  1.11e-15 &      \\ 
     &           &    6 &  3.88e-08 &  1.11e-15 &      \\ 
     &           &    7 &  3.90e-08 &  1.11e-15 &      \\ 
     &           &    8 &  3.88e-08 &  1.11e-15 &      \\ 
     &           &    9 &  3.90e-08 &  1.11e-15 &      \\ 
     &           &   10 &  3.88e-08 &  1.11e-15 &      \\ 
1717 &  1.72e+04 &   10 &           &           & iters  \\ 
 \hdashline 
     &           &    1 &  1.56e-08 &  3.92e-09 &      \\ 
     &           &    2 &  1.55e-08 &  4.44e-16 &      \\ 
     &           &    3 &  1.55e-08 &  4.43e-16 &      \\ 
     &           &    4 &  6.22e-08 &  4.43e-16 &      \\ 
     &           &    5 &  1.56e-08 &  1.78e-15 &      \\ 
     &           &    6 &  1.56e-08 &  4.45e-16 &      \\ 
     &           &    7 &  1.55e-08 &  4.44e-16 &      \\ 
     &           &    8 &  1.55e-08 &  4.43e-16 &      \\ 
     &           &    9 &  6.22e-08 &  4.43e-16 &      \\ 
     &           &   10 &  1.56e-08 &  1.78e-15 &      \\ 
1718 &  1.72e+04 &   10 &           &           & iters  \\ 
 \hdashline 
     &           &    1 &  6.81e-08 &  4.61e-09 &      \\ 
     &           &    2 &  8.74e-08 &  1.94e-15 &      \\ 
     &           &    3 &  6.81e-08 &  2.49e-15 &      \\ 
     &           &    4 &  9.66e-09 &  1.94e-15 &      \\ 
     &           &    5 &  9.65e-09 &  2.76e-16 &      \\ 
     &           &    6 &  9.63e-09 &  2.75e-16 &      \\ 
     &           &    7 &  9.62e-09 &  2.75e-16 &      \\ 
     &           &    8 &  9.61e-09 &  2.75e-16 &      \\ 
     &           &    9 &  9.59e-09 &  2.74e-16 &      \\ 
     &           &   10 &  6.81e-08 &  2.74e-16 &      \\ 
1719 &  1.72e+04 &   10 &           &           & iters  \\ 
 \hdashline 
     &           &    1 &  6.94e-08 &  7.40e-10 &      \\ 
     &           &    2 &  8.39e-09 &  1.98e-15 &      \\ 
     &           &    3 &  8.38e-09 &  2.39e-16 &      \\ 
     &           &    4 &  8.37e-09 &  2.39e-16 &      \\ 
     &           &    5 &  6.93e-08 &  2.39e-16 &      \\ 
     &           &    6 &  8.61e-08 &  1.98e-15 &      \\ 
     &           &    7 &  6.94e-08 &  2.46e-15 &      \\ 
     &           &    8 &  8.43e-09 &  1.98e-15 &      \\ 
     &           &    9 &  8.42e-09 &  2.41e-16 &      \\ 
     &           &   10 &  8.41e-09 &  2.40e-16 &      \\ 
1720 &  1.72e+04 &   10 &           &           & iters  \\ 
 \hdashline 
     &           &    1 &  5.66e-08 &  1.89e-09 &      \\ 
     &           &    2 &  2.12e-08 &  1.62e-15 &      \\ 
     &           &    3 &  2.11e-08 &  6.04e-16 &      \\ 
     &           &    4 &  5.66e-08 &  6.03e-16 &      \\ 
     &           &    5 &  2.12e-08 &  1.61e-15 &      \\ 
     &           &    6 &  2.12e-08 &  6.05e-16 &      \\ 
     &           &    7 &  2.11e-08 &  6.04e-16 &      \\ 
     &           &    8 &  5.66e-08 &  6.03e-16 &      \\ 
     &           &    9 &  2.12e-08 &  1.62e-15 &      \\ 
     &           &   10 &  2.12e-08 &  6.05e-16 &      \\ 
1721 &  1.72e+04 &   10 &           &           & iters  \\ 
 \hdashline 
     &           &    1 &  2.59e-08 &  1.52e-09 &      \\ 
     &           &    2 &  5.18e-08 &  7.39e-16 &      \\ 
     &           &    3 &  2.59e-08 &  1.48e-15 &      \\ 
     &           &    4 &  2.59e-08 &  7.40e-16 &      \\ 
     &           &    5 &  5.18e-08 &  7.39e-16 &      \\ 
     &           &    6 &  2.59e-08 &  1.48e-15 &      \\ 
     &           &    7 &  2.59e-08 &  7.40e-16 &      \\ 
     &           &    8 &  5.18e-08 &  7.39e-16 &      \\ 
     &           &    9 &  2.59e-08 &  1.48e-15 &      \\ 
     &           &   10 &  2.59e-08 &  7.40e-16 &      \\ 
1722 &  1.72e+04 &   10 &           &           & iters  \\ 
 \hdashline 
     &           &    1 &  7.40e-08 &  8.59e-11 &      \\ 
     &           &    2 &  8.15e-08 &  2.11e-15 &      \\ 
     &           &    3 &  7.40e-08 &  2.33e-15 &      \\ 
     &           &    4 &  3.81e-09 &  2.11e-15 &      \\ 
     &           &    5 &  3.80e-09 &  1.09e-16 &      \\ 
     &           &    6 &  3.80e-09 &  1.09e-16 &      \\ 
     &           &    7 &  3.79e-09 &  1.08e-16 &      \\ 
     &           &    8 &  3.79e-09 &  1.08e-16 &      \\ 
     &           &    9 &  3.78e-09 &  1.08e-16 &      \\ 
     &           &   10 &  3.78e-09 &  1.08e-16 &      \\ 
1723 &  1.72e+04 &   10 &           &           & iters  \\ 
 \hdashline 
     &           &    1 &  2.99e-08 &  2.27e-10 &      \\ 
     &           &    2 &  4.79e-08 &  8.52e-16 &      \\ 
     &           &    3 &  2.99e-08 &  1.37e-15 &      \\ 
     &           &    4 &  2.98e-08 &  8.53e-16 &      \\ 
     &           &    5 &  4.79e-08 &  8.51e-16 &      \\ 
     &           &    6 &  2.99e-08 &  1.37e-15 &      \\ 
     &           &    7 &  4.79e-08 &  8.52e-16 &      \\ 
     &           &    8 &  2.99e-08 &  1.37e-15 &      \\ 
     &           &    9 &  2.98e-08 &  8.53e-16 &      \\ 
     &           &   10 &  4.79e-08 &  8.52e-16 &      \\ 
1724 &  1.72e+04 &   10 &           &           & iters  \\ 
 \hdashline 
     &           &    1 &  8.63e-08 &  1.83e-09 &      \\ 
     &           &    2 &  6.92e-08 &  2.46e-15 &      \\ 
     &           &    3 &  8.58e-09 &  1.98e-15 &      \\ 
     &           &    4 &  8.57e-09 &  2.45e-16 &      \\ 
     &           &    5 &  8.55e-09 &  2.44e-16 &      \\ 
     &           &    6 &  8.54e-09 &  2.44e-16 &      \\ 
     &           &    7 &  8.53e-09 &  2.44e-16 &      \\ 
     &           &    8 &  8.52e-09 &  2.43e-16 &      \\ 
     &           &    9 &  6.92e-08 &  2.43e-16 &      \\ 
     &           &   10 &  8.63e-08 &  1.97e-15 &      \\ 
1725 &  1.72e+04 &   10 &           &           & iters  \\ 
 \hdashline 
     &           &    1 &  5.52e-08 &  1.55e-09 &      \\ 
     &           &    2 &  5.51e-08 &  1.58e-15 &      \\ 
     &           &    3 &  2.27e-08 &  1.57e-15 &      \\ 
     &           &    4 &  2.26e-08 &  6.47e-16 &      \\ 
     &           &    5 &  2.26e-08 &  6.46e-16 &      \\ 
     &           &    6 &  5.51e-08 &  6.45e-16 &      \\ 
     &           &    7 &  2.26e-08 &  1.57e-15 &      \\ 
     &           &    8 &  2.26e-08 &  6.46e-16 &      \\ 
     &           &    9 &  5.51e-08 &  6.45e-16 &      \\ 
     &           &   10 &  2.27e-08 &  1.57e-15 &      \\ 
1726 &  1.72e+04 &   10 &           &           & iters  \\ 
 \hdashline 
     &           &    1 &  3.55e-08 &  8.46e-10 &      \\ 
     &           &    2 &  4.22e-08 &  1.01e-15 &      \\ 
     &           &    3 &  3.55e-08 &  1.20e-15 &      \\ 
     &           &    4 &  4.22e-08 &  1.01e-15 &      \\ 
     &           &    5 &  3.55e-08 &  1.20e-15 &      \\ 
     &           &    6 &  4.22e-08 &  1.01e-15 &      \\ 
     &           &    7 &  3.55e-08 &  1.20e-15 &      \\ 
     &           &    8 &  4.22e-08 &  1.01e-15 &      \\ 
     &           &    9 &  3.56e-08 &  1.20e-15 &      \\ 
     &           &   10 &  4.22e-08 &  1.01e-15 &      \\ 
1727 &  1.73e+04 &   10 &           &           & iters  \\ 
 \hdashline 
     &           &    1 &  2.48e-08 &  1.27e-10 &      \\ 
     &           &    2 &  2.48e-08 &  7.09e-16 &      \\ 
     &           &    3 &  2.48e-08 &  7.08e-16 &      \\ 
     &           &    4 &  1.41e-08 &  7.07e-16 &      \\ 
     &           &    5 &  2.48e-08 &  4.03e-16 &      \\ 
     &           &    6 &  1.41e-08 &  7.07e-16 &      \\ 
     &           &    7 &  1.41e-08 &  4.03e-16 &      \\ 
     &           &    8 &  2.48e-08 &  4.03e-16 &      \\ 
     &           &    9 &  1.41e-08 &  7.07e-16 &      \\ 
     &           &   10 &  1.41e-08 &  4.03e-16 &      \\ 
1728 &  1.73e+04 &   10 &           &           & iters  \\ 
 \hdashline 
     &           &    1 &  3.48e-08 &  3.90e-10 &      \\ 
     &           &    2 &  4.29e-08 &  9.93e-16 &      \\ 
     &           &    3 &  3.48e-08 &  1.23e-15 &      \\ 
     &           &    4 &  4.29e-08 &  9.93e-16 &      \\ 
     &           &    5 &  3.48e-08 &  1.23e-15 &      \\ 
     &           &    6 &  3.48e-08 &  9.94e-16 &      \\ 
     &           &    7 &  4.30e-08 &  9.92e-16 &      \\ 
     &           &    8 &  3.48e-08 &  1.23e-15 &      \\ 
     &           &    9 &  4.30e-08 &  9.92e-16 &      \\ 
     &           &   10 &  3.48e-08 &  1.23e-15 &      \\ 
1729 &  1.73e+04 &   10 &           &           & iters  \\ 
 \hdashline 
     &           &    1 &  1.69e-08 &  2.38e-10 &      \\ 
     &           &    2 &  1.69e-08 &  4.83e-16 &      \\ 
     &           &    3 &  1.69e-08 &  4.82e-16 &      \\ 
     &           &    4 &  6.08e-08 &  4.81e-16 &      \\ 
     &           &    5 &  1.69e-08 &  1.74e-15 &      \\ 
     &           &    6 &  1.69e-08 &  4.83e-16 &      \\ 
     &           &    7 &  1.69e-08 &  4.82e-16 &      \\ 
     &           &    8 &  1.69e-08 &  4.82e-16 &      \\ 
     &           &    9 &  6.09e-08 &  4.81e-16 &      \\ 
     &           &   10 &  1.69e-08 &  1.74e-15 &      \\ 
1730 &  1.73e+04 &   10 &           &           & iters  \\ 
 \hdashline 
     &           &    1 &  3.92e-11 &  8.77e-10 &      \\ 
     &           &    2 &  3.92e-11 &  1.12e-18 &      \\ 
     &           &    3 &  3.91e-11 &  1.12e-18 &      \\ 
     &           &    4 &  3.91e-11 &  1.12e-18 &      \\ 
     &           &    5 &  3.90e-11 &  1.11e-18 &      \\ 
     &           &    6 &  3.89e-11 &  1.11e-18 &      \\ 
     &           &    7 &  3.89e-11 &  1.11e-18 &      \\ 
     &           &    8 &  3.88e-11 &  1.11e-18 &      \\ 
     &           &    9 &  3.88e-11 &  1.11e-18 &      \\ 
     &           &   10 &  3.87e-11 &  1.11e-18 &      \\ 
1731 &  1.73e+04 &   10 &           &           & iters  \\ 
 \hdashline 
     &           &    1 &  2.37e-08 &  1.33e-09 &      \\ 
     &           &    2 &  5.41e-08 &  6.75e-16 &      \\ 
     &           &    3 &  2.37e-08 &  1.54e-15 &      \\ 
     &           &    4 &  2.37e-08 &  6.76e-16 &      \\ 
     &           &    5 &  5.41e-08 &  6.75e-16 &      \\ 
     &           &    6 &  2.37e-08 &  1.54e-15 &      \\ 
     &           &    7 &  2.37e-08 &  6.77e-16 &      \\ 
     &           &    8 &  5.40e-08 &  6.76e-16 &      \\ 
     &           &    9 &  2.37e-08 &  1.54e-15 &      \\ 
     &           &   10 &  2.37e-08 &  6.77e-16 &      \\ 
1732 &  1.73e+04 &   10 &           &           & iters  \\ 
 \hdashline 
     &           &    1 &  2.05e-08 &  1.17e-09 &      \\ 
     &           &    2 &  5.72e-08 &  5.86e-16 &      \\ 
     &           &    3 &  2.06e-08 &  1.63e-15 &      \\ 
     &           &    4 &  2.06e-08 &  5.88e-16 &      \\ 
     &           &    5 &  2.05e-08 &  5.87e-16 &      \\ 
     &           &    6 &  5.72e-08 &  5.86e-16 &      \\ 
     &           &    7 &  2.06e-08 &  1.63e-15 &      \\ 
     &           &    8 &  2.06e-08 &  5.87e-16 &      \\ 
     &           &    9 &  5.72e-08 &  5.87e-16 &      \\ 
     &           &   10 &  2.06e-08 &  1.63e-15 &      \\ 
1733 &  1.73e+04 &   10 &           &           & iters  \\ 
 \hdashline 
     &           &    1 &  3.09e-08 &  1.82e-09 &      \\ 
     &           &    2 &  4.68e-08 &  8.82e-16 &      \\ 
     &           &    3 &  3.09e-08 &  1.34e-15 &      \\ 
     &           &    4 &  3.09e-08 &  8.82e-16 &      \\ 
     &           &    5 &  4.69e-08 &  8.81e-16 &      \\ 
     &           &    6 &  3.09e-08 &  1.34e-15 &      \\ 
     &           &    7 &  4.68e-08 &  8.82e-16 &      \\ 
     &           &    8 &  3.09e-08 &  1.34e-15 &      \\ 
     &           &    9 &  3.09e-08 &  8.82e-16 &      \\ 
     &           &   10 &  4.69e-08 &  8.81e-16 &      \\ 
1734 &  1.73e+04 &   10 &           &           & iters  \\ 
 \hdashline 
     &           &    1 &  1.74e-08 &  2.82e-10 &      \\ 
     &           &    2 &  2.14e-08 &  4.97e-16 &      \\ 
     &           &    3 &  1.74e-08 &  6.12e-16 &      \\ 
     &           &    4 &  2.14e-08 &  4.98e-16 &      \\ 
     &           &    5 &  1.74e-08 &  6.12e-16 &      \\ 
     &           &    6 &  1.74e-08 &  4.98e-16 &      \\ 
     &           &    7 &  2.15e-08 &  4.97e-16 &      \\ 
     &           &    8 &  1.74e-08 &  6.12e-16 &      \\ 
     &           &    9 &  2.14e-08 &  4.97e-16 &      \\ 
     &           &   10 &  1.74e-08 &  6.12e-16 &      \\ 
1735 &  1.73e+04 &   10 &           &           & iters  \\ 
 \hdashline 
     &           &    1 &  1.78e-08 &  1.59e-09 &      \\ 
     &           &    2 &  1.78e-08 &  5.08e-16 &      \\ 
     &           &    3 &  1.77e-08 &  5.07e-16 &      \\ 
     &           &    4 &  6.00e-08 &  5.06e-16 &      \\ 
     &           &    5 &  1.78e-08 &  1.71e-15 &      \\ 
     &           &    6 &  1.78e-08 &  5.08e-16 &      \\ 
     &           &    7 &  1.77e-08 &  5.07e-16 &      \\ 
     &           &    8 &  1.77e-08 &  5.06e-16 &      \\ 
     &           &    9 &  6.00e-08 &  5.06e-16 &      \\ 
     &           &   10 &  1.78e-08 &  1.71e-15 &      \\ 
1736 &  1.74e+04 &   10 &           &           & iters  \\ 
 \hdashline 
     &           &    1 &  9.85e-10 &  5.20e-10 &      \\ 
     &           &    2 &  9.84e-10 &  2.81e-17 &      \\ 
     &           &    3 &  9.83e-10 &  2.81e-17 &      \\ 
     &           &    4 &  9.81e-10 &  2.80e-17 &      \\ 
     &           &    5 &  9.80e-10 &  2.80e-17 &      \\ 
     &           &    6 &  9.78e-10 &  2.80e-17 &      \\ 
     &           &    7 &  9.77e-10 &  2.79e-17 &      \\ 
     &           &    8 &  9.76e-10 &  2.79e-17 &      \\ 
     &           &    9 &  9.74e-10 &  2.78e-17 &      \\ 
     &           &   10 &  9.73e-10 &  2.78e-17 &      \\ 
1737 &  1.74e+04 &   10 &           &           & iters  \\ 
 \hdashline 
     &           &    1 &  1.56e-08 &  1.12e-09 &      \\ 
     &           &    2 &  1.55e-08 &  4.44e-16 &      \\ 
     &           &    3 &  6.22e-08 &  4.44e-16 &      \\ 
     &           &    4 &  1.56e-08 &  1.77e-15 &      \\ 
     &           &    5 &  1.56e-08 &  4.46e-16 &      \\ 
     &           &    6 &  1.56e-08 &  4.45e-16 &      \\ 
     &           &    7 &  1.55e-08 &  4.44e-16 &      \\ 
     &           &    8 &  6.22e-08 &  4.44e-16 &      \\ 
     &           &    9 &  1.56e-08 &  1.77e-15 &      \\ 
     &           &   10 &  1.56e-08 &  4.46e-16 &      \\ 
1738 &  1.74e+04 &   10 &           &           & iters  \\ 
 \hdashline 
     &           &    1 &  4.73e-08 &  4.14e-10 &      \\ 
     &           &    2 &  3.04e-08 &  1.35e-15 &      \\ 
     &           &    3 &  3.04e-08 &  8.69e-16 &      \\ 
     &           &    4 &  4.73e-08 &  8.68e-16 &      \\ 
     &           &    5 &  3.04e-08 &  1.35e-15 &      \\ 
     &           &    6 &  4.73e-08 &  8.68e-16 &      \\ 
     &           &    7 &  3.04e-08 &  1.35e-15 &      \\ 
     &           &    8 &  3.04e-08 &  8.69e-16 &      \\ 
     &           &    9 &  4.73e-08 &  8.68e-16 &      \\ 
     &           &   10 &  3.04e-08 &  1.35e-15 &      \\ 
1739 &  1.74e+04 &   10 &           &           & iters  \\ 
 \hdashline 
     &           &    1 &  8.35e-09 &  1.47e-09 &      \\ 
     &           &    2 &  8.34e-09 &  2.38e-16 &      \\ 
     &           &    3 &  8.33e-09 &  2.38e-16 &      \\ 
     &           &    4 &  8.32e-09 &  2.38e-16 &      \\ 
     &           &    5 &  8.31e-09 &  2.37e-16 &      \\ 
     &           &    6 &  8.29e-09 &  2.37e-16 &      \\ 
     &           &    7 &  8.28e-09 &  2.37e-16 &      \\ 
     &           &    8 &  6.94e-08 &  2.36e-16 &      \\ 
     &           &    9 &  8.37e-09 &  1.98e-15 &      \\ 
     &           &   10 &  8.36e-09 &  2.39e-16 &      \\ 
1740 &  1.74e+04 &   10 &           &           & iters  \\ 
 \hdashline 
     &           &    1 &  3.56e-08 &  9.89e-10 &      \\ 
     &           &    2 &  4.22e-08 &  1.02e-15 &      \\ 
     &           &    3 &  3.56e-08 &  1.20e-15 &      \\ 
     &           &    4 &  4.21e-08 &  1.02e-15 &      \\ 
     &           &    5 &  3.56e-08 &  1.20e-15 &      \\ 
     &           &    6 &  4.21e-08 &  1.02e-15 &      \\ 
     &           &    7 &  3.56e-08 &  1.20e-15 &      \\ 
     &           &    8 &  4.21e-08 &  1.02e-15 &      \\ 
     &           &    9 &  3.56e-08 &  1.20e-15 &      \\ 
     &           &   10 &  4.21e-08 &  1.02e-15 &      \\ 
1741 &  1.74e+04 &   10 &           &           & iters  \\ 
 \hdashline 
     &           &    1 &  4.24e-08 &  7.19e-10 &      \\ 
     &           &    2 &  3.53e-08 &  1.21e-15 &      \\ 
     &           &    3 &  4.24e-08 &  1.01e-15 &      \\ 
     &           &    4 &  3.53e-08 &  1.21e-15 &      \\ 
     &           &    5 &  4.24e-08 &  1.01e-15 &      \\ 
     &           &    6 &  3.53e-08 &  1.21e-15 &      \\ 
     &           &    7 &  4.24e-08 &  1.01e-15 &      \\ 
     &           &    8 &  3.53e-08 &  1.21e-15 &      \\ 
     &           &    9 &  4.24e-08 &  1.01e-15 &      \\ 
     &           &   10 &  3.54e-08 &  1.21e-15 &      \\ 
1742 &  1.74e+04 &   10 &           &           & iters  \\ 
 \hdashline 
     &           &    1 &  3.54e-08 &  5.24e-10 &      \\ 
     &           &    2 &  4.24e-08 &  1.01e-15 &      \\ 
     &           &    3 &  3.54e-08 &  1.21e-15 &      \\ 
     &           &    4 &  4.24e-08 &  1.01e-15 &      \\ 
     &           &    5 &  3.54e-08 &  1.21e-15 &      \\ 
     &           &    6 &  4.23e-08 &  1.01e-15 &      \\ 
     &           &    7 &  3.54e-08 &  1.21e-15 &      \\ 
     &           &    8 &  4.23e-08 &  1.01e-15 &      \\ 
     &           &    9 &  3.54e-08 &  1.21e-15 &      \\ 
     &           &   10 &  4.23e-08 &  1.01e-15 &      \\ 
1743 &  1.74e+04 &   10 &           &           & iters  \\ 
 \hdashline 
     &           &    1 &  3.75e-08 &  5.05e-10 &      \\ 
     &           &    2 &  4.03e-08 &  1.07e-15 &      \\ 
     &           &    3 &  3.75e-08 &  1.15e-15 &      \\ 
     &           &    4 &  3.74e-08 &  1.07e-15 &      \\ 
     &           &    5 &  4.03e-08 &  1.07e-15 &      \\ 
     &           &    6 &  3.74e-08 &  1.15e-15 &      \\ 
     &           &    7 &  4.03e-08 &  1.07e-15 &      \\ 
     &           &    8 &  3.74e-08 &  1.15e-15 &      \\ 
     &           &    9 &  4.03e-08 &  1.07e-15 &      \\ 
     &           &   10 &  3.74e-08 &  1.15e-15 &      \\ 
1744 &  1.74e+04 &   10 &           &           & iters  \\ 
 \hdashline 
     &           &    1 &  2.62e-08 &  3.34e-10 &      \\ 
     &           &    2 &  2.62e-08 &  7.49e-16 &      \\ 
     &           &    3 &  5.15e-08 &  7.48e-16 &      \\ 
     &           &    4 &  2.62e-08 &  1.47e-15 &      \\ 
     &           &    5 &  2.62e-08 &  7.49e-16 &      \\ 
     &           &    6 &  5.15e-08 &  7.48e-16 &      \\ 
     &           &    7 &  2.62e-08 &  1.47e-15 &      \\ 
     &           &    8 &  2.62e-08 &  7.49e-16 &      \\ 
     &           &    9 &  5.15e-08 &  7.48e-16 &      \\ 
     &           &   10 &  2.62e-08 &  1.47e-15 &      \\ 
1745 &  1.74e+04 &   10 &           &           & iters  \\ 
 \hdashline 
     &           &    1 &  2.49e-08 &  4.12e-10 &      \\ 
     &           &    2 &  2.49e-08 &  7.11e-16 &      \\ 
     &           &    3 &  5.29e-08 &  7.10e-16 &      \\ 
     &           &    4 &  2.49e-08 &  1.51e-15 &      \\ 
     &           &    5 &  2.49e-08 &  7.11e-16 &      \\ 
     &           &    6 &  5.29e-08 &  7.10e-16 &      \\ 
     &           &    7 &  2.49e-08 &  1.51e-15 &      \\ 
     &           &    8 &  2.49e-08 &  7.11e-16 &      \\ 
     &           &    9 &  5.28e-08 &  7.10e-16 &      \\ 
     &           &   10 &  2.49e-08 &  1.51e-15 &      \\ 
1746 &  1.74e+04 &   10 &           &           & iters  \\ 
 \hdashline 
     &           &    1 &  5.30e-08 &  4.87e-10 &      \\ 
     &           &    2 &  2.48e-08 &  1.51e-15 &      \\ 
     &           &    3 &  2.48e-08 &  7.08e-16 &      \\ 
     &           &    4 &  5.29e-08 &  7.07e-16 &      \\ 
     &           &    5 &  2.48e-08 &  1.51e-15 &      \\ 
     &           &    6 &  2.48e-08 &  7.08e-16 &      \\ 
     &           &    7 &  5.29e-08 &  7.07e-16 &      \\ 
     &           &    8 &  2.48e-08 &  1.51e-15 &      \\ 
     &           &    9 &  2.48e-08 &  7.08e-16 &      \\ 
     &           &   10 &  5.29e-08 &  7.07e-16 &      \\ 
1747 &  1.75e+04 &   10 &           &           & iters  \\ 
 \hdashline 
     &           &    1 &  4.86e-08 &  3.43e-10 &      \\ 
     &           &    2 &  4.86e-08 &  1.39e-15 &      \\ 
     &           &    3 &  2.92e-08 &  1.39e-15 &      \\ 
     &           &    4 &  2.91e-08 &  8.33e-16 &      \\ 
     &           &    5 &  4.86e-08 &  8.32e-16 &      \\ 
     &           &    6 &  2.92e-08 &  1.39e-15 &      \\ 
     &           &    7 &  4.86e-08 &  8.32e-16 &      \\ 
     &           &    8 &  2.92e-08 &  1.39e-15 &      \\ 
     &           &    9 &  2.92e-08 &  8.33e-16 &      \\ 
     &           &   10 &  4.86e-08 &  8.32e-16 &      \\ 
1748 &  1.75e+04 &   10 &           &           & iters  \\ 
 \hdashline 
     &           &    1 &  5.85e-08 &  5.08e-10 &      \\ 
     &           &    2 &  1.92e-08 &  1.67e-15 &      \\ 
     &           &    3 &  1.92e-08 &  5.49e-16 &      \\ 
     &           &    4 &  5.85e-08 &  5.48e-16 &      \\ 
     &           &    5 &  1.93e-08 &  1.67e-15 &      \\ 
     &           &    6 &  1.92e-08 &  5.50e-16 &      \\ 
     &           &    7 &  1.92e-08 &  5.49e-16 &      \\ 
     &           &    8 &  5.85e-08 &  5.48e-16 &      \\ 
     &           &    9 &  1.93e-08 &  1.67e-15 &      \\ 
     &           &   10 &  1.92e-08 &  5.50e-16 &      \\ 
1749 &  1.75e+04 &   10 &           &           & iters  \\ 
 \hdashline 
     &           &    1 &  4.32e-08 &  7.22e-11 &      \\ 
     &           &    2 &  3.45e-08 &  1.23e-15 &      \\ 
     &           &    3 &  4.32e-08 &  9.85e-16 &      \\ 
     &           &    4 &  3.45e-08 &  1.23e-15 &      \\ 
     &           &    5 &  4.32e-08 &  9.86e-16 &      \\ 
     &           &    6 &  3.45e-08 &  1.23e-15 &      \\ 
     &           &    7 &  3.45e-08 &  9.86e-16 &      \\ 
     &           &    8 &  4.32e-08 &  9.84e-16 &      \\ 
     &           &    9 &  3.45e-08 &  1.23e-15 &      \\ 
     &           &   10 &  4.32e-08 &  9.85e-16 &      \\ 
1750 &  1.75e+04 &   10 &           &           & iters  \\ 
 \hdashline 
     &           &    1 &  1.13e-08 &  9.30e-10 &      \\ 
     &           &    2 &  6.64e-08 &  3.22e-16 &      \\ 
     &           &    3 &  1.14e-08 &  1.90e-15 &      \\ 
     &           &    4 &  1.14e-08 &  3.25e-16 &      \\ 
     &           &    5 &  1.13e-08 &  3.24e-16 &      \\ 
     &           &    6 &  1.13e-08 &  3.24e-16 &      \\ 
     &           &    7 &  1.13e-08 &  3.23e-16 &      \\ 
     &           &    8 &  1.13e-08 &  3.23e-16 &      \\ 
     &           &    9 &  6.64e-08 &  3.22e-16 &      \\ 
     &           &   10 &  1.14e-08 &  1.90e-15 &      \\ 
1751 &  1.75e+04 &   10 &           &           & iters  \\ 
 \hdashline 
     &           &    1 &  2.40e-08 &  7.48e-11 &      \\ 
     &           &    2 &  5.37e-08 &  6.86e-16 &      \\ 
     &           &    3 &  2.41e-08 &  1.53e-15 &      \\ 
     &           &    4 &  2.40e-08 &  6.87e-16 &      \\ 
     &           &    5 &  5.37e-08 &  6.86e-16 &      \\ 
     &           &    6 &  2.41e-08 &  1.53e-15 &      \\ 
     &           &    7 &  2.40e-08 &  6.87e-16 &      \\ 
     &           &    8 &  5.37e-08 &  6.86e-16 &      \\ 
     &           &    9 &  2.41e-08 &  1.53e-15 &      \\ 
     &           &   10 &  2.41e-08 &  6.87e-16 &      \\ 
1752 &  1.75e+04 &   10 &           &           & iters  \\ 
 \hdashline 
     &           &    1 &  1.26e-08 &  2.96e-09 &      \\ 
     &           &    2 &  1.26e-08 &  3.60e-16 &      \\ 
     &           &    3 &  1.26e-08 &  3.59e-16 &      \\ 
     &           &    4 &  6.51e-08 &  3.58e-16 &      \\ 
     &           &    5 &  1.26e-08 &  1.86e-15 &      \\ 
     &           &    6 &  1.26e-08 &  3.61e-16 &      \\ 
     &           &    7 &  1.26e-08 &  3.60e-16 &      \\ 
     &           &    8 &  1.26e-08 &  3.60e-16 &      \\ 
     &           &    9 &  1.26e-08 &  3.59e-16 &      \\ 
     &           &   10 &  6.51e-08 &  3.59e-16 &      \\ 
1753 &  1.75e+04 &   10 &           &           & iters  \\ 
 \hdashline 
     &           &    1 &  1.39e-08 &  3.18e-09 &      \\ 
     &           &    2 &  1.39e-08 &  3.98e-16 &      \\ 
     &           &    3 &  1.39e-08 &  3.97e-16 &      \\ 
     &           &    4 &  2.50e-08 &  3.97e-16 &      \\ 
     &           &    5 &  1.39e-08 &  7.12e-16 &      \\ 
     &           &    6 &  1.39e-08 &  3.97e-16 &      \\ 
     &           &    7 &  2.50e-08 &  3.97e-16 &      \\ 
     &           &    8 &  1.39e-08 &  7.13e-16 &      \\ 
     &           &    9 &  1.39e-08 &  3.97e-16 &      \\ 
     &           &   10 &  2.50e-08 &  3.96e-16 &      \\ 
1754 &  1.75e+04 &   10 &           &           & iters  \\ 
 \hdashline 
     &           &    1 &  6.49e-09 &  1.13e-09 &      \\ 
     &           &    2 &  6.48e-09 &  1.85e-16 &      \\ 
     &           &    3 &  6.47e-09 &  1.85e-16 &      \\ 
     &           &    4 &  7.12e-08 &  1.85e-16 &      \\ 
     &           &    5 &  4.54e-08 &  2.03e-15 &      \\ 
     &           &    6 &  6.50e-09 &  1.30e-15 &      \\ 
     &           &    7 &  6.49e-09 &  1.85e-16 &      \\ 
     &           &    8 &  6.48e-09 &  1.85e-16 &      \\ 
     &           &    9 &  6.47e-09 &  1.85e-16 &      \\ 
     &           &   10 &  7.12e-08 &  1.85e-16 &      \\ 
1755 &  1.75e+04 &   10 &           &           & iters  \\ 
 \hdashline 
     &           &    1 &  4.12e-08 &  2.33e-09 &      \\ 
     &           &    2 &  4.12e-08 &  1.18e-15 &      \\ 
     &           &    3 &  3.66e-08 &  1.18e-15 &      \\ 
     &           &    4 &  3.65e-08 &  1.04e-15 &      \\ 
     &           &    5 &  4.12e-08 &  1.04e-15 &      \\ 
     &           &    6 &  3.65e-08 &  1.18e-15 &      \\ 
     &           &    7 &  4.12e-08 &  1.04e-15 &      \\ 
     &           &    8 &  3.65e-08 &  1.18e-15 &      \\ 
     &           &    9 &  4.12e-08 &  1.04e-15 &      \\ 
     &           &   10 &  3.65e-08 &  1.18e-15 &      \\ 
1756 &  1.76e+04 &   10 &           &           & iters  \\ 
 \hdashline 
     &           &    1 &  4.47e-08 &  1.56e-09 &      \\ 
     &           &    2 &  3.31e-08 &  1.28e-15 &      \\ 
     &           &    3 &  4.47e-08 &  9.44e-16 &      \\ 
     &           &    4 &  3.31e-08 &  1.27e-15 &      \\ 
     &           &    5 &  3.30e-08 &  9.45e-16 &      \\ 
     &           &    6 &  4.47e-08 &  9.43e-16 &      \\ 
     &           &    7 &  3.31e-08 &  1.28e-15 &      \\ 
     &           &    8 &  4.47e-08 &  9.44e-16 &      \\ 
     &           &    9 &  3.31e-08 &  1.27e-15 &      \\ 
     &           &   10 &  4.47e-08 &  9.44e-16 &      \\ 
1757 &  1.76e+04 &   10 &           &           & iters  \\ 
 \hdashline 
     &           &    1 &  2.52e-08 &  2.18e-09 &      \\ 
     &           &    2 &  2.52e-08 &  7.19e-16 &      \\ 
     &           &    3 &  1.37e-08 &  7.18e-16 &      \\ 
     &           &    4 &  2.51e-08 &  3.92e-16 &      \\ 
     &           &    5 &  1.37e-08 &  7.17e-16 &      \\ 
     &           &    6 &  1.37e-08 &  3.92e-16 &      \\ 
     &           &    7 &  2.51e-08 &  3.92e-16 &      \\ 
     &           &    8 &  1.37e-08 &  7.17e-16 &      \\ 
     &           &    9 &  1.37e-08 &  3.92e-16 &      \\ 
     &           &   10 &  2.51e-08 &  3.92e-16 &      \\ 
1758 &  1.76e+04 &   10 &           &           & iters  \\ 
 \hdashline 
     &           &    1 &  2.50e-08 &  1.08e-09 &      \\ 
     &           &    2 &  2.50e-08 &  7.14e-16 &      \\ 
     &           &    3 &  5.27e-08 &  7.13e-16 &      \\ 
     &           &    4 &  2.50e-08 &  1.51e-15 &      \\ 
     &           &    5 &  2.50e-08 &  7.14e-16 &      \\ 
     &           &    6 &  5.27e-08 &  7.13e-16 &      \\ 
     &           &    7 &  2.50e-08 &  1.51e-15 &      \\ 
     &           &    8 &  2.50e-08 &  7.14e-16 &      \\ 
     &           &    9 &  5.27e-08 &  7.13e-16 &      \\ 
     &           &   10 &  2.50e-08 &  1.51e-15 &      \\ 
1759 &  1.76e+04 &   10 &           &           & iters  \\ 
 \hdashline 
     &           &    1 &  2.61e-08 &  2.25e-09 &      \\ 
     &           &    2 &  5.16e-08 &  7.46e-16 &      \\ 
     &           &    3 &  2.62e-08 &  1.47e-15 &      \\ 
     &           &    4 &  2.61e-08 &  7.47e-16 &      \\ 
     &           &    5 &  5.16e-08 &  7.46e-16 &      \\ 
     &           &    6 &  2.62e-08 &  1.47e-15 &      \\ 
     &           &    7 &  2.61e-08 &  7.47e-16 &      \\ 
     &           &    8 &  5.16e-08 &  7.46e-16 &      \\ 
     &           &    9 &  2.62e-08 &  1.47e-15 &      \\ 
     &           &   10 &  2.61e-08 &  7.47e-16 &      \\ 
1760 &  1.76e+04 &   10 &           &           & iters  \\ 
 \hdashline 
     &           &    1 &  3.37e-08 &  1.05e-10 &      \\ 
     &           &    2 &  3.37e-08 &  9.63e-16 &      \\ 
     &           &    3 &  4.40e-08 &  9.62e-16 &      \\ 
     &           &    4 &  3.37e-08 &  1.26e-15 &      \\ 
     &           &    5 &  4.40e-08 &  9.62e-16 &      \\ 
     &           &    6 &  3.37e-08 &  1.26e-15 &      \\ 
     &           &    7 &  4.40e-08 &  9.63e-16 &      \\ 
     &           &    8 &  3.37e-08 &  1.26e-15 &      \\ 
     &           &    9 &  3.37e-08 &  9.63e-16 &      \\ 
     &           &   10 &  4.40e-08 &  9.62e-16 &      \\ 
1761 &  1.76e+04 &   10 &           &           & iters  \\ 
 \hdashline 
     &           &    1 &  3.85e-09 &  2.08e-09 &      \\ 
     &           &    2 &  3.84e-09 &  1.10e-16 &      \\ 
     &           &    3 &  3.84e-09 &  1.10e-16 &      \\ 
     &           &    4 &  3.83e-09 &  1.10e-16 &      \\ 
     &           &    5 &  3.83e-09 &  1.09e-16 &      \\ 
     &           &    6 &  3.82e-09 &  1.09e-16 &      \\ 
     &           &    7 &  7.39e-08 &  1.09e-16 &      \\ 
     &           &    8 &  8.16e-08 &  2.11e-15 &      \\ 
     &           &    9 &  7.39e-08 &  2.33e-15 &      \\ 
     &           &   10 &  8.16e-08 &  2.11e-15 &      \\ 
1762 &  1.76e+04 &   10 &           &           & iters  \\ 
 \hdashline 
     &           &    1 &  4.12e-08 &  1.48e-09 &      \\ 
     &           &    2 &  3.66e-08 &  1.18e-15 &      \\ 
     &           &    3 &  4.12e-08 &  1.04e-15 &      \\ 
     &           &    4 &  3.66e-08 &  1.18e-15 &      \\ 
     &           &    5 &  4.12e-08 &  1.04e-15 &      \\ 
     &           &    6 &  3.66e-08 &  1.18e-15 &      \\ 
     &           &    7 &  4.12e-08 &  1.04e-15 &      \\ 
     &           &    8 &  3.66e-08 &  1.17e-15 &      \\ 
     &           &    9 &  4.12e-08 &  1.04e-15 &      \\ 
     &           &   10 &  3.66e-08 &  1.17e-15 &      \\ 
1763 &  1.76e+04 &   10 &           &           & iters  \\ 
 \hdashline 
     &           &    1 &  2.85e-08 &  5.20e-10 &      \\ 
     &           &    2 &  1.04e-08 &  8.13e-16 &      \\ 
     &           &    3 &  1.04e-08 &  2.97e-16 &      \\ 
     &           &    4 &  2.85e-08 &  2.96e-16 &      \\ 
     &           &    5 &  1.04e-08 &  8.13e-16 &      \\ 
     &           &    6 &  1.04e-08 &  2.97e-16 &      \\ 
     &           &    7 &  2.85e-08 &  2.97e-16 &      \\ 
     &           &    8 &  1.04e-08 &  8.12e-16 &      \\ 
     &           &    9 &  1.04e-08 &  2.98e-16 &      \\ 
     &           &   10 &  1.04e-08 &  2.97e-16 &      \\ 
1764 &  1.76e+04 &   10 &           &           & iters  \\ 
 \hdashline 
     &           &    1 &  1.79e-08 &  1.95e-09 &      \\ 
     &           &    2 &  1.79e-08 &  5.12e-16 &      \\ 
     &           &    3 &  5.98e-08 &  5.11e-16 &      \\ 
     &           &    4 &  1.80e-08 &  1.71e-15 &      \\ 
     &           &    5 &  1.79e-08 &  5.13e-16 &      \\ 
     &           &    6 &  1.79e-08 &  5.12e-16 &      \\ 
     &           &    7 &  5.98e-08 &  5.11e-16 &      \\ 
     &           &    8 &  1.80e-08 &  1.71e-15 &      \\ 
     &           &    9 &  1.80e-08 &  5.13e-16 &      \\ 
     &           &   10 &  1.79e-08 &  5.12e-16 &      \\ 
1765 &  1.76e+04 &   10 &           &           & iters  \\ 
 \hdashline 
     &           &    1 &  3.83e-08 &  2.32e-09 &      \\ 
     &           &    2 &  3.94e-08 &  1.09e-15 &      \\ 
     &           &    3 &  3.83e-08 &  1.12e-15 &      \\ 
     &           &    4 &  3.94e-08 &  1.09e-15 &      \\ 
     &           &    5 &  3.83e-08 &  1.12e-15 &      \\ 
     &           &    6 &  3.94e-08 &  1.09e-15 &      \\ 
     &           &    7 &  3.83e-08 &  1.12e-15 &      \\ 
     &           &    8 &  3.94e-08 &  1.09e-15 &      \\ 
     &           &    9 &  3.83e-08 &  1.12e-15 &      \\ 
     &           &   10 &  3.94e-08 &  1.09e-15 &      \\ 
1766 &  1.76e+04 &   10 &           &           & iters  \\ 
 \hdashline 
     &           &    1 &  4.41e-10 &  6.41e-10 &      \\ 
     &           &    2 &  4.40e-10 &  1.26e-17 &      \\ 
     &           &    3 &  4.40e-10 &  1.26e-17 &      \\ 
     &           &    4 &  4.39e-10 &  1.26e-17 &      \\ 
     &           &    5 &  4.39e-10 &  1.25e-17 &      \\ 
     &           &    6 &  4.38e-10 &  1.25e-17 &      \\ 
     &           &    7 &  4.37e-10 &  1.25e-17 &      \\ 
     &           &    8 &  4.37e-10 &  1.25e-17 &      \\ 
     &           &    9 &  4.36e-10 &  1.25e-17 &      \\ 
     &           &   10 &  4.35e-10 &  1.24e-17 &      \\ 
1767 &  1.77e+04 &   10 &           &           & iters  \\ 
 \hdashline 
     &           &    1 &  1.20e-08 &  2.33e-09 &      \\ 
     &           &    2 &  1.19e-08 &  3.41e-16 &      \\ 
     &           &    3 &  1.19e-08 &  3.41e-16 &      \\ 
     &           &    4 &  2.69e-08 &  3.40e-16 &      \\ 
     &           &    5 &  1.19e-08 &  7.69e-16 &      \\ 
     &           &    6 &  1.19e-08 &  3.41e-16 &      \\ 
     &           &    7 &  1.19e-08 &  3.40e-16 &      \\ 
     &           &    8 &  2.70e-08 &  3.40e-16 &      \\ 
     &           &    9 &  1.19e-08 &  7.69e-16 &      \\ 
     &           &   10 &  1.19e-08 &  3.40e-16 &      \\ 
1768 &  1.77e+04 &   10 &           &           & iters  \\ 
 \hdashline 
     &           &    1 &  9.64e-09 &  2.45e-09 &      \\ 
     &           &    2 &  9.62e-09 &  2.75e-16 &      \\ 
     &           &    3 &  9.61e-09 &  2.75e-16 &      \\ 
     &           &    4 &  9.60e-09 &  2.74e-16 &      \\ 
     &           &    5 &  2.93e-08 &  2.74e-16 &      \\ 
     &           &    6 &  9.62e-09 &  8.35e-16 &      \\ 
     &           &    7 &  9.61e-09 &  2.75e-16 &      \\ 
     &           &    8 &  9.60e-09 &  2.74e-16 &      \\ 
     &           &    9 &  2.93e-08 &  2.74e-16 &      \\ 
     &           &   10 &  9.63e-09 &  8.35e-16 &      \\ 
1769 &  1.77e+04 &   10 &           &           & iters  \\ 
 \hdashline 
     &           &    1 &  6.81e-09 &  9.18e-10 &      \\ 
     &           &    2 &  6.80e-09 &  1.94e-16 &      \\ 
     &           &    3 &  6.79e-09 &  1.94e-16 &      \\ 
     &           &    4 &  3.21e-08 &  1.94e-16 &      \\ 
     &           &    5 &  6.83e-09 &  9.15e-16 &      \\ 
     &           &    6 &  6.82e-09 &  1.95e-16 &      \\ 
     &           &    7 &  6.81e-09 &  1.95e-16 &      \\ 
     &           &    8 &  6.80e-09 &  1.94e-16 &      \\ 
     &           &    9 &  6.79e-09 &  1.94e-16 &      \\ 
     &           &   10 &  3.21e-08 &  1.94e-16 &      \\ 
1770 &  1.77e+04 &   10 &           &           & iters  \\ 
 \hdashline 
     &           &    1 &  8.91e-09 &  2.77e-09 &      \\ 
     &           &    2 &  8.90e-09 &  2.54e-16 &      \\ 
     &           &    3 &  8.88e-09 &  2.54e-16 &      \\ 
     &           &    4 &  8.87e-09 &  2.54e-16 &      \\ 
     &           &    5 &  8.86e-09 &  2.53e-16 &      \\ 
     &           &    6 &  6.88e-08 &  2.53e-16 &      \\ 
     &           &    7 &  4.78e-08 &  1.96e-15 &      \\ 
     &           &    8 &  8.88e-09 &  1.36e-15 &      \\ 
     &           &    9 &  8.86e-09 &  2.53e-16 &      \\ 
     &           &   10 &  6.88e-08 &  2.53e-16 &      \\ 
1771 &  1.77e+04 &   10 &           &           & iters  \\ 
 \hdashline 
     &           &    1 &  4.80e-08 &  2.40e-09 &      \\ 
     &           &    2 &  2.98e-08 &  1.37e-15 &      \\ 
     &           &    3 &  4.80e-08 &  8.49e-16 &      \\ 
     &           &    4 &  2.98e-08 &  1.37e-15 &      \\ 
     &           &    5 &  2.97e-08 &  8.50e-16 &      \\ 
     &           &    6 &  4.80e-08 &  8.49e-16 &      \\ 
     &           &    7 &  2.98e-08 &  1.37e-15 &      \\ 
     &           &    8 &  4.80e-08 &  8.49e-16 &      \\ 
     &           &    9 &  2.98e-08 &  1.37e-15 &      \\ 
     &           &   10 &  2.97e-08 &  8.50e-16 &      \\ 
1772 &  1.77e+04 &   10 &           &           & iters  \\ 
 \hdashline 
     &           &    1 &  2.76e-08 &  1.26e-09 &      \\ 
     &           &    2 &  2.76e-08 &  7.88e-16 &      \\ 
     &           &    3 &  2.75e-08 &  7.87e-16 &      \\ 
     &           &    4 &  5.02e-08 &  7.86e-16 &      \\ 
     &           &    5 &  2.76e-08 &  1.43e-15 &      \\ 
     &           &    6 &  5.02e-08 &  7.87e-16 &      \\ 
     &           &    7 &  2.76e-08 &  1.43e-15 &      \\ 
     &           &    8 &  2.76e-08 &  7.88e-16 &      \\ 
     &           &    9 &  5.02e-08 &  7.87e-16 &      \\ 
     &           &   10 &  2.76e-08 &  1.43e-15 &      \\ 
1773 &  1.77e+04 &   10 &           &           & iters  \\ 
 \hdashline 
     &           &    1 &  3.14e-08 &  2.14e-09 &      \\ 
     &           &    2 &  3.14e-08 &  8.97e-16 &      \\ 
     &           &    3 &  4.64e-08 &  8.96e-16 &      \\ 
     &           &    4 &  3.14e-08 &  1.32e-15 &      \\ 
     &           &    5 &  3.14e-08 &  8.96e-16 &      \\ 
     &           &    6 &  4.64e-08 &  8.95e-16 &      \\ 
     &           &    7 &  3.14e-08 &  1.32e-15 &      \\ 
     &           &    8 &  4.64e-08 &  8.96e-16 &      \\ 
     &           &    9 &  3.14e-08 &  1.32e-15 &      \\ 
     &           &   10 &  3.14e-08 &  8.96e-16 &      \\ 
1774 &  1.77e+04 &   10 &           &           & iters  \\ 
 \hdashline 
     &           &    1 &  3.39e-08 &  2.42e-09 &      \\ 
     &           &    2 &  3.39e-08 &  9.68e-16 &      \\ 
     &           &    3 &  5.04e-09 &  9.66e-16 &      \\ 
     &           &    4 &  5.03e-09 &  1.44e-16 &      \\ 
     &           &    5 &  5.02e-09 &  1.43e-16 &      \\ 
     &           &    6 &  5.01e-09 &  1.43e-16 &      \\ 
     &           &    7 &  3.38e-08 &  1.43e-16 &      \\ 
     &           &    8 &  5.05e-09 &  9.66e-16 &      \\ 
     &           &    9 &  5.05e-09 &  1.44e-16 &      \\ 
     &           &   10 &  5.04e-09 &  1.44e-16 &      \\ 
1775 &  1.77e+04 &   10 &           &           & iters  \\ 
 \hdashline 
     &           &    1 &  2.43e-08 &  1.19e-09 &      \\ 
     &           &    2 &  2.43e-08 &  6.94e-16 &      \\ 
     &           &    3 &  5.34e-08 &  6.93e-16 &      \\ 
     &           &    4 &  2.43e-08 &  1.53e-15 &      \\ 
     &           &    5 &  2.43e-08 &  6.94e-16 &      \\ 
     &           &    6 &  5.34e-08 &  6.93e-16 &      \\ 
     &           &    7 &  2.43e-08 &  1.53e-15 &      \\ 
     &           &    8 &  2.43e-08 &  6.94e-16 &      \\ 
     &           &    9 &  5.34e-08 &  6.93e-16 &      \\ 
     &           &   10 &  2.43e-08 &  1.52e-15 &      \\ 
1776 &  1.78e+04 &   10 &           &           & iters  \\ 
 \hdashline 
     &           &    1 &  3.61e-08 &  5.16e-10 &      \\ 
     &           &    2 &  4.17e-08 &  1.03e-15 &      \\ 
     &           &    3 &  3.61e-08 &  1.19e-15 &      \\ 
     &           &    4 &  4.17e-08 &  1.03e-15 &      \\ 
     &           &    5 &  3.61e-08 &  1.19e-15 &      \\ 
     &           &    6 &  4.17e-08 &  1.03e-15 &      \\ 
     &           &    7 &  3.61e-08 &  1.19e-15 &      \\ 
     &           &    8 &  4.16e-08 &  1.03e-15 &      \\ 
     &           &    9 &  3.61e-08 &  1.19e-15 &      \\ 
     &           &   10 &  4.16e-08 &  1.03e-15 &      \\ 
1777 &  1.78e+04 &   10 &           &           & iters  \\ 
 \hdashline 
     &           &    1 &  3.15e-08 &  1.82e-09 &      \\ 
     &           &    2 &  3.14e-08 &  8.99e-16 &      \\ 
     &           &    3 &  4.63e-08 &  8.98e-16 &      \\ 
     &           &    4 &  3.15e-08 &  1.32e-15 &      \\ 
     &           &    5 &  4.63e-08 &  8.98e-16 &      \\ 
     &           &    6 &  3.15e-08 &  1.32e-15 &      \\ 
     &           &    7 &  3.14e-08 &  8.99e-16 &      \\ 
     &           &    8 &  4.63e-08 &  8.98e-16 &      \\ 
     &           &    9 &  3.15e-08 &  1.32e-15 &      \\ 
     &           &   10 &  4.63e-08 &  8.98e-16 &      \\ 
1778 &  1.78e+04 &   10 &           &           & iters  \\ 
 \hdashline 
     &           &    1 &  4.97e-09 &  1.57e-09 &      \\ 
     &           &    2 &  4.96e-09 &  1.42e-16 &      \\ 
     &           &    3 &  4.96e-09 &  1.42e-16 &      \\ 
     &           &    4 &  4.95e-09 &  1.41e-16 &      \\ 
     &           &    5 &  4.94e-09 &  1.41e-16 &      \\ 
     &           &    6 &  4.94e-09 &  1.41e-16 &      \\ 
     &           &    7 &  3.39e-08 &  1.41e-16 &      \\ 
     &           &    8 &  4.98e-09 &  9.68e-16 &      \\ 
     &           &    9 &  4.97e-09 &  1.42e-16 &      \\ 
     &           &   10 &  4.96e-09 &  1.42e-16 &      \\ 
1779 &  1.78e+04 &   10 &           &           & iters  \\ 
 \hdashline 
     &           &    1 &  3.56e-09 &  1.39e-09 &      \\ 
     &           &    2 &  3.55e-09 &  1.02e-16 &      \\ 
     &           &    3 &  3.55e-09 &  1.01e-16 &      \\ 
     &           &    4 &  3.54e-09 &  1.01e-16 &      \\ 
     &           &    5 &  3.54e-09 &  1.01e-16 &      \\ 
     &           &    6 &  3.53e-09 &  1.01e-16 &      \\ 
     &           &    7 &  3.53e-08 &  1.01e-16 &      \\ 
     &           &    8 &  3.58e-09 &  1.01e-15 &      \\ 
     &           &    9 &  3.57e-09 &  1.02e-16 &      \\ 
     &           &   10 &  3.57e-09 &  1.02e-16 &      \\ 
1780 &  1.78e+04 &   10 &           &           & iters  \\ 
 \hdashline 
     &           &    1 &  6.59e-09 &  2.11e-09 &      \\ 
     &           &    2 &  6.58e-09 &  1.88e-16 &      \\ 
     &           &    3 &  6.57e-09 &  1.88e-16 &      \\ 
     &           &    4 &  6.56e-09 &  1.88e-16 &      \\ 
     &           &    5 &  7.11e-08 &  1.87e-16 &      \\ 
     &           &    6 &  4.55e-08 &  2.03e-15 &      \\ 
     &           &    7 &  6.59e-09 &  1.30e-15 &      \\ 
     &           &    8 &  6.58e-09 &  1.88e-16 &      \\ 
     &           &    9 &  6.57e-09 &  1.88e-16 &      \\ 
     &           &   10 &  6.56e-09 &  1.88e-16 &      \\ 
1781 &  1.78e+04 &   10 &           &           & iters  \\ 
 \hdashline 
     &           &    1 &  6.92e-08 &  1.28e-09 &      \\ 
     &           &    2 &  8.64e-09 &  1.97e-15 &      \\ 
     &           &    3 &  8.63e-09 &  2.47e-16 &      \\ 
     &           &    4 &  8.61e-09 &  2.46e-16 &      \\ 
     &           &    5 &  8.60e-09 &  2.46e-16 &      \\ 
     &           &    6 &  8.59e-09 &  2.45e-16 &      \\ 
     &           &    7 &  8.58e-09 &  2.45e-16 &      \\ 
     &           &    8 &  8.56e-09 &  2.45e-16 &      \\ 
     &           &    9 &  6.91e-08 &  2.44e-16 &      \\ 
     &           &   10 &  8.65e-09 &  1.97e-15 &      \\ 
1782 &  1.78e+04 &   10 &           &           & iters  \\ 
 \hdashline 
     &           &    1 &  4.40e-08 &  2.73e-09 &      \\ 
     &           &    2 &  3.38e-08 &  1.26e-15 &      \\ 
     &           &    3 &  4.40e-08 &  9.63e-16 &      \\ 
     &           &    4 &  3.38e-08 &  1.26e-15 &      \\ 
     &           &    5 &  4.40e-08 &  9.64e-16 &      \\ 
     &           &    6 &  3.38e-08 &  1.25e-15 &      \\ 
     &           &    7 &  3.37e-08 &  9.64e-16 &      \\ 
     &           &    8 &  4.40e-08 &  9.63e-16 &      \\ 
     &           &    9 &  3.38e-08 &  1.26e-15 &      \\ 
     &           &   10 &  4.40e-08 &  9.63e-16 &      \\ 
1783 &  1.78e+04 &   10 &           &           & iters  \\ 
 \hdashline 
     &           &    1 &  1.15e-08 &  5.49e-09 &      \\ 
     &           &    2 &  1.14e-08 &  3.27e-16 &      \\ 
     &           &    3 &  1.14e-08 &  3.27e-16 &      \\ 
     &           &    4 &  1.14e-08 &  3.26e-16 &      \\ 
     &           &    5 &  1.14e-08 &  3.26e-16 &      \\ 
     &           &    6 &  1.14e-08 &  3.25e-16 &      \\ 
     &           &    7 &  1.14e-08 &  3.25e-16 &      \\ 
     &           &    8 &  6.63e-08 &  3.24e-16 &      \\ 
     &           &    9 &  1.14e-08 &  1.89e-15 &      \\ 
     &           &   10 &  1.14e-08 &  3.26e-16 &      \\ 
1784 &  1.78e+04 &   10 &           &           & iters  \\ 
 \hdashline 
     &           &    1 &  6.16e-08 &  1.15e-09 &      \\ 
     &           &    2 &  1.62e-08 &  1.76e-15 &      \\ 
     &           &    3 &  1.62e-08 &  4.62e-16 &      \\ 
     &           &    4 &  1.61e-08 &  4.61e-16 &      \\ 
     &           &    5 &  6.16e-08 &  4.61e-16 &      \\ 
     &           &    6 &  1.62e-08 &  1.76e-15 &      \\ 
     &           &    7 &  1.62e-08 &  4.63e-16 &      \\ 
     &           &    8 &  1.62e-08 &  4.62e-16 &      \\ 
     &           &    9 &  1.61e-08 &  4.61e-16 &      \\ 
     &           &   10 &  6.16e-08 &  4.61e-16 &      \\ 
1785 &  1.78e+04 &   10 &           &           & iters  \\ 
 \hdashline 
     &           &    1 &  2.43e-08 &  3.37e-09 &      \\ 
     &           &    2 &  2.42e-08 &  6.93e-16 &      \\ 
     &           &    3 &  2.42e-08 &  6.92e-16 &      \\ 
     &           &    4 &  1.47e-08 &  6.91e-16 &      \\ 
     &           &    5 &  1.47e-08 &  4.19e-16 &      \\ 
     &           &    6 &  2.42e-08 &  4.18e-16 &      \\ 
     &           &    7 &  1.47e-08 &  6.91e-16 &      \\ 
     &           &    8 &  2.42e-08 &  4.19e-16 &      \\ 
     &           &    9 &  1.47e-08 &  6.91e-16 &      \\ 
     &           &   10 &  1.47e-08 &  4.19e-16 &      \\ 
1786 &  1.78e+04 &   10 &           &           & iters  \\ 
 \hdashline 
     &           &    1 &  2.15e-08 &  3.74e-10 &      \\ 
     &           &    2 &  5.62e-08 &  6.15e-16 &      \\ 
     &           &    3 &  2.16e-08 &  1.60e-15 &      \\ 
     &           &    4 &  2.16e-08 &  6.16e-16 &      \\ 
     &           &    5 &  2.15e-08 &  6.15e-16 &      \\ 
     &           &    6 &  5.62e-08 &  6.14e-16 &      \\ 
     &           &    7 &  2.16e-08 &  1.60e-15 &      \\ 
     &           &    8 &  2.15e-08 &  6.16e-16 &      \\ 
     &           &    9 &  5.62e-08 &  6.15e-16 &      \\ 
     &           &   10 &  2.16e-08 &  1.60e-15 &      \\ 
1787 &  1.79e+04 &   10 &           &           & iters  \\ 
 \hdashline 
     &           &    1 &  8.25e-09 &  3.67e-09 &      \\ 
     &           &    2 &  8.24e-09 &  2.36e-16 &      \\ 
     &           &    3 &  6.95e-08 &  2.35e-16 &      \\ 
     &           &    4 &  8.60e-08 &  1.98e-15 &      \\ 
     &           &    5 &  6.95e-08 &  2.45e-15 &      \\ 
     &           &    6 &  8.30e-09 &  1.98e-15 &      \\ 
     &           &    7 &  8.29e-09 &  2.37e-16 &      \\ 
     &           &    8 &  8.28e-09 &  2.37e-16 &      \\ 
     &           &    9 &  8.27e-09 &  2.36e-16 &      \\ 
     &           &   10 &  8.26e-09 &  2.36e-16 &      \\ 
1788 &  1.79e+04 &   10 &           &           & iters  \\ 
 \hdashline 
     &           &    1 &  1.92e-08 &  4.08e-10 &      \\ 
     &           &    2 &  5.85e-08 &  5.48e-16 &      \\ 
     &           &    3 &  1.93e-08 &  1.67e-15 &      \\ 
     &           &    4 &  1.92e-08 &  5.50e-16 &      \\ 
     &           &    5 &  1.92e-08 &  5.49e-16 &      \\ 
     &           &    6 &  5.85e-08 &  5.48e-16 &      \\ 
     &           &    7 &  1.93e-08 &  1.67e-15 &      \\ 
     &           &    8 &  1.92e-08 &  5.50e-16 &      \\ 
     &           &    9 &  1.92e-08 &  5.49e-16 &      \\ 
     &           &   10 &  5.85e-08 &  5.48e-16 &      \\ 
1789 &  1.79e+04 &   10 &           &           & iters  \\ 
 \hdashline 
     &           &    1 &  4.28e-08 &  2.54e-09 &      \\ 
     &           &    2 &  4.28e-08 &  1.22e-15 &      \\ 
     &           &    3 &  3.50e-08 &  1.22e-15 &      \\ 
     &           &    4 &  4.28e-08 &  9.98e-16 &      \\ 
     &           &    5 &  3.50e-08 &  1.22e-15 &      \\ 
     &           &    6 &  4.27e-08 &  9.99e-16 &      \\ 
     &           &    7 &  3.50e-08 &  1.22e-15 &      \\ 
     &           &    8 &  3.50e-08 &  9.99e-16 &      \\ 
     &           &    9 &  4.28e-08 &  9.98e-16 &      \\ 
     &           &   10 &  3.50e-08 &  1.22e-15 &      \\ 
1790 &  1.79e+04 &   10 &           &           & iters  \\ 
 \hdashline 
     &           &    1 &  2.84e-08 &  2.86e-09 &      \\ 
     &           &    2 &  1.05e-08 &  8.10e-16 &      \\ 
     &           &    3 &  1.05e-08 &  3.00e-16 &      \\ 
     &           &    4 &  1.05e-08 &  3.00e-16 &      \\ 
     &           &    5 &  2.84e-08 &  2.99e-16 &      \\ 
     &           &    6 &  1.05e-08 &  8.10e-16 &      \\ 
     &           &    7 &  1.05e-08 &  3.00e-16 &      \\ 
     &           &    8 &  1.05e-08 &  3.00e-16 &      \\ 
     &           &    9 &  2.84e-08 &  2.99e-16 &      \\ 
     &           &   10 &  1.05e-08 &  8.10e-16 &      \\ 
1791 &  1.79e+04 &   10 &           &           & iters  \\ 
 \hdashline 
     &           &    1 &  5.25e-09 &  2.40e-09 &      \\ 
     &           &    2 &  5.24e-09 &  1.50e-16 &      \\ 
     &           &    3 &  5.24e-09 &  1.50e-16 &      \\ 
     &           &    4 &  5.23e-09 &  1.49e-16 &      \\ 
     &           &    5 &  5.22e-09 &  1.49e-16 &      \\ 
     &           &    6 &  3.36e-08 &  1.49e-16 &      \\ 
     &           &    7 &  5.26e-09 &  9.60e-16 &      \\ 
     &           &    8 &  5.26e-09 &  1.50e-16 &      \\ 
     &           &    9 &  5.25e-09 &  1.50e-16 &      \\ 
     &           &   10 &  5.24e-09 &  1.50e-16 &      \\ 
1792 &  1.79e+04 &   10 &           &           & iters  \\ 
 \hdashline 
     &           &    1 &  3.53e-08 &  7.14e-10 &      \\ 
     &           &    2 &  4.25e-08 &  1.01e-15 &      \\ 
     &           &    3 &  3.53e-08 &  1.21e-15 &      \\ 
     &           &    4 &  4.25e-08 &  1.01e-15 &      \\ 
     &           &    5 &  3.53e-08 &  1.21e-15 &      \\ 
     &           &    6 &  4.24e-08 &  1.01e-15 &      \\ 
     &           &    7 &  3.53e-08 &  1.21e-15 &      \\ 
     &           &    8 &  3.53e-08 &  1.01e-15 &      \\ 
     &           &    9 &  4.25e-08 &  1.01e-15 &      \\ 
     &           &   10 &  3.53e-08 &  1.21e-15 &      \\ 
1793 &  1.79e+04 &   10 &           &           & iters  \\ 
 \hdashline 
     &           &    1 &  5.85e-08 &  2.83e-09 &      \\ 
     &           &    2 &  1.92e-08 &  1.67e-15 &      \\ 
     &           &    3 &  1.92e-08 &  5.49e-16 &      \\ 
     &           &    4 &  5.85e-08 &  5.48e-16 &      \\ 
     &           &    5 &  1.93e-08 &  1.67e-15 &      \\ 
     &           &    6 &  1.92e-08 &  5.50e-16 &      \\ 
     &           &    7 &  1.92e-08 &  5.49e-16 &      \\ 
     &           &    8 &  5.85e-08 &  5.48e-16 &      \\ 
     &           &    9 &  1.93e-08 &  1.67e-15 &      \\ 
     &           &   10 &  1.92e-08 &  5.50e-16 &      \\ 
1794 &  1.79e+04 &   10 &           &           & iters  \\ 
 \hdashline 
     &           &    1 &  1.15e-08 &  1.06e-09 &      \\ 
     &           &    2 &  1.14e-08 &  3.27e-16 &      \\ 
     &           &    3 &  2.74e-08 &  3.27e-16 &      \\ 
     &           &    4 &  1.15e-08 &  7.83e-16 &      \\ 
     &           &    5 &  1.14e-08 &  3.27e-16 &      \\ 
     &           &    6 &  2.74e-08 &  3.27e-16 &      \\ 
     &           &    7 &  1.15e-08 &  7.82e-16 &      \\ 
     &           &    8 &  1.15e-08 &  3.27e-16 &      \\ 
     &           &    9 &  2.74e-08 &  3.27e-16 &      \\ 
     &           &   10 &  1.15e-08 &  7.82e-16 &      \\ 
1795 &  1.79e+04 &   10 &           &           & iters  \\ 
 \hdashline 
     &           &    1 &  2.42e-08 &  1.98e-10 &      \\ 
     &           &    2 &  5.35e-08 &  6.91e-16 &      \\ 
     &           &    3 &  2.43e-08 &  1.53e-15 &      \\ 
     &           &    4 &  2.42e-08 &  6.92e-16 &      \\ 
     &           &    5 &  5.35e-08 &  6.91e-16 &      \\ 
     &           &    6 &  2.43e-08 &  1.53e-15 &      \\ 
     &           &    7 &  2.42e-08 &  6.92e-16 &      \\ 
     &           &    8 &  5.35e-08 &  6.91e-16 &      \\ 
     &           &    9 &  2.43e-08 &  1.53e-15 &      \\ 
     &           &   10 &  2.42e-08 &  6.93e-16 &      \\ 
1796 &  1.80e+04 &   10 &           &           & iters  \\ 
 \hdashline 
     &           &    1 &  2.87e-08 &  3.11e-10 &      \\ 
     &           &    2 &  2.86e-08 &  8.18e-16 &      \\ 
     &           &    3 &  4.91e-08 &  8.17e-16 &      \\ 
     &           &    4 &  2.87e-08 &  1.40e-15 &      \\ 
     &           &    5 &  4.91e-08 &  8.18e-16 &      \\ 
     &           &    6 &  2.87e-08 &  1.40e-15 &      \\ 
     &           &    7 &  2.87e-08 &  8.19e-16 &      \\ 
     &           &    8 &  4.91e-08 &  8.18e-16 &      \\ 
     &           &    9 &  2.87e-08 &  1.40e-15 &      \\ 
     &           &   10 &  2.86e-08 &  8.19e-16 &      \\ 
1797 &  1.80e+04 &   10 &           &           & iters  \\ 
 \hdashline 
     &           &    1 &  5.86e-08 &  6.98e-10 &      \\ 
     &           &    2 &  1.91e-08 &  1.67e-15 &      \\ 
     &           &    3 &  1.91e-08 &  5.46e-16 &      \\ 
     &           &    4 &  1.91e-08 &  5.46e-16 &      \\ 
     &           &    5 &  5.86e-08 &  5.45e-16 &      \\ 
     &           &    6 &  1.91e-08 &  1.67e-15 &      \\ 
     &           &    7 &  1.91e-08 &  5.46e-16 &      \\ 
     &           &    8 &  1.91e-08 &  5.46e-16 &      \\ 
     &           &    9 &  5.86e-08 &  5.45e-16 &      \\ 
     &           &   10 &  1.91e-08 &  1.67e-15 &      \\ 
1798 &  1.80e+04 &   10 &           &           & iters  \\ 
 \hdashline 
     &           &    1 &  2.93e-08 &  3.83e-09 &      \\ 
     &           &    2 &  4.84e-08 &  8.37e-16 &      \\ 
     &           &    3 &  2.93e-08 &  1.38e-15 &      \\ 
     &           &    4 &  4.84e-08 &  8.38e-16 &      \\ 
     &           &    5 &  2.94e-08 &  1.38e-15 &      \\ 
     &           &    6 &  2.93e-08 &  8.38e-16 &      \\ 
     &           &    7 &  4.84e-08 &  8.37e-16 &      \\ 
     &           &    8 &  2.94e-08 &  1.38e-15 &      \\ 
     &           &    9 &  2.93e-08 &  8.38e-16 &      \\ 
     &           &   10 &  4.84e-08 &  8.37e-16 &      \\ 
1799 &  1.80e+04 &   10 &           &           & iters  \\ 
 \hdashline 
     &           &    1 &  1.27e-08 &  3.34e-09 &      \\ 
     &           &    2 &  1.27e-08 &  3.62e-16 &      \\ 
     &           &    3 &  1.27e-08 &  3.62e-16 &      \\ 
     &           &    4 &  1.26e-08 &  3.61e-16 &      \\ 
     &           &    5 &  6.51e-08 &  3.61e-16 &      \\ 
     &           &    6 &  1.27e-08 &  1.86e-15 &      \\ 
     &           &    7 &  1.27e-08 &  3.63e-16 &      \\ 
     &           &    8 &  1.27e-08 &  3.62e-16 &      \\ 
     &           &    9 &  1.27e-08 &  3.62e-16 &      \\ 
     &           &   10 &  1.26e-08 &  3.61e-16 &      \\ 
1800 &  1.80e+04 &   10 &           &           & iters  \\ 
 \hdashline 
     &           &    1 &  4.87e-10 &  8.95e-10 &      \\ 
     &           &    2 &  4.87e-10 &  1.39e-17 &      \\ 
     &           &    3 &  4.86e-10 &  1.39e-17 &      \\ 
     &           &    4 &  4.85e-10 &  1.39e-17 &      \\ 
     &           &    5 &  4.85e-10 &  1.38e-17 &      \\ 
     &           &    6 &  4.84e-10 &  1.38e-17 &      \\ 
     &           &    7 &  4.83e-10 &  1.38e-17 &      \\ 
     &           &    8 &  4.82e-10 &  1.38e-17 &      \\ 
     &           &    9 &  4.82e-10 &  1.38e-17 &      \\ 
     &           &   10 &  4.81e-10 &  1.38e-17 &      \\ 
1801 &  1.80e+04 &   10 &           &           & iters  \\ 
 \hdashline 
     &           &    1 &  1.38e-08 &  4.96e-09 &      \\ 
     &           &    2 &  1.38e-08 &  3.95e-16 &      \\ 
     &           &    3 &  1.38e-08 &  3.94e-16 &      \\ 
     &           &    4 &  6.39e-08 &  3.94e-16 &      \\ 
     &           &    5 &  1.39e-08 &  1.82e-15 &      \\ 
     &           &    6 &  1.38e-08 &  3.96e-16 &      \\ 
     &           &    7 &  1.38e-08 &  3.95e-16 &      \\ 
     &           &    8 &  1.38e-08 &  3.95e-16 &      \\ 
     &           &    9 &  1.38e-08 &  3.94e-16 &      \\ 
     &           &   10 &  6.39e-08 &  3.93e-16 &      \\ 
1802 &  1.80e+04 &   10 &           &           & iters  \\ 
 \hdashline 
     &           &    1 &  2.57e-08 &  3.23e-09 &      \\ 
     &           &    2 &  2.56e-08 &  7.32e-16 &      \\ 
     &           &    3 &  2.56e-08 &  7.31e-16 &      \\ 
     &           &    4 &  5.21e-08 &  7.30e-16 &      \\ 
     &           &    5 &  2.56e-08 &  1.49e-15 &      \\ 
     &           &    6 &  2.56e-08 &  7.31e-16 &      \\ 
     &           &    7 &  5.21e-08 &  7.30e-16 &      \\ 
     &           &    8 &  2.56e-08 &  1.49e-15 &      \\ 
     &           &    9 &  2.56e-08 &  7.31e-16 &      \\ 
     &           &   10 &  5.21e-08 &  7.30e-16 &      \\ 
1803 &  1.80e+04 &   10 &           &           & iters  \\ 
 \hdashline 
     &           &    1 &  3.10e-08 &  2.63e-09 &      \\ 
     &           &    2 &  3.09e-08 &  8.85e-16 &      \\ 
     &           &    3 &  4.68e-08 &  8.83e-16 &      \\ 
     &           &    4 &  3.10e-08 &  1.34e-15 &      \\ 
     &           &    5 &  4.68e-08 &  8.84e-16 &      \\ 
     &           &    6 &  3.10e-08 &  1.33e-15 &      \\ 
     &           &    7 &  3.09e-08 &  8.85e-16 &      \\ 
     &           &    8 &  4.68e-08 &  8.83e-16 &      \\ 
     &           &    9 &  3.10e-08 &  1.34e-15 &      \\ 
     &           &   10 &  4.68e-08 &  8.84e-16 &      \\ 
1804 &  1.80e+04 &   10 &           &           & iters  \\ 
 \hdashline 
     &           &    1 &  2.45e-09 &  2.79e-09 &      \\ 
     &           &    2 &  2.45e-09 &  7.00e-17 &      \\ 
     &           &    3 &  2.44e-09 &  6.99e-17 &      \\ 
     &           &    4 &  2.44e-09 &  6.98e-17 &      \\ 
     &           &    5 &  2.44e-09 &  6.97e-17 &      \\ 
     &           &    6 &  2.43e-09 &  6.96e-17 &      \\ 
     &           &    7 &  2.43e-09 &  6.95e-17 &      \\ 
     &           &    8 &  2.43e-09 &  6.94e-17 &      \\ 
     &           &    9 &  2.42e-09 &  6.93e-17 &      \\ 
     &           &   10 &  2.42e-09 &  6.92e-17 &      \\ 
1805 &  1.80e+04 &   10 &           &           & iters  \\ 
 \hdashline 
     &           &    1 &  3.03e-08 &  7.82e-10 &      \\ 
     &           &    2 &  3.03e-08 &  8.65e-16 &      \\ 
     &           &    3 &  4.75e-08 &  8.64e-16 &      \\ 
     &           &    4 &  3.03e-08 &  1.35e-15 &      \\ 
     &           &    5 &  4.74e-08 &  8.65e-16 &      \\ 
     &           &    6 &  3.03e-08 &  1.35e-15 &      \\ 
     &           &    7 &  3.03e-08 &  8.66e-16 &      \\ 
     &           &    8 &  4.74e-08 &  8.64e-16 &      \\ 
     &           &    9 &  3.03e-08 &  1.35e-15 &      \\ 
     &           &   10 &  3.03e-08 &  8.65e-16 &      \\ 
1806 &  1.80e+04 &   10 &           &           & iters  \\ 
 \hdashline 
     &           &    1 &  5.35e-08 &  1.05e-09 &      \\ 
     &           &    2 &  2.42e-08 &  1.53e-15 &      \\ 
     &           &    3 &  5.35e-08 &  6.92e-16 &      \\ 
     &           &    4 &  2.43e-08 &  1.53e-15 &      \\ 
     &           &    5 &  2.42e-08 &  6.93e-16 &      \\ 
     &           &    6 &  5.35e-08 &  6.92e-16 &      \\ 
     &           &    7 &  2.43e-08 &  1.53e-15 &      \\ 
     &           &    8 &  2.42e-08 &  6.93e-16 &      \\ 
     &           &    9 &  5.35e-08 &  6.92e-16 &      \\ 
     &           &   10 &  2.43e-08 &  1.53e-15 &      \\ 
1807 &  1.81e+04 &   10 &           &           & iters  \\ 
 \hdashline 
     &           &    1 &  4.71e-08 &  6.75e-11 &      \\ 
     &           &    2 &  3.07e-08 &  1.34e-15 &      \\ 
     &           &    3 &  3.07e-08 &  8.76e-16 &      \\ 
     &           &    4 &  4.71e-08 &  8.75e-16 &      \\ 
     &           &    5 &  3.07e-08 &  1.34e-15 &      \\ 
     &           &    6 &  4.71e-08 &  8.76e-16 &      \\ 
     &           &    7 &  3.07e-08 &  1.34e-15 &      \\ 
     &           &    8 &  3.07e-08 &  8.76e-16 &      \\ 
     &           &    9 &  4.71e-08 &  8.75e-16 &      \\ 
     &           &   10 &  3.07e-08 &  1.34e-15 &      \\ 
1808 &  1.81e+04 &   10 &           &           & iters  \\ 
 \hdashline 
     &           &    1 &  2.31e-08 &  3.93e-12 &      \\ 
     &           &    2 &  2.30e-08 &  6.58e-16 &      \\ 
     &           &    3 &  5.47e-08 &  6.57e-16 &      \\ 
     &           &    4 &  2.31e-08 &  1.56e-15 &      \\ 
     &           &    5 &  2.30e-08 &  6.58e-16 &      \\ 
     &           &    6 &  5.47e-08 &  6.57e-16 &      \\ 
     &           &    7 &  2.31e-08 &  1.56e-15 &      \\ 
     &           &    8 &  2.30e-08 &  6.59e-16 &      \\ 
     &           &    9 &  2.30e-08 &  6.58e-16 &      \\ 
     &           &   10 &  5.47e-08 &  6.57e-16 &      \\ 
1809 &  1.81e+04 &   10 &           &           & iters  \\ 
 \hdashline 
     &           &    1 &  6.03e-09 &  7.27e-10 &      \\ 
     &           &    2 &  6.02e-09 &  1.72e-16 &      \\ 
     &           &    3 &  7.17e-08 &  1.72e-16 &      \\ 
     &           &    4 &  8.38e-08 &  2.05e-15 &      \\ 
     &           &    5 &  7.17e-08 &  2.39e-15 &      \\ 
     &           &    6 &  6.09e-09 &  2.05e-15 &      \\ 
     &           &    7 &  6.09e-09 &  1.74e-16 &      \\ 
     &           &    8 &  6.08e-09 &  1.74e-16 &      \\ 
     &           &    9 &  6.07e-09 &  1.73e-16 &      \\ 
     &           &   10 &  6.06e-09 &  1.73e-16 &      \\ 
1810 &  1.81e+04 &   10 &           &           & iters  \\ 
 \hdashline 
     &           &    1 &  1.80e-08 &  1.26e-09 &      \\ 
     &           &    2 &  5.97e-08 &  5.14e-16 &      \\ 
     &           &    3 &  1.81e-08 &  1.70e-15 &      \\ 
     &           &    4 &  1.80e-08 &  5.16e-16 &      \\ 
     &           &    5 &  1.80e-08 &  5.15e-16 &      \\ 
     &           &    6 &  5.97e-08 &  5.14e-16 &      \\ 
     &           &    7 &  1.81e-08 &  1.70e-15 &      \\ 
     &           &    8 &  1.80e-08 &  5.16e-16 &      \\ 
     &           &    9 &  1.80e-08 &  5.15e-16 &      \\ 
     &           &   10 &  1.80e-08 &  5.14e-16 &      \\ 
1811 &  1.81e+04 &   10 &           &           & iters  \\ 
 \hdashline 
     &           &    1 &  7.71e-08 &  9.39e-10 &      \\ 
     &           &    2 &  6.65e-10 &  2.20e-15 &      \\ 
     &           &    3 &  6.64e-10 &  1.90e-17 &      \\ 
     &           &    4 &  6.63e-10 &  1.90e-17 &      \\ 
     &           &    5 &  6.62e-10 &  1.89e-17 &      \\ 
     &           &    6 &  6.62e-10 &  1.89e-17 &      \\ 
     &           &    7 &  6.61e-10 &  1.89e-17 &      \\ 
     &           &    8 &  6.60e-10 &  1.89e-17 &      \\ 
     &           &    9 &  6.59e-10 &  1.88e-17 &      \\ 
     &           &   10 &  6.58e-10 &  1.88e-17 &      \\ 
1812 &  1.81e+04 &   10 &           &           & iters  \\ 
 \hdashline 
     &           &    1 &  3.45e-08 &  3.38e-10 &      \\ 
     &           &    2 &  3.45e-08 &  9.85e-16 &      \\ 
     &           &    3 &  4.33e-08 &  9.84e-16 &      \\ 
     &           &    4 &  3.45e-08 &  1.23e-15 &      \\ 
     &           &    5 &  4.33e-08 &  9.84e-16 &      \\ 
     &           &    6 &  3.45e-08 &  1.23e-15 &      \\ 
     &           &    7 &  3.44e-08 &  9.84e-16 &      \\ 
     &           &    8 &  4.33e-08 &  9.83e-16 &      \\ 
     &           &    9 &  3.45e-08 &  1.24e-15 &      \\ 
     &           &   10 &  4.33e-08 &  9.83e-16 &      \\ 
1813 &  1.81e+04 &   10 &           &           & iters  \\ 
 \hdashline 
     &           &    1 &  3.92e-08 &  2.06e-09 &      \\ 
     &           &    2 &  3.86e-08 &  1.12e-15 &      \\ 
     &           &    3 &  3.92e-08 &  1.10e-15 &      \\ 
     &           &    4 &  3.86e-08 &  1.12e-15 &      \\ 
     &           &    5 &  3.92e-08 &  1.10e-15 &      \\ 
     &           &    6 &  3.86e-08 &  1.12e-15 &      \\ 
     &           &    7 &  3.92e-08 &  1.10e-15 &      \\ 
     &           &    8 &  3.86e-08 &  1.12e-15 &      \\ 
     &           &    9 &  3.92e-08 &  1.10e-15 &      \\ 
     &           &   10 &  3.86e-08 &  1.12e-15 &      \\ 
1814 &  1.81e+04 &   10 &           &           & iters  \\ 
 \hdashline 
     &           &    1 &  2.49e-09 &  1.52e-09 &      \\ 
     &           &    2 &  2.49e-09 &  7.11e-17 &      \\ 
     &           &    3 &  2.48e-09 &  7.10e-17 &      \\ 
     &           &    4 &  2.48e-09 &  7.09e-17 &      \\ 
     &           &    5 &  2.48e-09 &  7.08e-17 &      \\ 
     &           &    6 &  2.47e-09 &  7.07e-17 &      \\ 
     &           &    7 &  2.47e-09 &  7.06e-17 &      \\ 
     &           &    8 &  2.47e-09 &  7.05e-17 &      \\ 
     &           &    9 &  2.46e-09 &  7.04e-17 &      \\ 
     &           &   10 &  2.46e-09 &  7.03e-17 &      \\ 
1815 &  1.81e+04 &   10 &           &           & iters  \\ 
 \hdashline 
     &           &    1 &  1.16e-08 &  4.26e-11 &      \\ 
     &           &    2 &  1.16e-08 &  3.31e-16 &      \\ 
     &           &    3 &  1.16e-08 &  3.31e-16 &      \\ 
     &           &    4 &  1.16e-08 &  3.30e-16 &      \\ 
     &           &    5 &  6.62e-08 &  3.30e-16 &      \\ 
     &           &    6 &  1.16e-08 &  1.89e-15 &      \\ 
     &           &    7 &  1.16e-08 &  3.32e-16 &      \\ 
     &           &    8 &  1.16e-08 &  3.32e-16 &      \\ 
     &           &    9 &  1.16e-08 &  3.31e-16 &      \\ 
     &           &   10 &  1.16e-08 &  3.31e-16 &      \\ 
1816 &  1.82e+04 &   10 &           &           & iters  \\ 
 \hdashline 
     &           &    1 &  4.67e-08 &  7.33e-10 &      \\ 
     &           &    2 &  3.10e-08 &  1.33e-15 &      \\ 
     &           &    3 &  3.10e-08 &  8.85e-16 &      \\ 
     &           &    4 &  4.68e-08 &  8.84e-16 &      \\ 
     &           &    5 &  3.10e-08 &  1.33e-15 &      \\ 
     &           &    6 &  4.67e-08 &  8.85e-16 &      \\ 
     &           &    7 &  3.10e-08 &  1.33e-15 &      \\ 
     &           &    8 &  3.10e-08 &  8.85e-16 &      \\ 
     &           &    9 &  4.68e-08 &  8.84e-16 &      \\ 
     &           &   10 &  3.10e-08 &  1.33e-15 &      \\ 
1817 &  1.82e+04 &   10 &           &           & iters  \\ 
 \hdashline 
     &           &    1 &  2.69e-09 &  1.82e-09 &      \\ 
     &           &    2 &  2.69e-09 &  7.68e-17 &      \\ 
     &           &    3 &  2.68e-09 &  7.67e-17 &      \\ 
     &           &    4 &  2.68e-09 &  7.66e-17 &      \\ 
     &           &    5 &  2.68e-09 &  7.65e-17 &      \\ 
     &           &    6 &  2.67e-09 &  7.64e-17 &      \\ 
     &           &    7 &  2.67e-09 &  7.63e-17 &      \\ 
     &           &    8 &  2.67e-09 &  7.62e-17 &      \\ 
     &           &    9 &  2.66e-09 &  7.61e-17 &      \\ 
     &           &   10 &  2.66e-09 &  7.60e-17 &      \\ 
1818 &  1.82e+04 &   10 &           &           & iters  \\ 
 \hdashline 
     &           &    1 &  4.25e-08 &  1.90e-09 &      \\ 
     &           &    2 &  3.52e-08 &  1.21e-15 &      \\ 
     &           &    3 &  4.25e-08 &  1.00e-15 &      \\ 
     &           &    4 &  3.52e-08 &  1.21e-15 &      \\ 
     &           &    5 &  4.25e-08 &  1.01e-15 &      \\ 
     &           &    6 &  3.52e-08 &  1.21e-15 &      \\ 
     &           &    7 &  4.25e-08 &  1.01e-15 &      \\ 
     &           &    8 &  3.52e-08 &  1.21e-15 &      \\ 
     &           &    9 &  3.52e-08 &  1.01e-15 &      \\ 
     &           &   10 &  4.26e-08 &  1.00e-15 &      \\ 
1819 &  1.82e+04 &   10 &           &           & iters  \\ 
 \hdashline 
     &           &    1 &  1.67e-08 &  1.92e-09 &      \\ 
     &           &    2 &  6.10e-08 &  4.76e-16 &      \\ 
     &           &    3 &  1.67e-08 &  1.74e-15 &      \\ 
     &           &    4 &  1.67e-08 &  4.78e-16 &      \\ 
     &           &    5 &  1.67e-08 &  4.77e-16 &      \\ 
     &           &    6 &  1.67e-08 &  4.77e-16 &      \\ 
     &           &    7 &  6.10e-08 &  4.76e-16 &      \\ 
     &           &    8 &  1.67e-08 &  1.74e-15 &      \\ 
     &           &    9 &  1.67e-08 &  4.78e-16 &      \\ 
     &           &   10 &  1.67e-08 &  4.77e-16 &      \\ 
1820 &  1.82e+04 &   10 &           &           & iters  \\ 
 \hdashline 
     &           &    1 &  2.68e-08 &  2.25e-10 &      \\ 
     &           &    2 &  5.09e-08 &  7.65e-16 &      \\ 
     &           &    3 &  2.68e-08 &  1.45e-15 &      \\ 
     &           &    4 &  5.09e-08 &  7.66e-16 &      \\ 
     &           &    5 &  2.69e-08 &  1.45e-15 &      \\ 
     &           &    6 &  2.68e-08 &  7.67e-16 &      \\ 
     &           &    7 &  5.09e-08 &  7.66e-16 &      \\ 
     &           &    8 &  2.69e-08 &  1.45e-15 &      \\ 
     &           &    9 &  2.68e-08 &  7.67e-16 &      \\ 
     &           &   10 &  5.09e-08 &  7.66e-16 &      \\ 
1821 &  1.82e+04 &   10 &           &           & iters  \\ 
 \hdashline 
     &           &    1 &  3.13e-08 &  3.10e-09 &      \\ 
     &           &    2 &  3.13e-08 &  8.94e-16 &      \\ 
     &           &    3 &  4.64e-08 &  8.93e-16 &      \\ 
     &           &    4 &  3.13e-08 &  1.33e-15 &      \\ 
     &           &    5 &  4.64e-08 &  8.94e-16 &      \\ 
     &           &    6 &  3.13e-08 &  1.32e-15 &      \\ 
     &           &    7 &  4.64e-08 &  8.94e-16 &      \\ 
     &           &    8 &  3.14e-08 &  1.32e-15 &      \\ 
     &           &    9 &  3.13e-08 &  8.95e-16 &      \\ 
     &           &   10 &  4.64e-08 &  8.94e-16 &      \\ 
1822 &  1.82e+04 &   10 &           &           & iters  \\ 
 \hdashline 
     &           &    1 &  4.17e-08 &  2.97e-09 &      \\ 
     &           &    2 &  3.61e-08 &  1.19e-15 &      \\ 
     &           &    3 &  4.17e-08 &  1.03e-15 &      \\ 
     &           &    4 &  3.61e-08 &  1.19e-15 &      \\ 
     &           &    5 &  4.17e-08 &  1.03e-15 &      \\ 
     &           &    6 &  3.61e-08 &  1.19e-15 &      \\ 
     &           &    7 &  3.60e-08 &  1.03e-15 &      \\ 
     &           &    8 &  4.17e-08 &  1.03e-15 &      \\ 
     &           &    9 &  3.60e-08 &  1.19e-15 &      \\ 
     &           &   10 &  4.17e-08 &  1.03e-15 &      \\ 
1823 &  1.82e+04 &   10 &           &           & iters  \\ 
 \hdashline 
     &           &    1 &  4.39e-08 &  1.70e-09 &      \\ 
     &           &    2 &  3.39e-08 &  1.25e-15 &      \\ 
     &           &    3 &  4.38e-08 &  9.67e-16 &      \\ 
     &           &    4 &  3.39e-08 &  1.25e-15 &      \\ 
     &           &    5 &  4.38e-08 &  9.68e-16 &      \\ 
     &           &    6 &  3.39e-08 &  1.25e-15 &      \\ 
     &           &    7 &  3.39e-08 &  9.68e-16 &      \\ 
     &           &    8 &  4.39e-08 &  9.67e-16 &      \\ 
     &           &    9 &  3.39e-08 &  1.25e-15 &      \\ 
     &           &   10 &  4.38e-08 &  9.67e-16 &      \\ 
1824 &  1.82e+04 &   10 &           &           & iters  \\ 
 \hdashline 
     &           &    1 &  3.60e-08 &  9.40e-10 &      \\ 
     &           &    2 &  3.59e-08 &  1.03e-15 &      \\ 
     &           &    3 &  4.18e-08 &  1.03e-15 &      \\ 
     &           &    4 &  3.59e-08 &  1.19e-15 &      \\ 
     &           &    5 &  3.59e-08 &  1.03e-15 &      \\ 
     &           &    6 &  4.19e-08 &  1.02e-15 &      \\ 
     &           &    7 &  3.59e-08 &  1.19e-15 &      \\ 
     &           &    8 &  4.19e-08 &  1.02e-15 &      \\ 
     &           &    9 &  3.59e-08 &  1.19e-15 &      \\ 
     &           &   10 &  4.19e-08 &  1.02e-15 &      \\ 
1825 &  1.82e+04 &   10 &           &           & iters  \\ 
 \hdashline 
     &           &    1 &  1.75e-08 &  8.36e-10 &      \\ 
     &           &    2 &  1.75e-08 &  5.00e-16 &      \\ 
     &           &    3 &  6.02e-08 &  4.99e-16 &      \\ 
     &           &    4 &  1.75e-08 &  1.72e-15 &      \\ 
     &           &    5 &  1.75e-08 &  5.01e-16 &      \\ 
     &           &    6 &  1.75e-08 &  5.00e-16 &      \\ 
     &           &    7 &  6.02e-08 &  4.99e-16 &      \\ 
     &           &    8 &  1.76e-08 &  1.72e-15 &      \\ 
     &           &    9 &  1.75e-08 &  5.01e-16 &      \\ 
     &           &   10 &  1.75e-08 &  5.00e-16 &      \\ 
1826 &  1.82e+04 &   10 &           &           & iters  \\ 
 \hdashline 
     &           &    1 &  1.49e-08 &  1.04e-09 &      \\ 
     &           &    2 &  6.28e-08 &  4.26e-16 &      \\ 
     &           &    3 &  1.50e-08 &  1.79e-15 &      \\ 
     &           &    4 &  1.50e-08 &  4.28e-16 &      \\ 
     &           &    5 &  1.50e-08 &  4.28e-16 &      \\ 
     &           &    6 &  1.49e-08 &  4.27e-16 &      \\ 
     &           &    7 &  6.28e-08 &  4.26e-16 &      \\ 
     &           &    8 &  1.50e-08 &  1.79e-15 &      \\ 
     &           &    9 &  1.50e-08 &  4.28e-16 &      \\ 
     &           &   10 &  1.50e-08 &  4.28e-16 &      \\ 
1827 &  1.83e+04 &   10 &           &           & iters  \\ 
 \hdashline 
     &           &    1 &  2.92e-08 &  8.15e-10 &      \\ 
     &           &    2 &  2.92e-08 &  8.34e-16 &      \\ 
     &           &    3 &  4.86e-08 &  8.32e-16 &      \\ 
     &           &    4 &  2.92e-08 &  1.39e-15 &      \\ 
     &           &    5 &  2.92e-08 &  8.33e-16 &      \\ 
     &           &    6 &  4.86e-08 &  8.32e-16 &      \\ 
     &           &    7 &  2.92e-08 &  1.39e-15 &      \\ 
     &           &    8 &  2.91e-08 &  8.33e-16 &      \\ 
     &           &    9 &  4.86e-08 &  8.32e-16 &      \\ 
     &           &   10 &  2.92e-08 &  1.39e-15 &      \\ 
1828 &  1.83e+04 &   10 &           &           & iters  \\ 
 \hdashline 
     &           &    1 &  8.90e-08 &  3.51e-10 &      \\ 
     &           &    2 &  6.65e-08 &  2.54e-15 &      \\ 
     &           &    3 &  1.13e-08 &  1.90e-15 &      \\ 
     &           &    4 &  1.13e-08 &  3.22e-16 &      \\ 
     &           &    5 &  1.12e-08 &  3.22e-16 &      \\ 
     &           &    6 &  1.12e-08 &  3.21e-16 &      \\ 
     &           &    7 &  1.12e-08 &  3.21e-16 &      \\ 
     &           &    8 &  6.65e-08 &  3.20e-16 &      \\ 
     &           &    9 &  8.90e-08 &  1.90e-15 &      \\ 
     &           &   10 &  6.65e-08 &  2.54e-15 &      \\ 
1829 &  1.83e+04 &   10 &           &           & iters  \\ 
 \hdashline 
     &           &    1 &  7.71e-09 &  1.70e-09 &      \\ 
     &           &    2 &  7.70e-09 &  2.20e-16 &      \\ 
     &           &    3 &  7.69e-09 &  2.20e-16 &      \\ 
     &           &    4 &  7.68e-09 &  2.19e-16 &      \\ 
     &           &    5 &  7.67e-09 &  2.19e-16 &      \\ 
     &           &    6 &  7.00e-08 &  2.19e-16 &      \\ 
     &           &    7 &  8.54e-08 &  2.00e-15 &      \\ 
     &           &    8 &  7.01e-08 &  2.44e-15 &      \\ 
     &           &    9 &  7.73e-09 &  2.00e-15 &      \\ 
     &           &   10 &  7.72e-09 &  2.21e-16 &      \\ 
1830 &  1.83e+04 &   10 &           &           & iters  \\ 
 \hdashline 
     &           &    1 &  5.08e-09 &  1.24e-09 &      \\ 
     &           &    2 &  5.08e-09 &  1.45e-16 &      \\ 
     &           &    3 &  5.07e-09 &  1.45e-16 &      \\ 
     &           &    4 &  5.06e-09 &  1.45e-16 &      \\ 
     &           &    5 &  5.05e-09 &  1.44e-16 &      \\ 
     &           &    6 &  5.05e-09 &  1.44e-16 &      \\ 
     &           &    7 &  5.04e-09 &  1.44e-16 &      \\ 
     &           &    8 &  5.03e-09 &  1.44e-16 &      \\ 
     &           &    9 &  5.02e-09 &  1.44e-16 &      \\ 
     &           &   10 &  5.02e-09 &  1.43e-16 &      \\ 
1831 &  1.83e+04 &   10 &           &           & iters  \\ 
 \hdashline 
     &           &    1 &  4.53e-08 &  2.91e-10 &      \\ 
     &           &    2 &  4.52e-08 &  1.29e-15 &      \\ 
     &           &    3 &  3.25e-08 &  1.29e-15 &      \\ 
     &           &    4 &  4.52e-08 &  9.28e-16 &      \\ 
     &           &    5 &  3.25e-08 &  1.29e-15 &      \\ 
     &           &    6 &  4.52e-08 &  9.29e-16 &      \\ 
     &           &    7 &  3.26e-08 &  1.29e-15 &      \\ 
     &           &    8 &  3.25e-08 &  9.29e-16 &      \\ 
     &           &    9 &  4.52e-08 &  9.28e-16 &      \\ 
     &           &   10 &  3.25e-08 &  1.29e-15 &      \\ 
1832 &  1.83e+04 &   10 &           &           & iters  \\ 
 \hdashline 
     &           &    1 &  5.10e-08 &  7.52e-10 &      \\ 
     &           &    2 &  2.67e-08 &  1.46e-15 &      \\ 
     &           &    3 &  2.67e-08 &  7.63e-16 &      \\ 
     &           &    4 &  5.10e-08 &  7.62e-16 &      \\ 
     &           &    5 &  2.67e-08 &  1.46e-15 &      \\ 
     &           &    6 &  2.67e-08 &  7.63e-16 &      \\ 
     &           &    7 &  5.10e-08 &  7.62e-16 &      \\ 
     &           &    8 &  2.67e-08 &  1.46e-15 &      \\ 
     &           &    9 &  2.67e-08 &  7.63e-16 &      \\ 
     &           &   10 &  5.10e-08 &  7.62e-16 &      \\ 
1833 &  1.83e+04 &   10 &           &           & iters  \\ 
 \hdashline 
     &           &    1 &  1.33e-08 &  5.89e-10 &      \\ 
     &           &    2 &  1.32e-08 &  3.79e-16 &      \\ 
     &           &    3 &  1.32e-08 &  3.78e-16 &      \\ 
     &           &    4 &  6.45e-08 &  3.77e-16 &      \\ 
     &           &    5 &  9.10e-08 &  1.84e-15 &      \\ 
     &           &    6 &  6.45e-08 &  2.60e-15 &      \\ 
     &           &    7 &  1.33e-08 &  1.84e-15 &      \\ 
     &           &    8 &  1.32e-08 &  3.79e-16 &      \\ 
     &           &    9 &  1.32e-08 &  3.78e-16 &      \\ 
     &           &   10 &  6.45e-08 &  3.77e-16 &      \\ 
1834 &  1.83e+04 &   10 &           &           & iters  \\ 
 \hdashline 
     &           &    1 &  8.49e-08 &  1.80e-09 &      \\ 
     &           &    2 &  7.09e-09 &  2.42e-15 &      \\ 
     &           &    3 &  7.08e-09 &  2.02e-16 &      \\ 
     &           &    4 &  7.06e-08 &  2.02e-16 &      \\ 
     &           &    5 &  8.49e-08 &  2.02e-15 &      \\ 
     &           &    6 &  7.06e-08 &  2.42e-15 &      \\ 
     &           &    7 &  7.15e-09 &  2.02e-15 &      \\ 
     &           &    8 &  7.14e-09 &  2.04e-16 &      \\ 
     &           &    9 &  7.13e-09 &  2.04e-16 &      \\ 
     &           &   10 &  7.12e-09 &  2.04e-16 &      \\ 
1835 &  1.83e+04 &   10 &           &           & iters  \\ 
 \hdashline 
     &           &    1 &  2.07e-09 &  1.45e-09 &      \\ 
     &           &    2 &  2.07e-09 &  5.90e-17 &      \\ 
     &           &    3 &  2.06e-09 &  5.90e-17 &      \\ 
     &           &    4 &  2.06e-09 &  5.89e-17 &      \\ 
     &           &    5 &  2.06e-09 &  5.88e-17 &      \\ 
     &           &    6 &  2.05e-09 &  5.87e-17 &      \\ 
     &           &    7 &  2.05e-09 &  5.86e-17 &      \\ 
     &           &    8 &  2.05e-09 &  5.85e-17 &      \\ 
     &           &    9 &  2.05e-09 &  5.85e-17 &      \\ 
     &           &   10 &  2.04e-09 &  5.84e-17 &      \\ 
1836 &  1.84e+04 &   10 &           &           & iters  \\ 
 \hdashline 
     &           &    1 &  2.89e-08 &  2.07e-09 &      \\ 
     &           &    2 &  4.88e-08 &  8.25e-16 &      \\ 
     &           &    3 &  2.89e-08 &  1.39e-15 &      \\ 
     &           &    4 &  2.89e-08 &  8.26e-16 &      \\ 
     &           &    5 &  4.88e-08 &  8.25e-16 &      \\ 
     &           &    6 &  2.89e-08 &  1.39e-15 &      \\ 
     &           &    7 &  4.88e-08 &  8.25e-16 &      \\ 
     &           &    8 &  2.89e-08 &  1.39e-15 &      \\ 
     &           &    9 &  2.89e-08 &  8.26e-16 &      \\ 
     &           &   10 &  4.88e-08 &  8.25e-16 &      \\ 
1837 &  1.84e+04 &   10 &           &           & iters  \\ 
 \hdashline 
     &           &    1 &  2.19e-08 &  3.37e-09 &      \\ 
     &           &    2 &  1.70e-08 &  6.25e-16 &      \\ 
     &           &    3 &  2.19e-08 &  4.84e-16 &      \\ 
     &           &    4 &  1.70e-08 &  6.25e-16 &      \\ 
     &           &    5 &  2.19e-08 &  4.84e-16 &      \\ 
     &           &    6 &  1.70e-08 &  6.25e-16 &      \\ 
     &           &    7 &  2.19e-08 &  4.85e-16 &      \\ 
     &           &    8 &  1.70e-08 &  6.25e-16 &      \\ 
     &           &    9 &  1.70e-08 &  4.85e-16 &      \\ 
     &           &   10 &  2.19e-08 &  4.84e-16 &      \\ 
1838 &  1.84e+04 &   10 &           &           & iters  \\ 
 \hdashline 
     &           &    1 &  3.10e-08 &  2.04e-09 &      \\ 
     &           &    2 &  4.68e-08 &  8.83e-16 &      \\ 
     &           &    3 &  3.10e-08 &  1.34e-15 &      \\ 
     &           &    4 &  4.68e-08 &  8.84e-16 &      \\ 
     &           &    5 &  3.10e-08 &  1.33e-15 &      \\ 
     &           &    6 &  3.10e-08 &  8.85e-16 &      \\ 
     &           &    7 &  4.68e-08 &  8.83e-16 &      \\ 
     &           &    8 &  3.10e-08 &  1.34e-15 &      \\ 
     &           &    9 &  4.68e-08 &  8.84e-16 &      \\ 
     &           &   10 &  3.10e-08 &  1.33e-15 &      \\ 
1839 &  1.84e+04 &   10 &           &           & iters  \\ 
 \hdashline 
     &           &    1 &  3.40e-09 &  2.62e-10 &      \\ 
     &           &    2 &  3.40e-09 &  9.71e-17 &      \\ 
     &           &    3 &  3.39e-09 &  9.70e-17 &      \\ 
     &           &    4 &  3.39e-09 &  9.68e-17 &      \\ 
     &           &    5 &  3.38e-09 &  9.67e-17 &      \\ 
     &           &    6 &  3.38e-09 &  9.65e-17 &      \\ 
     &           &    7 &  7.43e-08 &  9.64e-17 &      \\ 
     &           &    8 &  8.12e-08 &  2.12e-15 &      \\ 
     &           &    9 &  7.43e-08 &  2.32e-15 &      \\ 
     &           &   10 &  8.12e-08 &  2.12e-15 &      \\ 
1840 &  1.84e+04 &   10 &           &           & iters  \\ 
 \hdashline 
     &           &    1 &  2.60e-08 &  7.96e-10 &      \\ 
     &           &    2 &  2.59e-08 &  7.41e-16 &      \\ 
     &           &    3 &  2.59e-08 &  7.40e-16 &      \\ 
     &           &    4 &  5.18e-08 &  7.39e-16 &      \\ 
     &           &    5 &  2.59e-08 &  1.48e-15 &      \\ 
     &           &    6 &  2.59e-08 &  7.40e-16 &      \\ 
     &           &    7 &  5.18e-08 &  7.39e-16 &      \\ 
     &           &    8 &  2.59e-08 &  1.48e-15 &      \\ 
     &           &    9 &  2.59e-08 &  7.40e-16 &      \\ 
     &           &   10 &  5.18e-08 &  7.39e-16 &      \\ 
1841 &  1.84e+04 &   10 &           &           & iters  \\ 
 \hdashline 
     &           &    1 &  9.06e-09 &  1.06e-09 &      \\ 
     &           &    2 &  6.86e-08 &  2.59e-16 &      \\ 
     &           &    3 &  8.68e-08 &  1.96e-15 &      \\ 
     &           &    4 &  6.87e-08 &  2.48e-15 &      \\ 
     &           &    5 &  9.12e-09 &  1.96e-15 &      \\ 
     &           &    6 &  9.11e-09 &  2.60e-16 &      \\ 
     &           &    7 &  9.09e-09 &  2.60e-16 &      \\ 
     &           &    8 &  9.08e-09 &  2.60e-16 &      \\ 
     &           &    9 &  9.07e-09 &  2.59e-16 &      \\ 
     &           &   10 &  6.86e-08 &  2.59e-16 &      \\ 
1842 &  1.84e+04 &   10 &           &           & iters  \\ 
 \hdashline 
     &           &    1 &  2.33e-08 &  2.83e-10 &      \\ 
     &           &    2 &  2.33e-08 &  6.66e-16 &      \\ 
     &           &    3 &  2.33e-08 &  6.65e-16 &      \\ 
     &           &    4 &  5.44e-08 &  6.65e-16 &      \\ 
     &           &    5 &  2.33e-08 &  1.55e-15 &      \\ 
     &           &    6 &  2.33e-08 &  6.66e-16 &      \\ 
     &           &    7 &  5.44e-08 &  6.65e-16 &      \\ 
     &           &    8 &  2.33e-08 &  1.55e-15 &      \\ 
     &           &    9 &  2.33e-08 &  6.66e-16 &      \\ 
     &           &   10 &  2.33e-08 &  6.65e-16 &      \\ 
1843 &  1.84e+04 &   10 &           &           & iters  \\ 
 \hdashline 
     &           &    1 &  1.60e-08 &  1.72e-09 &      \\ 
     &           &    2 &  1.60e-08 &  4.56e-16 &      \\ 
     &           &    3 &  2.29e-08 &  4.55e-16 &      \\ 
     &           &    4 &  1.60e-08 &  6.54e-16 &      \\ 
     &           &    5 &  1.59e-08 &  4.56e-16 &      \\ 
     &           &    6 &  2.29e-08 &  4.55e-16 &      \\ 
     &           &    7 &  1.60e-08 &  6.54e-16 &      \\ 
     &           &    8 &  2.29e-08 &  4.55e-16 &      \\ 
     &           &    9 &  1.60e-08 &  6.54e-16 &      \\ 
     &           &   10 &  1.59e-08 &  4.56e-16 &      \\ 
1844 &  1.84e+04 &   10 &           &           & iters  \\ 
 \hdashline 
     &           &    1 &  2.30e-09 &  2.66e-09 &      \\ 
     &           &    2 &  2.29e-09 &  6.55e-17 &      \\ 
     &           &    3 &  2.29e-09 &  6.54e-17 &      \\ 
     &           &    4 &  2.29e-09 &  6.53e-17 &      \\ 
     &           &    5 &  2.28e-09 &  6.53e-17 &      \\ 
     &           &    6 &  2.28e-09 &  6.52e-17 &      \\ 
     &           &    7 &  2.28e-09 &  6.51e-17 &      \\ 
     &           &    8 &  2.27e-09 &  6.50e-17 &      \\ 
     &           &    9 &  2.27e-09 &  6.49e-17 &      \\ 
     &           &   10 &  2.27e-09 &  6.48e-17 &      \\ 
1845 &  1.84e+04 &   10 &           &           & iters  \\ 
 \hdashline 
     &           &    1 &  4.42e-08 &  2.92e-09 &      \\ 
     &           &    2 &  3.35e-08 &  1.26e-15 &      \\ 
     &           &    3 &  4.42e-08 &  9.57e-16 &      \\ 
     &           &    4 &  3.35e-08 &  1.26e-15 &      \\ 
     &           &    5 &  3.35e-08 &  9.57e-16 &      \\ 
     &           &    6 &  4.42e-08 &  9.56e-16 &      \\ 
     &           &    7 &  3.35e-08 &  1.26e-15 &      \\ 
     &           &    8 &  4.42e-08 &  9.57e-16 &      \\ 
     &           &    9 &  3.35e-08 &  1.26e-15 &      \\ 
     &           &   10 &  4.42e-08 &  9.57e-16 &      \\ 
1846 &  1.84e+04 &   10 &           &           & iters  \\ 
 \hdashline 
     &           &    1 &  8.01e-08 &  2.97e-09 &      \\ 
     &           &    2 &  7.54e-08 &  2.29e-15 &      \\ 
     &           &    3 &  8.01e-08 &  2.15e-15 &      \\ 
     &           &    4 &  7.54e-08 &  2.29e-15 &      \\ 
     &           &    5 &  2.37e-09 &  2.15e-15 &      \\ 
     &           &    6 &  2.37e-09 &  6.77e-17 &      \\ 
     &           &    7 &  2.37e-09 &  6.76e-17 &      \\ 
     &           &    8 &  2.36e-09 &  6.75e-17 &      \\ 
     &           &    9 &  2.36e-09 &  6.74e-17 &      \\ 
     &           &   10 &  2.36e-09 &  6.73e-17 &      \\ 
1847 &  1.85e+04 &   10 &           &           & iters  \\ 
 \hdashline 
     &           &    1 &  4.24e-08 &  2.92e-09 &      \\ 
     &           &    2 &  3.53e-08 &  1.21e-15 &      \\ 
     &           &    3 &  4.24e-08 &  1.01e-15 &      \\ 
     &           &    4 &  3.53e-08 &  1.21e-15 &      \\ 
     &           &    5 &  4.24e-08 &  1.01e-15 &      \\ 
     &           &    6 &  3.53e-08 &  1.21e-15 &      \\ 
     &           &    7 &  4.24e-08 &  1.01e-15 &      \\ 
     &           &    8 &  3.53e-08 &  1.21e-15 &      \\ 
     &           &    9 &  3.53e-08 &  1.01e-15 &      \\ 
     &           &   10 &  4.25e-08 &  1.01e-15 &      \\ 
1848 &  1.85e+04 &   10 &           &           & iters  \\ 
 \hdashline 
     &           &    1 &  2.85e-08 &  4.03e-10 &      \\ 
     &           &    2 &  2.85e-08 &  8.14e-16 &      \\ 
     &           &    3 &  4.92e-08 &  8.13e-16 &      \\ 
     &           &    4 &  2.85e-08 &  1.41e-15 &      \\ 
     &           &    5 &  4.92e-08 &  8.14e-16 &      \\ 
     &           &    6 &  2.85e-08 &  1.40e-15 &      \\ 
     &           &    7 &  2.85e-08 &  8.15e-16 &      \\ 
     &           &    8 &  4.92e-08 &  8.14e-16 &      \\ 
     &           &    9 &  2.85e-08 &  1.40e-15 &      \\ 
     &           &   10 &  2.85e-08 &  8.14e-16 &      \\ 
1849 &  1.85e+04 &   10 &           &           & iters  \\ 
 \hdashline 
     &           &    1 &  2.23e-08 &  2.11e-09 &      \\ 
     &           &    2 &  5.54e-08 &  6.37e-16 &      \\ 
     &           &    3 &  2.24e-08 &  1.58e-15 &      \\ 
     &           &    4 &  2.23e-08 &  6.38e-16 &      \\ 
     &           &    5 &  5.54e-08 &  6.37e-16 &      \\ 
     &           &    6 &  2.24e-08 &  1.58e-15 &      \\ 
     &           &    7 &  2.23e-08 &  6.38e-16 &      \\ 
     &           &    8 &  2.23e-08 &  6.37e-16 &      \\ 
     &           &    9 &  5.54e-08 &  6.37e-16 &      \\ 
     &           &   10 &  2.24e-08 &  1.58e-15 &      \\ 
1850 &  1.85e+04 &   10 &           &           & iters  \\ 
 \hdashline 
     &           &    1 &  3.02e-08 &  7.24e-11 &      \\ 
     &           &    2 &  4.75e-08 &  8.63e-16 &      \\ 
     &           &    3 &  3.02e-08 &  1.36e-15 &      \\ 
     &           &    4 &  3.02e-08 &  8.63e-16 &      \\ 
     &           &    5 &  4.75e-08 &  8.62e-16 &      \\ 
     &           &    6 &  3.02e-08 &  1.36e-15 &      \\ 
     &           &    7 &  3.02e-08 &  8.63e-16 &      \\ 
     &           &    8 &  4.75e-08 &  8.62e-16 &      \\ 
     &           &    9 &  3.02e-08 &  1.36e-15 &      \\ 
     &           &   10 &  4.75e-08 &  8.62e-16 &      \\ 
1851 &  1.85e+04 &   10 &           &           & iters  \\ 
 \hdashline 
     &           &    1 &  4.10e-08 &  1.78e-09 &      \\ 
     &           &    2 &  3.68e-08 &  1.17e-15 &      \\ 
     &           &    3 &  4.10e-08 &  1.05e-15 &      \\ 
     &           &    4 &  3.68e-08 &  1.17e-15 &      \\ 
     &           &    5 &  4.10e-08 &  1.05e-15 &      \\ 
     &           &    6 &  3.68e-08 &  1.17e-15 &      \\ 
     &           &    7 &  4.10e-08 &  1.05e-15 &      \\ 
     &           &    8 &  3.68e-08 &  1.17e-15 &      \\ 
     &           &    9 &  4.10e-08 &  1.05e-15 &      \\ 
     &           &   10 &  3.68e-08 &  1.17e-15 &      \\ 
1852 &  1.85e+04 &   10 &           &           & iters  \\ 
 \hdashline 
     &           &    1 &  5.95e-08 &  5.89e-10 &      \\ 
     &           &    2 &  1.83e-08 &  1.70e-15 &      \\ 
     &           &    3 &  1.83e-08 &  5.22e-16 &      \\ 
     &           &    4 &  1.82e-08 &  5.21e-16 &      \\ 
     &           &    5 &  1.82e-08 &  5.20e-16 &      \\ 
     &           &    6 &  5.95e-08 &  5.20e-16 &      \\ 
     &           &    7 &  1.83e-08 &  1.70e-15 &      \\ 
     &           &    8 &  1.82e-08 &  5.21e-16 &      \\ 
     &           &    9 &  1.82e-08 &  5.21e-16 &      \\ 
     &           &   10 &  5.95e-08 &  5.20e-16 &      \\ 
1853 &  1.85e+04 &   10 &           &           & iters  \\ 
 \hdashline 
     &           &    1 &  5.59e-08 &  4.33e-10 &      \\ 
     &           &    2 &  2.19e-08 &  1.59e-15 &      \\ 
     &           &    3 &  5.58e-08 &  6.25e-16 &      \\ 
     &           &    4 &  2.19e-08 &  1.59e-15 &      \\ 
     &           &    5 &  2.19e-08 &  6.26e-16 &      \\ 
     &           &    6 &  2.19e-08 &  6.25e-16 &      \\ 
     &           &    7 &  5.58e-08 &  6.24e-16 &      \\ 
     &           &    8 &  2.19e-08 &  1.59e-15 &      \\ 
     &           &    9 &  2.19e-08 &  6.26e-16 &      \\ 
     &           &   10 &  5.58e-08 &  6.25e-16 &      \\ 
1854 &  1.85e+04 &   10 &           &           & iters  \\ 
 \hdashline 
     &           &    1 &  6.83e-08 &  1.72e-09 &      \\ 
     &           &    2 &  9.50e-09 &  1.95e-15 &      \\ 
     &           &    3 &  9.49e-09 &  2.71e-16 &      \\ 
     &           &    4 &  9.48e-09 &  2.71e-16 &      \\ 
     &           &    5 &  9.46e-09 &  2.70e-16 &      \\ 
     &           &    6 &  9.45e-09 &  2.70e-16 &      \\ 
     &           &    7 &  9.44e-09 &  2.70e-16 &      \\ 
     &           &    8 &  6.83e-08 &  2.69e-16 &      \\ 
     &           &    9 &  8.72e-08 &  1.95e-15 &      \\ 
     &           &   10 &  6.83e-08 &  2.49e-15 &      \\ 
1855 &  1.85e+04 &   10 &           &           & iters  \\ 
 \hdashline 
     &           &    1 &  2.37e-08 &  3.00e-09 &      \\ 
     &           &    2 &  5.40e-08 &  6.76e-16 &      \\ 
     &           &    3 &  2.37e-08 &  1.54e-15 &      \\ 
     &           &    4 &  2.37e-08 &  6.77e-16 &      \\ 
     &           &    5 &  5.40e-08 &  6.76e-16 &      \\ 
     &           &    6 &  2.37e-08 &  1.54e-15 &      \\ 
     &           &    7 &  2.37e-08 &  6.77e-16 &      \\ 
     &           &    8 &  2.37e-08 &  6.76e-16 &      \\ 
     &           &    9 &  5.41e-08 &  6.75e-16 &      \\ 
     &           &   10 &  2.37e-08 &  1.54e-15 &      \\ 
1856 &  1.86e+04 &   10 &           &           & iters  \\ 
 \hdashline 
     &           &    1 &  4.62e-09 &  1.25e-09 &      \\ 
     &           &    2 &  4.61e-09 &  1.32e-16 &      \\ 
     &           &    3 &  4.60e-09 &  1.32e-16 &      \\ 
     &           &    4 &  4.60e-09 &  1.31e-16 &      \\ 
     &           &    5 &  4.59e-09 &  1.31e-16 &      \\ 
     &           &    6 &  4.58e-09 &  1.31e-16 &      \\ 
     &           &    7 &  4.58e-09 &  1.31e-16 &      \\ 
     &           &    8 &  4.57e-09 &  1.31e-16 &      \\ 
     &           &    9 &  4.56e-09 &  1.30e-16 &      \\ 
     &           &   10 &  7.31e-08 &  1.30e-16 &      \\ 
1857 &  1.86e+04 &   10 &           &           & iters  \\ 
 \hdashline 
     &           &    1 &  1.21e-09 &  1.77e-09 &      \\ 
     &           &    2 &  1.20e-09 &  3.44e-17 &      \\ 
     &           &    3 &  1.20e-09 &  3.44e-17 &      \\ 
     &           &    4 &  1.20e-09 &  3.43e-17 &      \\ 
     &           &    5 &  1.20e-09 &  3.43e-17 &      \\ 
     &           &    6 &  1.20e-09 &  3.42e-17 &      \\ 
     &           &    7 &  1.20e-09 &  3.42e-17 &      \\ 
     &           &    8 &  1.19e-09 &  3.41e-17 &      \\ 
     &           &    9 &  1.19e-09 &  3.41e-17 &      \\ 
     &           &   10 &  1.19e-09 &  3.40e-17 &      \\ 
1858 &  1.86e+04 &   10 &           &           & iters  \\ 
 \hdashline 
     &           &    1 &  5.94e-08 &  2.02e-09 &      \\ 
     &           &    2 &  1.83e-08 &  1.70e-15 &      \\ 
     &           &    3 &  1.83e-08 &  5.24e-16 &      \\ 
     &           &    4 &  1.83e-08 &  5.23e-16 &      \\ 
     &           &    5 &  1.83e-08 &  5.22e-16 &      \\ 
     &           &    6 &  5.95e-08 &  5.21e-16 &      \\ 
     &           &    7 &  1.83e-08 &  1.70e-15 &      \\ 
     &           &    8 &  1.83e-08 &  5.23e-16 &      \\ 
     &           &    9 &  1.83e-08 &  5.22e-16 &      \\ 
     &           &   10 &  5.94e-08 &  5.21e-16 &      \\ 
1859 &  1.86e+04 &   10 &           &           & iters  \\ 
 \hdashline 
     &           &    1 &  2.66e-08 &  5.64e-10 &      \\ 
     &           &    2 &  2.66e-08 &  7.59e-16 &      \\ 
     &           &    3 &  2.65e-08 &  7.58e-16 &      \\ 
     &           &    4 &  5.12e-08 &  7.57e-16 &      \\ 
     &           &    5 &  2.66e-08 &  1.46e-15 &      \\ 
     &           &    6 &  5.12e-08 &  7.58e-16 &      \\ 
     &           &    7 &  2.66e-08 &  1.46e-15 &      \\ 
     &           &    8 &  2.66e-08 &  7.59e-16 &      \\ 
     &           &    9 &  5.12e-08 &  7.58e-16 &      \\ 
     &           &   10 &  2.66e-08 &  1.46e-15 &      \\ 
1860 &  1.86e+04 &   10 &           &           & iters  \\ 
 \hdashline 
     &           &    1 &  7.64e-09 &  7.20e-10 &      \\ 
     &           &    2 &  7.63e-09 &  2.18e-16 &      \\ 
     &           &    3 &  7.62e-09 &  2.18e-16 &      \\ 
     &           &    4 &  7.61e-09 &  2.17e-16 &      \\ 
     &           &    5 &  7.60e-09 &  2.17e-16 &      \\ 
     &           &    6 &  7.58e-09 &  2.17e-16 &      \\ 
     &           &    7 &  7.01e-08 &  2.16e-16 &      \\ 
     &           &    8 &  8.54e-08 &  2.00e-15 &      \\ 
     &           &    9 &  7.01e-08 &  2.44e-15 &      \\ 
     &           &   10 &  7.65e-09 &  2.00e-15 &      \\ 
1861 &  1.86e+04 &   10 &           &           & iters  \\ 
 \hdashline 
     &           &    1 &  3.31e-09 &  8.23e-10 &      \\ 
     &           &    2 &  3.30e-09 &  9.44e-17 &      \\ 
     &           &    3 &  3.30e-09 &  9.42e-17 &      \\ 
     &           &    4 &  3.29e-09 &  9.41e-17 &      \\ 
     &           &    5 &  3.29e-09 &  9.40e-17 &      \\ 
     &           &    6 &  3.28e-09 &  9.38e-17 &      \\ 
     &           &    7 &  3.28e-09 &  9.37e-17 &      \\ 
     &           &    8 &  3.27e-09 &  9.36e-17 &      \\ 
     &           &    9 &  7.44e-08 &  9.34e-17 &      \\ 
     &           &   10 &  8.11e-08 &  2.12e-15 &      \\ 
1862 &  1.86e+04 &   10 &           &           & iters  \\ 
 \hdashline 
     &           &    1 &  6.76e-09 &  7.99e-11 &      \\ 
     &           &    2 &  6.75e-09 &  1.93e-16 &      \\ 
     &           &    3 &  6.74e-09 &  1.93e-16 &      \\ 
     &           &    4 &  6.73e-09 &  1.92e-16 &      \\ 
     &           &    5 &  6.72e-09 &  1.92e-16 &      \\ 
     &           &    6 &  6.71e-09 &  1.92e-16 &      \\ 
     &           &    7 &  6.70e-09 &  1.91e-16 &      \\ 
     &           &    8 &  7.10e-08 &  1.91e-16 &      \\ 
     &           &    9 &  8.45e-08 &  2.03e-15 &      \\ 
     &           &   10 &  7.10e-08 &  2.41e-15 &      \\ 
1863 &  1.86e+04 &   10 &           &           & iters  \\ 
 \hdashline 
     &           &    1 &  1.13e-09 &  1.17e-10 &      \\ 
     &           &    2 &  1.13e-09 &  3.23e-17 &      \\ 
     &           &    3 &  1.13e-09 &  3.23e-17 &      \\ 
     &           &    4 &  1.13e-09 &  3.22e-17 &      \\ 
     &           &    5 &  1.13e-09 &  3.22e-17 &      \\ 
     &           &    6 &  1.12e-09 &  3.21e-17 &      \\ 
     &           &    7 &  1.12e-09 &  3.21e-17 &      \\ 
     &           &    8 &  1.12e-09 &  3.21e-17 &      \\ 
     &           &    9 &  1.12e-09 &  3.20e-17 &      \\ 
     &           &   10 &  1.12e-09 &  3.20e-17 &      \\ 
1864 &  1.86e+04 &   10 &           &           & iters  \\ 
 \hdashline 
     &           &    1 &  4.78e-08 &  1.10e-09 &      \\ 
     &           &    2 &  3.00e-08 &  1.36e-15 &      \\ 
     &           &    3 &  2.99e-08 &  8.56e-16 &      \\ 
     &           &    4 &  4.78e-08 &  8.55e-16 &      \\ 
     &           &    5 &  3.00e-08 &  1.36e-15 &      \\ 
     &           &    6 &  4.78e-08 &  8.55e-16 &      \\ 
     &           &    7 &  3.00e-08 &  1.36e-15 &      \\ 
     &           &    8 &  3.00e-08 &  8.56e-16 &      \\ 
     &           &    9 &  4.78e-08 &  8.55e-16 &      \\ 
     &           &   10 &  3.00e-08 &  1.36e-15 &      \\ 
1865 &  1.86e+04 &   10 &           &           & iters  \\ 
 \hdashline 
     &           &    1 &  5.82e-08 &  1.07e-09 &      \\ 
     &           &    2 &  1.96e-08 &  1.66e-15 &      \\ 
     &           &    3 &  1.95e-08 &  5.59e-16 &      \\ 
     &           &    4 &  1.95e-08 &  5.58e-16 &      \\ 
     &           &    5 &  5.82e-08 &  5.57e-16 &      \\ 
     &           &    6 &  1.96e-08 &  1.66e-15 &      \\ 
     &           &    7 &  1.95e-08 &  5.59e-16 &      \\ 
     &           &    8 &  1.95e-08 &  5.58e-16 &      \\ 
     &           &    9 &  5.82e-08 &  5.57e-16 &      \\ 
     &           &   10 &  1.96e-08 &  1.66e-15 &      \\ 
1866 &  1.86e+04 &   10 &           &           & iters  \\ 
 \hdashline 
     &           &    1 &  5.25e-09 &  2.54e-10 &      \\ 
     &           &    2 &  5.25e-09 &  1.50e-16 &      \\ 
     &           &    3 &  5.24e-09 &  1.50e-16 &      \\ 
     &           &    4 &  5.23e-09 &  1.50e-16 &      \\ 
     &           &    5 &  5.22e-09 &  1.49e-16 &      \\ 
     &           &    6 &  5.22e-09 &  1.49e-16 &      \\ 
     &           &    7 &  7.25e-08 &  1.49e-16 &      \\ 
     &           &    8 &  4.42e-08 &  2.07e-15 &      \\ 
     &           &    9 &  5.25e-09 &  1.26e-15 &      \\ 
     &           &   10 &  5.24e-09 &  1.50e-16 &      \\ 
1867 &  1.87e+04 &   10 &           &           & iters  \\ 
 \hdashline 
     &           &    1 &  1.36e-08 &  1.83e-10 &      \\ 
     &           &    2 &  1.36e-08 &  3.88e-16 &      \\ 
     &           &    3 &  1.36e-08 &  3.87e-16 &      \\ 
     &           &    4 &  1.35e-08 &  3.87e-16 &      \\ 
     &           &    5 &  6.42e-08 &  3.86e-16 &      \\ 
     &           &    6 &  1.36e-08 &  1.83e-15 &      \\ 
     &           &    7 &  1.36e-08 &  3.88e-16 &      \\ 
     &           &    8 &  1.36e-08 &  3.88e-16 &      \\ 
     &           &    9 &  1.36e-08 &  3.87e-16 &      \\ 
     &           &   10 &  1.35e-08 &  3.87e-16 &      \\ 
1868 &  1.87e+04 &   10 &           &           & iters  \\ 
 \hdashline 
     &           &    1 &  6.87e-08 &  2.78e-10 &      \\ 
     &           &    2 &  9.10e-09 &  1.96e-15 &      \\ 
     &           &    3 &  9.09e-09 &  2.60e-16 &      \\ 
     &           &    4 &  9.08e-09 &  2.59e-16 &      \\ 
     &           &    5 &  9.06e-09 &  2.59e-16 &      \\ 
     &           &    6 &  9.05e-09 &  2.59e-16 &      \\ 
     &           &    7 &  9.04e-09 &  2.58e-16 &      \\ 
     &           &    8 &  9.02e-09 &  2.58e-16 &      \\ 
     &           &    9 &  6.87e-08 &  2.58e-16 &      \\ 
     &           &   10 &  8.68e-08 &  1.96e-15 &      \\ 
1869 &  1.87e+04 &   10 &           &           & iters  \\ 
 \hdashline 
     &           &    1 &  5.26e-08 &  1.55e-10 &      \\ 
     &           &    2 &  1.37e-08 &  1.50e-15 &      \\ 
     &           &    3 &  1.36e-08 &  3.90e-16 &      \\ 
     &           &    4 &  6.41e-08 &  3.89e-16 &      \\ 
     &           &    5 &  1.37e-08 &  1.83e-15 &      \\ 
     &           &    6 &  1.37e-08 &  3.91e-16 &      \\ 
     &           &    7 &  1.37e-08 &  3.91e-16 &      \\ 
     &           &    8 &  1.37e-08 &  3.90e-16 &      \\ 
     &           &    9 &  6.41e-08 &  3.90e-16 &      \\ 
     &           &   10 &  1.37e-08 &  1.83e-15 &      \\ 
1870 &  1.87e+04 &   10 &           &           & iters  \\ 
 \hdashline 
     &           &    1 &  5.33e-09 &  7.14e-10 &      \\ 
     &           &    2 &  5.32e-09 &  1.52e-16 &      \\ 
     &           &    3 &  5.32e-09 &  1.52e-16 &      \\ 
     &           &    4 &  5.31e-09 &  1.52e-16 &      \\ 
     &           &    5 &  5.30e-09 &  1.52e-16 &      \\ 
     &           &    6 &  5.29e-09 &  1.51e-16 &      \\ 
     &           &    7 &  5.29e-09 &  1.51e-16 &      \\ 
     &           &    8 &  5.28e-09 &  1.51e-16 &      \\ 
     &           &    9 &  5.27e-09 &  1.51e-16 &      \\ 
     &           &   10 &  5.26e-09 &  1.50e-16 &      \\ 
1871 &  1.87e+04 &   10 &           &           & iters  \\ 
 \hdashline 
     &           &    1 &  4.81e-08 &  1.11e-09 &      \\ 
     &           &    2 &  2.96e-08 &  1.37e-15 &      \\ 
     &           &    3 &  4.81e-08 &  8.46e-16 &      \\ 
     &           &    4 &  2.97e-08 &  1.37e-15 &      \\ 
     &           &    5 &  2.96e-08 &  8.46e-16 &      \\ 
     &           &    6 &  4.81e-08 &  8.45e-16 &      \\ 
     &           &    7 &  2.96e-08 &  1.37e-15 &      \\ 
     &           &    8 &  2.96e-08 &  8.46e-16 &      \\ 
     &           &    9 &  4.81e-08 &  8.45e-16 &      \\ 
     &           &   10 &  2.96e-08 &  1.37e-15 &      \\ 
1872 &  1.87e+04 &   10 &           &           & iters  \\ 
 \hdashline 
     &           &    1 &  2.48e-08 &  9.97e-10 &      \\ 
     &           &    2 &  2.48e-08 &  7.07e-16 &      \\ 
     &           &    3 &  5.30e-08 &  7.06e-16 &      \\ 
     &           &    4 &  2.48e-08 &  1.51e-15 &      \\ 
     &           &    5 &  2.48e-08 &  7.08e-16 &      \\ 
     &           &    6 &  5.30e-08 &  7.07e-16 &      \\ 
     &           &    7 &  2.48e-08 &  1.51e-15 &      \\ 
     &           &    8 &  2.48e-08 &  7.08e-16 &      \\ 
     &           &    9 &  5.30e-08 &  7.07e-16 &      \\ 
     &           &   10 &  2.48e-08 &  1.51e-15 &      \\ 
1873 &  1.87e+04 &   10 &           &           & iters  \\ 
 \hdashline 
     &           &    1 &  2.47e-09 &  1.13e-09 &      \\ 
     &           &    2 &  2.47e-09 &  7.06e-17 &      \\ 
     &           &    3 &  2.47e-09 &  7.05e-17 &      \\ 
     &           &    4 &  2.46e-09 &  7.04e-17 &      \\ 
     &           &    5 &  2.46e-09 &  7.03e-17 &      \\ 
     &           &    6 &  3.64e-08 &  7.02e-17 &      \\ 
     &           &    7 &  2.51e-09 &  1.04e-15 &      \\ 
     &           &    8 &  2.50e-09 &  7.16e-17 &      \\ 
     &           &    9 &  2.50e-09 &  7.15e-17 &      \\ 
     &           &   10 &  2.50e-09 &  7.13e-17 &      \\ 
1874 &  1.87e+04 &   10 &           &           & iters  \\ 
 \hdashline 
     &           &    1 &  3.34e-08 &  1.03e-09 &      \\ 
     &           &    2 &  5.49e-09 &  9.53e-16 &      \\ 
     &           &    3 &  5.48e-09 &  1.57e-16 &      \\ 
     &           &    4 &  5.47e-09 &  1.56e-16 &      \\ 
     &           &    5 &  5.46e-09 &  1.56e-16 &      \\ 
     &           &    6 &  5.45e-09 &  1.56e-16 &      \\ 
     &           &    7 &  5.45e-09 &  1.56e-16 &      \\ 
     &           &    8 &  3.34e-08 &  1.55e-16 &      \\ 
     &           &    9 &  5.49e-09 &  9.53e-16 &      \\ 
     &           &   10 &  5.48e-09 &  1.57e-16 &      \\ 
1875 &  1.87e+04 &   10 &           &           & iters  \\ 
 \hdashline 
     &           &    1 &  1.98e-08 &  1.61e-10 &      \\ 
     &           &    2 &  1.98e-08 &  5.66e-16 &      \\ 
     &           &    3 &  5.79e-08 &  5.65e-16 &      \\ 
     &           &    4 &  1.98e-08 &  1.65e-15 &      \\ 
     &           &    5 &  1.98e-08 &  5.66e-16 &      \\ 
     &           &    6 &  1.98e-08 &  5.66e-16 &      \\ 
     &           &    7 &  5.79e-08 &  5.65e-16 &      \\ 
     &           &    8 &  1.98e-08 &  1.65e-15 &      \\ 
     &           &    9 &  1.98e-08 &  5.66e-16 &      \\ 
     &           &   10 &  1.98e-08 &  5.65e-16 &      \\ 
1876 &  1.88e+04 &   10 &           &           & iters  \\ 
 \hdashline 
     &           &    1 &  2.00e-08 &  1.71e-09 &      \\ 
     &           &    2 &  1.99e-08 &  5.70e-16 &      \\ 
     &           &    3 &  1.99e-08 &  5.69e-16 &      \\ 
     &           &    4 &  5.78e-08 &  5.68e-16 &      \\ 
     &           &    5 &  2.00e-08 &  1.65e-15 &      \\ 
     &           &    6 &  1.99e-08 &  5.70e-16 &      \\ 
     &           &    7 &  1.99e-08 &  5.69e-16 &      \\ 
     &           &    8 &  5.78e-08 &  5.68e-16 &      \\ 
     &           &    9 &  2.00e-08 &  1.65e-15 &      \\ 
     &           &   10 &  1.99e-08 &  5.70e-16 &      \\ 
1877 &  1.88e+04 &   10 &           &           & iters  \\ 
 \hdashline 
     &           &    1 &  1.32e-08 &  1.82e-09 &      \\ 
     &           &    2 &  1.32e-08 &  3.78e-16 &      \\ 
     &           &    3 &  6.45e-08 &  3.78e-16 &      \\ 
     &           &    4 &  1.33e-08 &  1.84e-15 &      \\ 
     &           &    5 &  1.33e-08 &  3.80e-16 &      \\ 
     &           &    6 &  1.33e-08 &  3.79e-16 &      \\ 
     &           &    7 &  1.32e-08 &  3.79e-16 &      \\ 
     &           &    8 &  1.32e-08 &  3.78e-16 &      \\ 
     &           &    9 &  6.45e-08 &  3.78e-16 &      \\ 
     &           &   10 &  1.33e-08 &  1.84e-15 &      \\ 
1878 &  1.88e+04 &   10 &           &           & iters  \\ 
 \hdashline 
     &           &    1 &  5.29e-09 &  9.33e-10 &      \\ 
     &           &    2 &  5.28e-09 &  1.51e-16 &      \\ 
     &           &    3 &  5.28e-09 &  1.51e-16 &      \\ 
     &           &    4 &  7.24e-08 &  1.51e-16 &      \\ 
     &           &    5 &  8.31e-08 &  2.07e-15 &      \\ 
     &           &    6 &  7.24e-08 &  2.37e-15 &      \\ 
     &           &    7 &  5.36e-09 &  2.07e-15 &      \\ 
     &           &    8 &  5.35e-09 &  1.53e-16 &      \\ 
     &           &    9 &  5.34e-09 &  1.53e-16 &      \\ 
     &           &   10 &  5.33e-09 &  1.52e-16 &      \\ 
1879 &  1.88e+04 &   10 &           &           & iters  \\ 
 \hdashline 
     &           &    1 &  7.29e-09 &  1.16e-09 &      \\ 
     &           &    2 &  7.28e-09 &  2.08e-16 &      \\ 
     &           &    3 &  7.27e-09 &  2.08e-16 &      \\ 
     &           &    4 &  7.04e-08 &  2.07e-16 &      \\ 
     &           &    5 &  4.62e-08 &  2.01e-15 &      \\ 
     &           &    6 &  7.29e-09 &  1.32e-15 &      \\ 
     &           &    7 &  7.28e-09 &  2.08e-16 &      \\ 
     &           &    8 &  7.27e-09 &  2.08e-16 &      \\ 
     &           &    9 &  7.04e-08 &  2.08e-16 &      \\ 
     &           &   10 &  4.62e-08 &  2.01e-15 &      \\ 
1880 &  1.88e+04 &   10 &           &           & iters  \\ 
 \hdashline 
     &           &    1 &  3.85e-08 &  2.58e-09 &      \\ 
     &           &    2 &  3.93e-08 &  1.10e-15 &      \\ 
     &           &    3 &  3.85e-08 &  1.12e-15 &      \\ 
     &           &    4 &  3.93e-08 &  1.10e-15 &      \\ 
     &           &    5 &  3.85e-08 &  1.12e-15 &      \\ 
     &           &    6 &  3.93e-08 &  1.10e-15 &      \\ 
     &           &    7 &  3.85e-08 &  1.12e-15 &      \\ 
     &           &    8 &  3.93e-08 &  1.10e-15 &      \\ 
     &           &    9 &  3.85e-08 &  1.12e-15 &      \\ 
     &           &   10 &  3.93e-08 &  1.10e-15 &      \\ 
1881 &  1.88e+04 &   10 &           &           & iters  \\ 
 \hdashline 
     &           &    1 &  6.74e-08 &  3.62e-09 &      \\ 
     &           &    2 &  1.04e-08 &  1.92e-15 &      \\ 
     &           &    3 &  1.04e-08 &  2.97e-16 &      \\ 
     &           &    4 &  1.04e-08 &  2.97e-16 &      \\ 
     &           &    5 &  1.04e-08 &  2.96e-16 &      \\ 
     &           &    6 &  1.04e-08 &  2.96e-16 &      \\ 
     &           &    7 &  1.03e-08 &  2.96e-16 &      \\ 
     &           &    8 &  6.74e-08 &  2.95e-16 &      \\ 
     &           &    9 &  8.81e-08 &  1.92e-15 &      \\ 
     &           &   10 &  6.74e-08 &  2.51e-15 &      \\ 
1882 &  1.88e+04 &   10 &           &           & iters  \\ 
 \hdashline 
     &           &    1 &  3.95e-08 &  1.97e-09 &      \\ 
     &           &    2 &  3.82e-08 &  1.13e-15 &      \\ 
     &           &    3 &  3.95e-08 &  1.09e-15 &      \\ 
     &           &    4 &  3.82e-08 &  1.13e-15 &      \\ 
     &           &    5 &  3.95e-08 &  1.09e-15 &      \\ 
     &           &    6 &  3.82e-08 &  1.13e-15 &      \\ 
     &           &    7 &  3.95e-08 &  1.09e-15 &      \\ 
     &           &    8 &  3.82e-08 &  1.13e-15 &      \\ 
     &           &    9 &  3.95e-08 &  1.09e-15 &      \\ 
     &           &   10 &  3.82e-08 &  1.13e-15 &      \\ 
1883 &  1.88e+04 &   10 &           &           & iters  \\ 
 \hdashline 
     &           &    1 &  2.05e-08 &  7.01e-10 &      \\ 
     &           &    2 &  2.05e-08 &  5.86e-16 &      \\ 
     &           &    3 &  2.05e-08 &  5.85e-16 &      \\ 
     &           &    4 &  5.72e-08 &  5.85e-16 &      \\ 
     &           &    5 &  2.05e-08 &  1.63e-15 &      \\ 
     &           &    6 &  2.05e-08 &  5.86e-16 &      \\ 
     &           &    7 &  2.05e-08 &  5.85e-16 &      \\ 
     &           &    8 &  5.72e-08 &  5.84e-16 &      \\ 
     &           &    9 &  2.05e-08 &  1.63e-15 &      \\ 
     &           &   10 &  2.05e-08 &  5.86e-16 &      \\ 
1884 &  1.88e+04 &   10 &           &           & iters  \\ 
 \hdashline 
     &           &    1 &  4.30e-08 &  7.15e-10 &      \\ 
     &           &    2 &  3.48e-08 &  1.23e-15 &      \\ 
     &           &    3 &  3.47e-08 &  9.92e-16 &      \\ 
     &           &    4 &  4.30e-08 &  9.91e-16 &      \\ 
     &           &    5 &  3.47e-08 &  1.23e-15 &      \\ 
     &           &    6 &  4.30e-08 &  9.91e-16 &      \\ 
     &           &    7 &  3.47e-08 &  1.23e-15 &      \\ 
     &           &    8 &  4.30e-08 &  9.91e-16 &      \\ 
     &           &    9 &  3.47e-08 &  1.23e-15 &      \\ 
     &           &   10 &  4.30e-08 &  9.92e-16 &      \\ 
1885 &  1.88e+04 &   10 &           &           & iters  \\ 
 \hdashline 
     &           &    1 &  2.29e-08 &  4.56e-10 &      \\ 
     &           &    2 &  2.28e-08 &  6.53e-16 &      \\ 
     &           &    3 &  2.28e-08 &  6.52e-16 &      \\ 
     &           &    4 &  5.49e-08 &  6.51e-16 &      \\ 
     &           &    5 &  2.28e-08 &  1.57e-15 &      \\ 
     &           &    6 &  2.28e-08 &  6.52e-16 &      \\ 
     &           &    7 &  5.49e-08 &  6.51e-16 &      \\ 
     &           &    8 &  2.29e-08 &  1.57e-15 &      \\ 
     &           &    9 &  2.28e-08 &  6.52e-16 &      \\ 
     &           &   10 &  2.28e-08 &  6.51e-16 &      \\ 
1886 &  1.88e+04 &   10 &           &           & iters  \\ 
 \hdashline 
     &           &    1 &  1.19e-08 &  1.05e-09 &      \\ 
     &           &    2 &  1.19e-08 &  3.39e-16 &      \\ 
     &           &    3 &  1.18e-08 &  3.38e-16 &      \\ 
     &           &    4 &  6.59e-08 &  3.38e-16 &      \\ 
     &           &    5 &  8.96e-08 &  1.88e-15 &      \\ 
     &           &    6 &  6.59e-08 &  2.56e-15 &      \\ 
     &           &    7 &  1.19e-08 &  1.88e-15 &      \\ 
     &           &    8 &  1.19e-08 &  3.39e-16 &      \\ 
     &           &    9 &  1.18e-08 &  3.39e-16 &      \\ 
     &           &   10 &  1.18e-08 &  3.38e-16 &      \\ 
1887 &  1.89e+04 &   10 &           &           & iters  \\ 
 \hdashline 
     &           &    1 &  7.78e-08 &  6.47e-10 &      \\ 
     &           &    2 &  7.76e-08 &  2.22e-15 &      \\ 
     &           &    3 &  7.78e-08 &  2.22e-15 &      \\ 
     &           &    4 &  7.76e-08 &  2.22e-15 &      \\ 
     &           &    5 &  7.78e-08 &  2.22e-15 &      \\ 
     &           &    6 &  7.76e-08 &  2.22e-15 &      \\ 
     &           &    7 &  7.78e-08 &  2.22e-15 &      \\ 
     &           &    8 &  7.76e-08 &  2.22e-15 &      \\ 
     &           &    9 &  7.78e-08 &  2.22e-15 &      \\ 
     &           &   10 &  7.76e-08 &  2.22e-15 &      \\ 
1888 &  1.89e+04 &   10 &           &           & iters  \\ 
 \hdashline 
     &           &    1 &  1.68e-08 &  4.22e-10 &      \\ 
     &           &    2 &  1.68e-08 &  4.80e-16 &      \\ 
     &           &    3 &  6.09e-08 &  4.79e-16 &      \\ 
     &           &    4 &  1.68e-08 &  1.74e-15 &      \\ 
     &           &    5 &  1.68e-08 &  4.81e-16 &      \\ 
     &           &    6 &  1.68e-08 &  4.80e-16 &      \\ 
     &           &    7 &  1.68e-08 &  4.79e-16 &      \\ 
     &           &    8 &  6.09e-08 &  4.79e-16 &      \\ 
     &           &    9 &  1.68e-08 &  1.74e-15 &      \\ 
     &           &   10 &  1.68e-08 &  4.80e-16 &      \\ 
1889 &  1.89e+04 &   10 &           &           & iters  \\ 
 \hdashline 
     &           &    1 &  8.21e-10 &  9.30e-10 &      \\ 
     &           &    2 &  8.20e-10 &  2.34e-17 &      \\ 
     &           &    3 &  8.18e-10 &  2.34e-17 &      \\ 
     &           &    4 &  8.17e-10 &  2.34e-17 &      \\ 
     &           &    5 &  8.16e-10 &  2.33e-17 &      \\ 
     &           &    6 &  8.15e-10 &  2.33e-17 &      \\ 
     &           &    7 &  8.14e-10 &  2.33e-17 &      \\ 
     &           &    8 &  8.13e-10 &  2.32e-17 &      \\ 
     &           &    9 &  8.11e-10 &  2.32e-17 &      \\ 
     &           &   10 &  8.10e-10 &  2.32e-17 &      \\ 
1890 &  1.89e+04 &   10 &           &           & iters  \\ 
 \hdashline 
     &           &    1 &  2.96e-08 &  8.90e-10 &      \\ 
     &           &    2 &  4.82e-08 &  8.44e-16 &      \\ 
     &           &    3 &  2.96e-08 &  1.37e-15 &      \\ 
     &           &    4 &  4.81e-08 &  8.45e-16 &      \\ 
     &           &    5 &  2.96e-08 &  1.37e-15 &      \\ 
     &           &    6 &  2.96e-08 &  8.46e-16 &      \\ 
     &           &    7 &  4.81e-08 &  8.45e-16 &      \\ 
     &           &    8 &  2.96e-08 &  1.37e-15 &      \\ 
     &           &    9 &  2.96e-08 &  8.45e-16 &      \\ 
     &           &   10 &  4.82e-08 &  8.44e-16 &      \\ 
1891 &  1.89e+04 &   10 &           &           & iters  \\ 
 \hdashline 
     &           &    1 &  3.08e-08 &  5.29e-10 &      \\ 
     &           &    2 &  4.70e-08 &  8.78e-16 &      \\ 
     &           &    3 &  3.08e-08 &  1.34e-15 &      \\ 
     &           &    4 &  4.69e-08 &  8.79e-16 &      \\ 
     &           &    5 &  3.08e-08 &  1.34e-15 &      \\ 
     &           &    6 &  3.08e-08 &  8.80e-16 &      \\ 
     &           &    7 &  4.70e-08 &  8.78e-16 &      \\ 
     &           &    8 &  3.08e-08 &  1.34e-15 &      \\ 
     &           &    9 &  4.69e-08 &  8.79e-16 &      \\ 
     &           &   10 &  3.08e-08 &  1.34e-15 &      \\ 
1892 &  1.89e+04 &   10 &           &           & iters  \\ 
 \hdashline 
     &           &    1 &  4.02e-08 &  1.68e-09 &      \\ 
     &           &    2 &  3.76e-08 &  1.15e-15 &      \\ 
     &           &    3 &  4.02e-08 &  1.07e-15 &      \\ 
     &           &    4 &  3.76e-08 &  1.15e-15 &      \\ 
     &           &    5 &  4.02e-08 &  1.07e-15 &      \\ 
     &           &    6 &  3.76e-08 &  1.15e-15 &      \\ 
     &           &    7 &  3.75e-08 &  1.07e-15 &      \\ 
     &           &    8 &  4.02e-08 &  1.07e-15 &      \\ 
     &           &    9 &  3.75e-08 &  1.15e-15 &      \\ 
     &           &   10 &  4.02e-08 &  1.07e-15 &      \\ 
1893 &  1.89e+04 &   10 &           &           & iters  \\ 
 \hdashline 
     &           &    1 &  3.11e-08 &  2.86e-09 &      \\ 
     &           &    2 &  3.10e-08 &  8.87e-16 &      \\ 
     &           &    3 &  4.67e-08 &  8.86e-16 &      \\ 
     &           &    4 &  3.11e-08 &  1.33e-15 &      \\ 
     &           &    5 &  3.10e-08 &  8.86e-16 &      \\ 
     &           &    6 &  4.67e-08 &  8.85e-16 &      \\ 
     &           &    7 &  3.10e-08 &  1.33e-15 &      \\ 
     &           &    8 &  4.67e-08 &  8.86e-16 &      \\ 
     &           &    9 &  3.11e-08 &  1.33e-15 &      \\ 
     &           &   10 &  3.10e-08 &  8.86e-16 &      \\ 
1894 &  1.89e+04 &   10 &           &           & iters  \\ 
 \hdashline 
     &           &    1 &  5.58e-08 &  3.48e-09 &      \\ 
     &           &    2 &  2.19e-08 &  1.59e-15 &      \\ 
     &           &    3 &  5.58e-08 &  6.26e-16 &      \\ 
     &           &    4 &  2.20e-08 &  1.59e-15 &      \\ 
     &           &    5 &  2.20e-08 &  6.28e-16 &      \\ 
     &           &    6 &  5.58e-08 &  6.27e-16 &      \\ 
     &           &    7 &  2.20e-08 &  1.59e-15 &      \\ 
     &           &    8 &  2.20e-08 &  6.28e-16 &      \\ 
     &           &    9 &  2.19e-08 &  6.27e-16 &      \\ 
     &           &   10 &  5.58e-08 &  6.26e-16 &      \\ 
1895 &  1.89e+04 &   10 &           &           & iters  \\ 
 \hdashline 
     &           &    1 &  5.80e-09 &  8.81e-10 &      \\ 
     &           &    2 &  5.79e-09 &  1.65e-16 &      \\ 
     &           &    3 &  5.78e-09 &  1.65e-16 &      \\ 
     &           &    4 &  5.77e-09 &  1.65e-16 &      \\ 
     &           &    5 &  5.76e-09 &  1.65e-16 &      \\ 
     &           &    6 &  5.76e-09 &  1.64e-16 &      \\ 
     &           &    7 &  5.75e-09 &  1.64e-16 &      \\ 
     &           &    8 &  5.74e-09 &  1.64e-16 &      \\ 
     &           &    9 &  5.73e-09 &  1.64e-16 &      \\ 
     &           &   10 &  5.72e-09 &  1.64e-16 &      \\ 
1896 &  1.90e+04 &   10 &           &           & iters  \\ 
 \hdashline 
     &           &    1 &  3.57e-08 &  1.94e-10 &      \\ 
     &           &    2 &  4.20e-08 &  1.02e-15 &      \\ 
     &           &    3 &  3.58e-08 &  1.20e-15 &      \\ 
     &           &    4 &  4.20e-08 &  1.02e-15 &      \\ 
     &           &    5 &  3.58e-08 &  1.20e-15 &      \\ 
     &           &    6 &  3.57e-08 &  1.02e-15 &      \\ 
     &           &    7 &  4.20e-08 &  1.02e-15 &      \\ 
     &           &    8 &  3.57e-08 &  1.20e-15 &      \\ 
     &           &    9 &  4.20e-08 &  1.02e-15 &      \\ 
     &           &   10 &  3.57e-08 &  1.20e-15 &      \\ 
1897 &  1.90e+04 &   10 &           &           & iters  \\ 
 \hdashline 
     &           &    1 &  2.52e-08 &  3.53e-11 &      \\ 
     &           &    2 &  5.26e-08 &  7.18e-16 &      \\ 
     &           &    3 &  2.52e-08 &  1.50e-15 &      \\ 
     &           &    4 &  2.52e-08 &  7.19e-16 &      \\ 
     &           &    5 &  5.26e-08 &  7.18e-16 &      \\ 
     &           &    6 &  2.52e-08 &  1.50e-15 &      \\ 
     &           &    7 &  2.52e-08 &  7.19e-16 &      \\ 
     &           &    8 &  2.51e-08 &  7.18e-16 &      \\ 
     &           &    9 &  5.26e-08 &  7.17e-16 &      \\ 
     &           &   10 &  2.52e-08 &  1.50e-15 &      \\ 
1898 &  1.90e+04 &   10 &           &           & iters  \\ 
 \hdashline 
     &           &    1 &  2.75e-08 &  1.23e-09 &      \\ 
     &           &    2 &  2.75e-08 &  7.86e-16 &      \\ 
     &           &    3 &  5.02e-08 &  7.85e-16 &      \\ 
     &           &    4 &  2.75e-08 &  1.43e-15 &      \\ 
     &           &    5 &  2.75e-08 &  7.86e-16 &      \\ 
     &           &    6 &  5.02e-08 &  7.84e-16 &      \\ 
     &           &    7 &  2.75e-08 &  1.43e-15 &      \\ 
     &           &    8 &  2.75e-08 &  7.85e-16 &      \\ 
     &           &    9 &  5.02e-08 &  7.84e-16 &      \\ 
     &           &   10 &  2.75e-08 &  1.43e-15 &      \\ 
1899 &  1.90e+04 &   10 &           &           & iters  \\ 
 \hdashline 
     &           &    1 &  4.59e-08 &  8.34e-10 &      \\ 
     &           &    2 &  3.18e-08 &  1.31e-15 &      \\ 
     &           &    3 &  3.18e-08 &  9.08e-16 &      \\ 
     &           &    4 &  4.60e-08 &  9.06e-16 &      \\ 
     &           &    5 &  3.18e-08 &  1.31e-15 &      \\ 
     &           &    6 &  4.60e-08 &  9.07e-16 &      \\ 
     &           &    7 &  3.18e-08 &  1.31e-15 &      \\ 
     &           &    8 &  4.59e-08 &  9.08e-16 &      \\ 
     &           &    9 &  3.18e-08 &  1.31e-15 &      \\ 
     &           &   10 &  3.18e-08 &  9.08e-16 &      \\ 
1900 &  1.90e+04 &   10 &           &           & iters  \\ 
 \hdashline 
     &           &    1 &  3.57e-08 &  4.47e-10 &      \\ 
     &           &    2 &  4.21e-08 &  1.02e-15 &      \\ 
     &           &    3 &  3.57e-08 &  1.20e-15 &      \\ 
     &           &    4 &  3.56e-08 &  1.02e-15 &      \\ 
     &           &    5 &  4.21e-08 &  1.02e-15 &      \\ 
     &           &    6 &  3.56e-08 &  1.20e-15 &      \\ 
     &           &    7 &  4.21e-08 &  1.02e-15 &      \\ 
     &           &    8 &  3.57e-08 &  1.20e-15 &      \\ 
     &           &    9 &  4.21e-08 &  1.02e-15 &      \\ 
     &           &   10 &  3.57e-08 &  1.20e-15 &      \\ 
1901 &  1.90e+04 &   10 &           &           & iters  \\ 
 \hdashline 
     &           &    1 &  1.16e-08 &  4.12e-10 &      \\ 
     &           &    2 &  1.16e-08 &  3.30e-16 &      \\ 
     &           &    3 &  1.15e-08 &  3.30e-16 &      \\ 
     &           &    4 &  6.62e-08 &  3.29e-16 &      \\ 
     &           &    5 &  1.16e-08 &  1.89e-15 &      \\ 
     &           &    6 &  1.16e-08 &  3.32e-16 &      \\ 
     &           &    7 &  1.16e-08 &  3.31e-16 &      \\ 
     &           &    8 &  1.16e-08 &  3.31e-16 &      \\ 
     &           &    9 &  1.16e-08 &  3.30e-16 &      \\ 
     &           &   10 &  6.61e-08 &  3.30e-16 &      \\ 
1902 &  1.90e+04 &   10 &           &           & iters  \\ 
 \hdashline 
     &           &    1 &  3.27e-08 &  9.34e-10 &      \\ 
     &           &    2 &  3.27e-08 &  9.34e-16 &      \\ 
     &           &    3 &  4.51e-08 &  9.32e-16 &      \\ 
     &           &    4 &  3.27e-08 &  1.29e-15 &      \\ 
     &           &    5 &  3.26e-08 &  9.33e-16 &      \\ 
     &           &    6 &  4.51e-08 &  9.31e-16 &      \\ 
     &           &    7 &  3.27e-08 &  1.29e-15 &      \\ 
     &           &    8 &  4.51e-08 &  9.32e-16 &      \\ 
     &           &    9 &  3.27e-08 &  1.29e-15 &      \\ 
     &           &   10 &  4.51e-08 &  9.32e-16 &      \\ 
1903 &  1.90e+04 &   10 &           &           & iters  \\ 
 \hdashline 
     &           &    1 &  2.72e-08 &  2.64e-09 &      \\ 
     &           &    2 &  5.05e-08 &  7.77e-16 &      \\ 
     &           &    3 &  2.73e-08 &  1.44e-15 &      \\ 
     &           &    4 &  2.72e-08 &  7.78e-16 &      \\ 
     &           &    5 &  5.05e-08 &  7.77e-16 &      \\ 
     &           &    6 &  2.73e-08 &  1.44e-15 &      \\ 
     &           &    7 &  2.72e-08 &  7.78e-16 &      \\ 
     &           &    8 &  5.05e-08 &  7.77e-16 &      \\ 
     &           &    9 &  2.73e-08 &  1.44e-15 &      \\ 
     &           &   10 &  2.72e-08 &  7.78e-16 &      \\ 
1904 &  1.90e+04 &   10 &           &           & iters  \\ 
 \hdashline 
     &           &    1 &  2.73e-08 &  3.41e-09 &      \\ 
     &           &    2 &  2.73e-08 &  7.80e-16 &      \\ 
     &           &    3 &  5.04e-08 &  7.79e-16 &      \\ 
     &           &    4 &  2.73e-08 &  1.44e-15 &      \\ 
     &           &    5 &  2.73e-08 &  7.80e-16 &      \\ 
     &           &    6 &  5.04e-08 &  7.79e-16 &      \\ 
     &           &    7 &  2.73e-08 &  1.44e-15 &      \\ 
     &           &    8 &  2.73e-08 &  7.80e-16 &      \\ 
     &           &    9 &  5.04e-08 &  7.79e-16 &      \\ 
     &           &   10 &  2.73e-08 &  1.44e-15 &      \\ 
1905 &  1.90e+04 &   10 &           &           & iters  \\ 
 \hdashline 
     &           &    1 &  2.28e-08 &  4.21e-09 &      \\ 
     &           &    2 &  1.61e-08 &  6.50e-16 &      \\ 
     &           &    3 &  1.61e-08 &  4.60e-16 &      \\ 
     &           &    4 &  2.28e-08 &  4.59e-16 &      \\ 
     &           &    5 &  1.61e-08 &  6.50e-16 &      \\ 
     &           &    6 &  2.28e-08 &  4.59e-16 &      \\ 
     &           &    7 &  1.61e-08 &  6.50e-16 &      \\ 
     &           &    8 &  2.28e-08 &  4.59e-16 &      \\ 
     &           &    9 &  1.61e-08 &  6.50e-16 &      \\ 
     &           &   10 &  1.61e-08 &  4.60e-16 &      \\ 
1906 &  1.90e+04 &   10 &           &           & iters  \\ 
 \hdashline 
     &           &    1 &  3.45e-08 &  4.81e-09 &      \\ 
     &           &    2 &  4.36e-09 &  9.86e-16 &      \\ 
     &           &    3 &  4.35e-09 &  1.24e-16 &      \\ 
     &           &    4 &  4.35e-09 &  1.24e-16 &      \\ 
     &           &    5 &  4.34e-09 &  1.24e-16 &      \\ 
     &           &    6 &  4.33e-09 &  1.24e-16 &      \\ 
     &           &    7 &  4.33e-09 &  1.24e-16 &      \\ 
     &           &    8 &  4.32e-09 &  1.23e-16 &      \\ 
     &           &    9 &  3.45e-08 &  1.23e-16 &      \\ 
     &           &   10 &  4.36e-09 &  9.86e-16 &      \\ 
1907 &  1.91e+04 &   10 &           &           & iters  \\ 
 \hdashline 
     &           &    1 &  7.48e-09 &  2.70e-09 &      \\ 
     &           &    2 &  7.47e-09 &  2.13e-16 &      \\ 
     &           &    3 &  7.46e-09 &  2.13e-16 &      \\ 
     &           &    4 &  7.45e-09 &  2.13e-16 &      \\ 
     &           &    5 &  7.44e-09 &  2.13e-16 &      \\ 
     &           &    6 &  3.14e-08 &  2.12e-16 &      \\ 
     &           &    7 &  7.47e-09 &  8.97e-16 &      \\ 
     &           &    8 &  7.46e-09 &  2.13e-16 &      \\ 
     &           &    9 &  7.45e-09 &  2.13e-16 &      \\ 
     &           &   10 &  7.44e-09 &  2.13e-16 &      \\ 
1908 &  1.91e+04 &   10 &           &           & iters  \\ 
 \hdashline 
     &           &    1 &  2.47e-08 &  1.79e-10 &      \\ 
     &           &    2 &  2.46e-08 &  7.04e-16 &      \\ 
     &           &    3 &  5.31e-08 &  7.03e-16 &      \\ 
     &           &    4 &  2.47e-08 &  1.51e-15 &      \\ 
     &           &    5 &  2.46e-08 &  7.04e-16 &      \\ 
     &           &    6 &  5.31e-08 &  7.03e-16 &      \\ 
     &           &    7 &  2.47e-08 &  1.51e-15 &      \\ 
     &           &    8 &  2.47e-08 &  7.05e-16 &      \\ 
     &           &    9 &  5.31e-08 &  7.04e-16 &      \\ 
     &           &   10 &  2.47e-08 &  1.51e-15 &      \\ 
1909 &  1.91e+04 &   10 &           &           & iters  \\ 
 \hdashline 
     &           &    1 &  5.72e-09 &  7.02e-10 &      \\ 
     &           &    2 &  5.71e-09 &  1.63e-16 &      \\ 
     &           &    3 &  5.70e-09 &  1.63e-16 &      \\ 
     &           &    4 &  5.69e-09 &  1.63e-16 &      \\ 
     &           &    5 &  5.68e-09 &  1.62e-16 &      \\ 
     &           &    6 &  7.20e-08 &  1.62e-16 &      \\ 
     &           &    7 &  4.46e-08 &  2.06e-15 &      \\ 
     &           &    8 &  5.71e-09 &  1.27e-15 &      \\ 
     &           &    9 &  5.71e-09 &  1.63e-16 &      \\ 
     &           &   10 &  5.70e-09 &  1.63e-16 &      \\ 
1910 &  1.91e+04 &   10 &           &           & iters  \\ 
 \hdashline 
     &           &    1 &  9.00e-09 &  2.52e-09 &      \\ 
     &           &    2 &  8.99e-09 &  2.57e-16 &      \\ 
     &           &    3 &  8.98e-09 &  2.57e-16 &      \\ 
     &           &    4 &  8.96e-09 &  2.56e-16 &      \\ 
     &           &    5 &  8.95e-09 &  2.56e-16 &      \\ 
     &           &    6 &  8.94e-09 &  2.55e-16 &      \\ 
     &           &    7 &  8.93e-09 &  2.55e-16 &      \\ 
     &           &    8 &  6.88e-08 &  2.55e-16 &      \\ 
     &           &    9 &  9.01e-09 &  1.96e-15 &      \\ 
     &           &   10 &  9.00e-09 &  2.57e-16 &      \\ 
1911 &  1.91e+04 &   10 &           &           & iters  \\ 
 \hdashline 
     &           &    1 &  1.82e-08 &  6.14e-10 &      \\ 
     &           &    2 &  1.81e-08 &  5.19e-16 &      \\ 
     &           &    3 &  1.81e-08 &  5.18e-16 &      \\ 
     &           &    4 &  5.96e-08 &  5.17e-16 &      \\ 
     &           &    5 &  1.82e-08 &  1.70e-15 &      \\ 
     &           &    6 &  1.82e-08 &  5.19e-16 &      \\ 
     &           &    7 &  1.81e-08 &  5.18e-16 &      \\ 
     &           &    8 &  5.96e-08 &  5.17e-16 &      \\ 
     &           &    9 &  1.82e-08 &  1.70e-15 &      \\ 
     &           &   10 &  1.82e-08 &  5.19e-16 &      \\ 
1912 &  1.91e+04 &   10 &           &           & iters  \\ 
 \hdashline 
     &           &    1 &  1.75e-08 &  5.76e-09 &      \\ 
     &           &    2 &  1.74e-08 &  4.99e-16 &      \\ 
     &           &    3 &  2.14e-08 &  4.98e-16 &      \\ 
     &           &    4 &  1.74e-08 &  6.12e-16 &      \\ 
     &           &    5 &  2.14e-08 &  4.98e-16 &      \\ 
     &           &    6 &  1.75e-08 &  6.11e-16 &      \\ 
     &           &    7 &  2.14e-08 &  4.98e-16 &      \\ 
     &           &    8 &  1.75e-08 &  6.11e-16 &      \\ 
     &           &    9 &  2.14e-08 &  4.98e-16 &      \\ 
     &           &   10 &  1.75e-08 &  6.11e-16 &      \\ 
1913 &  1.91e+04 &   10 &           &           & iters  \\ 
 \hdashline 
     &           &    1 &  1.09e-08 &  7.62e-09 &      \\ 
     &           &    2 &  1.09e-08 &  3.11e-16 &      \\ 
     &           &    3 &  1.09e-08 &  3.11e-16 &      \\ 
     &           &    4 &  6.68e-08 &  3.10e-16 &      \\ 
     &           &    5 &  1.10e-08 &  1.91e-15 &      \\ 
     &           &    6 &  1.09e-08 &  3.13e-16 &      \\ 
     &           &    7 &  1.09e-08 &  3.12e-16 &      \\ 
     &           &    8 &  1.09e-08 &  3.12e-16 &      \\ 
     &           &    9 &  1.09e-08 &  3.11e-16 &      \\ 
     &           &   10 &  1.09e-08 &  3.11e-16 &      \\ 
1914 &  1.91e+04 &   10 &           &           & iters  \\ 
 \hdashline 
     &           &    1 &  1.14e-08 &  6.53e-09 &      \\ 
     &           &    2 &  1.14e-08 &  3.26e-16 &      \\ 
     &           &    3 &  1.14e-08 &  3.26e-16 &      \\ 
     &           &    4 &  1.14e-08 &  3.25e-16 &      \\ 
     &           &    5 &  6.63e-08 &  3.25e-16 &      \\ 
     &           &    6 &  8.92e-08 &  1.89e-15 &      \\ 
     &           &    7 &  6.63e-08 &  2.54e-15 &      \\ 
     &           &    8 &  1.14e-08 &  1.89e-15 &      \\ 
     &           &    9 &  1.14e-08 &  3.26e-16 &      \\ 
     &           &   10 &  1.14e-08 &  3.26e-16 &      \\ 
1915 &  1.91e+04 &   10 &           &           & iters  \\ 
 \hdashline 
     &           &    1 &  2.68e-08 &  4.87e-09 &      \\ 
     &           &    2 &  2.68e-08 &  7.65e-16 &      \\ 
     &           &    3 &  5.10e-08 &  7.64e-16 &      \\ 
     &           &    4 &  2.68e-08 &  1.45e-15 &      \\ 
     &           &    5 &  2.67e-08 &  7.65e-16 &      \\ 
     &           &    6 &  5.10e-08 &  7.63e-16 &      \\ 
     &           &    7 &  2.68e-08 &  1.45e-15 &      \\ 
     &           &    8 &  2.67e-08 &  7.64e-16 &      \\ 
     &           &    9 &  5.10e-08 &  7.63e-16 &      \\ 
     &           &   10 &  2.68e-08 &  1.46e-15 &      \\ 
1916 &  1.92e+04 &   10 &           &           & iters  \\ 
 \hdashline 
     &           &    1 &  2.42e-08 &  2.12e-09 &      \\ 
     &           &    2 &  2.42e-08 &  6.91e-16 &      \\ 
     &           &    3 &  5.36e-08 &  6.90e-16 &      \\ 
     &           &    4 &  2.42e-08 &  1.53e-15 &      \\ 
     &           &    5 &  2.42e-08 &  6.91e-16 &      \\ 
     &           &    6 &  5.36e-08 &  6.90e-16 &      \\ 
     &           &    7 &  2.42e-08 &  1.53e-15 &      \\ 
     &           &    8 &  2.42e-08 &  6.91e-16 &      \\ 
     &           &    9 &  5.35e-08 &  6.90e-16 &      \\ 
     &           &   10 &  2.42e-08 &  1.53e-15 &      \\ 
1917 &  1.92e+04 &   10 &           &           & iters  \\ 
 \hdashline 
     &           &    1 &  3.84e-08 &  2.64e-10 &      \\ 
     &           &    2 &  3.94e-08 &  1.10e-15 &      \\ 
     &           &    3 &  3.84e-08 &  1.12e-15 &      \\ 
     &           &    4 &  3.94e-08 &  1.10e-15 &      \\ 
     &           &    5 &  3.84e-08 &  1.12e-15 &      \\ 
     &           &    6 &  3.94e-08 &  1.10e-15 &      \\ 
     &           &    7 &  3.84e-08 &  1.12e-15 &      \\ 
     &           &    8 &  3.94e-08 &  1.10e-15 &      \\ 
     &           &    9 &  3.84e-08 &  1.12e-15 &      \\ 
     &           &   10 &  3.94e-08 &  1.10e-15 &      \\ 
1918 &  1.92e+04 &   10 &           &           & iters  \\ 
 \hdashline 
     &           &    1 &  2.91e-08 &  3.87e-10 &      \\ 
     &           &    2 &  2.90e-08 &  8.30e-16 &      \\ 
     &           &    3 &  4.87e-08 &  8.29e-16 &      \\ 
     &           &    4 &  2.91e-08 &  1.39e-15 &      \\ 
     &           &    5 &  2.90e-08 &  8.30e-16 &      \\ 
     &           &    6 &  4.87e-08 &  8.28e-16 &      \\ 
     &           &    7 &  2.91e-08 &  1.39e-15 &      \\ 
     &           &    8 &  4.87e-08 &  8.29e-16 &      \\ 
     &           &    9 &  2.91e-08 &  1.39e-15 &      \\ 
     &           &   10 &  2.90e-08 &  8.30e-16 &      \\ 
1919 &  1.92e+04 &   10 &           &           & iters  \\ 
 \hdashline 
     &           &    1 &  2.99e-08 &  1.01e-09 &      \\ 
     &           &    2 &  2.99e-08 &  8.53e-16 &      \\ 
     &           &    3 &  4.79e-08 &  8.52e-16 &      \\ 
     &           &    4 &  2.99e-08 &  1.37e-15 &      \\ 
     &           &    5 &  4.78e-08 &  8.53e-16 &      \\ 
     &           &    6 &  2.99e-08 &  1.37e-15 &      \\ 
     &           &    7 &  2.99e-08 &  8.54e-16 &      \\ 
     &           &    8 &  4.79e-08 &  8.52e-16 &      \\ 
     &           &    9 &  2.99e-08 &  1.37e-15 &      \\ 
     &           &   10 &  4.78e-08 &  8.53e-16 &      \\ 
1920 &  1.92e+04 &   10 &           &           & iters  \\ 
 \hdashline 
     &           &    1 &  8.76e-08 &  4.05e-09 &      \\ 
     &           &    2 &  9.82e-09 &  2.50e-15 &      \\ 
     &           &    3 &  6.79e-08 &  2.80e-16 &      \\ 
     &           &    4 &  8.76e-08 &  1.94e-15 &      \\ 
     &           &    5 &  6.79e-08 &  2.50e-15 &      \\ 
     &           &    6 &  9.88e-09 &  1.94e-15 &      \\ 
     &           &    7 &  9.86e-09 &  2.82e-16 &      \\ 
     &           &    8 &  9.85e-09 &  2.82e-16 &      \\ 
     &           &    9 &  9.84e-09 &  2.81e-16 &      \\ 
     &           &   10 &  9.82e-09 &  2.81e-16 &      \\ 
1921 &  1.92e+04 &   10 &           &           & iters  \\ 
 \hdashline 
     &           &    1 &  2.14e-08 &  2.22e-09 &      \\ 
     &           &    2 &  2.13e-08 &  6.10e-16 &      \\ 
     &           &    3 &  5.64e-08 &  6.09e-16 &      \\ 
     &           &    4 &  2.14e-08 &  1.61e-15 &      \\ 
     &           &    5 &  2.14e-08 &  6.10e-16 &      \\ 
     &           &    6 &  2.13e-08 &  6.09e-16 &      \\ 
     &           &    7 &  5.64e-08 &  6.09e-16 &      \\ 
     &           &    8 &  2.14e-08 &  1.61e-15 &      \\ 
     &           &    9 &  2.13e-08 &  6.10e-16 &      \\ 
     &           &   10 &  5.64e-08 &  6.09e-16 &      \\ 
1922 &  1.92e+04 &   10 &           &           & iters  \\ 
 \hdashline 
     &           &    1 &  8.39e-09 &  7.21e-10 &      \\ 
     &           &    2 &  8.38e-09 &  2.39e-16 &      \\ 
     &           &    3 &  8.37e-09 &  2.39e-16 &      \\ 
     &           &    4 &  8.35e-09 &  2.39e-16 &      \\ 
     &           &    5 &  8.34e-09 &  2.38e-16 &      \\ 
     &           &    6 &  8.33e-09 &  2.38e-16 &      \\ 
     &           &    7 &  8.32e-09 &  2.38e-16 &      \\ 
     &           &    8 &  8.31e-09 &  2.37e-16 &      \\ 
     &           &    9 &  8.29e-09 &  2.37e-16 &      \\ 
     &           &   10 &  8.28e-09 &  2.37e-16 &      \\ 
1923 &  1.92e+04 &   10 &           &           & iters  \\ 
 \hdashline 
     &           &    1 &  6.33e-08 &  3.14e-09 &      \\ 
     &           &    2 &  1.45e-08 &  1.81e-15 &      \\ 
     &           &    3 &  1.44e-08 &  4.13e-16 &      \\ 
     &           &    4 &  1.44e-08 &  4.12e-16 &      \\ 
     &           &    5 &  6.33e-08 &  4.12e-16 &      \\ 
     &           &    6 &  1.45e-08 &  1.81e-15 &      \\ 
     &           &    7 &  1.45e-08 &  4.14e-16 &      \\ 
     &           &    8 &  1.45e-08 &  4.13e-16 &      \\ 
     &           &    9 &  1.44e-08 &  4.13e-16 &      \\ 
     &           &   10 &  1.44e-08 &  4.12e-16 &      \\ 
1924 &  1.92e+04 &   10 &           &           & iters  \\ 
 \hdashline 
     &           &    1 &  4.31e-09 &  3.32e-09 &      \\ 
     &           &    2 &  4.30e-09 &  1.23e-16 &      \\ 
     &           &    3 &  4.29e-09 &  1.23e-16 &      \\ 
     &           &    4 &  7.34e-08 &  1.23e-16 &      \\ 
     &           &    5 &  8.21e-08 &  2.09e-15 &      \\ 
     &           &    6 &  7.34e-08 &  2.34e-15 &      \\ 
     &           &    7 &  8.21e-08 &  2.10e-15 &      \\ 
     &           &    8 &  7.34e-08 &  2.34e-15 &      \\ 
     &           &    9 &  4.37e-09 &  2.10e-15 &      \\ 
     &           &   10 &  4.36e-09 &  1.25e-16 &      \\ 
1925 &  1.92e+04 &   10 &           &           & iters  \\ 
 \hdashline 
     &           &    1 &  8.33e-09 &  1.90e-09 &      \\ 
     &           &    2 &  8.32e-09 &  2.38e-16 &      \\ 
     &           &    3 &  8.31e-09 &  2.38e-16 &      \\ 
     &           &    4 &  8.30e-09 &  2.37e-16 &      \\ 
     &           &    5 &  8.29e-09 &  2.37e-16 &      \\ 
     &           &    6 &  8.27e-09 &  2.37e-16 &      \\ 
     &           &    7 &  8.26e-09 &  2.36e-16 &      \\ 
     &           &    8 &  6.94e-08 &  2.36e-16 &      \\ 
     &           &    9 &  4.72e-08 &  1.98e-15 &      \\ 
     &           &   10 &  8.28e-09 &  1.35e-15 &      \\ 
1926 &  1.92e+04 &   10 &           &           & iters  \\ 
 \hdashline 
     &           &    1 &  5.32e-08 &  1.88e-09 &      \\ 
     &           &    2 &  2.45e-08 &  1.52e-15 &      \\ 
     &           &    3 &  2.45e-08 &  7.00e-16 &      \\ 
     &           &    4 &  5.32e-08 &  6.99e-16 &      \\ 
     &           &    5 &  2.45e-08 &  1.52e-15 &      \\ 
     &           &    6 &  2.45e-08 &  7.01e-16 &      \\ 
     &           &    7 &  5.32e-08 &  7.00e-16 &      \\ 
     &           &    8 &  2.46e-08 &  1.52e-15 &      \\ 
     &           &    9 &  2.45e-08 &  7.01e-16 &      \\ 
     &           &   10 &  5.32e-08 &  7.00e-16 &      \\ 
1927 &  1.93e+04 &   10 &           &           & iters  \\ 
 \hdashline 
     &           &    1 &  5.59e-08 &  1.22e-10 &      \\ 
     &           &    2 &  2.18e-08 &  1.60e-15 &      \\ 
     &           &    3 &  2.18e-08 &  6.23e-16 &      \\ 
     &           &    4 &  5.59e-08 &  6.22e-16 &      \\ 
     &           &    5 &  2.19e-08 &  1.60e-15 &      \\ 
     &           &    6 &  2.18e-08 &  6.24e-16 &      \\ 
     &           &    7 &  2.18e-08 &  6.23e-16 &      \\ 
     &           &    8 &  5.59e-08 &  6.22e-16 &      \\ 
     &           &    9 &  2.18e-08 &  1.60e-15 &      \\ 
     &           &   10 &  2.18e-08 &  6.23e-16 &      \\ 
1928 &  1.93e+04 &   10 &           &           & iters  \\ 
 \hdashline 
     &           &    1 &  3.88e-08 &  1.97e-09 &      \\ 
     &           &    2 &  3.89e-08 &  1.11e-15 &      \\ 
     &           &    3 &  3.88e-08 &  1.11e-15 &      \\ 
     &           &    4 &  3.89e-08 &  1.11e-15 &      \\ 
     &           &    5 &  3.88e-08 &  1.11e-15 &      \\ 
     &           &    6 &  3.89e-08 &  1.11e-15 &      \\ 
     &           &    7 &  3.88e-08 &  1.11e-15 &      \\ 
     &           &    8 &  3.89e-08 &  1.11e-15 &      \\ 
     &           &    9 &  3.88e-08 &  1.11e-15 &      \\ 
     &           &   10 &  3.89e-08 &  1.11e-15 &      \\ 
1929 &  1.93e+04 &   10 &           &           & iters  \\ 
 \hdashline 
     &           &    1 &  1.89e-08 &  3.93e-10 &      \\ 
     &           &    2 &  1.89e-08 &  5.40e-16 &      \\ 
     &           &    3 &  1.88e-08 &  5.39e-16 &      \\ 
     &           &    4 &  5.89e-08 &  5.38e-16 &      \\ 
     &           &    5 &  1.89e-08 &  1.68e-15 &      \\ 
     &           &    6 &  1.89e-08 &  5.40e-16 &      \\ 
     &           &    7 &  1.89e-08 &  5.39e-16 &      \\ 
     &           &    8 &  5.89e-08 &  5.38e-16 &      \\ 
     &           &    9 &  1.89e-08 &  1.68e-15 &      \\ 
     &           &   10 &  1.89e-08 &  5.40e-16 &      \\ 
1930 &  1.93e+04 &   10 &           &           & iters  \\ 
 \hdashline 
     &           &    1 &  2.00e-08 &  7.90e-10 &      \\ 
     &           &    2 &  5.77e-08 &  5.70e-16 &      \\ 
     &           &    3 &  2.00e-08 &  1.65e-15 &      \\ 
     &           &    4 &  2.00e-08 &  5.72e-16 &      \\ 
     &           &    5 &  2.00e-08 &  5.71e-16 &      \\ 
     &           &    6 &  5.78e-08 &  5.70e-16 &      \\ 
     &           &    7 &  2.00e-08 &  1.65e-15 &      \\ 
     &           &    8 &  2.00e-08 &  5.71e-16 &      \\ 
     &           &    9 &  2.00e-08 &  5.71e-16 &      \\ 
     &           &   10 &  5.78e-08 &  5.70e-16 &      \\ 
1931 &  1.93e+04 &   10 &           &           & iters  \\ 
 \hdashline 
     &           &    1 &  6.23e-08 &  1.45e-09 &      \\ 
     &           &    2 &  1.55e-08 &  1.78e-15 &      \\ 
     &           &    3 &  1.55e-08 &  4.43e-16 &      \\ 
     &           &    4 &  1.55e-08 &  4.42e-16 &      \\ 
     &           &    5 &  1.55e-08 &  4.42e-16 &      \\ 
     &           &    6 &  6.23e-08 &  4.41e-16 &      \\ 
     &           &    7 &  1.55e-08 &  1.78e-15 &      \\ 
     &           &    8 &  1.55e-08 &  4.43e-16 &      \\ 
     &           &    9 &  1.55e-08 &  4.42e-16 &      \\ 
     &           &   10 &  1.55e-08 &  4.42e-16 &      \\ 
1932 &  1.93e+04 &   10 &           &           & iters  \\ 
 \hdashline 
     &           &    1 &  7.59e-08 &  2.59e-09 &      \\ 
     &           &    2 &  7.96e-08 &  2.17e-15 &      \\ 
     &           &    3 &  7.59e-08 &  2.27e-15 &      \\ 
     &           &    4 &  7.96e-08 &  2.17e-15 &      \\ 
     &           &    5 &  7.59e-08 &  2.27e-15 &      \\ 
     &           &    6 &  1.91e-09 &  2.17e-15 &      \\ 
     &           &    7 &  1.91e-09 &  5.45e-17 &      \\ 
     &           &    8 &  1.90e-09 &  5.44e-17 &      \\ 
     &           &    9 &  1.90e-09 &  5.44e-17 &      \\ 
     &           &   10 &  1.90e-09 &  5.43e-17 &      \\ 
1933 &  1.93e+04 &   10 &           &           & iters  \\ 
 \hdashline 
     &           &    1 &  6.55e-08 &  1.81e-09 &      \\ 
     &           &    2 &  1.23e-08 &  1.87e-15 &      \\ 
     &           &    3 &  1.22e-08 &  3.50e-16 &      \\ 
     &           &    4 &  1.22e-08 &  3.49e-16 &      \\ 
     &           &    5 &  1.22e-08 &  3.49e-16 &      \\ 
     &           &    6 &  6.55e-08 &  3.48e-16 &      \\ 
     &           &    7 &  9.00e-08 &  1.87e-15 &      \\ 
     &           &    8 &  6.55e-08 &  2.57e-15 &      \\ 
     &           &    9 &  1.22e-08 &  1.87e-15 &      \\ 
     &           &   10 &  1.22e-08 &  3.50e-16 &      \\ 
1934 &  1.93e+04 &   10 &           &           & iters  \\ 
 \hdashline 
     &           &    1 &  4.74e-08 &  1.06e-09 &      \\ 
     &           &    2 &  3.04e-08 &  1.35e-15 &      \\ 
     &           &    3 &  4.74e-08 &  8.66e-16 &      \\ 
     &           &    4 &  3.04e-08 &  1.35e-15 &      \\ 
     &           &    5 &  3.03e-08 &  8.67e-16 &      \\ 
     &           &    6 &  4.74e-08 &  8.66e-16 &      \\ 
     &           &    7 &  3.04e-08 &  1.35e-15 &      \\ 
     &           &    8 &  4.74e-08 &  8.66e-16 &      \\ 
     &           &    9 &  3.04e-08 &  1.35e-15 &      \\ 
     &           &   10 &  3.03e-08 &  8.67e-16 &      \\ 
1935 &  1.93e+04 &   10 &           &           & iters  \\ 
 \hdashline 
     &           &    1 &  3.40e-08 &  8.43e-10 &      \\ 
     &           &    2 &  4.37e-08 &  9.71e-16 &      \\ 
     &           &    3 &  3.40e-08 &  1.25e-15 &      \\ 
     &           &    4 &  4.37e-08 &  9.72e-16 &      \\ 
     &           &    5 &  3.41e-08 &  1.25e-15 &      \\ 
     &           &    6 &  3.40e-08 &  9.72e-16 &      \\ 
     &           &    7 &  4.37e-08 &  9.71e-16 &      \\ 
     &           &    8 &  3.40e-08 &  1.25e-15 &      \\ 
     &           &    9 &  4.37e-08 &  9.71e-16 &      \\ 
     &           &   10 &  3.40e-08 &  1.25e-15 &      \\ 
1936 &  1.94e+04 &   10 &           &           & iters  \\ 
 \hdashline 
     &           &    1 &  4.22e-08 &  1.77e-09 &      \\ 
     &           &    2 &  4.21e-08 &  1.20e-15 &      \\ 
     &           &    3 &  3.57e-08 &  1.20e-15 &      \\ 
     &           &    4 &  4.21e-08 &  1.02e-15 &      \\ 
     &           &    5 &  3.57e-08 &  1.20e-15 &      \\ 
     &           &    6 &  4.21e-08 &  1.02e-15 &      \\ 
     &           &    7 &  3.57e-08 &  1.20e-15 &      \\ 
     &           &    8 &  3.56e-08 &  1.02e-15 &      \\ 
     &           &    9 &  4.21e-08 &  1.02e-15 &      \\ 
     &           &   10 &  3.56e-08 &  1.20e-15 &      \\ 
1937 &  1.94e+04 &   10 &           &           & iters  \\ 
 \hdashline 
     &           &    1 &  5.89e-08 &  9.05e-11 &      \\ 
     &           &    2 &  1.89e-08 &  1.68e-15 &      \\ 
     &           &    3 &  1.89e-08 &  5.39e-16 &      \\ 
     &           &    4 &  1.88e-08 &  5.38e-16 &      \\ 
     &           &    5 &  1.88e-08 &  5.37e-16 &      \\ 
     &           &    6 &  5.89e-08 &  5.37e-16 &      \\ 
     &           &    7 &  1.89e-08 &  1.68e-15 &      \\ 
     &           &    8 &  1.88e-08 &  5.38e-16 &      \\ 
     &           &    9 &  1.88e-08 &  5.38e-16 &      \\ 
     &           &   10 &  5.89e-08 &  5.37e-16 &      \\ 
1938 &  1.94e+04 &   10 &           &           & iters  \\ 
 \hdashline 
     &           &    1 &  3.27e-08 &  2.53e-09 &      \\ 
     &           &    2 &  3.27e-08 &  9.34e-16 &      \\ 
     &           &    3 &  4.51e-08 &  9.32e-16 &      \\ 
     &           &    4 &  3.27e-08 &  1.29e-15 &      \\ 
     &           &    5 &  4.50e-08 &  9.33e-16 &      \\ 
     &           &    6 &  3.27e-08 &  1.29e-15 &      \\ 
     &           &    7 &  3.27e-08 &  9.33e-16 &      \\ 
     &           &    8 &  4.51e-08 &  9.32e-16 &      \\ 
     &           &    9 &  3.27e-08 &  1.29e-15 &      \\ 
     &           &   10 &  4.51e-08 &  9.33e-16 &      \\ 
1939 &  1.94e+04 &   10 &           &           & iters  \\ 
 \hdashline 
     &           &    1 &  6.18e-08 &  2.71e-09 &      \\ 
     &           &    2 &  1.60e-08 &  1.76e-15 &      \\ 
     &           &    3 &  1.60e-08 &  4.57e-16 &      \\ 
     &           &    4 &  1.60e-08 &  4.56e-16 &      \\ 
     &           &    5 &  1.59e-08 &  4.56e-16 &      \\ 
     &           &    6 &  6.18e-08 &  4.55e-16 &      \\ 
     &           &    7 &  1.60e-08 &  1.76e-15 &      \\ 
     &           &    8 &  1.60e-08 &  4.57e-16 &      \\ 
     &           &    9 &  1.60e-08 &  4.56e-16 &      \\ 
     &           &   10 &  1.59e-08 &  4.55e-16 &      \\ 
1940 &  1.94e+04 &   10 &           &           & iters  \\ 
 \hdashline 
     &           &    1 &  1.56e-08 &  4.35e-10 &      \\ 
     &           &    2 &  1.56e-08 &  4.46e-16 &      \\ 
     &           &    3 &  1.56e-08 &  4.45e-16 &      \\ 
     &           &    4 &  1.55e-08 &  4.44e-16 &      \\ 
     &           &    5 &  1.55e-08 &  4.44e-16 &      \\ 
     &           &    6 &  1.55e-08 &  4.43e-16 &      \\ 
     &           &    7 &  6.22e-08 &  4.42e-16 &      \\ 
     &           &    8 &  1.56e-08 &  1.78e-15 &      \\ 
     &           &    9 &  1.55e-08 &  4.44e-16 &      \\ 
     &           &   10 &  1.55e-08 &  4.44e-16 &      \\ 
1941 &  1.94e+04 &   10 &           &           & iters  \\ 
 \hdashline 
     &           &    1 &  4.27e-08 &  6.49e-10 &      \\ 
     &           &    2 &  3.50e-08 &  1.22e-15 &      \\ 
     &           &    3 &  4.27e-08 &  1.00e-15 &      \\ 
     &           &    4 &  3.51e-08 &  1.22e-15 &      \\ 
     &           &    5 &  4.27e-08 &  1.00e-15 &      \\ 
     &           &    6 &  3.51e-08 &  1.22e-15 &      \\ 
     &           &    7 &  4.27e-08 &  1.00e-15 &      \\ 
     &           &    8 &  3.51e-08 &  1.22e-15 &      \\ 
     &           &    9 &  3.50e-08 &  1.00e-15 &      \\ 
     &           &   10 &  4.27e-08 &  1.00e-15 &      \\ 
1942 &  1.94e+04 &   10 &           &           & iters  \\ 
 \hdashline 
     &           &    1 &  4.04e-08 &  7.85e-10 &      \\ 
     &           &    2 &  3.73e-08 &  1.15e-15 &      \\ 
     &           &    3 &  4.04e-08 &  1.06e-15 &      \\ 
     &           &    4 &  3.73e-08 &  1.15e-15 &      \\ 
     &           &    5 &  4.04e-08 &  1.07e-15 &      \\ 
     &           &    6 &  3.73e-08 &  1.15e-15 &      \\ 
     &           &    7 &  4.04e-08 &  1.07e-15 &      \\ 
     &           &    8 &  3.73e-08 &  1.15e-15 &      \\ 
     &           &    9 &  4.04e-08 &  1.07e-15 &      \\ 
     &           &   10 &  3.73e-08 &  1.15e-15 &      \\ 
1943 &  1.94e+04 &   10 &           &           & iters  \\ 
 \hdashline 
     &           &    1 &  4.07e-08 &  1.59e-09 &      \\ 
     &           &    2 &  3.70e-08 &  1.16e-15 &      \\ 
     &           &    3 &  4.07e-08 &  1.06e-15 &      \\ 
     &           &    4 &  3.70e-08 &  1.16e-15 &      \\ 
     &           &    5 &  4.07e-08 &  1.06e-15 &      \\ 
     &           &    6 &  3.70e-08 &  1.16e-15 &      \\ 
     &           &    7 &  3.70e-08 &  1.06e-15 &      \\ 
     &           &    8 &  4.07e-08 &  1.06e-15 &      \\ 
     &           &    9 &  3.70e-08 &  1.16e-15 &      \\ 
     &           &   10 &  4.07e-08 &  1.06e-15 &      \\ 
1944 &  1.94e+04 &   10 &           &           & iters  \\ 
 \hdashline 
     &           &    1 &  3.57e-08 &  1.62e-09 &      \\ 
     &           &    2 &  4.21e-08 &  1.02e-15 &      \\ 
     &           &    3 &  3.57e-08 &  1.20e-15 &      \\ 
     &           &    4 &  4.21e-08 &  1.02e-15 &      \\ 
     &           &    5 &  3.57e-08 &  1.20e-15 &      \\ 
     &           &    6 &  3.56e-08 &  1.02e-15 &      \\ 
     &           &    7 &  4.21e-08 &  1.02e-15 &      \\ 
     &           &    8 &  3.56e-08 &  1.20e-15 &      \\ 
     &           &    9 &  4.21e-08 &  1.02e-15 &      \\ 
     &           &   10 &  3.56e-08 &  1.20e-15 &      \\ 
1945 &  1.94e+04 &   10 &           &           & iters  \\ 
 \hdashline 
     &           &    1 &  1.26e-08 &  1.28e-09 &      \\ 
     &           &    2 &  1.26e-08 &  3.60e-16 &      \\ 
     &           &    3 &  6.51e-08 &  3.60e-16 &      \\ 
     &           &    4 &  1.27e-08 &  1.86e-15 &      \\ 
     &           &    5 &  1.27e-08 &  3.62e-16 &      \\ 
     &           &    6 &  1.26e-08 &  3.61e-16 &      \\ 
     &           &    7 &  1.26e-08 &  3.61e-16 &      \\ 
     &           &    8 &  1.26e-08 &  3.60e-16 &      \\ 
     &           &    9 &  6.51e-08 &  3.60e-16 &      \\ 
     &           &   10 &  1.27e-08 &  1.86e-15 &      \\ 
1946 &  1.94e+04 &   10 &           &           & iters  \\ 
 \hdashline 
     &           &    1 &  5.02e-09 &  1.59e-09 &      \\ 
     &           &    2 &  5.01e-09 &  1.43e-16 &      \\ 
     &           &    3 &  5.01e-09 &  1.43e-16 &      \\ 
     &           &    4 &  5.00e-09 &  1.43e-16 &      \\ 
     &           &    5 &  4.99e-09 &  1.43e-16 &      \\ 
     &           &    6 &  4.98e-09 &  1.42e-16 &      \\ 
     &           &    7 &  4.98e-09 &  1.42e-16 &      \\ 
     &           &    8 &  3.39e-08 &  1.42e-16 &      \\ 
     &           &    9 &  5.02e-09 &  9.67e-16 &      \\ 
     &           &   10 &  5.01e-09 &  1.43e-16 &      \\ 
1947 &  1.95e+04 &   10 &           &           & iters  \\ 
 \hdashline 
     &           &    1 &  2.21e-08 &  1.07e-09 &      \\ 
     &           &    2 &  5.56e-08 &  6.32e-16 &      \\ 
     &           &    3 &  2.22e-08 &  1.59e-15 &      \\ 
     &           &    4 &  2.22e-08 &  6.33e-16 &      \\ 
     &           &    5 &  2.21e-08 &  6.32e-16 &      \\ 
     &           &    6 &  5.56e-08 &  6.31e-16 &      \\ 
     &           &    7 &  2.22e-08 &  1.59e-15 &      \\ 
     &           &    8 &  2.21e-08 &  6.33e-16 &      \\ 
     &           &    9 &  2.21e-08 &  6.32e-16 &      \\ 
     &           &   10 &  5.56e-08 &  6.31e-16 &      \\ 
1948 &  1.95e+04 &   10 &           &           & iters  \\ 
 \hdashline 
     &           &    1 &  9.63e-09 &  3.01e-10 &      \\ 
     &           &    2 &  9.62e-09 &  2.75e-16 &      \\ 
     &           &    3 &  2.92e-08 &  2.74e-16 &      \\ 
     &           &    4 &  9.64e-09 &  8.35e-16 &      \\ 
     &           &    5 &  9.63e-09 &  2.75e-16 &      \\ 
     &           &    6 &  9.62e-09 &  2.75e-16 &      \\ 
     &           &    7 &  2.92e-08 &  2.74e-16 &      \\ 
     &           &    8 &  9.64e-09 &  8.35e-16 &      \\ 
     &           &    9 &  9.63e-09 &  2.75e-16 &      \\ 
     &           &   10 &  9.62e-09 &  2.75e-16 &      \\ 
1949 &  1.95e+04 &   10 &           &           & iters  \\ 
 \hdashline 
     &           &    1 &  9.73e-09 &  1.26e-09 &      \\ 
     &           &    2 &  9.71e-09 &  2.78e-16 &      \\ 
     &           &    3 &  6.80e-08 &  2.77e-16 &      \\ 
     &           &    4 &  9.80e-09 &  1.94e-15 &      \\ 
     &           &    5 &  9.78e-09 &  2.80e-16 &      \\ 
     &           &    6 &  9.77e-09 &  2.79e-16 &      \\ 
     &           &    7 &  9.75e-09 &  2.79e-16 &      \\ 
     &           &    8 &  9.74e-09 &  2.78e-16 &      \\ 
     &           &    9 &  9.73e-09 &  2.78e-16 &      \\ 
     &           &   10 &  9.71e-09 &  2.78e-16 &      \\ 
1950 &  1.95e+04 &   10 &           &           & iters  \\ 
 \hdashline 
     &           &    1 &  5.14e-08 &  2.46e-09 &      \\ 
     &           &    2 &  1.24e-08 &  1.47e-15 &      \\ 
     &           &    3 &  6.53e-08 &  3.55e-16 &      \\ 
     &           &    4 &  5.14e-08 &  1.86e-15 &      \\ 
     &           &    5 &  1.24e-08 &  1.47e-15 &      \\ 
     &           &    6 &  1.24e-08 &  3.55e-16 &      \\ 
     &           &    7 &  6.53e-08 &  3.55e-16 &      \\ 
     &           &    8 &  1.25e-08 &  1.86e-15 &      \\ 
     &           &    9 &  1.25e-08 &  3.57e-16 &      \\ 
     &           &   10 &  1.25e-08 &  3.56e-16 &      \\ 
1951 &  1.95e+04 &   10 &           &           & iters  \\ 
 \hdashline 
     &           &    1 &  3.89e-08 &  2.19e-09 &      \\ 
     &           &    2 &  3.89e-08 &  1.11e-15 &      \\ 
     &           &    3 &  3.89e-08 &  1.11e-15 &      \\ 
     &           &    4 &  3.89e-08 &  1.11e-15 &      \\ 
     &           &    5 &  3.89e-08 &  1.11e-15 &      \\ 
     &           &    6 &  3.89e-08 &  1.11e-15 &      \\ 
     &           &    7 &  3.89e-08 &  1.11e-15 &      \\ 
     &           &    8 &  3.89e-08 &  1.11e-15 &      \\ 
     &           &    9 &  3.89e-08 &  1.11e-15 &      \\ 
     &           &   10 &  3.89e-08 &  1.11e-15 &      \\ 
1952 &  1.95e+04 &   10 &           &           & iters  \\ 
 \hdashline 
     &           &    1 &  2.36e-08 &  5.98e-11 &      \\ 
     &           &    2 &  2.35e-08 &  6.72e-16 &      \\ 
     &           &    3 &  5.42e-08 &  6.71e-16 &      \\ 
     &           &    4 &  2.36e-08 &  1.55e-15 &      \\ 
     &           &    5 &  2.35e-08 &  6.73e-16 &      \\ 
     &           &    6 &  5.42e-08 &  6.72e-16 &      \\ 
     &           &    7 &  2.36e-08 &  1.55e-15 &      \\ 
     &           &    8 &  2.35e-08 &  6.73e-16 &      \\ 
     &           &    9 &  5.42e-08 &  6.72e-16 &      \\ 
     &           &   10 &  2.36e-08 &  1.55e-15 &      \\ 
1953 &  1.95e+04 &   10 &           &           & iters  \\ 
 \hdashline 
     &           &    1 &  3.68e-08 &  2.28e-09 &      \\ 
     &           &    2 &  2.09e-09 &  1.05e-15 &      \\ 
     &           &    3 &  2.09e-09 &  5.98e-17 &      \\ 
     &           &    4 &  2.09e-09 &  5.97e-17 &      \\ 
     &           &    5 &  2.09e-09 &  5.96e-17 &      \\ 
     &           &    6 &  2.08e-09 &  5.95e-17 &      \\ 
     &           &    7 &  2.08e-09 &  5.94e-17 &      \\ 
     &           &    8 &  2.08e-09 &  5.94e-17 &      \\ 
     &           &    9 &  2.07e-09 &  5.93e-17 &      \\ 
     &           &   10 &  2.07e-09 &  5.92e-17 &      \\ 
1954 &  1.95e+04 &   10 &           &           & iters  \\ 
 \hdashline 
     &           &    1 &  8.14e-09 &  2.55e-09 &      \\ 
     &           &    2 &  8.13e-09 &  2.32e-16 &      \\ 
     &           &    3 &  3.07e-08 &  2.32e-16 &      \\ 
     &           &    4 &  8.16e-09 &  8.77e-16 &      \\ 
     &           &    5 &  8.15e-09 &  2.33e-16 &      \\ 
     &           &    6 &  8.14e-09 &  2.33e-16 &      \\ 
     &           &    7 &  8.13e-09 &  2.32e-16 &      \\ 
     &           &    8 &  3.07e-08 &  2.32e-16 &      \\ 
     &           &    9 &  8.16e-09 &  8.77e-16 &      \\ 
     &           &   10 &  8.15e-09 &  2.33e-16 &      \\ 
1955 &  1.95e+04 &   10 &           &           & iters  \\ 
 \hdashline 
     &           &    1 &  2.52e-08 &  2.14e-09 &      \\ 
     &           &    2 &  2.52e-08 &  7.21e-16 &      \\ 
     &           &    3 &  5.25e-08 &  7.19e-16 &      \\ 
     &           &    4 &  2.52e-08 &  1.50e-15 &      \\ 
     &           &    5 &  2.52e-08 &  7.21e-16 &      \\ 
     &           &    6 &  5.25e-08 &  7.20e-16 &      \\ 
     &           &    7 &  2.53e-08 &  1.50e-15 &      \\ 
     &           &    8 &  2.52e-08 &  7.21e-16 &      \\ 
     &           &    9 &  5.25e-08 &  7.20e-16 &      \\ 
     &           &   10 &  2.53e-08 &  1.50e-15 &      \\ 
1956 &  1.96e+04 &   10 &           &           & iters  \\ 
 \hdashline 
     &           &    1 &  1.69e-08 &  2.02e-09 &      \\ 
     &           &    2 &  1.69e-08 &  4.82e-16 &      \\ 
     &           &    3 &  6.09e-08 &  4.81e-16 &      \\ 
     &           &    4 &  1.69e-08 &  1.74e-15 &      \\ 
     &           &    5 &  1.69e-08 &  4.83e-16 &      \\ 
     &           &    6 &  1.69e-08 &  4.82e-16 &      \\ 
     &           &    7 &  1.68e-08 &  4.81e-16 &      \\ 
     &           &    8 &  6.09e-08 &  4.81e-16 &      \\ 
     &           &    9 &  1.69e-08 &  1.74e-15 &      \\ 
     &           &   10 &  1.69e-08 &  4.83e-16 &      \\ 
1957 &  1.96e+04 &   10 &           &           & iters  \\ 
 \hdashline 
     &           &    1 &  2.93e-09 &  1.20e-09 &      \\ 
     &           &    2 &  2.92e-09 &  8.35e-17 &      \\ 
     &           &    3 &  2.92e-09 &  8.34e-17 &      \\ 
     &           &    4 &  2.91e-09 &  8.33e-17 &      \\ 
     &           &    5 &  2.91e-09 &  8.31e-17 &      \\ 
     &           &    6 &  2.90e-09 &  8.30e-17 &      \\ 
     &           &    7 &  2.90e-09 &  8.29e-17 &      \\ 
     &           &    8 &  2.90e-09 &  8.28e-17 &      \\ 
     &           &    9 &  2.89e-09 &  8.27e-17 &      \\ 
     &           &   10 &  2.89e-09 &  8.25e-17 &      \\ 
1958 &  1.96e+04 &   10 &           &           & iters  \\ 
 \hdashline 
     &           &    1 &  2.90e-08 &  4.61e-10 &      \\ 
     &           &    2 &  2.90e-08 &  8.28e-16 &      \\ 
     &           &    3 &  4.88e-08 &  8.26e-16 &      \\ 
     &           &    4 &  2.90e-08 &  1.39e-15 &      \\ 
     &           &    5 &  4.87e-08 &  8.27e-16 &      \\ 
     &           &    6 &  2.90e-08 &  1.39e-15 &      \\ 
     &           &    7 &  2.90e-08 &  8.28e-16 &      \\ 
     &           &    8 &  4.88e-08 &  8.27e-16 &      \\ 
     &           &    9 &  2.90e-08 &  1.39e-15 &      \\ 
     &           &   10 &  2.90e-08 &  8.28e-16 &      \\ 
1959 &  1.96e+04 &   10 &           &           & iters  \\ 
 \hdashline 
     &           &    1 &  3.78e-08 &  9.71e-10 &      \\ 
     &           &    2 &  3.99e-08 &  1.08e-15 &      \\ 
     &           &    3 &  3.78e-08 &  1.14e-15 &      \\ 
     &           &    4 &  3.99e-08 &  1.08e-15 &      \\ 
     &           &    5 &  3.78e-08 &  1.14e-15 &      \\ 
     &           &    6 &  3.99e-08 &  1.08e-15 &      \\ 
     &           &    7 &  3.78e-08 &  1.14e-15 &      \\ 
     &           &    8 &  3.99e-08 &  1.08e-15 &      \\ 
     &           &    9 &  3.79e-08 &  1.14e-15 &      \\ 
     &           &   10 &  3.99e-08 &  1.08e-15 &      \\ 
1960 &  1.96e+04 &   10 &           &           & iters  \\ 
 \hdashline 
     &           &    1 &  1.71e-08 &  1.92e-09 &      \\ 
     &           &    2 &  1.71e-08 &  4.89e-16 &      \\ 
     &           &    3 &  1.71e-08 &  4.88e-16 &      \\ 
     &           &    4 &  1.71e-08 &  4.88e-16 &      \\ 
     &           &    5 &  1.70e-08 &  4.87e-16 &      \\ 
     &           &    6 &  6.07e-08 &  4.86e-16 &      \\ 
     &           &    7 &  1.71e-08 &  1.73e-15 &      \\ 
     &           &    8 &  1.71e-08 &  4.88e-16 &      \\ 
     &           &    9 &  1.70e-08 &  4.87e-16 &      \\ 
     &           &   10 &  1.70e-08 &  4.87e-16 &      \\ 
1961 &  1.96e+04 &   10 &           &           & iters  \\ 
 \hdashline 
     &           &    1 &  8.00e-08 &  1.69e-09 &      \\ 
     &           &    2 &  2.23e-09 &  2.28e-15 &      \\ 
     &           &    3 &  2.22e-09 &  6.36e-17 &      \\ 
     &           &    4 &  7.55e-08 &  6.35e-17 &      \\ 
     &           &    5 &  8.00e-08 &  2.15e-15 &      \\ 
     &           &    6 &  7.55e-08 &  2.28e-15 &      \\ 
     &           &    7 &  8.00e-08 &  2.15e-15 &      \\ 
     &           &    8 &  7.55e-08 &  2.28e-15 &      \\ 
     &           &    9 &  8.00e-08 &  2.15e-15 &      \\ 
     &           &   10 &  7.55e-08 &  2.28e-15 &      \\ 
1962 &  1.96e+04 &   10 &           &           & iters  \\ 
 \hdashline 
     &           &    1 &  2.56e-08 &  7.01e-10 &      \\ 
     &           &    2 &  5.21e-08 &  7.31e-16 &      \\ 
     &           &    3 &  2.57e-08 &  1.49e-15 &      \\ 
     &           &    4 &  2.56e-08 &  7.32e-16 &      \\ 
     &           &    5 &  5.21e-08 &  7.31e-16 &      \\ 
     &           &    6 &  2.57e-08 &  1.49e-15 &      \\ 
     &           &    7 &  2.56e-08 &  7.32e-16 &      \\ 
     &           &    8 &  5.21e-08 &  7.31e-16 &      \\ 
     &           &    9 &  2.57e-08 &  1.49e-15 &      \\ 
     &           &   10 &  2.56e-08 &  7.32e-16 &      \\ 
1963 &  1.96e+04 &   10 &           &           & iters  \\ 
 \hdashline 
     &           &    1 &  3.23e-08 &  6.74e-10 &      \\ 
     &           &    2 &  3.22e-08 &  9.21e-16 &      \\ 
     &           &    3 &  4.55e-08 &  9.19e-16 &      \\ 
     &           &    4 &  3.22e-08 &  1.30e-15 &      \\ 
     &           &    5 &  4.55e-08 &  9.20e-16 &      \\ 
     &           &    6 &  3.22e-08 &  1.30e-15 &      \\ 
     &           &    7 &  3.22e-08 &  9.20e-16 &      \\ 
     &           &    8 &  4.55e-08 &  9.19e-16 &      \\ 
     &           &    9 &  3.22e-08 &  1.30e-15 &      \\ 
     &           &   10 &  4.55e-08 &  9.20e-16 &      \\ 
1964 &  1.96e+04 &   10 &           &           & iters  \\ 
 \hdashline 
     &           &    1 &  5.54e-08 &  2.42e-10 &      \\ 
     &           &    2 &  2.23e-08 &  1.58e-15 &      \\ 
     &           &    3 &  2.23e-08 &  6.37e-16 &      \\ 
     &           &    4 &  2.23e-08 &  6.36e-16 &      \\ 
     &           &    5 &  5.55e-08 &  6.35e-16 &      \\ 
     &           &    6 &  2.23e-08 &  1.58e-15 &      \\ 
     &           &    7 &  2.23e-08 &  6.37e-16 &      \\ 
     &           &    8 &  5.54e-08 &  6.36e-16 &      \\ 
     &           &    9 &  2.23e-08 &  1.58e-15 &      \\ 
     &           &   10 &  2.23e-08 &  6.37e-16 &      \\ 
1965 &  1.96e+04 &   10 &           &           & iters  \\ 
 \hdashline 
     &           &    1 &  1.60e-08 &  1.76e-09 &      \\ 
     &           &    2 &  1.59e-08 &  4.56e-16 &      \\ 
     &           &    3 &  1.59e-08 &  4.55e-16 &      \\ 
     &           &    4 &  6.18e-08 &  4.55e-16 &      \\ 
     &           &    5 &  1.60e-08 &  1.76e-15 &      \\ 
     &           &    6 &  1.60e-08 &  4.56e-16 &      \\ 
     &           &    7 &  1.59e-08 &  4.56e-16 &      \\ 
     &           &    8 &  6.18e-08 &  4.55e-16 &      \\ 
     &           &    9 &  1.60e-08 &  1.76e-15 &      \\ 
     &           &   10 &  1.60e-08 &  4.57e-16 &      \\ 
1966 &  1.96e+04 &   10 &           &           & iters  \\ 
 \hdashline 
     &           &    1 &  8.73e-10 &  2.48e-09 &      \\ 
     &           &    2 &  8.72e-10 &  2.49e-17 &      \\ 
     &           &    3 &  7.68e-08 &  2.49e-17 &      \\ 
     &           &    4 &  7.87e-08 &  2.19e-15 &      \\ 
     &           &    5 &  7.68e-08 &  2.25e-15 &      \\ 
     &           &    6 &  7.87e-08 &  2.19e-15 &      \\ 
     &           &    7 &  7.68e-08 &  2.25e-15 &      \\ 
     &           &    8 &  7.87e-08 &  2.19e-15 &      \\ 
     &           &    9 &  7.68e-08 &  2.25e-15 &      \\ 
     &           &   10 &  7.87e-08 &  2.19e-15 &      \\ 
1967 &  1.97e+04 &   10 &           &           & iters  \\ 
 \hdashline 
     &           &    1 &  5.47e-08 &  1.69e-09 &      \\ 
     &           &    2 &  2.31e-08 &  1.56e-15 &      \\ 
     &           &    3 &  5.46e-08 &  6.59e-16 &      \\ 
     &           &    4 &  2.31e-08 &  1.56e-15 &      \\ 
     &           &    5 &  2.31e-08 &  6.60e-16 &      \\ 
     &           &    6 &  5.46e-08 &  6.59e-16 &      \\ 
     &           &    7 &  2.31e-08 &  1.56e-15 &      \\ 
     &           &    8 &  2.31e-08 &  6.60e-16 &      \\ 
     &           &    9 &  2.31e-08 &  6.59e-16 &      \\ 
     &           &   10 &  5.46e-08 &  6.58e-16 &      \\ 
1968 &  1.97e+04 &   10 &           &           & iters  \\ 
 \hdashline 
     &           &    1 &  3.46e-08 &  2.08e-09 &      \\ 
     &           &    2 &  3.45e-08 &  9.86e-16 &      \\ 
     &           &    3 &  4.32e-08 &  9.85e-16 &      \\ 
     &           &    4 &  3.45e-08 &  1.23e-15 &      \\ 
     &           &    5 &  4.32e-08 &  9.85e-16 &      \\ 
     &           &    6 &  3.45e-08 &  1.23e-15 &      \\ 
     &           &    7 &  4.32e-08 &  9.85e-16 &      \\ 
     &           &    8 &  3.45e-08 &  1.23e-15 &      \\ 
     &           &    9 &  4.32e-08 &  9.86e-16 &      \\ 
     &           &   10 &  3.46e-08 &  1.23e-15 &      \\ 
1969 &  1.97e+04 &   10 &           &           & iters  \\ 
 \hdashline 
     &           &    1 &  2.66e-08 &  2.39e-09 &      \\ 
     &           &    2 &  5.11e-08 &  7.60e-16 &      \\ 
     &           &    3 &  2.67e-08 &  1.46e-15 &      \\ 
     &           &    4 &  2.66e-08 &  7.61e-16 &      \\ 
     &           &    5 &  5.11e-08 &  7.60e-16 &      \\ 
     &           &    6 &  2.67e-08 &  1.46e-15 &      \\ 
     &           &    7 &  2.66e-08 &  7.61e-16 &      \\ 
     &           &    8 &  5.11e-08 &  7.60e-16 &      \\ 
     &           &    9 &  2.67e-08 &  1.46e-15 &      \\ 
     &           &   10 &  5.11e-08 &  7.61e-16 &      \\ 
1970 &  1.97e+04 &   10 &           &           & iters  \\ 
 \hdashline 
     &           &    1 &  2.04e-08 &  9.43e-10 &      \\ 
     &           &    2 &  2.04e-08 &  5.82e-16 &      \\ 
     &           &    3 &  1.85e-08 &  5.82e-16 &      \\ 
     &           &    4 &  2.04e-08 &  5.28e-16 &      \\ 
     &           &    5 &  1.85e-08 &  5.81e-16 &      \\ 
     &           &    6 &  2.04e-08 &  5.28e-16 &      \\ 
     &           &    7 &  1.85e-08 &  5.81e-16 &      \\ 
     &           &    8 &  2.04e-08 &  5.28e-16 &      \\ 
     &           &    9 &  1.85e-08 &  5.81e-16 &      \\ 
     &           &   10 &  2.04e-08 &  5.28e-16 &      \\ 
1971 &  1.97e+04 &   10 &           &           & iters  \\ 
 \hdashline 
     &           &    1 &  5.60e-08 &  3.84e-12 &      \\ 
     &           &    2 &  2.18e-08 &  1.60e-15 &      \\ 
     &           &    3 &  2.18e-08 &  6.23e-16 &      \\ 
     &           &    4 &  2.18e-08 &  6.22e-16 &      \\ 
     &           &    5 &  5.60e-08 &  6.21e-16 &      \\ 
     &           &    6 &  2.18e-08 &  1.60e-15 &      \\ 
     &           &    7 &  2.18e-08 &  6.22e-16 &      \\ 
     &           &    8 &  2.17e-08 &  6.21e-16 &      \\ 
     &           &    9 &  5.60e-08 &  6.21e-16 &      \\ 
     &           &   10 &  2.18e-08 &  1.60e-15 &      \\ 
1972 &  1.97e+04 &   10 &           &           & iters  \\ 
 \hdashline 
     &           &    1 &  3.38e-08 &  5.62e-10 &      \\ 
     &           &    2 &  4.39e-08 &  9.66e-16 &      \\ 
     &           &    3 &  3.39e-08 &  1.25e-15 &      \\ 
     &           &    4 &  4.39e-08 &  9.66e-16 &      \\ 
     &           &    5 &  3.39e-08 &  1.25e-15 &      \\ 
     &           &    6 &  4.39e-08 &  9.67e-16 &      \\ 
     &           &    7 &  3.39e-08 &  1.25e-15 &      \\ 
     &           &    8 &  3.38e-08 &  9.67e-16 &      \\ 
     &           &    9 &  4.39e-08 &  9.66e-16 &      \\ 
     &           &   10 &  3.39e-08 &  1.25e-15 &      \\ 
1973 &  1.97e+04 &   10 &           &           & iters  \\ 
 \hdashline 
     &           &    1 &  2.19e-08 &  2.25e-10 &      \\ 
     &           &    2 &  2.19e-08 &  6.25e-16 &      \\ 
     &           &    3 &  5.58e-08 &  6.25e-16 &      \\ 
     &           &    4 &  2.19e-08 &  1.59e-15 &      \\ 
     &           &    5 &  2.19e-08 &  6.26e-16 &      \\ 
     &           &    6 &  2.19e-08 &  6.25e-16 &      \\ 
     &           &    7 &  5.59e-08 &  6.24e-16 &      \\ 
     &           &    8 &  2.19e-08 &  1.59e-15 &      \\ 
     &           &    9 &  2.19e-08 &  6.25e-16 &      \\ 
     &           &   10 &  5.58e-08 &  6.25e-16 &      \\ 
1974 &  1.97e+04 &   10 &           &           & iters  \\ 
 \hdashline 
     &           &    1 &  7.18e-08 &  1.33e-09 &      \\ 
     &           &    2 &  4.48e-08 &  2.05e-15 &      \\ 
     &           &    3 &  5.90e-09 &  1.28e-15 &      \\ 
     &           &    4 &  5.89e-09 &  1.68e-16 &      \\ 
     &           &    5 &  5.88e-09 &  1.68e-16 &      \\ 
     &           &    6 &  5.87e-09 &  1.68e-16 &      \\ 
     &           &    7 &  7.18e-08 &  1.68e-16 &      \\ 
     &           &    8 &  4.48e-08 &  2.05e-15 &      \\ 
     &           &    9 &  5.90e-09 &  1.28e-15 &      \\ 
     &           &   10 &  5.89e-09 &  1.68e-16 &      \\ 
1975 &  1.97e+04 &   10 &           &           & iters  \\ 
 \hdashline 
     &           &    1 &  1.19e-08 &  1.96e-09 &      \\ 
     &           &    2 &  1.19e-08 &  3.39e-16 &      \\ 
     &           &    3 &  1.18e-08 &  3.38e-16 &      \\ 
     &           &    4 &  2.70e-08 &  3.38e-16 &      \\ 
     &           &    5 &  1.19e-08 &  7.71e-16 &      \\ 
     &           &    6 &  1.18e-08 &  3.38e-16 &      \\ 
     &           &    7 &  2.70e-08 &  3.38e-16 &      \\ 
     &           &    8 &  1.19e-08 &  7.71e-16 &      \\ 
     &           &    9 &  1.18e-08 &  3.39e-16 &      \\ 
     &           &   10 &  1.18e-08 &  3.38e-16 &      \\ 
1976 &  1.98e+04 &   10 &           &           & iters  \\ 
 \hdashline 
     &           &    1 &  5.52e-08 &  3.26e-10 &      \\ 
     &           &    2 &  2.26e-08 &  1.58e-15 &      \\ 
     &           &    3 &  2.25e-08 &  6.44e-16 &      \\ 
     &           &    4 &  5.52e-08 &  6.43e-16 &      \\ 
     &           &    5 &  2.26e-08 &  1.58e-15 &      \\ 
     &           &    6 &  2.25e-08 &  6.44e-16 &      \\ 
     &           &    7 &  2.25e-08 &  6.43e-16 &      \\ 
     &           &    8 &  5.52e-08 &  6.42e-16 &      \\ 
     &           &    9 &  2.25e-08 &  1.58e-15 &      \\ 
     &           &   10 &  2.25e-08 &  6.44e-16 &      \\ 
1977 &  1.98e+04 &   10 &           &           & iters  \\ 
 \hdashline 
     &           &    1 &  4.91e-09 &  1.82e-09 &      \\ 
     &           &    2 &  4.90e-09 &  1.40e-16 &      \\ 
     &           &    3 &  4.90e-09 &  1.40e-16 &      \\ 
     &           &    4 &  4.89e-09 &  1.40e-16 &      \\ 
     &           &    5 &  4.88e-09 &  1.40e-16 &      \\ 
     &           &    6 &  4.88e-09 &  1.39e-16 &      \\ 
     &           &    7 &  4.87e-09 &  1.39e-16 &      \\ 
     &           &    8 &  4.86e-09 &  1.39e-16 &      \\ 
     &           &    9 &  4.85e-09 &  1.39e-16 &      \\ 
     &           &   10 &  4.85e-09 &  1.39e-16 &      \\ 
1978 &  1.98e+04 &   10 &           &           & iters  \\ 
 \hdashline 
     &           &    1 &  2.11e-08 &  9.42e-10 &      \\ 
     &           &    2 &  2.11e-08 &  6.02e-16 &      \\ 
     &           &    3 &  2.10e-08 &  6.01e-16 &      \\ 
     &           &    4 &  5.67e-08 &  6.00e-16 &      \\ 
     &           &    5 &  2.11e-08 &  1.62e-15 &      \\ 
     &           &    6 &  2.10e-08 &  6.02e-16 &      \\ 
     &           &    7 &  2.10e-08 &  6.01e-16 &      \\ 
     &           &    8 &  5.67e-08 &  6.00e-16 &      \\ 
     &           &    9 &  2.11e-08 &  1.62e-15 &      \\ 
     &           &   10 &  2.10e-08 &  6.01e-16 &      \\ 
1979 &  1.98e+04 &   10 &           &           & iters  \\ 
 \hdashline 
     &           &    1 &  4.64e-08 &  1.58e-09 &      \\ 
     &           &    2 &  3.13e-08 &  1.33e-15 &      \\ 
     &           &    3 &  3.13e-08 &  8.94e-16 &      \\ 
     &           &    4 &  4.65e-08 &  8.92e-16 &      \\ 
     &           &    5 &  3.13e-08 &  1.33e-15 &      \\ 
     &           &    6 &  4.64e-08 &  8.93e-16 &      \\ 
     &           &    7 &  3.13e-08 &  1.33e-15 &      \\ 
     &           &    8 &  3.13e-08 &  8.93e-16 &      \\ 
     &           &    9 &  4.65e-08 &  8.92e-16 &      \\ 
     &           &   10 &  3.13e-08 &  1.33e-15 &      \\ 
1980 &  1.98e+04 &   10 &           &           & iters  \\ 
 \hdashline 
     &           &    1 &  3.30e-08 &  2.16e-09 &      \\ 
     &           &    2 &  3.29e-08 &  9.41e-16 &      \\ 
     &           &    3 &  4.48e-08 &  9.40e-16 &      \\ 
     &           &    4 &  3.29e-08 &  1.28e-15 &      \\ 
     &           &    5 &  4.48e-08 &  9.40e-16 &      \\ 
     &           &    6 &  3.30e-08 &  1.28e-15 &      \\ 
     &           &    7 &  3.29e-08 &  9.41e-16 &      \\ 
     &           &    8 &  4.48e-08 &  9.39e-16 &      \\ 
     &           &    9 &  3.29e-08 &  1.28e-15 &      \\ 
     &           &   10 &  4.48e-08 &  9.40e-16 &      \\ 
1981 &  1.98e+04 &   10 &           &           & iters  \\ 
 \hdashline 
     &           &    1 &  3.49e-08 &  1.14e-09 &      \\ 
     &           &    2 &  4.28e-08 &  9.97e-16 &      \\ 
     &           &    3 &  3.49e-08 &  1.22e-15 &      \\ 
     &           &    4 &  4.28e-08 &  9.97e-16 &      \\ 
     &           &    5 &  3.50e-08 &  1.22e-15 &      \\ 
     &           &    6 &  4.28e-08 &  9.98e-16 &      \\ 
     &           &    7 &  3.50e-08 &  1.22e-15 &      \\ 
     &           &    8 &  3.49e-08 &  9.98e-16 &      \\ 
     &           &    9 &  4.28e-08 &  9.96e-16 &      \\ 
     &           &   10 &  3.49e-08 &  1.22e-15 &      \\ 
1982 &  1.98e+04 &   10 &           &           & iters  \\ 
 \hdashline 
     &           &    1 &  1.21e-08 &  5.77e-11 &      \\ 
     &           &    2 &  1.21e-08 &  3.46e-16 &      \\ 
     &           &    3 &  6.56e-08 &  3.45e-16 &      \\ 
     &           &    4 &  1.22e-08 &  1.87e-15 &      \\ 
     &           &    5 &  1.21e-08 &  3.47e-16 &      \\ 
     &           &    6 &  1.21e-08 &  3.47e-16 &      \\ 
     &           &    7 &  1.21e-08 &  3.46e-16 &      \\ 
     &           &    8 &  1.21e-08 &  3.46e-16 &      \\ 
     &           &    9 &  6.56e-08 &  3.45e-16 &      \\ 
     &           &   10 &  1.22e-08 &  1.87e-15 &      \\ 
1983 &  1.98e+04 &   10 &           &           & iters  \\ 
 \hdashline 
     &           &    1 &  1.47e-08 &  6.30e-11 &      \\ 
     &           &    2 &  1.47e-08 &  4.21e-16 &      \\ 
     &           &    3 &  1.47e-08 &  4.20e-16 &      \\ 
     &           &    4 &  1.47e-08 &  4.19e-16 &      \\ 
     &           &    5 &  6.30e-08 &  4.19e-16 &      \\ 
     &           &    6 &  1.47e-08 &  1.80e-15 &      \\ 
     &           &    7 &  1.47e-08 &  4.21e-16 &      \\ 
     &           &    8 &  1.47e-08 &  4.20e-16 &      \\ 
     &           &    9 &  1.47e-08 &  4.20e-16 &      \\ 
     &           &   10 &  6.30e-08 &  4.19e-16 &      \\ 
1984 &  1.98e+04 &   10 &           &           & iters  \\ 
 \hdashline 
     &           &    1 &  1.16e-08 &  1.70e-09 &      \\ 
     &           &    2 &  1.15e-08 &  3.30e-16 &      \\ 
     &           &    3 &  6.62e-08 &  3.29e-16 &      \\ 
     &           &    4 &  1.16e-08 &  1.89e-15 &      \\ 
     &           &    5 &  1.16e-08 &  3.32e-16 &      \\ 
     &           &    6 &  1.16e-08 &  3.31e-16 &      \\ 
     &           &    7 &  1.16e-08 &  3.31e-16 &      \\ 
     &           &    8 &  1.16e-08 &  3.30e-16 &      \\ 
     &           &    9 &  6.61e-08 &  3.30e-16 &      \\ 
     &           &   10 &  8.93e-08 &  1.89e-15 &      \\ 
1985 &  1.98e+04 &   10 &           &           & iters  \\ 
 \hdashline 
     &           &    1 &  1.43e-08 &  2.71e-09 &      \\ 
     &           &    2 &  1.43e-08 &  4.09e-16 &      \\ 
     &           &    3 &  1.43e-08 &  4.09e-16 &      \\ 
     &           &    4 &  1.43e-08 &  4.08e-16 &      \\ 
     &           &    5 &  1.43e-08 &  4.08e-16 &      \\ 
     &           &    6 &  6.34e-08 &  4.07e-16 &      \\ 
     &           &    7 &  1.43e-08 &  1.81e-15 &      \\ 
     &           &    8 &  1.43e-08 &  4.09e-16 &      \\ 
     &           &    9 &  1.43e-08 &  4.08e-16 &      \\ 
     &           &   10 &  1.43e-08 &  4.08e-16 &      \\ 
1986 &  1.98e+04 &   10 &           &           & iters  \\ 
 \hdashline 
     &           &    1 &  2.53e-08 &  1.17e-09 &      \\ 
     &           &    2 &  2.52e-08 &  7.22e-16 &      \\ 
     &           &    3 &  5.25e-08 &  7.21e-16 &      \\ 
     &           &    4 &  2.53e-08 &  1.50e-15 &      \\ 
     &           &    5 &  2.52e-08 &  7.22e-16 &      \\ 
     &           &    6 &  5.25e-08 &  7.21e-16 &      \\ 
     &           &    7 &  2.53e-08 &  1.50e-15 &      \\ 
     &           &    8 &  2.53e-08 &  7.22e-16 &      \\ 
     &           &    9 &  5.25e-08 &  7.21e-16 &      \\ 
     &           &   10 &  2.53e-08 &  1.50e-15 &      \\ 
1987 &  1.99e+04 &   10 &           &           & iters  \\ 
 \hdashline 
     &           &    1 &  2.52e-08 &  1.36e-09 &      \\ 
     &           &    2 &  5.25e-08 &  7.20e-16 &      \\ 
     &           &    3 &  2.53e-08 &  1.50e-15 &      \\ 
     &           &    4 &  2.52e-08 &  7.21e-16 &      \\ 
     &           &    5 &  5.25e-08 &  7.20e-16 &      \\ 
     &           &    6 &  2.53e-08 &  1.50e-15 &      \\ 
     &           &    7 &  2.52e-08 &  7.21e-16 &      \\ 
     &           &    8 &  5.25e-08 &  7.20e-16 &      \\ 
     &           &    9 &  2.53e-08 &  1.50e-15 &      \\ 
     &           &   10 &  2.52e-08 &  7.21e-16 &      \\ 
1988 &  1.99e+04 &   10 &           &           & iters  \\ 
 \hdashline 
     &           &    1 &  4.84e-08 &  2.67e-09 &      \\ 
     &           &    2 &  2.94e-08 &  1.38e-15 &      \\ 
     &           &    3 &  2.93e-08 &  8.38e-16 &      \\ 
     &           &    4 &  4.84e-08 &  8.37e-16 &      \\ 
     &           &    5 &  2.93e-08 &  1.38e-15 &      \\ 
     &           &    6 &  4.84e-08 &  8.37e-16 &      \\ 
     &           &    7 &  2.94e-08 &  1.38e-15 &      \\ 
     &           &    8 &  2.93e-08 &  8.38e-16 &      \\ 
     &           &    9 &  4.84e-08 &  8.37e-16 &      \\ 
     &           &   10 &  2.94e-08 &  1.38e-15 &      \\ 
1989 &  1.99e+04 &   10 &           &           & iters  \\ 
 \hdashline 
     &           &    1 &  5.19e-09 &  2.50e-09 &      \\ 
     &           &    2 &  5.19e-09 &  1.48e-16 &      \\ 
     &           &    3 &  5.18e-09 &  1.48e-16 &      \\ 
     &           &    4 &  7.25e-08 &  1.48e-16 &      \\ 
     &           &    5 &  8.30e-08 &  2.07e-15 &      \\ 
     &           &    6 &  7.25e-08 &  2.37e-15 &      \\ 
     &           &    7 &  5.26e-09 &  2.07e-15 &      \\ 
     &           &    8 &  5.25e-09 &  1.50e-16 &      \\ 
     &           &    9 &  5.25e-09 &  1.50e-16 &      \\ 
     &           &   10 &  5.24e-09 &  1.50e-16 &      \\ 
1990 &  1.99e+04 &   10 &           &           & iters  \\ 
 \hdashline 
     &           &    1 &  5.94e-08 &  1.25e-10 &      \\ 
     &           &    2 &  1.83e-08 &  1.70e-15 &      \\ 
     &           &    3 &  1.83e-08 &  5.23e-16 &      \\ 
     &           &    4 &  1.83e-08 &  5.23e-16 &      \\ 
     &           &    5 &  5.94e-08 &  5.22e-16 &      \\ 
     &           &    6 &  1.83e-08 &  1.70e-15 &      \\ 
     &           &    7 &  1.83e-08 &  5.24e-16 &      \\ 
     &           &    8 &  1.83e-08 &  5.23e-16 &      \\ 
     &           &    9 &  5.94e-08 &  5.22e-16 &      \\ 
     &           &   10 &  1.84e-08 &  1.70e-15 &      \\ 
1991 &  1.99e+04 &   10 &           &           & iters  \\ 
 \hdashline 
     &           &    1 &  5.22e-08 &  2.53e-09 &      \\ 
     &           &    2 &  2.56e-08 &  1.49e-15 &      \\ 
     &           &    3 &  2.55e-08 &  7.29e-16 &      \\ 
     &           &    4 &  5.22e-08 &  7.28e-16 &      \\ 
     &           &    5 &  2.56e-08 &  1.49e-15 &      \\ 
     &           &    6 &  2.55e-08 &  7.30e-16 &      \\ 
     &           &    7 &  5.22e-08 &  7.28e-16 &      \\ 
     &           &    8 &  2.56e-08 &  1.49e-15 &      \\ 
     &           &    9 &  2.55e-08 &  7.30e-16 &      \\ 
     &           &   10 &  2.55e-08 &  7.29e-16 &      \\ 
1992 &  1.99e+04 &   10 &           &           & iters  \\ 
 \hdashline 
     &           &    1 &  4.66e-08 &  1.59e-09 &      \\ 
     &           &    2 &  3.11e-08 &  1.33e-15 &      \\ 
     &           &    3 &  4.66e-08 &  8.88e-16 &      \\ 
     &           &    4 &  3.11e-08 &  1.33e-15 &      \\ 
     &           &    5 &  4.66e-08 &  8.89e-16 &      \\ 
     &           &    6 &  3.12e-08 &  1.33e-15 &      \\ 
     &           &    7 &  3.11e-08 &  8.90e-16 &      \\ 
     &           &    8 &  4.66e-08 &  8.88e-16 &      \\ 
     &           &    9 &  3.11e-08 &  1.33e-15 &      \\ 
     &           &   10 &  4.66e-08 &  8.89e-16 &      \\ 
1993 &  1.99e+04 &   10 &           &           & iters  \\ 
 \hdashline 
     &           &    1 &  5.68e-08 &  3.88e-10 &      \\ 
     &           &    2 &  2.09e-08 &  1.62e-15 &      \\ 
     &           &    3 &  2.09e-08 &  5.98e-16 &      \\ 
     &           &    4 &  2.09e-08 &  5.97e-16 &      \\ 
     &           &    5 &  5.68e-08 &  5.96e-16 &      \\ 
     &           &    6 &  2.09e-08 &  1.62e-15 &      \\ 
     &           &    7 &  2.09e-08 &  5.97e-16 &      \\ 
     &           &    8 &  2.09e-08 &  5.97e-16 &      \\ 
     &           &    9 &  5.68e-08 &  5.96e-16 &      \\ 
     &           &   10 &  2.09e-08 &  1.62e-15 &      \\ 
1994 &  1.99e+04 &   10 &           &           & iters  \\ 
 \hdashline 
     &           &    1 &  5.80e-10 &  5.65e-10 &      \\ 
     &           &    2 &  5.79e-10 &  1.66e-17 &      \\ 
     &           &    3 &  5.79e-10 &  1.65e-17 &      \\ 
     &           &    4 &  5.78e-10 &  1.65e-17 &      \\ 
     &           &    5 &  5.77e-10 &  1.65e-17 &      \\ 
     &           &    6 &  5.76e-10 &  1.65e-17 &      \\ 
     &           &    7 &  5.75e-10 &  1.64e-17 &      \\ 
     &           &    8 &  5.74e-10 &  1.64e-17 &      \\ 
     &           &    9 &  5.74e-10 &  1.64e-17 &      \\ 
     &           &   10 &  5.73e-10 &  1.64e-17 &      \\ 
1995 &  1.99e+04 &   10 &           &           & iters  \\ 
 \hdashline 
     &           &    1 &  3.56e-08 &  2.19e-09 &      \\ 
     &           &    2 &  4.21e-08 &  1.02e-15 &      \\ 
     &           &    3 &  3.57e-08 &  1.20e-15 &      \\ 
     &           &    4 &  4.21e-08 &  1.02e-15 &      \\ 
     &           &    5 &  3.57e-08 &  1.20e-15 &      \\ 
     &           &    6 &  3.56e-08 &  1.02e-15 &      \\ 
     &           &    7 &  4.21e-08 &  1.02e-15 &      \\ 
     &           &    8 &  3.56e-08 &  1.20e-15 &      \\ 
     &           &    9 &  4.21e-08 &  1.02e-15 &      \\ 
     &           &   10 &  3.56e-08 &  1.20e-15 &      \\ 
1996 &  2.00e+04 &   10 &           &           & iters  \\ 
 \hdashline 
     &           &    1 &  1.47e-08 &  1.40e-09 &      \\ 
     &           &    2 &  1.47e-08 &  4.20e-16 &      \\ 
     &           &    3 &  1.47e-08 &  4.19e-16 &      \\ 
     &           &    4 &  6.30e-08 &  4.19e-16 &      \\ 
     &           &    5 &  1.47e-08 &  1.80e-15 &      \\ 
     &           &    6 &  1.47e-08 &  4.21e-16 &      \\ 
     &           &    7 &  1.47e-08 &  4.20e-16 &      \\ 
     &           &    8 &  1.47e-08 &  4.19e-16 &      \\ 
     &           &    9 &  6.30e-08 &  4.19e-16 &      \\ 
     &           &   10 &  1.47e-08 &  1.80e-15 &      \\ 
1997 &  2.00e+04 &   10 &           &           & iters  \\ 
 \hdashline 
     &           &    1 &  4.33e-08 &  9.10e-10 &      \\ 
     &           &    2 &  3.45e-08 &  1.24e-15 &      \\ 
     &           &    3 &  4.33e-08 &  9.84e-16 &      \\ 
     &           &    4 &  3.45e-08 &  1.23e-15 &      \\ 
     &           &    5 &  4.33e-08 &  9.84e-16 &      \\ 
     &           &    6 &  3.45e-08 &  1.23e-15 &      \\ 
     &           &    7 &  4.32e-08 &  9.84e-16 &      \\ 
     &           &    8 &  3.45e-08 &  1.23e-15 &      \\ 
     &           &    9 &  4.32e-08 &  9.85e-16 &      \\ 
     &           &   10 &  3.45e-08 &  1.23e-15 &      \\ 
1998 &  2.00e+04 &   10 &           &           & iters  \\ 
 \hdashline 
     &           &    1 &  4.04e-08 &  1.83e-09 &      \\ 
     &           &    2 &  3.73e-08 &  1.15e-15 &      \\ 
     &           &    3 &  3.73e-08 &  1.07e-15 &      \\ 
     &           &    4 &  4.05e-08 &  1.06e-15 &      \\ 
     &           &    5 &  3.73e-08 &  1.15e-15 &      \\ 
     &           &    6 &  4.04e-08 &  1.06e-15 &      \\ 
     &           &    7 &  3.73e-08 &  1.15e-15 &      \\ 
     &           &    8 &  4.04e-08 &  1.06e-15 &      \\ 
     &           &    9 &  3.73e-08 &  1.15e-15 &      \\ 
     &           &   10 &  4.04e-08 &  1.06e-15 &      \\ 
1999 &  2.00e+04 &   10 &           &           & iters  \\ 
 \hdashline 
     &           &    1 &  1.44e-08 &  1.93e-09 &      \\ 
     &           &    2 &  6.33e-08 &  4.11e-16 &      \\ 
     &           &    3 &  1.45e-08 &  1.81e-15 &      \\ 
     &           &    4 &  1.44e-08 &  4.13e-16 &      \\ 
     &           &    5 &  1.44e-08 &  4.12e-16 &      \\ 
     &           &    6 &  1.44e-08 &  4.11e-16 &      \\ 
     &           &    7 &  6.33e-08 &  4.11e-16 &      \\ 
     &           &    8 &  1.45e-08 &  1.81e-15 &      \\ 
     &           &    9 &  1.44e-08 &  4.13e-16 &      \\ 
     &           &   10 &  1.44e-08 &  4.12e-16 &      \\ 
2000 &  2.00e+04 &   10 &           &           & iters  \\ 
 \hdashline 
     &           &    1 &  1.13e-08 &  1.06e-09 &      \\ 
     &           &    2 &  1.13e-08 &  3.22e-16 &      \\ 
     &           &    3 &  2.76e-08 &  3.21e-16 &      \\ 
     &           &    4 &  1.13e-08 &  7.88e-16 &      \\ 
     &           &    5 &  1.13e-08 &  3.22e-16 &      \\ 
     &           &    6 &  2.76e-08 &  3.22e-16 &      \\ 
     &           &    7 &  1.13e-08 &  7.87e-16 &      \\ 
     &           &    8 &  1.13e-08 &  3.22e-16 &      \\ 
     &           &    9 &  2.76e-08 &  3.22e-16 &      \\ 
     &           &   10 &  1.13e-08 &  7.87e-16 &      \\ 
2001 &  2.00e+04 &   10 &           &           & iters  \\ 
 \hdashline 
     &           &    1 &  2.96e-08 &  3.40e-10 &      \\ 
     &           &    2 &  4.81e-08 &  8.45e-16 &      \\ 
     &           &    3 &  2.96e-08 &  1.37e-15 &      \\ 
     &           &    4 &  4.81e-08 &  8.45e-16 &      \\ 
     &           &    5 &  2.96e-08 &  1.37e-15 &      \\ 
     &           &    6 &  2.96e-08 &  8.46e-16 &      \\ 
     &           &    7 &  4.81e-08 &  8.45e-16 &      \\ 
     &           &    8 &  2.96e-08 &  1.37e-15 &      \\ 
     &           &    9 &  2.96e-08 &  8.46e-16 &      \\ 
     &           &   10 &  4.81e-08 &  8.45e-16 &      \\ 
2002 &  2.00e+04 &   10 &           &           & iters  \\ 
 \hdashline 
     &           &    1 &  3.34e-09 &  8.88e-10 &      \\ 
     &           &    2 &  3.33e-09 &  9.52e-17 &      \\ 
     &           &    3 &  3.33e-09 &  9.51e-17 &      \\ 
     &           &    4 &  3.32e-09 &  9.49e-17 &      \\ 
     &           &    5 &  3.32e-09 &  9.48e-17 &      \\ 
     &           &    6 &  3.31e-09 &  9.47e-17 &      \\ 
     &           &    7 &  3.31e-09 &  9.45e-17 &      \\ 
     &           &    8 &  3.30e-09 &  9.44e-17 &      \\ 
     &           &    9 &  3.30e-09 &  9.43e-17 &      \\ 
     &           &   10 &  3.29e-09 &  9.41e-17 &      \\ 
2003 &  2.00e+04 &   10 &           &           & iters  \\ 
 \hdashline 
     &           &    1 &  3.30e-08 &  2.11e-09 &      \\ 
     &           &    2 &  3.29e-08 &  9.41e-16 &      \\ 
     &           &    3 &  4.48e-08 &  9.40e-16 &      \\ 
     &           &    4 &  3.29e-08 &  1.28e-15 &      \\ 
     &           &    5 &  4.48e-08 &  9.40e-16 &      \\ 
     &           &    6 &  3.30e-08 &  1.28e-15 &      \\ 
     &           &    7 &  4.48e-08 &  9.41e-16 &      \\ 
     &           &    8 &  3.30e-08 &  1.28e-15 &      \\ 
     &           &    9 &  3.29e-08 &  9.41e-16 &      \\ 
     &           &   10 &  4.48e-08 &  9.40e-16 &      \\ 
2004 &  2.00e+04 &   10 &           &           & iters  \\ 
 \hdashline 
     &           &    1 &  9.60e-09 &  3.48e-09 &      \\ 
     &           &    2 &  9.59e-09 &  2.74e-16 &      \\ 
     &           &    3 &  9.57e-09 &  2.74e-16 &      \\ 
     &           &    4 &  9.56e-09 &  2.73e-16 &      \\ 
     &           &    5 &  6.81e-08 &  2.73e-16 &      \\ 
     &           &    6 &  8.73e-08 &  1.94e-15 &      \\ 
     &           &    7 &  6.82e-08 &  2.49e-15 &      \\ 
     &           &    8 &  9.62e-09 &  1.95e-15 &      \\ 
     &           &    9 &  9.60e-09 &  2.74e-16 &      \\ 
     &           &   10 &  9.59e-09 &  2.74e-16 &      \\ 
2005 &  2.00e+04 &   10 &           &           & iters  \\ 
 \hdashline 
     &           &    1 &  4.48e-08 &  1.79e-09 &      \\ 
     &           &    2 &  3.29e-08 &  1.28e-15 &      \\ 
     &           &    3 &  4.48e-08 &  9.40e-16 &      \\ 
     &           &    4 &  3.29e-08 &  1.28e-15 &      \\ 
     &           &    5 &  4.48e-08 &  9.40e-16 &      \\ 
     &           &    6 &  3.30e-08 &  1.28e-15 &      \\ 
     &           &    7 &  4.48e-08 &  9.41e-16 &      \\ 
     &           &    8 &  3.30e-08 &  1.28e-15 &      \\ 
     &           &    9 &  3.29e-08 &  9.41e-16 &      \\ 
     &           &   10 &  4.48e-08 &  9.40e-16 &      \\ 
2006 &  2.00e+04 &   10 &           &           & iters  \\ 
 \hdashline 
     &           &    1 &  6.18e-08 &  8.34e-10 &      \\ 
     &           &    2 &  1.60e-08 &  1.76e-15 &      \\ 
     &           &    3 &  1.60e-08 &  4.57e-16 &      \\ 
     &           &    4 &  6.17e-08 &  4.57e-16 &      \\ 
     &           &    5 &  1.61e-08 &  1.76e-15 &      \\ 
     &           &    6 &  1.60e-08 &  4.59e-16 &      \\ 
     &           &    7 &  1.60e-08 &  4.58e-16 &      \\ 
     &           &    8 &  1.60e-08 &  4.57e-16 &      \\ 
     &           &    9 &  6.17e-08 &  4.57e-16 &      \\ 
     &           &   10 &  1.61e-08 &  1.76e-15 &      \\ 
2007 &  2.01e+04 &   10 &           &           & iters  \\ 
 \hdashline 
     &           &    1 &  4.45e-08 &  1.90e-09 &      \\ 
     &           &    2 &  3.33e-08 &  1.27e-15 &      \\ 
     &           &    3 &  4.44e-08 &  9.50e-16 &      \\ 
     &           &    4 &  3.33e-08 &  1.27e-15 &      \\ 
     &           &    5 &  3.33e-08 &  9.51e-16 &      \\ 
     &           &    6 &  4.45e-08 &  9.49e-16 &      \\ 
     &           &    7 &  3.33e-08 &  1.27e-15 &      \\ 
     &           &    8 &  4.45e-08 &  9.50e-16 &      \\ 
     &           &    9 &  3.33e-08 &  1.27e-15 &      \\ 
     &           &   10 &  4.44e-08 &  9.50e-16 &      \\ 
2008 &  2.01e+04 &   10 &           &           & iters  \\ 
 \hdashline 
     &           &    1 &  1.99e-08 &  9.16e-10 &      \\ 
     &           &    2 &  1.99e-08 &  5.68e-16 &      \\ 
     &           &    3 &  1.98e-08 &  5.67e-16 &      \\ 
     &           &    4 &  1.98e-08 &  5.66e-16 &      \\ 
     &           &    5 &  5.79e-08 &  5.66e-16 &      \\ 
     &           &    6 &  1.99e-08 &  1.65e-15 &      \\ 
     &           &    7 &  1.98e-08 &  5.67e-16 &      \\ 
     &           &    8 &  1.98e-08 &  5.66e-16 &      \\ 
     &           &    9 &  5.79e-08 &  5.66e-16 &      \\ 
     &           &   10 &  1.99e-08 &  1.65e-15 &      \\ 
2009 &  2.01e+04 &   10 &           &           & iters  \\ 
 \hdashline 
     &           &    1 &  3.63e-08 &  6.85e-10 &      \\ 
     &           &    2 &  3.63e-08 &  1.04e-15 &      \\ 
     &           &    3 &  4.14e-08 &  1.04e-15 &      \\ 
     &           &    4 &  3.63e-08 &  1.18e-15 &      \\ 
     &           &    5 &  4.14e-08 &  1.04e-15 &      \\ 
     &           &    6 &  3.63e-08 &  1.18e-15 &      \\ 
     &           &    7 &  4.14e-08 &  1.04e-15 &      \\ 
     &           &    8 &  3.63e-08 &  1.18e-15 &      \\ 
     &           &    9 &  4.14e-08 &  1.04e-15 &      \\ 
     &           &   10 &  3.63e-08 &  1.18e-15 &      \\ 
2010 &  2.01e+04 &   10 &           &           & iters  \\ 
 \hdashline 
     &           &    1 &  2.34e-08 &  4.08e-10 &      \\ 
     &           &    2 &  2.34e-08 &  6.68e-16 &      \\ 
     &           &    3 &  2.34e-08 &  6.67e-16 &      \\ 
     &           &    4 &  5.44e-08 &  6.66e-16 &      \\ 
     &           &    5 &  2.34e-08 &  1.55e-15 &      \\ 
     &           &    6 &  2.34e-08 &  6.68e-16 &      \\ 
     &           &    7 &  2.33e-08 &  6.67e-16 &      \\ 
     &           &    8 &  5.44e-08 &  6.66e-16 &      \\ 
     &           &    9 &  2.34e-08 &  1.55e-15 &      \\ 
     &           &   10 &  2.33e-08 &  6.67e-16 &      \\ 
2011 &  2.01e+04 &   10 &           &           & iters  \\ 
 \hdashline 
     &           &    1 &  6.55e-08 &  4.91e-10 &      \\ 
     &           &    2 &  1.23e-08 &  1.87e-15 &      \\ 
     &           &    3 &  1.23e-08 &  3.51e-16 &      \\ 
     &           &    4 &  6.54e-08 &  3.51e-16 &      \\ 
     &           &    5 &  1.24e-08 &  1.87e-15 &      \\ 
     &           &    6 &  1.23e-08 &  3.53e-16 &      \\ 
     &           &    7 &  1.23e-08 &  3.52e-16 &      \\ 
     &           &    8 &  1.23e-08 &  3.52e-16 &      \\ 
     &           &    9 &  1.23e-08 &  3.51e-16 &      \\ 
     &           &   10 &  6.54e-08 &  3.51e-16 &      \\ 
2012 &  2.01e+04 &   10 &           &           & iters  \\ 
 \hdashline 
     &           &    1 &  6.34e-08 &  1.60e-09 &      \\ 
     &           &    2 &  1.44e-08 &  1.81e-15 &      \\ 
     &           &    3 &  6.33e-08 &  4.11e-16 &      \\ 
     &           &    4 &  1.45e-08 &  1.81e-15 &      \\ 
     &           &    5 &  1.44e-08 &  4.13e-16 &      \\ 
     &           &    6 &  1.44e-08 &  4.12e-16 &      \\ 
     &           &    7 &  1.44e-08 &  4.12e-16 &      \\ 
     &           &    8 &  6.33e-08 &  4.11e-16 &      \\ 
     &           &    9 &  1.45e-08 &  1.81e-15 &      \\ 
     &           &   10 &  1.45e-08 &  4.13e-16 &      \\ 
2013 &  2.01e+04 &   10 &           &           & iters  \\ 
 \hdashline 
     &           &    1 &  7.16e-08 &  1.59e-09 &      \\ 
     &           &    2 &  6.17e-09 &  2.04e-15 &      \\ 
     &           &    3 &  6.17e-09 &  1.76e-16 &      \\ 
     &           &    4 &  6.16e-09 &  1.76e-16 &      \\ 
     &           &    5 &  6.15e-09 &  1.76e-16 &      \\ 
     &           &    6 &  6.14e-09 &  1.75e-16 &      \\ 
     &           &    7 &  6.13e-09 &  1.75e-16 &      \\ 
     &           &    8 &  6.12e-09 &  1.75e-16 &      \\ 
     &           &    9 &  6.11e-09 &  1.75e-16 &      \\ 
     &           &   10 &  6.10e-09 &  1.74e-16 &      \\ 
2014 &  2.01e+04 &   10 &           &           & iters  \\ 
 \hdashline 
     &           &    1 &  5.17e-08 &  1.38e-09 &      \\ 
     &           &    2 &  2.61e-08 &  1.47e-15 &      \\ 
     &           &    3 &  2.61e-08 &  7.45e-16 &      \\ 
     &           &    4 &  5.17e-08 &  7.44e-16 &      \\ 
     &           &    5 &  2.61e-08 &  1.47e-15 &      \\ 
     &           &    6 &  2.61e-08 &  7.45e-16 &      \\ 
     &           &    7 &  5.17e-08 &  7.44e-16 &      \\ 
     &           &    8 &  2.61e-08 &  1.47e-15 &      \\ 
     &           &    9 &  2.61e-08 &  7.45e-16 &      \\ 
     &           &   10 &  5.17e-08 &  7.44e-16 &      \\ 
2015 &  2.01e+04 &   10 &           &           & iters  \\ 
 \hdashline 
     &           &    1 &  1.83e-08 &  2.37e-09 &      \\ 
     &           &    2 &  1.83e-08 &  5.23e-16 &      \\ 
     &           &    3 &  1.83e-08 &  5.23e-16 &      \\ 
     &           &    4 &  5.94e-08 &  5.22e-16 &      \\ 
     &           &    5 &  1.83e-08 &  1.70e-15 &      \\ 
     &           &    6 &  1.83e-08 &  5.24e-16 &      \\ 
     &           &    7 &  1.83e-08 &  5.23e-16 &      \\ 
     &           &    8 &  1.83e-08 &  5.22e-16 &      \\ 
     &           &    9 &  5.94e-08 &  5.21e-16 &      \\ 
     &           &   10 &  1.83e-08 &  1.70e-15 &      \\ 
2016 &  2.02e+04 &   10 &           &           & iters  \\ 
 \hdashline 
     &           &    1 &  6.33e-09 &  3.14e-09 &      \\ 
     &           &    2 &  6.33e-09 &  1.81e-16 &      \\ 
     &           &    3 &  6.32e-09 &  1.81e-16 &      \\ 
     &           &    4 &  6.31e-09 &  1.80e-16 &      \\ 
     &           &    5 &  7.14e-08 &  1.80e-16 &      \\ 
     &           &    6 &  4.52e-08 &  2.04e-15 &      \\ 
     &           &    7 &  6.34e-09 &  1.29e-15 &      \\ 
     &           &    8 &  6.33e-09 &  1.81e-16 &      \\ 
     &           &    9 &  6.32e-09 &  1.81e-16 &      \\ 
     &           &   10 &  6.31e-09 &  1.80e-16 &      \\ 
2017 &  2.02e+04 &   10 &           &           & iters  \\ 
 \hdashline 
     &           &    1 &  2.29e-08 &  2.28e-09 &      \\ 
     &           &    2 &  2.29e-08 &  6.54e-16 &      \\ 
     &           &    3 &  1.60e-08 &  6.53e-16 &      \\ 
     &           &    4 &  1.60e-08 &  4.56e-16 &      \\ 
     &           &    5 &  2.29e-08 &  4.56e-16 &      \\ 
     &           &    6 &  1.60e-08 &  6.53e-16 &      \\ 
     &           &    7 &  2.29e-08 &  4.56e-16 &      \\ 
     &           &    8 &  1.60e-08 &  6.53e-16 &      \\ 
     &           &    9 &  2.29e-08 &  4.56e-16 &      \\ 
     &           &   10 &  1.60e-08 &  6.53e-16 &      \\ 
2018 &  2.02e+04 &   10 &           &           & iters  \\ 
 \hdashline 
     &           &    1 &  2.19e-08 &  2.50e-10 &      \\ 
     &           &    2 &  5.58e-08 &  6.24e-16 &      \\ 
     &           &    3 &  2.19e-08 &  1.59e-15 &      \\ 
     &           &    4 &  2.19e-08 &  6.26e-16 &      \\ 
     &           &    5 &  5.58e-08 &  6.25e-16 &      \\ 
     &           &    6 &  2.19e-08 &  1.59e-15 &      \\ 
     &           &    7 &  2.19e-08 &  6.26e-16 &      \\ 
     &           &    8 &  2.19e-08 &  6.25e-16 &      \\ 
     &           &    9 &  5.58e-08 &  6.25e-16 &      \\ 
     &           &   10 &  2.19e-08 &  1.59e-15 &      \\ 
2019 &  2.02e+04 &   10 &           &           & iters  \\ 
 \hdashline 
     &           &    1 &  1.26e-08 &  1.43e-09 &      \\ 
     &           &    2 &  1.25e-08 &  3.59e-16 &      \\ 
     &           &    3 &  1.25e-08 &  3.58e-16 &      \\ 
     &           &    4 &  1.25e-08 &  3.58e-16 &      \\ 
     &           &    5 &  6.52e-08 &  3.57e-16 &      \\ 
     &           &    6 &  1.26e-08 &  1.86e-15 &      \\ 
     &           &    7 &  1.26e-08 &  3.59e-16 &      \\ 
     &           &    8 &  1.26e-08 &  3.59e-16 &      \\ 
     &           &    9 &  1.25e-08 &  3.58e-16 &      \\ 
     &           &   10 &  1.25e-08 &  3.58e-16 &      \\ 
2020 &  2.02e+04 &   10 &           &           & iters  \\ 
 \hdashline 
     &           &    1 &  4.07e-08 &  4.95e-10 &      \\ 
     &           &    2 &  3.70e-08 &  1.16e-15 &      \\ 
     &           &    3 &  4.07e-08 &  1.06e-15 &      \\ 
     &           &    4 &  3.70e-08 &  1.16e-15 &      \\ 
     &           &    5 &  4.07e-08 &  1.06e-15 &      \\ 
     &           &    6 &  3.70e-08 &  1.16e-15 &      \\ 
     &           &    7 &  4.07e-08 &  1.06e-15 &      \\ 
     &           &    8 &  3.70e-08 &  1.16e-15 &      \\ 
     &           &    9 &  4.07e-08 &  1.06e-15 &      \\ 
     &           &   10 &  3.70e-08 &  1.16e-15 &      \\ 
2021 &  2.02e+04 &   10 &           &           & iters  \\ 
 \hdashline 
     &           &    1 &  2.74e-08 &  8.69e-10 &      \\ 
     &           &    2 &  5.04e-08 &  7.81e-16 &      \\ 
     &           &    3 &  2.74e-08 &  1.44e-15 &      \\ 
     &           &    4 &  2.74e-08 &  7.82e-16 &      \\ 
     &           &    5 &  5.04e-08 &  7.81e-16 &      \\ 
     &           &    6 &  2.74e-08 &  1.44e-15 &      \\ 
     &           &    7 &  2.74e-08 &  7.82e-16 &      \\ 
     &           &    8 &  5.04e-08 &  7.81e-16 &      \\ 
     &           &    9 &  2.74e-08 &  1.44e-15 &      \\ 
     &           &   10 &  2.74e-08 &  7.82e-16 &      \\ 
2022 &  2.02e+04 &   10 &           &           & iters  \\ 
 \hdashline 
     &           &    1 &  2.84e-08 &  1.62e-09 &      \\ 
     &           &    2 &  4.94e-08 &  8.10e-16 &      \\ 
     &           &    3 &  2.84e-08 &  1.41e-15 &      \\ 
     &           &    4 &  2.84e-08 &  8.11e-16 &      \\ 
     &           &    5 &  4.94e-08 &  8.10e-16 &      \\ 
     &           &    6 &  2.84e-08 &  1.41e-15 &      \\ 
     &           &    7 &  2.84e-08 &  8.10e-16 &      \\ 
     &           &    8 &  4.94e-08 &  8.09e-16 &      \\ 
     &           &    9 &  2.84e-08 &  1.41e-15 &      \\ 
     &           &   10 &  2.83e-08 &  8.10e-16 &      \\ 
2023 &  2.02e+04 &   10 &           &           & iters  \\ 
 \hdashline 
     &           &    1 &  5.50e-08 &  2.09e-10 &      \\ 
     &           &    2 &  2.28e-08 &  1.57e-15 &      \\ 
     &           &    3 &  2.27e-08 &  6.50e-16 &      \\ 
     &           &    4 &  5.50e-08 &  6.49e-16 &      \\ 
     &           &    5 &  2.28e-08 &  1.57e-15 &      \\ 
     &           &    6 &  2.27e-08 &  6.50e-16 &      \\ 
     &           &    7 &  2.27e-08 &  6.49e-16 &      \\ 
     &           &    8 &  5.50e-08 &  6.48e-16 &      \\ 
     &           &    9 &  2.28e-08 &  1.57e-15 &      \\ 
     &           &   10 &  2.27e-08 &  6.49e-16 &      \\ 
2024 &  2.02e+04 &   10 &           &           & iters  \\ 
 \hdashline 
     &           &    1 &  4.48e-08 &  4.99e-09 &      \\ 
     &           &    2 &  3.29e-08 &  1.28e-15 &      \\ 
     &           &    3 &  4.48e-08 &  9.40e-16 &      \\ 
     &           &    4 &  3.29e-08 &  1.28e-15 &      \\ 
     &           &    5 &  3.29e-08 &  9.40e-16 &      \\ 
     &           &    6 &  4.48e-08 &  9.39e-16 &      \\ 
     &           &    7 &  3.29e-08 &  1.28e-15 &      \\ 
     &           &    8 &  4.48e-08 &  9.39e-16 &      \\ 
     &           &    9 &  3.29e-08 &  1.28e-15 &      \\ 
     &           &   10 &  4.48e-08 &  9.40e-16 &      \\ 
2025 &  2.02e+04 &   10 &           &           & iters  \\ 
 \hdashline 
     &           &    1 &  6.49e-09 &  7.66e-09 &      \\ 
     &           &    2 &  7.12e-08 &  1.85e-16 &      \\ 
     &           &    3 &  8.43e-08 &  2.03e-15 &      \\ 
     &           &    4 &  7.12e-08 &  2.41e-15 &      \\ 
     &           &    5 &  6.57e-09 &  2.03e-15 &      \\ 
     &           &    6 &  6.56e-09 &  1.87e-16 &      \\ 
     &           &    7 &  6.55e-09 &  1.87e-16 &      \\ 
     &           &    8 &  6.54e-09 &  1.87e-16 &      \\ 
     &           &    9 &  6.53e-09 &  1.87e-16 &      \\ 
     &           &   10 &  6.52e-09 &  1.86e-16 &      \\ 
2026 &  2.02e+04 &   10 &           &           & iters  \\ 
 \hdashline 
     &           &    1 &  4.20e-08 &  2.57e-09 &      \\ 
     &           &    2 &  3.58e-08 &  1.20e-15 &      \\ 
     &           &    3 &  4.20e-08 &  1.02e-15 &      \\ 
     &           &    4 &  3.58e-08 &  1.20e-15 &      \\ 
     &           &    5 &  4.20e-08 &  1.02e-15 &      \\ 
     &           &    6 &  3.58e-08 &  1.20e-15 &      \\ 
     &           &    7 &  4.19e-08 &  1.02e-15 &      \\ 
     &           &    8 &  3.58e-08 &  1.20e-15 &      \\ 
     &           &    9 &  4.19e-08 &  1.02e-15 &      \\ 
     &           &   10 &  3.58e-08 &  1.20e-15 &      \\ 
2027 &  2.03e+04 &   10 &           &           & iters  \\ 
 \hdashline 
     &           &    1 &  1.86e-08 &  2.18e-09 &      \\ 
     &           &    2 &  5.91e-08 &  5.32e-16 &      \\ 
     &           &    3 &  1.87e-08 &  1.69e-15 &      \\ 
     &           &    4 &  1.87e-08 &  5.34e-16 &      \\ 
     &           &    5 &  1.87e-08 &  5.33e-16 &      \\ 
     &           &    6 &  5.91e-08 &  5.32e-16 &      \\ 
     &           &    7 &  1.87e-08 &  1.69e-15 &      \\ 
     &           &    8 &  1.87e-08 &  5.34e-16 &      \\ 
     &           &    9 &  1.87e-08 &  5.33e-16 &      \\ 
     &           &   10 &  5.91e-08 &  5.32e-16 &      \\ 
2028 &  2.03e+04 &   10 &           &           & iters  \\ 
 \hdashline 
     &           &    1 &  4.65e-08 &  6.35e-10 &      \\ 
     &           &    2 &  3.13e-08 &  1.33e-15 &      \\ 
     &           &    3 &  4.64e-08 &  8.93e-16 &      \\ 
     &           &    4 &  3.13e-08 &  1.33e-15 &      \\ 
     &           &    5 &  4.64e-08 &  8.94e-16 &      \\ 
     &           &    6 &  3.13e-08 &  1.32e-15 &      \\ 
     &           &    7 &  3.13e-08 &  8.94e-16 &      \\ 
     &           &    8 &  4.64e-08 &  8.93e-16 &      \\ 
     &           &    9 &  3.13e-08 &  1.33e-15 &      \\ 
     &           &   10 &  4.64e-08 &  8.94e-16 &      \\ 
2029 &  2.03e+04 &   10 &           &           & iters  \\ 
 \hdashline 
     &           &    1 &  6.84e-08 &  9.48e-10 &      \\ 
     &           &    2 &  9.37e-09 &  1.95e-15 &      \\ 
     &           &    3 &  9.35e-09 &  2.67e-16 &      \\ 
     &           &    4 &  9.34e-09 &  2.67e-16 &      \\ 
     &           &    5 &  9.33e-09 &  2.67e-16 &      \\ 
     &           &    6 &  9.31e-09 &  2.66e-16 &      \\ 
     &           &    7 &  9.30e-09 &  2.66e-16 &      \\ 
     &           &    8 &  6.84e-08 &  2.65e-16 &      \\ 
     &           &    9 &  9.39e-09 &  1.95e-15 &      \\ 
     &           &   10 &  9.37e-09 &  2.68e-16 &      \\ 
2030 &  2.03e+04 &   10 &           &           & iters  \\ 
 \hdashline 
     &           &    1 &  8.19e-08 &  6.83e-10 &      \\ 
     &           &    2 &  7.36e-08 &  2.34e-15 &      \\ 
     &           &    3 &  8.19e-08 &  2.10e-15 &      \\ 
     &           &    4 &  7.36e-08 &  2.34e-15 &      \\ 
     &           &    5 &  8.19e-08 &  2.10e-15 &      \\ 
     &           &    6 &  7.36e-08 &  2.34e-15 &      \\ 
     &           &    7 &  4.16e-09 &  2.10e-15 &      \\ 
     &           &    8 &  4.16e-09 &  1.19e-16 &      \\ 
     &           &    9 &  4.15e-09 &  1.19e-16 &      \\ 
     &           &   10 &  4.14e-09 &  1.18e-16 &      \\ 
2031 &  2.03e+04 &   10 &           &           & iters  \\ 
 \hdashline 
     &           &    1 &  4.89e-09 &  3.44e-10 &      \\ 
     &           &    2 &  4.88e-09 &  1.40e-16 &      \\ 
     &           &    3 &  4.88e-09 &  1.39e-16 &      \\ 
     &           &    4 &  4.87e-09 &  1.39e-16 &      \\ 
     &           &    5 &  4.86e-09 &  1.39e-16 &      \\ 
     &           &    6 &  7.28e-08 &  1.39e-16 &      \\ 
     &           &    7 &  8.26e-08 &  2.08e-15 &      \\ 
     &           &    8 &  7.28e-08 &  2.36e-15 &      \\ 
     &           &    9 &  4.94e-09 &  2.08e-15 &      \\ 
     &           &   10 &  4.94e-09 &  1.41e-16 &      \\ 
2032 &  2.03e+04 &   10 &           &           & iters  \\ 
 \hdashline 
     &           &    1 &  2.17e-08 &  1.16e-09 &      \\ 
     &           &    2 &  2.17e-08 &  6.20e-16 &      \\ 
     &           &    3 &  5.60e-08 &  6.19e-16 &      \\ 
     &           &    4 &  2.17e-08 &  1.60e-15 &      \\ 
     &           &    5 &  2.17e-08 &  6.20e-16 &      \\ 
     &           &    6 &  2.17e-08 &  6.19e-16 &      \\ 
     &           &    7 &  5.60e-08 &  6.19e-16 &      \\ 
     &           &    8 &  2.17e-08 &  1.60e-15 &      \\ 
     &           &    9 &  2.17e-08 &  6.20e-16 &      \\ 
     &           &   10 &  5.60e-08 &  6.19e-16 &      \\ 
2033 &  2.03e+04 &   10 &           &           & iters  \\ 
 \hdashline 
     &           &    1 &  6.40e-08 &  3.58e-09 &      \\ 
     &           &    2 &  1.38e-08 &  1.83e-15 &      \\ 
     &           &    3 &  1.38e-08 &  3.94e-16 &      \\ 
     &           &    4 &  1.37e-08 &  3.93e-16 &      \\ 
     &           &    5 &  1.37e-08 &  3.92e-16 &      \\ 
     &           &    6 &  1.37e-08 &  3.92e-16 &      \\ 
     &           &    7 &  6.40e-08 &  3.91e-16 &      \\ 
     &           &    8 &  1.38e-08 &  1.83e-15 &      \\ 
     &           &    9 &  1.38e-08 &  3.93e-16 &      \\ 
     &           &   10 &  1.37e-08 &  3.93e-16 &      \\ 
2034 &  2.03e+04 &   10 &           &           & iters  \\ 
 \hdashline 
     &           &    1 &  3.14e-08 &  2.28e-09 &      \\ 
     &           &    2 &  3.13e-08 &  8.96e-16 &      \\ 
     &           &    3 &  4.64e-08 &  8.95e-16 &      \\ 
     &           &    4 &  3.14e-08 &  1.32e-15 &      \\ 
     &           &    5 &  3.13e-08 &  8.95e-16 &      \\ 
     &           &    6 &  4.64e-08 &  8.94e-16 &      \\ 
     &           &    7 &  3.13e-08 &  1.32e-15 &      \\ 
     &           &    8 &  4.64e-08 &  8.95e-16 &      \\ 
     &           &    9 &  3.14e-08 &  1.32e-15 &      \\ 
     &           &   10 &  3.13e-08 &  8.95e-16 &      \\ 
2035 &  2.03e+04 &   10 &           &           & iters  \\ 
 \hdashline 
     &           &    1 &  2.46e-08 &  1.41e-09 &      \\ 
     &           &    2 &  2.46e-08 &  7.02e-16 &      \\ 
     &           &    3 &  2.45e-08 &  7.01e-16 &      \\ 
     &           &    4 &  2.45e-08 &  7.00e-16 &      \\ 
     &           &    5 &  5.32e-08 &  6.99e-16 &      \\ 
     &           &    6 &  2.45e-08 &  1.52e-15 &      \\ 
     &           &    7 &  2.45e-08 &  7.00e-16 &      \\ 
     &           &    8 &  2.45e-08 &  6.99e-16 &      \\ 
     &           &    9 &  5.33e-08 &  6.98e-16 &      \\ 
     &           &   10 &  2.45e-08 &  1.52e-15 &      \\ 
2036 &  2.04e+04 &   10 &           &           & iters  \\ 
 \hdashline 
     &           &    1 &  7.98e-08 &  2.02e-09 &      \\ 
     &           &    2 &  7.57e-08 &  2.28e-15 &      \\ 
     &           &    3 &  7.97e-08 &  2.16e-15 &      \\ 
     &           &    4 &  7.57e-08 &  2.28e-15 &      \\ 
     &           &    5 &  7.97e-08 &  2.16e-15 &      \\ 
     &           &    6 &  7.57e-08 &  2.28e-15 &      \\ 
     &           &    7 &  7.97e-08 &  2.16e-15 &      \\ 
     &           &    8 &  7.58e-08 &  2.28e-15 &      \\ 
     &           &    9 &  2.04e-09 &  2.16e-15 &      \\ 
     &           &   10 &  2.04e-09 &  5.83e-17 &      \\ 
2037 &  2.04e+04 &   10 &           &           & iters  \\ 
 \hdashline 
     &           &    1 &  1.47e-08 &  4.36e-10 &      \\ 
     &           &    2 &  1.47e-08 &  4.20e-16 &      \\ 
     &           &    3 &  6.30e-08 &  4.19e-16 &      \\ 
     &           &    4 &  1.47e-08 &  1.80e-15 &      \\ 
     &           &    5 &  1.47e-08 &  4.21e-16 &      \\ 
     &           &    6 &  1.47e-08 &  4.20e-16 &      \\ 
     &           &    7 &  1.47e-08 &  4.20e-16 &      \\ 
     &           &    8 &  6.30e-08 &  4.19e-16 &      \\ 
     &           &    9 &  1.48e-08 &  1.80e-15 &      \\ 
     &           &   10 &  1.47e-08 &  4.21e-16 &      \\ 
2038 &  2.04e+04 &   10 &           &           & iters  \\ 
 \hdashline 
     &           &    1 &  8.26e-09 &  2.69e-09 &      \\ 
     &           &    2 &  8.25e-09 &  2.36e-16 &      \\ 
     &           &    3 &  8.24e-09 &  2.35e-16 &      \\ 
     &           &    4 &  8.22e-09 &  2.35e-16 &      \\ 
     &           &    5 &  8.21e-09 &  2.35e-16 &      \\ 
     &           &    6 &  8.20e-09 &  2.34e-16 &      \\ 
     &           &    7 &  8.19e-09 &  2.34e-16 &      \\ 
     &           &    8 &  8.18e-09 &  2.34e-16 &      \\ 
     &           &    9 &  6.95e-08 &  2.33e-16 &      \\ 
     &           &   10 &  8.26e-09 &  1.98e-15 &      \\ 
2039 &  2.04e+04 &   10 &           &           & iters  \\ 
 \hdashline 
     &           &    1 &  4.68e-08 &  4.69e-09 &      \\ 
     &           &    2 &  3.09e-08 &  1.34e-15 &      \\ 
     &           &    3 &  3.09e-08 &  8.83e-16 &      \\ 
     &           &    4 &  4.68e-08 &  8.81e-16 &      \\ 
     &           &    5 &  3.09e-08 &  1.34e-15 &      \\ 
     &           &    6 &  3.09e-08 &  8.82e-16 &      \\ 
     &           &    7 &  4.69e-08 &  8.81e-16 &      \\ 
     &           &    8 &  3.09e-08 &  1.34e-15 &      \\ 
     &           &    9 &  4.68e-08 &  8.81e-16 &      \\ 
     &           &   10 &  3.09e-08 &  1.34e-15 &      \\ 
2040 &  2.04e+04 &   10 &           &           & iters  \\ 
 \hdashline 
     &           &    1 &  1.61e-10 &  1.27e-09 &      \\ 
     &           &    2 &  1.60e-10 &  4.58e-18 &      \\ 
     &           &    3 &  1.60e-10 &  4.58e-18 &      \\ 
     &           &    4 &  1.60e-10 &  4.57e-18 &      \\ 
     &           &    5 &  1.60e-10 &  4.56e-18 &      \\ 
     &           &    6 &  1.59e-10 &  4.56e-18 &      \\ 
     &           &    7 &  1.59e-10 &  4.55e-18 &      \\ 
     &           &    8 &  1.59e-10 &  4.54e-18 &      \\ 
     &           &    9 &  1.59e-10 &  4.54e-18 &      \\ 
     &           &   10 &  1.59e-10 &  4.53e-18 &      \\ 
2041 &  2.04e+04 &   10 &           &           & iters  \\ 
 \hdashline 
     &           &    1 &  1.45e-08 &  1.43e-09 &      \\ 
     &           &    2 &  2.43e-08 &  4.14e-16 &      \\ 
     &           &    3 &  1.45e-08 &  6.95e-16 &      \\ 
     &           &    4 &  2.43e-08 &  4.15e-16 &      \\ 
     &           &    5 &  1.45e-08 &  6.94e-16 &      \\ 
     &           &    6 &  1.45e-08 &  4.15e-16 &      \\ 
     &           &    7 &  2.43e-08 &  4.15e-16 &      \\ 
     &           &    8 &  1.45e-08 &  6.95e-16 &      \\ 
     &           &    9 &  1.45e-08 &  4.15e-16 &      \\ 
     &           &   10 &  2.43e-08 &  4.14e-16 &      \\ 
2042 &  2.04e+04 &   10 &           &           & iters  \\ 
 \hdashline 
     &           &    1 &  2.16e-08 &  1.00e-11 &      \\ 
     &           &    2 &  5.61e-08 &  6.16e-16 &      \\ 
     &           &    3 &  2.16e-08 &  1.60e-15 &      \\ 
     &           &    4 &  2.16e-08 &  6.17e-16 &      \\ 
     &           &    5 &  2.16e-08 &  6.16e-16 &      \\ 
     &           &    6 &  5.62e-08 &  6.15e-16 &      \\ 
     &           &    7 &  2.16e-08 &  1.60e-15 &      \\ 
     &           &    8 &  2.16e-08 &  6.17e-16 &      \\ 
     &           &    9 &  5.61e-08 &  6.16e-16 &      \\ 
     &           &   10 &  2.16e-08 &  1.60e-15 &      \\ 
2043 &  2.04e+04 &   10 &           &           & iters  \\ 
 \hdashline 
     &           &    1 &  9.55e-09 &  7.78e-10 &      \\ 
     &           &    2 &  9.54e-09 &  2.73e-16 &      \\ 
     &           &    3 &  9.52e-09 &  2.72e-16 &      \\ 
     &           &    4 &  9.51e-09 &  2.72e-16 &      \\ 
     &           &    5 &  6.82e-08 &  2.71e-16 &      \\ 
     &           &    6 &  4.84e-08 &  1.95e-15 &      \\ 
     &           &    7 &  9.53e-09 &  1.38e-15 &      \\ 
     &           &    8 &  9.51e-09 &  2.72e-16 &      \\ 
     &           &    9 &  6.82e-08 &  2.71e-16 &      \\ 
     &           &   10 &  4.84e-08 &  1.95e-15 &      \\ 
2044 &  2.04e+04 &   10 &           &           & iters  \\ 
 \hdashline 
     &           &    1 &  5.32e-08 &  5.09e-10 &      \\ 
     &           &    2 &  2.46e-08 &  1.52e-15 &      \\ 
     &           &    3 &  2.45e-08 &  7.01e-16 &      \\ 
     &           &    4 &  5.32e-08 &  7.00e-16 &      \\ 
     &           &    5 &  2.46e-08 &  1.52e-15 &      \\ 
     &           &    6 &  2.45e-08 &  7.02e-16 &      \\ 
     &           &    7 &  5.32e-08 &  7.01e-16 &      \\ 
     &           &    8 &  2.46e-08 &  1.52e-15 &      \\ 
     &           &    9 &  2.46e-08 &  7.02e-16 &      \\ 
     &           &   10 &  5.32e-08 &  7.01e-16 &      \\ 
2045 &  2.04e+04 &   10 &           &           & iters  \\ 
 \hdashline 
     &           &    1 &  4.92e-08 &  5.57e-10 &      \\ 
     &           &    2 &  2.86e-08 &  1.40e-15 &      \\ 
     &           &    3 &  4.91e-08 &  8.16e-16 &      \\ 
     &           &    4 &  2.86e-08 &  1.40e-15 &      \\ 
     &           &    5 &  4.91e-08 &  8.17e-16 &      \\ 
     &           &    6 &  2.87e-08 &  1.40e-15 &      \\ 
     &           &    7 &  2.86e-08 &  8.18e-16 &      \\ 
     &           &    8 &  4.91e-08 &  8.17e-16 &      \\ 
     &           &    9 &  2.86e-08 &  1.40e-15 &      \\ 
     &           &   10 &  2.86e-08 &  8.18e-16 &      \\ 
2046 &  2.04e+04 &   10 &           &           & iters  \\ 
 \hdashline 
     &           &    1 &  9.94e-09 &  7.53e-10 &      \\ 
     &           &    2 &  9.93e-09 &  2.84e-16 &      \\ 
     &           &    3 &  6.78e-08 &  2.83e-16 &      \\ 
     &           &    4 &  1.00e-08 &  1.93e-15 &      \\ 
     &           &    5 &  9.99e-09 &  2.86e-16 &      \\ 
     &           &    6 &  9.98e-09 &  2.85e-16 &      \\ 
     &           &    7 &  9.97e-09 &  2.85e-16 &      \\ 
     &           &    8 &  9.95e-09 &  2.84e-16 &      \\ 
     &           &    9 &  9.94e-09 &  2.84e-16 &      \\ 
     &           &   10 &  6.78e-08 &  2.84e-16 &      \\ 
2047 &  2.05e+04 &   10 &           &           & iters  \\ 
 \hdashline 
     &           &    1 &  9.50e-09 &  9.36e-10 &      \\ 
     &           &    2 &  9.49e-09 &  2.71e-16 &      \\ 
     &           &    3 &  9.47e-09 &  2.71e-16 &      \\ 
     &           &    4 &  9.46e-09 &  2.70e-16 &      \\ 
     &           &    5 &  9.45e-09 &  2.70e-16 &      \\ 
     &           &    6 &  2.94e-08 &  2.70e-16 &      \\ 
     &           &    7 &  9.47e-09 &  8.39e-16 &      \\ 
     &           &    8 &  9.46e-09 &  2.70e-16 &      \\ 
     &           &    9 &  9.45e-09 &  2.70e-16 &      \\ 
     &           &   10 &  2.94e-08 &  2.70e-16 &      \\ 
2048 &  2.05e+04 &   10 &           &           & iters  \\ 
 \hdashline 
     &           &    1 &  5.18e-08 &  1.64e-10 &      \\ 
     &           &    2 &  2.60e-08 &  1.48e-15 &      \\ 
     &           &    3 &  2.59e-08 &  7.41e-16 &      \\ 
     &           &    4 &  5.18e-08 &  7.40e-16 &      \\ 
     &           &    5 &  2.60e-08 &  1.48e-15 &      \\ 
     &           &    6 &  2.59e-08 &  7.41e-16 &      \\ 
     &           &    7 &  5.18e-08 &  7.40e-16 &      \\ 
     &           &    8 &  2.60e-08 &  1.48e-15 &      \\ 
     &           &    9 &  2.59e-08 &  7.41e-16 &      \\ 
     &           &   10 &  5.18e-08 &  7.40e-16 &      \\ 
2049 &  2.05e+04 &   10 &           &           & iters  \\ 
 \hdashline 
     &           &    1 &  3.42e-08 &  1.23e-09 &      \\ 
     &           &    2 &  3.42e-08 &  9.77e-16 &      \\ 
     &           &    3 &  4.36e-08 &  9.75e-16 &      \\ 
     &           &    4 &  3.42e-08 &  1.24e-15 &      \\ 
     &           &    5 &  4.36e-08 &  9.76e-16 &      \\ 
     &           &    6 &  3.42e-08 &  1.24e-15 &      \\ 
     &           &    7 &  4.35e-08 &  9.76e-16 &      \\ 
     &           &    8 &  3.42e-08 &  1.24e-15 &      \\ 
     &           &    9 &  3.42e-08 &  9.76e-16 &      \\ 
     &           &   10 &  4.36e-08 &  9.75e-16 &      \\ 
2050 &  2.05e+04 &   10 &           &           & iters  \\ 
 \hdashline 
     &           &    1 &  4.16e-09 &  1.39e-09 &      \\ 
     &           &    2 &  4.15e-09 &  1.19e-16 &      \\ 
     &           &    3 &  4.15e-09 &  1.19e-16 &      \\ 
     &           &    4 &  3.47e-08 &  1.18e-16 &      \\ 
     &           &    5 &  4.19e-09 &  9.90e-16 &      \\ 
     &           &    6 &  4.18e-09 &  1.20e-16 &      \\ 
     &           &    7 &  4.18e-09 &  1.19e-16 &      \\ 
     &           &    8 &  4.17e-09 &  1.19e-16 &      \\ 
     &           &    9 &  4.17e-09 &  1.19e-16 &      \\ 
     &           &   10 &  4.16e-09 &  1.19e-16 &      \\ 
2051 &  2.05e+04 &   10 &           &           & iters  \\ 
 \hdashline 
     &           &    1 &  3.34e-08 &  7.91e-10 &      \\ 
     &           &    2 &  4.43e-08 &  9.54e-16 &      \\ 
     &           &    3 &  3.34e-08 &  1.26e-15 &      \\ 
     &           &    4 &  4.43e-08 &  9.54e-16 &      \\ 
     &           &    5 &  3.35e-08 &  1.26e-15 &      \\ 
     &           &    6 &  4.43e-08 &  9.55e-16 &      \\ 
     &           &    7 &  3.35e-08 &  1.26e-15 &      \\ 
     &           &    8 &  3.34e-08 &  9.55e-16 &      \\ 
     &           &    9 &  4.43e-08 &  9.54e-16 &      \\ 
     &           &   10 &  3.34e-08 &  1.26e-15 &      \\ 
2052 &  2.05e+04 &   10 &           &           & iters  \\ 
 \hdashline 
     &           &    1 &  5.04e-09 &  3.42e-10 &      \\ 
     &           &    2 &  5.03e-09 &  1.44e-16 &      \\ 
     &           &    3 &  5.03e-09 &  1.44e-16 &      \\ 
     &           &    4 &  5.02e-09 &  1.43e-16 &      \\ 
     &           &    5 &  5.01e-09 &  1.43e-16 &      \\ 
     &           &    6 &  5.00e-09 &  1.43e-16 &      \\ 
     &           &    7 &  5.00e-09 &  1.43e-16 &      \\ 
     &           &    8 &  4.99e-09 &  1.43e-16 &      \\ 
     &           &    9 &  3.39e-08 &  1.42e-16 &      \\ 
     &           &   10 &  5.03e-09 &  9.66e-16 &      \\ 
2053 &  2.05e+04 &   10 &           &           & iters  \\ 
 \hdashline 
     &           &    1 &  4.70e-08 &  5.37e-10 &      \\ 
     &           &    2 &  3.07e-08 &  1.34e-15 &      \\ 
     &           &    3 &  3.07e-08 &  8.77e-16 &      \\ 
     &           &    4 &  4.70e-08 &  8.76e-16 &      \\ 
     &           &    5 &  3.07e-08 &  1.34e-15 &      \\ 
     &           &    6 &  4.70e-08 &  8.77e-16 &      \\ 
     &           &    7 &  3.07e-08 &  1.34e-15 &      \\ 
     &           &    8 &  3.07e-08 &  8.77e-16 &      \\ 
     &           &    9 &  4.70e-08 &  8.76e-16 &      \\ 
     &           &   10 &  3.07e-08 &  1.34e-15 &      \\ 
2054 &  2.05e+04 &   10 &           &           & iters  \\ 
 \hdashline 
     &           &    1 &  7.87e-09 &  1.63e-09 &      \\ 
     &           &    2 &  7.85e-09 &  2.24e-16 &      \\ 
     &           &    3 &  7.84e-09 &  2.24e-16 &      \\ 
     &           &    4 &  7.83e-09 &  2.24e-16 &      \\ 
     &           &    5 &  7.82e-09 &  2.24e-16 &      \\ 
     &           &    6 &  6.99e-08 &  2.23e-16 &      \\ 
     &           &    7 &  8.56e-08 &  1.99e-15 &      \\ 
     &           &    8 &  6.99e-08 &  2.44e-15 &      \\ 
     &           &    9 &  7.89e-09 &  2.00e-15 &      \\ 
     &           &   10 &  7.88e-09 &  2.25e-16 &      \\ 
2055 &  2.05e+04 &   10 &           &           & iters  \\ 
 \hdashline 
     &           &    1 &  3.07e-08 &  2.14e-09 &      \\ 
     &           &    2 &  8.21e-09 &  8.76e-16 &      \\ 
     &           &    3 &  8.20e-09 &  2.34e-16 &      \\ 
     &           &    4 &  8.18e-09 &  2.34e-16 &      \\ 
     &           &    5 &  8.17e-09 &  2.34e-16 &      \\ 
     &           &    6 &  3.07e-08 &  2.33e-16 &      \\ 
     &           &    7 &  8.20e-09 &  8.76e-16 &      \\ 
     &           &    8 &  8.19e-09 &  2.34e-16 &      \\ 
     &           &    9 &  8.18e-09 &  2.34e-16 &      \\ 
     &           &   10 &  8.17e-09 &  2.33e-16 &      \\ 
2056 &  2.06e+04 &   10 &           &           & iters  \\ 
 \hdashline 
     &           &    1 &  5.41e-08 &  1.02e-09 &      \\ 
     &           &    2 &  2.37e-08 &  1.54e-15 &      \\ 
     &           &    3 &  2.36e-08 &  6.75e-16 &      \\ 
     &           &    4 &  5.41e-08 &  6.74e-16 &      \\ 
     &           &    5 &  2.37e-08 &  1.54e-15 &      \\ 
     &           &    6 &  2.36e-08 &  6.76e-16 &      \\ 
     &           &    7 &  5.41e-08 &  6.75e-16 &      \\ 
     &           &    8 &  2.37e-08 &  1.54e-15 &      \\ 
     &           &    9 &  2.37e-08 &  6.76e-16 &      \\ 
     &           &   10 &  5.41e-08 &  6.75e-16 &      \\ 
2057 &  2.06e+04 &   10 &           &           & iters  \\ 
 \hdashline 
     &           &    1 &  2.60e-08 &  1.43e-09 &      \\ 
     &           &    2 &  2.60e-08 &  7.43e-16 &      \\ 
     &           &    3 &  5.17e-08 &  7.42e-16 &      \\ 
     &           &    4 &  2.60e-08 &  1.48e-15 &      \\ 
     &           &    5 &  2.60e-08 &  7.43e-16 &      \\ 
     &           &    6 &  5.17e-08 &  7.42e-16 &      \\ 
     &           &    7 &  2.60e-08 &  1.48e-15 &      \\ 
     &           &    8 &  2.60e-08 &  7.43e-16 &      \\ 
     &           &    9 &  5.17e-08 &  7.42e-16 &      \\ 
     &           &   10 &  2.60e-08 &  1.48e-15 &      \\ 
2058 &  2.06e+04 &   10 &           &           & iters  \\ 
 \hdashline 
     &           &    1 &  2.46e-08 &  1.74e-09 &      \\ 
     &           &    2 &  5.31e-08 &  7.01e-16 &      \\ 
     &           &    3 &  2.46e-08 &  1.52e-15 &      \\ 
     &           &    4 &  2.46e-08 &  7.03e-16 &      \\ 
     &           &    5 &  5.31e-08 &  7.02e-16 &      \\ 
     &           &    6 &  2.46e-08 &  1.52e-15 &      \\ 
     &           &    7 &  2.46e-08 &  7.03e-16 &      \\ 
     &           &    8 &  5.31e-08 &  7.02e-16 &      \\ 
     &           &    9 &  2.46e-08 &  1.52e-15 &      \\ 
     &           &   10 &  2.46e-08 &  7.03e-16 &      \\ 
2059 &  2.06e+04 &   10 &           &           & iters  \\ 
 \hdashline 
     &           &    1 &  6.19e-08 &  1.19e-09 &      \\ 
     &           &    2 &  1.59e-08 &  1.77e-15 &      \\ 
     &           &    3 &  1.58e-08 &  4.53e-16 &      \\ 
     &           &    4 &  1.58e-08 &  4.52e-16 &      \\ 
     &           &    5 &  1.58e-08 &  4.52e-16 &      \\ 
     &           &    6 &  6.19e-08 &  4.51e-16 &      \\ 
     &           &    7 &  1.59e-08 &  1.77e-15 &      \\ 
     &           &    8 &  1.58e-08 &  4.53e-16 &      \\ 
     &           &    9 &  1.58e-08 &  4.52e-16 &      \\ 
     &           &   10 &  6.19e-08 &  4.52e-16 &      \\ 
2060 &  2.06e+04 &   10 &           &           & iters  \\ 
 \hdashline 
     &           &    1 &  4.50e-08 &  1.16e-09 &      \\ 
     &           &    2 &  3.27e-08 &  1.29e-15 &      \\ 
     &           &    3 &  4.50e-08 &  9.34e-16 &      \\ 
     &           &    4 &  3.27e-08 &  1.28e-15 &      \\ 
     &           &    5 &  3.27e-08 &  9.34e-16 &      \\ 
     &           &    6 &  4.50e-08 &  9.33e-16 &      \\ 
     &           &    7 &  3.27e-08 &  1.29e-15 &      \\ 
     &           &    8 &  4.50e-08 &  9.33e-16 &      \\ 
     &           &    9 &  3.27e-08 &  1.29e-15 &      \\ 
     &           &   10 &  3.27e-08 &  9.34e-16 &      \\ 
2061 &  2.06e+04 &   10 &           &           & iters  \\ 
 \hdashline 
     &           &    1 &  3.63e-08 &  1.74e-09 &      \\ 
     &           &    2 &  3.63e-08 &  1.04e-15 &      \\ 
     &           &    3 &  4.15e-08 &  1.03e-15 &      \\ 
     &           &    4 &  3.63e-08 &  1.18e-15 &      \\ 
     &           &    5 &  4.15e-08 &  1.03e-15 &      \\ 
     &           &    6 &  3.63e-08 &  1.18e-15 &      \\ 
     &           &    7 &  4.15e-08 &  1.04e-15 &      \\ 
     &           &    8 &  3.63e-08 &  1.18e-15 &      \\ 
     &           &    9 &  4.15e-08 &  1.04e-15 &      \\ 
     &           &   10 &  3.63e-08 &  1.18e-15 &      \\ 
2062 &  2.06e+04 &   10 &           &           & iters  \\ 
 \hdashline 
     &           &    1 &  4.20e-08 &  2.26e-09 &      \\ 
     &           &    2 &  3.57e-08 &  1.20e-15 &      \\ 
     &           &    3 &  4.20e-08 &  1.02e-15 &      \\ 
     &           &    4 &  3.57e-08 &  1.20e-15 &      \\ 
     &           &    5 &  4.20e-08 &  1.02e-15 &      \\ 
     &           &    6 &  3.58e-08 &  1.20e-15 &      \\ 
     &           &    7 &  4.20e-08 &  1.02e-15 &      \\ 
     &           &    8 &  3.58e-08 &  1.20e-15 &      \\ 
     &           &    9 &  4.20e-08 &  1.02e-15 &      \\ 
     &           &   10 &  3.58e-08 &  1.20e-15 &      \\ 
2063 &  2.06e+04 &   10 &           &           & iters  \\ 
 \hdashline 
     &           &    1 &  2.08e-08 &  8.72e-10 &      \\ 
     &           &    2 &  2.08e-08 &  5.94e-16 &      \\ 
     &           &    3 &  2.08e-08 &  5.94e-16 &      \\ 
     &           &    4 &  2.07e-08 &  5.93e-16 &      \\ 
     &           &    5 &  5.70e-08 &  5.92e-16 &      \\ 
     &           &    6 &  2.08e-08 &  1.63e-15 &      \\ 
     &           &    7 &  2.08e-08 &  5.93e-16 &      \\ 
     &           &    8 &  5.70e-08 &  5.93e-16 &      \\ 
     &           &    9 &  2.08e-08 &  1.63e-15 &      \\ 
     &           &   10 &  2.08e-08 &  5.94e-16 &      \\ 
2064 &  2.06e+04 &   10 &           &           & iters  \\ 
 \hdashline 
     &           &    1 &  5.03e-08 &  1.92e-09 &      \\ 
     &           &    2 &  2.75e-08 &  1.43e-15 &      \\ 
     &           &    3 &  5.02e-08 &  7.85e-16 &      \\ 
     &           &    4 &  2.75e-08 &  1.43e-15 &      \\ 
     &           &    5 &  2.75e-08 &  7.86e-16 &      \\ 
     &           &    6 &  5.02e-08 &  7.85e-16 &      \\ 
     &           &    7 &  2.75e-08 &  1.43e-15 &      \\ 
     &           &    8 &  2.75e-08 &  7.86e-16 &      \\ 
     &           &    9 &  5.02e-08 &  7.85e-16 &      \\ 
     &           &   10 &  2.75e-08 &  1.43e-15 &      \\ 
2065 &  2.06e+04 &   10 &           &           & iters  \\ 
 \hdashline 
     &           &    1 &  2.30e-08 &  2.81e-09 &      \\ 
     &           &    2 &  2.30e-08 &  6.57e-16 &      \\ 
     &           &    3 &  2.30e-08 &  6.56e-16 &      \\ 
     &           &    4 &  5.48e-08 &  6.55e-16 &      \\ 
     &           &    5 &  2.30e-08 &  1.56e-15 &      \\ 
     &           &    6 &  2.30e-08 &  6.56e-16 &      \\ 
     &           &    7 &  5.48e-08 &  6.55e-16 &      \\ 
     &           &    8 &  2.30e-08 &  1.56e-15 &      \\ 
     &           &    9 &  2.30e-08 &  6.57e-16 &      \\ 
     &           &   10 &  5.47e-08 &  6.56e-16 &      \\ 
2066 &  2.06e+04 &   10 &           &           & iters  \\ 
 \hdashline 
     &           &    1 &  2.34e-08 &  3.34e-09 &      \\ 
     &           &    2 &  5.44e-08 &  6.67e-16 &      \\ 
     &           &    3 &  2.34e-08 &  1.55e-15 &      \\ 
     &           &    4 &  2.34e-08 &  6.68e-16 &      \\ 
     &           &    5 &  2.33e-08 &  6.67e-16 &      \\ 
     &           &    6 &  5.44e-08 &  6.66e-16 &      \\ 
     &           &    7 &  2.34e-08 &  1.55e-15 &      \\ 
     &           &    8 &  2.33e-08 &  6.67e-16 &      \\ 
     &           &    9 &  5.44e-08 &  6.66e-16 &      \\ 
     &           &   10 &  2.34e-08 &  1.55e-15 &      \\ 
2067 &  2.07e+04 &   10 &           &           & iters  \\ 
 \hdashline 
     &           &    1 &  1.56e-08 &  4.90e-09 &      \\ 
     &           &    2 &  6.21e-08 &  4.46e-16 &      \\ 
     &           &    3 &  1.57e-08 &  1.77e-15 &      \\ 
     &           &    4 &  1.57e-08 &  4.48e-16 &      \\ 
     &           &    5 &  1.57e-08 &  4.48e-16 &      \\ 
     &           &    6 &  1.56e-08 &  4.47e-16 &      \\ 
     &           &    7 &  6.21e-08 &  4.46e-16 &      \\ 
     &           &    8 &  1.57e-08 &  1.77e-15 &      \\ 
     &           &    9 &  1.57e-08 &  4.48e-16 &      \\ 
     &           &   10 &  1.57e-08 &  4.48e-16 &      \\ 
2068 &  2.07e+04 &   10 &           &           & iters  \\ 
 \hdashline 
     &           &    1 &  1.32e-09 &  4.52e-09 &      \\ 
     &           &    2 &  1.32e-09 &  3.77e-17 &      \\ 
     &           &    3 &  1.32e-09 &  3.77e-17 &      \\ 
     &           &    4 &  1.32e-09 &  3.76e-17 &      \\ 
     &           &    5 &  1.31e-09 &  3.76e-17 &      \\ 
     &           &    6 &  1.31e-09 &  3.75e-17 &      \\ 
     &           &    7 &  1.31e-09 &  3.75e-17 &      \\ 
     &           &    8 &  3.75e-08 &  3.74e-17 &      \\ 
     &           &    9 &  1.36e-09 &  1.07e-15 &      \\ 
     &           &   10 &  1.36e-09 &  3.89e-17 &      \\ 
2069 &  2.07e+04 &   10 &           &           & iters  \\ 
 \hdashline 
     &           &    1 &  2.16e-08 &  2.01e-09 &      \\ 
     &           &    2 &  1.72e-08 &  6.17e-16 &      \\ 
     &           &    3 &  2.16e-08 &  4.92e-16 &      \\ 
     &           &    4 &  1.72e-08 &  6.17e-16 &      \\ 
     &           &    5 &  2.16e-08 &  4.92e-16 &      \\ 
     &           &    6 &  1.73e-08 &  6.17e-16 &      \\ 
     &           &    7 &  2.16e-08 &  4.92e-16 &      \\ 
     &           &    8 &  1.73e-08 &  6.17e-16 &      \\ 
     &           &    9 &  1.72e-08 &  4.93e-16 &      \\ 
     &           &   10 &  2.16e-08 &  4.92e-16 &      \\ 
2070 &  2.07e+04 &   10 &           &           & iters  \\ 
 \hdashline 
     &           &    1 &  2.94e-08 &  3.99e-10 &      \\ 
     &           &    2 &  4.83e-08 &  8.40e-16 &      \\ 
     &           &    3 &  2.94e-08 &  1.38e-15 &      \\ 
     &           &    4 &  4.83e-08 &  8.40e-16 &      \\ 
     &           &    5 &  2.95e-08 &  1.38e-15 &      \\ 
     &           &    6 &  2.94e-08 &  8.41e-16 &      \\ 
     &           &    7 &  4.83e-08 &  8.40e-16 &      \\ 
     &           &    8 &  2.95e-08 &  1.38e-15 &      \\ 
     &           &    9 &  2.94e-08 &  8.41e-16 &      \\ 
     &           &   10 &  4.83e-08 &  8.40e-16 &      \\ 
2071 &  2.07e+04 &   10 &           &           & iters  \\ 
 \hdashline 
     &           &    1 &  1.72e-08 &  4.91e-10 &      \\ 
     &           &    2 &  1.72e-08 &  4.91e-16 &      \\ 
     &           &    3 &  6.05e-08 &  4.91e-16 &      \\ 
     &           &    4 &  1.73e-08 &  1.73e-15 &      \\ 
     &           &    5 &  1.72e-08 &  4.93e-16 &      \\ 
     &           &    6 &  1.72e-08 &  4.92e-16 &      \\ 
     &           &    7 &  6.05e-08 &  4.91e-16 &      \\ 
     &           &    8 &  1.73e-08 &  1.73e-15 &      \\ 
     &           &    9 &  1.72e-08 &  4.93e-16 &      \\ 
     &           &   10 &  1.72e-08 &  4.92e-16 &      \\ 
2072 &  2.07e+04 &   10 &           &           & iters  \\ 
 \hdashline 
     &           &    1 &  2.93e-08 &  7.80e-11 &      \\ 
     &           &    2 &  4.85e-08 &  8.35e-16 &      \\ 
     &           &    3 &  2.93e-08 &  1.38e-15 &      \\ 
     &           &    4 &  4.84e-08 &  8.36e-16 &      \\ 
     &           &    5 &  2.93e-08 &  1.38e-15 &      \\ 
     &           &    6 &  2.93e-08 &  8.37e-16 &      \\ 
     &           &    7 &  4.85e-08 &  8.35e-16 &      \\ 
     &           &    8 &  2.93e-08 &  1.38e-15 &      \\ 
     &           &    9 &  4.84e-08 &  8.36e-16 &      \\ 
     &           &   10 &  2.93e-08 &  1.38e-15 &      \\ 
2073 &  2.07e+04 &   10 &           &           & iters  \\ 
 \hdashline 
     &           &    1 &  2.77e-09 &  2.71e-09 &      \\ 
     &           &    2 &  2.76e-09 &  7.89e-17 &      \\ 
     &           &    3 &  2.76e-09 &  7.88e-17 &      \\ 
     &           &    4 &  2.75e-09 &  7.87e-17 &      \\ 
     &           &    5 &  2.75e-09 &  7.86e-17 &      \\ 
     &           &    6 &  3.61e-08 &  7.85e-17 &      \\ 
     &           &    7 &  2.80e-09 &  1.03e-15 &      \\ 
     &           &    8 &  2.79e-09 &  7.99e-17 &      \\ 
     &           &    9 &  2.79e-09 &  7.97e-17 &      \\ 
     &           &   10 &  2.79e-09 &  7.96e-17 &      \\ 
2074 &  2.07e+04 &   10 &           &           & iters  \\ 
 \hdashline 
     &           &    1 &  2.68e-09 &  3.71e-09 &      \\ 
     &           &    2 &  2.68e-09 &  7.65e-17 &      \\ 
     &           &    3 &  2.67e-09 &  7.64e-17 &      \\ 
     &           &    4 &  2.67e-09 &  7.63e-17 &      \\ 
     &           &    5 &  2.66e-09 &  7.61e-17 &      \\ 
     &           &    6 &  2.66e-09 &  7.60e-17 &      \\ 
     &           &    7 &  2.66e-09 &  7.59e-17 &      \\ 
     &           &    8 &  2.65e-09 &  7.58e-17 &      \\ 
     &           &    9 &  2.65e-09 &  7.57e-17 &      \\ 
     &           &   10 &  3.62e-08 &  7.56e-17 &      \\ 
2075 &  2.07e+04 &   10 &           &           & iters  \\ 
 \hdashline 
     &           &    1 &  2.09e-08 &  7.20e-10 &      \\ 
     &           &    2 &  5.68e-08 &  5.96e-16 &      \\ 
     &           &    3 &  2.09e-08 &  1.62e-15 &      \\ 
     &           &    4 &  2.09e-08 &  5.97e-16 &      \\ 
     &           &    5 &  2.09e-08 &  5.96e-16 &      \\ 
     &           &    6 &  5.69e-08 &  5.95e-16 &      \\ 
     &           &    7 &  2.09e-08 &  1.62e-15 &      \\ 
     &           &    8 &  2.09e-08 &  5.97e-16 &      \\ 
     &           &    9 &  5.68e-08 &  5.96e-16 &      \\ 
     &           &   10 &  2.09e-08 &  1.62e-15 &      \\ 
2076 &  2.08e+04 &   10 &           &           & iters  \\ 
 \hdashline 
     &           &    1 &  1.62e-08 &  1.43e-10 &      \\ 
     &           &    2 &  2.27e-08 &  4.63e-16 &      \\ 
     &           &    3 &  1.62e-08 &  6.47e-16 &      \\ 
     &           &    4 &  2.27e-08 &  4.63e-16 &      \\ 
     &           &    5 &  1.62e-08 &  6.47e-16 &      \\ 
     &           &    6 &  1.62e-08 &  4.63e-16 &      \\ 
     &           &    7 &  2.27e-08 &  4.62e-16 &      \\ 
     &           &    8 &  1.62e-08 &  6.47e-16 &      \\ 
     &           &    9 &  2.27e-08 &  4.63e-16 &      \\ 
     &           &   10 &  1.62e-08 &  6.47e-16 &      \\ 
2077 &  2.08e+04 &   10 &           &           & iters  \\ 
 \hdashline 
     &           &    1 &  3.04e-08 &  1.36e-09 &      \\ 
     &           &    2 &  4.74e-08 &  8.66e-16 &      \\ 
     &           &    3 &  3.04e-08 &  1.35e-15 &      \\ 
     &           &    4 &  4.74e-08 &  8.67e-16 &      \\ 
     &           &    5 &  3.04e-08 &  1.35e-15 &      \\ 
     &           &    6 &  3.04e-08 &  8.68e-16 &      \\ 
     &           &    7 &  4.74e-08 &  8.66e-16 &      \\ 
     &           &    8 &  3.04e-08 &  1.35e-15 &      \\ 
     &           &    9 &  3.03e-08 &  8.67e-16 &      \\ 
     &           &   10 &  4.74e-08 &  8.66e-16 &      \\ 
2078 &  2.08e+04 &   10 &           &           & iters  \\ 
 \hdashline 
     &           &    1 &  3.82e-08 &  1.32e-09 &      \\ 
     &           &    2 &  3.95e-08 &  1.09e-15 &      \\ 
     &           &    3 &  3.82e-08 &  1.13e-15 &      \\ 
     &           &    4 &  3.95e-08 &  1.09e-15 &      \\ 
     &           &    5 &  3.82e-08 &  1.13e-15 &      \\ 
     &           &    6 &  3.95e-08 &  1.09e-15 &      \\ 
     &           &    7 &  3.82e-08 &  1.13e-15 &      \\ 
     &           &    8 &  3.95e-08 &  1.09e-15 &      \\ 
     &           &    9 &  3.82e-08 &  1.13e-15 &      \\ 
     &           &   10 &  3.95e-08 &  1.09e-15 &      \\ 
2079 &  2.08e+04 &   10 &           &           & iters  \\ 
 \hdashline 
     &           &    1 &  2.53e-08 &  2.32e-09 &      \\ 
     &           &    2 &  5.25e-08 &  7.21e-16 &      \\ 
     &           &    3 &  2.53e-08 &  1.50e-15 &      \\ 
     &           &    4 &  2.53e-08 &  7.22e-16 &      \\ 
     &           &    5 &  5.25e-08 &  7.21e-16 &      \\ 
     &           &    6 &  2.53e-08 &  1.50e-15 &      \\ 
     &           &    7 &  2.53e-08 &  7.22e-16 &      \\ 
     &           &    8 &  5.24e-08 &  7.21e-16 &      \\ 
     &           &    9 &  2.53e-08 &  1.50e-15 &      \\ 
     &           &   10 &  2.53e-08 &  7.22e-16 &      \\ 
2080 &  2.08e+04 &   10 &           &           & iters  \\ 
 \hdashline 
     &           &    1 &  2.92e-08 &  1.62e-09 &      \\ 
     &           &    2 &  2.91e-08 &  8.32e-16 &      \\ 
     &           &    3 &  4.86e-08 &  8.31e-16 &      \\ 
     &           &    4 &  2.91e-08 &  1.39e-15 &      \\ 
     &           &    5 &  4.86e-08 &  8.32e-16 &      \\ 
     &           &    6 &  2.92e-08 &  1.39e-15 &      \\ 
     &           &    7 &  2.91e-08 &  8.33e-16 &      \\ 
     &           &    8 &  4.86e-08 &  8.31e-16 &      \\ 
     &           &    9 &  2.92e-08 &  1.39e-15 &      \\ 
     &           &   10 &  2.91e-08 &  8.32e-16 &      \\ 
2081 &  2.08e+04 &   10 &           &           & iters  \\ 
 \hdashline 
     &           &    1 &  5.95e-08 &  1.02e-09 &      \\ 
     &           &    2 &  1.82e-08 &  1.70e-15 &      \\ 
     &           &    3 &  5.95e-08 &  5.20e-16 &      \\ 
     &           &    4 &  1.83e-08 &  1.70e-15 &      \\ 
     &           &    5 &  1.83e-08 &  5.22e-16 &      \\ 
     &           &    6 &  1.82e-08 &  5.21e-16 &      \\ 
     &           &    7 &  5.95e-08 &  5.21e-16 &      \\ 
     &           &    8 &  1.83e-08 &  1.70e-15 &      \\ 
     &           &    9 &  1.83e-08 &  5.22e-16 &      \\ 
     &           &   10 &  1.82e-08 &  5.22e-16 &      \\ 
2082 &  2.08e+04 &   10 &           &           & iters  \\ 
 \hdashline 
     &           &    1 &  7.08e-08 &  9.28e-10 &      \\ 
     &           &    2 &  8.47e-08 &  2.02e-15 &      \\ 
     &           &    3 &  7.08e-08 &  2.42e-15 &      \\ 
     &           &    4 &  7.01e-09 &  2.02e-15 &      \\ 
     &           &    5 &  7.00e-09 &  2.00e-16 &      \\ 
     &           &    6 &  6.99e-09 &  2.00e-16 &      \\ 
     &           &    7 &  6.98e-09 &  1.99e-16 &      \\ 
     &           &    8 &  6.97e-09 &  1.99e-16 &      \\ 
     &           &    9 &  6.96e-09 &  1.99e-16 &      \\ 
     &           &   10 &  6.95e-09 &  1.99e-16 &      \\ 
2083 &  2.08e+04 &   10 &           &           & iters  \\ 
 \hdashline 
     &           &    1 &  2.17e-09 &  5.38e-10 &      \\ 
     &           &    2 &  2.17e-09 &  6.19e-17 &      \\ 
     &           &    3 &  2.16e-09 &  6.18e-17 &      \\ 
     &           &    4 &  2.16e-09 &  6.17e-17 &      \\ 
     &           &    5 &  2.16e-09 &  6.16e-17 &      \\ 
     &           &    6 &  7.55e-08 &  6.16e-17 &      \\ 
     &           &    7 &  8.00e-08 &  2.16e-15 &      \\ 
     &           &    8 &  7.55e-08 &  2.28e-15 &      \\ 
     &           &    9 &  7.99e-08 &  2.16e-15 &      \\ 
     &           &   10 &  7.55e-08 &  2.28e-15 &      \\ 
2084 &  2.08e+04 &   10 &           &           & iters  \\ 
 \hdashline 
     &           &    1 &  1.53e-08 &  3.50e-10 &      \\ 
     &           &    2 &  1.52e-08 &  4.36e-16 &      \\ 
     &           &    3 &  1.52e-08 &  4.35e-16 &      \\ 
     &           &    4 &  1.52e-08 &  4.34e-16 &      \\ 
     &           &    5 &  6.25e-08 &  4.34e-16 &      \\ 
     &           &    6 &  1.53e-08 &  1.78e-15 &      \\ 
     &           &    7 &  1.52e-08 &  4.36e-16 &      \\ 
     &           &    8 &  1.52e-08 &  4.35e-16 &      \\ 
     &           &    9 &  1.52e-08 &  4.34e-16 &      \\ 
     &           &   10 &  6.25e-08 &  4.34e-16 &      \\ 
2085 &  2.08e+04 &   10 &           &           & iters  \\ 
 \hdashline 
     &           &    1 &  4.45e-08 &  1.45e-09 &      \\ 
     &           &    2 &  4.44e-08 &  1.27e-15 &      \\ 
     &           &    3 &  3.33e-08 &  1.27e-15 &      \\ 
     &           &    4 &  4.44e-08 &  9.51e-16 &      \\ 
     &           &    5 &  3.33e-08 &  1.27e-15 &      \\ 
     &           &    6 &  3.33e-08 &  9.51e-16 &      \\ 
     &           &    7 &  4.45e-08 &  9.50e-16 &      \\ 
     &           &    8 &  3.33e-08 &  1.27e-15 &      \\ 
     &           &    9 &  4.44e-08 &  9.50e-16 &      \\ 
     &           &   10 &  3.33e-08 &  1.27e-15 &      \\ 
2086 &  2.08e+04 &   10 &           &           & iters  \\ 
 \hdashline 
     &           &    1 &  1.00e-08 &  2.87e-09 &      \\ 
     &           &    2 &  1.00e-08 &  2.86e-16 &      \\ 
     &           &    3 &  9.99e-09 &  2.86e-16 &      \\ 
     &           &    4 &  6.77e-08 &  2.85e-16 &      \\ 
     &           &    5 &  4.89e-08 &  1.93e-15 &      \\ 
     &           &    6 &  1.00e-08 &  1.40e-15 &      \\ 
     &           &    7 &  9.99e-09 &  2.86e-16 &      \\ 
     &           &    8 &  6.77e-08 &  2.85e-16 &      \\ 
     &           &    9 &  4.89e-08 &  1.93e-15 &      \\ 
     &           &   10 &  1.00e-08 &  1.40e-15 &      \\ 
2087 &  2.09e+04 &   10 &           &           & iters  \\ 
 \hdashline 
     &           &    1 &  3.88e-09 &  2.61e-09 &      \\ 
     &           &    2 &  3.88e-09 &  1.11e-16 &      \\ 
     &           &    3 &  3.87e-09 &  1.11e-16 &      \\ 
     &           &    4 &  3.87e-09 &  1.10e-16 &      \\ 
     &           &    5 &  3.86e-09 &  1.10e-16 &      \\ 
     &           &    6 &  3.85e-09 &  1.10e-16 &      \\ 
     &           &    7 &  3.85e-09 &  1.10e-16 &      \\ 
     &           &    8 &  3.84e-09 &  1.10e-16 &      \\ 
     &           &    9 &  3.84e-09 &  1.10e-16 &      \\ 
     &           &   10 &  3.50e-08 &  1.10e-16 &      \\ 
2088 &  2.09e+04 &   10 &           &           & iters  \\ 
 \hdashline 
     &           &    1 &  5.35e-08 &  9.21e-10 &      \\ 
     &           &    2 &  2.42e-08 &  1.53e-15 &      \\ 
     &           &    3 &  2.42e-08 &  6.92e-16 &      \\ 
     &           &    4 &  5.35e-08 &  6.91e-16 &      \\ 
     &           &    5 &  2.43e-08 &  1.53e-15 &      \\ 
     &           &    6 &  2.42e-08 &  6.92e-16 &      \\ 
     &           &    7 &  2.42e-08 &  6.91e-16 &      \\ 
     &           &    8 &  5.35e-08 &  6.90e-16 &      \\ 
     &           &    9 &  2.42e-08 &  1.53e-15 &      \\ 
     &           &   10 &  2.42e-08 &  6.92e-16 &      \\ 
2089 &  2.09e+04 &   10 &           &           & iters  \\ 
 \hdashline 
     &           &    1 &  5.56e-09 &  1.57e-09 &      \\ 
     &           &    2 &  5.55e-09 &  1.59e-16 &      \\ 
     &           &    3 &  7.21e-08 &  1.58e-16 &      \\ 
     &           &    4 &  4.45e-08 &  2.06e-15 &      \\ 
     &           &    5 &  5.58e-09 &  1.27e-15 &      \\ 
     &           &    6 &  5.58e-09 &  1.59e-16 &      \\ 
     &           &    7 &  5.57e-09 &  1.59e-16 &      \\ 
     &           &    8 &  5.56e-09 &  1.59e-16 &      \\ 
     &           &    9 &  5.55e-09 &  1.59e-16 &      \\ 
     &           &   10 &  7.21e-08 &  1.58e-16 &      \\ 
2090 &  2.09e+04 &   10 &           &           & iters  \\ 
 \hdashline 
     &           &    1 &  2.88e-08 &  2.48e-09 &      \\ 
     &           &    2 &  2.88e-08 &  8.23e-16 &      \\ 
     &           &    3 &  4.89e-08 &  8.22e-16 &      \\ 
     &           &    4 &  2.88e-08 &  1.40e-15 &      \\ 
     &           &    5 &  2.88e-08 &  8.23e-16 &      \\ 
     &           &    6 &  4.89e-08 &  8.22e-16 &      \\ 
     &           &    7 &  2.88e-08 &  1.40e-15 &      \\ 
     &           &    8 &  4.89e-08 &  8.22e-16 &      \\ 
     &           &    9 &  2.88e-08 &  1.40e-15 &      \\ 
     &           &   10 &  2.88e-08 &  8.23e-16 &      \\ 
2091 &  2.09e+04 &   10 &           &           & iters  \\ 
 \hdashline 
     &           &    1 &  6.01e-08 &  1.93e-09 &      \\ 
     &           &    2 &  1.77e-08 &  1.72e-15 &      \\ 
     &           &    3 &  1.76e-08 &  5.04e-16 &      \\ 
     &           &    4 &  1.76e-08 &  5.03e-16 &      \\ 
     &           &    5 &  6.01e-08 &  5.03e-16 &      \\ 
     &           &    6 &  1.77e-08 &  1.72e-15 &      \\ 
     &           &    7 &  1.77e-08 &  5.04e-16 &      \\ 
     &           &    8 &  1.76e-08 &  5.04e-16 &      \\ 
     &           &    9 &  1.76e-08 &  5.03e-16 &      \\ 
     &           &   10 &  6.01e-08 &  5.02e-16 &      \\ 
2092 &  2.09e+04 &   10 &           &           & iters  \\ 
 \hdashline 
     &           &    1 &  4.73e-08 &  8.00e-10 &      \\ 
     &           &    2 &  3.05e-08 &  1.35e-15 &      \\ 
     &           &    3 &  3.04e-08 &  8.70e-16 &      \\ 
     &           &    4 &  4.73e-08 &  8.68e-16 &      \\ 
     &           &    5 &  3.04e-08 &  1.35e-15 &      \\ 
     &           &    6 &  4.73e-08 &  8.69e-16 &      \\ 
     &           &    7 &  3.05e-08 &  1.35e-15 &      \\ 
     &           &    8 &  3.04e-08 &  8.70e-16 &      \\ 
     &           &    9 &  4.73e-08 &  8.68e-16 &      \\ 
     &           &   10 &  3.05e-08 &  1.35e-15 &      \\ 
2093 &  2.09e+04 &   10 &           &           & iters  \\ 
 \hdashline 
     &           &    1 &  3.05e-08 &  2.18e-10 &      \\ 
     &           &    2 &  3.04e-08 &  8.70e-16 &      \\ 
     &           &    3 &  4.73e-08 &  8.69e-16 &      \\ 
     &           &    4 &  3.05e-08 &  1.35e-15 &      \\ 
     &           &    5 &  4.73e-08 &  8.70e-16 &      \\ 
     &           &    6 &  3.05e-08 &  1.35e-15 &      \\ 
     &           &    7 &  3.04e-08 &  8.70e-16 &      \\ 
     &           &    8 &  4.73e-08 &  8.69e-16 &      \\ 
     &           &    9 &  3.05e-08 &  1.35e-15 &      \\ 
     &           &   10 &  3.04e-08 &  8.70e-16 &      \\ 
2094 &  2.09e+04 &   10 &           &           & iters  \\ 
 \hdashline 
     &           &    1 &  9.67e-11 &  8.21e-10 &      \\ 
     &           &    2 &  9.65e-11 &  2.76e-18 &      \\ 
     &           &    3 &  9.64e-11 &  2.75e-18 &      \\ 
     &           &    4 &  9.63e-11 &  2.75e-18 &      \\ 
     &           &    5 &  9.61e-11 &  2.75e-18 &      \\ 
     &           &    6 &  9.60e-11 &  2.74e-18 &      \\ 
     &           &    7 &  9.58e-11 &  2.74e-18 &      \\ 
     &           &    8 &  9.57e-11 &  2.74e-18 &      \\ 
     &           &    9 &  9.56e-11 &  2.73e-18 &      \\ 
     &           &   10 &  9.54e-11 &  2.73e-18 &      \\ 
2095 &  2.09e+04 &   10 &           &           & iters  \\ 
 \hdashline 
     &           &    1 &  6.63e-08 &  1.93e-09 &      \\ 
     &           &    2 &  1.15e-08 &  1.89e-15 &      \\ 
     &           &    3 &  1.15e-08 &  3.28e-16 &      \\ 
     &           &    4 &  1.14e-08 &  3.27e-16 &      \\ 
     &           &    5 &  1.14e-08 &  3.27e-16 &      \\ 
     &           &    6 &  1.14e-08 &  3.26e-16 &      \\ 
     &           &    7 &  6.63e-08 &  3.26e-16 &      \\ 
     &           &    8 &  8.92e-08 &  1.89e-15 &      \\ 
     &           &    9 &  6.63e-08 &  2.55e-15 &      \\ 
     &           &   10 &  1.15e-08 &  1.89e-15 &      \\ 
2096 &  2.10e+04 &   10 &           &           & iters  \\ 
 \hdashline 
     &           &    1 &  3.87e-08 &  9.05e-10 &      \\ 
     &           &    2 &  3.90e-08 &  1.10e-15 &      \\ 
     &           &    3 &  3.87e-08 &  1.11e-15 &      \\ 
     &           &    4 &  3.90e-08 &  1.10e-15 &      \\ 
     &           &    5 &  3.87e-08 &  1.11e-15 &      \\ 
     &           &    6 &  3.90e-08 &  1.10e-15 &      \\ 
     &           &    7 &  3.87e-08 &  1.11e-15 &      \\ 
     &           &    8 &  3.90e-08 &  1.10e-15 &      \\ 
     &           &    9 &  3.87e-08 &  1.11e-15 &      \\ 
     &           &   10 &  3.90e-08 &  1.10e-15 &      \\ 
2097 &  2.10e+04 &   10 &           &           & iters  \\ 
 \hdashline 
     &           &    1 &  6.24e-09 &  9.57e-10 &      \\ 
     &           &    2 &  6.23e-09 &  1.78e-16 &      \\ 
     &           &    3 &  6.22e-09 &  1.78e-16 &      \\ 
     &           &    4 &  6.22e-09 &  1.78e-16 &      \\ 
     &           &    5 &  6.21e-09 &  1.77e-16 &      \\ 
     &           &    6 &  6.20e-09 &  1.77e-16 &      \\ 
     &           &    7 &  6.19e-09 &  1.77e-16 &      \\ 
     &           &    8 &  6.18e-09 &  1.77e-16 &      \\ 
     &           &    9 &  6.17e-09 &  1.76e-16 &      \\ 
     &           &   10 &  7.15e-08 &  1.76e-16 &      \\ 
2098 &  2.10e+04 &   10 &           &           & iters  \\ 
 \hdashline 
     &           &    1 &  4.64e-09 &  1.53e-09 &      \\ 
     &           &    2 &  4.64e-09 &  1.33e-16 &      \\ 
     &           &    3 &  4.63e-09 &  1.32e-16 &      \\ 
     &           &    4 &  7.31e-08 &  1.32e-16 &      \\ 
     &           &    5 &  8.24e-08 &  2.09e-15 &      \\ 
     &           &    6 &  7.31e-08 &  2.35e-15 &      \\ 
     &           &    7 &  8.24e-08 &  2.09e-15 &      \\ 
     &           &    8 &  7.31e-08 &  2.35e-15 &      \\ 
     &           &    9 &  4.70e-09 &  2.09e-15 &      \\ 
     &           &   10 &  4.70e-09 &  1.34e-16 &      \\ 
2099 &  2.10e+04 &   10 &           &           & iters  \\ 
 \hdashline 
     &           &    1 &  1.60e-08 &  8.48e-10 &      \\ 
     &           &    2 &  6.18e-08 &  4.55e-16 &      \\ 
     &           &    3 &  1.60e-08 &  1.76e-15 &      \\ 
     &           &    4 &  1.60e-08 &  4.57e-16 &      \\ 
     &           &    5 &  1.60e-08 &  4.56e-16 &      \\ 
     &           &    6 &  1.59e-08 &  4.56e-16 &      \\ 
     &           &    7 &  6.18e-08 &  4.55e-16 &      \\ 
     &           &    8 &  1.60e-08 &  1.76e-15 &      \\ 
     &           &    9 &  1.60e-08 &  4.57e-16 &      \\ 
     &           &   10 &  1.60e-08 &  4.56e-16 &      \\ 
2100 &  2.10e+04 &   10 &           &           & iters  \\ 
 \hdashline 
     &           &    1 &  6.82e-08 &  1.20e-10 &      \\ 
     &           &    2 &  8.73e-08 &  1.95e-15 &      \\ 
     &           &    3 &  6.82e-08 &  2.49e-15 &      \\ 
     &           &    4 &  9.56e-09 &  1.95e-15 &      \\ 
     &           &    5 &  9.55e-09 &  2.73e-16 &      \\ 
     &           &    6 &  9.53e-09 &  2.72e-16 &      \\ 
     &           &    7 &  9.52e-09 &  2.72e-16 &      \\ 
     &           &    8 &  9.51e-09 &  2.72e-16 &      \\ 
     &           &    9 &  9.49e-09 &  2.71e-16 &      \\ 
     &           &   10 &  6.82e-08 &  2.71e-16 &      \\ 
2101 &  2.10e+04 &   10 &           &           & iters  \\ 
 \hdashline 
     &           &    1 &  3.97e-08 &  4.73e-10 &      \\ 
     &           &    2 &  3.97e-08 &  1.13e-15 &      \\ 
     &           &    3 &  3.81e-08 &  1.13e-15 &      \\ 
     &           &    4 &  3.97e-08 &  1.09e-15 &      \\ 
     &           &    5 &  3.81e-08 &  1.13e-15 &      \\ 
     &           &    6 &  3.97e-08 &  1.09e-15 &      \\ 
     &           &    7 &  3.81e-08 &  1.13e-15 &      \\ 
     &           &    8 &  3.97e-08 &  1.09e-15 &      \\ 
     &           &    9 &  3.81e-08 &  1.13e-15 &      \\ 
     &           &   10 &  3.80e-08 &  1.09e-15 &      \\ 
2102 &  2.10e+04 &   10 &           &           & iters  \\ 
 \hdashline 
     &           &    1 &  4.33e-08 &  1.29e-09 &      \\ 
     &           &    2 &  3.45e-08 &  1.24e-15 &      \\ 
     &           &    3 &  4.33e-08 &  9.83e-16 &      \\ 
     &           &    4 &  3.45e-08 &  1.24e-15 &      \\ 
     &           &    5 &  4.33e-08 &  9.84e-16 &      \\ 
     &           &    6 &  3.45e-08 &  1.24e-15 &      \\ 
     &           &    7 &  4.33e-08 &  9.84e-16 &      \\ 
     &           &    8 &  3.45e-08 &  1.23e-15 &      \\ 
     &           &    9 &  3.44e-08 &  9.84e-16 &      \\ 
     &           &   10 &  4.33e-08 &  9.83e-16 &      \\ 
2103 &  2.10e+04 &   10 &           &           & iters  \\ 
 \hdashline 
     &           &    1 &  2.12e-08 &  2.40e-09 &      \\ 
     &           &    2 &  1.77e-08 &  6.05e-16 &      \\ 
     &           &    3 &  2.12e-08 &  5.04e-16 &      \\ 
     &           &    4 &  1.77e-08 &  6.05e-16 &      \\ 
     &           &    5 &  2.12e-08 &  5.04e-16 &      \\ 
     &           &    6 &  1.77e-08 &  6.05e-16 &      \\ 
     &           &    7 &  1.77e-08 &  5.05e-16 &      \\ 
     &           &    8 &  2.12e-08 &  5.04e-16 &      \\ 
     &           &    9 &  1.77e-08 &  6.05e-16 &      \\ 
     &           &   10 &  2.12e-08 &  5.04e-16 &      \\ 
2104 &  2.10e+04 &   10 &           &           & iters  \\ 
 \hdashline 
     &           &    1 &  3.43e-08 &  1.49e-09 &      \\ 
     &           &    2 &  4.35e-08 &  9.79e-16 &      \\ 
     &           &    3 &  3.43e-08 &  1.24e-15 &      \\ 
     &           &    4 &  4.34e-08 &  9.79e-16 &      \\ 
     &           &    5 &  3.43e-08 &  1.24e-15 &      \\ 
     &           &    6 &  4.34e-08 &  9.79e-16 &      \\ 
     &           &    7 &  3.43e-08 &  1.24e-15 &      \\ 
     &           &    8 &  4.34e-08 &  9.80e-16 &      \\ 
     &           &    9 &  3.43e-08 &  1.24e-15 &      \\ 
     &           &   10 &  3.43e-08 &  9.80e-16 &      \\ 
2105 &  2.10e+04 &   10 &           &           & iters  \\ 
 \hdashline 
     &           &    1 &  3.51e-08 &  8.47e-10 &      \\ 
     &           &    2 &  3.77e-09 &  1.00e-15 &      \\ 
     &           &    3 &  3.76e-09 &  1.08e-16 &      \\ 
     &           &    4 &  3.76e-09 &  1.07e-16 &      \\ 
     &           &    5 &  3.75e-09 &  1.07e-16 &      \\ 
     &           &    6 &  3.75e-09 &  1.07e-16 &      \\ 
     &           &    7 &  3.74e-09 &  1.07e-16 &      \\ 
     &           &    8 &  3.74e-09 &  1.07e-16 &      \\ 
     &           &    9 &  3.51e-08 &  1.07e-16 &      \\ 
     &           &   10 &  3.78e-09 &  1.00e-15 &      \\ 
2106 &  2.10e+04 &   10 &           &           & iters  \\ 
 \hdashline 
     &           &    1 &  1.38e-09 &  1.65e-10 &      \\ 
     &           &    2 &  1.38e-09 &  3.93e-17 &      \\ 
     &           &    3 &  1.37e-09 &  3.93e-17 &      \\ 
     &           &    4 &  1.37e-09 &  3.92e-17 &      \\ 
     &           &    5 &  1.37e-09 &  3.92e-17 &      \\ 
     &           &    6 &  1.37e-09 &  3.91e-17 &      \\ 
     &           &    7 &  1.37e-09 &  3.91e-17 &      \\ 
     &           &    8 &  1.36e-09 &  3.90e-17 &      \\ 
     &           &    9 &  1.36e-09 &  3.89e-17 &      \\ 
     &           &   10 &  1.36e-09 &  3.89e-17 &      \\ 
2107 &  2.11e+04 &   10 &           &           & iters  \\ 
 \hdashline 
     &           &    1 &  1.34e-08 &  5.26e-11 &      \\ 
     &           &    2 &  1.34e-08 &  3.83e-16 &      \\ 
     &           &    3 &  1.34e-08 &  3.83e-16 &      \\ 
     &           &    4 &  1.34e-08 &  3.82e-16 &      \\ 
     &           &    5 &  1.34e-08 &  3.82e-16 &      \\ 
     &           &    6 &  6.43e-08 &  3.81e-16 &      \\ 
     &           &    7 &  1.34e-08 &  1.84e-15 &      \\ 
     &           &    8 &  1.34e-08 &  3.83e-16 &      \\ 
     &           &    9 &  1.34e-08 &  3.83e-16 &      \\ 
     &           &   10 &  1.34e-08 &  3.82e-16 &      \\ 
2108 &  2.11e+04 &   10 &           &           & iters  \\ 
 \hdashline 
     &           &    1 &  6.46e-09 &  7.11e-10 &      \\ 
     &           &    2 &  6.46e-09 &  1.85e-16 &      \\ 
     &           &    3 &  6.45e-09 &  1.84e-16 &      \\ 
     &           &    4 &  6.44e-09 &  1.84e-16 &      \\ 
     &           &    5 &  6.43e-09 &  1.84e-16 &      \\ 
     &           &    6 &  6.42e-09 &  1.83e-16 &      \\ 
     &           &    7 &  6.41e-09 &  1.83e-16 &      \\ 
     &           &    8 &  6.40e-09 &  1.83e-16 &      \\ 
     &           &    9 &  6.39e-09 &  1.83e-16 &      \\ 
     &           &   10 &  7.13e-08 &  1.82e-16 &      \\ 
2109 &  2.11e+04 &   10 &           &           & iters  \\ 
 \hdashline 
     &           &    1 &  2.25e-08 &  1.09e-10 &      \\ 
     &           &    2 &  2.24e-08 &  6.41e-16 &      \\ 
     &           &    3 &  2.24e-08 &  6.40e-16 &      \\ 
     &           &    4 &  5.53e-08 &  6.39e-16 &      \\ 
     &           &    5 &  2.24e-08 &  1.58e-15 &      \\ 
     &           &    6 &  2.24e-08 &  6.40e-16 &      \\ 
     &           &    7 &  2.24e-08 &  6.39e-16 &      \\ 
     &           &    8 &  5.54e-08 &  6.38e-16 &      \\ 
     &           &    9 &  2.24e-08 &  1.58e-15 &      \\ 
     &           &   10 &  2.24e-08 &  6.40e-16 &      \\ 
2110 &  2.11e+04 &   10 &           &           & iters  \\ 
 \hdashline 
     &           &    1 &  3.98e-08 &  1.59e-09 &      \\ 
     &           &    2 &  3.79e-08 &  1.14e-15 &      \\ 
     &           &    3 &  3.98e-08 &  1.08e-15 &      \\ 
     &           &    4 &  3.79e-08 &  1.14e-15 &      \\ 
     &           &    5 &  3.98e-08 &  1.08e-15 &      \\ 
     &           &    6 &  3.79e-08 &  1.14e-15 &      \\ 
     &           &    7 &  3.98e-08 &  1.08e-15 &      \\ 
     &           &    8 &  3.79e-08 &  1.14e-15 &      \\ 
     &           &    9 &  3.98e-08 &  1.08e-15 &      \\ 
     &           &   10 &  3.79e-08 &  1.14e-15 &      \\ 
2111 &  2.11e+04 &   10 &           &           & iters  \\ 
 \hdashline 
     &           &    1 &  5.12e-08 &  2.17e-09 &      \\ 
     &           &    2 &  2.66e-08 &  1.46e-15 &      \\ 
     &           &    3 &  2.65e-08 &  7.58e-16 &      \\ 
     &           &    4 &  5.12e-08 &  7.57e-16 &      \\ 
     &           &    5 &  2.66e-08 &  1.46e-15 &      \\ 
     &           &    6 &  2.65e-08 &  7.58e-16 &      \\ 
     &           &    7 &  5.12e-08 &  7.57e-16 &      \\ 
     &           &    8 &  2.66e-08 &  1.46e-15 &      \\ 
     &           &    9 &  2.65e-08 &  7.58e-16 &      \\ 
     &           &   10 &  5.12e-08 &  7.57e-16 &      \\ 
2112 &  2.11e+04 &   10 &           &           & iters  \\ 
 \hdashline 
     &           &    1 &  5.86e-08 &  9.16e-10 &      \\ 
     &           &    2 &  1.92e-08 &  1.67e-15 &      \\ 
     &           &    3 &  1.92e-08 &  5.48e-16 &      \\ 
     &           &    4 &  1.92e-08 &  5.48e-16 &      \\ 
     &           &    5 &  5.86e-08 &  5.47e-16 &      \\ 
     &           &    6 &  1.92e-08 &  1.67e-15 &      \\ 
     &           &    7 &  1.92e-08 &  5.49e-16 &      \\ 
     &           &    8 &  1.92e-08 &  5.48e-16 &      \\ 
     &           &    9 &  5.86e-08 &  5.47e-16 &      \\ 
     &           &   10 &  1.92e-08 &  1.67e-15 &      \\ 
2113 &  2.11e+04 &   10 &           &           & iters  \\ 
 \hdashline 
     &           &    1 &  3.14e-08 &  1.36e-09 &      \\ 
     &           &    2 &  4.63e-08 &  8.97e-16 &      \\ 
     &           &    3 &  3.15e-08 &  1.32e-15 &      \\ 
     &           &    4 &  3.14e-08 &  8.98e-16 &      \\ 
     &           &    5 &  4.63e-08 &  8.97e-16 &      \\ 
     &           &    6 &  3.14e-08 &  1.32e-15 &      \\ 
     &           &    7 &  4.63e-08 &  8.97e-16 &      \\ 
     &           &    8 &  3.15e-08 &  1.32e-15 &      \\ 
     &           &    9 &  4.63e-08 &  8.98e-16 &      \\ 
     &           &   10 &  3.15e-08 &  1.32e-15 &      \\ 
2114 &  2.11e+04 &   10 &           &           & iters  \\ 
 \hdashline 
     &           &    1 &  2.29e-08 &  2.36e-09 &      \\ 
     &           &    2 &  1.60e-08 &  6.54e-16 &      \\ 
     &           &    3 &  2.29e-08 &  4.56e-16 &      \\ 
     &           &    4 &  1.60e-08 &  6.54e-16 &      \\ 
     &           &    5 &  1.60e-08 &  4.56e-16 &      \\ 
     &           &    6 &  2.29e-08 &  4.55e-16 &      \\ 
     &           &    7 &  1.60e-08 &  6.54e-16 &      \\ 
     &           &    8 &  2.29e-08 &  4.56e-16 &      \\ 
     &           &    9 &  1.60e-08 &  6.54e-16 &      \\ 
     &           &   10 &  1.60e-08 &  4.56e-16 &      \\ 
2115 &  2.11e+04 &   10 &           &           & iters  \\ 
 \hdashline 
     &           &    1 &  4.67e-09 &  1.24e-09 &      \\ 
     &           &    2 &  4.66e-09 &  1.33e-16 &      \\ 
     &           &    3 &  4.66e-09 &  1.33e-16 &      \\ 
     &           &    4 &  4.65e-09 &  1.33e-16 &      \\ 
     &           &    5 &  4.64e-09 &  1.33e-16 &      \\ 
     &           &    6 &  4.64e-09 &  1.33e-16 &      \\ 
     &           &    7 &  4.63e-09 &  1.32e-16 &      \\ 
     &           &    8 &  4.62e-09 &  1.32e-16 &      \\ 
     &           &    9 &  4.62e-09 &  1.32e-16 &      \\ 
     &           &   10 &  7.31e-08 &  1.32e-16 &      \\ 
2116 &  2.12e+04 &   10 &           &           & iters  \\ 
 \hdashline 
     &           &    1 &  4.69e-08 &  4.15e-10 &      \\ 
     &           &    2 &  3.09e-08 &  1.34e-15 &      \\ 
     &           &    3 &  4.69e-08 &  8.81e-16 &      \\ 
     &           &    4 &  3.09e-08 &  1.34e-15 &      \\ 
     &           &    5 &  4.68e-08 &  8.82e-16 &      \\ 
     &           &    6 &  3.09e-08 &  1.34e-15 &      \\ 
     &           &    7 &  3.09e-08 &  8.83e-16 &      \\ 
     &           &    8 &  4.69e-08 &  8.81e-16 &      \\ 
     &           &    9 &  3.09e-08 &  1.34e-15 &      \\ 
     &           &   10 &  4.68e-08 &  8.82e-16 &      \\ 
2117 &  2.12e+04 &   10 &           &           & iters  \\ 
 \hdashline 
     &           &    1 &  5.96e-08 &  4.83e-10 &      \\ 
     &           &    2 &  1.82e-08 &  1.70e-15 &      \\ 
     &           &    3 &  1.81e-08 &  5.19e-16 &      \\ 
     &           &    4 &  5.96e-08 &  5.18e-16 &      \\ 
     &           &    5 &  1.82e-08 &  1.70e-15 &      \\ 
     &           &    6 &  1.82e-08 &  5.20e-16 &      \\ 
     &           &    7 &  1.82e-08 &  5.19e-16 &      \\ 
     &           &    8 &  1.81e-08 &  5.18e-16 &      \\ 
     &           &    9 &  5.96e-08 &  5.17e-16 &      \\ 
     &           &   10 &  1.82e-08 &  1.70e-15 &      \\ 
2118 &  2.12e+04 &   10 &           &           & iters  \\ 
 \hdashline 
     &           &    1 &  1.32e-08 &  5.07e-10 &      \\ 
     &           &    2 &  1.32e-08 &  3.76e-16 &      \\ 
     &           &    3 &  6.45e-08 &  3.76e-16 &      \\ 
     &           &    4 &  1.32e-08 &  1.84e-15 &      \\ 
     &           &    5 &  1.32e-08 &  3.78e-16 &      \\ 
     &           &    6 &  1.32e-08 &  3.77e-16 &      \\ 
     &           &    7 &  1.32e-08 &  3.77e-16 &      \\ 
     &           &    8 &  1.32e-08 &  3.76e-16 &      \\ 
     &           &    9 &  6.45e-08 &  3.76e-16 &      \\ 
     &           &   10 &  1.32e-08 &  1.84e-15 &      \\ 
2119 &  2.12e+04 &   10 &           &           & iters  \\ 
 \hdashline 
     &           &    1 &  2.72e-09 &  4.08e-10 &      \\ 
     &           &    2 &  2.71e-09 &  7.75e-17 &      \\ 
     &           &    3 &  2.71e-09 &  7.74e-17 &      \\ 
     &           &    4 &  2.70e-09 &  7.73e-17 &      \\ 
     &           &    5 &  2.70e-09 &  7.72e-17 &      \\ 
     &           &    6 &  2.70e-09 &  7.71e-17 &      \\ 
     &           &    7 &  2.69e-09 &  7.70e-17 &      \\ 
     &           &    8 &  2.69e-09 &  7.68e-17 &      \\ 
     &           &    9 &  2.68e-09 &  7.67e-17 &      \\ 
     &           &   10 &  2.68e-09 &  7.66e-17 &      \\ 
2120 &  2.12e+04 &   10 &           &           & iters  \\ 
 \hdashline 
     &           &    1 &  1.32e-08 &  5.81e-10 &      \\ 
     &           &    2 &  1.32e-08 &  3.76e-16 &      \\ 
     &           &    3 &  1.31e-08 &  3.76e-16 &      \\ 
     &           &    4 &  1.31e-08 &  3.75e-16 &      \\ 
     &           &    5 &  6.46e-08 &  3.75e-16 &      \\ 
     &           &    6 &  1.32e-08 &  1.84e-15 &      \\ 
     &           &    7 &  1.32e-08 &  3.77e-16 &      \\ 
     &           &    8 &  1.32e-08 &  3.76e-16 &      \\ 
     &           &    9 &  1.31e-08 &  3.76e-16 &      \\ 
     &           &   10 &  1.31e-08 &  3.75e-16 &      \\ 
2121 &  2.12e+04 &   10 &           &           & iters  \\ 
 \hdashline 
     &           &    1 &  7.72e-08 &  5.04e-10 &      \\ 
     &           &    2 &  7.83e-08 &  2.20e-15 &      \\ 
     &           &    3 &  7.72e-08 &  2.23e-15 &      \\ 
     &           &    4 &  7.83e-08 &  2.20e-15 &      \\ 
     &           &    5 &  7.72e-08 &  2.23e-15 &      \\ 
     &           &    6 &  7.83e-08 &  2.20e-15 &      \\ 
     &           &    7 &  7.72e-08 &  2.23e-15 &      \\ 
     &           &    8 &  7.83e-08 &  2.20e-15 &      \\ 
     &           &    9 &  7.72e-08 &  2.23e-15 &      \\ 
     &           &   10 &  7.83e-08 &  2.20e-15 &      \\ 
2122 &  2.12e+04 &   10 &           &           & iters  \\ 
 \hdashline 
     &           &    1 &  4.37e-08 &  9.11e-10 &      \\ 
     &           &    2 &  3.41e-08 &  1.25e-15 &      \\ 
     &           &    3 &  4.37e-08 &  9.72e-16 &      \\ 
     &           &    4 &  3.41e-08 &  1.25e-15 &      \\ 
     &           &    5 &  4.37e-08 &  9.73e-16 &      \\ 
     &           &    6 &  3.41e-08 &  1.25e-15 &      \\ 
     &           &    7 &  4.36e-08 &  9.73e-16 &      \\ 
     &           &    8 &  3.41e-08 &  1.25e-15 &      \\ 
     &           &    9 &  3.41e-08 &  9.73e-16 &      \\ 
     &           &   10 &  4.37e-08 &  9.72e-16 &      \\ 
2123 &  2.12e+04 &   10 &           &           & iters  \\ 
 \hdashline 
     &           &    1 &  8.91e-10 &  2.00e-09 &      \\ 
     &           &    2 &  8.89e-10 &  2.54e-17 &      \\ 
     &           &    3 &  8.88e-10 &  2.54e-17 &      \\ 
     &           &    4 &  8.87e-10 &  2.53e-17 &      \\ 
     &           &    5 &  8.86e-10 &  2.53e-17 &      \\ 
     &           &    6 &  8.84e-10 &  2.53e-17 &      \\ 
     &           &    7 &  8.83e-10 &  2.52e-17 &      \\ 
     &           &    8 &  8.82e-10 &  2.52e-17 &      \\ 
     &           &    9 &  8.81e-10 &  2.52e-17 &      \\ 
     &           &   10 &  8.79e-10 &  2.51e-17 &      \\ 
2124 &  2.12e+04 &   10 &           &           & iters  \\ 
 \hdashline 
     &           &    1 &  5.06e-08 &  1.78e-09 &      \\ 
     &           &    2 &  2.71e-08 &  1.45e-15 &      \\ 
     &           &    3 &  2.71e-08 &  7.74e-16 &      \\ 
     &           &    4 &  5.06e-08 &  7.73e-16 &      \\ 
     &           &    5 &  2.71e-08 &  1.45e-15 &      \\ 
     &           &    6 &  2.71e-08 &  7.74e-16 &      \\ 
     &           &    7 &  5.07e-08 &  7.73e-16 &      \\ 
     &           &    8 &  2.71e-08 &  1.45e-15 &      \\ 
     &           &    9 &  2.71e-08 &  7.74e-16 &      \\ 
     &           &   10 &  5.07e-08 &  7.73e-16 &      \\ 
2125 &  2.12e+04 &   10 &           &           & iters  \\ 
 \hdashline 
     &           &    1 &  2.04e-08 &  1.79e-09 &      \\ 
     &           &    2 &  2.04e-08 &  5.83e-16 &      \\ 
     &           &    3 &  5.73e-08 &  5.83e-16 &      \\ 
     &           &    4 &  2.05e-08 &  1.64e-15 &      \\ 
     &           &    5 &  2.04e-08 &  5.84e-16 &      \\ 
     &           &    6 &  2.04e-08 &  5.83e-16 &      \\ 
     &           &    7 &  5.73e-08 &  5.82e-16 &      \\ 
     &           &    8 &  2.05e-08 &  1.64e-15 &      \\ 
     &           &    9 &  2.04e-08 &  5.84e-16 &      \\ 
     &           &   10 &  2.04e-08 &  5.83e-16 &      \\ 
2126 &  2.12e+04 &   10 &           &           & iters  \\ 
 \hdashline 
     &           &    1 &  5.18e-08 &  1.22e-09 &      \\ 
     &           &    2 &  2.60e-08 &  1.48e-15 &      \\ 
     &           &    3 &  5.17e-08 &  7.42e-16 &      \\ 
     &           &    4 &  2.60e-08 &  1.48e-15 &      \\ 
     &           &    5 &  2.60e-08 &  7.43e-16 &      \\ 
     &           &    6 &  5.17e-08 &  7.42e-16 &      \\ 
     &           &    7 &  2.60e-08 &  1.48e-15 &      \\ 
     &           &    8 &  2.60e-08 &  7.43e-16 &      \\ 
     &           &    9 &  5.17e-08 &  7.42e-16 &      \\ 
     &           &   10 &  2.60e-08 &  1.48e-15 &      \\ 
2127 &  2.13e+04 &   10 &           &           & iters  \\ 
 \hdashline 
     &           &    1 &  4.14e-08 &  2.77e-10 &      \\ 
     &           &    2 &  3.64e-08 &  1.18e-15 &      \\ 
     &           &    3 &  3.63e-08 &  1.04e-15 &      \\ 
     &           &    4 &  4.14e-08 &  1.04e-15 &      \\ 
     &           &    5 &  3.63e-08 &  1.18e-15 &      \\ 
     &           &    6 &  4.14e-08 &  1.04e-15 &      \\ 
     &           &    7 &  3.63e-08 &  1.18e-15 &      \\ 
     &           &    8 &  4.14e-08 &  1.04e-15 &      \\ 
     &           &    9 &  3.63e-08 &  1.18e-15 &      \\ 
     &           &   10 &  4.14e-08 &  1.04e-15 &      \\ 
2128 &  2.13e+04 &   10 &           &           & iters  \\ 
 \hdashline 
     &           &    1 &  5.97e-08 &  4.86e-10 &      \\ 
     &           &    2 &  1.80e-08 &  1.70e-15 &      \\ 
     &           &    3 &  1.80e-08 &  5.15e-16 &      \\ 
     &           &    4 &  1.80e-08 &  5.14e-16 &      \\ 
     &           &    5 &  1.80e-08 &  5.13e-16 &      \\ 
     &           &    6 &  5.98e-08 &  5.13e-16 &      \\ 
     &           &    7 &  1.80e-08 &  1.71e-15 &      \\ 
     &           &    8 &  1.80e-08 &  5.14e-16 &      \\ 
     &           &    9 &  1.80e-08 &  5.14e-16 &      \\ 
     &           &   10 &  5.97e-08 &  5.13e-16 &      \\ 
2129 &  2.13e+04 &   10 &           &           & iters  \\ 
 \hdashline 
     &           &    1 &  7.74e-08 &  6.55e-10 &      \\ 
     &           &    2 &  4.40e-10 &  2.21e-15 &      \\ 
     &           &    3 &  4.39e-10 &  1.26e-17 &      \\ 
     &           &    4 &  4.39e-10 &  1.25e-17 &      \\ 
     &           &    5 &  4.38e-10 &  1.25e-17 &      \\ 
     &           &    6 &  4.38e-10 &  1.25e-17 &      \\ 
     &           &    7 &  4.37e-10 &  1.25e-17 &      \\ 
     &           &    8 &  4.36e-10 &  1.25e-17 &      \\ 
     &           &    9 &  4.36e-10 &  1.25e-17 &      \\ 
     &           &   10 &  4.35e-10 &  1.24e-17 &      \\ 
2130 &  2.13e+04 &   10 &           &           & iters  \\ 
 \hdashline 
     &           &    1 &  4.30e-08 &  5.66e-10 &      \\ 
     &           &    2 &  3.47e-08 &  1.23e-15 &      \\ 
     &           &    3 &  3.47e-08 &  9.92e-16 &      \\ 
     &           &    4 &  4.30e-08 &  9.90e-16 &      \\ 
     &           &    5 &  3.47e-08 &  1.23e-15 &      \\ 
     &           &    6 &  4.30e-08 &  9.91e-16 &      \\ 
     &           &    7 &  3.47e-08 &  1.23e-15 &      \\ 
     &           &    8 &  4.30e-08 &  9.91e-16 &      \\ 
     &           &    9 &  3.47e-08 &  1.23e-15 &      \\ 
     &           &   10 &  4.30e-08 &  9.91e-16 &      \\ 
2131 &  2.13e+04 &   10 &           &           & iters  \\ 
 \hdashline 
     &           &    1 &  2.10e-08 &  3.59e-10 &      \\ 
     &           &    2 &  2.10e-08 &  6.00e-16 &      \\ 
     &           &    3 &  2.10e-08 &  5.99e-16 &      \\ 
     &           &    4 &  5.67e-08 &  5.98e-16 &      \\ 
     &           &    5 &  2.10e-08 &  1.62e-15 &      \\ 
     &           &    6 &  2.10e-08 &  6.00e-16 &      \\ 
     &           &    7 &  2.10e-08 &  5.99e-16 &      \\ 
     &           &    8 &  5.68e-08 &  5.98e-16 &      \\ 
     &           &    9 &  2.10e-08 &  1.62e-15 &      \\ 
     &           &   10 &  2.10e-08 &  6.00e-16 &      \\ 
2132 &  2.13e+04 &   10 &           &           & iters  \\ 
 \hdashline 
     &           &    1 &  3.02e-08 &  5.42e-10 &      \\ 
     &           &    2 &  4.76e-08 &  8.61e-16 &      \\ 
     &           &    3 &  3.02e-08 &  1.36e-15 &      \\ 
     &           &    4 &  3.01e-08 &  8.61e-16 &      \\ 
     &           &    5 &  4.76e-08 &  8.60e-16 &      \\ 
     &           &    6 &  3.02e-08 &  1.36e-15 &      \\ 
     &           &    7 &  3.01e-08 &  8.61e-16 &      \\ 
     &           &    8 &  4.76e-08 &  8.60e-16 &      \\ 
     &           &    9 &  3.01e-08 &  1.36e-15 &      \\ 
     &           &   10 &  4.76e-08 &  8.60e-16 &      \\ 
2133 &  2.13e+04 &   10 &           &           & iters  \\ 
 \hdashline 
     &           &    1 &  1.51e-08 &  5.53e-10 &      \\ 
     &           &    2 &  1.51e-08 &  4.32e-16 &      \\ 
     &           &    3 &  6.26e-08 &  4.31e-16 &      \\ 
     &           &    4 &  1.52e-08 &  1.79e-15 &      \\ 
     &           &    5 &  1.51e-08 &  4.33e-16 &      \\ 
     &           &    6 &  1.51e-08 &  4.32e-16 &      \\ 
     &           &    7 &  1.51e-08 &  4.32e-16 &      \\ 
     &           &    8 &  6.26e-08 &  4.31e-16 &      \\ 
     &           &    9 &  1.52e-08 &  1.79e-15 &      \\ 
     &           &   10 &  1.52e-08 &  4.33e-16 &      \\ 
2134 &  2.13e+04 &   10 &           &           & iters  \\ 
 \hdashline 
     &           &    1 &  1.94e-08 &  4.60e-10 &      \\ 
     &           &    2 &  5.83e-08 &  5.54e-16 &      \\ 
     &           &    3 &  1.95e-08 &  1.66e-15 &      \\ 
     &           &    4 &  1.94e-08 &  5.56e-16 &      \\ 
     &           &    5 &  1.94e-08 &  5.55e-16 &      \\ 
     &           &    6 &  5.83e-08 &  5.54e-16 &      \\ 
     &           &    7 &  1.95e-08 &  1.66e-15 &      \\ 
     &           &    8 &  1.94e-08 &  5.56e-16 &      \\ 
     &           &    9 &  1.94e-08 &  5.55e-16 &      \\ 
     &           &   10 &  5.83e-08 &  5.54e-16 &      \\ 
2135 &  2.13e+04 &   10 &           &           & iters  \\ 
 \hdashline 
     &           &    1 &  3.80e-08 &  3.08e-10 &      \\ 
     &           &    2 &  3.97e-08 &  1.09e-15 &      \\ 
     &           &    3 &  3.81e-08 &  1.13e-15 &      \\ 
     &           &    4 &  3.97e-08 &  1.09e-15 &      \\ 
     &           &    5 &  3.81e-08 &  1.13e-15 &      \\ 
     &           &    6 &  3.97e-08 &  1.09e-15 &      \\ 
     &           &    7 &  3.81e-08 &  1.13e-15 &      \\ 
     &           &    8 &  3.97e-08 &  1.09e-15 &      \\ 
     &           &    9 &  3.81e-08 &  1.13e-15 &      \\ 
     &           &   10 &  3.97e-08 &  1.09e-15 &      \\ 
2136 &  2.14e+04 &   10 &           &           & iters  \\ 
 \hdashline 
     &           &    1 &  7.69e-08 &  8.27e-10 &      \\ 
     &           &    2 &  8.60e-10 &  2.20e-15 &      \\ 
     &           &    3 &  8.59e-10 &  2.46e-17 &      \\ 
     &           &    4 &  8.58e-10 &  2.45e-17 &      \\ 
     &           &    5 &  8.57e-10 &  2.45e-17 &      \\ 
     &           &    6 &  8.56e-10 &  2.45e-17 &      \\ 
     &           &    7 &  8.54e-10 &  2.44e-17 &      \\ 
     &           &    8 &  8.53e-10 &  2.44e-17 &      \\ 
     &           &    9 &  8.52e-10 &  2.43e-17 &      \\ 
     &           &   10 &  8.51e-10 &  2.43e-17 &      \\ 
2137 &  2.14e+04 &   10 &           &           & iters  \\ 
 \hdashline 
     &           &    1 &  1.52e-08 &  1.14e-09 &      \\ 
     &           &    2 &  1.52e-08 &  4.34e-16 &      \\ 
     &           &    3 &  6.25e-08 &  4.33e-16 &      \\ 
     &           &    4 &  1.52e-08 &  1.78e-15 &      \\ 
     &           &    5 &  1.52e-08 &  4.35e-16 &      \\ 
     &           &    6 &  1.52e-08 &  4.34e-16 &      \\ 
     &           &    7 &  1.52e-08 &  4.34e-16 &      \\ 
     &           &    8 &  6.25e-08 &  4.33e-16 &      \\ 
     &           &    9 &  1.52e-08 &  1.78e-15 &      \\ 
     &           &   10 &  1.52e-08 &  4.35e-16 &      \\ 
2138 &  2.14e+04 &   10 &           &           & iters  \\ 
 \hdashline 
     &           &    1 &  4.75e-08 &  5.80e-10 &      \\ 
     &           &    2 &  3.03e-08 &  1.36e-15 &      \\ 
     &           &    3 &  4.75e-08 &  8.64e-16 &      \\ 
     &           &    4 &  3.03e-08 &  1.35e-15 &      \\ 
     &           &    5 &  3.03e-08 &  8.65e-16 &      \\ 
     &           &    6 &  4.75e-08 &  8.64e-16 &      \\ 
     &           &    7 &  3.03e-08 &  1.35e-15 &      \\ 
     &           &    8 &  3.02e-08 &  8.64e-16 &      \\ 
     &           &    9 &  4.75e-08 &  8.63e-16 &      \\ 
     &           &   10 &  3.03e-08 &  1.36e-15 &      \\ 
2139 &  2.14e+04 &   10 &           &           & iters  \\ 
 \hdashline 
     &           &    1 &  5.93e-08 &  4.65e-10 &      \\ 
     &           &    2 &  1.85e-08 &  1.69e-15 &      \\ 
     &           &    3 &  1.85e-08 &  5.28e-16 &      \\ 
     &           &    4 &  1.84e-08 &  5.27e-16 &      \\ 
     &           &    5 &  5.93e-08 &  5.27e-16 &      \\ 
     &           &    6 &  1.85e-08 &  1.69e-15 &      \\ 
     &           &    7 &  1.85e-08 &  5.28e-16 &      \\ 
     &           &    8 &  1.85e-08 &  5.27e-16 &      \\ 
     &           &    9 &  1.84e-08 &  5.27e-16 &      \\ 
     &           &   10 &  5.93e-08 &  5.26e-16 &      \\ 
2140 &  2.14e+04 &   10 &           &           & iters  \\ 
 \hdashline 
     &           &    1 &  9.76e-09 &  1.02e-09 &      \\ 
     &           &    2 &  9.74e-09 &  2.78e-16 &      \\ 
     &           &    3 &  9.73e-09 &  2.78e-16 &      \\ 
     &           &    4 &  9.72e-09 &  2.78e-16 &      \\ 
     &           &    5 &  2.91e-08 &  2.77e-16 &      \\ 
     &           &    6 &  9.74e-09 &  8.32e-16 &      \\ 
     &           &    7 &  9.73e-09 &  2.78e-16 &      \\ 
     &           &    8 &  9.72e-09 &  2.78e-16 &      \\ 
     &           &    9 &  2.91e-08 &  2.77e-16 &      \\ 
     &           &   10 &  9.74e-09 &  8.32e-16 &      \\ 
2141 &  2.14e+04 &   10 &           &           & iters  \\ 
 \hdashline 
     &           &    1 &  1.22e-08 &  1.61e-09 &      \\ 
     &           &    2 &  6.55e-08 &  3.49e-16 &      \\ 
     &           &    3 &  1.23e-08 &  1.87e-15 &      \\ 
     &           &    4 &  1.23e-08 &  3.52e-16 &      \\ 
     &           &    5 &  1.23e-08 &  3.51e-16 &      \\ 
     &           &    6 &  1.23e-08 &  3.51e-16 &      \\ 
     &           &    7 &  1.22e-08 &  3.50e-16 &      \\ 
     &           &    8 &  6.55e-08 &  3.50e-16 &      \\ 
     &           &    9 &  9.00e-08 &  1.87e-15 &      \\ 
     &           &   10 &  6.55e-08 &  2.57e-15 &      \\ 
2142 &  2.14e+04 &   10 &           &           & iters  \\ 
 \hdashline 
     &           &    1 &  8.89e-09 &  1.44e-09 &      \\ 
     &           &    2 &  8.87e-09 &  2.54e-16 &      \\ 
     &           &    3 &  8.86e-09 &  2.53e-16 &      \\ 
     &           &    4 &  8.85e-09 &  2.53e-16 &      \\ 
     &           &    5 &  8.83e-09 &  2.53e-16 &      \\ 
     &           &    6 &  6.89e-08 &  2.52e-16 &      \\ 
     &           &    7 &  8.66e-08 &  1.97e-15 &      \\ 
     &           &    8 &  6.89e-08 &  2.47e-15 &      \\ 
     &           &    9 &  8.90e-09 &  1.97e-15 &      \\ 
     &           &   10 &  8.88e-09 &  2.54e-16 &      \\ 
2143 &  2.14e+04 &   10 &           &           & iters  \\ 
 \hdashline 
     &           &    1 &  2.62e-08 &  7.65e-10 &      \\ 
     &           &    2 &  5.15e-08 &  7.48e-16 &      \\ 
     &           &    3 &  2.62e-08 &  1.47e-15 &      \\ 
     &           &    4 &  2.62e-08 &  7.49e-16 &      \\ 
     &           &    5 &  5.15e-08 &  7.47e-16 &      \\ 
     &           &    6 &  2.62e-08 &  1.47e-15 &      \\ 
     &           &    7 &  2.62e-08 &  7.49e-16 &      \\ 
     &           &    8 &  5.15e-08 &  7.47e-16 &      \\ 
     &           &    9 &  2.62e-08 &  1.47e-15 &      \\ 
     &           &   10 &  2.62e-08 &  7.48e-16 &      \\ 
2144 &  2.14e+04 &   10 &           &           & iters  \\ 
 \hdashline 
     &           &    1 &  3.69e-08 &  8.22e-10 &      \\ 
     &           &    2 &  4.09e-08 &  1.05e-15 &      \\ 
     &           &    3 &  3.69e-08 &  1.17e-15 &      \\ 
     &           &    4 &  3.68e-08 &  1.05e-15 &      \\ 
     &           &    5 &  4.09e-08 &  1.05e-15 &      \\ 
     &           &    6 &  3.68e-08 &  1.17e-15 &      \\ 
     &           &    7 &  4.09e-08 &  1.05e-15 &      \\ 
     &           &    8 &  3.68e-08 &  1.17e-15 &      \\ 
     &           &    9 &  4.09e-08 &  1.05e-15 &      \\ 
     &           &   10 &  3.68e-08 &  1.17e-15 &      \\ 
2145 &  2.14e+04 &   10 &           &           & iters  \\ 
 \hdashline 
     &           &    1 &  1.61e-08 &  9.72e-10 &      \\ 
     &           &    2 &  1.61e-08 &  4.59e-16 &      \\ 
     &           &    3 &  2.28e-08 &  4.59e-16 &      \\ 
     &           &    4 &  1.61e-08 &  6.51e-16 &      \\ 
     &           &    5 &  2.28e-08 &  4.59e-16 &      \\ 
     &           &    6 &  1.61e-08 &  6.50e-16 &      \\ 
     &           &    7 &  2.28e-08 &  4.59e-16 &      \\ 
     &           &    8 &  1.61e-08 &  6.50e-16 &      \\ 
     &           &    9 &  1.61e-08 &  4.60e-16 &      \\ 
     &           &   10 &  2.28e-08 &  4.59e-16 &      \\ 
2146 &  2.14e+04 &   10 &           &           & iters  \\ 
 \hdashline 
     &           &    1 &  2.64e-08 &  7.92e-11 &      \\ 
     &           &    2 &  1.25e-08 &  7.52e-16 &      \\ 
     &           &    3 &  1.25e-08 &  3.58e-16 &      \\ 
     &           &    4 &  2.64e-08 &  3.57e-16 &      \\ 
     &           &    5 &  1.25e-08 &  7.52e-16 &      \\ 
     &           &    6 &  1.25e-08 &  3.58e-16 &      \\ 
     &           &    7 &  2.64e-08 &  3.57e-16 &      \\ 
     &           &    8 &  1.25e-08 &  7.52e-16 &      \\ 
     &           &    9 &  1.25e-08 &  3.58e-16 &      \\ 
     &           &   10 &  2.63e-08 &  3.57e-16 &      \\ 
2147 &  2.15e+04 &   10 &           &           & iters  \\ 
 \hdashline 
     &           &    1 &  2.90e-08 &  1.19e-09 &      \\ 
     &           &    2 &  2.89e-08 &  8.27e-16 &      \\ 
     &           &    3 &  4.88e-08 &  8.26e-16 &      \\ 
     &           &    4 &  2.90e-08 &  1.39e-15 &      \\ 
     &           &    5 &  4.88e-08 &  8.26e-16 &      \\ 
     &           &    6 &  2.90e-08 &  1.39e-15 &      \\ 
     &           &    7 &  2.89e-08 &  8.27e-16 &      \\ 
     &           &    8 &  4.88e-08 &  8.26e-16 &      \\ 
     &           &    9 &  2.90e-08 &  1.39e-15 &      \\ 
     &           &   10 &  2.89e-08 &  8.27e-16 &      \\ 
2148 &  2.15e+04 &   10 &           &           & iters  \\ 
 \hdashline 
     &           &    1 &  3.17e-08 &  1.25e-09 &      \\ 
     &           &    2 &  3.16e-08 &  9.03e-16 &      \\ 
     &           &    3 &  4.61e-08 &  9.02e-16 &      \\ 
     &           &    4 &  3.16e-08 &  1.32e-15 &      \\ 
     &           &    5 &  3.16e-08 &  9.03e-16 &      \\ 
     &           &    6 &  4.61e-08 &  9.02e-16 &      \\ 
     &           &    7 &  3.16e-08 &  1.32e-15 &      \\ 
     &           &    8 &  4.61e-08 &  9.02e-16 &      \\ 
     &           &    9 &  3.16e-08 &  1.32e-15 &      \\ 
     &           &   10 &  3.16e-08 &  9.03e-16 &      \\ 
2149 &  2.15e+04 &   10 &           &           & iters  \\ 
 \hdashline 
     &           &    1 &  9.06e-08 &  1.36e-09 &      \\ 
     &           &    2 &  6.49e-08 &  2.59e-15 &      \\ 
     &           &    3 &  1.29e-08 &  1.85e-15 &      \\ 
     &           &    4 &  1.28e-08 &  3.67e-16 &      \\ 
     &           &    5 &  1.28e-08 &  3.66e-16 &      \\ 
     &           &    6 &  6.49e-08 &  3.66e-16 &      \\ 
     &           &    7 &  9.06e-08 &  1.85e-15 &      \\ 
     &           &    8 &  6.49e-08 &  2.59e-15 &      \\ 
     &           &    9 &  1.29e-08 &  1.85e-15 &      \\ 
     &           &   10 &  1.28e-08 &  3.67e-16 &      \\ 
2150 &  2.15e+04 &   10 &           &           & iters  \\ 
 \hdashline 
     &           &    1 &  2.75e-08 &  1.78e-09 &      \\ 
     &           &    2 &  5.03e-08 &  7.84e-16 &      \\ 
     &           &    3 &  2.75e-08 &  1.43e-15 &      \\ 
     &           &    4 &  2.75e-08 &  7.85e-16 &      \\ 
     &           &    5 &  5.03e-08 &  7.84e-16 &      \\ 
     &           &    6 &  2.75e-08 &  1.43e-15 &      \\ 
     &           &    7 &  5.02e-08 &  7.85e-16 &      \\ 
     &           &    8 &  2.75e-08 &  1.43e-15 &      \\ 
     &           &    9 &  2.75e-08 &  7.86e-16 &      \\ 
     &           &   10 &  5.02e-08 &  7.84e-16 &      \\ 
2151 &  2.15e+04 &   10 &           &           & iters  \\ 
 \hdashline 
     &           &    1 &  2.37e-08 &  1.70e-09 &      \\ 
     &           &    2 &  2.37e-08 &  6.76e-16 &      \\ 
     &           &    3 &  2.36e-08 &  6.76e-16 &      \\ 
     &           &    4 &  5.41e-08 &  6.75e-16 &      \\ 
     &           &    5 &  2.37e-08 &  1.54e-15 &      \\ 
     &           &    6 &  2.36e-08 &  6.76e-16 &      \\ 
     &           &    7 &  5.41e-08 &  6.75e-16 &      \\ 
     &           &    8 &  2.37e-08 &  1.54e-15 &      \\ 
     &           &    9 &  2.37e-08 &  6.76e-16 &      \\ 
     &           &   10 &  5.41e-08 &  6.75e-16 &      \\ 
2152 &  2.15e+04 &   10 &           &           & iters  \\ 
 \hdashline 
     &           &    1 &  2.04e-08 &  3.98e-10 &      \\ 
     &           &    2 &  2.04e-08 &  5.83e-16 &      \\ 
     &           &    3 &  5.73e-08 &  5.82e-16 &      \\ 
     &           &    4 &  2.05e-08 &  1.64e-15 &      \\ 
     &           &    5 &  2.04e-08 &  5.84e-16 &      \\ 
     &           &    6 &  5.73e-08 &  5.83e-16 &      \\ 
     &           &    7 &  2.05e-08 &  1.64e-15 &      \\ 
     &           &    8 &  2.04e-08 &  5.84e-16 &      \\ 
     &           &    9 &  2.04e-08 &  5.84e-16 &      \\ 
     &           &   10 &  5.73e-08 &  5.83e-16 &      \\ 
2153 &  2.15e+04 &   10 &           &           & iters  \\ 
 \hdashline 
     &           &    1 &  7.05e-08 &  1.30e-09 &      \\ 
     &           &    2 &  4.62e-08 &  2.01e-15 &      \\ 
     &           &    3 &  3.16e-08 &  1.32e-15 &      \\ 
     &           &    4 &  3.15e-08 &  9.01e-16 &      \\ 
     &           &    5 &  4.62e-08 &  9.00e-16 &      \\ 
     &           &    6 &  3.16e-08 &  1.32e-15 &      \\ 
     &           &    7 &  4.62e-08 &  9.00e-16 &      \\ 
     &           &    8 &  3.16e-08 &  1.32e-15 &      \\ 
     &           &    9 &  3.15e-08 &  9.01e-16 &      \\ 
     &           &   10 &  4.62e-08 &  9.00e-16 &      \\ 
2154 &  2.15e+04 &   10 &           &           & iters  \\ 
 \hdashline 
     &           &    1 &  2.04e-08 &  5.43e-10 &      \\ 
     &           &    2 &  5.73e-08 &  5.83e-16 &      \\ 
     &           &    3 &  2.05e-08 &  1.64e-15 &      \\ 
     &           &    4 &  2.05e-08 &  5.85e-16 &      \\ 
     &           &    5 &  2.04e-08 &  5.84e-16 &      \\ 
     &           &    6 &  5.73e-08 &  5.83e-16 &      \\ 
     &           &    7 &  2.05e-08 &  1.64e-15 &      \\ 
     &           &    8 &  2.04e-08 &  5.84e-16 &      \\ 
     &           &    9 &  2.04e-08 &  5.84e-16 &      \\ 
     &           &   10 &  5.73e-08 &  5.83e-16 &      \\ 
2155 &  2.15e+04 &   10 &           &           & iters  \\ 
 \hdashline 
     &           &    1 &  2.87e-09 &  9.15e-10 &      \\ 
     &           &    2 &  2.86e-09 &  8.18e-17 &      \\ 
     &           &    3 &  2.86e-09 &  8.17e-17 &      \\ 
     &           &    4 &  2.85e-09 &  8.16e-17 &      \\ 
     &           &    5 &  2.85e-09 &  8.14e-17 &      \\ 
     &           &    6 &  2.85e-09 &  8.13e-17 &      \\ 
     &           &    7 &  2.84e-09 &  8.12e-17 &      \\ 
     &           &    8 &  2.84e-09 &  8.11e-17 &      \\ 
     &           &    9 &  2.83e-09 &  8.10e-17 &      \\ 
     &           &   10 &  2.83e-09 &  8.09e-17 &      \\ 
2156 &  2.16e+04 &   10 &           &           & iters  \\ 
 \hdashline 
     &           &    1 &  3.28e-08 &  5.24e-10 &      \\ 
     &           &    2 &  4.50e-08 &  9.35e-16 &      \\ 
     &           &    3 &  3.28e-08 &  1.28e-15 &      \\ 
     &           &    4 &  4.49e-08 &  9.36e-16 &      \\ 
     &           &    5 &  3.28e-08 &  1.28e-15 &      \\ 
     &           &    6 &  3.28e-08 &  9.36e-16 &      \\ 
     &           &    7 &  4.50e-08 &  9.35e-16 &      \\ 
     &           &    8 &  3.28e-08 &  1.28e-15 &      \\ 
     &           &    9 &  4.50e-08 &  9.35e-16 &      \\ 
     &           &   10 &  3.28e-08 &  1.28e-15 &      \\ 
2157 &  2.16e+04 &   10 &           &           & iters  \\ 
 \hdashline 
     &           &    1 &  2.30e-08 &  1.98e-09 &      \\ 
     &           &    2 &  5.47e-08 &  6.56e-16 &      \\ 
     &           &    3 &  2.30e-08 &  1.56e-15 &      \\ 
     &           &    4 &  2.30e-08 &  6.57e-16 &      \\ 
     &           &    5 &  5.47e-08 &  6.56e-16 &      \\ 
     &           &    6 &  2.30e-08 &  1.56e-15 &      \\ 
     &           &    7 &  2.30e-08 &  6.57e-16 &      \\ 
     &           &    8 &  5.47e-08 &  6.57e-16 &      \\ 
     &           &    9 &  2.30e-08 &  1.56e-15 &      \\ 
     &           &   10 &  2.30e-08 &  6.58e-16 &      \\ 
2158 &  2.16e+04 &   10 &           &           & iters  \\ 
 \hdashline 
     &           &    1 &  1.63e-08 &  9.56e-10 &      \\ 
     &           &    2 &  1.63e-08 &  4.66e-16 &      \\ 
     &           &    3 &  6.14e-08 &  4.65e-16 &      \\ 
     &           &    4 &  1.64e-08 &  1.75e-15 &      \\ 
     &           &    5 &  1.63e-08 &  4.67e-16 &      \\ 
     &           &    6 &  1.63e-08 &  4.66e-16 &      \\ 
     &           &    7 &  6.14e-08 &  4.66e-16 &      \\ 
     &           &    8 &  1.64e-08 &  1.75e-15 &      \\ 
     &           &    9 &  1.64e-08 &  4.68e-16 &      \\ 
     &           &   10 &  1.63e-08 &  4.67e-16 &      \\ 
2159 &  2.16e+04 &   10 &           &           & iters  \\ 
 \hdashline 
     &           &    1 &  1.36e-08 &  4.27e-10 &      \\ 
     &           &    2 &  6.41e-08 &  3.89e-16 &      \\ 
     &           &    3 &  1.37e-08 &  1.83e-15 &      \\ 
     &           &    4 &  1.37e-08 &  3.91e-16 &      \\ 
     &           &    5 &  1.37e-08 &  3.90e-16 &      \\ 
     &           &    6 &  1.36e-08 &  3.90e-16 &      \\ 
     &           &    7 &  1.36e-08 &  3.89e-16 &      \\ 
     &           &    8 &  6.41e-08 &  3.89e-16 &      \\ 
     &           &    9 &  1.37e-08 &  1.83e-15 &      \\ 
     &           &   10 &  1.37e-08 &  3.91e-16 &      \\ 
2160 &  2.16e+04 &   10 &           &           & iters  \\ 
 \hdashline 
     &           &    1 &  3.73e-08 &  1.12e-09 &      \\ 
     &           &    2 &  4.04e-08 &  1.06e-15 &      \\ 
     &           &    3 &  3.73e-08 &  1.15e-15 &      \\ 
     &           &    4 &  4.04e-08 &  1.06e-15 &      \\ 
     &           &    5 &  3.73e-08 &  1.15e-15 &      \\ 
     &           &    6 &  4.04e-08 &  1.06e-15 &      \\ 
     &           &    7 &  3.73e-08 &  1.15e-15 &      \\ 
     &           &    8 &  4.04e-08 &  1.06e-15 &      \\ 
     &           &    9 &  3.73e-08 &  1.15e-15 &      \\ 
     &           &   10 &  4.04e-08 &  1.07e-15 &      \\ 
2161 &  2.16e+04 &   10 &           &           & iters  \\ 
 \hdashline 
     &           &    1 &  2.02e-08 &  1.12e-09 &      \\ 
     &           &    2 &  1.87e-08 &  5.76e-16 &      \\ 
     &           &    3 &  2.02e-08 &  5.33e-16 &      \\ 
     &           &    4 &  1.87e-08 &  5.76e-16 &      \\ 
     &           &    5 &  2.02e-08 &  5.33e-16 &      \\ 
     &           &    6 &  1.87e-08 &  5.76e-16 &      \\ 
     &           &    7 &  2.02e-08 &  5.34e-16 &      \\ 
     &           &    8 &  1.87e-08 &  5.76e-16 &      \\ 
     &           &    9 &  2.02e-08 &  5.34e-16 &      \\ 
     &           &   10 &  1.87e-08 &  5.76e-16 &      \\ 
2162 &  2.16e+04 &   10 &           &           & iters  \\ 
 \hdashline 
     &           &    1 &  3.36e-08 &  1.33e-09 &      \\ 
     &           &    2 &  4.42e-08 &  9.58e-16 &      \\ 
     &           &    3 &  3.36e-08 &  1.26e-15 &      \\ 
     &           &    4 &  3.35e-08 &  9.59e-16 &      \\ 
     &           &    5 &  4.42e-08 &  9.57e-16 &      \\ 
     &           &    6 &  3.36e-08 &  1.26e-15 &      \\ 
     &           &    7 &  4.42e-08 &  9.58e-16 &      \\ 
     &           &    8 &  3.36e-08 &  1.26e-15 &      \\ 
     &           &    9 &  4.42e-08 &  9.58e-16 &      \\ 
     &           &   10 &  3.36e-08 &  1.26e-15 &      \\ 
2163 &  2.16e+04 &   10 &           &           & iters  \\ 
 \hdashline 
     &           &    1 &  2.46e-08 &  2.18e-09 &      \\ 
     &           &    2 &  2.45e-08 &  7.01e-16 &      \\ 
     &           &    3 &  5.32e-08 &  7.00e-16 &      \\ 
     &           &    4 &  2.46e-08 &  1.52e-15 &      \\ 
     &           &    5 &  2.45e-08 &  7.01e-16 &      \\ 
     &           &    6 &  5.32e-08 &  7.00e-16 &      \\ 
     &           &    7 &  2.46e-08 &  1.52e-15 &      \\ 
     &           &    8 &  2.45e-08 &  7.02e-16 &      \\ 
     &           &    9 &  5.32e-08 &  7.00e-16 &      \\ 
     &           &   10 &  2.46e-08 &  1.52e-15 &      \\ 
2164 &  2.16e+04 &   10 &           &           & iters  \\ 
 \hdashline 
     &           &    1 &  2.80e-09 &  1.29e-09 &      \\ 
     &           &    2 &  7.49e-08 &  8.00e-17 &      \\ 
     &           &    3 &  8.06e-08 &  2.14e-15 &      \\ 
     &           &    4 &  7.49e-08 &  2.30e-15 &      \\ 
     &           &    5 &  8.06e-08 &  2.14e-15 &      \\ 
     &           &    6 &  7.49e-08 &  2.30e-15 &      \\ 
     &           &    7 &  2.89e-09 &  2.14e-15 &      \\ 
     &           &    8 &  2.89e-09 &  8.25e-17 &      \\ 
     &           &    9 &  2.88e-09 &  8.24e-17 &      \\ 
     &           &   10 &  2.88e-09 &  8.23e-17 &      \\ 
2165 &  2.16e+04 &   10 &           &           & iters  \\ 
 \hdashline 
     &           &    1 &  4.27e-08 &  1.07e-09 &      \\ 
     &           &    2 &  3.50e-08 &  1.22e-15 &      \\ 
     &           &    3 &  4.27e-08 &  1.00e-15 &      \\ 
     &           &    4 &  3.51e-08 &  1.22e-15 &      \\ 
     &           &    5 &  4.27e-08 &  1.00e-15 &      \\ 
     &           &    6 &  3.51e-08 &  1.22e-15 &      \\ 
     &           &    7 &  4.27e-08 &  1.00e-15 &      \\ 
     &           &    8 &  3.51e-08 &  1.22e-15 &      \\ 
     &           &    9 &  3.50e-08 &  1.00e-15 &      \\ 
     &           &   10 &  4.27e-08 &  1.00e-15 &      \\ 
2166 &  2.16e+04 &   10 &           &           & iters  \\ 
 \hdashline 
     &           &    1 &  3.13e-09 &  1.44e-09 &      \\ 
     &           &    2 &  3.13e-09 &  8.94e-17 &      \\ 
     &           &    3 &  3.12e-09 &  8.93e-17 &      \\ 
     &           &    4 &  3.12e-09 &  8.92e-17 &      \\ 
     &           &    5 &  7.46e-08 &  8.91e-17 &      \\ 
     &           &    6 &  8.09e-08 &  2.13e-15 &      \\ 
     &           &    7 &  7.46e-08 &  2.31e-15 &      \\ 
     &           &    8 &  8.09e-08 &  2.13e-15 &      \\ 
     &           &    9 &  7.46e-08 &  2.31e-15 &      \\ 
     &           &   10 &  3.20e-09 &  2.13e-15 &      \\ 
2167 &  2.17e+04 &   10 &           &           & iters  \\ 
 \hdashline 
     &           &    1 &  3.65e-08 &  9.74e-10 &      \\ 
     &           &    2 &  4.13e-08 &  1.04e-15 &      \\ 
     &           &    3 &  3.65e-08 &  1.18e-15 &      \\ 
     &           &    4 &  4.13e-08 &  1.04e-15 &      \\ 
     &           &    5 &  3.65e-08 &  1.18e-15 &      \\ 
     &           &    6 &  4.13e-08 &  1.04e-15 &      \\ 
     &           &    7 &  3.65e-08 &  1.18e-15 &      \\ 
     &           &    8 &  3.64e-08 &  1.04e-15 &      \\ 
     &           &    9 &  4.13e-08 &  1.04e-15 &      \\ 
     &           &   10 &  3.64e-08 &  1.18e-15 &      \\ 
2168 &  2.17e+04 &   10 &           &           & iters  \\ 
 \hdashline 
     &           &    1 &  2.83e-09 &  1.69e-09 &      \\ 
     &           &    2 &  2.83e-09 &  8.09e-17 &      \\ 
     &           &    3 &  2.83e-09 &  8.08e-17 &      \\ 
     &           &    4 &  2.82e-09 &  8.07e-17 &      \\ 
     &           &    5 &  2.82e-09 &  8.05e-17 &      \\ 
     &           &    6 &  2.81e-09 &  8.04e-17 &      \\ 
     &           &    7 &  2.81e-09 &  8.03e-17 &      \\ 
     &           &    8 &  2.81e-09 &  8.02e-17 &      \\ 
     &           &    9 &  2.80e-09 &  8.01e-17 &      \\ 
     &           &   10 &  2.80e-09 &  8.00e-17 &      \\ 
2169 &  2.17e+04 &   10 &           &           & iters  \\ 
 \hdashline 
     &           &    1 &  3.62e-08 &  1.50e-09 &      \\ 
     &           &    2 &  2.73e-09 &  1.03e-15 &      \\ 
     &           &    3 &  2.73e-09 &  7.80e-17 &      \\ 
     &           &    4 &  2.72e-09 &  7.79e-17 &      \\ 
     &           &    5 &  2.72e-09 &  7.77e-17 &      \\ 
     &           &    6 &  2.72e-09 &  7.76e-17 &      \\ 
     &           &    7 &  2.71e-09 &  7.75e-17 &      \\ 
     &           &    8 &  2.71e-09 &  7.74e-17 &      \\ 
     &           &    9 &  2.70e-09 &  7.73e-17 &      \\ 
     &           &   10 &  2.70e-09 &  7.72e-17 &      \\ 
2170 &  2.17e+04 &   10 &           &           & iters  \\ 
 \hdashline 
     &           &    1 &  2.18e-09 &  7.89e-10 &      \\ 
     &           &    2 &  2.18e-09 &  6.23e-17 &      \\ 
     &           &    3 &  2.18e-09 &  6.22e-17 &      \\ 
     &           &    4 &  2.17e-09 &  6.21e-17 &      \\ 
     &           &    5 &  2.17e-09 &  6.20e-17 &      \\ 
     &           &    6 &  2.17e-09 &  6.19e-17 &      \\ 
     &           &    7 &  2.16e-09 &  6.18e-17 &      \\ 
     &           &    8 &  2.16e-09 &  6.17e-17 &      \\ 
     &           &    9 &  2.16e-09 &  6.16e-17 &      \\ 
     &           &   10 &  2.15e-09 &  6.16e-17 &      \\ 
2171 &  2.17e+04 &   10 &           &           & iters  \\ 
 \hdashline 
     &           &    1 &  3.11e-08 &  1.17e-09 &      \\ 
     &           &    2 &  4.66e-08 &  8.88e-16 &      \\ 
     &           &    3 &  3.11e-08 &  1.33e-15 &      \\ 
     &           &    4 &  3.11e-08 &  8.89e-16 &      \\ 
     &           &    5 &  4.66e-08 &  8.87e-16 &      \\ 
     &           &    6 &  3.11e-08 &  1.33e-15 &      \\ 
     &           &    7 &  4.66e-08 &  8.88e-16 &      \\ 
     &           &    8 &  3.11e-08 &  1.33e-15 &      \\ 
     &           &    9 &  3.11e-08 &  8.89e-16 &      \\ 
     &           &   10 &  4.66e-08 &  8.87e-16 &      \\ 
2172 &  2.17e+04 &   10 &           &           & iters  \\ 
 \hdashline 
     &           &    1 &  1.24e-08 &  9.71e-10 &      \\ 
     &           &    2 &  1.24e-08 &  3.55e-16 &      \\ 
     &           &    3 &  2.64e-08 &  3.55e-16 &      \\ 
     &           &    4 &  1.24e-08 &  7.55e-16 &      \\ 
     &           &    5 &  1.24e-08 &  3.55e-16 &      \\ 
     &           &    6 &  2.64e-08 &  3.55e-16 &      \\ 
     &           &    7 &  1.24e-08 &  7.55e-16 &      \\ 
     &           &    8 &  1.24e-08 &  3.55e-16 &      \\ 
     &           &    9 &  2.64e-08 &  3.55e-16 &      \\ 
     &           &   10 &  1.24e-08 &  7.54e-16 &      \\ 
2173 &  2.17e+04 &   10 &           &           & iters  \\ 
 \hdashline 
     &           &    1 &  2.94e-08 &  6.15e-10 &      \\ 
     &           &    2 &  4.83e-08 &  8.39e-16 &      \\ 
     &           &    3 &  2.94e-08 &  1.38e-15 &      \\ 
     &           &    4 &  4.83e-08 &  8.40e-16 &      \\ 
     &           &    5 &  2.94e-08 &  1.38e-15 &      \\ 
     &           &    6 &  2.94e-08 &  8.40e-16 &      \\ 
     &           &    7 &  4.83e-08 &  8.39e-16 &      \\ 
     &           &    8 &  2.94e-08 &  1.38e-15 &      \\ 
     &           &    9 &  2.94e-08 &  8.40e-16 &      \\ 
     &           &   10 &  4.83e-08 &  8.39e-16 &      \\ 
2174 &  2.17e+04 &   10 &           &           & iters  \\ 
 \hdashline 
     &           &    1 &  7.18e-09 &  1.53e-09 &      \\ 
     &           &    2 &  7.17e-09 &  2.05e-16 &      \\ 
     &           &    3 &  7.16e-09 &  2.05e-16 &      \\ 
     &           &    4 &  7.15e-09 &  2.04e-16 &      \\ 
     &           &    5 &  7.14e-09 &  2.04e-16 &      \\ 
     &           &    6 &  7.13e-09 &  2.04e-16 &      \\ 
     &           &    7 &  7.06e-08 &  2.03e-16 &      \\ 
     &           &    8 &  4.61e-08 &  2.01e-15 &      \\ 
     &           &    9 &  7.15e-09 &  1.31e-15 &      \\ 
     &           &   10 &  7.14e-09 &  2.04e-16 &      \\ 
2175 &  2.17e+04 &   10 &           &           & iters  \\ 
 \hdashline 
     &           &    1 &  5.18e-08 &  5.26e-10 &      \\ 
     &           &    2 &  2.60e-08 &  1.48e-15 &      \\ 
     &           &    3 &  2.59e-08 &  7.41e-16 &      \\ 
     &           &    4 &  5.18e-08 &  7.40e-16 &      \\ 
     &           &    5 &  2.60e-08 &  1.48e-15 &      \\ 
     &           &    6 &  2.59e-08 &  7.41e-16 &      \\ 
     &           &    7 &  5.18e-08 &  7.40e-16 &      \\ 
     &           &    8 &  2.60e-08 &  1.48e-15 &      \\ 
     &           &    9 &  2.59e-08 &  7.41e-16 &      \\ 
     &           &   10 &  5.18e-08 &  7.40e-16 &      \\ 
2176 &  2.18e+04 &   10 &           &           & iters  \\ 
 \hdashline 
     &           &    1 &  2.21e-08 &  1.21e-09 &      \\ 
     &           &    2 &  2.21e-08 &  6.32e-16 &      \\ 
     &           &    3 &  2.21e-08 &  6.31e-16 &      \\ 
     &           &    4 &  5.56e-08 &  6.30e-16 &      \\ 
     &           &    5 &  2.21e-08 &  1.59e-15 &      \\ 
     &           &    6 &  2.21e-08 &  6.32e-16 &      \\ 
     &           &    7 &  5.56e-08 &  6.31e-16 &      \\ 
     &           &    8 &  2.21e-08 &  1.59e-15 &      \\ 
     &           &    9 &  2.21e-08 &  6.32e-16 &      \\ 
     &           &   10 &  2.21e-08 &  6.31e-16 &      \\ 
2177 &  2.18e+04 &   10 &           &           & iters  \\ 
 \hdashline 
     &           &    1 &  1.87e-08 &  2.06e-09 &      \\ 
     &           &    2 &  1.86e-08 &  5.33e-16 &      \\ 
     &           &    3 &  1.86e-08 &  5.32e-16 &      \\ 
     &           &    4 &  5.91e-08 &  5.31e-16 &      \\ 
     &           &    5 &  1.87e-08 &  1.69e-15 &      \\ 
     &           &    6 &  1.87e-08 &  5.33e-16 &      \\ 
     &           &    7 &  1.86e-08 &  5.32e-16 &      \\ 
     &           &    8 &  5.91e-08 &  5.32e-16 &      \\ 
     &           &    9 &  1.87e-08 &  1.69e-15 &      \\ 
     &           &   10 &  1.87e-08 &  5.33e-16 &      \\ 
2178 &  2.18e+04 &   10 &           &           & iters  \\ 
 \hdashline 
     &           &    1 &  2.18e-08 &  8.34e-10 &      \\ 
     &           &    2 &  5.59e-08 &  6.21e-16 &      \\ 
     &           &    3 &  2.18e-08 &  1.60e-15 &      \\ 
     &           &    4 &  2.18e-08 &  6.23e-16 &      \\ 
     &           &    5 &  2.18e-08 &  6.22e-16 &      \\ 
     &           &    6 &  5.60e-08 &  6.21e-16 &      \\ 
     &           &    7 &  2.18e-08 &  1.60e-15 &      \\ 
     &           &    8 &  2.18e-08 &  6.22e-16 &      \\ 
     &           &    9 &  2.17e-08 &  6.21e-16 &      \\ 
     &           &   10 &  5.60e-08 &  6.21e-16 &      \\ 
2179 &  2.18e+04 &   10 &           &           & iters  \\ 
 \hdashline 
     &           &    1 &  6.77e-09 &  2.36e-09 &      \\ 
     &           &    2 &  6.76e-09 &  1.93e-16 &      \\ 
     &           &    3 &  6.75e-09 &  1.93e-16 &      \\ 
     &           &    4 &  6.74e-09 &  1.93e-16 &      \\ 
     &           &    5 &  6.73e-09 &  1.92e-16 &      \\ 
     &           &    6 &  6.72e-09 &  1.92e-16 &      \\ 
     &           &    7 &  7.10e-08 &  1.92e-16 &      \\ 
     &           &    8 &  8.45e-08 &  2.03e-15 &      \\ 
     &           &    9 &  7.10e-08 &  2.41e-15 &      \\ 
     &           &   10 &  6.80e-09 &  2.03e-15 &      \\ 
2180 &  2.18e+04 &   10 &           &           & iters  \\ 
 \hdashline 
     &           &    1 &  1.31e-08 &  4.11e-09 &      \\ 
     &           &    2 &  1.31e-08 &  3.74e-16 &      \\ 
     &           &    3 &  6.46e-08 &  3.73e-16 &      \\ 
     &           &    4 &  1.31e-08 &  1.84e-15 &      \\ 
     &           &    5 &  1.31e-08 &  3.75e-16 &      \\ 
     &           &    6 &  1.31e-08 &  3.75e-16 &      \\ 
     &           &    7 &  1.31e-08 &  3.74e-16 &      \\ 
     &           &    8 &  6.46e-08 &  3.74e-16 &      \\ 
     &           &    9 &  5.20e-08 &  1.84e-15 &      \\ 
     &           &   10 &  1.31e-08 &  1.48e-15 &      \\ 
2181 &  2.18e+04 &   10 &           &           & iters  \\ 
 \hdashline 
     &           &    1 &  4.21e-08 &  1.39e-09 &      \\ 
     &           &    2 &  3.56e-08 &  1.20e-15 &      \\ 
     &           &    3 &  4.21e-08 &  1.02e-15 &      \\ 
     &           &    4 &  3.56e-08 &  1.20e-15 &      \\ 
     &           &    5 &  4.21e-08 &  1.02e-15 &      \\ 
     &           &    6 &  3.56e-08 &  1.20e-15 &      \\ 
     &           &    7 &  4.21e-08 &  1.02e-15 &      \\ 
     &           &    8 &  3.56e-08 &  1.20e-15 &      \\ 
     &           &    9 &  4.21e-08 &  1.02e-15 &      \\ 
     &           &   10 &  3.56e-08 &  1.20e-15 &      \\ 
2182 &  2.18e+04 &   10 &           &           & iters  \\ 
 \hdashline 
     &           &    1 &  3.07e-08 &  1.78e-09 &      \\ 
     &           &    2 &  4.70e-08 &  8.77e-16 &      \\ 
     &           &    3 &  3.08e-08 &  1.34e-15 &      \\ 
     &           &    4 &  3.07e-08 &  8.78e-16 &      \\ 
     &           &    5 &  4.70e-08 &  8.77e-16 &      \\ 
     &           &    6 &  3.07e-08 &  1.34e-15 &      \\ 
     &           &    7 &  4.70e-08 &  8.77e-16 &      \\ 
     &           &    8 &  3.08e-08 &  1.34e-15 &      \\ 
     &           &    9 &  3.07e-08 &  8.78e-16 &      \\ 
     &           &   10 &  4.70e-08 &  8.77e-16 &      \\ 
2183 &  2.18e+04 &   10 &           &           & iters  \\ 
 \hdashline 
     &           &    1 &  6.91e-09 &  1.38e-09 &      \\ 
     &           &    2 &  6.90e-09 &  1.97e-16 &      \\ 
     &           &    3 &  6.89e-09 &  1.97e-16 &      \\ 
     &           &    4 &  6.88e-09 &  1.97e-16 &      \\ 
     &           &    5 &  6.87e-09 &  1.96e-16 &      \\ 
     &           &    6 &  6.86e-09 &  1.96e-16 &      \\ 
     &           &    7 &  7.08e-08 &  1.96e-16 &      \\ 
     &           &    8 &  4.58e-08 &  2.02e-15 &      \\ 
     &           &    9 &  6.89e-09 &  1.31e-15 &      \\ 
     &           &   10 &  6.88e-09 &  1.97e-16 &      \\ 
2184 &  2.18e+04 &   10 &           &           & iters  \\ 
 \hdashline 
     &           &    1 &  5.42e-08 &  3.01e-10 &      \\ 
     &           &    2 &  2.36e-08 &  1.55e-15 &      \\ 
     &           &    3 &  2.35e-08 &  6.73e-16 &      \\ 
     &           &    4 &  5.42e-08 &  6.72e-16 &      \\ 
     &           &    5 &  2.36e-08 &  1.55e-15 &      \\ 
     &           &    6 &  2.36e-08 &  6.73e-16 &      \\ 
     &           &    7 &  5.42e-08 &  6.72e-16 &      \\ 
     &           &    8 &  2.36e-08 &  1.55e-15 &      \\ 
     &           &    9 &  2.36e-08 &  6.74e-16 &      \\ 
     &           &   10 &  2.35e-08 &  6.73e-16 &      \\ 
2185 &  2.18e+04 &   10 &           &           & iters  \\ 
 \hdashline 
     &           &    1 &  1.35e-08 &  1.53e-10 &      \\ 
     &           &    2 &  1.35e-08 &  3.86e-16 &      \\ 
     &           &    3 &  1.35e-08 &  3.86e-16 &      \\ 
     &           &    4 &  1.35e-08 &  3.85e-16 &      \\ 
     &           &    5 &  6.42e-08 &  3.85e-16 &      \\ 
     &           &    6 &  5.24e-08 &  1.83e-15 &      \\ 
     &           &    7 &  1.35e-08 &  1.50e-15 &      \\ 
     &           &    8 &  6.42e-08 &  3.85e-16 &      \\ 
     &           &    9 &  1.35e-08 &  1.83e-15 &      \\ 
     &           &   10 &  1.35e-08 &  3.87e-16 &      \\ 
2186 &  2.18e+04 &   10 &           &           & iters  \\ 
 \hdashline 
     &           &    1 &  2.91e-08 &  6.21e-10 &      \\ 
     &           &    2 &  4.86e-08 &  8.32e-16 &      \\ 
     &           &    3 &  2.92e-08 &  1.39e-15 &      \\ 
     &           &    4 &  4.86e-08 &  8.32e-16 &      \\ 
     &           &    5 &  2.92e-08 &  1.39e-15 &      \\ 
     &           &    6 &  2.92e-08 &  8.33e-16 &      \\ 
     &           &    7 &  4.86e-08 &  8.32e-16 &      \\ 
     &           &    8 &  2.92e-08 &  1.39e-15 &      \\ 
     &           &    9 &  2.91e-08 &  8.33e-16 &      \\ 
     &           &   10 &  4.86e-08 &  8.32e-16 &      \\ 
2187 &  2.19e+04 &   10 &           &           & iters  \\ 
 \hdashline 
     &           &    1 &  5.45e-09 &  9.28e-10 &      \\ 
     &           &    2 &  5.44e-09 &  1.56e-16 &      \\ 
     &           &    3 &  5.44e-09 &  1.55e-16 &      \\ 
     &           &    4 &  5.43e-09 &  1.55e-16 &      \\ 
     &           &    5 &  5.42e-09 &  1.55e-16 &      \\ 
     &           &    6 &  5.41e-09 &  1.55e-16 &      \\ 
     &           &    7 &  5.41e-09 &  1.54e-16 &      \\ 
     &           &    8 &  7.23e-08 &  1.54e-16 &      \\ 
     &           &    9 &  8.32e-08 &  2.06e-15 &      \\ 
     &           &   10 &  7.23e-08 &  2.37e-15 &      \\ 
2188 &  2.19e+04 &   10 &           &           & iters  \\ 
 \hdashline 
     &           &    1 &  4.60e-09 &  1.35e-09 &      \\ 
     &           &    2 &  4.60e-09 &  1.31e-16 &      \\ 
     &           &    3 &  4.59e-09 &  1.31e-16 &      \\ 
     &           &    4 &  4.58e-09 &  1.31e-16 &      \\ 
     &           &    5 &  4.58e-09 &  1.31e-16 &      \\ 
     &           &    6 &  4.57e-09 &  1.31e-16 &      \\ 
     &           &    7 &  7.31e-08 &  1.30e-16 &      \\ 
     &           &    8 &  4.67e-09 &  2.09e-15 &      \\ 
     &           &    9 &  4.66e-09 &  1.33e-16 &      \\ 
     &           &   10 &  4.66e-09 &  1.33e-16 &      \\ 
2189 &  2.19e+04 &   10 &           &           & iters  \\ 
 \hdashline 
     &           &    1 &  6.78e-09 &  8.76e-10 &      \\ 
     &           &    2 &  7.09e-08 &  1.93e-16 &      \\ 
     &           &    3 &  8.46e-08 &  2.02e-15 &      \\ 
     &           &    4 &  7.09e-08 &  2.41e-15 &      \\ 
     &           &    5 &  6.85e-09 &  2.02e-15 &      \\ 
     &           &    6 &  6.84e-09 &  1.96e-16 &      \\ 
     &           &    7 &  6.83e-09 &  1.95e-16 &      \\ 
     &           &    8 &  6.82e-09 &  1.95e-16 &      \\ 
     &           &    9 &  6.81e-09 &  1.95e-16 &      \\ 
     &           &   10 &  6.80e-09 &  1.94e-16 &      \\ 
2190 &  2.19e+04 &   10 &           &           & iters  \\ 
 \hdashline 
     &           &    1 &  4.56e-08 &  6.52e-10 &      \\ 
     &           &    2 &  3.22e-08 &  1.30e-15 &      \\ 
     &           &    3 &  3.21e-08 &  9.18e-16 &      \\ 
     &           &    4 &  4.56e-08 &  9.17e-16 &      \\ 
     &           &    5 &  3.21e-08 &  1.30e-15 &      \\ 
     &           &    6 &  4.56e-08 &  9.17e-16 &      \\ 
     &           &    7 &  3.22e-08 &  1.30e-15 &      \\ 
     &           &    8 &  4.56e-08 &  9.18e-16 &      \\ 
     &           &    9 &  3.22e-08 &  1.30e-15 &      \\ 
     &           &   10 &  3.21e-08 &  9.19e-16 &      \\ 
2191 &  2.19e+04 &   10 &           &           & iters  \\ 
 \hdashline 
     &           &    1 &  9.49e-09 &  1.23e-09 &      \\ 
     &           &    2 &  9.47e-09 &  2.71e-16 &      \\ 
     &           &    3 &  9.46e-09 &  2.70e-16 &      \\ 
     &           &    4 &  9.45e-09 &  2.70e-16 &      \\ 
     &           &    5 &  9.43e-09 &  2.70e-16 &      \\ 
     &           &    6 &  9.42e-09 &  2.69e-16 &      \\ 
     &           &    7 &  9.40e-09 &  2.69e-16 &      \\ 
     &           &    8 &  6.83e-08 &  2.68e-16 &      \\ 
     &           &    9 &  9.49e-09 &  1.95e-15 &      \\ 
     &           &   10 &  9.48e-09 &  2.71e-16 &      \\ 
2192 &  2.19e+04 &   10 &           &           & iters  \\ 
 \hdashline 
     &           &    1 &  2.66e-08 &  4.37e-10 &      \\ 
     &           &    2 &  2.66e-08 &  7.59e-16 &      \\ 
     &           &    3 &  5.12e-08 &  7.58e-16 &      \\ 
     &           &    4 &  2.66e-08 &  1.46e-15 &      \\ 
     &           &    5 &  2.66e-08 &  7.59e-16 &      \\ 
     &           &    6 &  5.12e-08 &  7.58e-16 &      \\ 
     &           &    7 &  2.66e-08 &  1.46e-15 &      \\ 
     &           &    8 &  2.66e-08 &  7.59e-16 &      \\ 
     &           &    9 &  5.12e-08 &  7.58e-16 &      \\ 
     &           &   10 &  2.66e-08 &  1.46e-15 &      \\ 
2193 &  2.19e+04 &   10 &           &           & iters  \\ 
 \hdashline 
     &           &    1 &  2.53e-08 &  1.51e-09 &      \\ 
     &           &    2 &  2.53e-08 &  7.23e-16 &      \\ 
     &           &    3 &  5.24e-08 &  7.21e-16 &      \\ 
     &           &    4 &  2.53e-08 &  1.50e-15 &      \\ 
     &           &    5 &  2.53e-08 &  7.23e-16 &      \\ 
     &           &    6 &  5.24e-08 &  7.22e-16 &      \\ 
     &           &    7 &  2.53e-08 &  1.50e-15 &      \\ 
     &           &    8 &  2.53e-08 &  7.23e-16 &      \\ 
     &           &    9 &  5.24e-08 &  7.22e-16 &      \\ 
     &           &   10 &  2.53e-08 &  1.50e-15 &      \\ 
2194 &  2.19e+04 &   10 &           &           & iters  \\ 
 \hdashline 
     &           &    1 &  4.94e-09 &  2.65e-09 &      \\ 
     &           &    2 &  4.93e-09 &  1.41e-16 &      \\ 
     &           &    3 &  4.92e-09 &  1.41e-16 &      \\ 
     &           &    4 &  4.92e-09 &  1.41e-16 &      \\ 
     &           &    5 &  3.39e-08 &  1.40e-16 &      \\ 
     &           &    6 &  4.96e-09 &  9.69e-16 &      \\ 
     &           &    7 &  4.95e-09 &  1.42e-16 &      \\ 
     &           &    8 &  4.94e-09 &  1.41e-16 &      \\ 
     &           &    9 &  4.94e-09 &  1.41e-16 &      \\ 
     &           &   10 &  4.93e-09 &  1.41e-16 &      \\ 
2195 &  2.19e+04 &   10 &           &           & iters  \\ 
 \hdashline 
     &           &    1 &  2.52e-08 &  1.84e-09 &      \\ 
     &           &    2 &  1.37e-08 &  7.19e-16 &      \\ 
     &           &    3 &  1.37e-08 &  3.91e-16 &      \\ 
     &           &    4 &  2.52e-08 &  3.90e-16 &      \\ 
     &           &    5 &  1.37e-08 &  7.19e-16 &      \\ 
     &           &    6 &  1.37e-08 &  3.91e-16 &      \\ 
     &           &    7 &  2.52e-08 &  3.90e-16 &      \\ 
     &           &    8 &  1.37e-08 &  7.19e-16 &      \\ 
     &           &    9 &  2.52e-08 &  3.91e-16 &      \\ 
     &           &   10 &  1.37e-08 &  7.19e-16 &      \\ 
2196 &  2.20e+04 &   10 &           &           & iters  \\ 
 \hdashline 
     &           &    1 &  1.41e-08 &  1.13e-09 &      \\ 
     &           &    2 &  1.41e-08 &  4.03e-16 &      \\ 
     &           &    3 &  6.36e-08 &  4.03e-16 &      \\ 
     &           &    4 &  1.42e-08 &  1.82e-15 &      \\ 
     &           &    5 &  1.42e-08 &  4.05e-16 &      \\ 
     &           &    6 &  1.41e-08 &  4.04e-16 &      \\ 
     &           &    7 &  1.41e-08 &  4.03e-16 &      \\ 
     &           &    8 &  1.41e-08 &  4.03e-16 &      \\ 
     &           &    9 &  6.36e-08 &  4.02e-16 &      \\ 
     &           &   10 &  1.42e-08 &  1.82e-15 &      \\ 
2197 &  2.20e+04 &   10 &           &           & iters  \\ 
 \hdashline 
     &           &    1 &  2.52e-08 &  1.92e-09 &      \\ 
     &           &    2 &  2.52e-08 &  7.19e-16 &      \\ 
     &           &    3 &  5.26e-08 &  7.18e-16 &      \\ 
     &           &    4 &  2.52e-08 &  1.50e-15 &      \\ 
     &           &    5 &  2.52e-08 &  7.19e-16 &      \\ 
     &           &    6 &  5.26e-08 &  7.18e-16 &      \\ 
     &           &    7 &  2.52e-08 &  1.50e-15 &      \\ 
     &           &    8 &  2.52e-08 &  7.19e-16 &      \\ 
     &           &    9 &  5.26e-08 &  7.18e-16 &      \\ 
     &           &   10 &  2.52e-08 &  1.50e-15 &      \\ 
2198 &  2.20e+04 &   10 &           &           & iters  \\ 
 \hdashline 
     &           &    1 &  1.38e-08 &  1.94e-09 &      \\ 
     &           &    2 &  1.38e-08 &  3.93e-16 &      \\ 
     &           &    3 &  6.39e-08 &  3.93e-16 &      \\ 
     &           &    4 &  1.38e-08 &  1.83e-15 &      \\ 
     &           &    5 &  1.38e-08 &  3.95e-16 &      \\ 
     &           &    6 &  1.38e-08 &  3.94e-16 &      \\ 
     &           &    7 &  1.38e-08 &  3.94e-16 &      \\ 
     &           &    8 &  1.38e-08 &  3.93e-16 &      \\ 
     &           &    9 &  6.40e-08 &  3.93e-16 &      \\ 
     &           &   10 &  1.38e-08 &  1.83e-15 &      \\ 
2199 &  2.20e+04 &   10 &           &           & iters  \\ 
 \hdashline 
     &           &    1 &  1.60e-08 &  1.30e-10 &      \\ 
     &           &    2 &  2.29e-08 &  4.55e-16 &      \\ 
     &           &    3 &  1.60e-08 &  6.54e-16 &      \\ 
     &           &    4 &  2.29e-08 &  4.56e-16 &      \\ 
     &           &    5 &  1.60e-08 &  6.54e-16 &      \\ 
     &           &    6 &  1.60e-08 &  4.56e-16 &      \\ 
     &           &    7 &  2.29e-08 &  4.55e-16 &      \\ 
     &           &    8 &  1.60e-08 &  6.54e-16 &      \\ 
     &           &    9 &  2.29e-08 &  4.56e-16 &      \\ 
     &           &   10 &  1.60e-08 &  6.54e-16 &      \\ 
2200 &  2.20e+04 &   10 &           &           & iters  \\ 
 \hdashline 
     &           &    1 &  5.76e-09 &  1.83e-09 &      \\ 
     &           &    2 &  5.75e-09 &  1.64e-16 &      \\ 
     &           &    3 &  5.74e-09 &  1.64e-16 &      \\ 
     &           &    4 &  5.73e-09 &  1.64e-16 &      \\ 
     &           &    5 &  5.72e-09 &  1.64e-16 &      \\ 
     &           &    6 &  7.20e-08 &  1.63e-16 &      \\ 
     &           &    7 &  8.35e-08 &  2.05e-15 &      \\ 
     &           &    8 &  7.20e-08 &  2.38e-15 &      \\ 
     &           &    9 &  5.80e-09 &  2.05e-15 &      \\ 
     &           &   10 &  5.80e-09 &  1.66e-16 &      \\ 
2201 &  2.20e+04 &   10 &           &           & iters  \\ 
 \hdashline 
     &           &    1 &  2.44e-08 &  1.71e-09 &      \\ 
     &           &    2 &  1.45e-08 &  6.96e-16 &      \\ 
     &           &    3 &  2.44e-08 &  4.14e-16 &      \\ 
     &           &    4 &  1.45e-08 &  6.95e-16 &      \\ 
     &           &    5 &  2.43e-08 &  4.14e-16 &      \\ 
     &           &    6 &  1.45e-08 &  6.95e-16 &      \\ 
     &           &    7 &  1.45e-08 &  4.15e-16 &      \\ 
     &           &    8 &  2.44e-08 &  4.14e-16 &      \\ 
     &           &    9 &  1.45e-08 &  6.95e-16 &      \\ 
     &           &   10 &  1.45e-08 &  4.15e-16 &      \\ 
2202 &  2.20e+04 &   10 &           &           & iters  \\ 
 \hdashline 
     &           &    1 &  1.33e-08 &  2.42e-09 &      \\ 
     &           &    2 &  1.32e-08 &  3.78e-16 &      \\ 
     &           &    3 &  2.56e-08 &  3.78e-16 &      \\ 
     &           &    4 &  1.33e-08 &  7.31e-16 &      \\ 
     &           &    5 &  1.32e-08 &  3.78e-16 &      \\ 
     &           &    6 &  2.56e-08 &  3.78e-16 &      \\ 
     &           &    7 &  1.33e-08 &  7.31e-16 &      \\ 
     &           &    8 &  1.32e-08 &  3.78e-16 &      \\ 
     &           &    9 &  2.56e-08 &  3.78e-16 &      \\ 
     &           &   10 &  1.33e-08 &  7.31e-16 &      \\ 
2203 &  2.20e+04 &   10 &           &           & iters  \\ 
 \hdashline 
     &           &    1 &  5.69e-08 &  3.43e-09 &      \\ 
     &           &    2 &  2.09e-08 &  1.62e-15 &      \\ 
     &           &    3 &  2.09e-08 &  5.96e-16 &      \\ 
     &           &    4 &  2.08e-08 &  5.95e-16 &      \\ 
     &           &    5 &  5.69e-08 &  5.95e-16 &      \\ 
     &           &    6 &  2.09e-08 &  1.62e-15 &      \\ 
     &           &    7 &  2.09e-08 &  5.96e-16 &      \\ 
     &           &    8 &  5.69e-08 &  5.95e-16 &      \\ 
     &           &    9 &  2.09e-08 &  1.62e-15 &      \\ 
     &           &   10 &  2.09e-08 &  5.97e-16 &      \\ 
2204 &  2.20e+04 &   10 &           &           & iters  \\ 
 \hdashline 
     &           &    1 &  3.74e-08 &  3.50e-09 &      \\ 
     &           &    2 &  4.04e-08 &  1.07e-15 &      \\ 
     &           &    3 &  3.74e-08 &  1.15e-15 &      \\ 
     &           &    4 &  4.04e-08 &  1.07e-15 &      \\ 
     &           &    5 &  3.74e-08 &  1.15e-15 &      \\ 
     &           &    6 &  4.04e-08 &  1.07e-15 &      \\ 
     &           &    7 &  3.74e-08 &  1.15e-15 &      \\ 
     &           &    8 &  4.04e-08 &  1.07e-15 &      \\ 
     &           &    9 &  3.74e-08 &  1.15e-15 &      \\ 
     &           &   10 &  4.04e-08 &  1.07e-15 &      \\ 
2205 &  2.20e+04 &   10 &           &           & iters  \\ 
 \hdashline 
     &           &    1 &  8.91e-09 &  1.74e-09 &      \\ 
     &           &    2 &  8.90e-09 &  2.54e-16 &      \\ 
     &           &    3 &  8.88e-09 &  2.54e-16 &      \\ 
     &           &    4 &  6.88e-08 &  2.54e-16 &      \\ 
     &           &    5 &  8.97e-09 &  1.96e-15 &      \\ 
     &           &    6 &  8.96e-09 &  2.56e-16 &      \\ 
     &           &    7 &  8.94e-09 &  2.56e-16 &      \\ 
     &           &    8 &  8.93e-09 &  2.55e-16 &      \\ 
     &           &    9 &  8.92e-09 &  2.55e-16 &      \\ 
     &           &   10 &  8.91e-09 &  2.55e-16 &      \\ 
2206 &  2.20e+04 &   10 &           &           & iters  \\ 
 \hdashline 
     &           &    1 &  3.58e-08 &  2.20e-09 &      \\ 
     &           &    2 &  4.19e-08 &  1.02e-15 &      \\ 
     &           &    3 &  3.58e-08 &  1.20e-15 &      \\ 
     &           &    4 &  4.19e-08 &  1.02e-15 &      \\ 
     &           &    5 &  3.58e-08 &  1.20e-15 &      \\ 
     &           &    6 &  4.19e-08 &  1.02e-15 &      \\ 
     &           &    7 &  3.58e-08 &  1.20e-15 &      \\ 
     &           &    8 &  4.19e-08 &  1.02e-15 &      \\ 
     &           &    9 &  3.58e-08 &  1.20e-15 &      \\ 
     &           &   10 &  4.19e-08 &  1.02e-15 &      \\ 
2207 &  2.21e+04 &   10 &           &           & iters  \\ 
 \hdashline 
     &           &    1 &  9.58e-09 &  1.83e-09 &      \\ 
     &           &    2 &  6.81e-08 &  2.73e-16 &      \\ 
     &           &    3 &  9.66e-09 &  1.94e-15 &      \\ 
     &           &    4 &  9.65e-09 &  2.76e-16 &      \\ 
     &           &    5 &  9.64e-09 &  2.75e-16 &      \\ 
     &           &    6 &  9.62e-09 &  2.75e-16 &      \\ 
     &           &    7 &  9.61e-09 &  2.75e-16 &      \\ 
     &           &    8 &  9.59e-09 &  2.74e-16 &      \\ 
     &           &    9 &  9.58e-09 &  2.74e-16 &      \\ 
     &           &   10 &  6.81e-08 &  2.73e-16 &      \\ 
2208 &  2.21e+04 &   10 &           &           & iters  \\ 
 \hdashline 
     &           &    1 &  2.11e-08 &  1.49e-09 &      \\ 
     &           &    2 &  5.66e-08 &  6.02e-16 &      \\ 
     &           &    3 &  2.11e-08 &  1.62e-15 &      \\ 
     &           &    4 &  2.11e-08 &  6.03e-16 &      \\ 
     &           &    5 &  2.11e-08 &  6.02e-16 &      \\ 
     &           &    6 &  5.66e-08 &  6.02e-16 &      \\ 
     &           &    7 &  2.11e-08 &  1.62e-15 &      \\ 
     &           &    8 &  2.11e-08 &  6.03e-16 &      \\ 
     &           &    9 &  5.66e-08 &  6.02e-16 &      \\ 
     &           &   10 &  2.11e-08 &  1.62e-15 &      \\ 
2209 &  2.21e+04 &   10 &           &           & iters  \\ 
 \hdashline 
     &           &    1 &  1.55e-08 &  2.56e-10 &      \\ 
     &           &    2 &  1.55e-08 &  4.42e-16 &      \\ 
     &           &    3 &  1.54e-08 &  4.41e-16 &      \\ 
     &           &    4 &  1.54e-08 &  4.40e-16 &      \\ 
     &           &    5 &  6.23e-08 &  4.40e-16 &      \\ 
     &           &    6 &  1.55e-08 &  1.78e-15 &      \\ 
     &           &    7 &  1.55e-08 &  4.42e-16 &      \\ 
     &           &    8 &  1.54e-08 &  4.41e-16 &      \\ 
     &           &    9 &  1.54e-08 &  4.40e-16 &      \\ 
     &           &   10 &  6.23e-08 &  4.40e-16 &      \\ 
2210 &  2.21e+04 &   10 &           &           & iters  \\ 
 \hdashline 
     &           &    1 &  1.76e-08 &  2.00e-09 &      \\ 
     &           &    2 &  6.01e-08 &  5.03e-16 &      \\ 
     &           &    3 &  1.77e-08 &  1.72e-15 &      \\ 
     &           &    4 &  1.76e-08 &  5.04e-16 &      \\ 
     &           &    5 &  1.76e-08 &  5.04e-16 &      \\ 
     &           &    6 &  6.01e-08 &  5.03e-16 &      \\ 
     &           &    7 &  1.77e-08 &  1.72e-15 &      \\ 
     &           &    8 &  1.77e-08 &  5.05e-16 &      \\ 
     &           &    9 &  1.76e-08 &  5.04e-16 &      \\ 
     &           &   10 &  1.76e-08 &  5.03e-16 &      \\ 
2211 &  2.21e+04 &   10 &           &           & iters  \\ 
 \hdashline 
     &           &    1 &  6.72e-09 &  1.39e-09 &      \\ 
     &           &    2 &  6.71e-09 &  1.92e-16 &      \\ 
     &           &    3 &  6.70e-09 &  1.91e-16 &      \\ 
     &           &    4 &  6.69e-09 &  1.91e-16 &      \\ 
     &           &    5 &  6.68e-09 &  1.91e-16 &      \\ 
     &           &    6 &  6.67e-09 &  1.91e-16 &      \\ 
     &           &    7 &  6.66e-09 &  1.90e-16 &      \\ 
     &           &    8 &  6.65e-09 &  1.90e-16 &      \\ 
     &           &    9 &  6.64e-09 &  1.90e-16 &      \\ 
     &           &   10 &  6.63e-09 &  1.89e-16 &      \\ 
2212 &  2.21e+04 &   10 &           &           & iters  \\ 
 \hdashline 
     &           &    1 &  3.73e-08 &  5.68e-10 &      \\ 
     &           &    2 &  4.04e-08 &  1.06e-15 &      \\ 
     &           &    3 &  3.73e-08 &  1.15e-15 &      \\ 
     &           &    4 &  4.04e-08 &  1.07e-15 &      \\ 
     &           &    5 &  3.73e-08 &  1.15e-15 &      \\ 
     &           &    6 &  4.04e-08 &  1.07e-15 &      \\ 
     &           &    7 &  3.73e-08 &  1.15e-15 &      \\ 
     &           &    8 &  4.04e-08 &  1.07e-15 &      \\ 
     &           &    9 &  3.73e-08 &  1.15e-15 &      \\ 
     &           &   10 &  4.04e-08 &  1.07e-15 &      \\ 
2213 &  2.21e+04 &   10 &           &           & iters  \\ 
 \hdashline 
     &           &    1 &  1.17e-08 &  7.45e-10 &      \\ 
     &           &    2 &  1.17e-08 &  3.34e-16 &      \\ 
     &           &    3 &  1.17e-08 &  3.33e-16 &      \\ 
     &           &    4 &  1.16e-08 &  3.33e-16 &      \\ 
     &           &    5 &  1.16e-08 &  3.32e-16 &      \\ 
     &           &    6 &  1.16e-08 &  3.32e-16 &      \\ 
     &           &    7 &  1.16e-08 &  3.31e-16 &      \\ 
     &           &    8 &  1.16e-08 &  3.31e-16 &      \\ 
     &           &    9 &  1.16e-08 &  3.30e-16 &      \\ 
     &           &   10 &  6.62e-08 &  3.30e-16 &      \\ 
2214 &  2.21e+04 &   10 &           &           & iters  \\ 
 \hdashline 
     &           &    1 &  5.08e-08 &  6.45e-10 &      \\ 
     &           &    2 &  2.69e-08 &  1.45e-15 &      \\ 
     &           &    3 &  2.69e-08 &  7.69e-16 &      \\ 
     &           &    4 &  5.08e-08 &  7.68e-16 &      \\ 
     &           &    5 &  2.69e-08 &  1.45e-15 &      \\ 
     &           &    6 &  2.69e-08 &  7.69e-16 &      \\ 
     &           &    7 &  5.08e-08 &  7.68e-16 &      \\ 
     &           &    8 &  2.69e-08 &  1.45e-15 &      \\ 
     &           &    9 &  2.69e-08 &  7.68e-16 &      \\ 
     &           &   10 &  5.08e-08 &  7.67e-16 &      \\ 
2215 &  2.21e+04 &   10 &           &           & iters  \\ 
 \hdashline 
     &           &    1 &  7.44e-08 &  5.82e-10 &      \\ 
     &           &    2 &  8.11e-08 &  2.12e-15 &      \\ 
     &           &    3 &  7.44e-08 &  2.31e-15 &      \\ 
     &           &    4 &  8.11e-08 &  2.12e-15 &      \\ 
     &           &    5 &  7.44e-08 &  2.31e-15 &      \\ 
     &           &    6 &  3.39e-09 &  2.12e-15 &      \\ 
     &           &    7 &  3.38e-09 &  9.67e-17 &      \\ 
     &           &    8 &  3.38e-09 &  9.65e-17 &      \\ 
     &           &    9 &  3.37e-09 &  9.64e-17 &      \\ 
     &           &   10 &  3.37e-09 &  9.62e-17 &      \\ 
2216 &  2.22e+04 &   10 &           &           & iters  \\ 
 \hdashline 
     &           &    1 &  1.26e-09 &  1.28e-09 &      \\ 
     &           &    2 &  1.26e-09 &  3.59e-17 &      \\ 
     &           &    3 &  1.25e-09 &  3.58e-17 &      \\ 
     &           &    4 &  1.25e-09 &  3.58e-17 &      \\ 
     &           &    5 &  1.25e-09 &  3.57e-17 &      \\ 
     &           &    6 &  1.25e-09 &  3.57e-17 &      \\ 
     &           &    7 &  1.25e-09 &  3.56e-17 &      \\ 
     &           &    8 &  1.25e-09 &  3.56e-17 &      \\ 
     &           &    9 &  1.24e-09 &  3.55e-17 &      \\ 
     &           &   10 &  1.24e-09 &  3.55e-17 &      \\ 
2217 &  2.22e+04 &   10 &           &           & iters  \\ 
 \hdashline 
     &           &    1 &  4.35e-08 &  5.72e-11 &      \\ 
     &           &    2 &  3.42e-08 &  1.24e-15 &      \\ 
     &           &    3 &  4.35e-08 &  9.77e-16 &      \\ 
     &           &    4 &  3.42e-08 &  1.24e-15 &      \\ 
     &           &    5 &  4.35e-08 &  9.77e-16 &      \\ 
     &           &    6 &  3.43e-08 &  1.24e-15 &      \\ 
     &           &    7 &  4.35e-08 &  9.78e-16 &      \\ 
     &           &    8 &  3.43e-08 &  1.24e-15 &      \\ 
     &           &    9 &  3.42e-08 &  9.78e-16 &      \\ 
     &           &   10 &  4.35e-08 &  9.77e-16 &      \\ 
2218 &  2.22e+04 &   10 &           &           & iters  \\ 
 \hdashline 
     &           &    1 &  2.75e-08 &  1.24e-09 &      \\ 
     &           &    2 &  5.02e-08 &  7.84e-16 &      \\ 
     &           &    3 &  2.75e-08 &  1.43e-15 &      \\ 
     &           &    4 &  2.75e-08 &  7.85e-16 &      \\ 
     &           &    5 &  5.03e-08 &  7.84e-16 &      \\ 
     &           &    6 &  2.75e-08 &  1.43e-15 &      \\ 
     &           &    7 &  2.75e-08 &  7.85e-16 &      \\ 
     &           &    8 &  5.03e-08 &  7.84e-16 &      \\ 
     &           &    9 &  2.75e-08 &  1.43e-15 &      \\ 
     &           &   10 &  2.75e-08 &  7.85e-16 &      \\ 
2219 &  2.22e+04 &   10 &           &           & iters  \\ 
 \hdashline 
     &           &    1 &  4.52e-08 &  6.99e-10 &      \\ 
     &           &    2 &  4.52e-08 &  1.29e-15 &      \\ 
     &           &    3 &  3.26e-08 &  1.29e-15 &      \\ 
     &           &    4 &  3.25e-08 &  9.30e-16 &      \\ 
     &           &    5 &  4.52e-08 &  9.29e-16 &      \\ 
     &           &    6 &  3.26e-08 &  1.29e-15 &      \\ 
     &           &    7 &  4.52e-08 &  9.29e-16 &      \\ 
     &           &    8 &  3.26e-08 &  1.29e-15 &      \\ 
     &           &    9 &  3.25e-08 &  9.30e-16 &      \\ 
     &           &   10 &  4.52e-08 &  9.28e-16 &      \\ 
2220 &  2.22e+04 &   10 &           &           & iters  \\ 
 \hdashline 
     &           &    1 &  1.70e-08 &  2.13e-10 &      \\ 
     &           &    2 &  2.19e-08 &  4.85e-16 &      \\ 
     &           &    3 &  1.70e-08 &  6.24e-16 &      \\ 
     &           &    4 &  2.19e-08 &  4.85e-16 &      \\ 
     &           &    5 &  1.70e-08 &  6.24e-16 &      \\ 
     &           &    6 &  2.19e-08 &  4.86e-16 &      \\ 
     &           &    7 &  1.70e-08 &  6.24e-16 &      \\ 
     &           &    8 &  2.18e-08 &  4.86e-16 &      \\ 
     &           &    9 &  1.70e-08 &  6.24e-16 &      \\ 
     &           &   10 &  1.70e-08 &  4.86e-16 &      \\ 
2221 &  2.22e+04 &   10 &           &           & iters  \\ 
 \hdashline 
     &           &    1 &  2.41e-09 &  3.94e-10 &      \\ 
     &           &    2 &  2.40e-09 &  6.87e-17 &      \\ 
     &           &    3 &  2.40e-09 &  6.86e-17 &      \\ 
     &           &    4 &  2.40e-09 &  6.85e-17 &      \\ 
     &           &    5 &  2.39e-09 &  6.84e-17 &      \\ 
     &           &    6 &  2.39e-09 &  6.83e-17 &      \\ 
     &           &    7 &  2.39e-09 &  6.82e-17 &      \\ 
     &           &    8 &  2.38e-09 &  6.81e-17 &      \\ 
     &           &    9 &  2.38e-09 &  6.80e-17 &      \\ 
     &           &   10 &  2.38e-09 &  6.79e-17 &      \\ 
2222 &  2.22e+04 &   10 &           &           & iters  \\ 
 \hdashline 
     &           &    1 &  1.51e-08 &  5.29e-11 &      \\ 
     &           &    2 &  6.26e-08 &  4.30e-16 &      \\ 
     &           &    3 &  1.51e-08 &  1.79e-15 &      \\ 
     &           &    4 &  1.51e-08 &  4.32e-16 &      \\ 
     &           &    5 &  1.51e-08 &  4.32e-16 &      \\ 
     &           &    6 &  1.51e-08 &  4.31e-16 &      \\ 
     &           &    7 &  6.26e-08 &  4.30e-16 &      \\ 
     &           &    8 &  9.28e-08 &  1.79e-15 &      \\ 
     &           &    9 &  6.27e-08 &  2.65e-15 &      \\ 
     &           &   10 &  1.51e-08 &  1.79e-15 &      \\ 
2223 &  2.22e+04 &   10 &           &           & iters  \\ 
 \hdashline 
     &           &    1 &  4.84e-09 &  5.91e-10 &      \\ 
     &           &    2 &  4.84e-09 &  1.38e-16 &      \\ 
     &           &    3 &  4.83e-09 &  1.38e-16 &      \\ 
     &           &    4 &  4.82e-09 &  1.38e-16 &      \\ 
     &           &    5 &  4.81e-09 &  1.38e-16 &      \\ 
     &           &    6 &  4.81e-09 &  1.37e-16 &      \\ 
     &           &    7 &  4.80e-09 &  1.37e-16 &      \\ 
     &           &    8 &  4.79e-09 &  1.37e-16 &      \\ 
     &           &    9 &  3.41e-08 &  1.37e-16 &      \\ 
     &           &   10 &  4.84e-09 &  9.72e-16 &      \\ 
2224 &  2.22e+04 &   10 &           &           & iters  \\ 
 \hdashline 
     &           &    1 &  3.14e-08 &  6.23e-10 &      \\ 
     &           &    2 &  4.64e-08 &  8.95e-16 &      \\ 
     &           &    3 &  3.14e-08 &  1.32e-15 &      \\ 
     &           &    4 &  4.64e-08 &  8.95e-16 &      \\ 
     &           &    5 &  3.14e-08 &  1.32e-15 &      \\ 
     &           &    6 &  3.13e-08 &  8.96e-16 &      \\ 
     &           &    7 &  4.64e-08 &  8.95e-16 &      \\ 
     &           &    8 &  3.14e-08 &  1.32e-15 &      \\ 
     &           &    9 &  4.64e-08 &  8.95e-16 &      \\ 
     &           &   10 &  3.14e-08 &  1.32e-15 &      \\ 
2225 &  2.22e+04 &   10 &           &           & iters  \\ 
 \hdashline 
     &           &    1 &  5.36e-08 &  1.27e-09 &      \\ 
     &           &    2 &  2.41e-08 &  1.53e-15 &      \\ 
     &           &    3 &  2.41e-08 &  6.89e-16 &      \\ 
     &           &    4 &  5.36e-08 &  6.88e-16 &      \\ 
     &           &    5 &  2.41e-08 &  1.53e-15 &      \\ 
     &           &    6 &  2.41e-08 &  6.89e-16 &      \\ 
     &           &    7 &  5.36e-08 &  6.88e-16 &      \\ 
     &           &    8 &  2.41e-08 &  1.53e-15 &      \\ 
     &           &    9 &  2.41e-08 &  6.89e-16 &      \\ 
     &           &   10 &  5.36e-08 &  6.88e-16 &      \\ 
2226 &  2.22e+04 &   10 &           &           & iters  \\ 
 \hdashline 
     &           &    1 &  7.45e-08 &  2.45e-09 &      \\ 
     &           &    2 &  3.31e-09 &  2.13e-15 &      \\ 
     &           &    3 &  3.30e-09 &  9.44e-17 &      \\ 
     &           &    4 &  3.30e-09 &  9.43e-17 &      \\ 
     &           &    5 &  3.29e-09 &  9.41e-17 &      \\ 
     &           &    6 &  3.29e-09 &  9.40e-17 &      \\ 
     &           &    7 &  3.28e-09 &  9.39e-17 &      \\ 
     &           &    8 &  3.28e-09 &  9.37e-17 &      \\ 
     &           &    9 &  3.28e-09 &  9.36e-17 &      \\ 
     &           &   10 &  3.27e-09 &  9.35e-17 &      \\ 
2227 &  2.23e+04 &   10 &           &           & iters  \\ 
 \hdashline 
     &           &    1 &  5.41e-08 &  9.84e-10 &      \\ 
     &           &    2 &  2.37e-08 &  1.54e-15 &      \\ 
     &           &    3 &  2.37e-08 &  6.76e-16 &      \\ 
     &           &    4 &  5.41e-08 &  6.75e-16 &      \\ 
     &           &    5 &  2.37e-08 &  1.54e-15 &      \\ 
     &           &    6 &  2.37e-08 &  6.76e-16 &      \\ 
     &           &    7 &  2.36e-08 &  6.75e-16 &      \\ 
     &           &    8 &  5.41e-08 &  6.74e-16 &      \\ 
     &           &    9 &  2.37e-08 &  1.54e-15 &      \\ 
     &           &   10 &  2.36e-08 &  6.76e-16 &      \\ 
2228 &  2.23e+04 &   10 &           &           & iters  \\ 
 \hdashline 
     &           &    1 &  1.95e-08 &  2.76e-10 &      \\ 
     &           &    2 &  1.95e-08 &  5.57e-16 &      \\ 
     &           &    3 &  5.82e-08 &  5.57e-16 &      \\ 
     &           &    4 &  1.96e-08 &  1.66e-15 &      \\ 
     &           &    5 &  1.95e-08 &  5.58e-16 &      \\ 
     &           &    6 &  1.95e-08 &  5.57e-16 &      \\ 
     &           &    7 &  5.82e-08 &  5.57e-16 &      \\ 
     &           &    8 &  1.96e-08 &  1.66e-15 &      \\ 
     &           &    9 &  1.95e-08 &  5.58e-16 &      \\ 
     &           &   10 &  1.95e-08 &  5.57e-16 &      \\ 
2229 &  2.23e+04 &   10 &           &           & iters  \\ 
 \hdashline 
     &           &    1 &  6.09e-08 &  5.14e-10 &      \\ 
     &           &    2 &  1.69e-08 &  1.74e-15 &      \\ 
     &           &    3 &  1.69e-08 &  4.82e-16 &      \\ 
     &           &    4 &  1.68e-08 &  4.81e-16 &      \\ 
     &           &    5 &  6.09e-08 &  4.80e-16 &      \\ 
     &           &    6 &  1.69e-08 &  1.74e-15 &      \\ 
     &           &    7 &  1.69e-08 &  4.82e-16 &      \\ 
     &           &    8 &  1.68e-08 &  4.82e-16 &      \\ 
     &           &    9 &  1.68e-08 &  4.81e-16 &      \\ 
     &           &   10 &  6.09e-08 &  4.80e-16 &      \\ 
2230 &  2.23e+04 &   10 &           &           & iters  \\ 
 \hdashline 
     &           &    1 &  6.93e-09 &  1.63e-10 &      \\ 
     &           &    2 &  6.92e-09 &  1.98e-16 &      \\ 
     &           &    3 &  6.91e-09 &  1.98e-16 &      \\ 
     &           &    4 &  6.90e-09 &  1.97e-16 &      \\ 
     &           &    5 &  6.89e-09 &  1.97e-16 &      \\ 
     &           &    6 &  6.88e-09 &  1.97e-16 &      \\ 
     &           &    7 &  6.87e-09 &  1.96e-16 &      \\ 
     &           &    8 &  7.08e-08 &  1.96e-16 &      \\ 
     &           &    9 &  8.47e-08 &  2.02e-15 &      \\ 
     &           &   10 &  7.08e-08 &  2.42e-15 &      \\ 
2231 &  2.23e+04 &   10 &           &           & iters  \\ 
 \hdashline 
     &           &    1 &  5.32e-09 &  1.87e-10 &      \\ 
     &           &    2 &  5.32e-09 &  1.52e-16 &      \\ 
     &           &    3 &  7.24e-08 &  1.52e-16 &      \\ 
     &           &    4 &  8.31e-08 &  2.07e-15 &      \\ 
     &           &    5 &  7.24e-08 &  2.37e-15 &      \\ 
     &           &    6 &  5.40e-09 &  2.07e-15 &      \\ 
     &           &    7 &  5.39e-09 &  1.54e-16 &      \\ 
     &           &    8 &  5.38e-09 &  1.54e-16 &      \\ 
     &           &    9 &  5.37e-09 &  1.54e-16 &      \\ 
     &           &   10 &  5.37e-09 &  1.53e-16 &      \\ 
2232 &  2.23e+04 &   10 &           &           & iters  \\ 
 \hdashline 
     &           &    1 &  1.42e-08 &  5.25e-10 &      \\ 
     &           &    2 &  1.42e-08 &  4.06e-16 &      \\ 
     &           &    3 &  1.42e-08 &  4.05e-16 &      \\ 
     &           &    4 &  6.35e-08 &  4.05e-16 &      \\ 
     &           &    5 &  9.19e-08 &  1.81e-15 &      \\ 
     &           &    6 &  6.36e-08 &  2.62e-15 &      \\ 
     &           &    7 &  1.42e-08 &  1.81e-15 &      \\ 
     &           &    8 &  1.42e-08 &  4.05e-16 &      \\ 
     &           &    9 &  1.42e-08 &  4.05e-16 &      \\ 
     &           &   10 &  6.35e-08 &  4.04e-16 &      \\ 
2233 &  2.23e+04 &   10 &           &           & iters  \\ 
 \hdashline 
     &           &    1 &  1.99e-08 &  6.38e-10 &      \\ 
     &           &    2 &  5.78e-08 &  5.69e-16 &      \\ 
     &           &    3 &  2.00e-08 &  1.65e-15 &      \\ 
     &           &    4 &  2.00e-08 &  5.70e-16 &      \\ 
     &           &    5 &  1.99e-08 &  5.69e-16 &      \\ 
     &           &    6 &  5.78e-08 &  5.69e-16 &      \\ 
     &           &    7 &  2.00e-08 &  1.65e-15 &      \\ 
     &           &    8 &  2.00e-08 &  5.70e-16 &      \\ 
     &           &    9 &  1.99e-08 &  5.69e-16 &      \\ 
     &           &   10 &  5.78e-08 &  5.69e-16 &      \\ 
2234 &  2.23e+04 &   10 &           &           & iters  \\ 
 \hdashline 
     &           &    1 &  1.18e-08 &  8.08e-10 &      \\ 
     &           &    2 &  1.18e-08 &  3.37e-16 &      \\ 
     &           &    3 &  1.18e-08 &  3.37e-16 &      \\ 
     &           &    4 &  1.18e-08 &  3.36e-16 &      \\ 
     &           &    5 &  1.17e-08 &  3.36e-16 &      \\ 
     &           &    6 &  1.17e-08 &  3.35e-16 &      \\ 
     &           &    7 &  6.60e-08 &  3.35e-16 &      \\ 
     &           &    8 &  1.18e-08 &  1.88e-15 &      \\ 
     &           &    9 &  1.18e-08 &  3.37e-16 &      \\ 
     &           &   10 &  1.18e-08 &  3.36e-16 &      \\ 
2235 &  2.23e+04 &   10 &           &           & iters  \\ 
 \hdashline 
     &           &    1 &  8.86e-08 &  7.37e-10 &      \\ 
     &           &    2 &  6.69e-08 &  2.53e-15 &      \\ 
     &           &    3 &  1.09e-08 &  1.91e-15 &      \\ 
     &           &    4 &  1.09e-08 &  3.11e-16 &      \\ 
     &           &    5 &  1.09e-08 &  3.11e-16 &      \\ 
     &           &    6 &  1.09e-08 &  3.10e-16 &      \\ 
     &           &    7 &  6.68e-08 &  3.10e-16 &      \\ 
     &           &    8 &  8.86e-08 &  1.91e-15 &      \\ 
     &           &    9 &  6.69e-08 &  2.53e-15 &      \\ 
     &           &   10 &  1.09e-08 &  1.91e-15 &      \\ 
2236 &  2.24e+04 &   10 &           &           & iters  \\ 
 \hdashline 
     &           &    1 &  2.81e-09 &  1.73e-09 &      \\ 
     &           &    2 &  2.81e-09 &  8.03e-17 &      \\ 
     &           &    3 &  2.81e-09 &  8.02e-17 &      \\ 
     &           &    4 &  2.80e-09 &  8.01e-17 &      \\ 
     &           &    5 &  2.80e-09 &  8.00e-17 &      \\ 
     &           &    6 &  2.79e-09 &  7.99e-17 &      \\ 
     &           &    7 &  2.79e-09 &  7.97e-17 &      \\ 
     &           &    8 &  2.79e-09 &  7.96e-17 &      \\ 
     &           &    9 &  2.78e-09 &  7.95e-17 &      \\ 
     &           &   10 &  2.78e-09 &  7.94e-17 &      \\ 
2237 &  2.24e+04 &   10 &           &           & iters  \\ 
 \hdashline 
     &           &    1 &  6.09e-08 &  2.30e-09 &      \\ 
     &           &    2 &  1.69e-08 &  1.74e-15 &      \\ 
     &           &    3 &  1.68e-08 &  4.81e-16 &      \\ 
     &           &    4 &  1.68e-08 &  4.81e-16 &      \\ 
     &           &    5 &  1.68e-08 &  4.80e-16 &      \\ 
     &           &    6 &  6.09e-08 &  4.79e-16 &      \\ 
     &           &    7 &  1.69e-08 &  1.74e-15 &      \\ 
     &           &    8 &  1.68e-08 &  4.81e-16 &      \\ 
     &           &    9 &  1.68e-08 &  4.80e-16 &      \\ 
     &           &   10 &  1.68e-08 &  4.80e-16 &      \\ 
2238 &  2.24e+04 &   10 &           &           & iters  \\ 
 \hdashline 
     &           &    1 &  5.04e-08 &  1.68e-09 &      \\ 
     &           &    2 &  2.73e-08 &  1.44e-15 &      \\ 
     &           &    3 &  5.04e-08 &  7.80e-16 &      \\ 
     &           &    4 &  2.74e-08 &  1.44e-15 &      \\ 
     &           &    5 &  2.73e-08 &  7.81e-16 &      \\ 
     &           &    6 &  5.04e-08 &  7.80e-16 &      \\ 
     &           &    7 &  2.73e-08 &  1.44e-15 &      \\ 
     &           &    8 &  2.73e-08 &  7.80e-16 &      \\ 
     &           &    9 &  5.04e-08 &  7.79e-16 &      \\ 
     &           &   10 &  2.73e-08 &  1.44e-15 &      \\ 
2239 &  2.24e+04 &   10 &           &           & iters  \\ 
 \hdashline 
     &           &    1 &  1.45e-08 &  2.87e-10 &      \\ 
     &           &    2 &  1.44e-08 &  4.13e-16 &      \\ 
     &           &    3 &  6.33e-08 &  4.12e-16 &      \\ 
     &           &    4 &  1.45e-08 &  1.81e-15 &      \\ 
     &           &    5 &  1.45e-08 &  4.14e-16 &      \\ 
     &           &    6 &  1.45e-08 &  4.13e-16 &      \\ 
     &           &    7 &  1.44e-08 &  4.13e-16 &      \\ 
     &           &    8 &  1.44e-08 &  4.12e-16 &      \\ 
     &           &    9 &  6.33e-08 &  4.12e-16 &      \\ 
     &           &   10 &  1.45e-08 &  1.81e-15 &      \\ 
2240 &  2.24e+04 &   10 &           &           & iters  \\ 
 \hdashline 
     &           &    1 &  4.89e-08 &  4.66e-10 &      \\ 
     &           &    2 &  2.89e-08 &  1.40e-15 &      \\ 
     &           &    3 &  2.88e-08 &  8.24e-16 &      \\ 
     &           &    4 &  4.89e-08 &  8.23e-16 &      \\ 
     &           &    5 &  2.89e-08 &  1.40e-15 &      \\ 
     &           &    6 &  2.88e-08 &  8.24e-16 &      \\ 
     &           &    7 &  4.89e-08 &  8.22e-16 &      \\ 
     &           &    8 &  2.88e-08 &  1.40e-15 &      \\ 
     &           &    9 &  4.89e-08 &  8.23e-16 &      \\ 
     &           &   10 &  2.89e-08 &  1.40e-15 &      \\ 
2241 &  2.24e+04 &   10 &           &           & iters  \\ 
 \hdashline 
     &           &    1 &  6.71e-09 &  8.83e-10 &      \\ 
     &           &    2 &  6.70e-09 &  1.91e-16 &      \\ 
     &           &    3 &  6.69e-09 &  1.91e-16 &      \\ 
     &           &    4 &  6.68e-09 &  1.91e-16 &      \\ 
     &           &    5 &  6.67e-09 &  1.91e-16 &      \\ 
     &           &    6 &  6.66e-09 &  1.90e-16 &      \\ 
     &           &    7 &  6.65e-09 &  1.90e-16 &      \\ 
     &           &    8 &  6.64e-09 &  1.90e-16 &      \\ 
     &           &    9 &  7.11e-08 &  1.89e-16 &      \\ 
     &           &   10 &  4.56e-08 &  2.03e-15 &      \\ 
2242 &  2.24e+04 &   10 &           &           & iters  \\ 
 \hdashline 
     &           &    1 &  1.56e-08 &  1.64e-09 &      \\ 
     &           &    2 &  1.55e-08 &  4.44e-16 &      \\ 
     &           &    3 &  2.33e-08 &  4.44e-16 &      \\ 
     &           &    4 &  1.56e-08 &  6.66e-16 &      \\ 
     &           &    5 &  2.33e-08 &  4.44e-16 &      \\ 
     &           &    6 &  1.56e-08 &  6.65e-16 &      \\ 
     &           &    7 &  1.55e-08 &  4.44e-16 &      \\ 
     &           &    8 &  2.33e-08 &  4.44e-16 &      \\ 
     &           &    9 &  1.56e-08 &  6.66e-16 &      \\ 
     &           &   10 &  2.33e-08 &  4.44e-16 &      \\ 
2243 &  2.24e+04 &   10 &           &           & iters  \\ 
 \hdashline 
     &           &    1 &  3.59e-08 &  2.00e-09 &      \\ 
     &           &    2 &  4.19e-08 &  1.02e-15 &      \\ 
     &           &    3 &  3.59e-08 &  1.19e-15 &      \\ 
     &           &    4 &  4.19e-08 &  1.02e-15 &      \\ 
     &           &    5 &  3.59e-08 &  1.19e-15 &      \\ 
     &           &    6 &  4.18e-08 &  1.02e-15 &      \\ 
     &           &    7 &  3.59e-08 &  1.19e-15 &      \\ 
     &           &    8 &  4.18e-08 &  1.02e-15 &      \\ 
     &           &    9 &  3.59e-08 &  1.19e-15 &      \\ 
     &           &   10 &  3.59e-08 &  1.02e-15 &      \\ 
2244 &  2.24e+04 &   10 &           &           & iters  \\ 
 \hdashline 
     &           &    1 &  1.17e-07 &  1.37e-09 &      \\ 
     &           &    2 &  3.83e-08 &  3.35e-15 &      \\ 
     &           &    3 &  3.82e-08 &  1.09e-15 &      \\ 
     &           &    4 &  3.95e-08 &  1.09e-15 &      \\ 
     &           &    5 &  3.82e-08 &  1.13e-15 &      \\ 
     &           &    6 &  3.95e-08 &  1.09e-15 &      \\ 
     &           &    7 &  3.82e-08 &  1.13e-15 &      \\ 
     &           &    8 &  3.95e-08 &  1.09e-15 &      \\ 
     &           &    9 &  3.82e-08 &  1.13e-15 &      \\ 
     &           &   10 &  3.95e-08 &  1.09e-15 &      \\ 
2245 &  2.24e+04 &   10 &           &           & iters  \\ 
 \hdashline 
     &           &    1 &  8.36e-09 &  9.46e-10 &      \\ 
     &           &    2 &  8.35e-09 &  2.39e-16 &      \\ 
     &           &    3 &  8.34e-09 &  2.38e-16 &      \\ 
     &           &    4 &  8.33e-09 &  2.38e-16 &      \\ 
     &           &    5 &  8.31e-09 &  2.38e-16 &      \\ 
     &           &    6 &  8.30e-09 &  2.37e-16 &      \\ 
     &           &    7 &  6.94e-08 &  2.37e-16 &      \\ 
     &           &    8 &  8.61e-08 &  1.98e-15 &      \\ 
     &           &    9 &  6.94e-08 &  2.46e-15 &      \\ 
     &           &   10 &  8.37e-09 &  1.98e-15 &      \\ 
2246 &  2.24e+04 &   10 &           &           & iters  \\ 
 \hdashline 
     &           &    1 &  6.07e-08 &  2.37e-09 &      \\ 
     &           &    2 &  1.71e-08 &  1.73e-15 &      \\ 
     &           &    3 &  1.71e-08 &  4.89e-16 &      \\ 
     &           &    4 &  1.71e-08 &  4.88e-16 &      \\ 
     &           &    5 &  1.70e-08 &  4.87e-16 &      \\ 
     &           &    6 &  6.07e-08 &  4.86e-16 &      \\ 
     &           &    7 &  1.71e-08 &  1.73e-15 &      \\ 
     &           &    8 &  1.71e-08 &  4.88e-16 &      \\ 
     &           &    9 &  1.71e-08 &  4.88e-16 &      \\ 
     &           &   10 &  6.07e-08 &  4.87e-16 &      \\ 
2247 &  2.25e+04 &   10 &           &           & iters  \\ 
 \hdashline 
     &           &    1 &  3.08e-08 &  3.47e-10 &      \\ 
     &           &    2 &  3.08e-08 &  8.79e-16 &      \\ 
     &           &    3 &  4.70e-08 &  8.78e-16 &      \\ 
     &           &    4 &  3.08e-08 &  1.34e-15 &      \\ 
     &           &    5 &  4.70e-08 &  8.78e-16 &      \\ 
     &           &    6 &  3.08e-08 &  1.34e-15 &      \\ 
     &           &    7 &  3.08e-08 &  8.79e-16 &      \\ 
     &           &    8 &  4.70e-08 &  8.78e-16 &      \\ 
     &           &    9 &  3.08e-08 &  1.34e-15 &      \\ 
     &           &   10 &  4.70e-08 &  8.78e-16 &      \\ 
2248 &  2.25e+04 &   10 &           &           & iters  \\ 
 \hdashline 
     &           &    1 &  5.44e-08 &  1.25e-09 &      \\ 
     &           &    2 &  2.34e-08 &  1.55e-15 &      \\ 
     &           &    3 &  2.34e-08 &  6.68e-16 &      \\ 
     &           &    4 &  5.43e-08 &  6.67e-16 &      \\ 
     &           &    5 &  2.34e-08 &  1.55e-15 &      \\ 
     &           &    6 &  2.34e-08 &  6.68e-16 &      \\ 
     &           &    7 &  5.43e-08 &  6.67e-16 &      \\ 
     &           &    8 &  2.34e-08 &  1.55e-15 &      \\ 
     &           &    9 &  2.34e-08 &  6.69e-16 &      \\ 
     &           &   10 &  2.34e-08 &  6.68e-16 &      \\ 
2249 &  2.25e+04 &   10 &           &           & iters  \\ 
 \hdashline 
     &           &    1 &  2.00e-08 &  1.36e-10 &      \\ 
     &           &    2 &  2.00e-08 &  5.71e-16 &      \\ 
     &           &    3 &  2.00e-08 &  5.70e-16 &      \\ 
     &           &    4 &  5.78e-08 &  5.69e-16 &      \\ 
     &           &    5 &  2.00e-08 &  1.65e-15 &      \\ 
     &           &    6 &  2.00e-08 &  5.71e-16 &      \\ 
     &           &    7 &  1.99e-08 &  5.70e-16 &      \\ 
     &           &    8 &  5.78e-08 &  5.69e-16 &      \\ 
     &           &    9 &  2.00e-08 &  1.65e-15 &      \\ 
     &           &   10 &  2.00e-08 &  5.71e-16 &      \\ 
2250 &  2.25e+04 &   10 &           &           & iters  \\ 
 \hdashline 
     &           &    1 &  3.79e-08 &  7.21e-10 &      \\ 
     &           &    2 &  3.98e-08 &  1.08e-15 &      \\ 
     &           &    3 &  3.79e-08 &  1.14e-15 &      \\ 
     &           &    4 &  3.98e-08 &  1.08e-15 &      \\ 
     &           &    5 &  3.79e-08 &  1.14e-15 &      \\ 
     &           &    6 &  3.98e-08 &  1.08e-15 &      \\ 
     &           &    7 &  3.79e-08 &  1.14e-15 &      \\ 
     &           &    8 &  3.98e-08 &  1.08e-15 &      \\ 
     &           &    9 &  3.79e-08 &  1.14e-15 &      \\ 
     &           &   10 &  3.98e-08 &  1.08e-15 &      \\ 
2251 &  2.25e+04 &   10 &           &           & iters  \\ 
 \hdashline 
     &           &    1 &  4.13e-08 &  2.13e-10 &      \\ 
     &           &    2 &  3.64e-08 &  1.18e-15 &      \\ 
     &           &    3 &  3.64e-08 &  1.04e-15 &      \\ 
     &           &    4 &  4.14e-08 &  1.04e-15 &      \\ 
     &           &    5 &  3.64e-08 &  1.18e-15 &      \\ 
     &           &    6 &  4.14e-08 &  1.04e-15 &      \\ 
     &           &    7 &  3.64e-08 &  1.18e-15 &      \\ 
     &           &    8 &  4.14e-08 &  1.04e-15 &      \\ 
     &           &    9 &  3.64e-08 &  1.18e-15 &      \\ 
     &           &   10 &  4.14e-08 &  1.04e-15 &      \\ 
2252 &  2.25e+04 &   10 &           &           & iters  \\ 
 \hdashline 
     &           &    1 &  1.20e-08 &  1.53e-10 &      \\ 
     &           &    2 &  1.20e-08 &  3.43e-16 &      \\ 
     &           &    3 &  6.57e-08 &  3.42e-16 &      \\ 
     &           &    4 &  1.21e-08 &  1.88e-15 &      \\ 
     &           &    5 &  1.20e-08 &  3.44e-16 &      \\ 
     &           &    6 &  1.20e-08 &  3.44e-16 &      \\ 
     &           &    7 &  1.20e-08 &  3.43e-16 &      \\ 
     &           &    8 &  1.20e-08 &  3.43e-16 &      \\ 
     &           &    9 &  6.57e-08 &  3.42e-16 &      \\ 
     &           &   10 &  5.09e-08 &  1.88e-15 &      \\ 
2253 &  2.25e+04 &   10 &           &           & iters  \\ 
 \hdashline 
     &           &    1 &  6.82e-09 &  5.73e-11 &      \\ 
     &           &    2 &  6.81e-09 &  1.95e-16 &      \\ 
     &           &    3 &  6.80e-09 &  1.94e-16 &      \\ 
     &           &    4 &  6.79e-09 &  1.94e-16 &      \\ 
     &           &    5 &  6.78e-09 &  1.94e-16 &      \\ 
     &           &    6 &  6.77e-09 &  1.94e-16 &      \\ 
     &           &    7 &  7.09e-08 &  1.93e-16 &      \\ 
     &           &    8 &  6.86e-09 &  2.02e-15 &      \\ 
     &           &    9 &  6.85e-09 &  1.96e-16 &      \\ 
     &           &   10 &  6.84e-09 &  1.96e-16 &      \\ 
2254 &  2.25e+04 &   10 &           &           & iters  \\ 
 \hdashline 
     &           &    1 &  3.02e-09 &  1.90e-10 &      \\ 
     &           &    2 &  3.01e-09 &  8.61e-17 &      \\ 
     &           &    3 &  3.01e-09 &  8.60e-17 &      \\ 
     &           &    4 &  3.58e-08 &  8.59e-17 &      \\ 
     &           &    5 &  3.06e-09 &  1.02e-15 &      \\ 
     &           &    6 &  3.05e-09 &  8.72e-17 &      \\ 
     &           &    7 &  3.05e-09 &  8.71e-17 &      \\ 
     &           &    8 &  3.04e-09 &  8.70e-17 &      \\ 
     &           &    9 &  3.04e-09 &  8.69e-17 &      \\ 
     &           &   10 &  3.03e-09 &  8.67e-17 &      \\ 
2255 &  2.25e+04 &   10 &           &           & iters  \\ 
 \hdashline 
     &           &    1 &  2.28e-08 &  7.13e-10 &      \\ 
     &           &    2 &  1.61e-08 &  6.51e-16 &      \\ 
     &           &    3 &  1.60e-08 &  4.58e-16 &      \\ 
     &           &    4 &  2.28e-08 &  4.58e-16 &      \\ 
     &           &    5 &  1.60e-08 &  6.52e-16 &      \\ 
     &           &    6 &  2.28e-08 &  4.58e-16 &      \\ 
     &           &    7 &  1.61e-08 &  6.51e-16 &      \\ 
     &           &    8 &  1.60e-08 &  4.58e-16 &      \\ 
     &           &    9 &  2.28e-08 &  4.58e-16 &      \\ 
     &           &   10 &  1.60e-08 &  6.52e-16 &      \\ 
2256 &  2.26e+04 &   10 &           &           & iters  \\ 
 \hdashline 
     &           &    1 &  1.81e-08 &  8.71e-10 &      \\ 
     &           &    2 &  2.07e-08 &  5.18e-16 &      \\ 
     &           &    3 &  1.81e-08 &  5.92e-16 &      \\ 
     &           &    4 &  2.07e-08 &  5.18e-16 &      \\ 
     &           &    5 &  1.81e-08 &  5.92e-16 &      \\ 
     &           &    6 &  2.07e-08 &  5.18e-16 &      \\ 
     &           &    7 &  1.82e-08 &  5.91e-16 &      \\ 
     &           &    8 &  2.07e-08 &  5.18e-16 &      \\ 
     &           &    9 &  1.82e-08 &  5.91e-16 &      \\ 
     &           &   10 &  2.07e-08 &  5.18e-16 &      \\ 
2257 &  2.26e+04 &   10 &           &           & iters  \\ 
 \hdashline 
     &           &    1 &  1.51e-08 &  8.82e-11 &      \\ 
     &           &    2 &  6.26e-08 &  4.30e-16 &      \\ 
     &           &    3 &  1.51e-08 &  1.79e-15 &      \\ 
     &           &    4 &  1.51e-08 &  4.32e-16 &      \\ 
     &           &    5 &  1.51e-08 &  4.31e-16 &      \\ 
     &           &    6 &  1.51e-08 &  4.31e-16 &      \\ 
     &           &    7 &  6.26e-08 &  4.30e-16 &      \\ 
     &           &    8 &  1.51e-08 &  1.79e-15 &      \\ 
     &           &    9 &  1.51e-08 &  4.32e-16 &      \\ 
     &           &   10 &  1.51e-08 &  4.31e-16 &      \\ 
2258 &  2.26e+04 &   10 &           &           & iters  \\ 
 \hdashline 
     &           &    1 &  3.87e-08 &  9.39e-10 &      \\ 
     &           &    2 &  3.90e-08 &  1.11e-15 &      \\ 
     &           &    3 &  3.87e-08 &  1.11e-15 &      \\ 
     &           &    4 &  3.90e-08 &  1.11e-15 &      \\ 
     &           &    5 &  3.87e-08 &  1.11e-15 &      \\ 
     &           &    6 &  3.90e-08 &  1.11e-15 &      \\ 
     &           &    7 &  3.87e-08 &  1.11e-15 &      \\ 
     &           &    8 &  3.90e-08 &  1.11e-15 &      \\ 
     &           &    9 &  3.87e-08 &  1.11e-15 &      \\ 
     &           &   10 &  3.90e-08 &  1.11e-15 &      \\ 
2259 &  2.26e+04 &   10 &           &           & iters  \\ 
 \hdashline 
     &           &    1 &  5.51e-08 &  1.09e-09 &      \\ 
     &           &    2 &  2.26e-08 &  1.57e-15 &      \\ 
     &           &    3 &  2.26e-08 &  6.46e-16 &      \\ 
     &           &    4 &  5.51e-08 &  6.45e-16 &      \\ 
     &           &    5 &  2.26e-08 &  1.57e-15 &      \\ 
     &           &    6 &  2.26e-08 &  6.46e-16 &      \\ 
     &           &    7 &  5.51e-08 &  6.45e-16 &      \\ 
     &           &    8 &  2.27e-08 &  1.57e-15 &      \\ 
     &           &    9 &  2.26e-08 &  6.47e-16 &      \\ 
     &           &   10 &  2.26e-08 &  6.46e-16 &      \\ 
2260 &  2.26e+04 &   10 &           &           & iters  \\ 
 \hdashline 
     &           &    1 &  7.17e-08 &  7.15e-10 &      \\ 
     &           &    2 &  8.38e-08 &  2.05e-15 &      \\ 
     &           &    3 &  7.17e-08 &  2.39e-15 &      \\ 
     &           &    4 &  6.09e-09 &  2.05e-15 &      \\ 
     &           &    5 &  6.09e-09 &  1.74e-16 &      \\ 
     &           &    6 &  6.08e-09 &  1.74e-16 &      \\ 
     &           &    7 &  6.07e-09 &  1.73e-16 &      \\ 
     &           &    8 &  6.06e-09 &  1.73e-16 &      \\ 
     &           &    9 &  6.05e-09 &  1.73e-16 &      \\ 
     &           &   10 &  6.04e-09 &  1.73e-16 &      \\ 
2261 &  2.26e+04 &   10 &           &           & iters  \\ 
 \hdashline 
     &           &    1 &  2.93e-08 &  4.26e-10 &      \\ 
     &           &    2 &  4.84e-08 &  8.37e-16 &      \\ 
     &           &    3 &  2.93e-08 &  1.38e-15 &      \\ 
     &           &    4 &  2.93e-08 &  8.37e-16 &      \\ 
     &           &    5 &  4.84e-08 &  8.36e-16 &      \\ 
     &           &    6 &  2.93e-08 &  1.38e-15 &      \\ 
     &           &    7 &  4.84e-08 &  8.37e-16 &      \\ 
     &           &    8 &  2.94e-08 &  1.38e-15 &      \\ 
     &           &    9 &  2.93e-08 &  8.38e-16 &      \\ 
     &           &   10 &  4.84e-08 &  8.37e-16 &      \\ 
2262 &  2.26e+04 &   10 &           &           & iters  \\ 
 \hdashline 
     &           &    1 &  1.45e-08 &  2.34e-10 &      \\ 
     &           &    2 &  1.45e-08 &  4.15e-16 &      \\ 
     &           &    3 &  6.32e-08 &  4.14e-16 &      \\ 
     &           &    4 &  1.46e-08 &  1.80e-15 &      \\ 
     &           &    5 &  1.46e-08 &  4.16e-16 &      \\ 
     &           &    6 &  1.45e-08 &  4.16e-16 &      \\ 
     &           &    7 &  1.45e-08 &  4.15e-16 &      \\ 
     &           &    8 &  1.45e-08 &  4.14e-16 &      \\ 
     &           &    9 &  6.32e-08 &  4.14e-16 &      \\ 
     &           &   10 &  1.46e-08 &  1.80e-15 &      \\ 
2263 &  2.26e+04 &   10 &           &           & iters  \\ 
 \hdashline 
     &           &    1 &  5.07e-09 &  3.56e-10 &      \\ 
     &           &    2 &  5.06e-09 &  1.45e-16 &      \\ 
     &           &    3 &  5.06e-09 &  1.45e-16 &      \\ 
     &           &    4 &  5.05e-09 &  1.44e-16 &      \\ 
     &           &    5 &  5.04e-09 &  1.44e-16 &      \\ 
     &           &    6 &  5.03e-09 &  1.44e-16 &      \\ 
     &           &    7 &  5.03e-09 &  1.44e-16 &      \\ 
     &           &    8 &  5.02e-09 &  1.43e-16 &      \\ 
     &           &    9 &  5.01e-09 &  1.43e-16 &      \\ 
     &           &   10 &  5.01e-09 &  1.43e-16 &      \\ 
2264 &  2.26e+04 &   10 &           &           & iters  \\ 
 \hdashline 
     &           &    1 &  5.74e-08 &  5.43e-10 &      \\ 
     &           &    2 &  2.04e-08 &  1.64e-15 &      \\ 
     &           &    3 &  2.04e-08 &  5.82e-16 &      \\ 
     &           &    4 &  5.74e-08 &  5.81e-16 &      \\ 
     &           &    5 &  2.04e-08 &  1.64e-15 &      \\ 
     &           &    6 &  2.04e-08 &  5.82e-16 &      \\ 
     &           &    7 &  5.73e-08 &  5.82e-16 &      \\ 
     &           &    8 &  2.04e-08 &  1.64e-15 &      \\ 
     &           &    9 &  2.04e-08 &  5.83e-16 &      \\ 
     &           &   10 &  2.04e-08 &  5.82e-16 &      \\ 
2265 &  2.26e+04 &   10 &           &           & iters  \\ 
 \hdashline 
     &           &    1 &  2.01e-08 &  6.26e-10 &      \\ 
     &           &    2 &  2.01e-08 &  5.74e-16 &      \\ 
     &           &    3 &  5.76e-08 &  5.73e-16 &      \\ 
     &           &    4 &  2.01e-08 &  1.64e-15 &      \\ 
     &           &    5 &  2.01e-08 &  5.75e-16 &      \\ 
     &           &    6 &  2.01e-08 &  5.74e-16 &      \\ 
     &           &    7 &  5.76e-08 &  5.73e-16 &      \\ 
     &           &    8 &  2.01e-08 &  1.64e-15 &      \\ 
     &           &    9 &  2.01e-08 &  5.75e-16 &      \\ 
     &           &   10 &  2.01e-08 &  5.74e-16 &      \\ 
2266 &  2.26e+04 &   10 &           &           & iters  \\ 
 \hdashline 
     &           &    1 &  5.04e-08 &  7.65e-10 &      \\ 
     &           &    2 &  5.04e-08 &  1.44e-15 &      \\ 
     &           &    3 &  2.74e-08 &  1.44e-15 &      \\ 
     &           &    4 &  2.74e-08 &  7.82e-16 &      \\ 
     &           &    5 &  5.04e-08 &  7.81e-16 &      \\ 
     &           &    6 &  2.74e-08 &  1.44e-15 &      \\ 
     &           &    7 &  2.74e-08 &  7.82e-16 &      \\ 
     &           &    8 &  5.04e-08 &  7.81e-16 &      \\ 
     &           &    9 &  2.74e-08 &  1.44e-15 &      \\ 
     &           &   10 &  2.73e-08 &  7.82e-16 &      \\ 
2267 &  2.27e+04 &   10 &           &           & iters  \\ 
 \hdashline 
     &           &    1 &  1.83e-08 &  6.14e-10 &      \\ 
     &           &    2 &  5.94e-08 &  5.22e-16 &      \\ 
     &           &    3 &  1.84e-08 &  1.70e-15 &      \\ 
     &           &    4 &  1.83e-08 &  5.24e-16 &      \\ 
     &           &    5 &  1.83e-08 &  5.23e-16 &      \\ 
     &           &    6 &  5.94e-08 &  5.22e-16 &      \\ 
     &           &    7 &  1.84e-08 &  1.70e-15 &      \\ 
     &           &    8 &  1.83e-08 &  5.24e-16 &      \\ 
     &           &    9 &  1.83e-08 &  5.23e-16 &      \\ 
     &           &   10 &  5.94e-08 &  5.22e-16 &      \\ 
2268 &  2.27e+04 &   10 &           &           & iters  \\ 
 \hdashline 
     &           &    1 &  4.74e-08 &  4.74e-10 &      \\ 
     &           &    2 &  3.03e-08 &  1.35e-15 &      \\ 
     &           &    3 &  4.74e-08 &  8.66e-16 &      \\ 
     &           &    4 &  3.04e-08 &  1.35e-15 &      \\ 
     &           &    5 &  3.03e-08 &  8.67e-16 &      \\ 
     &           &    6 &  4.74e-08 &  8.65e-16 &      \\ 
     &           &    7 &  3.03e-08 &  1.35e-15 &      \\ 
     &           &    8 &  4.74e-08 &  8.66e-16 &      \\ 
     &           &    9 &  3.04e-08 &  1.35e-15 &      \\ 
     &           &   10 &  3.03e-08 &  8.67e-16 &      \\ 
2269 &  2.27e+04 &   10 &           &           & iters  \\ 
 \hdashline 
     &           &    1 &  6.34e-08 &  1.01e-09 &      \\ 
     &           &    2 &  1.44e-08 &  1.81e-15 &      \\ 
     &           &    3 &  1.44e-08 &  4.11e-16 &      \\ 
     &           &    4 &  1.44e-08 &  4.10e-16 &      \\ 
     &           &    5 &  1.43e-08 &  4.10e-16 &      \\ 
     &           &    6 &  1.43e-08 &  4.09e-16 &      \\ 
     &           &    7 &  6.34e-08 &  4.08e-16 &      \\ 
     &           &    8 &  1.44e-08 &  1.81e-15 &      \\ 
     &           &    9 &  1.44e-08 &  4.10e-16 &      \\ 
     &           &   10 &  1.43e-08 &  4.10e-16 &      \\ 
2270 &  2.27e+04 &   10 &           &           & iters  \\ 
 \hdashline 
     &           &    1 &  6.78e-08 &  5.52e-10 &      \\ 
     &           &    2 &  9.99e-09 &  1.93e-15 &      \\ 
     &           &    3 &  9.97e-09 &  2.85e-16 &      \\ 
     &           &    4 &  9.96e-09 &  2.85e-16 &      \\ 
     &           &    5 &  9.94e-09 &  2.84e-16 &      \\ 
     &           &    6 &  9.93e-09 &  2.84e-16 &      \\ 
     &           &    7 &  9.92e-09 &  2.83e-16 &      \\ 
     &           &    8 &  6.78e-08 &  2.83e-16 &      \\ 
     &           &    9 &  1.00e-08 &  1.93e-15 &      \\ 
     &           &   10 &  9.98e-09 &  2.85e-16 &      \\ 
2271 &  2.27e+04 &   10 &           &           & iters  \\ 
 \hdashline 
     &           &    1 &  2.21e-08 &  1.53e-09 &      \\ 
     &           &    2 &  2.21e-08 &  6.31e-16 &      \\ 
     &           &    3 &  2.20e-08 &  6.30e-16 &      \\ 
     &           &    4 &  5.57e-08 &  6.29e-16 &      \\ 
     &           &    5 &  2.21e-08 &  1.59e-15 &      \\ 
     &           &    6 &  2.20e-08 &  6.30e-16 &      \\ 
     &           &    7 &  2.20e-08 &  6.29e-16 &      \\ 
     &           &    8 &  5.57e-08 &  6.28e-16 &      \\ 
     &           &    9 &  2.21e-08 &  1.59e-15 &      \\ 
     &           &   10 &  2.20e-08 &  6.30e-16 &      \\ 
2272 &  2.27e+04 &   10 &           &           & iters  \\ 
 \hdashline 
     &           &    1 &  6.84e-08 &  1.17e-09 &      \\ 
     &           &    2 &  9.38e-09 &  1.95e-15 &      \\ 
     &           &    3 &  9.36e-09 &  2.68e-16 &      \\ 
     &           &    4 &  9.35e-09 &  2.67e-16 &      \\ 
     &           &    5 &  9.34e-09 &  2.67e-16 &      \\ 
     &           &    6 &  9.32e-09 &  2.66e-16 &      \\ 
     &           &    7 &  9.31e-09 &  2.66e-16 &      \\ 
     &           &    8 &  9.30e-09 &  2.66e-16 &      \\ 
     &           &    9 &  6.84e-08 &  2.65e-16 &      \\ 
     &           &   10 &  9.38e-09 &  1.95e-15 &      \\ 
2273 &  2.27e+04 &   10 &           &           & iters  \\ 
 \hdashline 
     &           &    1 &  3.55e-08 &  3.10e-10 &      \\ 
     &           &    2 &  4.23e-08 &  1.01e-15 &      \\ 
     &           &    3 &  3.55e-08 &  1.21e-15 &      \\ 
     &           &    4 &  4.23e-08 &  1.01e-15 &      \\ 
     &           &    5 &  3.55e-08 &  1.21e-15 &      \\ 
     &           &    6 &  4.23e-08 &  1.01e-15 &      \\ 
     &           &    7 &  3.55e-08 &  1.21e-15 &      \\ 
     &           &    8 &  4.23e-08 &  1.01e-15 &      \\ 
     &           &    9 &  3.55e-08 &  1.21e-15 &      \\ 
     &           &   10 &  3.54e-08 &  1.01e-15 &      \\ 
2274 &  2.27e+04 &   10 &           &           & iters  \\ 
 \hdashline 
     &           &    1 &  5.54e-08 &  1.67e-09 &      \\ 
     &           &    2 &  2.24e-08 &  1.58e-15 &      \\ 
     &           &    3 &  2.23e-08 &  6.38e-16 &      \\ 
     &           &    4 &  2.23e-08 &  6.37e-16 &      \\ 
     &           &    5 &  5.54e-08 &  6.37e-16 &      \\ 
     &           &    6 &  2.23e-08 &  1.58e-15 &      \\ 
     &           &    7 &  2.23e-08 &  6.38e-16 &      \\ 
     &           &    8 &  5.54e-08 &  6.37e-16 &      \\ 
     &           &    9 &  2.24e-08 &  1.58e-15 &      \\ 
     &           &   10 &  2.23e-08 &  6.38e-16 &      \\ 
2275 &  2.27e+04 &   10 &           &           & iters  \\ 
 \hdashline 
     &           &    1 &  1.49e-09 &  2.28e-09 &      \\ 
     &           &    2 &  1.49e-09 &  4.27e-17 &      \\ 
     &           &    3 &  1.49e-09 &  4.26e-17 &      \\ 
     &           &    4 &  1.49e-09 &  4.25e-17 &      \\ 
     &           &    5 &  1.49e-09 &  4.25e-17 &      \\ 
     &           &    6 &  1.48e-09 &  4.24e-17 &      \\ 
     &           &    7 &  1.48e-09 &  4.24e-17 &      \\ 
     &           &    8 &  1.48e-09 &  4.23e-17 &      \\ 
     &           &    9 &  1.48e-09 &  4.22e-17 &      \\ 
     &           &   10 &  1.48e-09 &  4.22e-17 &      \\ 
2276 &  2.28e+04 &   10 &           &           & iters  \\ 
 \hdashline 
     &           &    1 &  4.40e-08 &  1.29e-09 &      \\ 
     &           &    2 &  3.37e-08 &  1.26e-15 &      \\ 
     &           &    3 &  3.37e-08 &  9.62e-16 &      \\ 
     &           &    4 &  4.41e-08 &  9.61e-16 &      \\ 
     &           &    5 &  3.37e-08 &  1.26e-15 &      \\ 
     &           &    6 &  4.41e-08 &  9.61e-16 &      \\ 
     &           &    7 &  3.37e-08 &  1.26e-15 &      \\ 
     &           &    8 &  4.40e-08 &  9.61e-16 &      \\ 
     &           &    9 &  3.37e-08 &  1.26e-15 &      \\ 
     &           &   10 &  3.37e-08 &  9.62e-16 &      \\ 
2277 &  2.28e+04 &   10 &           &           & iters  \\ 
 \hdashline 
     &           &    1 &  1.94e-08 &  7.54e-10 &      \\ 
     &           &    2 &  1.94e-08 &  5.53e-16 &      \\ 
     &           &    3 &  1.93e-08 &  5.53e-16 &      \\ 
     &           &    4 &  5.84e-08 &  5.52e-16 &      \\ 
     &           &    5 &  1.94e-08 &  1.67e-15 &      \\ 
     &           &    6 &  1.94e-08 &  5.53e-16 &      \\ 
     &           &    7 &  1.93e-08 &  5.53e-16 &      \\ 
     &           &    8 &  5.84e-08 &  5.52e-16 &      \\ 
     &           &    9 &  1.94e-08 &  1.67e-15 &      \\ 
     &           &   10 &  1.94e-08 &  5.53e-16 &      \\ 
2278 &  2.28e+04 &   10 &           &           & iters  \\ 
 \hdashline 
     &           &    1 &  6.35e-08 &  6.67e-10 &      \\ 
     &           &    2 &  1.42e-08 &  1.81e-15 &      \\ 
     &           &    3 &  1.42e-08 &  4.07e-16 &      \\ 
     &           &    4 &  1.42e-08 &  4.06e-16 &      \\ 
     &           &    5 &  1.42e-08 &  4.05e-16 &      \\ 
     &           &    6 &  6.35e-08 &  4.05e-16 &      \\ 
     &           &    7 &  9.19e-08 &  1.81e-15 &      \\ 
     &           &    8 &  6.36e-08 &  2.62e-15 &      \\ 
     &           &    9 &  1.42e-08 &  1.81e-15 &      \\ 
     &           &   10 &  1.42e-08 &  4.06e-16 &      \\ 
2279 &  2.28e+04 &   10 &           &           & iters  \\ 
 \hdashline 
     &           &    1 &  1.13e-08 &  1.12e-09 &      \\ 
     &           &    2 &  1.12e-08 &  3.21e-16 &      \\ 
     &           &    3 &  1.12e-08 &  3.21e-16 &      \\ 
     &           &    4 &  1.12e-08 &  3.20e-16 &      \\ 
     &           &    5 &  6.65e-08 &  3.20e-16 &      \\ 
     &           &    6 &  1.13e-08 &  1.90e-15 &      \\ 
     &           &    7 &  1.13e-08 &  3.22e-16 &      \\ 
     &           &    8 &  1.13e-08 &  3.22e-16 &      \\ 
     &           &    9 &  1.12e-08 &  3.21e-16 &      \\ 
     &           &   10 &  1.12e-08 &  3.21e-16 &      \\ 
2280 &  2.28e+04 &   10 &           &           & iters  \\ 
 \hdashline 
     &           &    1 &  2.12e-08 &  1.53e-09 &      \\ 
     &           &    2 &  2.11e-08 &  6.04e-16 &      \\ 
     &           &    3 &  5.66e-08 &  6.03e-16 &      \\ 
     &           &    4 &  2.12e-08 &  1.61e-15 &      \\ 
     &           &    5 &  2.12e-08 &  6.05e-16 &      \\ 
     &           &    6 &  2.11e-08 &  6.04e-16 &      \\ 
     &           &    7 &  5.66e-08 &  6.03e-16 &      \\ 
     &           &    8 &  2.12e-08 &  1.62e-15 &      \\ 
     &           &    9 &  2.11e-08 &  6.04e-16 &      \\ 
     &           &   10 &  2.11e-08 &  6.04e-16 &      \\ 
2281 &  2.28e+04 &   10 &           &           & iters  \\ 
 \hdashline 
     &           &    1 &  1.74e-08 &  1.94e-10 &      \\ 
     &           &    2 &  1.74e-08 &  4.98e-16 &      \\ 
     &           &    3 &  1.74e-08 &  4.97e-16 &      \\ 
     &           &    4 &  6.03e-08 &  4.96e-16 &      \\ 
     &           &    5 &  1.75e-08 &  1.72e-15 &      \\ 
     &           &    6 &  1.74e-08 &  4.98e-16 &      \\ 
     &           &    7 &  1.74e-08 &  4.97e-16 &      \\ 
     &           &    8 &  1.74e-08 &  4.97e-16 &      \\ 
     &           &    9 &  6.03e-08 &  4.96e-16 &      \\ 
     &           &   10 &  1.74e-08 &  1.72e-15 &      \\ 
2282 &  2.28e+04 &   10 &           &           & iters  \\ 
 \hdashline 
     &           &    1 &  2.65e-08 &  1.54e-09 &      \\ 
     &           &    2 &  1.24e-08 &  7.57e-16 &      \\ 
     &           &    3 &  1.23e-08 &  3.52e-16 &      \\ 
     &           &    4 &  2.65e-08 &  3.52e-16 &      \\ 
     &           &    5 &  1.24e-08 &  7.57e-16 &      \\ 
     &           &    6 &  1.23e-08 &  3.53e-16 &      \\ 
     &           &    7 &  2.65e-08 &  3.52e-16 &      \\ 
     &           &    8 &  1.24e-08 &  7.57e-16 &      \\ 
     &           &    9 &  1.23e-08 &  3.53e-16 &      \\ 
     &           &   10 &  2.65e-08 &  3.52e-16 &      \\ 
2283 &  2.28e+04 &   10 &           &           & iters  \\ 
 \hdashline 
     &           &    1 &  2.23e-08 &  1.66e-09 &      \\ 
     &           &    2 &  5.54e-08 &  6.37e-16 &      \\ 
     &           &    3 &  2.24e-08 &  1.58e-15 &      \\ 
     &           &    4 &  2.23e-08 &  6.38e-16 &      \\ 
     &           &    5 &  5.54e-08 &  6.37e-16 &      \\ 
     &           &    6 &  2.24e-08 &  1.58e-15 &      \\ 
     &           &    7 &  2.23e-08 &  6.39e-16 &      \\ 
     &           &    8 &  2.23e-08 &  6.38e-16 &      \\ 
     &           &    9 &  5.54e-08 &  6.37e-16 &      \\ 
     &           &   10 &  2.24e-08 &  1.58e-15 &      \\ 
2284 &  2.28e+04 &   10 &           &           & iters  \\ 
 \hdashline 
     &           &    1 &  2.62e-08 &  1.81e-09 &      \\ 
     &           &    2 &  2.62e-08 &  7.48e-16 &      \\ 
     &           &    3 &  5.16e-08 &  7.47e-16 &      \\ 
     &           &    4 &  2.62e-08 &  1.47e-15 &      \\ 
     &           &    5 &  2.62e-08 &  7.48e-16 &      \\ 
     &           &    6 &  5.16e-08 &  7.46e-16 &      \\ 
     &           &    7 &  2.62e-08 &  1.47e-15 &      \\ 
     &           &    8 &  2.62e-08 &  7.48e-16 &      \\ 
     &           &    9 &  5.16e-08 &  7.46e-16 &      \\ 
     &           &   10 &  2.62e-08 &  1.47e-15 &      \\ 
2285 &  2.28e+04 &   10 &           &           & iters  \\ 
 \hdashline 
     &           &    1 &  5.68e-08 &  8.87e-10 &      \\ 
     &           &    2 &  2.10e-08 &  1.62e-15 &      \\ 
     &           &    3 &  2.10e-08 &  5.99e-16 &      \\ 
     &           &    4 &  2.09e-08 &  5.98e-16 &      \\ 
     &           &    5 &  5.68e-08 &  5.97e-16 &      \\ 
     &           &    6 &  2.10e-08 &  1.62e-15 &      \\ 
     &           &    7 &  2.10e-08 &  5.99e-16 &      \\ 
     &           &    8 &  5.68e-08 &  5.98e-16 &      \\ 
     &           &    9 &  2.10e-08 &  1.62e-15 &      \\ 
     &           &   10 &  2.10e-08 &  5.99e-16 &      \\ 
2286 &  2.28e+04 &   10 &           &           & iters  \\ 
 \hdashline 
     &           &    1 &  1.36e-08 &  1.11e-09 &      \\ 
     &           &    2 &  1.36e-08 &  3.89e-16 &      \\ 
     &           &    3 &  1.36e-08 &  3.88e-16 &      \\ 
     &           &    4 &  6.41e-08 &  3.88e-16 &      \\ 
     &           &    5 &  1.37e-08 &  1.83e-15 &      \\ 
     &           &    6 &  1.36e-08 &  3.90e-16 &      \\ 
     &           &    7 &  1.36e-08 &  3.89e-16 &      \\ 
     &           &    8 &  1.36e-08 &  3.89e-16 &      \\ 
     &           &    9 &  6.41e-08 &  3.88e-16 &      \\ 
     &           &   10 &  1.37e-08 &  1.83e-15 &      \\ 
2287 &  2.29e+04 &   10 &           &           & iters  \\ 
 \hdashline 
     &           &    1 &  9.98e-09 &  6.45e-10 &      \\ 
     &           &    2 &  9.97e-09 &  2.85e-16 &      \\ 
     &           &    3 &  9.96e-09 &  2.85e-16 &      \\ 
     &           &    4 &  6.77e-08 &  2.84e-16 &      \\ 
     &           &    5 &  1.00e-08 &  1.93e-15 &      \\ 
     &           &    6 &  1.00e-08 &  2.86e-16 &      \\ 
     &           &    7 &  1.00e-08 &  2.86e-16 &      \\ 
     &           &    8 &  9.99e-09 &  2.86e-16 &      \\ 
     &           &    9 &  9.98e-09 &  2.85e-16 &      \\ 
     &           &   10 &  9.97e-09 &  2.85e-16 &      \\ 
2288 &  2.29e+04 &   10 &           &           & iters  \\ 
 \hdashline 
     &           &    1 &  2.56e-08 &  8.12e-10 &      \\ 
     &           &    2 &  5.21e-08 &  7.30e-16 &      \\ 
     &           &    3 &  2.56e-08 &  1.49e-15 &      \\ 
     &           &    4 &  2.56e-08 &  7.31e-16 &      \\ 
     &           &    5 &  5.21e-08 &  7.30e-16 &      \\ 
     &           &    6 &  2.56e-08 &  1.49e-15 &      \\ 
     &           &    7 &  2.56e-08 &  7.31e-16 &      \\ 
     &           &    8 &  5.21e-08 &  7.30e-16 &      \\ 
     &           &    9 &  2.56e-08 &  1.49e-15 &      \\ 
     &           &   10 &  2.56e-08 &  7.31e-16 &      \\ 
2289 &  2.29e+04 &   10 &           &           & iters  \\ 
 \hdashline 
     &           &    1 &  7.95e-09 &  1.11e-09 &      \\ 
     &           &    2 &  7.94e-09 &  2.27e-16 &      \\ 
     &           &    3 &  7.92e-09 &  2.26e-16 &      \\ 
     &           &    4 &  7.91e-09 &  2.26e-16 &      \\ 
     &           &    5 &  7.90e-09 &  2.26e-16 &      \\ 
     &           &    6 &  7.89e-09 &  2.26e-16 &      \\ 
     &           &    7 &  7.88e-09 &  2.25e-16 &      \\ 
     &           &    8 &  6.98e-08 &  2.25e-16 &      \\ 
     &           &    9 &  8.57e-08 &  1.99e-15 &      \\ 
     &           &   10 &  6.98e-08 &  2.44e-15 &      \\ 
2290 &  2.29e+04 &   10 &           &           & iters  \\ 
 \hdashline 
     &           &    1 &  9.00e-08 &  1.71e-09 &      \\ 
     &           &    2 &  6.55e-08 &  2.57e-15 &      \\ 
     &           &    3 &  1.22e-08 &  1.87e-15 &      \\ 
     &           &    4 &  1.22e-08 &  3.49e-16 &      \\ 
     &           &    5 &  1.22e-08 &  3.49e-16 &      \\ 
     &           &    6 &  1.22e-08 &  3.48e-16 &      \\ 
     &           &    7 &  6.55e-08 &  3.48e-16 &      \\ 
     &           &    8 &  1.23e-08 &  1.87e-15 &      \\ 
     &           &    9 &  1.22e-08 &  3.50e-16 &      \\ 
     &           &   10 &  1.22e-08 &  3.50e-16 &      \\ 
2291 &  2.29e+04 &   10 &           &           & iters  \\ 
 \hdashline 
     &           &    1 &  4.82e-08 &  1.47e-09 &      \\ 
     &           &    2 &  2.95e-08 &  1.38e-15 &      \\ 
     &           &    3 &  2.95e-08 &  8.43e-16 &      \\ 
     &           &    4 &  4.82e-08 &  8.41e-16 &      \\ 
     &           &    5 &  2.95e-08 &  1.38e-15 &      \\ 
     &           &    6 &  4.82e-08 &  8.42e-16 &      \\ 
     &           &    7 &  2.95e-08 &  1.38e-15 &      \\ 
     &           &    8 &  2.95e-08 &  8.43e-16 &      \\ 
     &           &    9 &  4.82e-08 &  8.42e-16 &      \\ 
     &           &   10 &  2.95e-08 &  1.38e-15 &      \\ 
2292 &  2.29e+04 &   10 &           &           & iters  \\ 
 \hdashline 
     &           &    1 &  4.34e-08 &  1.83e-09 &      \\ 
     &           &    2 &  3.43e-08 &  1.24e-15 &      \\ 
     &           &    3 &  3.43e-08 &  9.79e-16 &      \\ 
     &           &    4 &  4.35e-08 &  9.78e-16 &      \\ 
     &           &    5 &  3.43e-08 &  1.24e-15 &      \\ 
     &           &    6 &  4.35e-08 &  9.78e-16 &      \\ 
     &           &    7 &  3.43e-08 &  1.24e-15 &      \\ 
     &           &    8 &  4.34e-08 &  9.79e-16 &      \\ 
     &           &    9 &  3.43e-08 &  1.24e-15 &      \\ 
     &           &   10 &  4.34e-08 &  9.79e-16 &      \\ 
2293 &  2.29e+04 &   10 &           &           & iters  \\ 
 \hdashline 
     &           &    1 &  5.56e-08 &  3.26e-09 &      \\ 
     &           &    2 &  2.22e-08 &  1.59e-15 &      \\ 
     &           &    3 &  5.56e-08 &  6.32e-16 &      \\ 
     &           &    4 &  2.22e-08 &  1.59e-15 &      \\ 
     &           &    5 &  2.22e-08 &  6.34e-16 &      \\ 
     &           &    6 &  5.55e-08 &  6.33e-16 &      \\ 
     &           &    7 &  2.22e-08 &  1.59e-15 &      \\ 
     &           &    8 &  2.22e-08 &  6.34e-16 &      \\ 
     &           &    9 &  2.22e-08 &  6.33e-16 &      \\ 
     &           &   10 &  5.56e-08 &  6.32e-16 &      \\ 
2294 &  2.29e+04 &   10 &           &           & iters  \\ 
 \hdashline 
     &           &    1 &  5.25e-08 &  3.44e-10 &      \\ 
     &           &    2 &  2.52e-08 &  1.50e-15 &      \\ 
     &           &    3 &  2.52e-08 &  7.20e-16 &      \\ 
     &           &    4 &  5.25e-08 &  7.19e-16 &      \\ 
     &           &    5 &  2.52e-08 &  1.50e-15 &      \\ 
     &           &    6 &  2.52e-08 &  7.20e-16 &      \\ 
     &           &    7 &  5.25e-08 &  7.19e-16 &      \\ 
     &           &    8 &  2.52e-08 &  1.50e-15 &      \\ 
     &           &    9 &  2.52e-08 &  7.20e-16 &      \\ 
     &           &   10 &  5.25e-08 &  7.19e-16 &      \\ 
2295 &  2.29e+04 &   10 &           &           & iters  \\ 
 \hdashline 
     &           &    1 &  1.01e-08 &  1.03e-09 &      \\ 
     &           &    2 &  1.01e-08 &  2.89e-16 &      \\ 
     &           &    3 &  1.01e-08 &  2.89e-16 &      \\ 
     &           &    4 &  6.76e-08 &  2.88e-16 &      \\ 
     &           &    5 &  1.02e-08 &  1.93e-15 &      \\ 
     &           &    6 &  1.02e-08 &  2.91e-16 &      \\ 
     &           &    7 &  1.02e-08 &  2.90e-16 &      \\ 
     &           &    8 &  1.01e-08 &  2.90e-16 &      \\ 
     &           &    9 &  1.01e-08 &  2.89e-16 &      \\ 
     &           &   10 &  1.01e-08 &  2.89e-16 &      \\ 
2296 &  2.30e+04 &   10 &           &           & iters  \\ 
 \hdashline 
     &           &    1 &  4.34e-08 &  2.18e-10 &      \\ 
     &           &    2 &  3.44e-08 &  1.24e-15 &      \\ 
     &           &    3 &  4.34e-08 &  9.81e-16 &      \\ 
     &           &    4 &  3.44e-08 &  1.24e-15 &      \\ 
     &           &    5 &  4.34e-08 &  9.81e-16 &      \\ 
     &           &    6 &  3.44e-08 &  1.24e-15 &      \\ 
     &           &    7 &  3.43e-08 &  9.82e-16 &      \\ 
     &           &    8 &  4.34e-08 &  9.80e-16 &      \\ 
     &           &    9 &  3.44e-08 &  1.24e-15 &      \\ 
     &           &   10 &  4.34e-08 &  9.81e-16 &      \\ 
2297 &  2.30e+04 &   10 &           &           & iters  \\ 
 \hdashline 
     &           &    1 &  1.21e-08 &  5.93e-10 &      \\ 
     &           &    2 &  1.21e-08 &  3.45e-16 &      \\ 
     &           &    3 &  1.21e-08 &  3.45e-16 &      \\ 
     &           &    4 &  6.56e-08 &  3.44e-16 &      \\ 
     &           &    5 &  1.21e-08 &  1.87e-15 &      \\ 
     &           &    6 &  1.21e-08 &  3.46e-16 &      \\ 
     &           &    7 &  1.21e-08 &  3.46e-16 &      \\ 
     &           &    8 &  1.21e-08 &  3.45e-16 &      \\ 
     &           &    9 &  1.21e-08 &  3.45e-16 &      \\ 
     &           &   10 &  6.56e-08 &  3.44e-16 &      \\ 
2298 &  2.30e+04 &   10 &           &           & iters  \\ 
 \hdashline 
     &           &    1 &  6.47e-10 &  6.78e-11 &      \\ 
     &           &    2 &  6.46e-10 &  1.85e-17 &      \\ 
     &           &    3 &  6.45e-10 &  1.84e-17 &      \\ 
     &           &    4 &  6.44e-10 &  1.84e-17 &      \\ 
     &           &    5 &  6.43e-10 &  1.84e-17 &      \\ 
     &           &    6 &  6.42e-10 &  1.84e-17 &      \\ 
     &           &    7 &  6.41e-10 &  1.83e-17 &      \\ 
     &           &    8 &  6.40e-10 &  1.83e-17 &      \\ 
     &           &    9 &  6.39e-10 &  1.83e-17 &      \\ 
     &           &   10 &  6.38e-10 &  1.82e-17 &      \\ 
2299 &  2.30e+04 &   10 &           &           & iters  \\ 
 \hdashline 
     &           &    1 &  9.04e-09 &  7.97e-10 &      \\ 
     &           &    2 &  9.03e-09 &  2.58e-16 &      \\ 
     &           &    3 &  9.01e-09 &  2.58e-16 &      \\ 
     &           &    4 &  9.00e-09 &  2.57e-16 &      \\ 
     &           &    5 &  8.99e-09 &  2.57e-16 &      \\ 
     &           &    6 &  6.87e-08 &  2.56e-16 &      \\ 
     &           &    7 &  8.68e-08 &  1.96e-15 &      \\ 
     &           &    8 &  6.87e-08 &  2.48e-15 &      \\ 
     &           &    9 &  9.05e-09 &  1.96e-15 &      \\ 
     &           &   10 &  9.03e-09 &  2.58e-16 &      \\ 
2300 &  2.30e+04 &   10 &           &           & iters  \\ 
 \hdashline 
     &           &    1 &  1.58e-08 &  1.61e-09 &      \\ 
     &           &    2 &  1.58e-08 &  4.51e-16 &      \\ 
     &           &    3 &  6.19e-08 &  4.51e-16 &      \\ 
     &           &    4 &  1.59e-08 &  1.77e-15 &      \\ 
     &           &    5 &  1.58e-08 &  4.53e-16 &      \\ 
     &           &    6 &  1.58e-08 &  4.52e-16 &      \\ 
     &           &    7 &  1.58e-08 &  4.51e-16 &      \\ 
     &           &    8 &  6.19e-08 &  4.51e-16 &      \\ 
     &           &    9 &  1.59e-08 &  1.77e-15 &      \\ 
     &           &   10 &  1.58e-08 &  4.53e-16 &      \\ 
2301 &  2.30e+04 &   10 &           &           & iters  \\ 
 \hdashline 
     &           &    1 &  3.18e-08 &  1.44e-09 &      \\ 
     &           &    2 &  4.59e-08 &  9.09e-16 &      \\ 
     &           &    3 &  3.19e-08 &  1.31e-15 &      \\ 
     &           &    4 &  4.59e-08 &  9.09e-16 &      \\ 
     &           &    5 &  3.19e-08 &  1.31e-15 &      \\ 
     &           &    6 &  3.18e-08 &  9.10e-16 &      \\ 
     &           &    7 &  4.59e-08 &  9.09e-16 &      \\ 
     &           &    8 &  3.19e-08 &  1.31e-15 &      \\ 
     &           &    9 &  4.59e-08 &  9.09e-16 &      \\ 
     &           &   10 &  3.19e-08 &  1.31e-15 &      \\ 
2302 &  2.30e+04 &   10 &           &           & iters  \\ 
 \hdashline 
     &           &    1 &  1.34e-08 &  1.27e-09 &      \\ 
     &           &    2 &  1.33e-08 &  3.81e-16 &      \\ 
     &           &    3 &  1.33e-08 &  3.81e-16 &      \\ 
     &           &    4 &  6.44e-08 &  3.80e-16 &      \\ 
     &           &    5 &  1.34e-08 &  1.84e-15 &      \\ 
     &           &    6 &  1.34e-08 &  3.82e-16 &      \\ 
     &           &    7 &  1.34e-08 &  3.82e-16 &      \\ 
     &           &    8 &  1.33e-08 &  3.81e-16 &      \\ 
     &           &    9 &  1.33e-08 &  3.81e-16 &      \\ 
     &           &   10 &  6.44e-08 &  3.80e-16 &      \\ 
2303 &  2.30e+04 &   10 &           &           & iters  \\ 
 \hdashline 
     &           &    1 &  2.32e-08 &  1.68e-09 &      \\ 
     &           &    2 &  5.45e-08 &  6.63e-16 &      \\ 
     &           &    3 &  2.33e-08 &  1.56e-15 &      \\ 
     &           &    4 &  2.32e-08 &  6.64e-16 &      \\ 
     &           &    5 &  5.45e-08 &  6.64e-16 &      \\ 
     &           &    6 &  2.33e-08 &  1.55e-15 &      \\ 
     &           &    7 &  2.33e-08 &  6.65e-16 &      \\ 
     &           &    8 &  2.32e-08 &  6.64e-16 &      \\ 
     &           &    9 &  5.45e-08 &  6.63e-16 &      \\ 
     &           &   10 &  2.33e-08 &  1.56e-15 &      \\ 
2304 &  2.30e+04 &   10 &           &           & iters  \\ 
 \hdashline 
     &           &    1 &  5.10e-09 &  3.92e-10 &      \\ 
     &           &    2 &  5.09e-09 &  1.46e-16 &      \\ 
     &           &    3 &  5.09e-09 &  1.45e-16 &      \\ 
     &           &    4 &  5.08e-09 &  1.45e-16 &      \\ 
     &           &    5 &  5.07e-09 &  1.45e-16 &      \\ 
     &           &    6 &  5.06e-09 &  1.45e-16 &      \\ 
     &           &    7 &  5.06e-09 &  1.45e-16 &      \\ 
     &           &    8 &  5.05e-09 &  1.44e-16 &      \\ 
     &           &    9 &  7.26e-08 &  1.44e-16 &      \\ 
     &           &   10 &  8.28e-08 &  2.07e-15 &      \\ 
2305 &  2.30e+04 &   10 &           &           & iters  \\ 
 \hdashline 
     &           &    1 &  4.13e-08 &  2.28e-09 &      \\ 
     &           &    2 &  3.65e-08 &  1.18e-15 &      \\ 
     &           &    3 &  4.13e-08 &  1.04e-15 &      \\ 
     &           &    4 &  3.65e-08 &  1.18e-15 &      \\ 
     &           &    5 &  4.13e-08 &  1.04e-15 &      \\ 
     &           &    6 &  3.65e-08 &  1.18e-15 &      \\ 
     &           &    7 &  4.12e-08 &  1.04e-15 &      \\ 
     &           &    8 &  3.65e-08 &  1.18e-15 &      \\ 
     &           &    9 &  4.12e-08 &  1.04e-15 &      \\ 
     &           &   10 &  3.65e-08 &  1.18e-15 &      \\ 
2306 &  2.30e+04 &   10 &           &           & iters  \\ 
 \hdashline 
     &           &    1 &  1.46e-08 &  2.38e-09 &      \\ 
     &           &    2 &  1.46e-08 &  4.18e-16 &      \\ 
     &           &    3 &  6.31e-08 &  4.17e-16 &      \\ 
     &           &    4 &  1.47e-08 &  1.80e-15 &      \\ 
     &           &    5 &  1.47e-08 &  4.19e-16 &      \\ 
     &           &    6 &  1.46e-08 &  4.19e-16 &      \\ 
     &           &    7 &  1.46e-08 &  4.18e-16 &      \\ 
     &           &    8 &  6.31e-08 &  4.17e-16 &      \\ 
     &           &    9 &  1.47e-08 &  1.80e-15 &      \\ 
     &           &   10 &  1.47e-08 &  4.19e-16 &      \\ 
2307 &  2.31e+04 &   10 &           &           & iters  \\ 
 \hdashline 
     &           &    1 &  4.12e-08 &  7.23e-10 &      \\ 
     &           &    2 &  4.12e-08 &  1.18e-15 &      \\ 
     &           &    3 &  3.66e-08 &  1.17e-15 &      \\ 
     &           &    4 &  3.65e-08 &  1.04e-15 &      \\ 
     &           &    5 &  4.12e-08 &  1.04e-15 &      \\ 
     &           &    6 &  3.65e-08 &  1.18e-15 &      \\ 
     &           &    7 &  4.12e-08 &  1.04e-15 &      \\ 
     &           &    8 &  3.65e-08 &  1.18e-15 &      \\ 
     &           &    9 &  4.12e-08 &  1.04e-15 &      \\ 
     &           &   10 &  3.66e-08 &  1.18e-15 &      \\ 
2308 &  2.31e+04 &   10 &           &           & iters  \\ 
 \hdashline 
     &           &    1 &  1.36e-08 &  5.15e-10 &      \\ 
     &           &    2 &  6.41e-08 &  3.88e-16 &      \\ 
     &           &    3 &  1.37e-08 &  1.83e-15 &      \\ 
     &           &    4 &  1.37e-08 &  3.90e-16 &      \\ 
     &           &    5 &  1.36e-08 &  3.90e-16 &      \\ 
     &           &    6 &  1.36e-08 &  3.89e-16 &      \\ 
     &           &    7 &  6.41e-08 &  3.89e-16 &      \\ 
     &           &    8 &  9.14e-08 &  1.83e-15 &      \\ 
     &           &    9 &  6.41e-08 &  2.61e-15 &      \\ 
     &           &   10 &  1.37e-08 &  1.83e-15 &      \\ 
2309 &  2.31e+04 &   10 &           &           & iters  \\ 
 \hdashline 
     &           &    1 &  2.69e-08 &  8.62e-10 &      \\ 
     &           &    2 &  2.68e-08 &  7.67e-16 &      \\ 
     &           &    3 &  5.09e-08 &  7.66e-16 &      \\ 
     &           &    4 &  2.69e-08 &  1.45e-15 &      \\ 
     &           &    5 &  2.68e-08 &  7.67e-16 &      \\ 
     &           &    6 &  5.09e-08 &  7.66e-16 &      \\ 
     &           &    7 &  2.69e-08 &  1.45e-15 &      \\ 
     &           &    8 &  2.68e-08 &  7.67e-16 &      \\ 
     &           &    9 &  5.09e-08 &  7.66e-16 &      \\ 
     &           &   10 &  2.69e-08 &  1.45e-15 &      \\ 
2310 &  2.31e+04 &   10 &           &           & iters  \\ 
 \hdashline 
     &           &    1 &  2.67e-08 &  1.00e-09 &      \\ 
     &           &    2 &  2.67e-08 &  7.62e-16 &      \\ 
     &           &    3 &  5.11e-08 &  7.61e-16 &      \\ 
     &           &    4 &  2.67e-08 &  1.46e-15 &      \\ 
     &           &    5 &  2.67e-08 &  7.62e-16 &      \\ 
     &           &    6 &  5.11e-08 &  7.61e-16 &      \\ 
     &           &    7 &  2.67e-08 &  1.46e-15 &      \\ 
     &           &    8 &  2.67e-08 &  7.62e-16 &      \\ 
     &           &    9 &  5.11e-08 &  7.61e-16 &      \\ 
     &           &   10 &  2.67e-08 &  1.46e-15 &      \\ 
2311 &  2.31e+04 &   10 &           &           & iters  \\ 
 \hdashline 
     &           &    1 &  2.63e-08 &  1.97e-09 &      \\ 
     &           &    2 &  5.14e-08 &  7.52e-16 &      \\ 
     &           &    3 &  2.64e-08 &  1.47e-15 &      \\ 
     &           &    4 &  2.63e-08 &  7.53e-16 &      \\ 
     &           &    5 &  5.14e-08 &  7.52e-16 &      \\ 
     &           &    6 &  2.64e-08 &  1.47e-15 &      \\ 
     &           &    7 &  5.13e-08 &  7.53e-16 &      \\ 
     &           &    8 &  2.64e-08 &  1.47e-15 &      \\ 
     &           &    9 &  2.64e-08 &  7.54e-16 &      \\ 
     &           &   10 &  5.13e-08 &  7.53e-16 &      \\ 
2312 &  2.31e+04 &   10 &           &           & iters  \\ 
 \hdashline 
     &           &    1 &  5.17e-08 &  2.25e-09 &      \\ 
     &           &    2 &  2.60e-08 &  1.48e-15 &      \\ 
     &           &    3 &  2.60e-08 &  7.43e-16 &      \\ 
     &           &    4 &  5.17e-08 &  7.42e-16 &      \\ 
     &           &    5 &  2.60e-08 &  1.48e-15 &      \\ 
     &           &    6 &  2.60e-08 &  7.43e-16 &      \\ 
     &           &    7 &  5.17e-08 &  7.41e-16 &      \\ 
     &           &    8 &  2.60e-08 &  1.48e-15 &      \\ 
     &           &    9 &  2.60e-08 &  7.43e-16 &      \\ 
     &           &   10 &  5.17e-08 &  7.41e-16 &      \\ 
2313 &  2.31e+04 &   10 &           &           & iters  \\ 
 \hdashline 
     &           &    1 &  6.02e-08 &  3.90e-10 &      \\ 
     &           &    2 &  1.76e-08 &  1.72e-15 &      \\ 
     &           &    3 &  1.76e-08 &  5.02e-16 &      \\ 
     &           &    4 &  1.75e-08 &  5.01e-16 &      \\ 
     &           &    5 &  1.75e-08 &  5.01e-16 &      \\ 
     &           &    6 &  6.02e-08 &  5.00e-16 &      \\ 
     &           &    7 &  1.76e-08 &  1.72e-15 &      \\ 
     &           &    8 &  1.76e-08 &  5.02e-16 &      \\ 
     &           &    9 &  1.75e-08 &  5.01e-16 &      \\ 
     &           &   10 &  6.02e-08 &  5.00e-16 &      \\ 
2314 &  2.31e+04 &   10 &           &           & iters  \\ 
 \hdashline 
     &           &    1 &  7.08e-08 &  9.20e-10 &      \\ 
     &           &    2 &  6.94e-09 &  2.02e-15 &      \\ 
     &           &    3 &  6.93e-09 &  1.98e-16 &      \\ 
     &           &    4 &  6.92e-09 &  1.98e-16 &      \\ 
     &           &    5 &  6.91e-09 &  1.98e-16 &      \\ 
     &           &    6 &  6.90e-09 &  1.97e-16 &      \\ 
     &           &    7 &  6.89e-09 &  1.97e-16 &      \\ 
     &           &    8 &  6.88e-09 &  1.97e-16 &      \\ 
     &           &    9 &  6.87e-09 &  1.96e-16 &      \\ 
     &           &   10 &  6.86e-09 &  1.96e-16 &      \\ 
2315 &  2.31e+04 &   10 &           &           & iters  \\ 
 \hdashline 
     &           &    1 &  7.32e-09 &  1.11e-10 &      \\ 
     &           &    2 &  7.31e-09 &  2.09e-16 &      \\ 
     &           &    3 &  7.04e-08 &  2.09e-16 &      \\ 
     &           &    4 &  4.62e-08 &  2.01e-15 &      \\ 
     &           &    5 &  7.34e-09 &  1.32e-15 &      \\ 
     &           &    6 &  7.33e-09 &  2.09e-16 &      \\ 
     &           &    7 &  7.31e-09 &  2.09e-16 &      \\ 
     &           &    8 &  7.04e-08 &  2.09e-16 &      \\ 
     &           &    9 &  4.62e-08 &  2.01e-15 &      \\ 
     &           &   10 &  7.34e-09 &  1.32e-15 &      \\ 
2316 &  2.32e+04 &   10 &           &           & iters  \\ 
 \hdashline 
     &           &    1 &  1.32e-08 &  9.69e-10 &      \\ 
     &           &    2 &  1.32e-08 &  3.76e-16 &      \\ 
     &           &    3 &  1.31e-08 &  3.76e-16 &      \\ 
     &           &    4 &  1.31e-08 &  3.75e-16 &      \\ 
     &           &    5 &  6.46e-08 &  3.75e-16 &      \\ 
     &           &    6 &  1.32e-08 &  1.84e-15 &      \\ 
     &           &    7 &  1.32e-08 &  3.77e-16 &      \\ 
     &           &    8 &  1.32e-08 &  3.76e-16 &      \\ 
     &           &    9 &  1.31e-08 &  3.76e-16 &      \\ 
     &           &   10 &  1.31e-08 &  3.75e-16 &      \\ 
2317 &  2.32e+04 &   10 &           &           & iters  \\ 
 \hdashline 
     &           &    1 &  3.51e-08 &  1.19e-09 &      \\ 
     &           &    2 &  4.26e-08 &  1.00e-15 &      \\ 
     &           &    3 &  3.51e-08 &  1.22e-15 &      \\ 
     &           &    4 &  4.26e-08 &  1.00e-15 &      \\ 
     &           &    5 &  3.51e-08 &  1.22e-15 &      \\ 
     &           &    6 &  3.51e-08 &  1.00e-15 &      \\ 
     &           &    7 &  4.27e-08 &  1.00e-15 &      \\ 
     &           &    8 &  3.51e-08 &  1.22e-15 &      \\ 
     &           &    9 &  4.26e-08 &  1.00e-15 &      \\ 
     &           &   10 &  3.51e-08 &  1.22e-15 &      \\ 
2318 &  2.32e+04 &   10 &           &           & iters  \\ 
 \hdashline 
     &           &    1 &  1.42e-08 &  9.05e-10 &      \\ 
     &           &    2 &  1.42e-08 &  4.05e-16 &      \\ 
     &           &    3 &  2.47e-08 &  4.04e-16 &      \\ 
     &           &    4 &  1.42e-08 &  7.05e-16 &      \\ 
     &           &    5 &  1.42e-08 &  4.05e-16 &      \\ 
     &           &    6 &  2.47e-08 &  4.04e-16 &      \\ 
     &           &    7 &  1.42e-08 &  7.05e-16 &      \\ 
     &           &    8 &  2.47e-08 &  4.05e-16 &      \\ 
     &           &    9 &  1.42e-08 &  7.04e-16 &      \\ 
     &           &   10 &  1.42e-08 &  4.05e-16 &      \\ 
2319 &  2.32e+04 &   10 &           &           & iters  \\ 
 \hdashline 
     &           &    1 &  4.15e-08 &  5.57e-10 &      \\ 
     &           &    2 &  3.62e-08 &  1.18e-15 &      \\ 
     &           &    3 &  4.15e-08 &  1.03e-15 &      \\ 
     &           &    4 &  3.63e-08 &  1.18e-15 &      \\ 
     &           &    5 &  4.15e-08 &  1.03e-15 &      \\ 
     &           &    6 &  3.63e-08 &  1.18e-15 &      \\ 
     &           &    7 &  3.62e-08 &  1.03e-15 &      \\ 
     &           &    8 &  4.15e-08 &  1.03e-15 &      \\ 
     &           &    9 &  3.62e-08 &  1.19e-15 &      \\ 
     &           &   10 &  4.15e-08 &  1.03e-15 &      \\ 
2320 &  2.32e+04 &   10 &           &           & iters  \\ 
 \hdashline 
     &           &    1 &  3.26e-08 &  1.94e-10 &      \\ 
     &           &    2 &  4.51e-08 &  9.30e-16 &      \\ 
     &           &    3 &  3.26e-08 &  1.29e-15 &      \\ 
     &           &    4 &  4.51e-08 &  9.31e-16 &      \\ 
     &           &    5 &  3.26e-08 &  1.29e-15 &      \\ 
     &           &    6 &  3.26e-08 &  9.31e-16 &      \\ 
     &           &    7 &  4.52e-08 &  9.30e-16 &      \\ 
     &           &    8 &  3.26e-08 &  1.29e-15 &      \\ 
     &           &    9 &  4.51e-08 &  9.30e-16 &      \\ 
     &           &   10 &  3.26e-08 &  1.29e-15 &      \\ 
2321 &  2.32e+04 &   10 &           &           & iters  \\ 
 \hdashline 
     &           &    1 &  9.77e-09 &  8.21e-10 &      \\ 
     &           &    2 &  9.76e-09 &  2.79e-16 &      \\ 
     &           &    3 &  2.91e-08 &  2.79e-16 &      \\ 
     &           &    4 &  9.79e-09 &  8.30e-16 &      \\ 
     &           &    5 &  9.77e-09 &  2.79e-16 &      \\ 
     &           &    6 &  9.76e-09 &  2.79e-16 &      \\ 
     &           &    7 &  2.91e-08 &  2.79e-16 &      \\ 
     &           &    8 &  9.79e-09 &  8.30e-16 &      \\ 
     &           &    9 &  9.77e-09 &  2.79e-16 &      \\ 
     &           &   10 &  9.76e-09 &  2.79e-16 &      \\ 
2322 &  2.32e+04 &   10 &           &           & iters  \\ 
 \hdashline 
     &           &    1 &  7.51e-09 &  1.54e-09 &      \\ 
     &           &    2 &  7.50e-09 &  2.14e-16 &      \\ 
     &           &    3 &  7.49e-09 &  2.14e-16 &      \\ 
     &           &    4 &  7.48e-09 &  2.14e-16 &      \\ 
     &           &    5 &  7.46e-09 &  2.13e-16 &      \\ 
     &           &    6 &  3.14e-08 &  2.13e-16 &      \\ 
     &           &    7 &  7.50e-09 &  8.96e-16 &      \\ 
     &           &    8 &  7.49e-09 &  2.14e-16 &      \\ 
     &           &    9 &  7.48e-09 &  2.14e-16 &      \\ 
     &           &   10 &  7.47e-09 &  2.13e-16 &      \\ 
2323 &  2.32e+04 &   10 &           &           & iters  \\ 
 \hdashline 
     &           &    1 &  2.89e-08 &  2.64e-09 &      \\ 
     &           &    2 &  2.88e-08 &  8.24e-16 &      \\ 
     &           &    3 &  4.89e-08 &  8.23e-16 &      \\ 
     &           &    4 &  2.89e-08 &  1.40e-15 &      \\ 
     &           &    5 &  4.89e-08 &  8.24e-16 &      \\ 
     &           &    6 &  2.89e-08 &  1.39e-15 &      \\ 
     &           &    7 &  2.88e-08 &  8.24e-16 &      \\ 
     &           &    8 &  4.89e-08 &  8.23e-16 &      \\ 
     &           &    9 &  2.89e-08 &  1.40e-15 &      \\ 
     &           &   10 &  2.88e-08 &  8.24e-16 &      \\ 
2324 &  2.32e+04 &   10 &           &           & iters  \\ 
 \hdashline 
     &           &    1 &  5.47e-08 &  2.53e-09 &      \\ 
     &           &    2 &  2.31e-08 &  1.56e-15 &      \\ 
     &           &    3 &  2.31e-08 &  6.59e-16 &      \\ 
     &           &    4 &  5.46e-08 &  6.58e-16 &      \\ 
     &           &    5 &  2.31e-08 &  1.56e-15 &      \\ 
     &           &    6 &  2.31e-08 &  6.60e-16 &      \\ 
     &           &    7 &  2.31e-08 &  6.59e-16 &      \\ 
     &           &    8 &  5.47e-08 &  6.58e-16 &      \\ 
     &           &    9 &  2.31e-08 &  1.56e-15 &      \\ 
     &           &   10 &  2.31e-08 &  6.59e-16 &      \\ 
2325 &  2.32e+04 &   10 &           &           & iters  \\ 
 \hdashline 
     &           &    1 &  4.03e-08 &  2.72e-09 &      \\ 
     &           &    2 &  3.74e-08 &  1.15e-15 &      \\ 
     &           &    3 &  4.03e-08 &  1.07e-15 &      \\ 
     &           &    4 &  3.74e-08 &  1.15e-15 &      \\ 
     &           &    5 &  4.03e-08 &  1.07e-15 &      \\ 
     &           &    6 &  3.74e-08 &  1.15e-15 &      \\ 
     &           &    7 &  3.74e-08 &  1.07e-15 &      \\ 
     &           &    8 &  4.04e-08 &  1.07e-15 &      \\ 
     &           &    9 &  3.74e-08 &  1.15e-15 &      \\ 
     &           &   10 &  4.04e-08 &  1.07e-15 &      \\ 
2326 &  2.32e+04 &   10 &           &           & iters  \\ 
 \hdashline 
     &           &    1 &  9.60e-09 &  3.49e-09 &      \\ 
     &           &    2 &  9.59e-09 &  2.74e-16 &      \\ 
     &           &    3 &  9.58e-09 &  2.74e-16 &      \\ 
     &           &    4 &  9.56e-09 &  2.73e-16 &      \\ 
     &           &    5 &  9.55e-09 &  2.73e-16 &      \\ 
     &           &    6 &  9.53e-09 &  2.72e-16 &      \\ 
     &           &    7 &  6.82e-08 &  2.72e-16 &      \\ 
     &           &    8 &  8.73e-08 &  1.95e-15 &      \\ 
     &           &    9 &  6.82e-08 &  2.49e-15 &      \\ 
     &           &   10 &  9.59e-09 &  1.95e-15 &      \\ 
2327 &  2.33e+04 &   10 &           &           & iters  \\ 
 \hdashline 
     &           &    1 &  1.01e-08 &  2.39e-09 &      \\ 
     &           &    2 &  1.00e-08 &  2.87e-16 &      \\ 
     &           &    3 &  1.00e-08 &  2.87e-16 &      \\ 
     &           &    4 &  6.77e-08 &  2.86e-16 &      \\ 
     &           &    5 &  1.01e-08 &  1.93e-15 &      \\ 
     &           &    6 &  1.01e-08 &  2.89e-16 &      \\ 
     &           &    7 &  1.01e-08 &  2.88e-16 &      \\ 
     &           &    8 &  1.01e-08 &  2.88e-16 &      \\ 
     &           &    9 &  1.01e-08 &  2.87e-16 &      \\ 
     &           &   10 &  1.00e-08 &  2.87e-16 &      \\ 
2328 &  2.33e+04 &   10 &           &           & iters  \\ 
 \hdashline 
     &           &    1 &  1.78e-08 &  1.24e-09 &      \\ 
     &           &    2 &  1.78e-08 &  5.08e-16 &      \\ 
     &           &    3 &  1.78e-08 &  5.07e-16 &      \\ 
     &           &    4 &  6.00e-08 &  5.07e-16 &      \\ 
     &           &    5 &  1.78e-08 &  1.71e-15 &      \\ 
     &           &    6 &  1.78e-08 &  5.08e-16 &      \\ 
     &           &    7 &  1.78e-08 &  5.08e-16 &      \\ 
     &           &    8 &  6.00e-08 &  5.07e-16 &      \\ 
     &           &    9 &  1.78e-08 &  1.71e-15 &      \\ 
     &           &   10 &  1.78e-08 &  5.09e-16 &      \\ 
2329 &  2.33e+04 &   10 &           &           & iters  \\ 
 \hdashline 
     &           &    1 &  7.82e-09 &  1.53e-09 &      \\ 
     &           &    2 &  7.81e-09 &  2.23e-16 &      \\ 
     &           &    3 &  7.80e-09 &  2.23e-16 &      \\ 
     &           &    4 &  7.79e-09 &  2.23e-16 &      \\ 
     &           &    5 &  7.78e-09 &  2.22e-16 &      \\ 
     &           &    6 &  7.77e-09 &  2.22e-16 &      \\ 
     &           &    7 &  7.76e-09 &  2.22e-16 &      \\ 
     &           &    8 &  7.75e-09 &  2.21e-16 &      \\ 
     &           &    9 &  7.00e-08 &  2.21e-16 &      \\ 
     &           &   10 &  7.84e-09 &  2.00e-15 &      \\ 
2330 &  2.33e+04 &   10 &           &           & iters  \\ 
 \hdashline 
     &           &    1 &  3.71e-08 &  1.20e-09 &      \\ 
     &           &    2 &  4.07e-08 &  1.06e-15 &      \\ 
     &           &    3 &  3.71e-08 &  1.16e-15 &      \\ 
     &           &    4 &  4.07e-08 &  1.06e-15 &      \\ 
     &           &    5 &  3.71e-08 &  1.16e-15 &      \\ 
     &           &    6 &  4.07e-08 &  1.06e-15 &      \\ 
     &           &    7 &  3.71e-08 &  1.16e-15 &      \\ 
     &           &    8 &  4.06e-08 &  1.06e-15 &      \\ 
     &           &    9 &  3.71e-08 &  1.16e-15 &      \\ 
     &           &   10 &  4.06e-08 &  1.06e-15 &      \\ 
2331 &  2.33e+04 &   10 &           &           & iters  \\ 
 \hdashline 
     &           &    1 &  4.04e-08 &  1.12e-10 &      \\ 
     &           &    2 &  3.74e-08 &  1.15e-15 &      \\ 
     &           &    3 &  4.04e-08 &  1.07e-15 &      \\ 
     &           &    4 &  3.74e-08 &  1.15e-15 &      \\ 
     &           &    5 &  3.73e-08 &  1.07e-15 &      \\ 
     &           &    6 &  4.04e-08 &  1.06e-15 &      \\ 
     &           &    7 &  3.73e-08 &  1.15e-15 &      \\ 
     &           &    8 &  4.04e-08 &  1.06e-15 &      \\ 
     &           &    9 &  3.73e-08 &  1.15e-15 &      \\ 
     &           &   10 &  4.04e-08 &  1.07e-15 &      \\ 
2332 &  2.33e+04 &   10 &           &           & iters  \\ 
 \hdashline 
     &           &    1 &  4.76e-08 &  8.06e-10 &      \\ 
     &           &    2 &  3.02e-08 &  1.36e-15 &      \\ 
     &           &    3 &  3.01e-08 &  8.62e-16 &      \\ 
     &           &    4 &  4.76e-08 &  8.60e-16 &      \\ 
     &           &    5 &  3.02e-08 &  1.36e-15 &      \\ 
     &           &    6 &  3.01e-08 &  8.61e-16 &      \\ 
     &           &    7 &  4.76e-08 &  8.60e-16 &      \\ 
     &           &    8 &  3.02e-08 &  1.36e-15 &      \\ 
     &           &    9 &  4.76e-08 &  8.61e-16 &      \\ 
     &           &   10 &  3.02e-08 &  1.36e-15 &      \\ 
2333 &  2.33e+04 &   10 &           &           & iters  \\ 
 \hdashline 
     &           &    1 &  4.80e-08 &  2.75e-09 &      \\ 
     &           &    2 &  2.97e-08 &  1.37e-15 &      \\ 
     &           &    3 &  4.80e-08 &  8.49e-16 &      \\ 
     &           &    4 &  2.98e-08 &  1.37e-15 &      \\ 
     &           &    5 &  2.97e-08 &  8.50e-16 &      \\ 
     &           &    6 &  4.80e-08 &  8.49e-16 &      \\ 
     &           &    7 &  2.98e-08 &  1.37e-15 &      \\ 
     &           &    8 &  2.97e-08 &  8.49e-16 &      \\ 
     &           &    9 &  4.80e-08 &  8.48e-16 &      \\ 
     &           &   10 &  2.97e-08 &  1.37e-15 &      \\ 
2334 &  2.33e+04 &   10 &           &           & iters  \\ 
 \hdashline 
     &           &    1 &  9.43e-10 &  4.23e-09 &      \\ 
     &           &    2 &  9.42e-10 &  2.69e-17 &      \\ 
     &           &    3 &  9.41e-10 &  2.69e-17 &      \\ 
     &           &    4 &  9.39e-10 &  2.68e-17 &      \\ 
     &           &    5 &  9.38e-10 &  2.68e-17 &      \\ 
     &           &    6 &  9.36e-10 &  2.68e-17 &      \\ 
     &           &    7 &  9.35e-10 &  2.67e-17 &      \\ 
     &           &    8 &  9.34e-10 &  2.67e-17 &      \\ 
     &           &    9 &  9.32e-10 &  2.67e-17 &      \\ 
     &           &   10 &  9.31e-10 &  2.66e-17 &      \\ 
2335 &  2.33e+04 &   10 &           &           & iters  \\ 
 \hdashline 
     &           &    1 &  5.00e-09 &  2.96e-09 &      \\ 
     &           &    2 &  4.99e-09 &  1.43e-16 &      \\ 
     &           &    3 &  4.99e-09 &  1.42e-16 &      \\ 
     &           &    4 &  4.98e-09 &  1.42e-16 &      \\ 
     &           &    5 &  4.97e-09 &  1.42e-16 &      \\ 
     &           &    6 &  4.96e-09 &  1.42e-16 &      \\ 
     &           &    7 &  4.96e-09 &  1.42e-16 &      \\ 
     &           &    8 &  4.95e-09 &  1.41e-16 &      \\ 
     &           &    9 &  4.94e-09 &  1.41e-16 &      \\ 
     &           &   10 &  4.94e-09 &  1.41e-16 &      \\ 
2336 &  2.34e+04 &   10 &           &           & iters  \\ 
 \hdashline 
     &           &    1 &  5.33e-09 &  1.91e-09 &      \\ 
     &           &    2 &  5.32e-09 &  1.52e-16 &      \\ 
     &           &    3 &  5.31e-09 &  1.52e-16 &      \\ 
     &           &    4 &  5.30e-09 &  1.52e-16 &      \\ 
     &           &    5 &  5.30e-09 &  1.51e-16 &      \\ 
     &           &    6 &  5.29e-09 &  1.51e-16 &      \\ 
     &           &    7 &  5.28e-09 &  1.51e-16 &      \\ 
     &           &    8 &  5.27e-09 &  1.51e-16 &      \\ 
     &           &    9 &  5.27e-09 &  1.50e-16 &      \\ 
     &           &   10 &  5.26e-09 &  1.50e-16 &      \\ 
2337 &  2.34e+04 &   10 &           &           & iters  \\ 
 \hdashline 
     &           &    1 &  4.59e-08 &  2.17e-09 &      \\ 
     &           &    2 &  3.19e-08 &  1.31e-15 &      \\ 
     &           &    3 &  3.19e-08 &  9.10e-16 &      \\ 
     &           &    4 &  4.59e-08 &  9.09e-16 &      \\ 
     &           &    5 &  3.19e-08 &  1.31e-15 &      \\ 
     &           &    6 &  4.59e-08 &  9.10e-16 &      \\ 
     &           &    7 &  3.19e-08 &  1.31e-15 &      \\ 
     &           &    8 &  3.18e-08 &  9.10e-16 &      \\ 
     &           &    9 &  4.59e-08 &  9.09e-16 &      \\ 
     &           &   10 &  3.19e-08 &  1.31e-15 &      \\ 
2338 &  2.34e+04 &   10 &           &           & iters  \\ 
 \hdashline 
     &           &    1 &  3.53e-08 &  1.87e-09 &      \\ 
     &           &    2 &  3.62e-09 &  1.01e-15 &      \\ 
     &           &    3 &  3.62e-09 &  1.03e-16 &      \\ 
     &           &    4 &  3.61e-09 &  1.03e-16 &      \\ 
     &           &    5 &  3.61e-09 &  1.03e-16 &      \\ 
     &           &    6 &  3.60e-09 &  1.03e-16 &      \\ 
     &           &    7 &  3.60e-09 &  1.03e-16 &      \\ 
     &           &    8 &  3.59e-09 &  1.03e-16 &      \\ 
     &           &    9 &  3.53e-08 &  1.03e-16 &      \\ 
     &           &   10 &  3.64e-09 &  1.01e-15 &      \\ 
2339 &  2.34e+04 &   10 &           &           & iters  \\ 
 \hdashline 
     &           &    1 &  9.39e-09 &  1.81e-09 &      \\ 
     &           &    2 &  9.37e-09 &  2.68e-16 &      \\ 
     &           &    3 &  9.36e-09 &  2.68e-16 &      \\ 
     &           &    4 &  9.35e-09 &  2.67e-16 &      \\ 
     &           &    5 &  9.33e-09 &  2.67e-16 &      \\ 
     &           &    6 &  9.32e-09 &  2.66e-16 &      \\ 
     &           &    7 &  6.84e-08 &  2.66e-16 &      \\ 
     &           &    8 &  9.40e-09 &  1.95e-15 &      \\ 
     &           &    9 &  9.39e-09 &  2.68e-16 &      \\ 
     &           &   10 &  9.38e-09 &  2.68e-16 &      \\ 
2340 &  2.34e+04 &   10 &           &           & iters  \\ 
 \hdashline 
     &           &    1 &  3.03e-08 &  1.18e-09 &      \\ 
     &           &    2 &  4.74e-08 &  8.66e-16 &      \\ 
     &           &    3 &  3.04e-08 &  1.35e-15 &      \\ 
     &           &    4 &  4.74e-08 &  8.67e-16 &      \\ 
     &           &    5 &  3.04e-08 &  1.35e-15 &      \\ 
     &           &    6 &  3.03e-08 &  8.67e-16 &      \\ 
     &           &    7 &  4.74e-08 &  8.66e-16 &      \\ 
     &           &    8 &  3.04e-08 &  1.35e-15 &      \\ 
     &           &    9 &  3.03e-08 &  8.67e-16 &      \\ 
     &           &   10 &  4.74e-08 &  8.65e-16 &      \\ 
2341 &  2.34e+04 &   10 &           &           & iters  \\ 
 \hdashline 
     &           &    1 &  3.84e-08 &  4.33e-10 &      \\ 
     &           &    2 &  3.93e-08 &  1.10e-15 &      \\ 
     &           &    3 &  3.84e-08 &  1.12e-15 &      \\ 
     &           &    4 &  3.93e-08 &  1.10e-15 &      \\ 
     &           &    5 &  3.84e-08 &  1.12e-15 &      \\ 
     &           &    6 &  3.93e-08 &  1.10e-15 &      \\ 
     &           &    7 &  3.84e-08 &  1.12e-15 &      \\ 
     &           &    8 &  3.93e-08 &  1.10e-15 &      \\ 
     &           &    9 &  3.84e-08 &  1.12e-15 &      \\ 
     &           &   10 &  3.93e-08 &  1.10e-15 &      \\ 
2342 &  2.34e+04 &   10 &           &           & iters  \\ 
 \hdashline 
     &           &    1 &  5.50e-09 &  1.42e-10 &      \\ 
     &           &    2 &  5.49e-09 &  1.57e-16 &      \\ 
     &           &    3 &  5.48e-09 &  1.57e-16 &      \\ 
     &           &    4 &  5.47e-09 &  1.56e-16 &      \\ 
     &           &    5 &  7.22e-08 &  1.56e-16 &      \\ 
     &           &    6 &  8.33e-08 &  2.06e-15 &      \\ 
     &           &    7 &  7.22e-08 &  2.38e-15 &      \\ 
     &           &    8 &  5.55e-09 &  2.06e-15 &      \\ 
     &           &    9 &  5.55e-09 &  1.59e-16 &      \\ 
     &           &   10 &  5.54e-09 &  1.58e-16 &      \\ 
2343 &  2.34e+04 &   10 &           &           & iters  \\ 
 \hdashline 
     &           &    1 &  1.58e-08 &  6.38e-10 &      \\ 
     &           &    2 &  6.19e-08 &  4.51e-16 &      \\ 
     &           &    3 &  1.59e-08 &  1.77e-15 &      \\ 
     &           &    4 &  1.58e-08 &  4.52e-16 &      \\ 
     &           &    5 &  1.58e-08 &  4.52e-16 &      \\ 
     &           &    6 &  1.58e-08 &  4.51e-16 &      \\ 
     &           &    7 &  6.19e-08 &  4.51e-16 &      \\ 
     &           &    8 &  1.59e-08 &  1.77e-15 &      \\ 
     &           &    9 &  1.58e-08 &  4.52e-16 &      \\ 
     &           &   10 &  1.58e-08 &  4.52e-16 &      \\ 
2344 &  2.34e+04 &   10 &           &           & iters  \\ 
 \hdashline 
     &           &    1 &  2.29e-09 &  3.25e-10 &      \\ 
     &           &    2 &  2.29e-09 &  6.54e-17 &      \\ 
     &           &    3 &  2.29e-09 &  6.53e-17 &      \\ 
     &           &    4 &  2.28e-09 &  6.52e-17 &      \\ 
     &           &    5 &  2.28e-09 &  6.51e-17 &      \\ 
     &           &    6 &  2.28e-09 &  6.50e-17 &      \\ 
     &           &    7 &  2.27e-09 &  6.49e-17 &      \\ 
     &           &    8 &  2.27e-09 &  6.48e-17 &      \\ 
     &           &    9 &  2.27e-09 &  6.47e-17 &      \\ 
     &           &   10 &  2.26e-09 &  6.47e-17 &      \\ 
2345 &  2.34e+04 &   10 &           &           & iters  \\ 
 \hdashline 
     &           &    1 &  1.12e-08 &  6.60e-10 &      \\ 
     &           &    2 &  2.77e-08 &  3.19e-16 &      \\ 
     &           &    3 &  1.12e-08 &  7.90e-16 &      \\ 
     &           &    4 &  1.12e-08 &  3.20e-16 &      \\ 
     &           &    5 &  1.12e-08 &  3.19e-16 &      \\ 
     &           &    6 &  2.77e-08 &  3.19e-16 &      \\ 
     &           &    7 &  1.12e-08 &  7.90e-16 &      \\ 
     &           &    8 &  1.12e-08 &  3.20e-16 &      \\ 
     &           &    9 &  2.77e-08 &  3.19e-16 &      \\ 
     &           &   10 &  1.12e-08 &  7.90e-16 &      \\ 
2346 &  2.34e+04 &   10 &           &           & iters  \\ 
 \hdashline 
     &           &    1 &  3.11e-08 &  3.31e-10 &      \\ 
     &           &    2 &  4.66e-08 &  8.89e-16 &      \\ 
     &           &    3 &  3.12e-08 &  1.33e-15 &      \\ 
     &           &    4 &  4.66e-08 &  8.89e-16 &      \\ 
     &           &    5 &  3.12e-08 &  1.33e-15 &      \\ 
     &           &    6 &  3.11e-08 &  8.90e-16 &      \\ 
     &           &    7 &  4.66e-08 &  8.89e-16 &      \\ 
     &           &    8 &  3.12e-08 &  1.33e-15 &      \\ 
     &           &    9 &  4.66e-08 &  8.89e-16 &      \\ 
     &           &   10 &  3.12e-08 &  1.33e-15 &      \\ 
2347 &  2.35e+04 &   10 &           &           & iters  \\ 
 \hdashline 
     &           &    1 &  1.86e-08 &  1.00e-09 &      \\ 
     &           &    2 &  2.03e-08 &  5.31e-16 &      \\ 
     &           &    3 &  1.86e-08 &  5.78e-16 &      \\ 
     &           &    4 &  2.03e-08 &  5.31e-16 &      \\ 
     &           &    5 &  1.86e-08 &  5.78e-16 &      \\ 
     &           &    6 &  2.02e-08 &  5.31e-16 &      \\ 
     &           &    7 &  1.86e-08 &  5.78e-16 &      \\ 
     &           &    8 &  2.02e-08 &  5.32e-16 &      \\ 
     &           &    9 &  1.86e-08 &  5.78e-16 &      \\ 
     &           &   10 &  2.02e-08 &  5.32e-16 &      \\ 
2348 &  2.35e+04 &   10 &           &           & iters  \\ 
 \hdashline 
     &           &    1 &  1.60e-08 &  1.35e-09 &      \\ 
     &           &    2 &  6.17e-08 &  4.56e-16 &      \\ 
     &           &    3 &  1.61e-08 &  1.76e-15 &      \\ 
     &           &    4 &  1.60e-08 &  4.58e-16 &      \\ 
     &           &    5 &  1.60e-08 &  4.58e-16 &      \\ 
     &           &    6 &  1.60e-08 &  4.57e-16 &      \\ 
     &           &    7 &  6.17e-08 &  4.56e-16 &      \\ 
     &           &    8 &  1.61e-08 &  1.76e-15 &      \\ 
     &           &    9 &  1.60e-08 &  4.58e-16 &      \\ 
     &           &   10 &  1.60e-08 &  4.58e-16 &      \\ 
2349 &  2.35e+04 &   10 &           &           & iters  \\ 
 \hdashline 
     &           &    1 &  2.94e-08 &  1.19e-09 &      \\ 
     &           &    2 &  9.49e-09 &  8.39e-16 &      \\ 
     &           &    3 &  9.47e-09 &  2.71e-16 &      \\ 
     &           &    4 &  9.46e-09 &  2.70e-16 &      \\ 
     &           &    5 &  2.94e-08 &  2.70e-16 &      \\ 
     &           &    6 &  9.49e-09 &  8.39e-16 &      \\ 
     &           &    7 &  9.48e-09 &  2.71e-16 &      \\ 
     &           &    8 &  9.46e-09 &  2.70e-16 &      \\ 
     &           &    9 &  2.94e-08 &  2.70e-16 &      \\ 
     &           &   10 &  9.49e-09 &  8.39e-16 &      \\ 
2350 &  2.35e+04 &   10 &           &           & iters  \\ 
 \hdashline 
     &           &    1 &  3.12e-08 &  9.33e-10 &      \\ 
     &           &    2 &  4.66e-08 &  8.89e-16 &      \\ 
     &           &    3 &  3.12e-08 &  1.33e-15 &      \\ 
     &           &    4 &  4.65e-08 &  8.90e-16 &      \\ 
     &           &    5 &  3.12e-08 &  1.33e-15 &      \\ 
     &           &    6 &  3.12e-08 &  8.91e-16 &      \\ 
     &           &    7 &  4.66e-08 &  8.89e-16 &      \\ 
     &           &    8 &  3.12e-08 &  1.33e-15 &      \\ 
     &           &    9 &  4.65e-08 &  8.90e-16 &      \\ 
     &           &   10 &  3.12e-08 &  1.33e-15 &      \\ 
2351 &  2.35e+04 &   10 &           &           & iters  \\ 
 \hdashline 
     &           &    1 &  1.06e-08 &  6.66e-12 &      \\ 
     &           &    2 &  2.83e-08 &  3.02e-16 &      \\ 
     &           &    3 &  1.06e-08 &  8.07e-16 &      \\ 
     &           &    4 &  1.06e-08 &  3.03e-16 &      \\ 
     &           &    5 &  2.83e-08 &  3.02e-16 &      \\ 
     &           &    6 &  1.06e-08 &  8.07e-16 &      \\ 
     &           &    7 &  1.06e-08 &  3.03e-16 &      \\ 
     &           &    8 &  1.06e-08 &  3.03e-16 &      \\ 
     &           &    9 &  2.83e-08 &  3.02e-16 &      \\ 
     &           &   10 &  1.06e-08 &  8.07e-16 &      \\ 
2352 &  2.35e+04 &   10 &           &           & iters  \\ 
 \hdashline 
     &           &    1 &  8.86e-09 &  3.77e-10 &      \\ 
     &           &    2 &  3.00e-08 &  2.53e-16 &      \\ 
     &           &    3 &  8.89e-09 &  8.56e-16 &      \\ 
     &           &    4 &  8.88e-09 &  2.54e-16 &      \\ 
     &           &    5 &  8.87e-09 &  2.53e-16 &      \\ 
     &           &    6 &  3.00e-08 &  2.53e-16 &      \\ 
     &           &    7 &  8.90e-09 &  8.56e-16 &      \\ 
     &           &    8 &  8.89e-09 &  2.54e-16 &      \\ 
     &           &    9 &  8.87e-09 &  2.54e-16 &      \\ 
     &           &   10 &  8.86e-09 &  2.53e-16 &      \\ 
2353 &  2.35e+04 &   10 &           &           & iters  \\ 
 \hdashline 
     &           &    1 &  4.39e-08 &  8.73e-11 &      \\ 
     &           &    2 &  3.39e-08 &  1.25e-15 &      \\ 
     &           &    3 &  3.38e-08 &  9.67e-16 &      \\ 
     &           &    4 &  4.39e-08 &  9.65e-16 &      \\ 
     &           &    5 &  3.38e-08 &  1.25e-15 &      \\ 
     &           &    6 &  4.39e-08 &  9.66e-16 &      \\ 
     &           &    7 &  3.39e-08 &  1.25e-15 &      \\ 
     &           &    8 &  4.39e-08 &  9.66e-16 &      \\ 
     &           &    9 &  3.39e-08 &  1.25e-15 &      \\ 
     &           &   10 &  4.39e-08 &  9.67e-16 &      \\ 
2354 &  2.35e+04 &   10 &           &           & iters  \\ 
 \hdashline 
     &           &    1 &  3.44e-08 &  9.72e-12 &      \\ 
     &           &    2 &  4.47e-09 &  9.82e-16 &      \\ 
     &           &    3 &  4.47e-09 &  1.28e-16 &      \\ 
     &           &    4 &  3.44e-08 &  1.28e-16 &      \\ 
     &           &    5 &  4.51e-09 &  9.81e-16 &      \\ 
     &           &    6 &  4.50e-09 &  1.29e-16 &      \\ 
     &           &    7 &  4.50e-09 &  1.29e-16 &      \\ 
     &           &    8 &  4.49e-09 &  1.28e-16 &      \\ 
     &           &    9 &  4.48e-09 &  1.28e-16 &      \\ 
     &           &   10 &  4.48e-09 &  1.28e-16 &      \\ 
2355 &  2.35e+04 &   10 &           &           & iters  \\ 
 \hdashline 
     &           &    1 &  3.47e-08 &  2.87e-10 &      \\ 
     &           &    2 &  4.30e-08 &  9.91e-16 &      \\ 
     &           &    3 &  3.47e-08 &  1.23e-15 &      \\ 
     &           &    4 &  4.30e-08 &  9.91e-16 &      \\ 
     &           &    5 &  3.47e-08 &  1.23e-15 &      \\ 
     &           &    6 &  4.30e-08 &  9.91e-16 &      \\ 
     &           &    7 &  3.48e-08 &  1.23e-15 &      \\ 
     &           &    8 &  4.30e-08 &  9.92e-16 &      \\ 
     &           &    9 &  3.48e-08 &  1.23e-15 &      \\ 
     &           &   10 &  3.47e-08 &  9.92e-16 &      \\ 
2356 &  2.36e+04 &   10 &           &           & iters  \\ 
 \hdashline 
     &           &    1 &  4.79e-08 &  3.97e-10 &      \\ 
     &           &    2 &  2.99e-08 &  1.37e-15 &      \\ 
     &           &    3 &  4.79e-08 &  8.53e-16 &      \\ 
     &           &    4 &  2.99e-08 &  1.37e-15 &      \\ 
     &           &    5 &  4.78e-08 &  8.54e-16 &      \\ 
     &           &    6 &  2.99e-08 &  1.36e-15 &      \\ 
     &           &    7 &  2.99e-08 &  8.54e-16 &      \\ 
     &           &    8 &  4.78e-08 &  8.53e-16 &      \\ 
     &           &    9 &  2.99e-08 &  1.37e-15 &      \\ 
     &           &   10 &  2.99e-08 &  8.54e-16 &      \\ 
2357 &  2.36e+04 &   10 &           &           & iters  \\ 
 \hdashline 
     &           &    1 &  2.34e-08 &  5.22e-12 &      \\ 
     &           &    2 &  2.33e-08 &  6.66e-16 &      \\ 
     &           &    3 &  5.44e-08 &  6.66e-16 &      \\ 
     &           &    4 &  2.34e-08 &  1.55e-15 &      \\ 
     &           &    5 &  2.33e-08 &  6.67e-16 &      \\ 
     &           &    6 &  2.33e-08 &  6.66e-16 &      \\ 
     &           &    7 &  5.44e-08 &  6.65e-16 &      \\ 
     &           &    8 &  2.33e-08 &  1.55e-15 &      \\ 
     &           &    9 &  2.33e-08 &  6.66e-16 &      \\ 
     &           &   10 &  5.44e-08 &  6.65e-16 &      \\ 
2358 &  2.36e+04 &   10 &           &           & iters  \\ 
 \hdashline 
     &           &    1 &  1.12e-08 &  3.66e-10 &      \\ 
     &           &    2 &  1.12e-08 &  3.19e-16 &      \\ 
     &           &    3 &  1.12e-08 &  3.19e-16 &      \\ 
     &           &    4 &  1.11e-08 &  3.18e-16 &      \\ 
     &           &    5 &  2.77e-08 &  3.18e-16 &      \\ 
     &           &    6 &  1.12e-08 &  7.91e-16 &      \\ 
     &           &    7 &  1.12e-08 &  3.19e-16 &      \\ 
     &           &    8 &  1.11e-08 &  3.18e-16 &      \\ 
     &           &    9 &  2.77e-08 &  3.18e-16 &      \\ 
     &           &   10 &  1.12e-08 &  7.91e-16 &      \\ 
2359 &  2.36e+04 &   10 &           &           & iters  \\ 
 \hdashline 
     &           &    1 &  6.57e-09 &  4.51e-10 &      \\ 
     &           &    2 &  6.56e-09 &  1.88e-16 &      \\ 
     &           &    3 &  6.55e-09 &  1.87e-16 &      \\ 
     &           &    4 &  6.54e-09 &  1.87e-16 &      \\ 
     &           &    5 &  7.12e-08 &  1.87e-16 &      \\ 
     &           &    6 &  8.43e-08 &  2.03e-15 &      \\ 
     &           &    7 &  7.12e-08 &  2.41e-15 &      \\ 
     &           &    8 &  6.62e-09 &  2.03e-15 &      \\ 
     &           &    9 &  6.61e-09 &  1.89e-16 &      \\ 
     &           &   10 &  6.60e-09 &  1.89e-16 &      \\ 
2360 &  2.36e+04 &   10 &           &           & iters  \\ 
 \hdashline 
     &           &    1 &  3.65e-08 &  6.83e-10 &      \\ 
     &           &    2 &  4.12e-08 &  1.04e-15 &      \\ 
     &           &    3 &  3.65e-08 &  1.18e-15 &      \\ 
     &           &    4 &  4.12e-08 &  1.04e-15 &      \\ 
     &           &    5 &  3.66e-08 &  1.18e-15 &      \\ 
     &           &    6 &  4.12e-08 &  1.04e-15 &      \\ 
     &           &    7 &  3.66e-08 &  1.18e-15 &      \\ 
     &           &    8 &  4.12e-08 &  1.04e-15 &      \\ 
     &           &    9 &  3.66e-08 &  1.18e-15 &      \\ 
     &           &   10 &  4.12e-08 &  1.04e-15 &      \\ 
2361 &  2.36e+04 &   10 &           &           & iters  \\ 
 \hdashline 
     &           &    1 &  3.78e-08 &  8.46e-10 &      \\ 
     &           &    2 &  3.99e-08 &  1.08e-15 &      \\ 
     &           &    3 &  3.78e-08 &  1.14e-15 &      \\ 
     &           &    4 &  3.99e-08 &  1.08e-15 &      \\ 
     &           &    5 &  3.78e-08 &  1.14e-15 &      \\ 
     &           &    6 &  3.99e-08 &  1.08e-15 &      \\ 
     &           &    7 &  3.78e-08 &  1.14e-15 &      \\ 
     &           &    8 &  3.99e-08 &  1.08e-15 &      \\ 
     &           &    9 &  3.78e-08 &  1.14e-15 &      \\ 
     &           &   10 &  3.99e-08 &  1.08e-15 &      \\ 
2362 &  2.36e+04 &   10 &           &           & iters  \\ 
 \hdashline 
     &           &    1 &  5.47e-08 &  7.29e-10 &      \\ 
     &           &    2 &  2.31e-08 &  1.56e-15 &      \\ 
     &           &    3 &  5.47e-08 &  6.58e-16 &      \\ 
     &           &    4 &  2.31e-08 &  1.56e-15 &      \\ 
     &           &    5 &  2.31e-08 &  6.59e-16 &      \\ 
     &           &    6 &  2.30e-08 &  6.58e-16 &      \\ 
     &           &    7 &  5.47e-08 &  6.57e-16 &      \\ 
     &           &    8 &  2.31e-08 &  1.56e-15 &      \\ 
     &           &    9 &  2.30e-08 &  6.59e-16 &      \\ 
     &           &   10 &  5.47e-08 &  6.58e-16 &      \\ 
2363 &  2.36e+04 &   10 &           &           & iters  \\ 
 \hdashline 
     &           &    1 &  3.41e-08 &  3.87e-10 &      \\ 
     &           &    2 &  3.40e-08 &  9.72e-16 &      \\ 
     &           &    3 &  4.37e-08 &  9.71e-16 &      \\ 
     &           &    4 &  3.40e-08 &  1.25e-15 &      \\ 
     &           &    5 &  4.37e-08 &  9.71e-16 &      \\ 
     &           &    6 &  3.40e-08 &  1.25e-15 &      \\ 
     &           &    7 &  3.40e-08 &  9.72e-16 &      \\ 
     &           &    8 &  4.37e-08 &  9.70e-16 &      \\ 
     &           &    9 &  3.40e-08 &  1.25e-15 &      \\ 
     &           &   10 &  4.37e-08 &  9.71e-16 &      \\ 
2364 &  2.36e+04 &   10 &           &           & iters  \\ 
 \hdashline 
     &           &    1 &  2.78e-08 &  6.75e-10 &      \\ 
     &           &    2 &  4.99e-08 &  7.93e-16 &      \\ 
     &           &    3 &  2.78e-08 &  1.43e-15 &      \\ 
     &           &    4 &  2.78e-08 &  7.94e-16 &      \\ 
     &           &    5 &  4.99e-08 &  7.93e-16 &      \\ 
     &           &    6 &  2.78e-08 &  1.43e-15 &      \\ 
     &           &    7 &  2.78e-08 &  7.94e-16 &      \\ 
     &           &    8 &  5.00e-08 &  7.93e-16 &      \\ 
     &           &    9 &  2.78e-08 &  1.43e-15 &      \\ 
     &           &   10 &  2.78e-08 &  7.94e-16 &      \\ 
2365 &  2.36e+04 &   10 &           &           & iters  \\ 
 \hdashline 
     &           &    1 &  4.68e-08 &  1.30e-09 &      \\ 
     &           &    2 &  7.85e-09 &  1.33e-15 &      \\ 
     &           &    3 &  7.84e-09 &  2.24e-16 &      \\ 
     &           &    4 &  7.83e-09 &  2.24e-16 &      \\ 
     &           &    5 &  7.82e-09 &  2.23e-16 &      \\ 
     &           &    6 &  6.99e-08 &  2.23e-16 &      \\ 
     &           &    7 &  4.68e-08 &  1.99e-15 &      \\ 
     &           &    8 &  7.84e-09 &  1.33e-15 &      \\ 
     &           &    9 &  7.83e-09 &  2.24e-16 &      \\ 
     &           &   10 &  7.82e-09 &  2.23e-16 &      \\ 
2366 &  2.36e+04 &   10 &           &           & iters  \\ 
 \hdashline 
     &           &    1 &  4.72e-08 &  1.55e-09 &      \\ 
     &           &    2 &  8.33e-09 &  1.35e-15 &      \\ 
     &           &    3 &  8.32e-09 &  2.38e-16 &      \\ 
     &           &    4 &  8.31e-09 &  2.37e-16 &      \\ 
     &           &    5 &  8.29e-09 &  2.37e-16 &      \\ 
     &           &    6 &  6.94e-08 &  2.37e-16 &      \\ 
     &           &    7 &  4.72e-08 &  1.98e-15 &      \\ 
     &           &    8 &  8.31e-09 &  1.35e-15 &      \\ 
     &           &    9 &  8.30e-09 &  2.37e-16 &      \\ 
     &           &   10 &  6.94e-08 &  2.37e-16 &      \\ 
2367 &  2.37e+04 &   10 &           &           & iters  \\ 
 \hdashline 
     &           &    1 &  3.51e-08 &  1.85e-09 &      \\ 
     &           &    2 &  3.51e-08 &  1.00e-15 &      \\ 
     &           &    3 &  4.27e-08 &  1.00e-15 &      \\ 
     &           &    4 &  3.51e-08 &  1.22e-15 &      \\ 
     &           &    5 &  3.50e-08 &  1.00e-15 &      \\ 
     &           &    6 &  4.27e-08 &  1.00e-15 &      \\ 
     &           &    7 &  3.51e-08 &  1.22e-15 &      \\ 
     &           &    8 &  4.27e-08 &  1.00e-15 &      \\ 
     &           &    9 &  3.51e-08 &  1.22e-15 &      \\ 
     &           &   10 &  4.27e-08 &  1.00e-15 &      \\ 
2368 &  2.37e+04 &   10 &           &           & iters  \\ 
 \hdashline 
     &           &    1 &  9.11e-09 &  8.17e-10 &      \\ 
     &           &    2 &  9.10e-09 &  2.60e-16 &      \\ 
     &           &    3 &  9.08e-09 &  2.60e-16 &      \\ 
     &           &    4 &  9.07e-09 &  2.59e-16 &      \\ 
     &           &    5 &  9.06e-09 &  2.59e-16 &      \\ 
     &           &    6 &  9.04e-09 &  2.58e-16 &      \\ 
     &           &    7 &  9.03e-09 &  2.58e-16 &      \\ 
     &           &    8 &  6.87e-08 &  2.58e-16 &      \\ 
     &           &    9 &  9.12e-09 &  1.96e-15 &      \\ 
     &           &   10 &  9.10e-09 &  2.60e-16 &      \\ 
2369 &  2.37e+04 &   10 &           &           & iters  \\ 
 \hdashline 
     &           &    1 &  1.06e-09 &  4.07e-11 &      \\ 
     &           &    2 &  1.05e-09 &  3.01e-17 &      \\ 
     &           &    3 &  1.05e-09 &  3.01e-17 &      \\ 
     &           &    4 &  1.05e-09 &  3.00e-17 &      \\ 
     &           &    5 &  1.05e-09 &  3.00e-17 &      \\ 
     &           &    6 &  1.05e-09 &  3.00e-17 &      \\ 
     &           &    7 &  1.05e-09 &  2.99e-17 &      \\ 
     &           &    8 &  1.05e-09 &  2.99e-17 &      \\ 
     &           &    9 &  1.04e-09 &  2.98e-17 &      \\ 
     &           &   10 &  1.04e-09 &  2.98e-17 &      \\ 
2370 &  2.37e+04 &   10 &           &           & iters  \\ 
 \hdashline 
     &           &    1 &  1.23e-08 &  6.40e-11 &      \\ 
     &           &    2 &  1.23e-08 &  3.51e-16 &      \\ 
     &           &    3 &  1.23e-08 &  3.50e-16 &      \\ 
     &           &    4 &  1.22e-08 &  3.50e-16 &      \\ 
     &           &    5 &  1.22e-08 &  3.49e-16 &      \\ 
     &           &    6 &  6.55e-08 &  3.49e-16 &      \\ 
     &           &    7 &  1.23e-08 &  1.87e-15 &      \\ 
     &           &    8 &  1.23e-08 &  3.51e-16 &      \\ 
     &           &    9 &  1.23e-08 &  3.50e-16 &      \\ 
     &           &   10 &  1.22e-08 &  3.50e-16 &      \\ 
2371 &  2.37e+04 &   10 &           &           & iters  \\ 
 \hdashline 
     &           &    1 &  3.99e-08 &  3.76e-11 &      \\ 
     &           &    2 &  3.78e-08 &  1.14e-15 &      \\ 
     &           &    3 &  3.99e-08 &  1.08e-15 &      \\ 
     &           &    4 &  3.78e-08 &  1.14e-15 &      \\ 
     &           &    5 &  3.99e-08 &  1.08e-15 &      \\ 
     &           &    6 &  3.78e-08 &  1.14e-15 &      \\ 
     &           &    7 &  3.99e-08 &  1.08e-15 &      \\ 
     &           &    8 &  3.78e-08 &  1.14e-15 &      \\ 
     &           &    9 &  3.99e-08 &  1.08e-15 &      \\ 
     &           &   10 &  3.78e-08 &  1.14e-15 &      \\ 
2372 &  2.37e+04 &   10 &           &           & iters  \\ 
 \hdashline 
     &           &    1 &  5.96e-09 &  3.37e-10 &      \\ 
     &           &    2 &  5.95e-09 &  1.70e-16 &      \\ 
     &           &    3 &  5.95e-09 &  1.70e-16 &      \\ 
     &           &    4 &  5.94e-09 &  1.70e-16 &      \\ 
     &           &    5 &  5.93e-09 &  1.69e-16 &      \\ 
     &           &    6 &  5.92e-09 &  1.69e-16 &      \\ 
     &           &    7 &  5.91e-09 &  1.69e-16 &      \\ 
     &           &    8 &  5.90e-09 &  1.69e-16 &      \\ 
     &           &    9 &  5.89e-09 &  1.68e-16 &      \\ 
     &           &   10 &  5.89e-09 &  1.68e-16 &      \\ 
2373 &  2.37e+04 &   10 &           &           & iters  \\ 
 \hdashline 
     &           &    1 &  7.26e-09 &  1.07e-09 &      \\ 
     &           &    2 &  7.25e-09 &  2.07e-16 &      \\ 
     &           &    3 &  7.24e-09 &  2.07e-16 &      \\ 
     &           &    4 &  7.23e-09 &  2.07e-16 &      \\ 
     &           &    5 &  7.22e-09 &  2.06e-16 &      \\ 
     &           &    6 &  7.05e-08 &  2.06e-16 &      \\ 
     &           &    7 &  8.50e-08 &  2.01e-15 &      \\ 
     &           &    8 &  7.05e-08 &  2.43e-15 &      \\ 
     &           &    9 &  7.29e-09 &  2.01e-15 &      \\ 
     &           &   10 &  7.28e-09 &  2.08e-16 &      \\ 
2374 &  2.37e+04 &   10 &           &           & iters  \\ 
 \hdashline 
     &           &    1 &  4.33e-08 &  1.47e-09 &      \\ 
     &           &    2 &  3.45e-08 &  1.23e-15 &      \\ 
     &           &    3 &  4.33e-08 &  9.84e-16 &      \\ 
     &           &    4 &  3.45e-08 &  1.23e-15 &      \\ 
     &           &    5 &  4.32e-08 &  9.85e-16 &      \\ 
     &           &    6 &  3.45e-08 &  1.23e-15 &      \\ 
     &           &    7 &  3.45e-08 &  9.85e-16 &      \\ 
     &           &    8 &  4.33e-08 &  9.83e-16 &      \\ 
     &           &    9 &  3.45e-08 &  1.24e-15 &      \\ 
     &           &   10 &  4.33e-08 &  9.84e-16 &      \\ 
2375 &  2.37e+04 &   10 &           &           & iters  \\ 
 \hdashline 
     &           &    1 &  1.59e-08 &  1.22e-09 &      \\ 
     &           &    2 &  6.18e-08 &  4.53e-16 &      \\ 
     &           &    3 &  1.59e-08 &  1.76e-15 &      \\ 
     &           &    4 &  1.59e-08 &  4.55e-16 &      \\ 
     &           &    5 &  1.59e-08 &  4.54e-16 &      \\ 
     &           &    6 &  1.59e-08 &  4.54e-16 &      \\ 
     &           &    7 &  6.18e-08 &  4.53e-16 &      \\ 
     &           &    8 &  1.59e-08 &  1.76e-15 &      \\ 
     &           &    9 &  1.59e-08 &  4.55e-16 &      \\ 
     &           &   10 &  1.59e-08 &  4.54e-16 &      \\ 
2376 &  2.38e+04 &   10 &           &           & iters  \\ 
 \hdashline 
     &           &    1 &  3.26e-08 &  1.39e-09 &      \\ 
     &           &    2 &  4.51e-08 &  9.30e-16 &      \\ 
     &           &    3 &  3.26e-08 &  1.29e-15 &      \\ 
     &           &    4 &  4.51e-08 &  9.31e-16 &      \\ 
     &           &    5 &  3.26e-08 &  1.29e-15 &      \\ 
     &           &    6 &  3.26e-08 &  9.31e-16 &      \\ 
     &           &    7 &  4.52e-08 &  9.30e-16 &      \\ 
     &           &    8 &  3.26e-08 &  1.29e-15 &      \\ 
     &           &    9 &  4.51e-08 &  9.30e-16 &      \\ 
     &           &   10 &  3.26e-08 &  1.29e-15 &      \\ 
2377 &  2.38e+04 &   10 &           &           & iters  \\ 
 \hdashline 
     &           &    1 &  2.25e-08 &  1.40e-09 &      \\ 
     &           &    2 &  2.25e-08 &  6.43e-16 &      \\ 
     &           &    3 &  5.52e-08 &  6.42e-16 &      \\ 
     &           &    4 &  2.26e-08 &  1.58e-15 &      \\ 
     &           &    5 &  2.25e-08 &  6.44e-16 &      \\ 
     &           &    6 &  5.52e-08 &  6.43e-16 &      \\ 
     &           &    7 &  2.26e-08 &  1.58e-15 &      \\ 
     &           &    8 &  2.25e-08 &  6.44e-16 &      \\ 
     &           &    9 &  2.25e-08 &  6.43e-16 &      \\ 
     &           &   10 &  5.52e-08 &  6.42e-16 &      \\ 
2378 &  2.38e+04 &   10 &           &           & iters  \\ 
 \hdashline 
     &           &    1 &  2.21e-08 &  2.33e-10 &      \\ 
     &           &    2 &  5.57e-08 &  6.30e-16 &      \\ 
     &           &    3 &  2.21e-08 &  1.59e-15 &      \\ 
     &           &    4 &  2.21e-08 &  6.31e-16 &      \\ 
     &           &    5 &  2.21e-08 &  6.30e-16 &      \\ 
     &           &    6 &  5.57e-08 &  6.29e-16 &      \\ 
     &           &    7 &  2.21e-08 &  1.59e-15 &      \\ 
     &           &    8 &  2.21e-08 &  6.31e-16 &      \\ 
     &           &    9 &  5.56e-08 &  6.30e-16 &      \\ 
     &           &   10 &  2.21e-08 &  1.59e-15 &      \\ 
2379 &  2.38e+04 &   10 &           &           & iters  \\ 
 \hdashline 
     &           &    1 &  5.05e-08 &  1.23e-09 &      \\ 
     &           &    2 &  2.73e-08 &  1.44e-15 &      \\ 
     &           &    3 &  2.73e-08 &  7.79e-16 &      \\ 
     &           &    4 &  5.05e-08 &  7.78e-16 &      \\ 
     &           &    5 &  2.73e-08 &  1.44e-15 &      \\ 
     &           &    6 &  2.73e-08 &  7.79e-16 &      \\ 
     &           &    7 &  5.05e-08 &  7.78e-16 &      \\ 
     &           &    8 &  2.73e-08 &  1.44e-15 &      \\ 
     &           &    9 &  2.73e-08 &  7.79e-16 &      \\ 
     &           &   10 &  5.05e-08 &  7.78e-16 &      \\ 
2380 &  2.38e+04 &   10 &           &           & iters  \\ 
 \hdashline 
     &           &    1 &  2.80e-08 &  1.52e-09 &      \\ 
     &           &    2 &  4.97e-08 &  7.99e-16 &      \\ 
     &           &    3 &  2.80e-08 &  1.42e-15 &      \\ 
     &           &    4 &  2.80e-08 &  8.00e-16 &      \\ 
     &           &    5 &  4.97e-08 &  7.99e-16 &      \\ 
     &           &    6 &  2.80e-08 &  1.42e-15 &      \\ 
     &           &    7 &  2.80e-08 &  8.00e-16 &      \\ 
     &           &    8 &  4.97e-08 &  7.99e-16 &      \\ 
     &           &    9 &  2.80e-08 &  1.42e-15 &      \\ 
     &           &   10 &  2.80e-08 &  8.00e-16 &      \\ 
2381 &  2.38e+04 &   10 &           &           & iters  \\ 
 \hdashline 
     &           &    1 &  7.03e-08 &  9.04e-10 &      \\ 
     &           &    2 &  8.52e-08 &  2.01e-15 &      \\ 
     &           &    3 &  7.03e-08 &  2.43e-15 &      \\ 
     &           &    4 &  7.47e-09 &  2.01e-15 &      \\ 
     &           &    5 &  7.46e-09 &  2.13e-16 &      \\ 
     &           &    6 &  7.44e-09 &  2.13e-16 &      \\ 
     &           &    7 &  7.43e-09 &  2.12e-16 &      \\ 
     &           &    8 &  7.42e-09 &  2.12e-16 &      \\ 
     &           &    9 &  7.41e-09 &  2.12e-16 &      \\ 
     &           &   10 &  7.40e-09 &  2.12e-16 &      \\ 
2382 &  2.38e+04 &   10 &           &           & iters  \\ 
 \hdashline 
     &           &    1 &  1.32e-08 &  9.53e-10 &      \\ 
     &           &    2 &  1.32e-08 &  3.78e-16 &      \\ 
     &           &    3 &  1.32e-08 &  3.77e-16 &      \\ 
     &           &    4 &  1.32e-08 &  3.77e-16 &      \\ 
     &           &    5 &  1.32e-08 &  3.76e-16 &      \\ 
     &           &    6 &  1.32e-08 &  3.76e-16 &      \\ 
     &           &    7 &  6.46e-08 &  3.75e-16 &      \\ 
     &           &    8 &  1.32e-08 &  1.84e-15 &      \\ 
     &           &    9 &  1.32e-08 &  3.77e-16 &      \\ 
     &           &   10 &  1.32e-08 &  3.77e-16 &      \\ 
2383 &  2.38e+04 &   10 &           &           & iters  \\ 
 \hdashline 
     &           &    1 &  5.94e-08 &  5.65e-10 &      \\ 
     &           &    2 &  1.84e-08 &  1.69e-15 &      \\ 
     &           &    3 &  1.84e-08 &  5.25e-16 &      \\ 
     &           &    4 &  1.84e-08 &  5.24e-16 &      \\ 
     &           &    5 &  5.94e-08 &  5.24e-16 &      \\ 
     &           &    6 &  1.84e-08 &  1.69e-15 &      \\ 
     &           &    7 &  1.84e-08 &  5.25e-16 &      \\ 
     &           &    8 &  1.84e-08 &  5.25e-16 &      \\ 
     &           &    9 &  5.94e-08 &  5.24e-16 &      \\ 
     &           &   10 &  1.84e-08 &  1.69e-15 &      \\ 
2384 &  2.38e+04 &   10 &           &           & iters  \\ 
 \hdashline 
     &           &    1 &  4.48e-08 &  8.61e-10 &      \\ 
     &           &    2 &  3.29e-08 &  1.28e-15 &      \\ 
     &           &    3 &  4.48e-08 &  9.40e-16 &      \\ 
     &           &    4 &  3.29e-08 &  1.28e-15 &      \\ 
     &           &    5 &  3.29e-08 &  9.40e-16 &      \\ 
     &           &    6 &  4.48e-08 &  9.39e-16 &      \\ 
     &           &    7 &  3.29e-08 &  1.28e-15 &      \\ 
     &           &    8 &  4.48e-08 &  9.39e-16 &      \\ 
     &           &    9 &  3.29e-08 &  1.28e-15 &      \\ 
     &           &   10 &  4.48e-08 &  9.40e-16 &      \\ 
2385 &  2.38e+04 &   10 &           &           & iters  \\ 
 \hdashline 
     &           &    1 &  1.22e-09 &  4.47e-10 &      \\ 
     &           &    2 &  1.22e-09 &  3.49e-17 &      \\ 
     &           &    3 &  1.22e-09 &  3.48e-17 &      \\ 
     &           &    4 &  1.22e-09 &  3.48e-17 &      \\ 
     &           &    5 &  1.22e-09 &  3.47e-17 &      \\ 
     &           &    6 &  1.21e-09 &  3.47e-17 &      \\ 
     &           &    7 &  1.21e-09 &  3.46e-17 &      \\ 
     &           &    8 &  1.21e-09 &  3.46e-17 &      \\ 
     &           &    9 &  1.21e-09 &  3.45e-17 &      \\ 
     &           &   10 &  1.21e-09 &  3.45e-17 &      \\ 
2386 &  2.38e+04 &   10 &           &           & iters  \\ 
 \hdashline 
     &           &    1 &  2.40e-08 &  6.75e-10 &      \\ 
     &           &    2 &  2.39e-08 &  6.84e-16 &      \\ 
     &           &    3 &  2.39e-08 &  6.83e-16 &      \\ 
     &           &    4 &  5.38e-08 &  6.82e-16 &      \\ 
     &           &    5 &  2.39e-08 &  1.54e-15 &      \\ 
     &           &    6 &  2.39e-08 &  6.83e-16 &      \\ 
     &           &    7 &  5.38e-08 &  6.82e-16 &      \\ 
     &           &    8 &  2.39e-08 &  1.54e-15 &      \\ 
     &           &    9 &  2.39e-08 &  6.83e-16 &      \\ 
     &           &   10 &  2.39e-08 &  6.82e-16 &      \\ 
2387 &  2.39e+04 &   10 &           &           & iters  \\ 
 \hdashline 
     &           &    1 &  9.61e-08 &  1.11e-10 &      \\ 
     &           &    2 &  2.05e-08 &  2.74e-15 &      \\ 
     &           &    3 &  2.05e-08 &  5.86e-16 &      \\ 
     &           &    4 &  1.84e-08 &  5.85e-16 &      \\ 
     &           &    5 &  2.05e-08 &  5.24e-16 &      \\ 
     &           &    6 &  1.84e-08 &  5.85e-16 &      \\ 
     &           &    7 &  2.05e-08 &  5.24e-16 &      \\ 
     &           &    8 &  1.84e-08 &  5.85e-16 &      \\ 
     &           &    9 &  2.05e-08 &  5.24e-16 &      \\ 
     &           &   10 &  1.84e-08 &  5.85e-16 &      \\ 
2388 &  2.39e+04 &   10 &           &           & iters  \\ 
 \hdashline 
     &           &    1 &  3.05e-08 &  8.48e-10 &      \\ 
     &           &    2 &  4.72e-08 &  8.71e-16 &      \\ 
     &           &    3 &  3.05e-08 &  1.35e-15 &      \\ 
     &           &    4 &  4.72e-08 &  8.71e-16 &      \\ 
     &           &    5 &  3.06e-08 &  1.35e-15 &      \\ 
     &           &    6 &  3.05e-08 &  8.72e-16 &      \\ 
     &           &    7 &  4.72e-08 &  8.71e-16 &      \\ 
     &           &    8 &  3.05e-08 &  1.35e-15 &      \\ 
     &           &    9 &  4.72e-08 &  8.71e-16 &      \\ 
     &           &   10 &  3.06e-08 &  1.35e-15 &      \\ 
2389 &  2.39e+04 &   10 &           &           & iters  \\ 
 \hdashline 
     &           &    1 &  1.00e-08 &  5.80e-10 &      \\ 
     &           &    2 &  1.00e-08 &  2.86e-16 &      \\ 
     &           &    3 &  1.00e-08 &  2.86e-16 &      \\ 
     &           &    4 &  9.99e-09 &  2.85e-16 &      \\ 
     &           &    5 &  6.77e-08 &  2.85e-16 &      \\ 
     &           &    6 &  4.89e-08 &  1.93e-15 &      \\ 
     &           &    7 &  1.00e-08 &  1.40e-15 &      \\ 
     &           &    8 &  9.99e-09 &  2.85e-16 &      \\ 
     &           &    9 &  6.77e-08 &  2.85e-16 &      \\ 
     &           &   10 &  4.89e-08 &  1.93e-15 &      \\ 
2390 &  2.39e+04 &   10 &           &           & iters  \\ 
 \hdashline 
     &           &    1 &  1.84e-08 &  4.40e-10 &      \\ 
     &           &    2 &  1.83e-08 &  5.24e-16 &      \\ 
     &           &    3 &  5.94e-08 &  5.23e-16 &      \\ 
     &           &    4 &  1.84e-08 &  1.69e-15 &      \\ 
     &           &    5 &  1.84e-08 &  5.25e-16 &      \\ 
     &           &    6 &  1.83e-08 &  5.24e-16 &      \\ 
     &           &    7 &  1.83e-08 &  5.23e-16 &      \\ 
     &           &    8 &  5.94e-08 &  5.23e-16 &      \\ 
     &           &    9 &  1.84e-08 &  1.70e-15 &      \\ 
     &           &   10 &  1.83e-08 &  5.24e-16 &      \\ 
2391 &  2.39e+04 &   10 &           &           & iters  \\ 
 \hdashline 
     &           &    1 &  8.14e-09 &  1.68e-10 &      \\ 
     &           &    2 &  8.13e-09 &  2.32e-16 &      \\ 
     &           &    3 &  3.07e-08 &  2.32e-16 &      \\ 
     &           &    4 &  8.16e-09 &  8.77e-16 &      \\ 
     &           &    5 &  8.15e-09 &  2.33e-16 &      \\ 
     &           &    6 &  8.14e-09 &  2.33e-16 &      \\ 
     &           &    7 &  8.13e-09 &  2.32e-16 &      \\ 
     &           &    8 &  3.07e-08 &  2.32e-16 &      \\ 
     &           &    9 &  8.16e-09 &  8.77e-16 &      \\ 
     &           &   10 &  8.15e-09 &  2.33e-16 &      \\ 
2392 &  2.39e+04 &   10 &           &           & iters  \\ 
 \hdashline 
     &           &    1 &  4.73e-09 &  1.26e-11 &      \\ 
     &           &    2 &  4.72e-09 &  1.35e-16 &      \\ 
     &           &    3 &  4.71e-09 &  1.35e-16 &      \\ 
     &           &    4 &  4.71e-09 &  1.34e-16 &      \\ 
     &           &    5 &  4.70e-09 &  1.34e-16 &      \\ 
     &           &    6 &  4.69e-09 &  1.34e-16 &      \\ 
     &           &    7 &  4.68e-09 &  1.34e-16 &      \\ 
     &           &    8 &  7.30e-08 &  1.34e-16 &      \\ 
     &           &    9 &  4.36e-08 &  2.08e-15 &      \\ 
     &           &   10 &  4.72e-09 &  1.25e-15 &      \\ 
2393 &  2.39e+04 &   10 &           &           & iters  \\ 
 \hdashline 
     &           &    1 &  4.43e-08 &  3.00e-10 &      \\ 
     &           &    2 &  3.35e-08 &  1.26e-15 &      \\ 
     &           &    3 &  4.42e-08 &  9.56e-16 &      \\ 
     &           &    4 &  3.35e-08 &  1.26e-15 &      \\ 
     &           &    5 &  3.35e-08 &  9.56e-16 &      \\ 
     &           &    6 &  4.43e-08 &  9.55e-16 &      \\ 
     &           &    7 &  3.35e-08 &  1.26e-15 &      \\ 
     &           &    8 &  4.43e-08 &  9.55e-16 &      \\ 
     &           &    9 &  3.35e-08 &  1.26e-15 &      \\ 
     &           &   10 &  4.42e-08 &  9.56e-16 &      \\ 
2394 &  2.39e+04 &   10 &           &           & iters  \\ 
 \hdashline 
     &           &    1 &  7.48e-08 &  4.77e-11 &      \\ 
     &           &    2 &  4.19e-08 &  2.13e-15 &      \\ 
     &           &    3 &  2.97e-09 &  1.20e-15 &      \\ 
     &           &    4 &  2.96e-09 &  8.47e-17 &      \\ 
     &           &    5 &  2.96e-09 &  8.45e-17 &      \\ 
     &           &    6 &  2.95e-09 &  8.44e-17 &      \\ 
     &           &    7 &  2.95e-09 &  8.43e-17 &      \\ 
     &           &    8 &  2.95e-09 &  8.42e-17 &      \\ 
     &           &    9 &  2.94e-09 &  8.41e-17 &      \\ 
     &           &   10 &  2.94e-09 &  8.39e-17 &      \\ 
2395 &  2.39e+04 &   10 &           &           & iters  \\ 
 \hdashline 
     &           &    1 &  1.74e-08 &  5.98e-10 &      \\ 
     &           &    2 &  1.74e-08 &  4.96e-16 &      \\ 
     &           &    3 &  1.73e-08 &  4.96e-16 &      \\ 
     &           &    4 &  6.04e-08 &  4.95e-16 &      \\ 
     &           &    5 &  1.74e-08 &  1.72e-15 &      \\ 
     &           &    6 &  1.74e-08 &  4.97e-16 &      \\ 
     &           &    7 &  1.74e-08 &  4.96e-16 &      \\ 
     &           &    8 &  1.73e-08 &  4.95e-16 &      \\ 
     &           &    9 &  6.04e-08 &  4.95e-16 &      \\ 
     &           &   10 &  1.74e-08 &  1.72e-15 &      \\ 
2396 &  2.40e+04 &   10 &           &           & iters  \\ 
 \hdashline 
     &           &    1 &  6.31e-09 &  4.20e-10 &      \\ 
     &           &    2 &  6.31e-09 &  1.80e-16 &      \\ 
     &           &    3 &  6.30e-09 &  1.80e-16 &      \\ 
     &           &    4 &  6.29e-09 &  1.80e-16 &      \\ 
     &           &    5 &  6.28e-09 &  1.79e-16 &      \\ 
     &           &    6 &  3.26e-08 &  1.79e-16 &      \\ 
     &           &    7 &  6.32e-09 &  9.30e-16 &      \\ 
     &           &    8 &  6.31e-09 &  1.80e-16 &      \\ 
     &           &    9 &  6.30e-09 &  1.80e-16 &      \\ 
     &           &   10 &  6.29e-09 &  1.80e-16 &      \\ 
2397 &  2.40e+04 &   10 &           &           & iters  \\ 
 \hdashline 
     &           &    1 &  2.32e-08 &  2.09e-10 &      \\ 
     &           &    2 &  5.45e-08 &  6.63e-16 &      \\ 
     &           &    3 &  2.33e-08 &  1.56e-15 &      \\ 
     &           &    4 &  2.32e-08 &  6.64e-16 &      \\ 
     &           &    5 &  5.45e-08 &  6.63e-16 &      \\ 
     &           &    6 &  2.33e-08 &  1.55e-15 &      \\ 
     &           &    7 &  2.33e-08 &  6.65e-16 &      \\ 
     &           &    8 &  2.32e-08 &  6.64e-16 &      \\ 
     &           &    9 &  5.45e-08 &  6.63e-16 &      \\ 
     &           &   10 &  2.33e-08 &  1.56e-15 &      \\ 
2398 &  2.40e+04 &   10 &           &           & iters  \\ 
 \hdashline 
     &           &    1 &  3.93e-08 &  1.62e-10 &      \\ 
     &           &    2 &  3.84e-08 &  1.12e-15 &      \\ 
     &           &    3 &  3.93e-08 &  1.10e-15 &      \\ 
     &           &    4 &  3.84e-08 &  1.12e-15 &      \\ 
     &           &    5 &  3.83e-08 &  1.10e-15 &      \\ 
     &           &    6 &  3.94e-08 &  1.09e-15 &      \\ 
     &           &    7 &  3.84e-08 &  1.12e-15 &      \\ 
     &           &    8 &  3.94e-08 &  1.09e-15 &      \\ 
     &           &    9 &  3.84e-08 &  1.12e-15 &      \\ 
     &           &   10 &  3.94e-08 &  1.09e-15 &      \\ 
2399 &  2.40e+04 &   10 &           &           & iters  \\ 
 \hdashline 
     &           &    1 &  1.15e-08 &  3.01e-11 &      \\ 
     &           &    2 &  1.15e-08 &  3.29e-16 &      \\ 
     &           &    3 &  1.15e-08 &  3.29e-16 &      \\ 
     &           &    4 &  6.62e-08 &  3.28e-16 &      \\ 
     &           &    5 &  1.16e-08 &  1.89e-15 &      \\ 
     &           &    6 &  1.16e-08 &  3.31e-16 &      \\ 
     &           &    7 &  1.16e-08 &  3.30e-16 &      \\ 
     &           &    8 &  1.15e-08 &  3.30e-16 &      \\ 
     &           &    9 &  1.15e-08 &  3.29e-16 &      \\ 
     &           &   10 &  6.62e-08 &  3.29e-16 &      \\ 
2400 &  2.40e+04 &   10 &           &           & iters  \\ 
 \hdashline 
     &           &    1 &  5.70e-08 &  1.65e-11 &      \\ 
     &           &    2 &  2.08e-08 &  1.63e-15 &      \\ 
     &           &    3 &  2.08e-08 &  5.93e-16 &      \\ 
     &           &    4 &  5.70e-08 &  5.92e-16 &      \\ 
     &           &    5 &  2.08e-08 &  1.63e-15 &      \\ 
     &           &    6 &  2.08e-08 &  5.94e-16 &      \\ 
     &           &    7 &  5.69e-08 &  5.93e-16 &      \\ 
     &           &    8 &  2.08e-08 &  1.63e-15 &      \\ 
     &           &    9 &  2.08e-08 &  5.94e-16 &      \\ 
     &           &   10 &  2.08e-08 &  5.94e-16 &      \\ 
2401 &  2.40e+04 &   10 &           &           & iters  \\ 
 \hdashline 
     &           &    1 &  2.41e-08 &  2.09e-10 &      \\ 
     &           &    2 &  5.36e-08 &  6.88e-16 &      \\ 
     &           &    3 &  2.42e-08 &  1.53e-15 &      \\ 
     &           &    4 &  2.41e-08 &  6.89e-16 &      \\ 
     &           &    5 &  5.36e-08 &  6.88e-16 &      \\ 
     &           &    6 &  2.42e-08 &  1.53e-15 &      \\ 
     &           &    7 &  2.41e-08 &  6.89e-16 &      \\ 
     &           &    8 &  5.36e-08 &  6.88e-16 &      \\ 
     &           &    9 &  2.42e-08 &  1.53e-15 &      \\ 
     &           &   10 &  2.41e-08 &  6.90e-16 &      \\ 
2402 &  2.40e+04 &   10 &           &           & iters  \\ 
 \hdashline 
     &           &    1 &  6.34e-08 &  5.61e-10 &      \\ 
     &           &    2 &  1.43e-08 &  1.81e-15 &      \\ 
     &           &    3 &  1.43e-08 &  4.09e-16 &      \\ 
     &           &    4 &  1.43e-08 &  4.09e-16 &      \\ 
     &           &    5 &  1.43e-08 &  4.08e-16 &      \\ 
     &           &    6 &  1.43e-08 &  4.07e-16 &      \\ 
     &           &    7 &  6.35e-08 &  4.07e-16 &      \\ 
     &           &    8 &  1.43e-08 &  1.81e-15 &      \\ 
     &           &    9 &  1.43e-08 &  4.09e-16 &      \\ 
     &           &   10 &  1.43e-08 &  4.08e-16 &      \\ 
2403 &  2.40e+04 &   10 &           &           & iters  \\ 
 \hdashline 
     &           &    1 &  8.38e-09 &  6.13e-10 &      \\ 
     &           &    2 &  8.37e-09 &  2.39e-16 &      \\ 
     &           &    3 &  8.36e-09 &  2.39e-16 &      \\ 
     &           &    4 &  8.35e-09 &  2.39e-16 &      \\ 
     &           &    5 &  8.34e-09 &  2.38e-16 &      \\ 
     &           &    6 &  8.32e-09 &  2.38e-16 &      \\ 
     &           &    7 &  8.31e-09 &  2.38e-16 &      \\ 
     &           &    8 &  6.94e-08 &  2.37e-16 &      \\ 
     &           &    9 &  4.72e-08 &  1.98e-15 &      \\ 
     &           &   10 &  8.33e-09 &  1.35e-15 &      \\ 
2404 &  2.40e+04 &   10 &           &           & iters  \\ 
 \hdashline 
     &           &    1 &  5.05e-08 &  1.30e-11 &      \\ 
     &           &    2 &  2.73e-08 &  1.44e-15 &      \\ 
     &           &    3 &  5.05e-08 &  7.78e-16 &      \\ 
     &           &    4 &  2.73e-08 &  1.44e-15 &      \\ 
     &           &    5 &  2.72e-08 &  7.79e-16 &      \\ 
     &           &    6 &  5.05e-08 &  7.78e-16 &      \\ 
     &           &    7 &  2.73e-08 &  1.44e-15 &      \\ 
     &           &    8 &  2.72e-08 &  7.79e-16 &      \\ 
     &           &    9 &  5.05e-08 &  7.77e-16 &      \\ 
     &           &   10 &  2.73e-08 &  1.44e-15 &      \\ 
2405 &  2.40e+04 &   10 &           &           & iters  \\ 
 \hdashline 
     &           &    1 &  2.60e-08 &  6.19e-10 &      \\ 
     &           &    2 &  5.17e-08 &  7.43e-16 &      \\ 
     &           &    3 &  2.61e-08 &  1.47e-15 &      \\ 
     &           &    4 &  2.60e-08 &  7.44e-16 &      \\ 
     &           &    5 &  5.17e-08 &  7.43e-16 &      \\ 
     &           &    6 &  2.61e-08 &  1.47e-15 &      \\ 
     &           &    7 &  2.60e-08 &  7.44e-16 &      \\ 
     &           &    8 &  5.17e-08 &  7.43e-16 &      \\ 
     &           &    9 &  2.61e-08 &  1.47e-15 &      \\ 
     &           &   10 &  2.60e-08 &  7.44e-16 &      \\ 
2406 &  2.40e+04 &   10 &           &           & iters  \\ 
 \hdashline 
     &           &    1 &  1.97e-09 &  4.14e-10 &      \\ 
     &           &    2 &  1.96e-09 &  5.61e-17 &      \\ 
     &           &    3 &  1.96e-09 &  5.60e-17 &      \\ 
     &           &    4 &  1.96e-09 &  5.59e-17 &      \\ 
     &           &    5 &  1.95e-09 &  5.59e-17 &      \\ 
     &           &    6 &  7.57e-08 &  5.58e-17 &      \\ 
     &           &    7 &  4.09e-08 &  2.16e-15 &      \\ 
     &           &    8 &  2.00e-09 &  1.17e-15 &      \\ 
     &           &    9 &  2.00e-09 &  5.71e-17 &      \\ 
     &           &   10 &  2.00e-09 &  5.70e-17 &      \\ 
2407 &  2.41e+04 &   10 &           &           & iters  \\ 
 \hdashline 
     &           &    1 &  5.05e-09 &  8.89e-11 &      \\ 
     &           &    2 &  5.04e-09 &  1.44e-16 &      \\ 
     &           &    3 &  5.03e-09 &  1.44e-16 &      \\ 
     &           &    4 &  5.02e-09 &  1.44e-16 &      \\ 
     &           &    5 &  5.02e-09 &  1.43e-16 &      \\ 
     &           &    6 &  5.01e-09 &  1.43e-16 &      \\ 
     &           &    7 &  5.00e-09 &  1.43e-16 &      \\ 
     &           &    8 &  4.99e-09 &  1.43e-16 &      \\ 
     &           &    9 &  4.99e-09 &  1.43e-16 &      \\ 
     &           &   10 &  4.98e-09 &  1.42e-16 &      \\ 
2408 &  2.41e+04 &   10 &           &           & iters  \\ 
 \hdashline 
     &           &    1 &  4.21e-08 &  7.44e-11 &      \\ 
     &           &    2 &  3.56e-08 &  1.20e-15 &      \\ 
     &           &    3 &  4.21e-08 &  1.02e-15 &      \\ 
     &           &    4 &  3.56e-08 &  1.20e-15 &      \\ 
     &           &    5 &  4.21e-08 &  1.02e-15 &      \\ 
     &           &    6 &  3.56e-08 &  1.20e-15 &      \\ 
     &           &    7 &  4.21e-08 &  1.02e-15 &      \\ 
     &           &    8 &  3.56e-08 &  1.20e-15 &      \\ 
     &           &    9 &  4.21e-08 &  1.02e-15 &      \\ 
     &           &   10 &  3.56e-08 &  1.20e-15 &      \\ 
2409 &  2.41e+04 &   10 &           &           & iters  \\ 
 \hdashline 
     &           &    1 &  1.35e-08 &  1.84e-10 &      \\ 
     &           &    2 &  2.54e-08 &  3.86e-16 &      \\ 
     &           &    3 &  1.35e-08 &  7.24e-16 &      \\ 
     &           &    4 &  1.35e-08 &  3.86e-16 &      \\ 
     &           &    5 &  2.54e-08 &  3.85e-16 &      \\ 
     &           &    6 &  1.35e-08 &  7.24e-16 &      \\ 
     &           &    7 &  1.35e-08 &  3.86e-16 &      \\ 
     &           &    8 &  2.54e-08 &  3.85e-16 &      \\ 
     &           &    9 &  1.35e-08 &  7.24e-16 &      \\ 
     &           &   10 &  1.35e-08 &  3.86e-16 &      \\ 
2410 &  2.41e+04 &   10 &           &           & iters  \\ 
 \hdashline 
     &           &    1 &  1.81e-08 &  3.33e-10 &      \\ 
     &           &    2 &  5.96e-08 &  5.17e-16 &      \\ 
     &           &    3 &  1.82e-08 &  1.70e-15 &      \\ 
     &           &    4 &  1.82e-08 &  5.19e-16 &      \\ 
     &           &    5 &  1.81e-08 &  5.18e-16 &      \\ 
     &           &    6 &  5.96e-08 &  5.18e-16 &      \\ 
     &           &    7 &  1.82e-08 &  1.70e-15 &      \\ 
     &           &    8 &  1.82e-08 &  5.19e-16 &      \\ 
     &           &    9 &  1.81e-08 &  5.18e-16 &      \\ 
     &           &   10 &  5.96e-08 &  5.18e-16 &      \\ 
2411 &  2.41e+04 &   10 &           &           & iters  \\ 
 \hdashline 
     &           &    1 &  1.37e-09 &  4.00e-12 &      \\ 
     &           &    2 &  1.36e-09 &  3.90e-17 &      \\ 
     &           &    3 &  1.36e-09 &  3.90e-17 &      \\ 
     &           &    4 &  1.36e-09 &  3.89e-17 &      \\ 
     &           &    5 &  1.36e-09 &  3.88e-17 &      \\ 
     &           &    6 &  1.36e-09 &  3.88e-17 &      \\ 
     &           &    7 &  1.36e-09 &  3.87e-17 &      \\ 
     &           &    8 &  1.35e-09 &  3.87e-17 &      \\ 
     &           &    9 &  1.35e-09 &  3.86e-17 &      \\ 
     &           &   10 &  7.63e-08 &  3.86e-17 &      \\ 
2412 &  2.41e+04 &   10 &           &           & iters  \\ 
 \hdashline 
     &           &    1 &  3.64e-08 &  7.94e-10 &      \\ 
     &           &    2 &  2.50e-09 &  1.04e-15 &      \\ 
     &           &    3 &  2.49e-09 &  7.13e-17 &      \\ 
     &           &    4 &  2.49e-09 &  7.12e-17 &      \\ 
     &           &    5 &  2.49e-09 &  7.11e-17 &      \\ 
     &           &    6 &  2.48e-09 &  7.09e-17 &      \\ 
     &           &    7 &  2.48e-09 &  7.08e-17 &      \\ 
     &           &    8 &  2.48e-09 &  7.07e-17 &      \\ 
     &           &    9 &  2.47e-09 &  7.06e-17 &      \\ 
     &           &   10 &  2.47e-09 &  7.05e-17 &      \\ 
2413 &  2.41e+04 &   10 &           &           & iters  \\ 
 \hdashline 
     &           &    1 &  1.60e-08 &  1.09e-09 &      \\ 
     &           &    2 &  2.29e-08 &  4.57e-16 &      \\ 
     &           &    3 &  1.60e-08 &  6.53e-16 &      \\ 
     &           &    4 &  1.60e-08 &  4.57e-16 &      \\ 
     &           &    5 &  2.29e-08 &  4.56e-16 &      \\ 
     &           &    6 &  1.60e-08 &  6.53e-16 &      \\ 
     &           &    7 &  2.29e-08 &  4.57e-16 &      \\ 
     &           &    8 &  1.60e-08 &  6.53e-16 &      \\ 
     &           &    9 &  1.60e-08 &  4.57e-16 &      \\ 
     &           &   10 &  2.29e-08 &  4.56e-16 &      \\ 
2414 &  2.41e+04 &   10 &           &           & iters  \\ 
 \hdashline 
     &           &    1 &  4.59e-08 &  1.05e-09 &      \\ 
     &           &    2 &  3.18e-08 &  1.31e-15 &      \\ 
     &           &    3 &  3.18e-08 &  9.09e-16 &      \\ 
     &           &    4 &  4.59e-08 &  9.08e-16 &      \\ 
     &           &    5 &  3.18e-08 &  1.31e-15 &      \\ 
     &           &    6 &  4.59e-08 &  9.08e-16 &      \\ 
     &           &    7 &  3.18e-08 &  1.31e-15 &      \\ 
     &           &    8 &  4.59e-08 &  9.09e-16 &      \\ 
     &           &    9 &  3.19e-08 &  1.31e-15 &      \\ 
     &           &   10 &  3.18e-08 &  9.09e-16 &      \\ 
2415 &  2.41e+04 &   10 &           &           & iters  \\ 
 \hdashline 
     &           &    1 &  4.45e-08 &  5.43e-10 &      \\ 
     &           &    2 &  3.33e-08 &  1.27e-15 &      \\ 
     &           &    3 &  4.45e-08 &  9.49e-16 &      \\ 
     &           &    4 &  3.33e-08 &  1.27e-15 &      \\ 
     &           &    5 &  3.32e-08 &  9.50e-16 &      \\ 
     &           &    6 &  4.45e-08 &  9.48e-16 &      \\ 
     &           &    7 &  3.32e-08 &  1.27e-15 &      \\ 
     &           &    8 &  4.45e-08 &  9.49e-16 &      \\ 
     &           &    9 &  3.33e-08 &  1.27e-15 &      \\ 
     &           &   10 &  4.45e-08 &  9.49e-16 &      \\ 
2416 &  2.42e+04 &   10 &           &           & iters  \\ 
 \hdashline 
     &           &    1 &  3.65e-08 &  4.45e-10 &      \\ 
     &           &    2 &  2.40e-09 &  1.04e-15 &      \\ 
     &           &    3 &  2.39e-09 &  6.84e-17 &      \\ 
     &           &    4 &  2.39e-09 &  6.83e-17 &      \\ 
     &           &    5 &  2.39e-09 &  6.82e-17 &      \\ 
     &           &    6 &  2.38e-09 &  6.81e-17 &      \\ 
     &           &    7 &  2.38e-09 &  6.80e-17 &      \\ 
     &           &    8 &  2.38e-09 &  6.79e-17 &      \\ 
     &           &    9 &  2.37e-09 &  6.79e-17 &      \\ 
     &           &   10 &  2.37e-09 &  6.78e-17 &      \\ 
2417 &  2.42e+04 &   10 &           &           & iters  \\ 
 \hdashline 
     &           &    1 &  2.54e-08 &  5.77e-10 &      \\ 
     &           &    2 &  2.53e-08 &  7.24e-16 &      \\ 
     &           &    3 &  5.24e-08 &  7.23e-16 &      \\ 
     &           &    4 &  2.54e-08 &  1.49e-15 &      \\ 
     &           &    5 &  2.53e-08 &  7.25e-16 &      \\ 
     &           &    6 &  5.24e-08 &  7.23e-16 &      \\ 
     &           &    7 &  2.54e-08 &  1.49e-15 &      \\ 
     &           &    8 &  2.54e-08 &  7.25e-16 &      \\ 
     &           &    9 &  5.24e-08 &  7.24e-16 &      \\ 
     &           &   10 &  2.54e-08 &  1.49e-15 &      \\ 
2418 &  2.42e+04 &   10 &           &           & iters  \\ 
 \hdashline 
     &           &    1 &  8.00e-09 &  6.12e-12 &      \\ 
     &           &    2 &  3.09e-08 &  2.28e-16 &      \\ 
     &           &    3 &  8.04e-09 &  8.81e-16 &      \\ 
     &           &    4 &  8.02e-09 &  2.29e-16 &      \\ 
     &           &    5 &  8.01e-09 &  2.29e-16 &      \\ 
     &           &    6 &  8.00e-09 &  2.29e-16 &      \\ 
     &           &    7 &  3.09e-08 &  2.28e-16 &      \\ 
     &           &    8 &  8.03e-09 &  8.81e-16 &      \\ 
     &           &    9 &  8.02e-09 &  2.29e-16 &      \\ 
     &           &   10 &  8.01e-09 &  2.29e-16 &      \\ 
2419 &  2.42e+04 &   10 &           &           & iters  \\ 
 \hdashline 
     &           &    1 &  7.05e-09 &  4.95e-10 &      \\ 
     &           &    2 &  3.18e-08 &  2.01e-16 &      \\ 
     &           &    3 &  7.08e-09 &  9.08e-16 &      \\ 
     &           &    4 &  7.07e-09 &  2.02e-16 &      \\ 
     &           &    5 &  7.06e-09 &  2.02e-16 &      \\ 
     &           &    6 &  7.05e-09 &  2.02e-16 &      \\ 
     &           &    7 &  7.04e-09 &  2.01e-16 &      \\ 
     &           &    8 &  3.18e-08 &  2.01e-16 &      \\ 
     &           &    9 &  7.08e-09 &  9.08e-16 &      \\ 
     &           &   10 &  7.07e-09 &  2.02e-16 &      \\ 
2420 &  2.42e+04 &   10 &           &           & iters  \\ 
 \hdashline 
     &           &    1 &  1.67e-08 &  1.67e-10 &      \\ 
     &           &    2 &  2.22e-08 &  4.77e-16 &      \\ 
     &           &    3 &  1.67e-08 &  6.33e-16 &      \\ 
     &           &    4 &  2.22e-08 &  4.77e-16 &      \\ 
     &           &    5 &  1.67e-08 &  6.33e-16 &      \\ 
     &           &    6 &  2.22e-08 &  4.77e-16 &      \\ 
     &           &    7 &  1.67e-08 &  6.32e-16 &      \\ 
     &           &    8 &  1.67e-08 &  4.77e-16 &      \\ 
     &           &    9 &  2.22e-08 &  4.77e-16 &      \\ 
     &           &   10 &  1.67e-08 &  6.33e-16 &      \\ 
2421 &  2.42e+04 &   10 &           &           & iters  \\ 
 \hdashline 
     &           &    1 &  1.83e-08 &  5.37e-10 &      \\ 
     &           &    2 &  1.83e-08 &  5.23e-16 &      \\ 
     &           &    3 &  5.94e-08 &  5.23e-16 &      \\ 
     &           &    4 &  1.84e-08 &  1.70e-15 &      \\ 
     &           &    5 &  1.83e-08 &  5.24e-16 &      \\ 
     &           &    6 &  1.83e-08 &  5.24e-16 &      \\ 
     &           &    7 &  5.94e-08 &  5.23e-16 &      \\ 
     &           &    8 &  1.84e-08 &  1.70e-15 &      \\ 
     &           &    9 &  1.84e-08 &  5.25e-16 &      \\ 
     &           &   10 &  1.83e-08 &  5.24e-16 &      \\ 
2422 &  2.42e+04 &   10 &           &           & iters  \\ 
 \hdashline 
     &           &    1 &  2.80e-08 &  4.33e-10 &      \\ 
     &           &    2 &  2.79e-08 &  7.98e-16 &      \\ 
     &           &    3 &  4.98e-08 &  7.97e-16 &      \\ 
     &           &    4 &  2.80e-08 &  1.42e-15 &      \\ 
     &           &    5 &  2.79e-08 &  7.98e-16 &      \\ 
     &           &    6 &  4.98e-08 &  7.97e-16 &      \\ 
     &           &    7 &  2.79e-08 &  1.42e-15 &      \\ 
     &           &    8 &  2.79e-08 &  7.98e-16 &      \\ 
     &           &    9 &  4.98e-08 &  7.96e-16 &      \\ 
     &           &   10 &  2.79e-08 &  1.42e-15 &      \\ 
2423 &  2.42e+04 &   10 &           &           & iters  \\ 
 \hdashline 
     &           &    1 &  1.20e-08 &  1.79e-10 &      \\ 
     &           &    2 &  6.57e-08 &  3.44e-16 &      \\ 
     &           &    3 &  8.98e-08 &  1.87e-15 &      \\ 
     &           &    4 &  6.57e-08 &  2.56e-15 &      \\ 
     &           &    5 &  1.21e-08 &  1.88e-15 &      \\ 
     &           &    6 &  1.21e-08 &  3.45e-16 &      \\ 
     &           &    7 &  1.20e-08 &  3.44e-16 &      \\ 
     &           &    8 &  1.20e-08 &  3.44e-16 &      \\ 
     &           &    9 &  6.57e-08 &  3.43e-16 &      \\ 
     &           &   10 &  1.21e-08 &  1.87e-15 &      \\ 
2424 &  2.42e+04 &   10 &           &           & iters  \\ 
 \hdashline 
     &           &    1 &  1.75e-08 &  7.00e-11 &      \\ 
     &           &    2 &  6.02e-08 &  4.99e-16 &      \\ 
     &           &    3 &  1.76e-08 &  1.72e-15 &      \\ 
     &           &    4 &  1.75e-08 &  5.01e-16 &      \\ 
     &           &    5 &  1.75e-08 &  5.00e-16 &      \\ 
     &           &    6 &  6.02e-08 &  5.00e-16 &      \\ 
     &           &    7 &  1.76e-08 &  1.72e-15 &      \\ 
     &           &    8 &  1.75e-08 &  5.01e-16 &      \\ 
     &           &    9 &  1.75e-08 &  5.01e-16 &      \\ 
     &           &   10 &  1.75e-08 &  5.00e-16 &      \\ 
2425 &  2.42e+04 &   10 &           &           & iters  \\ 
 \hdashline 
     &           &    1 &  2.57e-08 &  2.92e-10 &      \\ 
     &           &    2 &  2.57e-08 &  7.34e-16 &      \\ 
     &           &    3 &  1.32e-08 &  7.33e-16 &      \\ 
     &           &    4 &  1.32e-08 &  3.77e-16 &      \\ 
     &           &    5 &  2.57e-08 &  3.76e-16 &      \\ 
     &           &    6 &  1.32e-08 &  7.33e-16 &      \\ 
     &           &    7 &  1.32e-08 &  3.77e-16 &      \\ 
     &           &    8 &  2.57e-08 &  3.76e-16 &      \\ 
     &           &    9 &  1.32e-08 &  7.33e-16 &      \\ 
     &           &   10 &  1.32e-08 &  3.77e-16 &      \\ 
2426 &  2.42e+04 &   10 &           &           & iters  \\ 
 \hdashline 
     &           &    1 &  5.23e-08 &  9.46e-11 &      \\ 
     &           &    2 &  2.54e-08 &  1.49e-15 &      \\ 
     &           &    3 &  2.54e-08 &  7.25e-16 &      \\ 
     &           &    4 &  2.53e-08 &  7.24e-16 &      \\ 
     &           &    5 &  5.24e-08 &  7.23e-16 &      \\ 
     &           &    6 &  2.54e-08 &  1.49e-15 &      \\ 
     &           &    7 &  2.53e-08 &  7.24e-16 &      \\ 
     &           &    8 &  5.24e-08 &  7.23e-16 &      \\ 
     &           &    9 &  2.54e-08 &  1.49e-15 &      \\ 
     &           &   10 &  2.53e-08 &  7.24e-16 &      \\ 
2427 &  2.43e+04 &   10 &           &           & iters  \\ 
 \hdashline 
     &           &    1 &  9.58e-09 &  9.97e-10 &      \\ 
     &           &    2 &  9.57e-09 &  2.73e-16 &      \\ 
     &           &    3 &  9.56e-09 &  2.73e-16 &      \\ 
     &           &    4 &  9.54e-09 &  2.73e-16 &      \\ 
     &           &    5 &  9.53e-09 &  2.72e-16 &      \\ 
     &           &    6 &  9.51e-09 &  2.72e-16 &      \\ 
     &           &    7 &  2.93e-08 &  2.72e-16 &      \\ 
     &           &    8 &  9.54e-09 &  8.37e-16 &      \\ 
     &           &    9 &  9.53e-09 &  2.72e-16 &      \\ 
     &           &   10 &  9.52e-09 &  2.72e-16 &      \\ 
2428 &  2.43e+04 &   10 &           &           & iters  \\ 
 \hdashline 
     &           &    1 &  1.40e-09 &  1.73e-09 &      \\ 
     &           &    2 &  1.40e-09 &  4.01e-17 &      \\ 
     &           &    3 &  1.40e-09 &  4.00e-17 &      \\ 
     &           &    4 &  1.40e-09 &  4.00e-17 &      \\ 
     &           &    5 &  1.40e-09 &  3.99e-17 &      \\ 
     &           &    6 &  1.39e-09 &  3.98e-17 &      \\ 
     &           &    7 &  1.39e-09 &  3.98e-17 &      \\ 
     &           &    8 &  1.39e-09 &  3.97e-17 &      \\ 
     &           &    9 &  1.39e-09 &  3.97e-17 &      \\ 
     &           &   10 &  1.39e-09 &  3.96e-17 &      \\ 
2429 &  2.43e+04 &   10 &           &           & iters  \\ 
 \hdashline 
     &           &    1 &  1.01e-08 &  1.01e-09 &      \\ 
     &           &    2 &  1.01e-08 &  2.89e-16 &      \\ 
     &           &    3 &  1.01e-08 &  2.89e-16 &      \\ 
     &           &    4 &  6.76e-08 &  2.88e-16 &      \\ 
     &           &    5 &  1.02e-08 &  1.93e-15 &      \\ 
     &           &    6 &  1.02e-08 &  2.91e-16 &      \\ 
     &           &    7 &  1.02e-08 &  2.90e-16 &      \\ 
     &           &    8 &  1.01e-08 &  2.90e-16 &      \\ 
     &           &    9 &  1.01e-08 &  2.89e-16 &      \\ 
     &           &   10 &  1.01e-08 &  2.89e-16 &      \\ 
2430 &  2.43e+04 &   10 &           &           & iters  \\ 
 \hdashline 
     &           &    1 &  1.13e-07 &  1.69e-10 &      \\ 
     &           &    2 &  4.21e-08 &  3.24e-15 &      \\ 
     &           &    3 &  4.21e-08 &  1.20e-15 &      \\ 
     &           &    4 &  3.57e-08 &  1.20e-15 &      \\ 
     &           &    5 &  4.21e-08 &  1.02e-15 &      \\ 
     &           &    6 &  3.57e-08 &  1.20e-15 &      \\ 
     &           &    7 &  3.56e-08 &  1.02e-15 &      \\ 
     &           &    8 &  4.21e-08 &  1.02e-15 &      \\ 
     &           &    9 &  3.56e-08 &  1.20e-15 &      \\ 
     &           &   10 &  4.21e-08 &  1.02e-15 &      \\ 
2431 &  2.43e+04 &   10 &           &           & iters  \\ 
 \hdashline 
     &           &    1 &  4.32e-08 &  2.10e-10 &      \\ 
     &           &    2 &  3.45e-08 &  1.23e-15 &      \\ 
     &           &    3 &  4.32e-08 &  9.85e-16 &      \\ 
     &           &    4 &  3.45e-08 &  1.23e-15 &      \\ 
     &           &    5 &  3.45e-08 &  9.85e-16 &      \\ 
     &           &    6 &  4.33e-08 &  9.84e-16 &      \\ 
     &           &    7 &  3.45e-08 &  1.23e-15 &      \\ 
     &           &    8 &  4.33e-08 &  9.84e-16 &      \\ 
     &           &    9 &  3.45e-08 &  1.23e-15 &      \\ 
     &           &   10 &  4.32e-08 &  9.85e-16 &      \\ 
2432 &  2.43e+04 &   10 &           &           & iters  \\ 
 \hdashline 
     &           &    1 &  1.62e-08 &  6.86e-11 &      \\ 
     &           &    2 &  1.61e-08 &  4.61e-16 &      \\ 
     &           &    3 &  1.61e-08 &  4.61e-16 &      \\ 
     &           &    4 &  1.61e-08 &  4.60e-16 &      \\ 
     &           &    5 &  6.16e-08 &  4.59e-16 &      \\ 
     &           &    6 &  1.62e-08 &  1.76e-15 &      \\ 
     &           &    7 &  1.61e-08 &  4.61e-16 &      \\ 
     &           &    8 &  1.61e-08 &  4.60e-16 &      \\ 
     &           &    9 &  6.16e-08 &  4.60e-16 &      \\ 
     &           &   10 &  1.62e-08 &  1.76e-15 &      \\ 
2433 &  2.43e+04 &   10 &           &           & iters  \\ 
 \hdashline 
     &           &    1 &  2.89e-08 &  5.72e-10 &      \\ 
     &           &    2 &  2.88e-08 &  8.24e-16 &      \\ 
     &           &    3 &  4.89e-08 &  8.23e-16 &      \\ 
     &           &    4 &  2.89e-08 &  1.40e-15 &      \\ 
     &           &    5 &  2.88e-08 &  8.24e-16 &      \\ 
     &           &    6 &  4.89e-08 &  8.23e-16 &      \\ 
     &           &    7 &  2.89e-08 &  1.40e-15 &      \\ 
     &           &    8 &  2.88e-08 &  8.24e-16 &      \\ 
     &           &    9 &  4.89e-08 &  8.22e-16 &      \\ 
     &           &   10 &  2.88e-08 &  1.40e-15 &      \\ 
2434 &  2.43e+04 &   10 &           &           & iters  \\ 
 \hdashline 
     &           &    1 &  5.43e-08 &  1.11e-09 &      \\ 
     &           &    2 &  2.34e-08 &  1.55e-15 &      \\ 
     &           &    3 &  2.34e-08 &  6.69e-16 &      \\ 
     &           &    4 &  5.43e-08 &  6.68e-16 &      \\ 
     &           &    5 &  2.34e-08 &  1.55e-15 &      \\ 
     &           &    6 &  2.34e-08 &  6.69e-16 &      \\ 
     &           &    7 &  5.43e-08 &  6.68e-16 &      \\ 
     &           &    8 &  2.34e-08 &  1.55e-15 &      \\ 
     &           &    9 &  2.34e-08 &  6.69e-16 &      \\ 
     &           &   10 &  2.34e-08 &  6.68e-16 &      \\ 
2435 &  2.43e+04 &   10 &           &           & iters  \\ 
 \hdashline 
     &           &    1 &  1.16e-08 &  4.80e-10 &      \\ 
     &           &    2 &  1.16e-08 &  3.30e-16 &      \\ 
     &           &    3 &  1.15e-08 &  3.30e-16 &      \\ 
     &           &    4 &  6.62e-08 &  3.30e-16 &      \\ 
     &           &    5 &  1.16e-08 &  1.89e-15 &      \\ 
     &           &    6 &  1.16e-08 &  3.32e-16 &      \\ 
     &           &    7 &  1.16e-08 &  3.31e-16 &      \\ 
     &           &    8 &  1.16e-08 &  3.31e-16 &      \\ 
     &           &    9 &  1.16e-08 &  3.30e-16 &      \\ 
     &           &   10 &  6.61e-08 &  3.30e-16 &      \\ 
2436 &  2.44e+04 &   10 &           &           & iters  \\ 
 \hdashline 
     &           &    1 &  4.83e-09 &  2.18e-10 &      \\ 
     &           &    2 &  4.82e-09 &  1.38e-16 &      \\ 
     &           &    3 &  4.82e-09 &  1.38e-16 &      \\ 
     &           &    4 &  4.81e-09 &  1.37e-16 &      \\ 
     &           &    5 &  4.80e-09 &  1.37e-16 &      \\ 
     &           &    6 &  4.80e-09 &  1.37e-16 &      \\ 
     &           &    7 &  4.79e-09 &  1.37e-16 &      \\ 
     &           &    8 &  7.29e-08 &  1.37e-16 &      \\ 
     &           &    9 &  4.89e-09 &  2.08e-15 &      \\ 
     &           &   10 &  4.88e-09 &  1.39e-16 &      \\ 
2437 &  2.44e+04 &   10 &           &           & iters  \\ 
 \hdashline 
     &           &    1 &  4.07e-08 &  7.83e-10 &      \\ 
     &           &    2 &  3.70e-08 &  1.16e-15 &      \\ 
     &           &    3 &  4.07e-08 &  1.06e-15 &      \\ 
     &           &    4 &  3.70e-08 &  1.16e-15 &      \\ 
     &           &    5 &  4.07e-08 &  1.06e-15 &      \\ 
     &           &    6 &  3.71e-08 &  1.16e-15 &      \\ 
     &           &    7 &  4.07e-08 &  1.06e-15 &      \\ 
     &           &    8 &  3.71e-08 &  1.16e-15 &      \\ 
     &           &    9 &  4.07e-08 &  1.06e-15 &      \\ 
     &           &   10 &  3.71e-08 &  1.16e-15 &      \\ 
2438 &  2.44e+04 &   10 &           &           & iters  \\ 
 \hdashline 
     &           &    1 &  6.28e-08 &  1.89e-09 &      \\ 
     &           &    2 &  9.26e-08 &  1.79e-15 &      \\ 
     &           &    3 &  6.29e-08 &  2.64e-15 &      \\ 
     &           &    4 &  1.49e-08 &  1.79e-15 &      \\ 
     &           &    5 &  1.49e-08 &  4.25e-16 &      \\ 
     &           &    6 &  1.48e-08 &  4.24e-16 &      \\ 
     &           &    7 &  6.29e-08 &  4.24e-16 &      \\ 
     &           &    8 &  1.49e-08 &  1.79e-15 &      \\ 
     &           &    9 &  1.49e-08 &  4.26e-16 &      \\ 
     &           &   10 &  1.49e-08 &  4.25e-16 &      \\ 
2439 &  2.44e+04 &   10 &           &           & iters  \\ 
 \hdashline 
     &           &    1 &  1.12e-08 &  1.53e-09 &      \\ 
     &           &    2 &  1.12e-08 &  3.20e-16 &      \\ 
     &           &    3 &  6.65e-08 &  3.20e-16 &      \\ 
     &           &    4 &  1.13e-08 &  1.90e-15 &      \\ 
     &           &    5 &  1.13e-08 &  3.22e-16 &      \\ 
     &           &    6 &  1.12e-08 &  3.21e-16 &      \\ 
     &           &    7 &  1.12e-08 &  3.21e-16 &      \\ 
     &           &    8 &  1.12e-08 &  3.21e-16 &      \\ 
     &           &    9 &  1.12e-08 &  3.20e-16 &      \\ 
     &           &   10 &  6.65e-08 &  3.20e-16 &      \\ 
2440 &  2.44e+04 &   10 &           &           & iters  \\ 
 \hdashline 
     &           &    1 &  3.20e-08 &  2.50e-11 &      \\ 
     &           &    2 &  3.20e-08 &  9.14e-16 &      \\ 
     &           &    3 &  4.58e-08 &  9.12e-16 &      \\ 
     &           &    4 &  3.20e-08 &  1.31e-15 &      \\ 
     &           &    5 &  4.57e-08 &  9.13e-16 &      \\ 
     &           &    6 &  3.20e-08 &  1.31e-15 &      \\ 
     &           &    7 &  4.57e-08 &  9.13e-16 &      \\ 
     &           &    8 &  3.20e-08 &  1.31e-15 &      \\ 
     &           &    9 &  3.20e-08 &  9.14e-16 &      \\ 
     &           &   10 &  4.58e-08 &  9.13e-16 &      \\ 
2441 &  2.44e+04 &   10 &           &           & iters  \\ 
 \hdashline 
     &           &    1 &  2.44e-08 &  1.22e-10 &      \\ 
     &           &    2 &  2.43e-08 &  6.95e-16 &      \\ 
     &           &    3 &  5.34e-08 &  6.94e-16 &      \\ 
     &           &    4 &  2.44e-08 &  1.52e-15 &      \\ 
     &           &    5 &  2.43e-08 &  6.95e-16 &      \\ 
     &           &    6 &  5.34e-08 &  6.94e-16 &      \\ 
     &           &    7 &  2.44e-08 &  1.52e-15 &      \\ 
     &           &    8 &  2.43e-08 &  6.96e-16 &      \\ 
     &           &    9 &  5.34e-08 &  6.95e-16 &      \\ 
     &           &   10 &  2.44e-08 &  1.52e-15 &      \\ 
2442 &  2.44e+04 &   10 &           &           & iters  \\ 
 \hdashline 
     &           &    1 &  2.09e-08 &  8.83e-10 &      \\ 
     &           &    2 &  1.80e-08 &  5.97e-16 &      \\ 
     &           &    3 &  2.09e-08 &  5.12e-16 &      \\ 
     &           &    4 &  1.80e-08 &  5.97e-16 &      \\ 
     &           &    5 &  2.09e-08 &  5.13e-16 &      \\ 
     &           &    6 &  1.80e-08 &  5.97e-16 &      \\ 
     &           &    7 &  2.09e-08 &  5.13e-16 &      \\ 
     &           &    8 &  1.80e-08 &  5.97e-16 &      \\ 
     &           &    9 &  2.09e-08 &  5.13e-16 &      \\ 
     &           &   10 &  1.80e-08 &  5.97e-16 &      \\ 
2443 &  2.44e+04 &   10 &           &           & iters  \\ 
 \hdashline 
     &           &    1 &  2.92e-08 &  1.18e-09 &      \\ 
     &           &    2 &  4.85e-08 &  8.34e-16 &      \\ 
     &           &    3 &  2.92e-08 &  1.38e-15 &      \\ 
     &           &    4 &  2.92e-08 &  8.34e-16 &      \\ 
     &           &    5 &  4.85e-08 &  8.33e-16 &      \\ 
     &           &    6 &  2.92e-08 &  1.39e-15 &      \\ 
     &           &    7 &  4.85e-08 &  8.34e-16 &      \\ 
     &           &    8 &  2.92e-08 &  1.38e-15 &      \\ 
     &           &    9 &  2.92e-08 &  8.35e-16 &      \\ 
     &           &   10 &  4.85e-08 &  8.34e-16 &      \\ 
2444 &  2.44e+04 &   10 &           &           & iters  \\ 
 \hdashline 
     &           &    1 &  8.25e-09 &  9.27e-10 &      \\ 
     &           &    2 &  8.24e-09 &  2.35e-16 &      \\ 
     &           &    3 &  6.95e-08 &  2.35e-16 &      \\ 
     &           &    4 &  4.72e-08 &  1.98e-15 &      \\ 
     &           &    5 &  8.26e-09 &  1.35e-15 &      \\ 
     &           &    6 &  8.25e-09 &  2.36e-16 &      \\ 
     &           &    7 &  8.23e-09 &  2.35e-16 &      \\ 
     &           &    8 &  6.95e-08 &  2.35e-16 &      \\ 
     &           &    9 &  8.32e-09 &  1.98e-15 &      \\ 
     &           &   10 &  8.31e-09 &  2.38e-16 &      \\ 
2445 &  2.44e+04 &   10 &           &           & iters  \\ 
 \hdashline 
     &           &    1 &  1.78e-08 &  5.33e-10 &      \\ 
     &           &    2 &  2.10e-08 &  5.09e-16 &      \\ 
     &           &    3 &  1.78e-08 &  6.01e-16 &      \\ 
     &           &    4 &  2.10e-08 &  5.09e-16 &      \\ 
     &           &    5 &  1.78e-08 &  6.00e-16 &      \\ 
     &           &    6 &  1.78e-08 &  5.09e-16 &      \\ 
     &           &    7 &  2.11e-08 &  5.08e-16 &      \\ 
     &           &    8 &  1.78e-08 &  6.01e-16 &      \\ 
     &           &    9 &  2.11e-08 &  5.08e-16 &      \\ 
     &           &   10 &  1.78e-08 &  6.01e-16 &      \\ 
2446 &  2.44e+04 &   10 &           &           & iters  \\ 
 \hdashline 
     &           &    1 &  2.00e-08 &  9.57e-10 &      \\ 
     &           &    2 &  5.77e-08 &  5.72e-16 &      \\ 
     &           &    3 &  2.01e-08 &  1.65e-15 &      \\ 
     &           &    4 &  2.01e-08 &  5.73e-16 &      \\ 
     &           &    5 &  2.00e-08 &  5.73e-16 &      \\ 
     &           &    6 &  5.77e-08 &  5.72e-16 &      \\ 
     &           &    7 &  2.01e-08 &  1.65e-15 &      \\ 
     &           &    8 &  2.01e-08 &  5.73e-16 &      \\ 
     &           &    9 &  2.00e-08 &  5.72e-16 &      \\ 
     &           &   10 &  5.77e-08 &  5.72e-16 &      \\ 
2447 &  2.45e+04 &   10 &           &           & iters  \\ 
 \hdashline 
     &           &    1 &  6.60e-09 &  1.98e-09 &      \\ 
     &           &    2 &  6.59e-09 &  1.88e-16 &      \\ 
     &           &    3 &  7.11e-08 &  1.88e-16 &      \\ 
     &           &    4 &  4.55e-08 &  2.03e-15 &      \\ 
     &           &    5 &  6.62e-09 &  1.30e-15 &      \\ 
     &           &    6 &  6.61e-09 &  1.89e-16 &      \\ 
     &           &    7 &  6.60e-09 &  1.89e-16 &      \\ 
     &           &    8 &  6.59e-09 &  1.88e-16 &      \\ 
     &           &    9 &  7.11e-08 &  1.88e-16 &      \\ 
     &           &   10 &  4.55e-08 &  2.03e-15 &      \\ 
2448 &  2.45e+04 &   10 &           &           & iters  \\ 
 \hdashline 
     &           &    1 &  7.31e-09 &  4.83e-10 &      \\ 
     &           &    2 &  3.15e-08 &  2.09e-16 &      \\ 
     &           &    3 &  7.35e-09 &  9.00e-16 &      \\ 
     &           &    4 &  7.34e-09 &  2.10e-16 &      \\ 
     &           &    5 &  7.32e-09 &  2.09e-16 &      \\ 
     &           &    6 &  7.31e-09 &  2.09e-16 &      \\ 
     &           &    7 &  3.15e-08 &  2.09e-16 &      \\ 
     &           &    8 &  7.35e-09 &  9.00e-16 &      \\ 
     &           &    9 &  7.34e-09 &  2.10e-16 &      \\ 
     &           &   10 &  7.33e-09 &  2.09e-16 &      \\ 
2449 &  2.45e+04 &   10 &           &           & iters  \\ 
 \hdashline 
     &           &    1 &  6.56e-09 &  1.11e-09 &      \\ 
     &           &    2 &  6.55e-09 &  1.87e-16 &      \\ 
     &           &    3 &  6.54e-09 &  1.87e-16 &      \\ 
     &           &    4 &  6.53e-09 &  1.87e-16 &      \\ 
     &           &    5 &  6.52e-09 &  1.86e-16 &      \\ 
     &           &    6 &  6.52e-09 &  1.86e-16 &      \\ 
     &           &    7 &  6.51e-09 &  1.86e-16 &      \\ 
     &           &    8 &  7.12e-08 &  1.86e-16 &      \\ 
     &           &    9 &  4.54e-08 &  2.03e-15 &      \\ 
     &           &   10 &  6.53e-09 &  1.30e-15 &      \\ 
2450 &  2.45e+04 &   10 &           &           & iters  \\ 
 \hdashline 
     &           &    1 &  6.18e-08 &  6.32e-10 &      \\ 
     &           &    2 &  1.60e-08 &  1.76e-15 &      \\ 
     &           &    3 &  1.60e-08 &  4.57e-16 &      \\ 
     &           &    4 &  1.60e-08 &  4.56e-16 &      \\ 
     &           &    5 &  1.59e-08 &  4.56e-16 &      \\ 
     &           &    6 &  6.18e-08 &  4.55e-16 &      \\ 
     &           &    7 &  1.60e-08 &  1.76e-15 &      \\ 
     &           &    8 &  1.60e-08 &  4.57e-16 &      \\ 
     &           &    9 &  1.60e-08 &  4.56e-16 &      \\ 
     &           &   10 &  1.59e-08 &  4.56e-16 &      \\ 
2451 &  2.45e+04 &   10 &           &           & iters  \\ 
 \hdashline 
     &           &    1 &  1.60e-08 &  3.27e-10 &      \\ 
     &           &    2 &  1.60e-08 &  4.57e-16 &      \\ 
     &           &    3 &  1.60e-08 &  4.56e-16 &      \\ 
     &           &    4 &  1.59e-08 &  4.56e-16 &      \\ 
     &           &    5 &  6.18e-08 &  4.55e-16 &      \\ 
     &           &    6 &  1.60e-08 &  1.76e-15 &      \\ 
     &           &    7 &  1.60e-08 &  4.57e-16 &      \\ 
     &           &    8 &  1.60e-08 &  4.56e-16 &      \\ 
     &           &    9 &  1.59e-08 &  4.56e-16 &      \\ 
     &           &   10 &  6.18e-08 &  4.55e-16 &      \\ 
2452 &  2.45e+04 &   10 &           &           & iters  \\ 
 \hdashline 
     &           &    1 &  1.94e-08 &  9.22e-10 &      \\ 
     &           &    2 &  1.93e-08 &  5.53e-16 &      \\ 
     &           &    3 &  5.84e-08 &  5.52e-16 &      \\ 
     &           &    4 &  1.94e-08 &  1.67e-15 &      \\ 
     &           &    5 &  1.94e-08 &  5.54e-16 &      \\ 
     &           &    6 &  1.93e-08 &  5.53e-16 &      \\ 
     &           &    7 &  5.84e-08 &  5.52e-16 &      \\ 
     &           &    8 &  1.94e-08 &  1.67e-15 &      \\ 
     &           &    9 &  1.94e-08 &  5.54e-16 &      \\ 
     &           &   10 &  1.93e-08 &  5.53e-16 &      \\ 
2453 &  2.45e+04 &   10 &           &           & iters  \\ 
 \hdashline 
     &           &    1 &  3.16e-09 &  1.37e-10 &      \\ 
     &           &    2 &  3.15e-09 &  9.02e-17 &      \\ 
     &           &    3 &  3.15e-09 &  9.00e-17 &      \\ 
     &           &    4 &  3.15e-09 &  8.99e-17 &      \\ 
     &           &    5 &  3.14e-09 &  8.98e-17 &      \\ 
     &           &    6 &  3.14e-09 &  8.96e-17 &      \\ 
     &           &    7 &  7.46e-08 &  8.95e-17 &      \\ 
     &           &    8 &  4.21e-08 &  2.13e-15 &      \\ 
     &           &    9 &  3.18e-09 &  1.20e-15 &      \\ 
     &           &   10 &  3.17e-09 &  9.07e-17 &      \\ 
2454 &  2.45e+04 &   10 &           &           & iters  \\ 
 \hdashline 
     &           &    1 &  3.18e-09 &  2.59e-09 &      \\ 
     &           &    2 &  3.18e-09 &  9.08e-17 &      \\ 
     &           &    3 &  3.17e-09 &  9.07e-17 &      \\ 
     &           &    4 &  3.17e-09 &  9.06e-17 &      \\ 
     &           &    5 &  3.16e-09 &  9.04e-17 &      \\ 
     &           &    6 &  3.16e-09 &  9.03e-17 &      \\ 
     &           &    7 &  3.16e-09 &  9.02e-17 &      \\ 
     &           &    8 &  3.15e-09 &  9.00e-17 &      \\ 
     &           &    9 &  3.15e-09 &  8.99e-17 &      \\ 
     &           &   10 &  3.14e-09 &  8.98e-17 &      \\ 
2455 &  2.45e+04 &   10 &           &           & iters  \\ 
 \hdashline 
     &           &    1 &  1.85e-09 &  2.45e-09 &      \\ 
     &           &    2 &  1.84e-09 &  5.27e-17 &      \\ 
     &           &    3 &  1.84e-09 &  5.26e-17 &      \\ 
     &           &    4 &  1.84e-09 &  5.25e-17 &      \\ 
     &           &    5 &  1.83e-09 &  5.24e-17 &      \\ 
     &           &    6 &  1.83e-09 &  5.24e-17 &      \\ 
     &           &    7 &  1.83e-09 &  5.23e-17 &      \\ 
     &           &    8 &  1.83e-09 &  5.22e-17 &      \\ 
     &           &    9 &  1.82e-09 &  5.21e-17 &      \\ 
     &           &   10 &  1.82e-09 &  5.21e-17 &      \\ 
2456 &  2.46e+04 &   10 &           &           & iters  \\ 
 \hdashline 
     &           &    1 &  4.06e-08 &  6.79e-10 &      \\ 
     &           &    2 &  3.72e-08 &  1.16e-15 &      \\ 
     &           &    3 &  4.06e-08 &  1.06e-15 &      \\ 
     &           &    4 &  3.72e-08 &  1.16e-15 &      \\ 
     &           &    5 &  3.71e-08 &  1.06e-15 &      \\ 
     &           &    6 &  4.06e-08 &  1.06e-15 &      \\ 
     &           &    7 &  3.71e-08 &  1.16e-15 &      \\ 
     &           &    8 &  4.06e-08 &  1.06e-15 &      \\ 
     &           &    9 &  3.71e-08 &  1.16e-15 &      \\ 
     &           &   10 &  4.06e-08 &  1.06e-15 &      \\ 
2457 &  2.46e+04 &   10 &           &           & iters  \\ 
 \hdashline 
     &           &    1 &  4.25e-08 &  1.26e-09 &      \\ 
     &           &    2 &  3.52e-08 &  1.21e-15 &      \\ 
     &           &    3 &  4.25e-08 &  1.00e-15 &      \\ 
     &           &    4 &  3.52e-08 &  1.21e-15 &      \\ 
     &           &    5 &  4.25e-08 &  1.01e-15 &      \\ 
     &           &    6 &  3.52e-08 &  1.21e-15 &      \\ 
     &           &    7 &  4.25e-08 &  1.01e-15 &      \\ 
     &           &    8 &  3.52e-08 &  1.21e-15 &      \\ 
     &           &    9 &  4.25e-08 &  1.01e-15 &      \\ 
     &           &   10 &  3.53e-08 &  1.21e-15 &      \\ 
2458 &  2.46e+04 &   10 &           &           & iters  \\ 
 \hdashline 
     &           &    1 &  1.86e-08 &  2.69e-09 &      \\ 
     &           &    2 &  1.86e-08 &  5.31e-16 &      \\ 
     &           &    3 &  5.91e-08 &  5.30e-16 &      \\ 
     &           &    4 &  1.86e-08 &  1.69e-15 &      \\ 
     &           &    5 &  1.86e-08 &  5.32e-16 &      \\ 
     &           &    6 &  1.86e-08 &  5.31e-16 &      \\ 
     &           &    7 &  5.91e-08 &  5.30e-16 &      \\ 
     &           &    8 &  1.86e-08 &  1.69e-15 &      \\ 
     &           &    9 &  1.86e-08 &  5.32e-16 &      \\ 
     &           &   10 &  1.86e-08 &  5.31e-16 &      \\ 
2459 &  2.46e+04 &   10 &           &           & iters  \\ 
 \hdashline 
     &           &    1 &  5.68e-08 &  3.13e-09 &      \\ 
     &           &    2 &  2.10e-08 &  1.62e-15 &      \\ 
     &           &    3 &  2.09e-08 &  5.99e-16 &      \\ 
     &           &    4 &  5.68e-08 &  5.98e-16 &      \\ 
     &           &    5 &  2.10e-08 &  1.62e-15 &      \\ 
     &           &    6 &  2.10e-08 &  5.99e-16 &      \\ 
     &           &    7 &  2.09e-08 &  5.98e-16 &      \\ 
     &           &    8 &  5.68e-08 &  5.98e-16 &      \\ 
     &           &    9 &  2.10e-08 &  1.62e-15 &      \\ 
     &           &   10 &  2.10e-08 &  5.99e-16 &      \\ 
2460 &  2.46e+04 &   10 &           &           & iters  \\ 
 \hdashline 
     &           &    1 &  1.80e-08 &  1.47e-09 &      \\ 
     &           &    2 &  1.80e-08 &  5.13e-16 &      \\ 
     &           &    3 &  5.98e-08 &  5.13e-16 &      \\ 
     &           &    4 &  1.80e-08 &  1.71e-15 &      \\ 
     &           &    5 &  1.80e-08 &  5.14e-16 &      \\ 
     &           &    6 &  1.80e-08 &  5.14e-16 &      \\ 
     &           &    7 &  1.79e-08 &  5.13e-16 &      \\ 
     &           &    8 &  5.98e-08 &  5.12e-16 &      \\ 
     &           &    9 &  1.80e-08 &  1.71e-15 &      \\ 
     &           &   10 &  1.80e-08 &  5.14e-16 &      \\ 
2461 &  2.46e+04 &   10 &           &           & iters  \\ 
 \hdashline 
     &           &    1 &  3.19e-08 &  1.63e-09 &      \\ 
     &           &    2 &  4.58e-08 &  9.11e-16 &      \\ 
     &           &    3 &  3.20e-08 &  1.31e-15 &      \\ 
     &           &    4 &  3.19e-08 &  9.12e-16 &      \\ 
     &           &    5 &  4.58e-08 &  9.11e-16 &      \\ 
     &           &    6 &  3.19e-08 &  1.31e-15 &      \\ 
     &           &    7 &  4.58e-08 &  9.11e-16 &      \\ 
     &           &    8 &  3.19e-08 &  1.31e-15 &      \\ 
     &           &    9 &  4.58e-08 &  9.12e-16 &      \\ 
     &           &   10 &  3.20e-08 &  1.31e-15 &      \\ 
2462 &  2.46e+04 &   10 &           &           & iters  \\ 
 \hdashline 
     &           &    1 &  3.01e-08 &  1.27e-09 &      \\ 
     &           &    2 &  4.76e-08 &  8.60e-16 &      \\ 
     &           &    3 &  3.02e-08 &  1.36e-15 &      \\ 
     &           &    4 &  3.01e-08 &  8.61e-16 &      \\ 
     &           &    5 &  4.76e-08 &  8.59e-16 &      \\ 
     &           &    6 &  3.01e-08 &  1.36e-15 &      \\ 
     &           &    7 &  4.76e-08 &  8.60e-16 &      \\ 
     &           &    8 &  3.02e-08 &  1.36e-15 &      \\ 
     &           &    9 &  3.01e-08 &  8.61e-16 &      \\ 
     &           &   10 &  4.76e-08 &  8.60e-16 &      \\ 
2463 &  2.46e+04 &   10 &           &           & iters  \\ 
 \hdashline 
     &           &    1 &  2.51e-08 &  7.47e-10 &      \\ 
     &           &    2 &  2.50e-08 &  7.15e-16 &      \\ 
     &           &    3 &  5.27e-08 &  7.14e-16 &      \\ 
     &           &    4 &  2.51e-08 &  1.50e-15 &      \\ 
     &           &    5 &  2.50e-08 &  7.16e-16 &      \\ 
     &           &    6 &  5.27e-08 &  7.14e-16 &      \\ 
     &           &    7 &  2.51e-08 &  1.50e-15 &      \\ 
     &           &    8 &  2.50e-08 &  7.16e-16 &      \\ 
     &           &    9 &  5.27e-08 &  7.15e-16 &      \\ 
     &           &   10 &  2.51e-08 &  1.50e-15 &      \\ 
2464 &  2.46e+04 &   10 &           &           & iters  \\ 
 \hdashline 
     &           &    1 &  6.52e-09 &  5.45e-10 &      \\ 
     &           &    2 &  6.51e-09 &  1.86e-16 &      \\ 
     &           &    3 &  6.50e-09 &  1.86e-16 &      \\ 
     &           &    4 &  6.49e-09 &  1.86e-16 &      \\ 
     &           &    5 &  6.48e-09 &  1.85e-16 &      \\ 
     &           &    6 &  6.48e-09 &  1.85e-16 &      \\ 
     &           &    7 &  6.47e-09 &  1.85e-16 &      \\ 
     &           &    8 &  7.12e-08 &  1.85e-16 &      \\ 
     &           &    9 &  8.42e-08 &  2.03e-15 &      \\ 
     &           &   10 &  7.13e-08 &  2.40e-15 &      \\ 
2465 &  2.46e+04 &   10 &           &           & iters  \\ 
 \hdashline 
     &           &    1 &  1.59e-08 &  3.05e-09 &      \\ 
     &           &    2 &  1.58e-08 &  4.53e-16 &      \\ 
     &           &    3 &  1.58e-08 &  4.52e-16 &      \\ 
     &           &    4 &  6.19e-08 &  4.52e-16 &      \\ 
     &           &    5 &  1.59e-08 &  1.77e-15 &      \\ 
     &           &    6 &  1.59e-08 &  4.54e-16 &      \\ 
     &           &    7 &  1.58e-08 &  4.53e-16 &      \\ 
     &           &    8 &  6.19e-08 &  4.52e-16 &      \\ 
     &           &    9 &  1.59e-08 &  1.77e-15 &      \\ 
     &           &   10 &  1.59e-08 &  4.54e-16 &      \\ 
2466 &  2.46e+04 &   10 &           &           & iters  \\ 
 \hdashline 
     &           &    1 &  2.89e-08 &  3.77e-09 &      \\ 
     &           &    2 &  2.89e-08 &  8.26e-16 &      \\ 
     &           &    3 &  4.88e-08 &  8.25e-16 &      \\ 
     &           &    4 &  2.89e-08 &  1.39e-15 &      \\ 
     &           &    5 &  2.89e-08 &  8.25e-16 &      \\ 
     &           &    6 &  4.89e-08 &  8.24e-16 &      \\ 
     &           &    7 &  2.89e-08 &  1.39e-15 &      \\ 
     &           &    8 &  2.89e-08 &  8.25e-16 &      \\ 
     &           &    9 &  4.89e-08 &  8.24e-16 &      \\ 
     &           &   10 &  2.89e-08 &  1.39e-15 &      \\ 
2467 &  2.47e+04 &   10 &           &           & iters  \\ 
 \hdashline 
     &           &    1 &  5.03e-08 &  2.27e-09 &      \\ 
     &           &    2 &  2.75e-08 &  1.43e-15 &      \\ 
     &           &    3 &  2.75e-08 &  7.85e-16 &      \\ 
     &           &    4 &  5.03e-08 &  7.84e-16 &      \\ 
     &           &    5 &  2.75e-08 &  1.43e-15 &      \\ 
     &           &    6 &  2.75e-08 &  7.85e-16 &      \\ 
     &           &    7 &  5.03e-08 &  7.83e-16 &      \\ 
     &           &    8 &  2.75e-08 &  1.43e-15 &      \\ 
     &           &    9 &  5.02e-08 &  7.84e-16 &      \\ 
     &           &   10 &  2.75e-08 &  1.43e-15 &      \\ 
2468 &  2.47e+04 &   10 &           &           & iters  \\ 
 \hdashline 
     &           &    1 &  4.91e-08 &  9.37e-10 &      \\ 
     &           &    2 &  2.87e-08 &  1.40e-15 &      \\ 
     &           &    3 &  2.86e-08 &  8.18e-16 &      \\ 
     &           &    4 &  4.91e-08 &  8.17e-16 &      \\ 
     &           &    5 &  2.86e-08 &  1.40e-15 &      \\ 
     &           &    6 &  2.86e-08 &  8.18e-16 &      \\ 
     &           &    7 &  4.91e-08 &  8.16e-16 &      \\ 
     &           &    8 &  2.86e-08 &  1.40e-15 &      \\ 
     &           &    9 &  4.91e-08 &  8.17e-16 &      \\ 
     &           &   10 &  2.87e-08 &  1.40e-15 &      \\ 
2469 &  2.47e+04 &   10 &           &           & iters  \\ 
 \hdashline 
     &           &    1 &  1.25e-08 &  9.81e-10 &      \\ 
     &           &    2 &  1.25e-08 &  3.58e-16 &      \\ 
     &           &    3 &  1.25e-08 &  3.58e-16 &      \\ 
     &           &    4 &  2.64e-08 &  3.57e-16 &      \\ 
     &           &    5 &  1.25e-08 &  7.52e-16 &      \\ 
     &           &    6 &  1.25e-08 &  3.58e-16 &      \\ 
     &           &    7 &  2.64e-08 &  3.57e-16 &      \\ 
     &           &    8 &  1.25e-08 &  7.52e-16 &      \\ 
     &           &    9 &  1.25e-08 &  3.58e-16 &      \\ 
     &           &   10 &  2.64e-08 &  3.57e-16 &      \\ 
2470 &  2.47e+04 &   10 &           &           & iters  \\ 
 \hdashline 
     &           &    1 &  4.46e-08 &  9.02e-10 &      \\ 
     &           &    2 &  3.32e-08 &  1.27e-15 &      \\ 
     &           &    3 &  3.32e-08 &  9.48e-16 &      \\ 
     &           &    4 &  4.46e-08 &  9.46e-16 &      \\ 
     &           &    5 &  3.32e-08 &  1.27e-15 &      \\ 
     &           &    6 &  4.46e-08 &  9.47e-16 &      \\ 
     &           &    7 &  3.32e-08 &  1.27e-15 &      \\ 
     &           &    8 &  4.46e-08 &  9.47e-16 &      \\ 
     &           &    9 &  3.32e-08 &  1.27e-15 &      \\ 
     &           &   10 &  3.32e-08 &  9.48e-16 &      \\ 
2471 &  2.47e+04 &   10 &           &           & iters  \\ 
 \hdashline 
     &           &    1 &  2.42e-09 &  7.48e-10 &      \\ 
     &           &    2 &  2.42e-09 &  6.90e-17 &      \\ 
     &           &    3 &  2.41e-09 &  6.89e-17 &      \\ 
     &           &    4 &  2.41e-09 &  6.88e-17 &      \\ 
     &           &    5 &  2.41e-09 &  6.87e-17 &      \\ 
     &           &    6 &  2.40e-09 &  6.86e-17 &      \\ 
     &           &    7 &  2.40e-09 &  6.86e-17 &      \\ 
     &           &    8 &  2.40e-09 &  6.85e-17 &      \\ 
     &           &    9 &  7.53e-08 &  6.84e-17 &      \\ 
     &           &   10 &  4.13e-08 &  2.15e-15 &      \\ 
2472 &  2.47e+04 &   10 &           &           & iters  \\ 
 \hdashline 
     &           &    1 &  2.87e-08 &  1.77e-11 &      \\ 
     &           &    2 &  1.02e-08 &  8.20e-16 &      \\ 
     &           &    3 &  1.02e-08 &  2.90e-16 &      \\ 
     &           &    4 &  1.01e-08 &  2.90e-16 &      \\ 
     &           &    5 &  2.87e-08 &  2.89e-16 &      \\ 
     &           &    6 &  1.02e-08 &  8.20e-16 &      \\ 
     &           &    7 &  1.02e-08 &  2.90e-16 &      \\ 
     &           &    8 &  2.87e-08 &  2.90e-16 &      \\ 
     &           &    9 &  1.02e-08 &  8.19e-16 &      \\ 
     &           &   10 &  1.02e-08 &  2.90e-16 &      \\ 
2473 &  2.47e+04 &   10 &           &           & iters  \\ 
 \hdashline 
     &           &    1 &  3.31e-08 &  1.22e-09 &      \\ 
     &           &    2 &  4.47e-08 &  9.44e-16 &      \\ 
     &           &    3 &  3.31e-08 &  1.27e-15 &      \\ 
     &           &    4 &  4.46e-08 &  9.45e-16 &      \\ 
     &           &    5 &  3.31e-08 &  1.27e-15 &      \\ 
     &           &    6 &  3.31e-08 &  9.45e-16 &      \\ 
     &           &    7 &  4.47e-08 &  9.44e-16 &      \\ 
     &           &    8 &  3.31e-08 &  1.27e-15 &      \\ 
     &           &    9 &  4.46e-08 &  9.44e-16 &      \\ 
     &           &   10 &  3.31e-08 &  1.27e-15 &      \\ 
2474 &  2.47e+04 &   10 &           &           & iters  \\ 
 \hdashline 
     &           &    1 &  3.42e-08 &  1.18e-09 &      \\ 
     &           &    2 &  4.69e-09 &  9.76e-16 &      \\ 
     &           &    3 &  4.69e-09 &  1.34e-16 &      \\ 
     &           &    4 &  4.68e-09 &  1.34e-16 &      \\ 
     &           &    5 &  4.67e-09 &  1.34e-16 &      \\ 
     &           &    6 &  4.67e-09 &  1.33e-16 &      \\ 
     &           &    7 &  3.42e-08 &  1.33e-16 &      \\ 
     &           &    8 &  4.71e-09 &  9.76e-16 &      \\ 
     &           &    9 &  4.70e-09 &  1.34e-16 &      \\ 
     &           &   10 &  4.70e-09 &  1.34e-16 &      \\ 
2475 &  2.47e+04 &   10 &           &           & iters  \\ 
 \hdashline 
     &           &    1 &  2.03e-08 &  6.11e-10 &      \\ 
     &           &    2 &  5.74e-08 &  5.80e-16 &      \\ 
     &           &    3 &  2.04e-08 &  1.64e-15 &      \\ 
     &           &    4 &  2.04e-08 &  5.82e-16 &      \\ 
     &           &    5 &  2.03e-08 &  5.81e-16 &      \\ 
     &           &    6 &  5.74e-08 &  5.80e-16 &      \\ 
     &           &    7 &  2.04e-08 &  1.64e-15 &      \\ 
     &           &    8 &  2.03e-08 &  5.82e-16 &      \\ 
     &           &    9 &  2.03e-08 &  5.81e-16 &      \\ 
     &           &   10 &  5.74e-08 &  5.80e-16 &      \\ 
2476 &  2.48e+04 &   10 &           &           & iters  \\ 
 \hdashline 
     &           &    1 &  1.94e-09 &  5.47e-10 &      \\ 
     &           &    2 &  1.94e-09 &  5.54e-17 &      \\ 
     &           &    3 &  1.94e-09 &  5.54e-17 &      \\ 
     &           &    4 &  1.93e-09 &  5.53e-17 &      \\ 
     &           &    5 &  1.93e-09 &  5.52e-17 &      \\ 
     &           &    6 &  1.93e-09 &  5.51e-17 &      \\ 
     &           &    7 &  1.93e-09 &  5.50e-17 &      \\ 
     &           &    8 &  1.92e-09 &  5.50e-17 &      \\ 
     &           &    9 &  1.92e-09 &  5.49e-17 &      \\ 
     &           &   10 &  1.92e-09 &  5.48e-17 &      \\ 
2477 &  2.48e+04 &   10 &           &           & iters  \\ 
 \hdashline 
     &           &    1 &  2.56e-08 &  9.81e-10 &      \\ 
     &           &    2 &  1.33e-08 &  7.31e-16 &      \\ 
     &           &    3 &  2.56e-08 &  3.79e-16 &      \\ 
     &           &    4 &  1.33e-08 &  7.30e-16 &      \\ 
     &           &    5 &  1.33e-08 &  3.80e-16 &      \\ 
     &           &    6 &  2.56e-08 &  3.79e-16 &      \\ 
     &           &    7 &  1.33e-08 &  7.30e-16 &      \\ 
     &           &    8 &  1.33e-08 &  3.79e-16 &      \\ 
     &           &    9 &  2.56e-08 &  3.79e-16 &      \\ 
     &           &   10 &  1.33e-08 &  7.30e-16 &      \\ 
2478 &  2.48e+04 &   10 &           &           & iters  \\ 
 \hdashline 
     &           &    1 &  3.22e-08 &  1.31e-09 &      \\ 
     &           &    2 &  4.55e-08 &  9.19e-16 &      \\ 
     &           &    3 &  3.22e-08 &  1.30e-15 &      \\ 
     &           &    4 &  3.22e-08 &  9.19e-16 &      \\ 
     &           &    5 &  4.56e-08 &  9.18e-16 &      \\ 
     &           &    6 &  3.22e-08 &  1.30e-15 &      \\ 
     &           &    7 &  4.56e-08 &  9.19e-16 &      \\ 
     &           &    8 &  3.22e-08 &  1.30e-15 &      \\ 
     &           &    9 &  3.22e-08 &  9.19e-16 &      \\ 
     &           &   10 &  4.56e-08 &  9.18e-16 &      \\ 
2479 &  2.48e+04 &   10 &           &           & iters  \\ 
 \hdashline 
     &           &    1 &  7.33e-09 &  9.14e-10 &      \\ 
     &           &    2 &  7.32e-09 &  2.09e-16 &      \\ 
     &           &    3 &  7.04e-08 &  2.09e-16 &      \\ 
     &           &    4 &  8.51e-08 &  2.01e-15 &      \\ 
     &           &    5 &  7.04e-08 &  2.43e-15 &      \\ 
     &           &    6 &  7.39e-09 &  2.01e-15 &      \\ 
     &           &    7 &  7.38e-09 &  2.11e-16 &      \\ 
     &           &    8 &  7.37e-09 &  2.11e-16 &      \\ 
     &           &    9 &  7.36e-09 &  2.10e-16 &      \\ 
     &           &   10 &  7.35e-09 &  2.10e-16 &      \\ 
2480 &  2.48e+04 &   10 &           &           & iters  \\ 
 \hdashline 
     &           &    1 &  1.75e-08 &  6.45e-10 &      \\ 
     &           &    2 &  6.02e-08 &  4.99e-16 &      \\ 
     &           &    3 &  1.75e-08 &  1.72e-15 &      \\ 
     &           &    4 &  1.75e-08 &  5.01e-16 &      \\ 
     &           &    5 &  1.75e-08 &  5.00e-16 &      \\ 
     &           &    6 &  6.02e-08 &  4.99e-16 &      \\ 
     &           &    7 &  1.76e-08 &  1.72e-15 &      \\ 
     &           &    8 &  1.75e-08 &  5.01e-16 &      \\ 
     &           &    9 &  1.75e-08 &  5.00e-16 &      \\ 
     &           &   10 &  1.75e-08 &  4.99e-16 &      \\ 
2481 &  2.48e+04 &   10 &           &           & iters  \\ 
 \hdashline 
     &           &    1 &  4.32e-08 &  8.51e-10 &      \\ 
     &           &    2 &  3.45e-08 &  1.23e-15 &      \\ 
     &           &    3 &  4.32e-08 &  9.86e-16 &      \\ 
     &           &    4 &  3.46e-08 &  1.23e-15 &      \\ 
     &           &    5 &  3.45e-08 &  9.86e-16 &      \\ 
     &           &    6 &  4.32e-08 &  9.85e-16 &      \\ 
     &           &    7 &  3.45e-08 &  1.23e-15 &      \\ 
     &           &    8 &  4.32e-08 &  9.85e-16 &      \\ 
     &           &    9 &  3.45e-08 &  1.23e-15 &      \\ 
     &           &   10 &  4.32e-08 &  9.85e-16 &      \\ 
2482 &  2.48e+04 &   10 &           &           & iters  \\ 
 \hdashline 
     &           &    1 &  6.38e-08 &  9.31e-10 &      \\ 
     &           &    2 &  1.40e-08 &  1.82e-15 &      \\ 
     &           &    3 &  6.37e-08 &  3.99e-16 &      \\ 
     &           &    4 &  9.18e-08 &  1.82e-15 &      \\ 
     &           &    5 &  6.38e-08 &  2.62e-15 &      \\ 
     &           &    6 &  1.40e-08 &  1.82e-15 &      \\ 
     &           &    7 &  1.40e-08 &  4.00e-16 &      \\ 
     &           &    8 &  1.40e-08 &  4.00e-16 &      \\ 
     &           &    9 &  6.37e-08 &  3.99e-16 &      \\ 
     &           &   10 &  1.41e-08 &  1.82e-15 &      \\ 
2483 &  2.48e+04 &   10 &           &           & iters  \\ 
 \hdashline 
     &           &    1 &  9.93e-08 &  9.28e-10 &      \\ 
     &           &    2 &  5.63e-08 &  2.83e-15 &      \\ 
     &           &    3 &  2.15e-08 &  1.61e-15 &      \\ 
     &           &    4 &  2.15e-08 &  6.14e-16 &      \\ 
     &           &    5 &  5.62e-08 &  6.13e-16 &      \\ 
     &           &    6 &  2.15e-08 &  1.61e-15 &      \\ 
     &           &    7 &  2.15e-08 &  6.14e-16 &      \\ 
     &           &    8 &  5.62e-08 &  6.14e-16 &      \\ 
     &           &    9 &  2.15e-08 &  1.60e-15 &      \\ 
     &           &   10 &  2.15e-08 &  6.15e-16 &      \\ 
2484 &  2.48e+04 &   10 &           &           & iters  \\ 
 \hdashline 
     &           &    1 &  4.54e-08 &  9.52e-10 &      \\ 
     &           &    2 &  3.24e-08 &  1.29e-15 &      \\ 
     &           &    3 &  3.23e-08 &  9.25e-16 &      \\ 
     &           &    4 &  4.54e-08 &  9.23e-16 &      \\ 
     &           &    5 &  3.24e-08 &  1.30e-15 &      \\ 
     &           &    6 &  4.54e-08 &  9.24e-16 &      \\ 
     &           &    7 &  3.24e-08 &  1.29e-15 &      \\ 
     &           &    8 &  3.23e-08 &  9.24e-16 &      \\ 
     &           &    9 &  4.54e-08 &  9.23e-16 &      \\ 
     &           &   10 &  3.24e-08 &  1.30e-15 &      \\ 
2485 &  2.48e+04 &   10 &           &           & iters  \\ 
 \hdashline 
     &           &    1 &  4.67e-08 &  9.17e-10 &      \\ 
     &           &    2 &  3.11e-08 &  1.33e-15 &      \\ 
     &           &    3 &  4.66e-08 &  8.87e-16 &      \\ 
     &           &    4 &  3.11e-08 &  1.33e-15 &      \\ 
     &           &    5 &  3.11e-08 &  8.88e-16 &      \\ 
     &           &    6 &  4.67e-08 &  8.87e-16 &      \\ 
     &           &    7 &  3.11e-08 &  1.33e-15 &      \\ 
     &           &    8 &  4.66e-08 &  8.87e-16 &      \\ 
     &           &    9 &  3.11e-08 &  1.33e-15 &      \\ 
     &           &   10 &  3.11e-08 &  8.88e-16 &      \\ 
2486 &  2.48e+04 &   10 &           &           & iters  \\ 
 \hdashline 
     &           &    1 &  2.22e-08 &  9.03e-10 &      \\ 
     &           &    2 &  2.22e-08 &  6.35e-16 &      \\ 
     &           &    3 &  2.22e-08 &  6.34e-16 &      \\ 
     &           &    4 &  5.56e-08 &  6.33e-16 &      \\ 
     &           &    5 &  2.22e-08 &  1.59e-15 &      \\ 
     &           &    6 &  2.22e-08 &  6.34e-16 &      \\ 
     &           &    7 &  2.22e-08 &  6.33e-16 &      \\ 
     &           &    8 &  5.56e-08 &  6.32e-16 &      \\ 
     &           &    9 &  2.22e-08 &  1.59e-15 &      \\ 
     &           &   10 &  2.22e-08 &  6.34e-16 &      \\ 
2487 &  2.49e+04 &   10 &           &           & iters  \\ 
 \hdashline 
     &           &    1 &  7.80e-08 &  1.19e-10 &      \\ 
     &           &    2 &  7.75e-08 &  2.22e-15 &      \\ 
     &           &    3 &  7.80e-08 &  2.21e-15 &      \\ 
     &           &    4 &  7.75e-08 &  2.22e-15 &      \\ 
     &           &    5 &  7.80e-08 &  2.21e-15 &      \\ 
     &           &    6 &  7.75e-08 &  2.22e-15 &      \\ 
     &           &    7 &  7.80e-08 &  2.21e-15 &      \\ 
     &           &    8 &  7.75e-08 &  2.22e-15 &      \\ 
     &           &    9 &  7.80e-08 &  2.21e-15 &      \\ 
     &           &   10 &  7.75e-08 &  2.22e-15 &      \\ 
2488 &  2.49e+04 &   10 &           &           & iters  \\ 
 \hdashline 
     &           &    1 &  8.27e-08 &  1.18e-09 &      \\ 
     &           &    2 &  7.28e-08 &  2.36e-15 &      \\ 
     &           &    3 &  8.27e-08 &  2.08e-15 &      \\ 
     &           &    4 &  7.28e-08 &  2.36e-15 &      \\ 
     &           &    5 &  4.98e-09 &  2.08e-15 &      \\ 
     &           &    6 &  4.98e-09 &  1.42e-16 &      \\ 
     &           &    7 &  4.97e-09 &  1.42e-16 &      \\ 
     &           &    8 &  4.96e-09 &  1.42e-16 &      \\ 
     &           &    9 &  4.96e-09 &  1.42e-16 &      \\ 
     &           &   10 &  4.95e-09 &  1.41e-16 &      \\ 
2489 &  2.49e+04 &   10 &           &           & iters  \\ 
 \hdashline 
     &           &    1 &  2.35e-08 &  1.16e-09 &      \\ 
     &           &    2 &  2.35e-08 &  6.70e-16 &      \\ 
     &           &    3 &  2.34e-08 &  6.69e-16 &      \\ 
     &           &    4 &  5.43e-08 &  6.68e-16 &      \\ 
     &           &    5 &  2.35e-08 &  1.55e-15 &      \\ 
     &           &    6 &  2.34e-08 &  6.70e-16 &      \\ 
     &           &    7 &  5.43e-08 &  6.69e-16 &      \\ 
     &           &    8 &  2.35e-08 &  1.55e-15 &      \\ 
     &           &    9 &  2.34e-08 &  6.70e-16 &      \\ 
     &           &   10 &  5.43e-08 &  6.69e-16 &      \\ 
2490 &  2.49e+04 &   10 &           &           & iters  \\ 
 \hdashline 
     &           &    1 &  3.46e-08 &  1.01e-09 &      \\ 
     &           &    2 &  3.46e-08 &  9.89e-16 &      \\ 
     &           &    3 &  4.32e-08 &  9.87e-16 &      \\ 
     &           &    4 &  3.46e-08 &  1.23e-15 &      \\ 
     &           &    5 &  4.31e-08 &  9.87e-16 &      \\ 
     &           &    6 &  3.46e-08 &  1.23e-15 &      \\ 
     &           &    7 &  4.31e-08 &  9.88e-16 &      \\ 
     &           &    8 &  3.46e-08 &  1.23e-15 &      \\ 
     &           &    9 &  4.31e-08 &  9.88e-16 &      \\ 
     &           &   10 &  3.46e-08 &  1.23e-15 &      \\ 
2491 &  2.49e+04 &   10 &           &           & iters  \\ 
 \hdashline 
     &           &    1 &  4.44e-08 &  1.85e-09 &      \\ 
     &           &    2 &  3.33e-08 &  1.27e-15 &      \\ 
     &           &    3 &  3.33e-08 &  9.51e-16 &      \\ 
     &           &    4 &  4.45e-08 &  9.50e-16 &      \\ 
     &           &    5 &  3.33e-08 &  1.27e-15 &      \\ 
     &           &    6 &  4.44e-08 &  9.50e-16 &      \\ 
     &           &    7 &  3.33e-08 &  1.27e-15 &      \\ 
     &           &    8 &  4.44e-08 &  9.51e-16 &      \\ 
     &           &    9 &  3.33e-08 &  1.27e-15 &      \\ 
     &           &   10 &  3.33e-08 &  9.51e-16 &      \\ 
2492 &  2.49e+04 &   10 &           &           & iters  \\ 
 \hdashline 
     &           &    1 &  8.59e-08 &  1.91e-09 &      \\ 
     &           &    2 &  6.96e-08 &  2.45e-15 &      \\ 
     &           &    3 &  8.59e-08 &  1.99e-15 &      \\ 
     &           &    4 &  6.96e-08 &  2.45e-15 &      \\ 
     &           &    5 &  8.16e-09 &  1.99e-15 &      \\ 
     &           &    6 &  8.15e-09 &  2.33e-16 &      \\ 
     &           &    7 &  8.14e-09 &  2.33e-16 &      \\ 
     &           &    8 &  8.13e-09 &  2.32e-16 &      \\ 
     &           &    9 &  8.12e-09 &  2.32e-16 &      \\ 
     &           &   10 &  8.10e-09 &  2.32e-16 &      \\ 
2493 &  2.49e+04 &   10 &           &           & iters  \\ 
 \hdashline 
     &           &    1 &  3.95e-08 &  1.27e-09 &      \\ 
     &           &    2 &  3.94e-08 &  1.13e-15 &      \\ 
     &           &    3 &  3.83e-08 &  1.12e-15 &      \\ 
     &           &    4 &  3.94e-08 &  1.09e-15 &      \\ 
     &           &    5 &  3.83e-08 &  1.12e-15 &      \\ 
     &           &    6 &  3.94e-08 &  1.09e-15 &      \\ 
     &           &    7 &  3.83e-08 &  1.12e-15 &      \\ 
     &           &    8 &  3.94e-08 &  1.09e-15 &      \\ 
     &           &    9 &  3.83e-08 &  1.12e-15 &      \\ 
     &           &   10 &  3.94e-08 &  1.09e-15 &      \\ 
2494 &  2.49e+04 &   10 &           &           & iters  \\ 
 \hdashline 
     &           &    1 &  3.14e-08 &  8.44e-10 &      \\ 
     &           &    2 &  3.14e-08 &  8.97e-16 &      \\ 
     &           &    3 &  4.63e-08 &  8.96e-16 &      \\ 
     &           &    4 &  3.14e-08 &  1.32e-15 &      \\ 
     &           &    5 &  3.14e-08 &  8.96e-16 &      \\ 
     &           &    6 &  4.64e-08 &  8.95e-16 &      \\ 
     &           &    7 &  3.14e-08 &  1.32e-15 &      \\ 
     &           &    8 &  4.63e-08 &  8.96e-16 &      \\ 
     &           &    9 &  3.14e-08 &  1.32e-15 &      \\ 
     &           &   10 &  3.14e-08 &  8.96e-16 &      \\ 
2495 &  2.49e+04 &   10 &           &           & iters  \\ 
 \hdashline 
     &           &    1 &  1.96e-08 &  7.65e-11 &      \\ 
     &           &    2 &  1.96e-08 &  5.59e-16 &      \\ 
     &           &    3 &  1.95e-08 &  5.58e-16 &      \\ 
     &           &    4 &  5.82e-08 &  5.58e-16 &      \\ 
     &           &    5 &  1.96e-08 &  1.66e-15 &      \\ 
     &           &    6 &  1.96e-08 &  5.59e-16 &      \\ 
     &           &    7 &  1.95e-08 &  5.58e-16 &      \\ 
     &           &    8 &  5.82e-08 &  5.58e-16 &      \\ 
     &           &    9 &  1.96e-08 &  1.66e-15 &      \\ 
     &           &   10 &  1.96e-08 &  5.59e-16 &      \\ 
2496 &  2.50e+04 &   10 &           &           & iters  \\ 
 \hdashline 
     &           &    1 &  2.20e-08 &  9.12e-10 &      \\ 
     &           &    2 &  5.57e-08 &  6.28e-16 &      \\ 
     &           &    3 &  2.21e-08 &  1.59e-15 &      \\ 
     &           &    4 &  2.20e-08 &  6.29e-16 &      \\ 
     &           &    5 &  5.57e-08 &  6.28e-16 &      \\ 
     &           &    6 &  2.21e-08 &  1.59e-15 &      \\ 
     &           &    7 &  2.20e-08 &  6.30e-16 &      \\ 
     &           &    8 &  2.20e-08 &  6.29e-16 &      \\ 
     &           &    9 &  5.57e-08 &  6.28e-16 &      \\ 
     &           &   10 &  2.21e-08 &  1.59e-15 &      \\ 
2497 &  2.50e+04 &   10 &           &           & iters  \\ 
 \hdashline 
     &           &    1 &  3.03e-08 &  5.38e-10 &      \\ 
     &           &    2 &  4.75e-08 &  8.64e-16 &      \\ 
     &           &    3 &  3.03e-08 &  1.35e-15 &      \\ 
     &           &    4 &  3.02e-08 &  8.64e-16 &      \\ 
     &           &    5 &  4.75e-08 &  8.63e-16 &      \\ 
     &           &    6 &  3.03e-08 &  1.36e-15 &      \\ 
     &           &    7 &  4.75e-08 &  8.64e-16 &      \\ 
     &           &    8 &  3.03e-08 &  1.35e-15 &      \\ 
     &           &    9 &  3.03e-08 &  8.65e-16 &      \\ 
     &           &   10 &  4.75e-08 &  8.63e-16 &      \\ 
2498 &  2.50e+04 &   10 &           &           & iters  \\ 
 \hdashline 
     &           &    1 &  1.28e-08 &  1.74e-10 &      \\ 
     &           &    2 &  1.28e-08 &  3.65e-16 &      \\ 
     &           &    3 &  1.28e-08 &  3.65e-16 &      \\ 
     &           &    4 &  6.49e-08 &  3.64e-16 &      \\ 
     &           &    5 &  9.05e-08 &  1.85e-15 &      \\ 
     &           &    6 &  6.50e-08 &  2.58e-15 &      \\ 
     &           &    7 &  1.28e-08 &  1.85e-15 &      \\ 
     &           &    8 &  1.28e-08 &  3.65e-16 &      \\ 
     &           &    9 &  1.28e-08 &  3.65e-16 &      \\ 
     &           &   10 &  6.49e-08 &  3.64e-16 &      \\ 
2499 &  2.50e+04 &   10 &           &           & iters  \\ 
 \hdashline 
     &           &    1 &  6.54e-08 &  1.50e-10 &      \\ 
     &           &    2 &  1.24e-08 &  1.87e-15 &      \\ 
     &           &    3 &  1.24e-08 &  3.55e-16 &      \\ 
     &           &    4 &  1.24e-08 &  3.54e-16 &      \\ 
     &           &    5 &  1.24e-08 &  3.54e-16 &      \\ 
     &           &    6 &  6.53e-08 &  3.53e-16 &      \\ 
     &           &    7 &  1.25e-08 &  1.86e-15 &      \\ 
     &           &    8 &  1.24e-08 &  3.55e-16 &      \\ 
     &           &    9 &  1.24e-08 &  3.55e-16 &      \\ 
     &           &   10 &  1.24e-08 &  3.54e-16 &      \\ 
2500 &  2.50e+04 &   10 &           &           & iters  \\ 
 \hdashline 
     &           &    1 &  4.11e-08 &  4.17e-10 &      \\ 
     &           &    2 &  3.67e-08 &  1.17e-15 &      \\ 
     &           &    3 &  4.11e-08 &  1.05e-15 &      \\ 
     &           &    4 &  3.67e-08 &  1.17e-15 &      \\ 
     &           &    5 &  4.11e-08 &  1.05e-15 &      \\ 
     &           &    6 &  3.67e-08 &  1.17e-15 &      \\ 
     &           &    7 &  4.10e-08 &  1.05e-15 &      \\ 
     &           &    8 &  3.67e-08 &  1.17e-15 &      \\ 
     &           &    9 &  4.10e-08 &  1.05e-15 &      \\ 
     &           &   10 &  3.67e-08 &  1.17e-15 &      \\ 
2501 &  2.50e+04 &   10 &           &           & iters  \\ 
 \hdashline 
     &           &    1 &  3.30e-08 &  6.79e-10 &      \\ 
     &           &    2 &  4.47e-08 &  9.43e-16 &      \\ 
     &           &    3 &  3.31e-08 &  1.28e-15 &      \\ 
     &           &    4 &  3.30e-08 &  9.43e-16 &      \\ 
     &           &    5 &  4.47e-08 &  9.42e-16 &      \\ 
     &           &    6 &  3.30e-08 &  1.28e-15 &      \\ 
     &           &    7 &  4.47e-08 &  9.43e-16 &      \\ 
     &           &    8 &  3.30e-08 &  1.28e-15 &      \\ 
     &           &    9 &  4.47e-08 &  9.43e-16 &      \\ 
     &           &   10 &  3.31e-08 &  1.28e-15 &      \\ 
2502 &  2.50e+04 &   10 &           &           & iters  \\ 
 \hdashline 
     &           &    1 &  3.01e-08 &  1.80e-10 &      \\ 
     &           &    2 &  3.01e-08 &  8.60e-16 &      \\ 
     &           &    3 &  4.76e-08 &  8.59e-16 &      \\ 
     &           &    4 &  3.01e-08 &  1.36e-15 &      \\ 
     &           &    5 &  3.01e-08 &  8.60e-16 &      \\ 
     &           &    6 &  4.77e-08 &  8.58e-16 &      \\ 
     &           &    7 &  3.01e-08 &  1.36e-15 &      \\ 
     &           &    8 &  4.76e-08 &  8.59e-16 &      \\ 
     &           &    9 &  3.01e-08 &  1.36e-15 &      \\ 
     &           &   10 &  3.01e-08 &  8.60e-16 &      \\ 
2503 &  2.50e+04 &   10 &           &           & iters  \\ 
 \hdashline 
     &           &    1 &  6.40e-08 &  8.24e-10 &      \\ 
     &           &    2 &  1.38e-08 &  1.83e-15 &      \\ 
     &           &    3 &  1.38e-08 &  3.94e-16 &      \\ 
     &           &    4 &  1.38e-08 &  3.94e-16 &      \\ 
     &           &    5 &  1.38e-08 &  3.93e-16 &      \\ 
     &           &    6 &  1.37e-08 &  3.93e-16 &      \\ 
     &           &    7 &  6.40e-08 &  3.92e-16 &      \\ 
     &           &    8 &  1.38e-08 &  1.83e-15 &      \\ 
     &           &    9 &  1.38e-08 &  3.94e-16 &      \\ 
     &           &   10 &  1.38e-08 &  3.93e-16 &      \\ 
2504 &  2.50e+04 &   10 &           &           & iters  \\ 
 \hdashline 
     &           &    1 &  5.07e-08 &  1.29e-09 &      \\ 
     &           &    2 &  2.71e-08 &  1.45e-15 &      \\ 
     &           &    3 &  2.70e-08 &  7.73e-16 &      \\ 
     &           &    4 &  5.07e-08 &  7.72e-16 &      \\ 
     &           &    5 &  2.71e-08 &  1.45e-15 &      \\ 
     &           &    6 &  2.70e-08 &  7.73e-16 &      \\ 
     &           &    7 &  5.07e-08 &  7.72e-16 &      \\ 
     &           &    8 &  2.71e-08 &  1.45e-15 &      \\ 
     &           &    9 &  2.70e-08 &  7.73e-16 &      \\ 
     &           &   10 &  5.07e-08 &  7.72e-16 &      \\ 
2505 &  2.50e+04 &   10 &           &           & iters  \\ 
 \hdashline 
     &           &    1 &  2.50e-08 &  5.75e-10 &      \\ 
     &           &    2 &  2.50e-08 &  7.14e-16 &      \\ 
     &           &    3 &  5.27e-08 &  7.13e-16 &      \\ 
     &           &    4 &  2.50e-08 &  1.51e-15 &      \\ 
     &           &    5 &  2.50e-08 &  7.14e-16 &      \\ 
     &           &    6 &  5.27e-08 &  7.13e-16 &      \\ 
     &           &    7 &  2.50e-08 &  1.51e-15 &      \\ 
     &           &    8 &  2.50e-08 &  7.14e-16 &      \\ 
     &           &    9 &  5.27e-08 &  7.13e-16 &      \\ 
     &           &   10 &  2.50e-08 &  1.51e-15 &      \\ 
2506 &  2.50e+04 &   10 &           &           & iters  \\ 
 \hdashline 
     &           &    1 &  3.73e-10 &  1.78e-10 &      \\ 
     &           &    2 &  3.73e-10 &  1.07e-17 &      \\ 
     &           &    3 &  3.72e-10 &  1.06e-17 &      \\ 
     &           &    4 &  3.72e-10 &  1.06e-17 &      \\ 
     &           &    5 &  3.71e-10 &  1.06e-17 &      \\ 
     &           &    6 &  3.71e-10 &  1.06e-17 &      \\ 
     &           &    7 &  3.70e-10 &  1.06e-17 &      \\ 
     &           &    8 &  3.70e-10 &  1.06e-17 &      \\ 
     &           &    9 &  3.69e-10 &  1.06e-17 &      \\ 
     &           &   10 &  3.69e-10 &  1.05e-17 &      \\ 
2507 &  2.51e+04 &   10 &           &           & iters  \\ 
 \hdashline 
     &           &    1 &  6.67e-09 &  5.79e-10 &      \\ 
     &           &    2 &  6.66e-09 &  1.90e-16 &      \\ 
     &           &    3 &  6.65e-09 &  1.90e-16 &      \\ 
     &           &    4 &  6.64e-09 &  1.90e-16 &      \\ 
     &           &    5 &  7.11e-08 &  1.90e-16 &      \\ 
     &           &    6 &  8.44e-08 &  2.03e-15 &      \\ 
     &           &    7 &  7.11e-08 &  2.41e-15 &      \\ 
     &           &    8 &  6.72e-09 &  2.03e-15 &      \\ 
     &           &    9 &  6.71e-09 &  1.92e-16 &      \\ 
     &           &   10 &  6.70e-09 &  1.91e-16 &      \\ 
2508 &  2.51e+04 &   10 &           &           & iters  \\ 
 \hdashline 
     &           &    1 &  9.28e-09 &  1.28e-09 &      \\ 
     &           &    2 &  9.27e-09 &  2.65e-16 &      \\ 
     &           &    3 &  6.84e-08 &  2.65e-16 &      \\ 
     &           &    4 &  8.70e-08 &  1.95e-15 &      \\ 
     &           &    5 &  6.85e-08 &  2.48e-15 &      \\ 
     &           &    6 &  9.33e-09 &  1.95e-15 &      \\ 
     &           &    7 &  9.31e-09 &  2.66e-16 &      \\ 
     &           &    8 &  9.30e-09 &  2.66e-16 &      \\ 
     &           &    9 &  9.29e-09 &  2.65e-16 &      \\ 
     &           &   10 &  9.27e-09 &  2.65e-16 &      \\ 
2509 &  2.51e+04 &   10 &           &           & iters  \\ 
 \hdashline 
     &           &    1 &  4.73e-08 &  1.53e-09 &      \\ 
     &           &    2 &  3.05e-08 &  1.35e-15 &      \\ 
     &           &    3 &  4.73e-08 &  8.70e-16 &      \\ 
     &           &    4 &  3.05e-08 &  1.35e-15 &      \\ 
     &           &    5 &  4.72e-08 &  8.70e-16 &      \\ 
     &           &    6 &  3.05e-08 &  1.35e-15 &      \\ 
     &           &    7 &  3.05e-08 &  8.71e-16 &      \\ 
     &           &    8 &  4.73e-08 &  8.70e-16 &      \\ 
     &           &    9 &  3.05e-08 &  1.35e-15 &      \\ 
     &           &   10 &  4.72e-08 &  8.70e-16 &      \\ 
2510 &  2.51e+04 &   10 &           &           & iters  \\ 
 \hdashline 
     &           &    1 &  9.86e-09 &  1.51e-09 &      \\ 
     &           &    2 &  9.85e-09 &  2.82e-16 &      \\ 
     &           &    3 &  9.84e-09 &  2.81e-16 &      \\ 
     &           &    4 &  9.82e-09 &  2.81e-16 &      \\ 
     &           &    5 &  9.81e-09 &  2.80e-16 &      \\ 
     &           &    6 &  9.79e-09 &  2.80e-16 &      \\ 
     &           &    7 &  9.78e-09 &  2.80e-16 &      \\ 
     &           &    8 &  9.77e-09 &  2.79e-16 &      \\ 
     &           &    9 &  9.75e-09 &  2.79e-16 &      \\ 
     &           &   10 &  9.74e-09 &  2.78e-16 &      \\ 
2511 &  2.51e+04 &   10 &           &           & iters  \\ 
 \hdashline 
     &           &    1 &  1.23e-08 &  1.42e-09 &      \\ 
     &           &    2 &  1.23e-08 &  3.51e-16 &      \\ 
     &           &    3 &  1.23e-08 &  3.51e-16 &      \\ 
     &           &    4 &  1.23e-08 &  3.50e-16 &      \\ 
     &           &    5 &  6.54e-08 &  3.50e-16 &      \\ 
     &           &    6 &  1.23e-08 &  1.87e-15 &      \\ 
     &           &    7 &  1.23e-08 &  3.52e-16 &      \\ 
     &           &    8 &  1.23e-08 &  3.51e-16 &      \\ 
     &           &    9 &  1.23e-08 &  3.51e-16 &      \\ 
     &           &   10 &  1.23e-08 &  3.50e-16 &      \\ 
2512 &  2.51e+04 &   10 &           &           & iters  \\ 
 \hdashline 
     &           &    1 &  4.25e-09 &  9.62e-11 &      \\ 
     &           &    2 &  4.25e-09 &  1.21e-16 &      \\ 
     &           &    3 &  4.24e-09 &  1.21e-16 &      \\ 
     &           &    4 &  4.23e-09 &  1.21e-16 &      \\ 
     &           &    5 &  4.23e-09 &  1.21e-16 &      \\ 
     &           &    6 &  7.35e-08 &  1.21e-16 &      \\ 
     &           &    7 &  8.20e-08 &  2.10e-15 &      \\ 
     &           &    8 &  7.35e-08 &  2.34e-15 &      \\ 
     &           &    9 &  4.32e-09 &  2.10e-15 &      \\ 
     &           &   10 &  4.31e-09 &  1.23e-16 &      \\ 
2513 &  2.51e+04 &   10 &           &           & iters  \\ 
 \hdashline 
     &           &    1 &  6.48e-08 &  1.12e-09 &      \\ 
     &           &    2 &  1.30e-08 &  1.85e-15 &      \\ 
     &           &    3 &  1.30e-08 &  3.70e-16 &      \\ 
     &           &    4 &  1.29e-08 &  3.70e-16 &      \\ 
     &           &    5 &  1.29e-08 &  3.69e-16 &      \\ 
     &           &    6 &  6.48e-08 &  3.69e-16 &      \\ 
     &           &    7 &  1.30e-08 &  1.85e-15 &      \\ 
     &           &    8 &  1.30e-08 &  3.71e-16 &      \\ 
     &           &    9 &  1.30e-08 &  3.70e-16 &      \\ 
     &           &   10 &  1.29e-08 &  3.70e-16 &      \\ 
2514 &  2.51e+04 &   10 &           &           & iters  \\ 
 \hdashline 
     &           &    1 &  3.57e-08 &  1.70e-10 &      \\ 
     &           &    2 &  4.20e-08 &  1.02e-15 &      \\ 
     &           &    3 &  3.57e-08 &  1.20e-15 &      \\ 
     &           &    4 &  4.20e-08 &  1.02e-15 &      \\ 
     &           &    5 &  3.57e-08 &  1.20e-15 &      \\ 
     &           &    6 &  3.57e-08 &  1.02e-15 &      \\ 
     &           &    7 &  4.21e-08 &  1.02e-15 &      \\ 
     &           &    8 &  3.57e-08 &  1.20e-15 &      \\ 
     &           &    9 &  4.21e-08 &  1.02e-15 &      \\ 
     &           &   10 &  3.57e-08 &  1.20e-15 &      \\ 
2515 &  2.51e+04 &   10 &           &           & iters  \\ 
 \hdashline 
     &           &    1 &  3.91e-08 &  6.87e-11 &      \\ 
     &           &    2 &  3.90e-08 &  1.12e-15 &      \\ 
     &           &    3 &  3.87e-08 &  1.11e-15 &      \\ 
     &           &    4 &  3.90e-08 &  1.10e-15 &      \\ 
     &           &    5 &  3.87e-08 &  1.11e-15 &      \\ 
     &           &    6 &  3.90e-08 &  1.10e-15 &      \\ 
     &           &    7 &  3.87e-08 &  1.11e-15 &      \\ 
     &           &    8 &  3.90e-08 &  1.10e-15 &      \\ 
     &           &    9 &  3.87e-08 &  1.11e-15 &      \\ 
     &           &   10 &  3.90e-08 &  1.10e-15 &      \\ 
2516 &  2.52e+04 &   10 &           &           & iters  \\ 
 \hdashline 
     &           &    1 &  2.88e-08 &  7.50e-10 &      \\ 
     &           &    2 &  2.88e-08 &  8.22e-16 &      \\ 
     &           &    3 &  2.87e-08 &  8.21e-16 &      \\ 
     &           &    4 &  4.90e-08 &  8.19e-16 &      \\ 
     &           &    5 &  2.87e-08 &  1.40e-15 &      \\ 
     &           &    6 &  2.87e-08 &  8.20e-16 &      \\ 
     &           &    7 &  4.90e-08 &  8.19e-16 &      \\ 
     &           &    8 &  2.87e-08 &  1.40e-15 &      \\ 
     &           &    9 &  4.90e-08 &  8.20e-16 &      \\ 
     &           &   10 &  2.88e-08 &  1.40e-15 &      \\ 
2517 &  2.52e+04 &   10 &           &           & iters  \\ 
 \hdashline 
     &           &    1 &  5.93e-08 &  1.48e-09 &      \\ 
     &           &    2 &  1.85e-08 &  1.69e-15 &      \\ 
     &           &    3 &  1.85e-08 &  5.29e-16 &      \\ 
     &           &    4 &  5.92e-08 &  5.28e-16 &      \\ 
     &           &    5 &  1.86e-08 &  1.69e-15 &      \\ 
     &           &    6 &  1.85e-08 &  5.29e-16 &      \\ 
     &           &    7 &  1.85e-08 &  5.29e-16 &      \\ 
     &           &    8 &  1.85e-08 &  5.28e-16 &      \\ 
     &           &    9 &  5.92e-08 &  5.27e-16 &      \\ 
     &           &   10 &  1.85e-08 &  1.69e-15 &      \\ 
2518 &  2.52e+04 &   10 &           &           & iters  \\ 
 \hdashline 
     &           &    1 &  6.70e-09 &  2.16e-09 &      \\ 
     &           &    2 &  6.69e-09 &  1.91e-16 &      \\ 
     &           &    3 &  7.10e-08 &  1.91e-16 &      \\ 
     &           &    4 &  4.56e-08 &  2.03e-15 &      \\ 
     &           &    5 &  6.72e-09 &  1.30e-15 &      \\ 
     &           &    6 &  6.71e-09 &  1.92e-16 &      \\ 
     &           &    7 &  6.70e-09 &  1.91e-16 &      \\ 
     &           &    8 &  6.69e-09 &  1.91e-16 &      \\ 
     &           &    9 &  7.10e-08 &  1.91e-16 &      \\ 
     &           &   10 &  4.56e-08 &  2.03e-15 &      \\ 
2519 &  2.52e+04 &   10 &           &           & iters  \\ 
 \hdashline 
     &           &    1 &  3.61e-08 &  1.23e-09 &      \\ 
     &           &    2 &  2.81e-09 &  1.03e-15 &      \\ 
     &           &    3 &  2.80e-09 &  8.01e-17 &      \\ 
     &           &    4 &  2.80e-09 &  8.00e-17 &      \\ 
     &           &    5 &  2.79e-09 &  7.99e-17 &      \\ 
     &           &    6 &  2.79e-09 &  7.98e-17 &      \\ 
     &           &    7 &  2.79e-09 &  7.96e-17 &      \\ 
     &           &    8 &  2.78e-09 &  7.95e-17 &      \\ 
     &           &    9 &  2.78e-09 &  7.94e-17 &      \\ 
     &           &   10 &  2.77e-09 &  7.93e-17 &      \\ 
2520 &  2.52e+04 &   10 &           &           & iters  \\ 
 \hdashline 
     &           &    1 &  5.74e-09 &  1.11e-09 &      \\ 
     &           &    2 &  3.31e-08 &  1.64e-16 &      \\ 
     &           &    3 &  5.78e-09 &  9.45e-16 &      \\ 
     &           &    4 &  5.77e-09 &  1.65e-16 &      \\ 
     &           &    5 &  5.76e-09 &  1.65e-16 &      \\ 
     &           &    6 &  5.75e-09 &  1.64e-16 &      \\ 
     &           &    7 &  5.74e-09 &  1.64e-16 &      \\ 
     &           &    8 &  5.74e-09 &  1.64e-16 &      \\ 
     &           &    9 &  3.31e-08 &  1.64e-16 &      \\ 
     &           &   10 &  5.78e-09 &  9.45e-16 &      \\ 
2521 &  2.52e+04 &   10 &           &           & iters  \\ 
 \hdashline 
     &           &    1 &  2.40e-08 &  1.94e-09 &      \\ 
     &           &    2 &  2.39e-08 &  6.84e-16 &      \\ 
     &           &    3 &  5.38e-08 &  6.83e-16 &      \\ 
     &           &    4 &  2.40e-08 &  1.54e-15 &      \\ 
     &           &    5 &  2.39e-08 &  6.84e-16 &      \\ 
     &           &    6 &  5.38e-08 &  6.83e-16 &      \\ 
     &           &    7 &  2.40e-08 &  1.54e-15 &      \\ 
     &           &    8 &  2.39e-08 &  6.84e-16 &      \\ 
     &           &    9 &  5.38e-08 &  6.83e-16 &      \\ 
     &           &   10 &  2.40e-08 &  1.53e-15 &      \\ 
2522 &  2.52e+04 &   10 &           &           & iters  \\ 
 \hdashline 
     &           &    1 &  3.67e-09 &  9.94e-10 &      \\ 
     &           &    2 &  3.66e-09 &  1.05e-16 &      \\ 
     &           &    3 &  7.40e-08 &  1.05e-16 &      \\ 
     &           &    4 &  8.15e-08 &  2.11e-15 &      \\ 
     &           &    5 &  7.40e-08 &  2.32e-15 &      \\ 
     &           &    6 &  8.14e-08 &  2.11e-15 &      \\ 
     &           &    7 &  7.41e-08 &  2.32e-15 &      \\ 
     &           &    8 &  3.74e-09 &  2.11e-15 &      \\ 
     &           &    9 &  3.74e-09 &  1.07e-16 &      \\ 
     &           &   10 &  3.73e-09 &  1.07e-16 &      \\ 
2523 &  2.52e+04 &   10 &           &           & iters  \\ 
 \hdashline 
     &           &    1 &  3.44e-08 &  2.47e-10 &      \\ 
     &           &    2 &  4.33e-08 &  9.82e-16 &      \\ 
     &           &    3 &  3.44e-08 &  1.24e-15 &      \\ 
     &           &    4 &  4.33e-08 &  9.83e-16 &      \\ 
     &           &    5 &  3.44e-08 &  1.24e-15 &      \\ 
     &           &    6 &  3.44e-08 &  9.83e-16 &      \\ 
     &           &    7 &  4.33e-08 &  9.82e-16 &      \\ 
     &           &    8 &  3.44e-08 &  1.24e-15 &      \\ 
     &           &    9 &  4.33e-08 &  9.82e-16 &      \\ 
     &           &   10 &  3.44e-08 &  1.24e-15 &      \\ 
2524 &  2.52e+04 &   10 &           &           & iters  \\ 
 \hdashline 
     &           &    1 &  3.01e-08 &  4.45e-10 &      \\ 
     &           &    2 &  4.77e-08 &  8.58e-16 &      \\ 
     &           &    3 &  3.01e-08 &  1.36e-15 &      \\ 
     &           &    4 &  3.00e-08 &  8.59e-16 &      \\ 
     &           &    5 &  4.77e-08 &  8.57e-16 &      \\ 
     &           &    6 &  3.01e-08 &  1.36e-15 &      \\ 
     &           &    7 &  3.00e-08 &  8.58e-16 &      \\ 
     &           &    8 &  4.77e-08 &  8.57e-16 &      \\ 
     &           &    9 &  3.00e-08 &  1.36e-15 &      \\ 
     &           &   10 &  4.77e-08 &  8.58e-16 &      \\ 
2525 &  2.52e+04 &   10 &           &           & iters  \\ 
 \hdashline 
     &           &    1 &  7.20e-08 &  7.49e-10 &      \\ 
     &           &    2 &  5.81e-09 &  2.05e-15 &      \\ 
     &           &    3 &  5.80e-09 &  1.66e-16 &      \\ 
     &           &    4 &  5.79e-09 &  1.65e-16 &      \\ 
     &           &    5 &  5.78e-09 &  1.65e-16 &      \\ 
     &           &    6 &  5.77e-09 &  1.65e-16 &      \\ 
     &           &    7 &  5.76e-09 &  1.65e-16 &      \\ 
     &           &    8 &  5.76e-09 &  1.65e-16 &      \\ 
     &           &    9 &  5.75e-09 &  1.64e-16 &      \\ 
     &           &   10 &  5.74e-09 &  1.64e-16 &      \\ 
2526 &  2.52e+04 &   10 &           &           & iters  \\ 
 \hdashline 
     &           &    1 &  7.38e-09 &  8.26e-11 &      \\ 
     &           &    2 &  7.37e-09 &  2.11e-16 &      \\ 
     &           &    3 &  7.03e-08 &  2.10e-16 &      \\ 
     &           &    4 &  8.52e-08 &  2.01e-15 &      \\ 
     &           &    5 &  7.03e-08 &  2.43e-15 &      \\ 
     &           &    6 &  7.44e-09 &  2.01e-15 &      \\ 
     &           &    7 &  7.43e-09 &  2.12e-16 &      \\ 
     &           &    8 &  7.42e-09 &  2.12e-16 &      \\ 
     &           &    9 &  7.41e-09 &  2.12e-16 &      \\ 
     &           &   10 &  7.40e-09 &  2.12e-16 &      \\ 
2527 &  2.53e+04 &   10 &           &           & iters  \\ 
 \hdashline 
     &           &    1 &  2.56e-09 &  8.77e-10 &      \\ 
     &           &    2 &  2.55e-09 &  7.29e-17 &      \\ 
     &           &    3 &  2.55e-09 &  7.28e-17 &      \\ 
     &           &    4 &  2.54e-09 &  7.27e-17 &      \\ 
     &           &    5 &  2.54e-09 &  7.26e-17 &      \\ 
     &           &    6 &  2.54e-09 &  7.25e-17 &      \\ 
     &           &    7 &  2.53e-09 &  7.24e-17 &      \\ 
     &           &    8 &  2.53e-09 &  7.23e-17 &      \\ 
     &           &    9 &  2.53e-09 &  7.22e-17 &      \\ 
     &           &   10 &  2.52e-09 &  7.21e-17 &      \\ 
2528 &  2.53e+04 &   10 &           &           & iters  \\ 
 \hdashline 
     &           &    1 &  2.49e-08 &  1.43e-09 &      \\ 
     &           &    2 &  2.49e-08 &  7.11e-16 &      \\ 
     &           &    3 &  2.48e-08 &  7.10e-16 &      \\ 
     &           &    4 &  5.29e-08 &  7.09e-16 &      \\ 
     &           &    5 &  2.49e-08 &  1.51e-15 &      \\ 
     &           &    6 &  2.49e-08 &  7.10e-16 &      \\ 
     &           &    7 &  5.29e-08 &  7.09e-16 &      \\ 
     &           &    8 &  2.49e-08 &  1.51e-15 &      \\ 
     &           &    9 &  2.49e-08 &  7.10e-16 &      \\ 
     &           &   10 &  5.29e-08 &  7.09e-16 &      \\ 
2529 &  2.53e+04 &   10 &           &           & iters  \\ 
 \hdashline 
     &           &    1 &  2.00e-08 &  2.40e-09 &      \\ 
     &           &    2 &  1.99e-08 &  5.70e-16 &      \\ 
     &           &    3 &  1.89e-08 &  5.69e-16 &      \\ 
     &           &    4 &  1.89e-08 &  5.40e-16 &      \\ 
     &           &    5 &  2.00e-08 &  5.39e-16 &      \\ 
     &           &    6 &  1.89e-08 &  5.70e-16 &      \\ 
     &           &    7 &  2.00e-08 &  5.39e-16 &      \\ 
     &           &    8 &  1.89e-08 &  5.70e-16 &      \\ 
     &           &    9 &  2.00e-08 &  5.39e-16 &      \\ 
     &           &   10 &  1.89e-08 &  5.70e-16 &      \\ 
2530 &  2.53e+04 &   10 &           &           & iters  \\ 
 \hdashline 
     &           &    1 &  3.30e-08 &  1.47e-09 &      \\ 
     &           &    2 &  4.47e-08 &  9.43e-16 &      \\ 
     &           &    3 &  3.30e-08 &  1.28e-15 &      \\ 
     &           &    4 &  3.30e-08 &  9.43e-16 &      \\ 
     &           &    5 &  4.47e-08 &  9.42e-16 &      \\ 
     &           &    6 &  3.30e-08 &  1.28e-15 &      \\ 
     &           &    7 &  4.47e-08 &  9.42e-16 &      \\ 
     &           &    8 &  3.30e-08 &  1.28e-15 &      \\ 
     &           &    9 &  4.47e-08 &  9.43e-16 &      \\ 
     &           &   10 &  3.30e-08 &  1.28e-15 &      \\ 
2531 &  2.53e+04 &   10 &           &           & iters  \\ 
 \hdashline 
     &           &    1 &  4.97e-08 &  8.57e-10 &      \\ 
     &           &    2 &  2.80e-08 &  1.42e-15 &      \\ 
     &           &    3 &  2.80e-08 &  8.00e-16 &      \\ 
     &           &    4 &  4.97e-08 &  7.99e-16 &      \\ 
     &           &    5 &  2.80e-08 &  1.42e-15 &      \\ 
     &           &    6 &  2.80e-08 &  8.00e-16 &      \\ 
     &           &    7 &  4.97e-08 &  7.99e-16 &      \\ 
     &           &    8 &  2.80e-08 &  1.42e-15 &      \\ 
     &           &    9 &  2.80e-08 &  8.00e-16 &      \\ 
     &           &   10 &  4.97e-08 &  7.99e-16 &      \\ 
2532 &  2.53e+04 &   10 &           &           & iters  \\ 
 \hdashline 
     &           &    1 &  2.73e-08 &  4.15e-10 &      \\ 
     &           &    2 &  2.73e-08 &  7.80e-16 &      \\ 
     &           &    3 &  5.04e-08 &  7.79e-16 &      \\ 
     &           &    4 &  2.73e-08 &  1.44e-15 &      \\ 
     &           &    5 &  2.73e-08 &  7.80e-16 &      \\ 
     &           &    6 &  5.04e-08 &  7.79e-16 &      \\ 
     &           &    7 &  2.73e-08 &  1.44e-15 &      \\ 
     &           &    8 &  2.73e-08 &  7.80e-16 &      \\ 
     &           &    9 &  5.05e-08 &  7.78e-16 &      \\ 
     &           &   10 &  2.73e-08 &  1.44e-15 &      \\ 
2533 &  2.53e+04 &   10 &           &           & iters  \\ 
 \hdashline 
     &           &    1 &  8.67e-10 &  3.73e-10 &      \\ 
     &           &    2 &  8.66e-10 &  2.48e-17 &      \\ 
     &           &    3 &  8.65e-10 &  2.47e-17 &      \\ 
     &           &    4 &  8.64e-10 &  2.47e-17 &      \\ 
     &           &    5 &  8.62e-10 &  2.46e-17 &      \\ 
     &           &    6 &  8.61e-10 &  2.46e-17 &      \\ 
     &           &    7 &  8.60e-10 &  2.46e-17 &      \\ 
     &           &    8 &  8.59e-10 &  2.45e-17 &      \\ 
     &           &    9 &  8.57e-10 &  2.45e-17 &      \\ 
     &           &   10 &  8.56e-10 &  2.45e-17 &      \\ 
2534 &  2.53e+04 &   10 &           &           & iters  \\ 
 \hdashline 
     &           &    1 &  3.71e-08 &  1.09e-10 &      \\ 
     &           &    2 &  1.82e-09 &  1.06e-15 &      \\ 
     &           &    3 &  1.82e-09 &  5.19e-17 &      \\ 
     &           &    4 &  1.81e-09 &  5.18e-17 &      \\ 
     &           &    5 &  1.81e-09 &  5.18e-17 &      \\ 
     &           &    6 &  1.81e-09 &  5.17e-17 &      \\ 
     &           &    7 &  1.81e-09 &  5.16e-17 &      \\ 
     &           &    8 &  1.80e-09 &  5.16e-17 &      \\ 
     &           &    9 &  1.80e-09 &  5.15e-17 &      \\ 
     &           &   10 &  1.80e-09 &  5.14e-17 &      \\ 
2535 &  2.53e+04 &   10 &           &           & iters  \\ 
 \hdashline 
     &           &    1 &  3.66e-08 &  9.72e-11 &      \\ 
     &           &    2 &  4.12e-08 &  1.04e-15 &      \\ 
     &           &    3 &  3.66e-08 &  1.17e-15 &      \\ 
     &           &    4 &  4.12e-08 &  1.04e-15 &      \\ 
     &           &    5 &  3.66e-08 &  1.17e-15 &      \\ 
     &           &    6 &  4.12e-08 &  1.04e-15 &      \\ 
     &           &    7 &  3.66e-08 &  1.17e-15 &      \\ 
     &           &    8 &  4.11e-08 &  1.04e-15 &      \\ 
     &           &    9 &  3.66e-08 &  1.17e-15 &      \\ 
     &           &   10 &  4.11e-08 &  1.04e-15 &      \\ 
2536 &  2.54e+04 &   10 &           &           & iters  \\ 
 \hdashline 
     &           &    1 &  3.16e-08 &  1.06e-09 &      \\ 
     &           &    2 &  7.26e-09 &  9.03e-16 &      \\ 
     &           &    3 &  7.25e-09 &  2.07e-16 &      \\ 
     &           &    4 &  3.16e-08 &  2.07e-16 &      \\ 
     &           &    5 &  7.28e-09 &  9.02e-16 &      \\ 
     &           &    6 &  7.27e-09 &  2.08e-16 &      \\ 
     &           &    7 &  7.26e-09 &  2.08e-16 &      \\ 
     &           &    8 &  7.25e-09 &  2.07e-16 &      \\ 
     &           &    9 &  7.24e-09 &  2.07e-16 &      \\ 
     &           &   10 &  3.16e-08 &  2.07e-16 &      \\ 
2537 &  2.54e+04 &   10 &           &           & iters  \\ 
 \hdashline 
     &           &    1 &  1.51e-10 &  2.11e-09 &      \\ 
     &           &    2 &  1.51e-10 &  4.32e-18 &      \\ 
     &           &    3 &  1.51e-10 &  4.31e-18 &      \\ 
     &           &    4 &  1.51e-10 &  4.30e-18 &      \\ 
     &           &    5 &  1.50e-10 &  4.30e-18 &      \\ 
     &           &    6 &  1.50e-10 &  4.29e-18 &      \\ 
     &           &    7 &  1.50e-10 &  4.29e-18 &      \\ 
     &           &    8 &  1.50e-10 &  4.28e-18 &      \\ 
     &           &    9 &  1.50e-10 &  4.27e-18 &      \\ 
     &           &   10 &  1.49e-10 &  4.27e-18 &      \\ 
2538 &  2.54e+04 &   10 &           &           & iters  \\ 
 \hdashline 
     &           &    1 &  5.62e-08 &  1.93e-09 &      \\ 
     &           &    2 &  2.16e-08 &  1.60e-15 &      \\ 
     &           &    3 &  2.15e-08 &  6.16e-16 &      \\ 
     &           &    4 &  2.15e-08 &  6.15e-16 &      \\ 
     &           &    5 &  5.62e-08 &  6.14e-16 &      \\ 
     &           &    6 &  2.16e-08 &  1.60e-15 &      \\ 
     &           &    7 &  2.15e-08 &  6.16e-16 &      \\ 
     &           &    8 &  5.62e-08 &  6.15e-16 &      \\ 
     &           &    9 &  2.16e-08 &  1.60e-15 &      \\ 
     &           &   10 &  2.16e-08 &  6.16e-16 &      \\ 
2539 &  2.54e+04 &   10 &           &           & iters  \\ 
 \hdashline 
     &           &    1 &  5.50e-08 &  1.26e-10 &      \\ 
     &           &    2 &  2.28e-08 &  1.57e-15 &      \\ 
     &           &    3 &  2.28e-08 &  6.51e-16 &      \\ 
     &           &    4 &  2.27e-08 &  6.50e-16 &      \\ 
     &           &    5 &  5.50e-08 &  6.49e-16 &      \\ 
     &           &    6 &  2.28e-08 &  1.57e-15 &      \\ 
     &           &    7 &  2.28e-08 &  6.51e-16 &      \\ 
     &           &    8 &  5.50e-08 &  6.50e-16 &      \\ 
     &           &    9 &  2.28e-08 &  1.57e-15 &      \\ 
     &           &   10 &  2.28e-08 &  6.51e-16 &      \\ 
2540 &  2.54e+04 &   10 &           &           & iters  \\ 
 \hdashline 
     &           &    1 &  4.25e-08 &  1.61e-09 &      \\ 
     &           &    2 &  3.53e-08 &  1.21e-15 &      \\ 
     &           &    3 &  4.25e-08 &  1.01e-15 &      \\ 
     &           &    4 &  3.53e-08 &  1.21e-15 &      \\ 
     &           &    5 &  4.25e-08 &  1.01e-15 &      \\ 
     &           &    6 &  3.53e-08 &  1.21e-15 &      \\ 
     &           &    7 &  4.24e-08 &  1.01e-15 &      \\ 
     &           &    8 &  3.53e-08 &  1.21e-15 &      \\ 
     &           &    9 &  3.53e-08 &  1.01e-15 &      \\ 
     &           &   10 &  4.25e-08 &  1.01e-15 &      \\ 
2541 &  2.54e+04 &   10 &           &           & iters  \\ 
 \hdashline 
     &           &    1 &  9.71e-09 &  3.29e-10 &      \\ 
     &           &    2 &  6.80e-08 &  2.77e-16 &      \\ 
     &           &    3 &  8.75e-08 &  1.94e-15 &      \\ 
     &           &    4 &  6.80e-08 &  2.50e-15 &      \\ 
     &           &    5 &  9.76e-09 &  1.94e-15 &      \\ 
     &           &    6 &  9.75e-09 &  2.79e-16 &      \\ 
     &           &    7 &  9.73e-09 &  2.78e-16 &      \\ 
     &           &    8 &  9.72e-09 &  2.78e-16 &      \\ 
     &           &    9 &  9.71e-09 &  2.77e-16 &      \\ 
     &           &   10 &  6.80e-08 &  2.77e-16 &      \\ 
2542 &  2.54e+04 &   10 &           &           & iters  \\ 
 \hdashline 
     &           &    1 &  3.46e-09 &  7.71e-10 &      \\ 
     &           &    2 &  3.46e-09 &  9.88e-17 &      \\ 
     &           &    3 &  3.45e-09 &  9.87e-17 &      \\ 
     &           &    4 &  3.45e-09 &  9.86e-17 &      \\ 
     &           &    5 &  3.44e-09 &  9.84e-17 &      \\ 
     &           &    6 &  3.44e-09 &  9.83e-17 &      \\ 
     &           &    7 &  3.43e-09 &  9.81e-17 &      \\ 
     &           &    8 &  3.43e-09 &  9.80e-17 &      \\ 
     &           &    9 &  3.42e-09 &  9.79e-17 &      \\ 
     &           &   10 &  3.42e-09 &  9.77e-17 &      \\ 
2543 &  2.54e+04 &   10 &           &           & iters  \\ 
 \hdashline 
     &           &    1 &  1.87e-08 &  2.65e-10 &      \\ 
     &           &    2 &  1.87e-08 &  5.34e-16 &      \\ 
     &           &    3 &  1.87e-08 &  5.33e-16 &      \\ 
     &           &    4 &  5.91e-08 &  5.33e-16 &      \\ 
     &           &    5 &  1.87e-08 &  1.69e-15 &      \\ 
     &           &    6 &  1.87e-08 &  5.34e-16 &      \\ 
     &           &    7 &  1.87e-08 &  5.34e-16 &      \\ 
     &           &    8 &  5.90e-08 &  5.33e-16 &      \\ 
     &           &    9 &  1.87e-08 &  1.69e-15 &      \\ 
     &           &   10 &  1.87e-08 &  5.34e-16 &      \\ 
2544 &  2.54e+04 &   10 &           &           & iters  \\ 
 \hdashline 
     &           &    1 &  3.52e-08 &  5.82e-10 &      \\ 
     &           &    2 &  3.52e-08 &  1.01e-15 &      \\ 
     &           &    3 &  4.26e-08 &  1.00e-15 &      \\ 
     &           &    4 &  3.52e-08 &  1.21e-15 &      \\ 
     &           &    5 &  4.26e-08 &  1.00e-15 &      \\ 
     &           &    6 &  3.52e-08 &  1.21e-15 &      \\ 
     &           &    7 &  4.25e-08 &  1.00e-15 &      \\ 
     &           &    8 &  3.52e-08 &  1.21e-15 &      \\ 
     &           &    9 &  3.51e-08 &  1.00e-15 &      \\ 
     &           &   10 &  4.26e-08 &  1.00e-15 &      \\ 
2545 &  2.54e+04 &   10 &           &           & iters  \\ 
 \hdashline 
     &           &    1 &  5.06e-08 &  1.32e-09 &      \\ 
     &           &    2 &  2.71e-08 &  1.44e-15 &      \\ 
     &           &    3 &  2.71e-08 &  7.75e-16 &      \\ 
     &           &    4 &  5.06e-08 &  7.73e-16 &      \\ 
     &           &    5 &  2.71e-08 &  1.44e-15 &      \\ 
     &           &    6 &  2.71e-08 &  7.74e-16 &      \\ 
     &           &    7 &  5.06e-08 &  7.73e-16 &      \\ 
     &           &    8 &  2.71e-08 &  1.45e-15 &      \\ 
     &           &    9 &  2.71e-08 &  7.74e-16 &      \\ 
     &           &   10 &  5.06e-08 &  7.73e-16 &      \\ 
2546 &  2.54e+04 &   10 &           &           & iters  \\ 
 \hdashline 
     &           &    1 &  2.97e-09 &  1.42e-09 &      \\ 
     &           &    2 &  2.97e-09 &  8.48e-17 &      \\ 
     &           &    3 &  2.96e-09 &  8.46e-17 &      \\ 
     &           &    4 &  2.96e-09 &  8.45e-17 &      \\ 
     &           &    5 &  2.95e-09 &  8.44e-17 &      \\ 
     &           &    6 &  2.95e-09 &  8.43e-17 &      \\ 
     &           &    7 &  2.94e-09 &  8.42e-17 &      \\ 
     &           &    8 &  2.94e-09 &  8.40e-17 &      \\ 
     &           &    9 &  2.94e-09 &  8.39e-17 &      \\ 
     &           &   10 &  2.93e-09 &  8.38e-17 &      \\ 
2547 &  2.55e+04 &   10 &           &           & iters  \\ 
 \hdashline 
     &           &    1 &  3.52e-08 &  9.92e-10 &      \\ 
     &           &    2 &  4.25e-08 &  1.00e-15 &      \\ 
     &           &    3 &  3.52e-08 &  1.21e-15 &      \\ 
     &           &    4 &  4.25e-08 &  1.01e-15 &      \\ 
     &           &    5 &  3.52e-08 &  1.21e-15 &      \\ 
     &           &    6 &  4.25e-08 &  1.01e-15 &      \\ 
     &           &    7 &  3.52e-08 &  1.21e-15 &      \\ 
     &           &    8 &  4.25e-08 &  1.01e-15 &      \\ 
     &           &    9 &  3.53e-08 &  1.21e-15 &      \\ 
     &           &   10 &  3.52e-08 &  1.01e-15 &      \\ 
2548 &  2.55e+04 &   10 &           &           & iters  \\ 
 \hdashline 
     &           &    1 &  7.88e-09 &  1.75e-10 &      \\ 
     &           &    2 &  6.98e-08 &  2.25e-16 &      \\ 
     &           &    3 &  4.68e-08 &  1.99e-15 &      \\ 
     &           &    4 &  7.90e-09 &  1.34e-15 &      \\ 
     &           &    5 &  7.89e-09 &  2.26e-16 &      \\ 
     &           &    6 &  7.88e-09 &  2.25e-16 &      \\ 
     &           &    7 &  6.98e-08 &  2.25e-16 &      \\ 
     &           &    8 &  7.97e-09 &  1.99e-15 &      \\ 
     &           &    9 &  7.96e-09 &  2.27e-16 &      \\ 
     &           &   10 &  7.95e-09 &  2.27e-16 &      \\ 
2549 &  2.55e+04 &   10 &           &           & iters  \\ 
 \hdashline 
     &           &    1 &  1.85e-08 &  6.41e-10 &      \\ 
     &           &    2 &  1.85e-08 &  5.28e-16 &      \\ 
     &           &    3 &  5.92e-08 &  5.27e-16 &      \\ 
     &           &    4 &  1.85e-08 &  1.69e-15 &      \\ 
     &           &    5 &  1.85e-08 &  5.29e-16 &      \\ 
     &           &    6 &  1.85e-08 &  5.28e-16 &      \\ 
     &           &    7 &  5.92e-08 &  5.27e-16 &      \\ 
     &           &    8 &  1.85e-08 &  1.69e-15 &      \\ 
     &           &    9 &  1.85e-08 &  5.29e-16 &      \\ 
     &           &   10 &  1.85e-08 &  5.28e-16 &      \\ 
2550 &  2.55e+04 &   10 &           &           & iters  \\ 
 \hdashline 
     &           &    1 &  4.37e-08 &  7.51e-10 &      \\ 
     &           &    2 &  4.80e-09 &  1.25e-15 &      \\ 
     &           &    3 &  4.79e-09 &  1.37e-16 &      \\ 
     &           &    4 &  4.78e-09 &  1.37e-16 &      \\ 
     &           &    5 &  4.78e-09 &  1.37e-16 &      \\ 
     &           &    6 &  4.77e-09 &  1.36e-16 &      \\ 
     &           &    7 &  7.29e-08 &  1.36e-16 &      \\ 
     &           &    8 &  4.37e-08 &  2.08e-15 &      \\ 
     &           &    9 &  4.80e-09 &  1.25e-15 &      \\ 
     &           &   10 &  4.80e-09 &  1.37e-16 &      \\ 
2551 &  2.55e+04 &   10 &           &           & iters  \\ 
 \hdashline 
     &           &    1 &  1.59e-08 &  1.11e-09 &      \\ 
     &           &    2 &  1.59e-08 &  4.55e-16 &      \\ 
     &           &    3 &  1.59e-08 &  4.54e-16 &      \\ 
     &           &    4 &  6.18e-08 &  4.54e-16 &      \\ 
     &           &    5 &  1.60e-08 &  1.76e-15 &      \\ 
     &           &    6 &  1.59e-08 &  4.56e-16 &      \\ 
     &           &    7 &  1.59e-08 &  4.55e-16 &      \\ 
     &           &    8 &  1.59e-08 &  4.54e-16 &      \\ 
     &           &    9 &  6.18e-08 &  4.54e-16 &      \\ 
     &           &   10 &  1.60e-08 &  1.76e-15 &      \\ 
2552 &  2.55e+04 &   10 &           &           & iters  \\ 
 \hdashline 
     &           &    1 &  1.97e-08 &  4.44e-10 &      \\ 
     &           &    2 &  5.80e-08 &  5.63e-16 &      \\ 
     &           &    3 &  1.98e-08 &  1.66e-15 &      \\ 
     &           &    4 &  1.97e-08 &  5.64e-16 &      \\ 
     &           &    5 &  1.97e-08 &  5.63e-16 &      \\ 
     &           &    6 &  5.80e-08 &  5.62e-16 &      \\ 
     &           &    7 &  1.98e-08 &  1.66e-15 &      \\ 
     &           &    8 &  1.97e-08 &  5.64e-16 &      \\ 
     &           &    9 &  1.97e-08 &  5.63e-16 &      \\ 
     &           &   10 &  5.80e-08 &  5.62e-16 &      \\ 
2553 &  2.55e+04 &   10 &           &           & iters  \\ 
 \hdashline 
     &           &    1 &  7.59e-09 &  2.25e-10 &      \\ 
     &           &    2 &  7.58e-09 &  2.17e-16 &      \\ 
     &           &    3 &  7.57e-09 &  2.16e-16 &      \\ 
     &           &    4 &  7.01e-08 &  2.16e-16 &      \\ 
     &           &    5 &  4.65e-08 &  2.00e-15 &      \\ 
     &           &    6 &  7.59e-09 &  1.33e-15 &      \\ 
     &           &    7 &  7.58e-09 &  2.17e-16 &      \\ 
     &           &    8 &  7.57e-09 &  2.16e-16 &      \\ 
     &           &    9 &  7.01e-08 &  2.16e-16 &      \\ 
     &           &   10 &  4.65e-08 &  2.00e-15 &      \\ 
2554 &  2.55e+04 &   10 &           &           & iters  \\ 
 \hdashline 
     &           &    1 &  9.40e-09 &  4.49e-10 &      \\ 
     &           &    2 &  6.83e-08 &  2.68e-16 &      \\ 
     &           &    3 &  9.49e-09 &  1.95e-15 &      \\ 
     &           &    4 &  9.47e-09 &  2.71e-16 &      \\ 
     &           &    5 &  9.46e-09 &  2.70e-16 &      \\ 
     &           &    6 &  9.45e-09 &  2.70e-16 &      \\ 
     &           &    7 &  9.43e-09 &  2.70e-16 &      \\ 
     &           &    8 &  9.42e-09 &  2.69e-16 &      \\ 
     &           &    9 &  9.41e-09 &  2.69e-16 &      \\ 
     &           &   10 &  6.83e-08 &  2.68e-16 &      \\ 
2555 &  2.55e+04 &   10 &           &           & iters  \\ 
 \hdashline 
     &           &    1 &  6.09e-08 &  7.28e-10 &      \\ 
     &           &    2 &  1.69e-08 &  1.74e-15 &      \\ 
     &           &    3 &  1.69e-08 &  4.82e-16 &      \\ 
     &           &    4 &  1.68e-08 &  4.81e-16 &      \\ 
     &           &    5 &  6.09e-08 &  4.80e-16 &      \\ 
     &           &    6 &  1.69e-08 &  1.74e-15 &      \\ 
     &           &    7 &  1.69e-08 &  4.82e-16 &      \\ 
     &           &    8 &  1.68e-08 &  4.82e-16 &      \\ 
     &           &    9 &  1.68e-08 &  4.81e-16 &      \\ 
     &           &   10 &  6.09e-08 &  4.80e-16 &      \\ 
2556 &  2.56e+04 &   10 &           &           & iters  \\ 
 \hdashline 
     &           &    1 &  5.58e-08 &  6.08e-10 &      \\ 
     &           &    2 &  2.20e-08 &  1.59e-15 &      \\ 
     &           &    3 &  2.20e-08 &  6.28e-16 &      \\ 
     &           &    4 &  2.19e-08 &  6.27e-16 &      \\ 
     &           &    5 &  5.58e-08 &  6.26e-16 &      \\ 
     &           &    6 &  2.20e-08 &  1.59e-15 &      \\ 
     &           &    7 &  2.20e-08 &  6.28e-16 &      \\ 
     &           &    8 &  5.58e-08 &  6.27e-16 &      \\ 
     &           &    9 &  2.20e-08 &  1.59e-15 &      \\ 
     &           &   10 &  2.20e-08 &  6.28e-16 &      \\ 
2557 &  2.56e+04 &   10 &           &           & iters  \\ 
 \hdashline 
     &           &    1 &  3.08e-09 &  1.01e-09 &      \\ 
     &           &    2 &  3.58e-08 &  8.80e-17 &      \\ 
     &           &    3 &  3.13e-09 &  1.02e-15 &      \\ 
     &           &    4 &  3.13e-09 &  8.94e-17 &      \\ 
     &           &    5 &  3.12e-09 &  8.92e-17 &      \\ 
     &           &    6 &  3.12e-09 &  8.91e-17 &      \\ 
     &           &    7 &  3.11e-09 &  8.90e-17 &      \\ 
     &           &    8 &  3.11e-09 &  8.88e-17 &      \\ 
     &           &    9 &  3.10e-09 &  8.87e-17 &      \\ 
     &           &   10 &  3.10e-09 &  8.86e-17 &      \\ 
2558 &  2.56e+04 &   10 &           &           & iters  \\ 
 \hdashline 
     &           &    1 &  6.60e-09 &  1.09e-09 &      \\ 
     &           &    2 &  6.59e-09 &  1.88e-16 &      \\ 
     &           &    3 &  6.59e-09 &  1.88e-16 &      \\ 
     &           &    4 &  3.23e-08 &  1.88e-16 &      \\ 
     &           &    5 &  6.62e-09 &  9.21e-16 &      \\ 
     &           &    6 &  6.61e-09 &  1.89e-16 &      \\ 
     &           &    7 &  6.60e-09 &  1.89e-16 &      \\ 
     &           &    8 &  6.59e-09 &  1.88e-16 &      \\ 
     &           &    9 &  6.58e-09 &  1.88e-16 &      \\ 
     &           &   10 &  3.23e-08 &  1.88e-16 &      \\ 
2559 &  2.56e+04 &   10 &           &           & iters  \\ 
 \hdashline 
     &           &    1 &  1.99e-08 &  9.16e-10 &      \\ 
     &           &    2 &  1.99e-08 &  5.68e-16 &      \\ 
     &           &    3 &  1.99e-08 &  5.68e-16 &      \\ 
     &           &    4 &  5.79e-08 &  5.67e-16 &      \\ 
     &           &    5 &  1.99e-08 &  1.65e-15 &      \\ 
     &           &    6 &  1.99e-08 &  5.68e-16 &      \\ 
     &           &    7 &  1.99e-08 &  5.67e-16 &      \\ 
     &           &    8 &  5.79e-08 &  5.67e-16 &      \\ 
     &           &    9 &  1.99e-08 &  1.65e-15 &      \\ 
     &           &   10 &  1.99e-08 &  5.68e-16 &      \\ 
2560 &  2.56e+04 &   10 &           &           & iters  \\ 
 \hdashline 
     &           &    1 &  4.79e-08 &  1.09e-09 &      \\ 
     &           &    2 &  2.99e-08 &  1.37e-15 &      \\ 
     &           &    3 &  4.78e-08 &  8.53e-16 &      \\ 
     &           &    4 &  2.99e-08 &  1.37e-15 &      \\ 
     &           &    5 &  2.99e-08 &  8.54e-16 &      \\ 
     &           &    6 &  4.79e-08 &  8.53e-16 &      \\ 
     &           &    7 &  2.99e-08 &  1.37e-15 &      \\ 
     &           &    8 &  4.78e-08 &  8.53e-16 &      \\ 
     &           &    9 &  2.99e-08 &  1.37e-15 &      \\ 
     &           &   10 &  2.99e-08 &  8.54e-16 &      \\ 
2561 &  2.56e+04 &   10 &           &           & iters  \\ 
 \hdashline 
     &           &    1 &  4.98e-08 &  8.25e-10 &      \\ 
     &           &    2 &  2.79e-08 &  1.42e-15 &      \\ 
     &           &    3 &  4.98e-08 &  7.98e-16 &      \\ 
     &           &    4 &  2.80e-08 &  1.42e-15 &      \\ 
     &           &    5 &  2.79e-08 &  7.98e-16 &      \\ 
     &           &    6 &  4.98e-08 &  7.97e-16 &      \\ 
     &           &    7 &  2.80e-08 &  1.42e-15 &      \\ 
     &           &    8 &  2.79e-08 &  7.98e-16 &      \\ 
     &           &    9 &  4.98e-08 &  7.97e-16 &      \\ 
     &           &   10 &  2.80e-08 &  1.42e-15 &      \\ 
2562 &  2.56e+04 &   10 &           &           & iters  \\ 
 \hdashline 
     &           &    1 &  3.32e-08 &  6.98e-10 &      \\ 
     &           &    2 &  4.45e-08 &  9.48e-16 &      \\ 
     &           &    3 &  3.32e-08 &  1.27e-15 &      \\ 
     &           &    4 &  4.45e-08 &  9.48e-16 &      \\ 
     &           &    5 &  3.32e-08 &  1.27e-15 &      \\ 
     &           &    6 &  4.45e-08 &  9.49e-16 &      \\ 
     &           &    7 &  3.33e-08 &  1.27e-15 &      \\ 
     &           &    8 &  3.32e-08 &  9.49e-16 &      \\ 
     &           &    9 &  4.45e-08 &  9.48e-16 &      \\ 
     &           &   10 &  3.32e-08 &  1.27e-15 &      \\ 
2563 &  2.56e+04 &   10 &           &           & iters  \\ 
 \hdashline 
     &           &    1 &  4.83e-08 &  1.61e-09 &      \\ 
     &           &    2 &  2.94e-08 &  1.38e-15 &      \\ 
     &           &    3 &  4.83e-08 &  8.39e-16 &      \\ 
     &           &    4 &  2.94e-08 &  1.38e-15 &      \\ 
     &           &    5 &  4.83e-08 &  8.40e-16 &      \\ 
     &           &    6 &  2.95e-08 &  1.38e-15 &      \\ 
     &           &    7 &  2.94e-08 &  8.41e-16 &      \\ 
     &           &    8 &  4.83e-08 &  8.40e-16 &      \\ 
     &           &    9 &  2.95e-08 &  1.38e-15 &      \\ 
     &           &   10 &  2.94e-08 &  8.41e-16 &      \\ 
2564 &  2.56e+04 &   10 &           &           & iters  \\ 
 \hdashline 
     &           &    1 &  2.05e-09 &  9.55e-10 &      \\ 
     &           &    2 &  2.05e-09 &  5.86e-17 &      \\ 
     &           &    3 &  2.05e-09 &  5.85e-17 &      \\ 
     &           &    4 &  2.05e-09 &  5.84e-17 &      \\ 
     &           &    5 &  2.04e-09 &  5.84e-17 &      \\ 
     &           &    6 &  2.04e-09 &  5.83e-17 &      \\ 
     &           &    7 &  2.04e-09 &  5.82e-17 &      \\ 
     &           &    8 &  2.03e-09 &  5.81e-17 &      \\ 
     &           &    9 &  2.03e-09 &  5.80e-17 &      \\ 
     &           &   10 &  2.03e-09 &  5.79e-17 &      \\ 
2565 &  2.56e+04 &   10 &           &           & iters  \\ 
 \hdashline 
     &           &    1 &  2.83e-08 &  9.87e-10 &      \\ 
     &           &    2 &  2.83e-08 &  8.08e-16 &      \\ 
     &           &    3 &  4.95e-08 &  8.07e-16 &      \\ 
     &           &    4 &  2.83e-08 &  1.41e-15 &      \\ 
     &           &    5 &  2.83e-08 &  8.07e-16 &      \\ 
     &           &    6 &  4.95e-08 &  8.06e-16 &      \\ 
     &           &    7 &  2.83e-08 &  1.41e-15 &      \\ 
     &           &    8 &  4.94e-08 &  8.07e-16 &      \\ 
     &           &    9 &  2.83e-08 &  1.41e-15 &      \\ 
     &           &   10 &  2.83e-08 &  8.08e-16 &      \\ 
2566 &  2.56e+04 &   10 &           &           & iters  \\ 
 \hdashline 
     &           &    1 &  3.65e-08 &  8.95e-10 &      \\ 
     &           &    2 &  4.12e-08 &  1.04e-15 &      \\ 
     &           &    3 &  3.66e-08 &  1.18e-15 &      \\ 
     &           &    4 &  4.12e-08 &  1.04e-15 &      \\ 
     &           &    5 &  3.66e-08 &  1.18e-15 &      \\ 
     &           &    6 &  4.12e-08 &  1.04e-15 &      \\ 
     &           &    7 &  3.66e-08 &  1.18e-15 &      \\ 
     &           &    8 &  4.12e-08 &  1.04e-15 &      \\ 
     &           &    9 &  3.66e-08 &  1.18e-15 &      \\ 
     &           &   10 &  4.12e-08 &  1.04e-15 &      \\ 
2567 &  2.57e+04 &   10 &           &           & iters  \\ 
 \hdashline 
     &           &    1 &  2.65e-08 &  2.44e-10 &      \\ 
     &           &    2 &  2.64e-08 &  7.55e-16 &      \\ 
     &           &    3 &  2.64e-08 &  7.54e-16 &      \\ 
     &           &    4 &  5.13e-08 &  7.53e-16 &      \\ 
     &           &    5 &  2.64e-08 &  1.47e-15 &      \\ 
     &           &    6 &  2.64e-08 &  7.54e-16 &      \\ 
     &           &    7 &  5.13e-08 &  7.53e-16 &      \\ 
     &           &    8 &  2.64e-08 &  1.47e-15 &      \\ 
     &           &    9 &  2.64e-08 &  7.54e-16 &      \\ 
     &           &   10 &  5.13e-08 &  7.53e-16 &      \\ 
2568 &  2.57e+04 &   10 &           &           & iters  \\ 
 \hdashline 
     &           &    1 &  2.38e-08 &  3.29e-10 &      \\ 
     &           &    2 &  5.39e-08 &  6.80e-16 &      \\ 
     &           &    3 &  2.39e-08 &  1.54e-15 &      \\ 
     &           &    4 &  2.38e-08 &  6.81e-16 &      \\ 
     &           &    5 &  5.39e-08 &  6.80e-16 &      \\ 
     &           &    6 &  2.39e-08 &  1.54e-15 &      \\ 
     &           &    7 &  2.38e-08 &  6.82e-16 &      \\ 
     &           &    8 &  2.38e-08 &  6.81e-16 &      \\ 
     &           &    9 &  5.39e-08 &  6.80e-16 &      \\ 
     &           &   10 &  2.39e-08 &  1.54e-15 &      \\ 
2569 &  2.57e+04 &   10 &           &           & iters  \\ 
 \hdashline 
     &           &    1 &  1.23e-08 &  8.43e-12 &      \\ 
     &           &    2 &  1.23e-08 &  3.51e-16 &      \\ 
     &           &    3 &  1.22e-08 &  3.50e-16 &      \\ 
     &           &    4 &  1.22e-08 &  3.50e-16 &      \\ 
     &           &    5 &  6.55e-08 &  3.49e-16 &      \\ 
     &           &    6 &  5.12e-08 &  1.87e-15 &      \\ 
     &           &    7 &  1.22e-08 &  1.46e-15 &      \\ 
     &           &    8 &  6.55e-08 &  3.49e-16 &      \\ 
     &           &    9 &  5.12e-08 &  1.87e-15 &      \\ 
     &           &   10 &  1.22e-08 &  1.46e-15 &      \\ 
2570 &  2.57e+04 &   10 &           &           & iters  \\ 
 \hdashline 
     &           &    1 &  5.84e-08 &  3.02e-10 &      \\ 
     &           &    2 &  1.94e-08 &  1.67e-15 &      \\ 
     &           &    3 &  1.94e-08 &  5.54e-16 &      \\ 
     &           &    4 &  1.94e-08 &  5.53e-16 &      \\ 
     &           &    5 &  5.84e-08 &  5.52e-16 &      \\ 
     &           &    6 &  1.94e-08 &  1.67e-15 &      \\ 
     &           &    7 &  1.94e-08 &  5.54e-16 &      \\ 
     &           &    8 &  1.94e-08 &  5.53e-16 &      \\ 
     &           &    9 &  5.84e-08 &  5.52e-16 &      \\ 
     &           &   10 &  1.94e-08 &  1.67e-15 &      \\ 
2571 &  2.57e+04 &   10 &           &           & iters  \\ 
 \hdashline 
     &           &    1 &  2.07e-08 &  3.40e-10 &      \\ 
     &           &    2 &  1.82e-08 &  5.91e-16 &      \\ 
     &           &    3 &  1.81e-08 &  5.19e-16 &      \\ 
     &           &    4 &  2.07e-08 &  5.18e-16 &      \\ 
     &           &    5 &  1.81e-08 &  5.92e-16 &      \\ 
     &           &    6 &  2.07e-08 &  5.18e-16 &      \\ 
     &           &    7 &  1.81e-08 &  5.91e-16 &      \\ 
     &           &    8 &  2.07e-08 &  5.18e-16 &      \\ 
     &           &    9 &  1.82e-08 &  5.91e-16 &      \\ 
     &           &   10 &  2.07e-08 &  5.18e-16 &      \\ 
2572 &  2.57e+04 &   10 &           &           & iters  \\ 
 \hdashline 
     &           &    1 &  2.47e-09 &  2.16e-10 &      \\ 
     &           &    2 &  2.47e-09 &  7.06e-17 &      \\ 
     &           &    3 &  2.47e-09 &  7.05e-17 &      \\ 
     &           &    4 &  2.46e-09 &  7.04e-17 &      \\ 
     &           &    5 &  2.46e-09 &  7.03e-17 &      \\ 
     &           &    6 &  2.46e-09 &  7.02e-17 &      \\ 
     &           &    7 &  2.45e-09 &  7.01e-17 &      \\ 
     &           &    8 &  2.45e-09 &  7.00e-17 &      \\ 
     &           &    9 &  2.45e-09 &  6.99e-17 &      \\ 
     &           &   10 &  2.44e-09 &  6.98e-17 &      \\ 
2573 &  2.57e+04 &   10 &           &           & iters  \\ 
 \hdashline 
     &           &    1 &  2.53e-09 &  2.79e-10 &      \\ 
     &           &    2 &  2.53e-09 &  7.23e-17 &      \\ 
     &           &    3 &  2.53e-09 &  7.22e-17 &      \\ 
     &           &    4 &  2.52e-09 &  7.21e-17 &      \\ 
     &           &    5 &  2.52e-09 &  7.20e-17 &      \\ 
     &           &    6 &  2.52e-09 &  7.19e-17 &      \\ 
     &           &    7 &  2.51e-09 &  7.18e-17 &      \\ 
     &           &    8 &  2.51e-09 &  7.17e-17 &      \\ 
     &           &    9 &  2.51e-09 &  7.16e-17 &      \\ 
     &           &   10 &  2.50e-09 &  7.15e-17 &      \\ 
2574 &  2.57e+04 &   10 &           &           & iters  \\ 
 \hdashline 
     &           &    1 &  4.26e-08 &  2.10e-10 &      \\ 
     &           &    2 &  3.52e-08 &  1.22e-15 &      \\ 
     &           &    3 &  3.51e-08 &  1.00e-15 &      \\ 
     &           &    4 &  4.26e-08 &  1.00e-15 &      \\ 
     &           &    5 &  3.51e-08 &  1.22e-15 &      \\ 
     &           &    6 &  4.26e-08 &  1.00e-15 &      \\ 
     &           &    7 &  3.51e-08 &  1.22e-15 &      \\ 
     &           &    8 &  4.26e-08 &  1.00e-15 &      \\ 
     &           &    9 &  3.51e-08 &  1.22e-15 &      \\ 
     &           &   10 &  4.26e-08 &  1.00e-15 &      \\ 
2575 &  2.57e+04 &   10 &           &           & iters  \\ 
 \hdashline 
     &           &    1 &  8.73e-10 &  9.04e-10 &      \\ 
     &           &    2 &  8.72e-10 &  2.49e-17 &      \\ 
     &           &    3 &  8.71e-10 &  2.49e-17 &      \\ 
     &           &    4 &  8.70e-10 &  2.49e-17 &      \\ 
     &           &    5 &  8.68e-10 &  2.48e-17 &      \\ 
     &           &    6 &  8.67e-10 &  2.48e-17 &      \\ 
     &           &    7 &  8.66e-10 &  2.48e-17 &      \\ 
     &           &    8 &  8.65e-10 &  2.47e-17 &      \\ 
     &           &    9 &  8.64e-10 &  2.47e-17 &      \\ 
     &           &   10 &  8.62e-10 &  2.46e-17 &      \\ 
2576 &  2.58e+04 &   10 &           &           & iters  \\ 
 \hdashline 
     &           &    1 &  5.08e-08 &  1.66e-10 &      \\ 
     &           &    2 &  1.19e-08 &  1.45e-15 &      \\ 
     &           &    3 &  1.19e-08 &  3.39e-16 &      \\ 
     &           &    4 &  6.58e-08 &  3.38e-16 &      \\ 
     &           &    5 &  5.08e-08 &  1.88e-15 &      \\ 
     &           &    6 &  1.19e-08 &  1.45e-15 &      \\ 
     &           &    7 &  1.18e-08 &  3.39e-16 &      \\ 
     &           &    8 &  6.59e-08 &  3.38e-16 &      \\ 
     &           &    9 &  1.19e-08 &  1.88e-15 &      \\ 
     &           &   10 &  1.19e-08 &  3.40e-16 &      \\ 
2577 &  2.58e+04 &   10 &           &           & iters  \\ 
 \hdashline 
     &           &    1 &  3.26e-08 &  9.47e-10 &      \\ 
     &           &    2 &  4.52e-08 &  9.30e-16 &      \\ 
     &           &    3 &  3.26e-08 &  1.29e-15 &      \\ 
     &           &    4 &  4.51e-08 &  9.30e-16 &      \\ 
     &           &    5 &  3.26e-08 &  1.29e-15 &      \\ 
     &           &    6 &  4.51e-08 &  9.31e-16 &      \\ 
     &           &    7 &  3.26e-08 &  1.29e-15 &      \\ 
     &           &    8 &  3.26e-08 &  9.31e-16 &      \\ 
     &           &    9 &  4.51e-08 &  9.30e-16 &      \\ 
     &           &   10 &  3.26e-08 &  1.29e-15 &      \\ 
2578 &  2.58e+04 &   10 &           &           & iters  \\ 
 \hdashline 
     &           &    1 &  2.60e-08 &  6.84e-10 &      \\ 
     &           &    2 &  1.29e-08 &  7.43e-16 &      \\ 
     &           &    3 &  2.60e-08 &  3.67e-16 &      \\ 
     &           &    4 &  1.29e-08 &  7.42e-16 &      \\ 
     &           &    5 &  1.29e-08 &  3.68e-16 &      \\ 
     &           &    6 &  2.60e-08 &  3.67e-16 &      \\ 
     &           &    7 &  1.29e-08 &  7.42e-16 &      \\ 
     &           &    8 &  1.29e-08 &  3.68e-16 &      \\ 
     &           &    9 &  2.60e-08 &  3.67e-16 &      \\ 
     &           &   10 &  1.29e-08 &  7.42e-16 &      \\ 
2579 &  2.58e+04 &   10 &           &           & iters  \\ 
 \hdashline 
     &           &    1 &  1.31e-08 &  1.26e-10 &      \\ 
     &           &    2 &  1.31e-08 &  3.75e-16 &      \\ 
     &           &    3 &  1.31e-08 &  3.74e-16 &      \\ 
     &           &    4 &  6.46e-08 &  3.74e-16 &      \\ 
     &           &    5 &  1.32e-08 &  1.84e-15 &      \\ 
     &           &    6 &  1.32e-08 &  3.76e-16 &      \\ 
     &           &    7 &  1.31e-08 &  3.75e-16 &      \\ 
     &           &    8 &  1.31e-08 &  3.75e-16 &      \\ 
     &           &    9 &  1.31e-08 &  3.74e-16 &      \\ 
     &           &   10 &  6.46e-08 &  3.74e-16 &      \\ 
2580 &  2.58e+04 &   10 &           &           & iters  \\ 
 \hdashline 
     &           &    1 &  6.66e-08 &  4.51e-11 &      \\ 
     &           &    2 &  1.12e-08 &  1.90e-15 &      \\ 
     &           &    3 &  1.12e-08 &  3.20e-16 &      \\ 
     &           &    4 &  1.12e-08 &  3.20e-16 &      \\ 
     &           &    5 &  1.12e-08 &  3.19e-16 &      \\ 
     &           &    6 &  1.12e-08 &  3.19e-16 &      \\ 
     &           &    7 &  6.66e-08 &  3.18e-16 &      \\ 
     &           &    8 &  8.89e-08 &  1.90e-15 &      \\ 
     &           &    9 &  6.66e-08 &  2.54e-15 &      \\ 
     &           &   10 &  1.12e-08 &  1.90e-15 &      \\ 
2581 &  2.58e+04 &   10 &           &           & iters  \\ 
 \hdashline 
     &           &    1 &  3.08e-08 &  1.97e-10 &      \\ 
     &           &    2 &  4.70e-08 &  8.78e-16 &      \\ 
     &           &    3 &  3.08e-08 &  1.34e-15 &      \\ 
     &           &    4 &  4.69e-08 &  8.79e-16 &      \\ 
     &           &    5 &  3.08e-08 &  1.34e-15 &      \\ 
     &           &    6 &  3.08e-08 &  8.80e-16 &      \\ 
     &           &    7 &  4.70e-08 &  8.78e-16 &      \\ 
     &           &    8 &  3.08e-08 &  1.34e-15 &      \\ 
     &           &    9 &  4.69e-08 &  8.79e-16 &      \\ 
     &           &   10 &  3.08e-08 &  1.34e-15 &      \\ 
2582 &  2.58e+04 &   10 &           &           & iters  \\ 
 \hdashline 
     &           &    1 &  9.91e-09 &  3.27e-10 &      \\ 
     &           &    2 &  9.90e-09 &  2.83e-16 &      \\ 
     &           &    3 &  6.78e-08 &  2.82e-16 &      \\ 
     &           &    4 &  4.88e-08 &  1.94e-15 &      \\ 
     &           &    5 &  9.91e-09 &  1.39e-15 &      \\ 
     &           &    6 &  9.90e-09 &  2.83e-16 &      \\ 
     &           &    7 &  6.78e-08 &  2.82e-16 &      \\ 
     &           &    8 &  4.88e-08 &  1.94e-15 &      \\ 
     &           &    9 &  9.91e-09 &  1.39e-15 &      \\ 
     &           &   10 &  9.89e-09 &  2.83e-16 &      \\ 
2583 &  2.58e+04 &   10 &           &           & iters  \\ 
 \hdashline 
     &           &    1 &  1.19e-08 &  3.03e-10 &      \\ 
     &           &    2 &  1.19e-08 &  3.39e-16 &      \\ 
     &           &    3 &  1.18e-08 &  3.38e-16 &      \\ 
     &           &    4 &  1.18e-08 &  3.38e-16 &      \\ 
     &           &    5 &  2.70e-08 &  3.37e-16 &      \\ 
     &           &    6 &  1.18e-08 &  7.72e-16 &      \\ 
     &           &    7 &  1.18e-08 &  3.38e-16 &      \\ 
     &           &    8 &  1.18e-08 &  3.38e-16 &      \\ 
     &           &    9 &  2.71e-08 &  3.37e-16 &      \\ 
     &           &   10 &  1.18e-08 &  7.72e-16 &      \\ 
2584 &  2.58e+04 &   10 &           &           & iters  \\ 
 \hdashline 
     &           &    1 &  3.65e-09 &  2.76e-11 &      \\ 
     &           &    2 &  3.64e-09 &  1.04e-16 &      \\ 
     &           &    3 &  3.64e-09 &  1.04e-16 &      \\ 
     &           &    4 &  3.63e-09 &  1.04e-16 &      \\ 
     &           &    5 &  3.62e-09 &  1.04e-16 &      \\ 
     &           &    6 &  3.62e-09 &  1.03e-16 &      \\ 
     &           &    7 &  3.52e-08 &  1.03e-16 &      \\ 
     &           &    8 &  3.66e-09 &  1.01e-15 &      \\ 
     &           &    9 &  3.66e-09 &  1.05e-16 &      \\ 
     &           &   10 &  3.65e-09 &  1.04e-16 &      \\ 
2585 &  2.58e+04 &   10 &           &           & iters  \\ 
 \hdashline 
     &           &    1 &  1.42e-08 &  2.93e-10 &      \\ 
     &           &    2 &  1.42e-08 &  4.06e-16 &      \\ 
     &           &    3 &  6.35e-08 &  4.06e-16 &      \\ 
     &           &    4 &  1.43e-08 &  1.81e-15 &      \\ 
     &           &    5 &  1.43e-08 &  4.08e-16 &      \\ 
     &           &    6 &  1.42e-08 &  4.07e-16 &      \\ 
     &           &    7 &  1.42e-08 &  4.07e-16 &      \\ 
     &           &    8 &  6.35e-08 &  4.06e-16 &      \\ 
     &           &    9 &  1.43e-08 &  1.81e-15 &      \\ 
     &           &   10 &  1.43e-08 &  4.08e-16 &      \\ 
2586 &  2.58e+04 &   10 &           &           & iters  \\ 
 \hdashline 
     &           &    1 &  1.61e-08 &  5.55e-11 &      \\ 
     &           &    2 &  2.28e-08 &  4.59e-16 &      \\ 
     &           &    3 &  1.61e-08 &  6.51e-16 &      \\ 
     &           &    4 &  2.28e-08 &  4.59e-16 &      \\ 
     &           &    5 &  1.61e-08 &  6.50e-16 &      \\ 
     &           &    6 &  1.61e-08 &  4.59e-16 &      \\ 
     &           &    7 &  2.28e-08 &  4.58e-16 &      \\ 
     &           &    8 &  1.61e-08 &  6.51e-16 &      \\ 
     &           &    9 &  2.28e-08 &  4.59e-16 &      \\ 
     &           &   10 &  1.61e-08 &  6.51e-16 &      \\ 
2587 &  2.59e+04 &   10 &           &           & iters  \\ 
 \hdashline 
     &           &    1 &  6.72e-09 &  2.86e-10 &      \\ 
     &           &    2 &  6.71e-09 &  1.92e-16 &      \\ 
     &           &    3 &  3.21e-08 &  1.92e-16 &      \\ 
     &           &    4 &  6.75e-09 &  9.17e-16 &      \\ 
     &           &    5 &  6.74e-09 &  1.93e-16 &      \\ 
     &           &    6 &  6.73e-09 &  1.92e-16 &      \\ 
     &           &    7 &  6.72e-09 &  1.92e-16 &      \\ 
     &           &    8 &  6.71e-09 &  1.92e-16 &      \\ 
     &           &    9 &  3.21e-08 &  1.92e-16 &      \\ 
     &           &   10 &  6.75e-09 &  9.17e-16 &      \\ 
2588 &  2.59e+04 &   10 &           &           & iters  \\ 
 \hdashline 
     &           &    1 &  2.15e-08 &  3.03e-10 &      \\ 
     &           &    2 &  5.62e-08 &  6.14e-16 &      \\ 
     &           &    3 &  2.16e-08 &  1.60e-15 &      \\ 
     &           &    4 &  2.15e-08 &  6.16e-16 &      \\ 
     &           &    5 &  5.62e-08 &  6.15e-16 &      \\ 
     &           &    6 &  2.16e-08 &  1.60e-15 &      \\ 
     &           &    7 &  2.16e-08 &  6.16e-16 &      \\ 
     &           &    8 &  2.15e-08 &  6.15e-16 &      \\ 
     &           &    9 &  5.62e-08 &  6.15e-16 &      \\ 
     &           &   10 &  2.16e-08 &  1.60e-15 &      \\ 
2589 &  2.59e+04 &   10 &           &           & iters  \\ 
 \hdashline 
     &           &    1 &  9.09e-09 &  2.06e-10 &      \\ 
     &           &    2 &  9.08e-09 &  2.60e-16 &      \\ 
     &           &    3 &  9.07e-09 &  2.59e-16 &      \\ 
     &           &    4 &  9.05e-09 &  2.59e-16 &      \\ 
     &           &    5 &  9.04e-09 &  2.58e-16 &      \\ 
     &           &    6 &  6.87e-08 &  2.58e-16 &      \\ 
     &           &    7 &  4.80e-08 &  1.96e-15 &      \\ 
     &           &    8 &  9.06e-09 &  1.37e-15 &      \\ 
     &           &    9 &  9.05e-09 &  2.59e-16 &      \\ 
     &           &   10 &  9.03e-09 &  2.58e-16 &      \\ 
2590 &  2.59e+04 &   10 &           &           & iters  \\ 
 \hdashline 
     &           &    1 &  6.62e-08 &  2.33e-10 &      \\ 
     &           &    2 &  1.16e-08 &  1.89e-15 &      \\ 
     &           &    3 &  1.15e-08 &  3.30e-16 &      \\ 
     &           &    4 &  1.15e-08 &  3.29e-16 &      \\ 
     &           &    5 &  1.15e-08 &  3.29e-16 &      \\ 
     &           &    6 &  1.15e-08 &  3.29e-16 &      \\ 
     &           &    7 &  6.62e-08 &  3.28e-16 &      \\ 
     &           &    8 &  1.16e-08 &  1.89e-15 &      \\ 
     &           &    9 &  1.16e-08 &  3.30e-16 &      \\ 
     &           &   10 &  1.15e-08 &  3.30e-16 &      \\ 
2591 &  2.59e+04 &   10 &           &           & iters  \\ 
 \hdashline 
     &           &    1 &  9.44e-09 &  4.35e-11 &      \\ 
     &           &    2 &  9.43e-09 &  2.69e-16 &      \\ 
     &           &    3 &  9.42e-09 &  2.69e-16 &      \\ 
     &           &    4 &  6.83e-08 &  2.69e-16 &      \\ 
     &           &    5 &  8.72e-08 &  1.95e-15 &      \\ 
     &           &    6 &  6.83e-08 &  2.49e-15 &      \\ 
     &           &    7 &  9.47e-09 &  1.95e-15 &      \\ 
     &           &    8 &  9.46e-09 &  2.70e-16 &      \\ 
     &           &    9 &  9.45e-09 &  2.70e-16 &      \\ 
     &           &   10 &  9.43e-09 &  2.70e-16 &      \\ 
2592 &  2.59e+04 &   10 &           &           & iters  \\ 
 \hdashline 
     &           &    1 &  2.80e-08 &  4.67e-10 &      \\ 
     &           &    2 &  2.79e-08 &  7.98e-16 &      \\ 
     &           &    3 &  4.98e-08 &  7.97e-16 &      \\ 
     &           &    4 &  2.79e-08 &  1.42e-15 &      \\ 
     &           &    5 &  4.98e-08 &  7.98e-16 &      \\ 
     &           &    6 &  2.80e-08 &  1.42e-15 &      \\ 
     &           &    7 &  2.79e-08 &  7.99e-16 &      \\ 
     &           &    8 &  4.98e-08 &  7.97e-16 &      \\ 
     &           &    9 &  2.80e-08 &  1.42e-15 &      \\ 
     &           &   10 &  2.79e-08 &  7.98e-16 &      \\ 
2593 &  2.59e+04 &   10 &           &           & iters  \\ 
 \hdashline 
     &           &    1 &  7.07e-09 &  7.01e-10 &      \\ 
     &           &    2 &  7.06e-09 &  2.02e-16 &      \\ 
     &           &    3 &  7.05e-09 &  2.02e-16 &      \\ 
     &           &    4 &  7.04e-09 &  2.01e-16 &      \\ 
     &           &    5 &  7.03e-09 &  2.01e-16 &      \\ 
     &           &    6 &  7.02e-09 &  2.01e-16 &      \\ 
     &           &    7 &  7.01e-09 &  2.00e-16 &      \\ 
     &           &    8 &  7.00e-09 &  2.00e-16 &      \\ 
     &           &    9 &  3.19e-08 &  2.00e-16 &      \\ 
     &           &   10 &  7.04e-09 &  9.09e-16 &      \\ 
2594 &  2.59e+04 &   10 &           &           & iters  \\ 
 \hdashline 
     &           &    1 &  6.80e-09 &  5.71e-10 &      \\ 
     &           &    2 &  6.79e-09 &  1.94e-16 &      \\ 
     &           &    3 &  6.78e-09 &  1.94e-16 &      \\ 
     &           &    4 &  6.77e-09 &  1.93e-16 &      \\ 
     &           &    5 &  6.76e-09 &  1.93e-16 &      \\ 
     &           &    6 &  7.09e-08 &  1.93e-16 &      \\ 
     &           &    7 &  8.45e-08 &  2.02e-15 &      \\ 
     &           &    8 &  7.10e-08 &  2.41e-15 &      \\ 
     &           &    9 &  6.83e-09 &  2.03e-15 &      \\ 
     &           &   10 &  6.82e-09 &  1.95e-16 &      \\ 
2595 &  2.59e+04 &   10 &           &           & iters  \\ 
 \hdashline 
     &           &    1 &  4.08e-08 &  1.18e-10 &      \\ 
     &           &    2 &  3.70e-08 &  1.16e-15 &      \\ 
     &           &    3 &  4.08e-08 &  1.06e-15 &      \\ 
     &           &    4 &  3.70e-08 &  1.16e-15 &      \\ 
     &           &    5 &  4.08e-08 &  1.06e-15 &      \\ 
     &           &    6 &  3.70e-08 &  1.16e-15 &      \\ 
     &           &    7 &  4.08e-08 &  1.06e-15 &      \\ 
     &           &    8 &  3.70e-08 &  1.16e-15 &      \\ 
     &           &    9 &  3.69e-08 &  1.06e-15 &      \\ 
     &           &   10 &  4.08e-08 &  1.05e-15 &      \\ 
2596 &  2.60e+04 &   10 &           &           & iters  \\ 
 \hdashline 
     &           &    1 &  2.43e-08 &  2.17e-10 &      \\ 
     &           &    2 &  5.34e-08 &  6.93e-16 &      \\ 
     &           &    3 &  2.43e-08 &  1.52e-15 &      \\ 
     &           &    4 &  2.43e-08 &  6.95e-16 &      \\ 
     &           &    5 &  5.34e-08 &  6.94e-16 &      \\ 
     &           &    6 &  2.43e-08 &  1.52e-15 &      \\ 
     &           &    7 &  2.43e-08 &  6.95e-16 &      \\ 
     &           &    8 &  5.34e-08 &  6.94e-16 &      \\ 
     &           &    9 &  2.44e-08 &  1.52e-15 &      \\ 
     &           &   10 &  2.43e-08 &  6.95e-16 &      \\ 
2597 &  2.60e+04 &   10 &           &           & iters  \\ 
 \hdashline 
     &           &    1 &  6.22e-09 &  3.82e-10 &      \\ 
     &           &    2 &  6.21e-09 &  1.77e-16 &      \\ 
     &           &    3 &  6.20e-09 &  1.77e-16 &      \\ 
     &           &    4 &  6.19e-09 &  1.77e-16 &      \\ 
     &           &    5 &  6.18e-09 &  1.77e-16 &      \\ 
     &           &    6 &  6.17e-09 &  1.76e-16 &      \\ 
     &           &    7 &  6.16e-09 &  1.76e-16 &      \\ 
     &           &    8 &  7.15e-08 &  1.76e-16 &      \\ 
     &           &    9 &  4.51e-08 &  2.04e-15 &      \\ 
     &           &   10 &  6.19e-09 &  1.29e-15 &      \\ 
2598 &  2.60e+04 &   10 &           &           & iters  \\ 
 \hdashline 
     &           &    1 &  9.35e-09 &  8.82e-10 &      \\ 
     &           &    2 &  9.34e-09 &  2.67e-16 &      \\ 
     &           &    3 &  9.32e-09 &  2.67e-16 &      \\ 
     &           &    4 &  9.31e-09 &  2.66e-16 &      \\ 
     &           &    5 &  9.30e-09 &  2.66e-16 &      \\ 
     &           &    6 &  6.84e-08 &  2.65e-16 &      \\ 
     &           &    7 &  9.38e-09 &  1.95e-15 &      \\ 
     &           &    8 &  9.37e-09 &  2.68e-16 &      \\ 
     &           &    9 &  9.36e-09 &  2.67e-16 &      \\ 
     &           &   10 &  9.34e-09 &  2.67e-16 &      \\ 
2599 &  2.60e+04 &   10 &           &           & iters  \\ 
 \hdashline 
     &           &    1 &  6.43e-10 &  6.96e-10 &      \\ 
     &           &    2 &  6.42e-10 &  1.83e-17 &      \\ 
     &           &    3 &  6.41e-10 &  1.83e-17 &      \\ 
     &           &    4 &  6.40e-10 &  1.83e-17 &      \\ 
     &           &    5 &  6.39e-10 &  1.83e-17 &      \\ 
     &           &    6 &  6.38e-10 &  1.82e-17 &      \\ 
     &           &    7 &  6.37e-10 &  1.82e-17 &      \\ 
     &           &    8 &  6.36e-10 &  1.82e-17 &      \\ 
     &           &    9 &  6.35e-10 &  1.82e-17 &      \\ 
     &           &   10 &  6.34e-10 &  1.81e-17 &      \\ 
2600 &  2.60e+04 &   10 &           &           & iters  \\ 
 \hdashline 
     &           &    1 &  1.89e-08 &  2.25e-10 &      \\ 
     &           &    2 &  5.88e-08 &  5.41e-16 &      \\ 
     &           &    3 &  1.90e-08 &  1.68e-15 &      \\ 
     &           &    4 &  1.90e-08 &  5.42e-16 &      \\ 
     &           &    5 &  1.89e-08 &  5.42e-16 &      \\ 
     &           &    6 &  5.88e-08 &  5.41e-16 &      \\ 
     &           &    7 &  1.90e-08 &  1.68e-15 &      \\ 
     &           &    8 &  1.90e-08 &  5.42e-16 &      \\ 
     &           &    9 &  1.90e-08 &  5.42e-16 &      \\ 
     &           &   10 &  5.88e-08 &  5.41e-16 &      \\ 
2601 &  2.60e+04 &   10 &           &           & iters  \\ 
 \hdashline 
     &           &    1 &  8.00e-08 &  5.24e-10 &      \\ 
     &           &    2 &  7.55e-08 &  2.28e-15 &      \\ 
     &           &    3 &  8.00e-08 &  2.15e-15 &      \\ 
     &           &    4 &  7.55e-08 &  2.28e-15 &      \\ 
     &           &    5 &  8.00e-08 &  2.15e-15 &      \\ 
     &           &    6 &  7.55e-08 &  2.28e-15 &      \\ 
     &           &    7 &  8.00e-08 &  2.15e-15 &      \\ 
     &           &    8 &  7.55e-08 &  2.28e-15 &      \\ 
     &           &    9 &  2.29e-09 &  2.16e-15 &      \\ 
     &           &   10 &  2.29e-09 &  6.53e-17 &      \\ 
2602 &  2.60e+04 &   10 &           &           & iters  \\ 
 \hdashline 
     &           &    1 &  1.23e-08 &  6.11e-10 &      \\ 
     &           &    2 &  1.23e-08 &  3.50e-16 &      \\ 
     &           &    3 &  1.22e-08 &  3.50e-16 &      \\ 
     &           &    4 &  6.55e-08 &  3.49e-16 &      \\ 
     &           &    5 &  1.23e-08 &  1.87e-15 &      \\ 
     &           &    6 &  1.23e-08 &  3.51e-16 &      \\ 
     &           &    7 &  1.23e-08 &  3.51e-16 &      \\ 
     &           &    8 &  1.23e-08 &  3.50e-16 &      \\ 
     &           &    9 &  1.22e-08 &  3.50e-16 &      \\ 
     &           &   10 &  6.55e-08 &  3.49e-16 &      \\ 
2603 &  2.60e+04 &   10 &           &           & iters  \\ 
 \hdashline 
     &           &    1 &  5.38e-08 &  3.75e-10 &      \\ 
     &           &    2 &  2.40e-08 &  1.54e-15 &      \\ 
     &           &    3 &  2.39e-08 &  6.84e-16 &      \\ 
     &           &    4 &  2.39e-08 &  6.83e-16 &      \\ 
     &           &    5 &  5.38e-08 &  6.82e-16 &      \\ 
     &           &    6 &  2.39e-08 &  1.54e-15 &      \\ 
     &           &    7 &  2.39e-08 &  6.83e-16 &      \\ 
     &           &    8 &  5.38e-08 &  6.82e-16 &      \\ 
     &           &    9 &  2.39e-08 &  1.54e-15 &      \\ 
     &           &   10 &  2.39e-08 &  6.83e-16 &      \\ 
2604 &  2.60e+04 &   10 &           &           & iters  \\ 
 \hdashline 
     &           &    1 &  9.44e-09 &  5.51e-10 &      \\ 
     &           &    2 &  9.42e-09 &  2.69e-16 &      \\ 
     &           &    3 &  9.41e-09 &  2.69e-16 &      \\ 
     &           &    4 &  6.83e-08 &  2.69e-16 &      \\ 
     &           &    5 &  4.83e-08 &  1.95e-15 &      \\ 
     &           &    6 &  9.42e-09 &  1.38e-15 &      \\ 
     &           &    7 &  9.41e-09 &  2.69e-16 &      \\ 
     &           &    8 &  6.83e-08 &  2.69e-16 &      \\ 
     &           &    9 &  4.83e-08 &  1.95e-15 &      \\ 
     &           &   10 &  9.43e-09 &  1.38e-15 &      \\ 
2605 &  2.60e+04 &   10 &           &           & iters  \\ 
 \hdashline 
     &           &    1 &  2.48e-08 &  4.73e-10 &      \\ 
     &           &    2 &  2.47e-08 &  7.07e-16 &      \\ 
     &           &    3 &  5.30e-08 &  7.06e-16 &      \\ 
     &           &    4 &  2.48e-08 &  1.51e-15 &      \\ 
     &           &    5 &  2.47e-08 &  7.07e-16 &      \\ 
     &           &    6 &  5.30e-08 &  7.06e-16 &      \\ 
     &           &    7 &  2.48e-08 &  1.51e-15 &      \\ 
     &           &    8 &  2.47e-08 &  7.07e-16 &      \\ 
     &           &    9 &  2.47e-08 &  7.06e-16 &      \\ 
     &           &   10 &  5.30e-08 &  7.05e-16 &      \\ 
2606 &  2.60e+04 &   10 &           &           & iters  \\ 
 \hdashline 
     &           &    1 &  1.39e-08 &  3.37e-10 &      \\ 
     &           &    2 &  1.39e-08 &  3.98e-16 &      \\ 
     &           &    3 &  2.49e-08 &  3.97e-16 &      \\ 
     &           &    4 &  1.39e-08 &  7.12e-16 &      \\ 
     &           &    5 &  1.39e-08 &  3.98e-16 &      \\ 
     &           &    6 &  2.49e-08 &  3.97e-16 &      \\ 
     &           &    7 &  1.39e-08 &  7.12e-16 &      \\ 
     &           &    8 &  2.49e-08 &  3.98e-16 &      \\ 
     &           &    9 &  1.39e-08 &  7.12e-16 &      \\ 
     &           &   10 &  1.39e-08 &  3.98e-16 &      \\ 
2607 &  2.61e+04 &   10 &           &           & iters  \\ 
 \hdashline 
     &           &    1 &  1.57e-08 &  5.51e-10 &      \\ 
     &           &    2 &  1.57e-08 &  4.48e-16 &      \\ 
     &           &    3 &  6.20e-08 &  4.48e-16 &      \\ 
     &           &    4 &  1.57e-08 &  1.77e-15 &      \\ 
     &           &    5 &  1.57e-08 &  4.49e-16 &      \\ 
     &           &    6 &  1.57e-08 &  4.49e-16 &      \\ 
     &           &    7 &  1.57e-08 &  4.48e-16 &      \\ 
     &           &    8 &  6.20e-08 &  4.47e-16 &      \\ 
     &           &    9 &  1.57e-08 &  1.77e-15 &      \\ 
     &           &   10 &  1.57e-08 &  4.49e-16 &      \\ 
2608 &  2.61e+04 &   10 &           &           & iters  \\ 
 \hdashline 
     &           &    1 &  2.90e-08 &  1.40e-10 &      \\ 
     &           &    2 &  4.87e-08 &  8.29e-16 &      \\ 
     &           &    3 &  2.91e-08 &  1.39e-15 &      \\ 
     &           &    4 &  2.90e-08 &  8.30e-16 &      \\ 
     &           &    5 &  4.87e-08 &  8.29e-16 &      \\ 
     &           &    6 &  2.91e-08 &  1.39e-15 &      \\ 
     &           &    7 &  4.87e-08 &  8.29e-16 &      \\ 
     &           &    8 &  2.91e-08 &  1.39e-15 &      \\ 
     &           &    9 &  2.90e-08 &  8.30e-16 &      \\ 
     &           &   10 &  4.87e-08 &  8.29e-16 &      \\ 
2609 &  2.61e+04 &   10 &           &           & iters  \\ 
 \hdashline 
     &           &    1 &  3.46e-08 &  9.43e-10 &      \\ 
     &           &    2 &  4.26e-09 &  9.88e-16 &      \\ 
     &           &    3 &  4.26e-09 &  1.22e-16 &      \\ 
     &           &    4 &  4.25e-09 &  1.22e-16 &      \\ 
     &           &    5 &  4.25e-09 &  1.21e-16 &      \\ 
     &           &    6 &  4.24e-09 &  1.21e-16 &      \\ 
     &           &    7 &  4.23e-09 &  1.21e-16 &      \\ 
     &           &    8 &  4.23e-09 &  1.21e-16 &      \\ 
     &           &    9 &  4.22e-09 &  1.21e-16 &      \\ 
     &           &   10 &  3.46e-08 &  1.20e-16 &      \\ 
2610 &  2.61e+04 &   10 &           &           & iters  \\ 
 \hdashline 
     &           &    1 &  2.73e-08 &  1.12e-09 &      \\ 
     &           &    2 &  5.05e-08 &  7.78e-16 &      \\ 
     &           &    3 &  2.73e-08 &  1.44e-15 &      \\ 
     &           &    4 &  2.72e-08 &  7.79e-16 &      \\ 
     &           &    5 &  5.05e-08 &  7.78e-16 &      \\ 
     &           &    6 &  2.73e-08 &  1.44e-15 &      \\ 
     &           &    7 &  2.72e-08 &  7.79e-16 &      \\ 
     &           &    8 &  5.05e-08 &  7.78e-16 &      \\ 
     &           &    9 &  2.73e-08 &  1.44e-15 &      \\ 
     &           &   10 &  2.72e-08 &  7.78e-16 &      \\ 
2611 &  2.61e+04 &   10 &           &           & iters  \\ 
 \hdashline 
     &           &    1 &  3.63e-08 &  8.72e-10 &      \\ 
     &           &    2 &  4.14e-08 &  1.04e-15 &      \\ 
     &           &    3 &  3.63e-08 &  1.18e-15 &      \\ 
     &           &    4 &  4.14e-08 &  1.04e-15 &      \\ 
     &           &    5 &  3.64e-08 &  1.18e-15 &      \\ 
     &           &    6 &  4.14e-08 &  1.04e-15 &      \\ 
     &           &    7 &  3.64e-08 &  1.18e-15 &      \\ 
     &           &    8 &  4.14e-08 &  1.04e-15 &      \\ 
     &           &    9 &  3.64e-08 &  1.18e-15 &      \\ 
     &           &   10 &  4.14e-08 &  1.04e-15 &      \\ 
2612 &  2.61e+04 &   10 &           &           & iters  \\ 
 \hdashline 
     &           &    1 &  9.65e-09 &  9.26e-10 &      \\ 
     &           &    2 &  6.81e-08 &  2.75e-16 &      \\ 
     &           &    3 &  9.73e-09 &  1.94e-15 &      \\ 
     &           &    4 &  9.72e-09 &  2.78e-16 &      \\ 
     &           &    5 &  9.70e-09 &  2.77e-16 &      \\ 
     &           &    6 &  9.69e-09 &  2.77e-16 &      \\ 
     &           &    7 &  9.68e-09 &  2.77e-16 &      \\ 
     &           &    8 &  9.66e-09 &  2.76e-16 &      \\ 
     &           &    9 &  9.65e-09 &  2.76e-16 &      \\ 
     &           &   10 &  6.81e-08 &  2.75e-16 &      \\ 
2613 &  2.61e+04 &   10 &           &           & iters  \\ 
 \hdashline 
     &           &    1 &  1.65e-09 &  1.86e-10 &      \\ 
     &           &    2 &  1.64e-09 &  4.70e-17 &      \\ 
     &           &    3 &  1.64e-09 &  4.69e-17 &      \\ 
     &           &    4 &  1.64e-09 &  4.68e-17 &      \\ 
     &           &    5 &  1.64e-09 &  4.68e-17 &      \\ 
     &           &    6 &  1.63e-09 &  4.67e-17 &      \\ 
     &           &    7 &  1.63e-09 &  4.66e-17 &      \\ 
     &           &    8 &  1.63e-09 &  4.66e-17 &      \\ 
     &           &    9 &  1.63e-09 &  4.65e-17 &      \\ 
     &           &   10 &  1.62e-09 &  4.64e-17 &      \\ 
2614 &  2.61e+04 &   10 &           &           & iters  \\ 
 \hdashline 
     &           &    1 &  8.66e-08 &  4.90e-10 &      \\ 
     &           &    2 &  6.89e-08 &  2.47e-15 &      \\ 
     &           &    3 &  8.92e-09 &  1.97e-15 &      \\ 
     &           &    4 &  8.90e-09 &  2.54e-16 &      \\ 
     &           &    5 &  8.89e-09 &  2.54e-16 &      \\ 
     &           &    6 &  8.88e-09 &  2.54e-16 &      \\ 
     &           &    7 &  8.87e-09 &  2.53e-16 &      \\ 
     &           &    8 &  8.85e-09 &  2.53e-16 &      \\ 
     &           &    9 &  6.88e-08 &  2.53e-16 &      \\ 
     &           &   10 &  8.66e-08 &  1.96e-15 &      \\ 
2615 &  2.61e+04 &   10 &           &           & iters  \\ 
 \hdashline 
     &           &    1 &  6.33e-08 &  2.46e-10 &      \\ 
     &           &    2 &  1.45e-08 &  1.81e-15 &      \\ 
     &           &    3 &  1.45e-08 &  4.14e-16 &      \\ 
     &           &    4 &  1.45e-08 &  4.13e-16 &      \\ 
     &           &    5 &  6.33e-08 &  4.13e-16 &      \\ 
     &           &    6 &  1.45e-08 &  1.81e-15 &      \\ 
     &           &    7 &  1.45e-08 &  4.15e-16 &      \\ 
     &           &    8 &  1.45e-08 &  4.14e-16 &      \\ 
     &           &    9 &  1.45e-08 &  4.13e-16 &      \\ 
     &           &   10 &  6.32e-08 &  4.13e-16 &      \\ 
2616 &  2.62e+04 &   10 &           &           & iters  \\ 
 \hdashline 
     &           &    1 &  9.07e-08 &  4.07e-10 &      \\ 
     &           &    2 &  6.48e-08 &  2.59e-15 &      \\ 
     &           &    3 &  1.30e-08 &  1.85e-15 &      \\ 
     &           &    4 &  1.30e-08 &  3.71e-16 &      \\ 
     &           &    5 &  1.30e-08 &  3.70e-16 &      \\ 
     &           &    6 &  6.47e-08 &  3.70e-16 &      \\ 
     &           &    7 &  9.07e-08 &  1.85e-15 &      \\ 
     &           &    8 &  6.48e-08 &  2.59e-15 &      \\ 
     &           &    9 &  1.30e-08 &  1.85e-15 &      \\ 
     &           &   10 &  1.30e-08 &  3.71e-16 &      \\ 
2617 &  2.62e+04 &   10 &           &           & iters  \\ 
 \hdashline 
     &           &    1 &  2.89e-08 &  7.37e-10 &      \\ 
     &           &    2 &  4.88e-08 &  8.25e-16 &      \\ 
     &           &    3 &  2.90e-08 &  1.39e-15 &      \\ 
     &           &    4 &  2.89e-08 &  8.26e-16 &      \\ 
     &           &    5 &  4.88e-08 &  8.25e-16 &      \\ 
     &           &    6 &  2.89e-08 &  1.39e-15 &      \\ 
     &           &    7 &  2.89e-08 &  8.26e-16 &      \\ 
     &           &    8 &  4.88e-08 &  8.25e-16 &      \\ 
     &           &    9 &  2.89e-08 &  1.39e-15 &      \\ 
     &           &   10 &  4.88e-08 &  8.26e-16 &      \\ 
2618 &  2.62e+04 &   10 &           &           & iters  \\ 
 \hdashline 
     &           &    1 &  4.96e-08 &  6.39e-10 &      \\ 
     &           &    2 &  2.81e-08 &  1.42e-15 &      \\ 
     &           &    3 &  4.96e-08 &  8.02e-16 &      \\ 
     &           &    4 &  2.81e-08 &  1.42e-15 &      \\ 
     &           &    5 &  2.81e-08 &  8.03e-16 &      \\ 
     &           &    6 &  4.96e-08 &  8.02e-16 &      \\ 
     &           &    7 &  2.81e-08 &  1.42e-15 &      \\ 
     &           &    8 &  2.81e-08 &  8.03e-16 &      \\ 
     &           &    9 &  4.96e-08 &  8.02e-16 &      \\ 
     &           &   10 &  2.81e-08 &  1.42e-15 &      \\ 
2619 &  2.62e+04 &   10 &           &           & iters  \\ 
 \hdashline 
     &           &    1 &  1.42e-08 &  6.92e-10 &      \\ 
     &           &    2 &  6.35e-08 &  4.06e-16 &      \\ 
     &           &    3 &  1.43e-08 &  1.81e-15 &      \\ 
     &           &    4 &  1.43e-08 &  4.08e-16 &      \\ 
     &           &    5 &  1.43e-08 &  4.08e-16 &      \\ 
     &           &    6 &  1.42e-08 &  4.07e-16 &      \\ 
     &           &    7 &  6.35e-08 &  4.06e-16 &      \\ 
     &           &    8 &  1.43e-08 &  1.81e-15 &      \\ 
     &           &    9 &  1.43e-08 &  4.08e-16 &      \\ 
     &           &   10 &  1.43e-08 &  4.08e-16 &      \\ 
2620 &  2.62e+04 &   10 &           &           & iters  \\ 
 \hdashline 
     &           &    1 &  1.04e-08 &  4.16e-10 &      \\ 
     &           &    2 &  2.85e-08 &  2.97e-16 &      \\ 
     &           &    3 &  1.04e-08 &  8.12e-16 &      \\ 
     &           &    4 &  1.04e-08 &  2.98e-16 &      \\ 
     &           &    5 &  1.04e-08 &  2.97e-16 &      \\ 
     &           &    6 &  2.85e-08 &  2.97e-16 &      \\ 
     &           &    7 &  1.04e-08 &  8.12e-16 &      \\ 
     &           &    8 &  1.04e-08 &  2.98e-16 &      \\ 
     &           &    9 &  2.85e-08 &  2.97e-16 &      \\ 
     &           &   10 &  1.04e-08 &  8.12e-16 &      \\ 
2621 &  2.62e+04 &   10 &           &           & iters  \\ 
 \hdashline 
     &           &    1 &  4.70e-08 &  3.20e-10 &      \\ 
     &           &    2 &  3.07e-08 &  1.34e-15 &      \\ 
     &           &    3 &  3.07e-08 &  8.77e-16 &      \\ 
     &           &    4 &  4.70e-08 &  8.76e-16 &      \\ 
     &           &    5 &  3.07e-08 &  1.34e-15 &      \\ 
     &           &    6 &  4.70e-08 &  8.76e-16 &      \\ 
     &           &    7 &  3.07e-08 &  1.34e-15 &      \\ 
     &           &    8 &  3.07e-08 &  8.77e-16 &      \\ 
     &           &    9 &  4.70e-08 &  8.76e-16 &      \\ 
     &           &   10 &  3.07e-08 &  1.34e-15 &      \\ 
2622 &  2.62e+04 &   10 &           &           & iters  \\ 
 \hdashline 
     &           &    1 &  1.29e-08 &  6.49e-10 &      \\ 
     &           &    2 &  1.29e-08 &  3.68e-16 &      \\ 
     &           &    3 &  2.60e-08 &  3.68e-16 &      \\ 
     &           &    4 &  1.29e-08 &  7.41e-16 &      \\ 
     &           &    5 &  1.29e-08 &  3.68e-16 &      \\ 
     &           &    6 &  2.60e-08 &  3.68e-16 &      \\ 
     &           &    7 &  1.29e-08 &  7.41e-16 &      \\ 
     &           &    8 &  1.29e-08 &  3.68e-16 &      \\ 
     &           &    9 &  2.60e-08 &  3.68e-16 &      \\ 
     &           &   10 &  1.29e-08 &  7.41e-16 &      \\ 
2623 &  2.62e+04 &   10 &           &           & iters  \\ 
 \hdashline 
     &           &    1 &  5.15e-08 &  7.93e-10 &      \\ 
     &           &    2 &  2.63e-08 &  1.47e-15 &      \\ 
     &           &    3 &  2.62e-08 &  7.49e-16 &      \\ 
     &           &    4 &  5.15e-08 &  7.48e-16 &      \\ 
     &           &    5 &  2.63e-08 &  1.47e-15 &      \\ 
     &           &    6 &  2.62e-08 &  7.49e-16 &      \\ 
     &           &    7 &  5.15e-08 &  7.48e-16 &      \\ 
     &           &    8 &  2.63e-08 &  1.47e-15 &      \\ 
     &           &    9 &  2.62e-08 &  7.49e-16 &      \\ 
     &           &   10 &  5.15e-08 &  7.48e-16 &      \\ 
2624 &  2.62e+04 &   10 &           &           & iters  \\ 
 \hdashline 
     &           &    1 &  7.60e-08 &  9.84e-10 &      \\ 
     &           &    2 &  1.77e-09 &  2.17e-15 &      \\ 
     &           &    3 &  1.77e-09 &  5.06e-17 &      \\ 
     &           &    4 &  1.77e-09 &  5.05e-17 &      \\ 
     &           &    5 &  1.77e-09 &  5.05e-17 &      \\ 
     &           &    6 &  1.76e-09 &  5.04e-17 &      \\ 
     &           &    7 &  1.76e-09 &  5.03e-17 &      \\ 
     &           &    8 &  1.76e-09 &  5.02e-17 &      \\ 
     &           &    9 &  1.76e-09 &  5.02e-17 &      \\ 
     &           &   10 &  1.75e-09 &  5.01e-17 &      \\ 
2625 &  2.62e+04 &   10 &           &           & iters  \\ 
 \hdashline 
     &           &    1 &  5.61e-08 &  5.84e-10 &      \\ 
     &           &    2 &  2.16e-08 &  1.60e-15 &      \\ 
     &           &    3 &  2.16e-08 &  6.17e-16 &      \\ 
     &           &    4 &  2.16e-08 &  6.16e-16 &      \\ 
     &           &    5 &  5.62e-08 &  6.15e-16 &      \\ 
     &           &    6 &  2.16e-08 &  1.60e-15 &      \\ 
     &           &    7 &  2.16e-08 &  6.17e-16 &      \\ 
     &           &    8 &  5.61e-08 &  6.16e-16 &      \\ 
     &           &    9 &  2.16e-08 &  1.60e-15 &      \\ 
     &           &   10 &  2.16e-08 &  6.17e-16 &      \\ 
2626 &  2.62e+04 &   10 &           &           & iters  \\ 
 \hdashline 
     &           &    1 &  1.91e-08 &  2.94e-10 &      \\ 
     &           &    2 &  5.87e-08 &  5.44e-16 &      \\ 
     &           &    3 &  1.91e-08 &  1.67e-15 &      \\ 
     &           &    4 &  1.91e-08 &  5.45e-16 &      \\ 
     &           &    5 &  1.91e-08 &  5.45e-16 &      \\ 
     &           &    6 &  5.87e-08 &  5.44e-16 &      \\ 
     &           &    7 &  1.91e-08 &  1.67e-15 &      \\ 
     &           &    8 &  1.91e-08 &  5.45e-16 &      \\ 
     &           &    9 &  1.91e-08 &  5.45e-16 &      \\ 
     &           &   10 &  5.87e-08 &  5.44e-16 &      \\ 
2627 &  2.63e+04 &   10 &           &           & iters  \\ 
 \hdashline 
     &           &    1 &  3.87e-08 &  5.39e-10 &      \\ 
     &           &    2 &  3.90e-08 &  1.10e-15 &      \\ 
     &           &    3 &  3.87e-08 &  1.11e-15 &      \\ 
     &           &    4 &  3.90e-08 &  1.10e-15 &      \\ 
     &           &    5 &  3.87e-08 &  1.11e-15 &      \\ 
     &           &    6 &  3.90e-08 &  1.10e-15 &      \\ 
     &           &    7 &  3.87e-08 &  1.11e-15 &      \\ 
     &           &    8 &  3.90e-08 &  1.10e-15 &      \\ 
     &           &    9 &  3.87e-08 &  1.11e-15 &      \\ 
     &           &   10 &  3.90e-08 &  1.10e-15 &      \\ 
2628 &  2.63e+04 &   10 &           &           & iters  \\ 
 \hdashline 
     &           &    1 &  5.22e-08 &  2.46e-10 &      \\ 
     &           &    2 &  2.56e-08 &  1.49e-15 &      \\ 
     &           &    3 &  2.55e-08 &  7.29e-16 &      \\ 
     &           &    4 &  5.22e-08 &  7.28e-16 &      \\ 
     &           &    5 &  2.56e-08 &  1.49e-15 &      \\ 
     &           &    6 &  2.55e-08 &  7.29e-16 &      \\ 
     &           &    7 &  5.22e-08 &  7.28e-16 &      \\ 
     &           &    8 &  2.56e-08 &  1.49e-15 &      \\ 
     &           &    9 &  2.55e-08 &  7.29e-16 &      \\ 
     &           &   10 &  5.22e-08 &  7.28e-16 &      \\ 
2629 &  2.63e+04 &   10 &           &           & iters  \\ 
 \hdashline 
     &           &    1 &  4.34e-09 &  1.38e-10 &      \\ 
     &           &    2 &  4.33e-09 &  1.24e-16 &      \\ 
     &           &    3 &  4.32e-09 &  1.24e-16 &      \\ 
     &           &    4 &  4.32e-09 &  1.23e-16 &      \\ 
     &           &    5 &  4.31e-09 &  1.23e-16 &      \\ 
     &           &    6 &  4.31e-09 &  1.23e-16 &      \\ 
     &           &    7 &  4.30e-09 &  1.23e-16 &      \\ 
     &           &    8 &  4.29e-09 &  1.23e-16 &      \\ 
     &           &    9 &  4.29e-09 &  1.23e-16 &      \\ 
     &           &   10 &  4.28e-09 &  1.22e-16 &      \\ 
2630 &  2.63e+04 &   10 &           &           & iters  \\ 
 \hdashline 
     &           &    1 &  6.14e-08 &  2.05e-11 &      \\ 
     &           &    2 &  1.63e-08 &  1.75e-15 &      \\ 
     &           &    3 &  1.63e-08 &  4.67e-16 &      \\ 
     &           &    4 &  6.14e-08 &  4.66e-16 &      \\ 
     &           &    5 &  1.64e-08 &  1.75e-15 &      \\ 
     &           &    6 &  1.64e-08 &  4.68e-16 &      \\ 
     &           &    7 &  1.63e-08 &  4.67e-16 &      \\ 
     &           &    8 &  1.63e-08 &  4.66e-16 &      \\ 
     &           &    9 &  6.14e-08 &  4.66e-16 &      \\ 
     &           &   10 &  1.64e-08 &  1.75e-15 &      \\ 
2631 &  2.63e+04 &   10 &           &           & iters  \\ 
 \hdashline 
     &           &    1 &  6.62e-08 &  4.02e-11 &      \\ 
     &           &    2 &  8.92e-08 &  1.89e-15 &      \\ 
     &           &    3 &  6.63e-08 &  2.55e-15 &      \\ 
     &           &    4 &  1.15e-08 &  1.89e-15 &      \\ 
     &           &    5 &  1.15e-08 &  3.29e-16 &      \\ 
     &           &    6 &  1.15e-08 &  3.28e-16 &      \\ 
     &           &    7 &  1.15e-08 &  3.28e-16 &      \\ 
     &           &    8 &  6.62e-08 &  3.27e-16 &      \\ 
     &           &    9 &  1.15e-08 &  1.89e-15 &      \\ 
     &           &   10 &  1.15e-08 &  3.30e-16 &      \\ 
2632 &  2.63e+04 &   10 &           &           & iters  \\ 
 \hdashline 
     &           &    1 &  2.15e-08 &  5.78e-10 &      \\ 
     &           &    2 &  2.14e-08 &  6.13e-16 &      \\ 
     &           &    3 &  5.63e-08 &  6.12e-16 &      \\ 
     &           &    4 &  2.15e-08 &  1.61e-15 &      \\ 
     &           &    5 &  2.15e-08 &  6.14e-16 &      \\ 
     &           &    6 &  2.14e-08 &  6.13e-16 &      \\ 
     &           &    7 &  5.63e-08 &  6.12e-16 &      \\ 
     &           &    8 &  2.15e-08 &  1.61e-15 &      \\ 
     &           &    9 &  2.15e-08 &  6.13e-16 &      \\ 
     &           &   10 &  2.14e-08 &  6.12e-16 &      \\ 
2633 &  2.63e+04 &   10 &           &           & iters  \\ 
 \hdashline 
     &           &    1 &  6.23e-08 &  1.25e-09 &      \\ 
     &           &    2 &  1.55e-08 &  1.78e-15 &      \\ 
     &           &    3 &  1.54e-08 &  4.41e-16 &      \\ 
     &           &    4 &  6.23e-08 &  4.41e-16 &      \\ 
     &           &    5 &  1.55e-08 &  1.78e-15 &      \\ 
     &           &    6 &  1.55e-08 &  4.43e-16 &      \\ 
     &           &    7 &  1.55e-08 &  4.42e-16 &      \\ 
     &           &    8 &  1.54e-08 &  4.41e-16 &      \\ 
     &           &    9 &  6.23e-08 &  4.41e-16 &      \\ 
     &           &   10 &  1.55e-08 &  1.78e-15 &      \\ 
2634 &  2.63e+04 &   10 &           &           & iters  \\ 
 \hdashline 
     &           &    1 &  5.40e-08 &  7.11e-10 &      \\ 
     &           &    2 &  2.38e-08 &  1.54e-15 &      \\ 
     &           &    3 &  2.38e-08 &  6.80e-16 &      \\ 
     &           &    4 &  5.39e-08 &  6.79e-16 &      \\ 
     &           &    5 &  2.38e-08 &  1.54e-15 &      \\ 
     &           &    6 &  2.38e-08 &  6.80e-16 &      \\ 
     &           &    7 &  2.38e-08 &  6.79e-16 &      \\ 
     &           &    8 &  5.40e-08 &  6.78e-16 &      \\ 
     &           &    9 &  2.38e-08 &  1.54e-15 &      \\ 
     &           &   10 &  2.38e-08 &  6.79e-16 &      \\ 
2635 &  2.63e+04 &   10 &           &           & iters  \\ 
 \hdashline 
     &           &    1 &  2.77e-08 &  1.23e-09 &      \\ 
     &           &    2 &  1.11e-08 &  7.92e-16 &      \\ 
     &           &    3 &  1.11e-08 &  3.18e-16 &      \\ 
     &           &    4 &  2.77e-08 &  3.17e-16 &      \\ 
     &           &    5 &  1.11e-08 &  7.92e-16 &      \\ 
     &           &    6 &  1.11e-08 &  3.18e-16 &      \\ 
     &           &    7 &  1.11e-08 &  3.18e-16 &      \\ 
     &           &    8 &  2.77e-08 &  3.17e-16 &      \\ 
     &           &    9 &  1.11e-08 &  7.92e-16 &      \\ 
     &           &   10 &  1.11e-08 &  3.18e-16 &      \\ 
2636 &  2.64e+04 &   10 &           &           & iters  \\ 
 \hdashline 
     &           &    1 &  3.79e-08 &  2.14e-09 &      \\ 
     &           &    2 &  3.98e-08 &  1.08e-15 &      \\ 
     &           &    3 &  3.80e-08 &  1.14e-15 &      \\ 
     &           &    4 &  3.98e-08 &  1.08e-15 &      \\ 
     &           &    5 &  3.80e-08 &  1.14e-15 &      \\ 
     &           &    6 &  3.98e-08 &  1.08e-15 &      \\ 
     &           &    7 &  3.80e-08 &  1.14e-15 &      \\ 
     &           &    8 &  3.98e-08 &  1.08e-15 &      \\ 
     &           &    9 &  3.80e-08 &  1.14e-15 &      \\ 
     &           &   10 &  3.98e-08 &  1.08e-15 &      \\ 
2637 &  2.64e+04 &   10 &           &           & iters  \\ 
 \hdashline 
     &           &    1 &  1.64e-08 &  1.41e-09 &      \\ 
     &           &    2 &  1.63e-08 &  4.67e-16 &      \\ 
     &           &    3 &  1.63e-08 &  4.66e-16 &      \\ 
     &           &    4 &  6.14e-08 &  4.66e-16 &      \\ 
     &           &    5 &  1.64e-08 &  1.75e-15 &      \\ 
     &           &    6 &  1.64e-08 &  4.67e-16 &      \\ 
     &           &    7 &  1.63e-08 &  4.67e-16 &      \\ 
     &           &    8 &  1.63e-08 &  4.66e-16 &      \\ 
     &           &    9 &  6.14e-08 &  4.65e-16 &      \\ 
     &           &   10 &  1.64e-08 &  1.75e-15 &      \\ 
2638 &  2.64e+04 &   10 &           &           & iters  \\ 
 \hdashline 
     &           &    1 &  4.28e-08 &  4.52e-10 &      \\ 
     &           &    2 &  3.49e-08 &  1.22e-15 &      \\ 
     &           &    3 &  3.49e-08 &  9.97e-16 &      \\ 
     &           &    4 &  4.29e-08 &  9.95e-16 &      \\ 
     &           &    5 &  3.49e-08 &  1.22e-15 &      \\ 
     &           &    6 &  4.28e-08 &  9.96e-16 &      \\ 
     &           &    7 &  3.49e-08 &  1.22e-15 &      \\ 
     &           &    8 &  4.28e-08 &  9.96e-16 &      \\ 
     &           &    9 &  3.49e-08 &  1.22e-15 &      \\ 
     &           &   10 &  4.28e-08 &  9.96e-16 &      \\ 
2639 &  2.64e+04 &   10 &           &           & iters  \\ 
 \hdashline 
     &           &    1 &  2.48e-08 &  5.51e-10 &      \\ 
     &           &    2 &  2.48e-08 &  7.09e-16 &      \\ 
     &           &    3 &  2.48e-08 &  7.08e-16 &      \\ 
     &           &    4 &  5.29e-08 &  7.07e-16 &      \\ 
     &           &    5 &  2.48e-08 &  1.51e-15 &      \\ 
     &           &    6 &  2.48e-08 &  7.08e-16 &      \\ 
     &           &    7 &  5.29e-08 &  7.07e-16 &      \\ 
     &           &    8 &  2.48e-08 &  1.51e-15 &      \\ 
     &           &    9 &  2.48e-08 &  7.08e-16 &      \\ 
     &           &   10 &  5.29e-08 &  7.07e-16 &      \\ 
2640 &  2.64e+04 &   10 &           &           & iters  \\ 
 \hdashline 
     &           &    1 &  6.75e-09 &  8.32e-10 &      \\ 
     &           &    2 &  7.10e-08 &  1.93e-16 &      \\ 
     &           &    3 &  8.45e-08 &  2.02e-15 &      \\ 
     &           &    4 &  7.10e-08 &  2.41e-15 &      \\ 
     &           &    5 &  6.82e-09 &  2.03e-15 &      \\ 
     &           &    6 &  6.81e-09 &  1.95e-16 &      \\ 
     &           &    7 &  6.80e-09 &  1.94e-16 &      \\ 
     &           &    8 &  6.79e-09 &  1.94e-16 &      \\ 
     &           &    9 &  6.78e-09 &  1.94e-16 &      \\ 
     &           &   10 &  6.77e-09 &  1.94e-16 &      \\ 
2641 &  2.64e+04 &   10 &           &           & iters  \\ 
 \hdashline 
     &           &    1 &  8.68e-09 &  6.09e-10 &      \\ 
     &           &    2 &  8.67e-09 &  2.48e-16 &      \\ 
     &           &    3 &  8.66e-09 &  2.47e-16 &      \\ 
     &           &    4 &  8.65e-09 &  2.47e-16 &      \\ 
     &           &    5 &  8.63e-09 &  2.47e-16 &      \\ 
     &           &    6 &  6.91e-08 &  2.46e-16 &      \\ 
     &           &    7 &  8.64e-08 &  1.97e-15 &      \\ 
     &           &    8 &  6.91e-08 &  2.47e-15 &      \\ 
     &           &    9 &  8.70e-09 &  1.97e-15 &      \\ 
     &           &   10 &  8.68e-09 &  2.48e-16 &      \\ 
2642 &  2.64e+04 &   10 &           &           & iters  \\ 
 \hdashline 
     &           &    1 &  1.39e-08 &  8.71e-10 &      \\ 
     &           &    2 &  6.38e-08 &  3.97e-16 &      \\ 
     &           &    3 &  1.40e-08 &  1.82e-15 &      \\ 
     &           &    4 &  1.40e-08 &  3.99e-16 &      \\ 
     &           &    5 &  1.39e-08 &  3.98e-16 &      \\ 
     &           &    6 &  1.39e-08 &  3.98e-16 &      \\ 
     &           &    7 &  1.39e-08 &  3.97e-16 &      \\ 
     &           &    8 &  6.38e-08 &  3.97e-16 &      \\ 
     &           &    9 &  1.40e-08 &  1.82e-15 &      \\ 
     &           &   10 &  1.40e-08 &  3.99e-16 &      \\ 
2643 &  2.64e+04 &   10 &           &           & iters  \\ 
 \hdashline 
     &           &    1 &  3.07e-09 &  8.61e-10 &      \\ 
     &           &    2 &  3.07e-09 &  8.77e-17 &      \\ 
     &           &    3 &  3.07e-09 &  8.76e-17 &      \\ 
     &           &    4 &  3.06e-09 &  8.75e-17 &      \\ 
     &           &    5 &  3.06e-09 &  8.74e-17 &      \\ 
     &           &    6 &  3.05e-09 &  8.72e-17 &      \\ 
     &           &    7 &  7.46e-08 &  8.71e-17 &      \\ 
     &           &    8 &  8.08e-08 &  2.13e-15 &      \\ 
     &           &    9 &  7.46e-08 &  2.31e-15 &      \\ 
     &           &   10 &  8.08e-08 &  2.13e-15 &      \\ 
2644 &  2.64e+04 &   10 &           &           & iters  \\ 
 \hdashline 
     &           &    1 &  3.47e-08 &  4.56e-10 &      \\ 
     &           &    2 &  4.23e-09 &  9.89e-16 &      \\ 
     &           &    3 &  4.23e-09 &  1.21e-16 &      \\ 
     &           &    4 &  4.22e-09 &  1.21e-16 &      \\ 
     &           &    5 &  3.46e-08 &  1.20e-16 &      \\ 
     &           &    6 &  4.27e-09 &  9.88e-16 &      \\ 
     &           &    7 &  4.26e-09 &  1.22e-16 &      \\ 
     &           &    8 &  4.25e-09 &  1.22e-16 &      \\ 
     &           &    9 &  4.25e-09 &  1.21e-16 &      \\ 
     &           &   10 &  4.24e-09 &  1.21e-16 &      \\ 
2645 &  2.64e+04 &   10 &           &           & iters  \\ 
 \hdashline 
     &           &    1 &  9.48e-09 &  3.98e-10 &      \\ 
     &           &    2 &  9.47e-09 &  2.71e-16 &      \\ 
     &           &    3 &  9.46e-09 &  2.70e-16 &      \\ 
     &           &    4 &  9.44e-09 &  2.70e-16 &      \\ 
     &           &    5 &  6.83e-08 &  2.70e-16 &      \\ 
     &           &    6 &  4.84e-08 &  1.95e-15 &      \\ 
     &           &    7 &  9.46e-09 &  1.38e-15 &      \\ 
     &           &    8 &  9.45e-09 &  2.70e-16 &      \\ 
     &           &    9 &  6.83e-08 &  2.70e-16 &      \\ 
     &           &   10 &  4.84e-08 &  1.95e-15 &      \\ 
2646 &  2.64e+04 &   10 &           &           & iters  \\ 
 \hdashline 
     &           &    1 &  7.51e-09 &  5.57e-10 &      \\ 
     &           &    2 &  7.50e-09 &  2.14e-16 &      \\ 
     &           &    3 &  7.49e-09 &  2.14e-16 &      \\ 
     &           &    4 &  7.48e-09 &  2.14e-16 &      \\ 
     &           &    5 &  7.47e-09 &  2.14e-16 &      \\ 
     &           &    6 &  7.02e-08 &  2.13e-16 &      \\ 
     &           &    7 &  8.52e-08 &  2.00e-15 &      \\ 
     &           &    8 &  7.02e-08 &  2.43e-15 &      \\ 
     &           &    9 &  7.54e-09 &  2.00e-15 &      \\ 
     &           &   10 &  7.53e-09 &  2.15e-16 &      \\ 
2647 &  2.65e+04 &   10 &           &           & iters  \\ 
 \hdashline 
     &           &    1 &  2.73e-08 &  5.56e-10 &      \\ 
     &           &    2 &  2.73e-08 &  7.80e-16 &      \\ 
     &           &    3 &  5.04e-08 &  7.79e-16 &      \\ 
     &           &    4 &  2.73e-08 &  1.44e-15 &      \\ 
     &           &    5 &  2.73e-08 &  7.80e-16 &      \\ 
     &           &    6 &  5.04e-08 &  7.79e-16 &      \\ 
     &           &    7 &  2.73e-08 &  1.44e-15 &      \\ 
     &           &    8 &  5.04e-08 &  7.80e-16 &      \\ 
     &           &    9 &  2.74e-08 &  1.44e-15 &      \\ 
     &           &   10 &  2.73e-08 &  7.81e-16 &      \\ 
2648 &  2.65e+04 &   10 &           &           & iters  \\ 
 \hdashline 
     &           &    1 &  9.32e-10 &  7.35e-11 &      \\ 
     &           &    2 &  9.31e-10 &  2.66e-17 &      \\ 
     &           &    3 &  9.29e-10 &  2.66e-17 &      \\ 
     &           &    4 &  9.28e-10 &  2.65e-17 &      \\ 
     &           &    5 &  9.27e-10 &  2.65e-17 &      \\ 
     &           &    6 &  9.25e-10 &  2.64e-17 &      \\ 
     &           &    7 &  9.24e-10 &  2.64e-17 &      \\ 
     &           &    8 &  9.23e-10 &  2.64e-17 &      \\ 
     &           &    9 &  9.21e-10 &  2.63e-17 &      \\ 
     &           &   10 &  9.20e-10 &  2.63e-17 &      \\ 
2649 &  2.65e+04 &   10 &           &           & iters  \\ 
 \hdashline 
     &           &    1 &  1.37e-08 &  6.20e-10 &      \\ 
     &           &    2 &  1.37e-08 &  3.91e-16 &      \\ 
     &           &    3 &  1.37e-08 &  3.91e-16 &      \\ 
     &           &    4 &  1.36e-08 &  3.90e-16 &      \\ 
     &           &    5 &  6.41e-08 &  3.89e-16 &      \\ 
     &           &    6 &  1.37e-08 &  1.83e-15 &      \\ 
     &           &    7 &  1.37e-08 &  3.92e-16 &      \\ 
     &           &    8 &  1.37e-08 &  3.91e-16 &      \\ 
     &           &    9 &  1.37e-08 &  3.90e-16 &      \\ 
     &           &   10 &  1.36e-08 &  3.90e-16 &      \\ 
2650 &  2.65e+04 &   10 &           &           & iters  \\ 
 \hdashline 
     &           &    1 &  5.11e-08 &  6.41e-10 &      \\ 
     &           &    2 &  2.66e-08 &  1.46e-15 &      \\ 
     &           &    3 &  5.11e-08 &  7.60e-16 &      \\ 
     &           &    4 &  2.66e-08 &  1.46e-15 &      \\ 
     &           &    5 &  2.66e-08 &  7.61e-16 &      \\ 
     &           &    6 &  5.11e-08 &  7.59e-16 &      \\ 
     &           &    7 &  2.66e-08 &  1.46e-15 &      \\ 
     &           &    8 &  2.66e-08 &  7.60e-16 &      \\ 
     &           &    9 &  5.11e-08 &  7.59e-16 &      \\ 
     &           &   10 &  2.66e-08 &  1.46e-15 &      \\ 
2651 &  2.65e+04 &   10 &           &           & iters  \\ 
 \hdashline 
     &           &    1 &  3.32e-08 &  5.36e-10 &      \\ 
     &           &    2 &  4.45e-08 &  9.48e-16 &      \\ 
     &           &    3 &  3.32e-08 &  1.27e-15 &      \\ 
     &           &    4 &  4.45e-08 &  9.48e-16 &      \\ 
     &           &    5 &  3.32e-08 &  1.27e-15 &      \\ 
     &           &    6 &  4.45e-08 &  9.49e-16 &      \\ 
     &           &    7 &  3.33e-08 &  1.27e-15 &      \\ 
     &           &    8 &  3.32e-08 &  9.49e-16 &      \\ 
     &           &    9 &  4.45e-08 &  9.48e-16 &      \\ 
     &           &   10 &  3.32e-08 &  1.27e-15 &      \\ 
2652 &  2.65e+04 &   10 &           &           & iters  \\ 
 \hdashline 
     &           &    1 &  2.78e-08 &  8.48e-10 &      \\ 
     &           &    2 &  2.78e-08 &  7.95e-16 &      \\ 
     &           &    3 &  4.99e-08 &  7.93e-16 &      \\ 
     &           &    4 &  2.78e-08 &  1.43e-15 &      \\ 
     &           &    5 &  2.78e-08 &  7.94e-16 &      \\ 
     &           &    6 &  4.99e-08 &  7.93e-16 &      \\ 
     &           &    7 &  2.78e-08 &  1.43e-15 &      \\ 
     &           &    8 &  2.78e-08 &  7.94e-16 &      \\ 
     &           &    9 &  4.99e-08 &  7.93e-16 &      \\ 
     &           &   10 &  2.78e-08 &  1.43e-15 &      \\ 
2653 &  2.65e+04 &   10 &           &           & iters  \\ 
 \hdashline 
     &           &    1 &  5.02e-09 &  9.59e-10 &      \\ 
     &           &    2 &  5.02e-09 &  1.43e-16 &      \\ 
     &           &    3 &  5.01e-09 &  1.43e-16 &      \\ 
     &           &    4 &  5.00e-09 &  1.43e-16 &      \\ 
     &           &    5 &  4.99e-09 &  1.43e-16 &      \\ 
     &           &    6 &  7.27e-08 &  1.43e-16 &      \\ 
     &           &    7 &  8.28e-08 &  2.07e-15 &      \\ 
     &           &    8 &  7.27e-08 &  2.36e-15 &      \\ 
     &           &    9 &  5.08e-09 &  2.08e-15 &      \\ 
     &           &   10 &  5.07e-09 &  1.45e-16 &      \\ 
2654 &  2.65e+04 &   10 &           &           & iters  \\ 
 \hdashline 
     &           &    1 &  1.09e-08 &  4.29e-10 &      \\ 
     &           &    2 &  1.09e-08 &  3.12e-16 &      \\ 
     &           &    3 &  6.68e-08 &  3.12e-16 &      \\ 
     &           &    4 &  1.10e-08 &  1.91e-15 &      \\ 
     &           &    5 &  1.10e-08 &  3.14e-16 &      \\ 
     &           &    6 &  1.10e-08 &  3.14e-16 &      \\ 
     &           &    7 &  1.10e-08 &  3.13e-16 &      \\ 
     &           &    8 &  1.09e-08 &  3.13e-16 &      \\ 
     &           &    9 &  1.09e-08 &  3.12e-16 &      \\ 
     &           &   10 &  6.68e-08 &  3.12e-16 &      \\ 
2655 &  2.65e+04 &   10 &           &           & iters  \\ 
 \hdashline 
     &           &    1 &  1.14e-08 &  6.72e-10 &      \\ 
     &           &    2 &  1.14e-08 &  3.25e-16 &      \\ 
     &           &    3 &  1.14e-08 &  3.24e-16 &      \\ 
     &           &    4 &  1.13e-08 &  3.24e-16 &      \\ 
     &           &    5 &  6.64e-08 &  3.24e-16 &      \\ 
     &           &    6 &  8.91e-08 &  1.89e-15 &      \\ 
     &           &    7 &  6.64e-08 &  2.54e-15 &      \\ 
     &           &    8 &  1.14e-08 &  1.90e-15 &      \\ 
     &           &    9 &  1.14e-08 &  3.25e-16 &      \\ 
     &           &   10 &  1.13e-08 &  3.24e-16 &      \\ 
2656 &  2.66e+04 &   10 &           &           & iters  \\ 
 \hdashline 
     &           &    1 &  5.49e-09 &  1.05e-09 &      \\ 
     &           &    2 &  5.48e-09 &  1.57e-16 &      \\ 
     &           &    3 &  5.47e-09 &  1.56e-16 &      \\ 
     &           &    4 &  5.47e-09 &  1.56e-16 &      \\ 
     &           &    5 &  5.46e-09 &  1.56e-16 &      \\ 
     &           &    6 &  7.22e-08 &  1.56e-16 &      \\ 
     &           &    7 &  8.32e-08 &  2.06e-15 &      \\ 
     &           &    8 &  7.23e-08 &  2.38e-15 &      \\ 
     &           &    9 &  5.54e-09 &  2.06e-15 &      \\ 
     &           &   10 &  5.53e-09 &  1.58e-16 &      \\ 
2657 &  2.66e+04 &   10 &           &           & iters  \\ 
 \hdashline 
     &           &    1 &  4.46e-08 &  1.23e-10 &      \\ 
     &           &    2 &  3.32e-08 &  1.27e-15 &      \\ 
     &           &    3 &  3.31e-08 &  9.47e-16 &      \\ 
     &           &    4 &  4.46e-08 &  9.46e-16 &      \\ 
     &           &    5 &  3.31e-08 &  1.27e-15 &      \\ 
     &           &    6 &  4.46e-08 &  9.46e-16 &      \\ 
     &           &    7 &  3.32e-08 &  1.27e-15 &      \\ 
     &           &    8 &  4.46e-08 &  9.47e-16 &      \\ 
     &           &    9 &  3.32e-08 &  1.27e-15 &      \\ 
     &           &   10 &  3.31e-08 &  9.47e-16 &      \\ 
2658 &  2.66e+04 &   10 &           &           & iters  \\ 
 \hdashline 
     &           &    1 &  1.02e-08 &  4.92e-10 &      \\ 
     &           &    2 &  1.02e-08 &  2.92e-16 &      \\ 
     &           &    3 &  1.02e-08 &  2.91e-16 &      \\ 
     &           &    4 &  1.02e-08 &  2.91e-16 &      \\ 
     &           &    5 &  1.02e-08 &  2.90e-16 &      \\ 
     &           &    6 &  1.01e-08 &  2.90e-16 &      \\ 
     &           &    7 &  1.01e-08 &  2.90e-16 &      \\ 
     &           &    8 &  1.01e-08 &  2.89e-16 &      \\ 
     &           &    9 &  1.01e-08 &  2.89e-16 &      \\ 
     &           &   10 &  1.01e-08 &  2.88e-16 &      \\ 
2659 &  2.66e+04 &   10 &           &           & iters  \\ 
 \hdashline 
     &           &    1 &  1.18e-07 &  2.13e-10 &      \\ 
     &           &    2 &  3.78e-08 &  3.36e-15 &      \\ 
     &           &    3 &  3.77e-08 &  1.08e-15 &      \\ 
     &           &    4 &  4.00e-08 &  1.08e-15 &      \\ 
     &           &    5 &  3.77e-08 &  1.14e-15 &      \\ 
     &           &    6 &  4.00e-08 &  1.08e-15 &      \\ 
     &           &    7 &  3.77e-08 &  1.14e-15 &      \\ 
     &           &    8 &  4.00e-08 &  1.08e-15 &      \\ 
     &           &    9 &  3.78e-08 &  1.14e-15 &      \\ 
     &           &   10 &  4.00e-08 &  1.08e-15 &      \\ 
2660 &  2.66e+04 &   10 &           &           & iters  \\ 
 \hdashline 
     &           &    1 &  1.65e-08 &  1.42e-10 &      \\ 
     &           &    2 &  1.65e-08 &  4.71e-16 &      \\ 
     &           &    3 &  1.65e-08 &  4.71e-16 &      \\ 
     &           &    4 &  6.12e-08 &  4.70e-16 &      \\ 
     &           &    5 &  1.65e-08 &  1.75e-15 &      \\ 
     &           &    6 &  1.65e-08 &  4.72e-16 &      \\ 
     &           &    7 &  1.65e-08 &  4.71e-16 &      \\ 
     &           &    8 &  1.65e-08 &  4.70e-16 &      \\ 
     &           &    9 &  6.13e-08 &  4.70e-16 &      \\ 
     &           &   10 &  1.65e-08 &  1.75e-15 &      \\ 
2661 &  2.66e+04 &   10 &           &           & iters  \\ 
 \hdashline 
     &           &    1 &  3.43e-08 &  3.74e-10 &      \\ 
     &           &    2 &  4.34e-08 &  9.79e-16 &      \\ 
     &           &    3 &  3.43e-08 &  1.24e-15 &      \\ 
     &           &    4 &  4.34e-08 &  9.80e-16 &      \\ 
     &           &    5 &  3.43e-08 &  1.24e-15 &      \\ 
     &           &    6 &  4.34e-08 &  9.80e-16 &      \\ 
     &           &    7 &  3.44e-08 &  1.24e-15 &      \\ 
     &           &    8 &  3.43e-08 &  9.80e-16 &      \\ 
     &           &    9 &  4.34e-08 &  9.79e-16 &      \\ 
     &           &   10 &  3.43e-08 &  1.24e-15 &      \\ 
2662 &  2.66e+04 &   10 &           &           & iters  \\ 
 \hdashline 
     &           &    1 &  5.59e-08 &  6.01e-10 &      \\ 
     &           &    2 &  5.58e-08 &  1.60e-15 &      \\ 
     &           &    3 &  2.20e-08 &  1.59e-15 &      \\ 
     &           &    4 &  2.19e-08 &  6.27e-16 &      \\ 
     &           &    5 &  2.19e-08 &  6.26e-16 &      \\ 
     &           &    6 &  5.58e-08 &  6.25e-16 &      \\ 
     &           &    7 &  2.19e-08 &  1.59e-15 &      \\ 
     &           &    8 &  2.19e-08 &  6.26e-16 &      \\ 
     &           &    9 &  2.19e-08 &  6.25e-16 &      \\ 
     &           &   10 &  5.58e-08 &  6.24e-16 &      \\ 
2663 &  2.66e+04 &   10 &           &           & iters  \\ 
 \hdashline 
     &           &    1 &  4.49e-08 &  7.66e-10 &      \\ 
     &           &    2 &  3.28e-08 &  1.28e-15 &      \\ 
     &           &    3 &  3.28e-08 &  9.37e-16 &      \\ 
     &           &    4 &  4.49e-08 &  9.36e-16 &      \\ 
     &           &    5 &  3.28e-08 &  1.28e-15 &      \\ 
     &           &    6 &  4.49e-08 &  9.36e-16 &      \\ 
     &           &    7 &  3.28e-08 &  1.28e-15 &      \\ 
     &           &    8 &  4.49e-08 &  9.37e-16 &      \\ 
     &           &    9 &  3.28e-08 &  1.28e-15 &      \\ 
     &           &   10 &  3.28e-08 &  9.37e-16 &      \\ 
2664 &  2.66e+04 &   10 &           &           & iters  \\ 
 \hdashline 
     &           &    1 &  4.76e-08 &  7.70e-10 &      \\ 
     &           &    2 &  4.76e-08 &  1.36e-15 &      \\ 
     &           &    3 &  3.02e-08 &  1.36e-15 &      \\ 
     &           &    4 &  3.02e-08 &  8.62e-16 &      \\ 
     &           &    5 &  4.76e-08 &  8.61e-16 &      \\ 
     &           &    6 &  3.02e-08 &  1.36e-15 &      \\ 
     &           &    7 &  4.75e-08 &  8.62e-16 &      \\ 
     &           &    8 &  3.02e-08 &  1.36e-15 &      \\ 
     &           &    9 &  3.02e-08 &  8.62e-16 &      \\ 
     &           &   10 &  4.76e-08 &  8.61e-16 &      \\ 
2665 &  2.66e+04 &   10 &           &           & iters  \\ 
 \hdashline 
     &           &    1 &  7.48e-09 &  7.30e-10 &      \\ 
     &           &    2 &  7.47e-09 &  2.13e-16 &      \\ 
     &           &    3 &  7.46e-09 &  2.13e-16 &      \\ 
     &           &    4 &  7.45e-09 &  2.13e-16 &      \\ 
     &           &    5 &  7.03e-08 &  2.13e-16 &      \\ 
     &           &    6 &  7.54e-09 &  2.01e-15 &      \\ 
     &           &    7 &  7.53e-09 &  2.15e-16 &      \\ 
     &           &    8 &  7.52e-09 &  2.15e-16 &      \\ 
     &           &    9 &  7.50e-09 &  2.14e-16 &      \\ 
     &           &   10 &  7.49e-09 &  2.14e-16 &      \\ 
2666 &  2.66e+04 &   10 &           &           & iters  \\ 
 \hdashline 
     &           &    1 &  2.85e-08 &  2.83e-10 &      \\ 
     &           &    2 &  2.84e-08 &  8.12e-16 &      \\ 
     &           &    3 &  4.93e-08 &  8.11e-16 &      \\ 
     &           &    4 &  2.85e-08 &  1.41e-15 &      \\ 
     &           &    5 &  2.84e-08 &  8.12e-16 &      \\ 
     &           &    6 &  4.93e-08 &  8.11e-16 &      \\ 
     &           &    7 &  2.84e-08 &  1.41e-15 &      \\ 
     &           &    8 &  2.84e-08 &  8.12e-16 &      \\ 
     &           &    9 &  4.93e-08 &  8.11e-16 &      \\ 
     &           &   10 &  2.84e-08 &  1.41e-15 &      \\ 
2667 &  2.67e+04 &   10 &           &           & iters  \\ 
 \hdashline 
     &           &    1 &  3.26e-08 &  5.56e-10 &      \\ 
     &           &    2 &  3.26e-08 &  9.31e-16 &      \\ 
     &           &    3 &  4.52e-08 &  9.30e-16 &      \\ 
     &           &    4 &  3.26e-08 &  1.29e-15 &      \\ 
     &           &    5 &  4.51e-08 &  9.30e-16 &      \\ 
     &           &    6 &  3.26e-08 &  1.29e-15 &      \\ 
     &           &    7 &  4.51e-08 &  9.31e-16 &      \\ 
     &           &    8 &  3.26e-08 &  1.29e-15 &      \\ 
     &           &    9 &  3.26e-08 &  9.31e-16 &      \\ 
     &           &   10 &  4.52e-08 &  9.30e-16 &      \\ 
2668 &  2.67e+04 &   10 &           &           & iters  \\ 
 \hdashline 
     &           &    1 &  3.10e-08 &  9.93e-10 &      \\ 
     &           &    2 &  4.68e-08 &  8.84e-16 &      \\ 
     &           &    3 &  3.10e-08 &  1.33e-15 &      \\ 
     &           &    4 &  4.67e-08 &  8.84e-16 &      \\ 
     &           &    5 &  3.10e-08 &  1.33e-15 &      \\ 
     &           &    6 &  3.10e-08 &  8.85e-16 &      \\ 
     &           &    7 &  4.68e-08 &  8.84e-16 &      \\ 
     &           &    8 &  3.10e-08 &  1.33e-15 &      \\ 
     &           &    9 &  4.67e-08 &  8.84e-16 &      \\ 
     &           &   10 &  3.10e-08 &  1.33e-15 &      \\ 
2669 &  2.67e+04 &   10 &           &           & iters  \\ 
 \hdashline 
     &           &    1 &  5.59e-08 &  8.48e-10 &      \\ 
     &           &    2 &  2.19e-08 &  1.60e-15 &      \\ 
     &           &    3 &  2.18e-08 &  6.24e-16 &      \\ 
     &           &    4 &  2.18e-08 &  6.23e-16 &      \\ 
     &           &    5 &  5.59e-08 &  6.22e-16 &      \\ 
     &           &    6 &  2.18e-08 &  1.60e-15 &      \\ 
     &           &    7 &  2.18e-08 &  6.23e-16 &      \\ 
     &           &    8 &  2.18e-08 &  6.23e-16 &      \\ 
     &           &    9 &  5.59e-08 &  6.22e-16 &      \\ 
     &           &   10 &  2.18e-08 &  1.60e-15 &      \\ 
2670 &  2.67e+04 &   10 &           &           & iters  \\ 
 \hdashline 
     &           &    1 &  1.74e-08 &  6.04e-10 &      \\ 
     &           &    2 &  1.73e-08 &  4.96e-16 &      \\ 
     &           &    3 &  1.73e-08 &  4.95e-16 &      \\ 
     &           &    4 &  6.04e-08 &  4.94e-16 &      \\ 
     &           &    5 &  1.74e-08 &  1.72e-15 &      \\ 
     &           &    6 &  1.74e-08 &  4.96e-16 &      \\ 
     &           &    7 &  1.73e-08 &  4.95e-16 &      \\ 
     &           &    8 &  6.04e-08 &  4.95e-16 &      \\ 
     &           &    9 &  1.74e-08 &  1.72e-15 &      \\ 
     &           &   10 &  1.74e-08 &  4.96e-16 &      \\ 
2671 &  2.67e+04 &   10 &           &           & iters  \\ 
 \hdashline 
     &           &    1 &  2.02e-08 &  7.32e-10 &      \\ 
     &           &    2 &  5.75e-08 &  5.78e-16 &      \\ 
     &           &    3 &  2.03e-08 &  1.64e-15 &      \\ 
     &           &    4 &  2.03e-08 &  5.79e-16 &      \\ 
     &           &    5 &  2.02e-08 &  5.78e-16 &      \\ 
     &           &    6 &  5.75e-08 &  5.77e-16 &      \\ 
     &           &    7 &  2.03e-08 &  1.64e-15 &      \\ 
     &           &    8 &  2.03e-08 &  5.79e-16 &      \\ 
     &           &    9 &  5.75e-08 &  5.78e-16 &      \\ 
     &           &   10 &  2.03e-08 &  1.64e-15 &      \\ 
2672 &  2.67e+04 &   10 &           &           & iters  \\ 
 \hdashline 
     &           &    1 &  2.82e-08 &  6.25e-10 &      \\ 
     &           &    2 &  4.95e-08 &  8.06e-16 &      \\ 
     &           &    3 &  2.83e-08 &  1.41e-15 &      \\ 
     &           &    4 &  2.82e-08 &  8.07e-16 &      \\ 
     &           &    5 &  4.95e-08 &  8.06e-16 &      \\ 
     &           &    6 &  2.83e-08 &  1.41e-15 &      \\ 
     &           &    7 &  4.95e-08 &  8.07e-16 &      \\ 
     &           &    8 &  2.83e-08 &  1.41e-15 &      \\ 
     &           &    9 &  2.83e-08 &  8.08e-16 &      \\ 
     &           &   10 &  4.95e-08 &  8.07e-16 &      \\ 
2673 &  2.67e+04 &   10 &           &           & iters  \\ 
 \hdashline 
     &           &    1 &  1.32e-09 &  1.56e-10 &      \\ 
     &           &    2 &  1.32e-09 &  3.77e-17 &      \\ 
     &           &    3 &  1.32e-09 &  3.76e-17 &      \\ 
     &           &    4 &  1.32e-09 &  3.76e-17 &      \\ 
     &           &    5 &  1.31e-09 &  3.75e-17 &      \\ 
     &           &    6 &  1.31e-09 &  3.75e-17 &      \\ 
     &           &    7 &  1.31e-09 &  3.74e-17 &      \\ 
     &           &    8 &  1.31e-09 &  3.74e-17 &      \\ 
     &           &    9 &  1.31e-09 &  3.73e-17 &      \\ 
     &           &   10 &  1.30e-09 &  3.73e-17 &      \\ 
2674 &  2.67e+04 &   10 &           &           & iters  \\ 
 \hdashline 
     &           &    1 &  3.27e-08 &  3.19e-11 &      \\ 
     &           &    2 &  6.20e-09 &  9.33e-16 &      \\ 
     &           &    3 &  6.19e-09 &  1.77e-16 &      \\ 
     &           &    4 &  6.18e-09 &  1.77e-16 &      \\ 
     &           &    5 &  6.17e-09 &  1.76e-16 &      \\ 
     &           &    6 &  6.16e-09 &  1.76e-16 &      \\ 
     &           &    7 &  3.27e-08 &  1.76e-16 &      \\ 
     &           &    8 &  6.20e-09 &  9.33e-16 &      \\ 
     &           &    9 &  6.19e-09 &  1.77e-16 &      \\ 
     &           &   10 &  6.18e-09 &  1.77e-16 &      \\ 
2675 &  2.67e+04 &   10 &           &           & iters  \\ 
 \hdashline 
     &           &    1 &  2.31e-08 &  8.78e-10 &      \\ 
     &           &    2 &  2.31e-08 &  6.59e-16 &      \\ 
     &           &    3 &  5.47e-08 &  6.58e-16 &      \\ 
     &           &    4 &  2.31e-08 &  1.56e-15 &      \\ 
     &           &    5 &  2.31e-08 &  6.60e-16 &      \\ 
     &           &    6 &  5.46e-08 &  6.59e-16 &      \\ 
     &           &    7 &  2.31e-08 &  1.56e-15 &      \\ 
     &           &    8 &  2.31e-08 &  6.60e-16 &      \\ 
     &           &    9 &  2.31e-08 &  6.59e-16 &      \\ 
     &           &   10 &  5.47e-08 &  6.58e-16 &      \\ 
2676 &  2.68e+04 &   10 &           &           & iters  \\ 
 \hdashline 
     &           &    1 &  2.93e-08 &  2.92e-10 &      \\ 
     &           &    2 &  4.84e-08 &  8.36e-16 &      \\ 
     &           &    3 &  2.93e-08 &  1.38e-15 &      \\ 
     &           &    4 &  4.84e-08 &  8.37e-16 &      \\ 
     &           &    5 &  2.94e-08 &  1.38e-15 &      \\ 
     &           &    6 &  2.93e-08 &  8.38e-16 &      \\ 
     &           &    7 &  4.84e-08 &  8.36e-16 &      \\ 
     &           &    8 &  2.93e-08 &  1.38e-15 &      \\ 
     &           &    9 &  2.93e-08 &  8.37e-16 &      \\ 
     &           &   10 &  4.84e-08 &  8.36e-16 &      \\ 
2677 &  2.68e+04 &   10 &           &           & iters  \\ 
 \hdashline 
     &           &    1 &  1.51e-08 &  4.16e-10 &      \\ 
     &           &    2 &  1.51e-08 &  4.31e-16 &      \\ 
     &           &    3 &  1.50e-08 &  4.30e-16 &      \\ 
     &           &    4 &  6.27e-08 &  4.29e-16 &      \\ 
     &           &    5 &  1.51e-08 &  1.79e-15 &      \\ 
     &           &    6 &  1.51e-08 &  4.31e-16 &      \\ 
     &           &    7 &  1.51e-08 &  4.31e-16 &      \\ 
     &           &    8 &  1.50e-08 &  4.30e-16 &      \\ 
     &           &    9 &  6.27e-08 &  4.29e-16 &      \\ 
     &           &   10 &  1.51e-08 &  1.79e-15 &      \\ 
2678 &  2.68e+04 &   10 &           &           & iters  \\ 
 \hdashline 
     &           &    1 &  2.99e-08 &  1.79e-10 &      \\ 
     &           &    2 &  4.78e-08 &  8.53e-16 &      \\ 
     &           &    3 &  2.99e-08 &  1.37e-15 &      \\ 
     &           &    4 &  2.99e-08 &  8.54e-16 &      \\ 
     &           &    5 &  4.79e-08 &  8.53e-16 &      \\ 
     &           &    6 &  2.99e-08 &  1.37e-15 &      \\ 
     &           &    7 &  4.78e-08 &  8.53e-16 &      \\ 
     &           &    8 &  2.99e-08 &  1.37e-15 &      \\ 
     &           &    9 &  2.99e-08 &  8.54e-16 &      \\ 
     &           &   10 &  4.78e-08 &  8.53e-16 &      \\ 
2679 &  2.68e+04 &   10 &           &           & iters  \\ 
 \hdashline 
     &           &    1 &  1.43e-08 &  8.91e-11 &      \\ 
     &           &    2 &  6.34e-08 &  4.09e-16 &      \\ 
     &           &    3 &  1.44e-08 &  1.81e-15 &      \\ 
     &           &    4 &  1.44e-08 &  4.11e-16 &      \\ 
     &           &    5 &  1.44e-08 &  4.11e-16 &      \\ 
     &           &    6 &  1.43e-08 &  4.10e-16 &      \\ 
     &           &    7 &  6.34e-08 &  4.09e-16 &      \\ 
     &           &    8 &  1.44e-08 &  1.81e-15 &      \\ 
     &           &    9 &  1.44e-08 &  4.11e-16 &      \\ 
     &           &   10 &  1.44e-08 &  4.11e-16 &      \\ 
2680 &  2.68e+04 &   10 &           &           & iters  \\ 
 \hdashline 
     &           &    1 &  7.14e-09 &  2.48e-10 &      \\ 
     &           &    2 &  7.13e-09 &  2.04e-16 &      \\ 
     &           &    3 &  7.12e-09 &  2.04e-16 &      \\ 
     &           &    4 &  7.11e-09 &  2.03e-16 &      \\ 
     &           &    5 &  3.17e-08 &  2.03e-16 &      \\ 
     &           &    6 &  7.15e-09 &  9.06e-16 &      \\ 
     &           &    7 &  7.14e-09 &  2.04e-16 &      \\ 
     &           &    8 &  7.13e-09 &  2.04e-16 &      \\ 
     &           &    9 &  7.12e-09 &  2.03e-16 &      \\ 
     &           &   10 &  7.11e-09 &  2.03e-16 &      \\ 
2681 &  2.68e+04 &   10 &           &           & iters  \\ 
 \hdashline 
     &           &    1 &  4.47e-08 &  3.03e-10 &      \\ 
     &           &    2 &  3.31e-08 &  1.28e-15 &      \\ 
     &           &    3 &  4.47e-08 &  9.43e-16 &      \\ 
     &           &    4 &  3.31e-08 &  1.28e-15 &      \\ 
     &           &    5 &  4.47e-08 &  9.44e-16 &      \\ 
     &           &    6 &  3.31e-08 &  1.27e-15 &      \\ 
     &           &    7 &  4.47e-08 &  9.44e-16 &      \\ 
     &           &    8 &  3.31e-08 &  1.27e-15 &      \\ 
     &           &    9 &  3.31e-08 &  9.45e-16 &      \\ 
     &           &   10 &  4.47e-08 &  9.43e-16 &      \\ 
2682 &  2.68e+04 &   10 &           &           & iters  \\ 
 \hdashline 
     &           &    1 &  4.56e-09 &  1.06e-10 &      \\ 
     &           &    2 &  7.31e-08 &  1.30e-16 &      \\ 
     &           &    3 &  8.23e-08 &  2.09e-15 &      \\ 
     &           &    4 &  7.32e-08 &  2.35e-15 &      \\ 
     &           &    5 &  8.23e-08 &  2.09e-15 &      \\ 
     &           &    6 &  7.32e-08 &  2.35e-15 &      \\ 
     &           &    7 &  4.63e-09 &  2.09e-15 &      \\ 
     &           &    8 &  4.62e-09 &  1.32e-16 &      \\ 
     &           &    9 &  4.61e-09 &  1.32e-16 &      \\ 
     &           &   10 &  4.61e-09 &  1.32e-16 &      \\ 
2683 &  2.68e+04 &   10 &           &           & iters  \\ 
 \hdashline 
     &           &    1 &  5.82e-09 &  8.52e-11 &      \\ 
     &           &    2 &  5.81e-09 &  1.66e-16 &      \\ 
     &           &    3 &  5.80e-09 &  1.66e-16 &      \\ 
     &           &    4 &  7.19e-08 &  1.66e-16 &      \\ 
     &           &    5 &  8.36e-08 &  2.05e-15 &      \\ 
     &           &    6 &  7.19e-08 &  2.39e-15 &      \\ 
     &           &    7 &  5.88e-09 &  2.05e-15 &      \\ 
     &           &    8 &  5.87e-09 &  1.68e-16 &      \\ 
     &           &    9 &  5.86e-09 &  1.67e-16 &      \\ 
     &           &   10 &  5.85e-09 &  1.67e-16 &      \\ 
2684 &  2.68e+04 &   10 &           &           & iters  \\ 
 \hdashline 
     &           &    1 &  2.77e-09 &  2.16e-10 &      \\ 
     &           &    2 &  2.77e-09 &  7.91e-17 &      \\ 
     &           &    3 &  2.76e-09 &  7.90e-17 &      \\ 
     &           &    4 &  2.76e-09 &  7.89e-17 &      \\ 
     &           &    5 &  2.76e-09 &  7.88e-17 &      \\ 
     &           &    6 &  2.75e-09 &  7.87e-17 &      \\ 
     &           &    7 &  2.75e-09 &  7.85e-17 &      \\ 
     &           &    8 &  2.74e-09 &  7.84e-17 &      \\ 
     &           &    9 &  2.74e-09 &  7.83e-17 &      \\ 
     &           &   10 &  2.74e-09 &  7.82e-17 &      \\ 
2685 &  2.68e+04 &   10 &           &           & iters  \\ 
 \hdashline 
     &           &    1 &  3.82e-08 &  5.97e-10 &      \\ 
     &           &    2 &  3.96e-08 &  1.09e-15 &      \\ 
     &           &    3 &  3.82e-08 &  1.13e-15 &      \\ 
     &           &    4 &  3.96e-08 &  1.09e-15 &      \\ 
     &           &    5 &  3.82e-08 &  1.13e-15 &      \\ 
     &           &    6 &  3.96e-08 &  1.09e-15 &      \\ 
     &           &    7 &  3.82e-08 &  1.13e-15 &      \\ 
     &           &    8 &  3.96e-08 &  1.09e-15 &      \\ 
     &           &    9 &  3.82e-08 &  1.13e-15 &      \\ 
     &           &   10 &  3.96e-08 &  1.09e-15 &      \\ 
2686 &  2.68e+04 &   10 &           &           & iters  \\ 
 \hdashline 
     &           &    1 &  2.89e-09 &  5.52e-10 &      \\ 
     &           &    2 &  2.89e-09 &  8.25e-17 &      \\ 
     &           &    3 &  2.88e-09 &  8.24e-17 &      \\ 
     &           &    4 &  2.88e-09 &  8.23e-17 &      \\ 
     &           &    5 &  2.88e-09 &  8.22e-17 &      \\ 
     &           &    6 &  2.87e-09 &  8.21e-17 &      \\ 
     &           &    7 &  2.87e-09 &  8.19e-17 &      \\ 
     &           &    8 &  2.86e-09 &  8.18e-17 &      \\ 
     &           &    9 &  2.86e-09 &  8.17e-17 &      \\ 
     &           &   10 &  2.85e-09 &  8.16e-17 &      \\ 
2687 &  2.69e+04 &   10 &           &           & iters  \\ 
 \hdashline 
     &           &    1 &  5.42e-08 &  5.25e-10 &      \\ 
     &           &    2 &  2.36e-08 &  1.55e-15 &      \\ 
     &           &    3 &  2.36e-08 &  6.73e-16 &      \\ 
     &           &    4 &  5.42e-08 &  6.72e-16 &      \\ 
     &           &    5 &  2.36e-08 &  1.55e-15 &      \\ 
     &           &    6 &  2.36e-08 &  6.74e-16 &      \\ 
     &           &    7 &  5.42e-08 &  6.73e-16 &      \\ 
     &           &    8 &  2.36e-08 &  1.55e-15 &      \\ 
     &           &    9 &  2.36e-08 &  6.74e-16 &      \\ 
     &           &   10 &  2.35e-08 &  6.73e-16 &      \\ 
2688 &  2.69e+04 &   10 &           &           & iters  \\ 
 \hdashline 
     &           &    1 &  2.77e-08 &  6.66e-10 &      \\ 
     &           &    2 &  5.00e-08 &  7.91e-16 &      \\ 
     &           &    3 &  2.77e-08 &  1.43e-15 &      \\ 
     &           &    4 &  2.77e-08 &  7.92e-16 &      \\ 
     &           &    5 &  5.00e-08 &  7.91e-16 &      \\ 
     &           &    6 &  2.77e-08 &  1.43e-15 &      \\ 
     &           &    7 &  2.77e-08 &  7.92e-16 &      \\ 
     &           &    8 &  5.00e-08 &  7.91e-16 &      \\ 
     &           &    9 &  2.77e-08 &  1.43e-15 &      \\ 
     &           &   10 &  5.00e-08 &  7.92e-16 &      \\ 
2689 &  2.69e+04 &   10 &           &           & iters  \\ 
 \hdashline 
     &           &    1 &  7.05e-08 &  3.41e-10 &      \\ 
     &           &    2 &  7.33e-09 &  2.01e-15 &      \\ 
     &           &    3 &  7.32e-09 &  2.09e-16 &      \\ 
     &           &    4 &  7.31e-09 &  2.09e-16 &      \\ 
     &           &    5 &  7.30e-09 &  2.09e-16 &      \\ 
     &           &    6 &  7.29e-09 &  2.08e-16 &      \\ 
     &           &    7 &  7.28e-09 &  2.08e-16 &      \\ 
     &           &    8 &  7.27e-09 &  2.08e-16 &      \\ 
     &           &    9 &  7.26e-09 &  2.07e-16 &      \\ 
     &           &   10 &  7.25e-09 &  2.07e-16 &      \\ 
2690 &  2.69e+04 &   10 &           &           & iters  \\ 
 \hdashline 
     &           &    1 &  5.66e-08 &  1.77e-10 &      \\ 
     &           &    2 &  2.11e-08 &  1.62e-15 &      \\ 
     &           &    3 &  2.11e-08 &  6.04e-16 &      \\ 
     &           &    4 &  5.66e-08 &  6.03e-16 &      \\ 
     &           &    5 &  2.12e-08 &  1.62e-15 &      \\ 
     &           &    6 &  2.11e-08 &  6.04e-16 &      \\ 
     &           &    7 &  5.66e-08 &  6.03e-16 &      \\ 
     &           &    8 &  2.12e-08 &  1.61e-15 &      \\ 
     &           &    9 &  2.12e-08 &  6.05e-16 &      \\ 
     &           &   10 &  2.11e-08 &  6.04e-16 &      \\ 
2691 &  2.69e+04 &   10 &           &           & iters  \\ 
 \hdashline 
     &           &    1 &  4.31e-08 &  1.19e-10 &      \\ 
     &           &    2 &  3.46e-08 &  1.23e-15 &      \\ 
     &           &    3 &  4.31e-08 &  9.89e-16 &      \\ 
     &           &    4 &  3.46e-08 &  1.23e-15 &      \\ 
     &           &    5 &  4.31e-08 &  9.89e-16 &      \\ 
     &           &    6 &  3.47e-08 &  1.23e-15 &      \\ 
     &           &    7 &  3.46e-08 &  9.89e-16 &      \\ 
     &           &    8 &  4.31e-08 &  9.88e-16 &      \\ 
     &           &    9 &  3.46e-08 &  1.23e-15 &      \\ 
     &           &   10 &  4.31e-08 &  9.88e-16 &      \\ 
2692 &  2.69e+04 &   10 &           &           & iters  \\ 
 \hdashline 
     &           &    1 &  1.27e-08 &  1.02e-10 &      \\ 
     &           &    2 &  6.50e-08 &  3.61e-16 &      \\ 
     &           &    3 &  1.27e-08 &  1.86e-15 &      \\ 
     &           &    4 &  1.27e-08 &  3.63e-16 &      \\ 
     &           &    5 &  1.27e-08 &  3.63e-16 &      \\ 
     &           &    6 &  1.27e-08 &  3.62e-16 &      \\ 
     &           &    7 &  1.27e-08 &  3.62e-16 &      \\ 
     &           &    8 &  6.50e-08 &  3.61e-16 &      \\ 
     &           &    9 &  9.04e-08 &  1.86e-15 &      \\ 
     &           &   10 &  6.51e-08 &  2.58e-15 &      \\ 
2693 &  2.69e+04 &   10 &           &           & iters  \\ 
 \hdashline 
     &           &    1 &  2.62e-08 &  1.35e-10 &      \\ 
     &           &    2 &  5.15e-08 &  7.49e-16 &      \\ 
     &           &    3 &  2.63e-08 &  1.47e-15 &      \\ 
     &           &    4 &  2.62e-08 &  7.50e-16 &      \\ 
     &           &    5 &  5.15e-08 &  7.49e-16 &      \\ 
     &           &    6 &  2.63e-08 &  1.47e-15 &      \\ 
     &           &    7 &  2.62e-08 &  7.50e-16 &      \\ 
     &           &    8 &  5.15e-08 &  7.49e-16 &      \\ 
     &           &    9 &  2.63e-08 &  1.47e-15 &      \\ 
     &           &   10 &  2.62e-08 &  7.50e-16 &      \\ 
2694 &  2.69e+04 &   10 &           &           & iters  \\ 
 \hdashline 
     &           &    1 &  2.63e-08 &  5.10e-11 &      \\ 
     &           &    2 &  2.62e-08 &  7.50e-16 &      \\ 
     &           &    3 &  5.15e-08 &  7.49e-16 &      \\ 
     &           &    4 &  2.63e-08 &  1.47e-15 &      \\ 
     &           &    5 &  2.62e-08 &  7.50e-16 &      \\ 
     &           &    6 &  5.15e-08 &  7.49e-16 &      \\ 
     &           &    7 &  2.63e-08 &  1.47e-15 &      \\ 
     &           &    8 &  2.62e-08 &  7.50e-16 &      \\ 
     &           &    9 &  5.15e-08 &  7.49e-16 &      \\ 
     &           &   10 &  2.63e-08 &  1.47e-15 &      \\ 
2695 &  2.69e+04 &   10 &           &           & iters  \\ 
 \hdashline 
     &           &    1 &  2.48e-09 &  5.12e-10 &      \\ 
     &           &    2 &  2.47e-09 &  7.07e-17 &      \\ 
     &           &    3 &  2.47e-09 &  7.06e-17 &      \\ 
     &           &    4 &  2.47e-09 &  7.05e-17 &      \\ 
     &           &    5 &  2.46e-09 &  7.04e-17 &      \\ 
     &           &    6 &  2.46e-09 &  7.03e-17 &      \\ 
     &           &    7 &  2.46e-09 &  7.02e-17 &      \\ 
     &           &    8 &  2.45e-09 &  7.01e-17 &      \\ 
     &           &    9 &  2.45e-09 &  7.00e-17 &      \\ 
     &           &   10 &  2.45e-09 &  6.99e-17 &      \\ 
2696 &  2.70e+04 &   10 &           &           & iters  \\ 
 \hdashline 
     &           &    1 &  5.31e-08 &  9.68e-10 &      \\ 
     &           &    2 &  2.47e-08 &  1.52e-15 &      \\ 
     &           &    3 &  2.46e-08 &  7.04e-16 &      \\ 
     &           &    4 &  5.31e-08 &  7.03e-16 &      \\ 
     &           &    5 &  2.47e-08 &  1.52e-15 &      \\ 
     &           &    6 &  2.46e-08 &  7.04e-16 &      \\ 
     &           &    7 &  2.46e-08 &  7.03e-16 &      \\ 
     &           &    8 &  5.31e-08 &  7.02e-16 &      \\ 
     &           &    9 &  2.46e-08 &  1.52e-15 &      \\ 
     &           &   10 &  2.46e-08 &  7.03e-16 &      \\ 
2697 &  2.70e+04 &   10 &           &           & iters  \\ 
 \hdashline 
     &           &    1 &  1.16e-07 &  7.32e-10 &      \\ 
     &           &    2 &  3.97e-08 &  3.31e-15 &      \\ 
     &           &    3 &  3.96e-08 &  1.13e-15 &      \\ 
     &           &    4 &  3.81e-08 &  1.13e-15 &      \\ 
     &           &    5 &  3.96e-08 &  1.09e-15 &      \\ 
     &           &    6 &  3.81e-08 &  1.13e-15 &      \\ 
     &           &    7 &  3.96e-08 &  1.09e-15 &      \\ 
     &           &    8 &  3.81e-08 &  1.13e-15 &      \\ 
     &           &    9 &  3.96e-08 &  1.09e-15 &      \\ 
     &           &   10 &  3.81e-08 &  1.13e-15 &      \\ 
2698 &  2.70e+04 &   10 &           &           & iters  \\ 
 \hdashline 
     &           &    1 &  2.83e-08 &  2.43e-10 &      \\ 
     &           &    2 &  2.83e-08 &  8.08e-16 &      \\ 
     &           &    3 &  4.94e-08 &  8.07e-16 &      \\ 
     &           &    4 &  2.83e-08 &  1.41e-15 &      \\ 
     &           &    5 &  4.94e-08 &  8.08e-16 &      \\ 
     &           &    6 &  2.83e-08 &  1.41e-15 &      \\ 
     &           &    7 &  2.83e-08 &  8.09e-16 &      \\ 
     &           &    8 &  4.94e-08 &  8.08e-16 &      \\ 
     &           &    9 &  2.83e-08 &  1.41e-15 &      \\ 
     &           &   10 &  2.83e-08 &  8.09e-16 &      \\ 
2699 &  2.70e+04 &   10 &           &           & iters  \\ 
 \hdashline 
     &           &    1 &  2.66e-08 &  2.11e-10 &      \\ 
     &           &    2 &  2.66e-08 &  7.60e-16 &      \\ 
     &           &    3 &  2.66e-08 &  7.59e-16 &      \\ 
     &           &    4 &  5.12e-08 &  7.58e-16 &      \\ 
     &           &    5 &  2.66e-08 &  1.46e-15 &      \\ 
     &           &    6 &  2.66e-08 &  7.59e-16 &      \\ 
     &           &    7 &  5.12e-08 &  7.58e-16 &      \\ 
     &           &    8 &  2.66e-08 &  1.46e-15 &      \\ 
     &           &    9 &  2.66e-08 &  7.59e-16 &      \\ 
     &           &   10 &  5.12e-08 &  7.58e-16 &      \\ 
2700 &  2.70e+04 &   10 &           &           & iters  \\ 
 \hdashline 
     &           &    1 &  4.42e-08 &  7.45e-11 &      \\ 
     &           &    2 &  3.35e-08 &  1.26e-15 &      \\ 
     &           &    3 &  3.35e-08 &  9.57e-16 &      \\ 
     &           &    4 &  4.43e-08 &  9.55e-16 &      \\ 
     &           &    5 &  3.35e-08 &  1.26e-15 &      \\ 
     &           &    6 &  4.43e-08 &  9.56e-16 &      \\ 
     &           &    7 &  3.35e-08 &  1.26e-15 &      \\ 
     &           &    8 &  4.42e-08 &  9.56e-16 &      \\ 
     &           &    9 &  3.35e-08 &  1.26e-15 &      \\ 
     &           &   10 &  3.35e-08 &  9.56e-16 &      \\ 
2701 &  2.70e+04 &   10 &           &           & iters  \\ 
 \hdashline 
     &           &    1 &  2.04e-08 &  4.11e-10 &      \\ 
     &           &    2 &  2.04e-08 &  5.82e-16 &      \\ 
     &           &    3 &  5.74e-08 &  5.81e-16 &      \\ 
     &           &    4 &  2.04e-08 &  1.64e-15 &      \\ 
     &           &    5 &  2.04e-08 &  5.82e-16 &      \\ 
     &           &    6 &  5.73e-08 &  5.82e-16 &      \\ 
     &           &    7 &  2.04e-08 &  1.64e-15 &      \\ 
     &           &    8 &  2.04e-08 &  5.83e-16 &      \\ 
     &           &    9 &  2.04e-08 &  5.82e-16 &      \\ 
     &           &   10 &  5.73e-08 &  5.81e-16 &      \\ 
2702 &  2.70e+04 &   10 &           &           & iters  \\ 
 \hdashline 
     &           &    1 &  8.21e-08 &  1.02e-10 &      \\ 
     &           &    2 &  4.28e-09 &  2.34e-15 &      \\ 
     &           &    3 &  7.34e-08 &  1.22e-16 &      \\ 
     &           &    4 &  8.21e-08 &  2.10e-15 &      \\ 
     &           &    5 &  7.34e-08 &  2.34e-15 &      \\ 
     &           &    6 &  8.21e-08 &  2.10e-15 &      \\ 
     &           &    7 &  7.34e-08 &  2.34e-15 &      \\ 
     &           &    8 &  4.35e-09 &  2.10e-15 &      \\ 
     &           &    9 &  4.34e-09 &  1.24e-16 &      \\ 
     &           &   10 &  4.34e-09 &  1.24e-16 &      \\ 
2703 &  2.70e+04 &   10 &           &           & iters  \\ 
 \hdashline 
     &           &    1 &  1.75e-08 &  4.58e-10 &      \\ 
     &           &    2 &  6.03e-08 &  4.98e-16 &      \\ 
     &           &    3 &  1.75e-08 &  1.72e-15 &      \\ 
     &           &    4 &  1.75e-08 &  5.00e-16 &      \\ 
     &           &    5 &  1.75e-08 &  4.99e-16 &      \\ 
     &           &    6 &  1.74e-08 &  4.99e-16 &      \\ 
     &           &    7 &  6.03e-08 &  4.98e-16 &      \\ 
     &           &    8 &  1.75e-08 &  1.72e-15 &      \\ 
     &           &    9 &  1.75e-08 &  5.00e-16 &      \\ 
     &           &   10 &  1.75e-08 &  4.99e-16 &      \\ 
2704 &  2.70e+04 &   10 &           &           & iters  \\ 
 \hdashline 
     &           &    1 &  9.02e-08 &  2.94e-10 &      \\ 
     &           &    2 &  6.54e-08 &  2.57e-15 &      \\ 
     &           &    3 &  1.24e-08 &  1.87e-15 &      \\ 
     &           &    4 &  1.24e-08 &  3.55e-16 &      \\ 
     &           &    5 &  1.24e-08 &  3.54e-16 &      \\ 
     &           &    6 &  1.24e-08 &  3.54e-16 &      \\ 
     &           &    7 &  6.53e-08 &  3.53e-16 &      \\ 
     &           &    8 &  1.24e-08 &  1.86e-15 &      \\ 
     &           &    9 &  1.24e-08 &  3.55e-16 &      \\ 
     &           &   10 &  1.24e-08 &  3.55e-16 &      \\ 
2705 &  2.70e+04 &   10 &           &           & iters  \\ 
 \hdashline 
     &           &    1 &  2.02e-09 &  1.36e-10 &      \\ 
     &           &    2 &  2.02e-09 &  5.78e-17 &      \\ 
     &           &    3 &  2.02e-09 &  5.77e-17 &      \\ 
     &           &    4 &  2.02e-09 &  5.76e-17 &      \\ 
     &           &    5 &  2.01e-09 &  5.75e-17 &      \\ 
     &           &    6 &  2.01e-09 &  5.74e-17 &      \\ 
     &           &    7 &  2.01e-09 &  5.74e-17 &      \\ 
     &           &    8 &  2.00e-09 &  5.73e-17 &      \\ 
     &           &    9 &  2.00e-09 &  5.72e-17 &      \\ 
     &           &   10 &  2.00e-09 &  5.71e-17 &      \\ 
2706 &  2.70e+04 &   10 &           &           & iters  \\ 
 \hdashline 
     &           &    1 &  2.58e-08 &  3.23e-11 &      \\ 
     &           &    2 &  5.19e-08 &  7.37e-16 &      \\ 
     &           &    3 &  2.59e-08 &  1.48e-15 &      \\ 
     &           &    4 &  2.58e-08 &  7.38e-16 &      \\ 
     &           &    5 &  5.19e-08 &  7.37e-16 &      \\ 
     &           &    6 &  2.59e-08 &  1.48e-15 &      \\ 
     &           &    7 &  2.58e-08 &  7.38e-16 &      \\ 
     &           &    8 &  5.19e-08 &  7.37e-16 &      \\ 
     &           &    9 &  2.59e-08 &  1.48e-15 &      \\ 
     &           &   10 &  2.58e-08 &  7.38e-16 &      \\ 
2707 &  2.71e+04 &   10 &           &           & iters  \\ 
 \hdashline 
     &           &    1 &  6.31e-08 &  4.97e-10 &      \\ 
     &           &    2 &  1.47e-08 &  1.80e-15 &      \\ 
     &           &    3 &  1.47e-08 &  4.20e-16 &      \\ 
     &           &    4 &  1.47e-08 &  4.19e-16 &      \\ 
     &           &    5 &  1.46e-08 &  4.19e-16 &      \\ 
     &           &    6 &  6.31e-08 &  4.18e-16 &      \\ 
     &           &    7 &  1.47e-08 &  1.80e-15 &      \\ 
     &           &    8 &  1.47e-08 &  4.20e-16 &      \\ 
     &           &    9 &  1.47e-08 &  4.19e-16 &      \\ 
     &           &   10 &  1.47e-08 &  4.19e-16 &      \\ 
2708 &  2.71e+04 &   10 &           &           & iters  \\ 
 \hdashline 
     &           &    1 &  2.72e-09 &  4.89e-10 &      \\ 
     &           &    2 &  2.72e-09 &  7.77e-17 &      \\ 
     &           &    3 &  2.72e-09 &  7.76e-17 &      \\ 
     &           &    4 &  2.71e-09 &  7.75e-17 &      \\ 
     &           &    5 &  2.71e-09 &  7.74e-17 &      \\ 
     &           &    6 &  2.70e-09 &  7.73e-17 &      \\ 
     &           &    7 &  2.70e-09 &  7.72e-17 &      \\ 
     &           &    8 &  2.70e-09 &  7.71e-17 &      \\ 
     &           &    9 &  2.69e-09 &  7.69e-17 &      \\ 
     &           &   10 &  2.69e-09 &  7.68e-17 &      \\ 
2709 &  2.71e+04 &   10 &           &           & iters  \\ 
 \hdashline 
     &           &    1 &  7.04e-08 &  3.49e-10 &      \\ 
     &           &    2 &  7.39e-09 &  2.01e-15 &      \\ 
     &           &    3 &  7.37e-09 &  2.11e-16 &      \\ 
     &           &    4 &  7.36e-09 &  2.10e-16 &      \\ 
     &           &    5 &  7.35e-09 &  2.10e-16 &      \\ 
     &           &    6 &  7.34e-09 &  2.10e-16 &      \\ 
     &           &    7 &  7.33e-09 &  2.10e-16 &      \\ 
     &           &    8 &  7.32e-09 &  2.09e-16 &      \\ 
     &           &    9 &  7.31e-09 &  2.09e-16 &      \\ 
     &           &   10 &  7.04e-08 &  2.09e-16 &      \\ 
2710 &  2.71e+04 &   10 &           &           & iters  \\ 
 \hdashline 
     &           &    1 &  5.65e-08 &  4.39e-10 &      \\ 
     &           &    2 &  1.75e-08 &  1.61e-15 &      \\ 
     &           &    3 &  1.75e-08 &  5.00e-16 &      \\ 
     &           &    4 &  6.02e-08 &  5.00e-16 &      \\ 
     &           &    5 &  1.76e-08 &  1.72e-15 &      \\ 
     &           &    6 &  1.75e-08 &  5.01e-16 &      \\ 
     &           &    7 &  1.75e-08 &  5.01e-16 &      \\ 
     &           &    8 &  6.02e-08 &  5.00e-16 &      \\ 
     &           &    9 &  1.76e-08 &  1.72e-15 &      \\ 
     &           &   10 &  1.76e-08 &  5.02e-16 &      \\ 
2711 &  2.71e+04 &   10 &           &           & iters  \\ 
 \hdashline 
     &           &    1 &  6.42e-10 &  4.37e-10 &      \\ 
     &           &    2 &  6.41e-10 &  1.83e-17 &      \\ 
     &           &    3 &  6.41e-10 &  1.83e-17 &      \\ 
     &           &    4 &  6.40e-10 &  1.83e-17 &      \\ 
     &           &    5 &  6.39e-10 &  1.83e-17 &      \\ 
     &           &    6 &  6.38e-10 &  1.82e-17 &      \\ 
     &           &    7 &  6.37e-10 &  1.82e-17 &      \\ 
     &           &    8 &  6.36e-10 &  1.82e-17 &      \\ 
     &           &    9 &  6.35e-10 &  1.82e-17 &      \\ 
     &           &   10 &  6.34e-10 &  1.81e-17 &      \\ 
2712 &  2.71e+04 &   10 &           &           & iters  \\ 
 \hdashline 
     &           &    1 &  2.43e-08 &  4.03e-10 &      \\ 
     &           &    2 &  1.46e-08 &  6.93e-16 &      \\ 
     &           &    3 &  1.46e-08 &  4.16e-16 &      \\ 
     &           &    4 &  2.43e-08 &  4.16e-16 &      \\ 
     &           &    5 &  1.46e-08 &  6.93e-16 &      \\ 
     &           &    6 &  2.43e-08 &  4.16e-16 &      \\ 
     &           &    7 &  1.46e-08 &  6.93e-16 &      \\ 
     &           &    8 &  1.46e-08 &  4.17e-16 &      \\ 
     &           &    9 &  2.43e-08 &  4.16e-16 &      \\ 
     &           &   10 &  1.46e-08 &  6.93e-16 &      \\ 
2713 &  2.71e+04 &   10 &           &           & iters  \\ 
 \hdashline 
     &           &    1 &  2.56e-08 &  4.76e-11 &      \\ 
     &           &    2 &  1.32e-08 &  7.32e-16 &      \\ 
     &           &    3 &  1.32e-08 &  3.78e-16 &      \\ 
     &           &    4 &  2.56e-08 &  3.77e-16 &      \\ 
     &           &    5 &  1.32e-08 &  7.32e-16 &      \\ 
     &           &    6 &  1.32e-08 &  3.78e-16 &      \\ 
     &           &    7 &  2.56e-08 &  3.77e-16 &      \\ 
     &           &    8 &  1.32e-08 &  7.32e-16 &      \\ 
     &           &    9 &  1.32e-08 &  3.78e-16 &      \\ 
     &           &   10 &  2.56e-08 &  3.77e-16 &      \\ 
2714 &  2.71e+04 &   10 &           &           & iters  \\ 
 \hdashline 
     &           &    1 &  5.10e-08 &  5.12e-10 &      \\ 
     &           &    2 &  2.68e-08 &  1.45e-15 &      \\ 
     &           &    3 &  2.68e-08 &  7.65e-16 &      \\ 
     &           &    4 &  5.10e-08 &  7.64e-16 &      \\ 
     &           &    5 &  2.68e-08 &  1.45e-15 &      \\ 
     &           &    6 &  2.68e-08 &  7.65e-16 &      \\ 
     &           &    7 &  5.10e-08 &  7.64e-16 &      \\ 
     &           &    8 &  2.68e-08 &  1.45e-15 &      \\ 
     &           &    9 &  2.67e-08 &  7.65e-16 &      \\ 
     &           &   10 &  5.10e-08 &  7.63e-16 &      \\ 
2715 &  2.71e+04 &   10 &           &           & iters  \\ 
 \hdashline 
     &           &    1 &  9.21e-09 &  3.60e-10 &      \\ 
     &           &    2 &  9.20e-09 &  2.63e-16 &      \\ 
     &           &    3 &  9.19e-09 &  2.63e-16 &      \\ 
     &           &    4 &  9.17e-09 &  2.62e-16 &      \\ 
     &           &    5 &  9.16e-09 &  2.62e-16 &      \\ 
     &           &    6 &  9.15e-09 &  2.61e-16 &      \\ 
     &           &    7 &  9.13e-09 &  2.61e-16 &      \\ 
     &           &    8 &  6.86e-08 &  2.61e-16 &      \\ 
     &           &    9 &  4.81e-08 &  1.96e-15 &      \\ 
     &           &   10 &  9.15e-09 &  1.37e-15 &      \\ 
2716 &  2.72e+04 &   10 &           &           & iters  \\ 
 \hdashline 
     &           &    1 &  8.81e-09 &  1.80e-10 &      \\ 
     &           &    2 &  8.80e-09 &  2.51e-16 &      \\ 
     &           &    3 &  8.78e-09 &  2.51e-16 &      \\ 
     &           &    4 &  8.77e-09 &  2.51e-16 &      \\ 
     &           &    5 &  8.76e-09 &  2.50e-16 &      \\ 
     &           &    6 &  3.01e-08 &  2.50e-16 &      \\ 
     &           &    7 &  8.79e-09 &  8.59e-16 &      \\ 
     &           &    8 &  8.78e-09 &  2.51e-16 &      \\ 
     &           &    9 &  8.76e-09 &  2.50e-16 &      \\ 
     &           &   10 &  8.75e-09 &  2.50e-16 &      \\ 
2717 &  2.72e+04 &   10 &           &           & iters  \\ 
 \hdashline 
     &           &    1 &  5.18e-08 &  2.85e-10 &      \\ 
     &           &    2 &  2.59e-08 &  1.48e-15 &      \\ 
     &           &    3 &  2.59e-08 &  7.40e-16 &      \\ 
     &           &    4 &  5.18e-08 &  7.39e-16 &      \\ 
     &           &    5 &  2.59e-08 &  1.48e-15 &      \\ 
     &           &    6 &  2.59e-08 &  7.40e-16 &      \\ 
     &           &    7 &  5.18e-08 &  7.39e-16 &      \\ 
     &           &    8 &  2.59e-08 &  1.48e-15 &      \\ 
     &           &    9 &  2.59e-08 &  7.40e-16 &      \\ 
     &           &   10 &  5.18e-08 &  7.39e-16 &      \\ 
2718 &  2.72e+04 &   10 &           &           & iters  \\ 
 \hdashline 
     &           &    1 &  2.96e-08 &  1.19e-10 &      \\ 
     &           &    2 &  2.95e-08 &  8.44e-16 &      \\ 
     &           &    3 &  4.82e-08 &  8.42e-16 &      \\ 
     &           &    4 &  2.95e-08 &  1.38e-15 &      \\ 
     &           &    5 &  4.82e-08 &  8.43e-16 &      \\ 
     &           &    6 &  2.96e-08 &  1.38e-15 &      \\ 
     &           &    7 &  2.95e-08 &  8.44e-16 &      \\ 
     &           &    8 &  4.82e-08 &  8.43e-16 &      \\ 
     &           &    9 &  2.96e-08 &  1.38e-15 &      \\ 
     &           &   10 &  2.95e-08 &  8.43e-16 &      \\ 
2719 &  2.72e+04 &   10 &           &           & iters  \\ 
 \hdashline 
     &           &    1 &  1.24e-08 &  3.48e-10 &      \\ 
     &           &    2 &  1.24e-08 &  3.55e-16 &      \\ 
     &           &    3 &  1.24e-08 &  3.55e-16 &      \\ 
     &           &    4 &  1.24e-08 &  3.54e-16 &      \\ 
     &           &    5 &  6.53e-08 &  3.54e-16 &      \\ 
     &           &    6 &  9.02e-08 &  1.86e-15 &      \\ 
     &           &    7 &  6.54e-08 &  2.57e-15 &      \\ 
     &           &    8 &  1.24e-08 &  1.87e-15 &      \\ 
     &           &    9 &  1.24e-08 &  3.55e-16 &      \\ 
     &           &   10 &  1.24e-08 &  3.54e-16 &      \\ 
2720 &  2.72e+04 &   10 &           &           & iters  \\ 
 \hdashline 
     &           &    1 &  1.24e-08 &  9.06e-10 &      \\ 
     &           &    2 &  1.24e-08 &  3.55e-16 &      \\ 
     &           &    3 &  2.64e-08 &  3.55e-16 &      \\ 
     &           &    4 &  1.24e-08 &  7.55e-16 &      \\ 
     &           &    5 &  1.24e-08 &  3.55e-16 &      \\ 
     &           &    6 &  2.64e-08 &  3.55e-16 &      \\ 
     &           &    7 &  1.24e-08 &  7.55e-16 &      \\ 
     &           &    8 &  1.24e-08 &  3.55e-16 &      \\ 
     &           &    9 &  2.64e-08 &  3.55e-16 &      \\ 
     &           &   10 &  1.24e-08 &  7.54e-16 &      \\ 
2721 &  2.72e+04 &   10 &           &           & iters  \\ 
 \hdashline 
     &           &    1 &  3.05e-09 &  6.88e-10 &      \\ 
     &           &    2 &  3.04e-09 &  8.70e-17 &      \\ 
     &           &    3 &  3.04e-09 &  8.68e-17 &      \\ 
     &           &    4 &  3.03e-09 &  8.67e-17 &      \\ 
     &           &    5 &  3.03e-09 &  8.66e-17 &      \\ 
     &           &    6 &  3.03e-09 &  8.65e-17 &      \\ 
     &           &    7 &  3.02e-09 &  8.63e-17 &      \\ 
     &           &    8 &  7.47e-08 &  8.62e-17 &      \\ 
     &           &    9 &  8.08e-08 &  2.13e-15 &      \\ 
     &           &   10 &  7.47e-08 &  2.31e-15 &      \\ 
2722 &  2.72e+04 &   10 &           &           & iters  \\ 
 \hdashline 
     &           &    1 &  8.36e-08 &  3.57e-11 &      \\ 
     &           &    2 &  7.19e-08 &  2.39e-15 &      \\ 
     &           &    3 &  8.36e-08 &  2.05e-15 &      \\ 
     &           &    4 &  7.19e-08 &  2.39e-15 &      \\ 
     &           &    5 &  5.88e-09 &  2.05e-15 &      \\ 
     &           &    6 &  5.87e-09 &  1.68e-16 &      \\ 
     &           &    7 &  5.86e-09 &  1.67e-16 &      \\ 
     &           &    8 &  5.85e-09 &  1.67e-16 &      \\ 
     &           &    9 &  5.84e-09 &  1.67e-16 &      \\ 
     &           &   10 &  5.83e-09 &  1.67e-16 &      \\ 
2723 &  2.72e+04 &   10 &           &           & iters  \\ 
 \hdashline 
     &           &    1 &  4.06e-08 &  1.13e-10 &      \\ 
     &           &    2 &  3.71e-08 &  1.16e-15 &      \\ 
     &           &    3 &  3.71e-08 &  1.06e-15 &      \\ 
     &           &    4 &  4.07e-08 &  1.06e-15 &      \\ 
     &           &    5 &  3.71e-08 &  1.16e-15 &      \\ 
     &           &    6 &  4.07e-08 &  1.06e-15 &      \\ 
     &           &    7 &  3.71e-08 &  1.16e-15 &      \\ 
     &           &    8 &  4.07e-08 &  1.06e-15 &      \\ 
     &           &    9 &  3.71e-08 &  1.16e-15 &      \\ 
     &           &   10 &  4.06e-08 &  1.06e-15 &      \\ 
2724 &  2.72e+04 &   10 &           &           & iters  \\ 
 \hdashline 
     &           &    1 &  9.54e-09 &  9.84e-11 &      \\ 
     &           &    2 &  9.52e-09 &  2.72e-16 &      \\ 
     &           &    3 &  9.51e-09 &  2.72e-16 &      \\ 
     &           &    4 &  9.50e-09 &  2.71e-16 &      \\ 
     &           &    5 &  6.82e-08 &  2.71e-16 &      \\ 
     &           &    6 &  9.58e-09 &  1.95e-15 &      \\ 
     &           &    7 &  9.57e-09 &  2.73e-16 &      \\ 
     &           &    8 &  9.55e-09 &  2.73e-16 &      \\ 
     &           &    9 &  9.54e-09 &  2.73e-16 &      \\ 
     &           &   10 &  9.52e-09 &  2.72e-16 &      \\ 
2725 &  2.72e+04 &   10 &           &           & iters  \\ 
 \hdashline 
     &           &    1 &  7.68e-09 &  1.79e-10 &      \\ 
     &           &    2 &  7.67e-09 &  2.19e-16 &      \\ 
     &           &    3 &  7.66e-09 &  2.19e-16 &      \\ 
     &           &    4 &  7.65e-09 &  2.19e-16 &      \\ 
     &           &    5 &  7.64e-09 &  2.18e-16 &      \\ 
     &           &    6 &  7.62e-09 &  2.18e-16 &      \\ 
     &           &    7 &  7.61e-09 &  2.18e-16 &      \\ 
     &           &    8 &  7.60e-09 &  2.17e-16 &      \\ 
     &           &    9 &  7.59e-09 &  2.17e-16 &      \\ 
     &           &   10 &  7.58e-09 &  2.17e-16 &      \\ 
2726 &  2.72e+04 &   10 &           &           & iters  \\ 
 \hdashline 
     &           &    1 &  2.97e-08 &  8.25e-11 &      \\ 
     &           &    2 &  2.97e-08 &  8.48e-16 &      \\ 
     &           &    3 &  2.96e-08 &  8.47e-16 &      \\ 
     &           &    4 &  4.81e-08 &  8.46e-16 &      \\ 
     &           &    5 &  2.97e-08 &  1.37e-15 &      \\ 
     &           &    6 &  2.96e-08 &  8.47e-16 &      \\ 
     &           &    7 &  4.81e-08 &  8.46e-16 &      \\ 
     &           &    8 &  2.97e-08 &  1.37e-15 &      \\ 
     &           &    9 &  4.81e-08 &  8.46e-16 &      \\ 
     &           &   10 &  2.97e-08 &  1.37e-15 &      \\ 
2727 &  2.73e+04 &   10 &           &           & iters  \\ 
 \hdashline 
     &           &    1 &  4.18e-08 &  2.44e-10 &      \\ 
     &           &    2 &  3.60e-08 &  1.19e-15 &      \\ 
     &           &    3 &  4.18e-08 &  1.03e-15 &      \\ 
     &           &    4 &  3.60e-08 &  1.19e-15 &      \\ 
     &           &    5 &  4.17e-08 &  1.03e-15 &      \\ 
     &           &    6 &  3.60e-08 &  1.19e-15 &      \\ 
     &           &    7 &  3.60e-08 &  1.03e-15 &      \\ 
     &           &    8 &  4.18e-08 &  1.03e-15 &      \\ 
     &           &    9 &  3.60e-08 &  1.19e-15 &      \\ 
     &           &   10 &  4.18e-08 &  1.03e-15 &      \\ 
2728 &  2.73e+04 &   10 &           &           & iters  \\ 
 \hdashline 
     &           &    1 &  3.81e-08 &  3.82e-10 &      \\ 
     &           &    2 &  3.97e-08 &  1.09e-15 &      \\ 
     &           &    3 &  3.81e-08 &  1.13e-15 &      \\ 
     &           &    4 &  3.97e-08 &  1.09e-15 &      \\ 
     &           &    5 &  3.81e-08 &  1.13e-15 &      \\ 
     &           &    6 &  3.97e-08 &  1.09e-15 &      \\ 
     &           &    7 &  3.81e-08 &  1.13e-15 &      \\ 
     &           &    8 &  3.97e-08 &  1.09e-15 &      \\ 
     &           &    9 &  3.81e-08 &  1.13e-15 &      \\ 
     &           &   10 &  3.97e-08 &  1.09e-15 &      \\ 
2729 &  2.73e+04 &   10 &           &           & iters  \\ 
 \hdashline 
     &           &    1 &  3.91e-08 &  1.31e-10 &      \\ 
     &           &    2 &  3.86e-08 &  1.12e-15 &      \\ 
     &           &    3 &  3.91e-08 &  1.10e-15 &      \\ 
     &           &    4 &  3.86e-08 &  1.12e-15 &      \\ 
     &           &    5 &  3.91e-08 &  1.10e-15 &      \\ 
     &           &    6 &  3.86e-08 &  1.12e-15 &      \\ 
     &           &    7 &  3.91e-08 &  1.10e-15 &      \\ 
     &           &    8 &  3.86e-08 &  1.12e-15 &      \\ 
     &           &    9 &  3.91e-08 &  1.10e-15 &      \\ 
     &           &   10 &  3.86e-08 &  1.12e-15 &      \\ 
2730 &  2.73e+04 &   10 &           &           & iters  \\ 
 \hdashline 
     &           &    1 &  4.09e-08 &  4.36e-10 &      \\ 
     &           &    2 &  3.68e-08 &  1.17e-15 &      \\ 
     &           &    3 &  4.09e-08 &  1.05e-15 &      \\ 
     &           &    4 &  3.68e-08 &  1.17e-15 &      \\ 
     &           &    5 &  4.09e-08 &  1.05e-15 &      \\ 
     &           &    6 &  3.68e-08 &  1.17e-15 &      \\ 
     &           &    7 &  4.09e-08 &  1.05e-15 &      \\ 
     &           &    8 &  3.68e-08 &  1.17e-15 &      \\ 
     &           &    9 &  3.68e-08 &  1.05e-15 &      \\ 
     &           &   10 &  4.09e-08 &  1.05e-15 &      \\ 
2731 &  2.73e+04 &   10 &           &           & iters  \\ 
 \hdashline 
     &           &    1 &  2.79e-08 &  2.26e-10 &      \\ 
     &           &    2 &  4.98e-08 &  7.96e-16 &      \\ 
     &           &    3 &  2.79e-08 &  1.42e-15 &      \\ 
     &           &    4 &  4.98e-08 &  7.97e-16 &      \\ 
     &           &    5 &  2.80e-08 &  1.42e-15 &      \\ 
     &           &    6 &  2.79e-08 &  7.98e-16 &      \\ 
     &           &    7 &  4.98e-08 &  7.97e-16 &      \\ 
     &           &    8 &  2.80e-08 &  1.42e-15 &      \\ 
     &           &    9 &  2.79e-08 &  7.98e-16 &      \\ 
     &           &   10 &  4.98e-08 &  7.97e-16 &      \\ 
2732 &  2.73e+04 &   10 &           &           & iters  \\ 
 \hdashline 
     &           &    1 &  7.52e-08 &  3.41e-10 &      \\ 
     &           &    2 &  2.60e-09 &  2.15e-15 &      \\ 
     &           &    3 &  2.60e-09 &  7.42e-17 &      \\ 
     &           &    4 &  2.59e-09 &  7.41e-17 &      \\ 
     &           &    5 &  2.59e-09 &  7.40e-17 &      \\ 
     &           &    6 &  2.58e-09 &  7.39e-17 &      \\ 
     &           &    7 &  2.58e-09 &  7.38e-17 &      \\ 
     &           &    8 &  2.58e-09 &  7.37e-17 &      \\ 
     &           &    9 &  2.57e-09 &  7.35e-17 &      \\ 
     &           &   10 &  2.57e-09 &  7.34e-17 &      \\ 
2733 &  2.73e+04 &   10 &           &           & iters  \\ 
 \hdashline 
     &           &    1 &  4.75e-08 &  4.62e-11 &      \\ 
     &           &    2 &  3.03e-08 &  1.36e-15 &      \\ 
     &           &    3 &  3.02e-08 &  8.64e-16 &      \\ 
     &           &    4 &  4.75e-08 &  8.63e-16 &      \\ 
     &           &    5 &  3.02e-08 &  1.36e-15 &      \\ 
     &           &    6 &  4.75e-08 &  8.63e-16 &      \\ 
     &           &    7 &  3.03e-08 &  1.36e-15 &      \\ 
     &           &    8 &  3.02e-08 &  8.64e-16 &      \\ 
     &           &    9 &  4.75e-08 &  8.63e-16 &      \\ 
     &           &   10 &  3.03e-08 &  1.36e-15 &      \\ 
2734 &  2.73e+04 &   10 &           &           & iters  \\ 
 \hdashline 
     &           &    1 &  5.71e-08 &  1.01e-09 &      \\ 
     &           &    2 &  2.06e-08 &  1.63e-15 &      \\ 
     &           &    3 &  2.06e-08 &  5.89e-16 &      \\ 
     &           &    4 &  5.71e-08 &  5.88e-16 &      \\ 
     &           &    5 &  2.07e-08 &  1.63e-15 &      \\ 
     &           &    6 &  2.06e-08 &  5.90e-16 &      \\ 
     &           &    7 &  2.06e-08 &  5.89e-16 &      \\ 
     &           &    8 &  5.71e-08 &  5.88e-16 &      \\ 
     &           &    9 &  2.07e-08 &  1.63e-15 &      \\ 
     &           &   10 &  2.06e-08 &  5.90e-16 &      \\ 
2735 &  2.73e+04 &   10 &           &           & iters  \\ 
 \hdashline 
     &           &    1 &  9.85e-09 &  1.00e-09 &      \\ 
     &           &    2 &  6.79e-08 &  2.81e-16 &      \\ 
     &           &    3 &  8.76e-08 &  1.94e-15 &      \\ 
     &           &    4 &  6.79e-08 &  2.50e-15 &      \\ 
     &           &    5 &  9.90e-09 &  1.94e-15 &      \\ 
     &           &    6 &  9.89e-09 &  2.83e-16 &      \\ 
     &           &    7 &  9.88e-09 &  2.82e-16 &      \\ 
     &           &    8 &  9.86e-09 &  2.82e-16 &      \\ 
     &           &    9 &  9.85e-09 &  2.81e-16 &      \\ 
     &           &   10 &  6.79e-08 &  2.81e-16 &      \\ 
2736 &  2.74e+04 &   10 &           &           & iters  \\ 
 \hdashline 
     &           &    1 &  4.40e-08 &  4.75e-10 &      \\ 
     &           &    2 &  3.38e-08 &  1.26e-15 &      \\ 
     &           &    3 &  4.40e-08 &  9.64e-16 &      \\ 
     &           &    4 &  3.38e-08 &  1.26e-15 &      \\ 
     &           &    5 &  4.40e-08 &  9.64e-16 &      \\ 
     &           &    6 &  3.38e-08 &  1.25e-15 &      \\ 
     &           &    7 &  4.39e-08 &  9.64e-16 &      \\ 
     &           &    8 &  3.38e-08 &  1.25e-15 &      \\ 
     &           &    9 &  3.38e-08 &  9.65e-16 &      \\ 
     &           &   10 &  4.40e-08 &  9.64e-16 &      \\ 
2737 &  2.74e+04 &   10 &           &           & iters  \\ 
 \hdashline 
     &           &    1 &  2.11e-08 &  4.96e-10 &      \\ 
     &           &    2 &  5.66e-08 &  6.04e-16 &      \\ 
     &           &    3 &  2.12e-08 &  1.61e-15 &      \\ 
     &           &    4 &  2.12e-08 &  6.05e-16 &      \\ 
     &           &    5 &  2.11e-08 &  6.04e-16 &      \\ 
     &           &    6 &  5.66e-08 &  6.03e-16 &      \\ 
     &           &    7 &  2.12e-08 &  1.61e-15 &      \\ 
     &           &    8 &  2.12e-08 &  6.05e-16 &      \\ 
     &           &    9 &  2.11e-08 &  6.04e-16 &      \\ 
     &           &   10 &  5.66e-08 &  6.03e-16 &      \\ 
2738 &  2.74e+04 &   10 &           &           & iters  \\ 
 \hdashline 
     &           &    1 &  2.33e-08 &  6.42e-11 &      \\ 
     &           &    2 &  2.33e-08 &  6.65e-16 &      \\ 
     &           &    3 &  5.44e-08 &  6.64e-16 &      \\ 
     &           &    4 &  2.33e-08 &  1.55e-15 &      \\ 
     &           &    5 &  2.33e-08 &  6.66e-16 &      \\ 
     &           &    6 &  2.33e-08 &  6.65e-16 &      \\ 
     &           &    7 &  5.45e-08 &  6.64e-16 &      \\ 
     &           &    8 &  2.33e-08 &  1.55e-15 &      \\ 
     &           &    9 &  2.33e-08 &  6.65e-16 &      \\ 
     &           &   10 &  5.45e-08 &  6.64e-16 &      \\ 
2739 &  2.74e+04 &   10 &           &           & iters  \\ 
 \hdashline 
     &           &    1 &  6.93e-09 &  1.80e-10 &      \\ 
     &           &    2 &  6.92e-09 &  1.98e-16 &      \\ 
     &           &    3 &  7.08e-08 &  1.98e-16 &      \\ 
     &           &    4 &  4.59e-08 &  2.02e-15 &      \\ 
     &           &    5 &  6.95e-09 &  1.31e-15 &      \\ 
     &           &    6 &  6.94e-09 &  1.98e-16 &      \\ 
     &           &    7 &  6.93e-09 &  1.98e-16 &      \\ 
     &           &    8 &  6.92e-09 &  1.98e-16 &      \\ 
     &           &    9 &  7.08e-08 &  1.97e-16 &      \\ 
     &           &   10 &  4.59e-08 &  2.02e-15 &      \\ 
2740 &  2.74e+04 &   10 &           &           & iters  \\ 
 \hdashline 
     &           &    1 &  2.17e-08 &  2.81e-11 &      \\ 
     &           &    2 &  1.71e-08 &  6.21e-16 &      \\ 
     &           &    3 &  2.17e-08 &  4.89e-16 &      \\ 
     &           &    4 &  1.71e-08 &  6.20e-16 &      \\ 
     &           &    5 &  2.17e-08 &  4.89e-16 &      \\ 
     &           &    6 &  1.71e-08 &  6.20e-16 &      \\ 
     &           &    7 &  1.71e-08 &  4.89e-16 &      \\ 
     &           &    8 &  2.18e-08 &  4.89e-16 &      \\ 
     &           &    9 &  1.71e-08 &  6.21e-16 &      \\ 
     &           &   10 &  2.17e-08 &  4.89e-16 &      \\ 
2741 &  2.74e+04 &   10 &           &           & iters  \\ 
 \hdashline 
     &           &    1 &  2.08e-08 &  3.11e-10 &      \\ 
     &           &    2 &  5.70e-08 &  5.93e-16 &      \\ 
     &           &    3 &  2.08e-08 &  1.63e-15 &      \\ 
     &           &    4 &  2.08e-08 &  5.94e-16 &      \\ 
     &           &    5 &  2.08e-08 &  5.93e-16 &      \\ 
     &           &    6 &  5.70e-08 &  5.92e-16 &      \\ 
     &           &    7 &  2.08e-08 &  1.63e-15 &      \\ 
     &           &    8 &  2.08e-08 &  5.94e-16 &      \\ 
     &           &    9 &  5.69e-08 &  5.93e-16 &      \\ 
     &           &   10 &  2.08e-08 &  1.63e-15 &      \\ 
2742 &  2.74e+04 &   10 &           &           & iters  \\ 
 \hdashline 
     &           &    1 &  1.34e-08 &  4.48e-10 &      \\ 
     &           &    2 &  1.33e-08 &  3.81e-16 &      \\ 
     &           &    3 &  1.33e-08 &  3.81e-16 &      \\ 
     &           &    4 &  1.33e-08 &  3.80e-16 &      \\ 
     &           &    5 &  6.44e-08 &  3.80e-16 &      \\ 
     &           &    6 &  1.34e-08 &  1.84e-15 &      \\ 
     &           &    7 &  1.34e-08 &  3.82e-16 &      \\ 
     &           &    8 &  1.33e-08 &  3.81e-16 &      \\ 
     &           &    9 &  1.33e-08 &  3.81e-16 &      \\ 
     &           &   10 &  1.33e-08 &  3.80e-16 &      \\ 
2743 &  2.74e+04 &   10 &           &           & iters  \\ 
 \hdashline 
     &           &    1 &  3.01e-08 &  2.75e-10 &      \\ 
     &           &    2 &  3.01e-08 &  8.60e-16 &      \\ 
     &           &    3 &  4.77e-08 &  8.58e-16 &      \\ 
     &           &    4 &  3.01e-08 &  1.36e-15 &      \\ 
     &           &    5 &  3.01e-08 &  8.59e-16 &      \\ 
     &           &    6 &  4.77e-08 &  8.58e-16 &      \\ 
     &           &    7 &  3.01e-08 &  1.36e-15 &      \\ 
     &           &    8 &  4.77e-08 &  8.58e-16 &      \\ 
     &           &    9 &  3.01e-08 &  1.36e-15 &      \\ 
     &           &   10 &  3.01e-08 &  8.59e-16 &      \\ 
2744 &  2.74e+04 &   10 &           &           & iters  \\ 
 \hdashline 
     &           &    1 &  2.66e-08 &  8.71e-11 &      \\ 
     &           &    2 &  1.23e-08 &  7.59e-16 &      \\ 
     &           &    3 &  2.66e-08 &  3.51e-16 &      \\ 
     &           &    4 &  1.23e-08 &  7.59e-16 &      \\ 
     &           &    5 &  1.23e-08 &  3.51e-16 &      \\ 
     &           &    6 &  2.66e-08 &  3.51e-16 &      \\ 
     &           &    7 &  1.23e-08 &  7.58e-16 &      \\ 
     &           &    8 &  1.23e-08 &  3.51e-16 &      \\ 
     &           &    9 &  2.66e-08 &  3.51e-16 &      \\ 
     &           &   10 &  1.23e-08 &  7.58e-16 &      \\ 
2745 &  2.74e+04 &   10 &           &           & iters  \\ 
 \hdashline 
     &           &    1 &  4.33e-08 &  4.74e-11 &      \\ 
     &           &    2 &  3.45e-08 &  1.24e-15 &      \\ 
     &           &    3 &  3.44e-08 &  9.83e-16 &      \\ 
     &           &    4 &  4.33e-08 &  9.82e-16 &      \\ 
     &           &    5 &  3.44e-08 &  1.24e-15 &      \\ 
     &           &    6 &  4.33e-08 &  9.82e-16 &      \\ 
     &           &    7 &  3.44e-08 &  1.24e-15 &      \\ 
     &           &    8 &  4.33e-08 &  9.83e-16 &      \\ 
     &           &    9 &  3.44e-08 &  1.24e-15 &      \\ 
     &           &   10 &  4.33e-08 &  9.83e-16 &      \\ 
2746 &  2.74e+04 &   10 &           &           & iters  \\ 
 \hdashline 
     &           &    1 &  1.49e-08 &  3.69e-10 &      \\ 
     &           &    2 &  1.49e-08 &  4.25e-16 &      \\ 
     &           &    3 &  1.48e-08 &  4.24e-16 &      \\ 
     &           &    4 &  1.48e-08 &  4.23e-16 &      \\ 
     &           &    5 &  1.48e-08 &  4.23e-16 &      \\ 
     &           &    6 &  1.48e-08 &  4.22e-16 &      \\ 
     &           &    7 &  1.48e-08 &  4.22e-16 &      \\ 
     &           &    8 &  6.30e-08 &  4.21e-16 &      \\ 
     &           &    9 &  1.48e-08 &  1.80e-15 &      \\ 
     &           &   10 &  1.48e-08 &  4.23e-16 &      \\ 
2747 &  2.75e+04 &   10 &           &           & iters  \\ 
 \hdashline 
     &           &    1 &  3.24e-08 &  3.88e-10 &      \\ 
     &           &    2 &  4.53e-08 &  9.25e-16 &      \\ 
     &           &    3 &  3.24e-08 &  1.29e-15 &      \\ 
     &           &    4 &  3.24e-08 &  9.25e-16 &      \\ 
     &           &    5 &  4.54e-08 &  9.24e-16 &      \\ 
     &           &    6 &  3.24e-08 &  1.29e-15 &      \\ 
     &           &    7 &  4.53e-08 &  9.24e-16 &      \\ 
     &           &    8 &  3.24e-08 &  1.29e-15 &      \\ 
     &           &    9 &  4.53e-08 &  9.25e-16 &      \\ 
     &           &   10 &  3.24e-08 &  1.29e-15 &      \\ 
2748 &  2.75e+04 &   10 &           &           & iters  \\ 
 \hdashline 
     &           &    1 &  2.07e-08 &  5.63e-10 &      \\ 
     &           &    2 &  5.70e-08 &  5.91e-16 &      \\ 
     &           &    3 &  2.08e-08 &  1.63e-15 &      \\ 
     &           &    4 &  2.07e-08 &  5.93e-16 &      \\ 
     &           &    5 &  2.07e-08 &  5.92e-16 &      \\ 
     &           &    6 &  5.70e-08 &  5.91e-16 &      \\ 
     &           &    7 &  2.08e-08 &  1.63e-15 &      \\ 
     &           &    8 &  2.07e-08 &  5.92e-16 &      \\ 
     &           &    9 &  5.70e-08 &  5.92e-16 &      \\ 
     &           &   10 &  2.08e-08 &  1.63e-15 &      \\ 
2749 &  2.75e+04 &   10 &           &           & iters  \\ 
 \hdashline 
     &           &    1 &  6.16e-08 &  9.26e-10 &      \\ 
     &           &    2 &  1.62e-08 &  1.76e-15 &      \\ 
     &           &    3 &  1.62e-08 &  4.62e-16 &      \\ 
     &           &    4 &  1.61e-08 &  4.61e-16 &      \\ 
     &           &    5 &  6.16e-08 &  4.61e-16 &      \\ 
     &           &    6 &  1.62e-08 &  1.76e-15 &      \\ 
     &           &    7 &  1.62e-08 &  4.62e-16 &      \\ 
     &           &    8 &  1.62e-08 &  4.62e-16 &      \\ 
     &           &    9 &  1.61e-08 &  4.61e-16 &      \\ 
     &           &   10 &  6.16e-08 &  4.60e-16 &      \\ 
2750 &  2.75e+04 &   10 &           &           & iters  \\ 
 \hdashline 
     &           &    1 &  2.54e-08 &  7.92e-10 &      \\ 
     &           &    2 &  2.54e-08 &  7.26e-16 &      \\ 
     &           &    3 &  5.23e-08 &  7.25e-16 &      \\ 
     &           &    4 &  2.54e-08 &  1.49e-15 &      \\ 
     &           &    5 &  2.54e-08 &  7.26e-16 &      \\ 
     &           &    6 &  5.23e-08 &  7.25e-16 &      \\ 
     &           &    7 &  2.54e-08 &  1.49e-15 &      \\ 
     &           &    8 &  2.54e-08 &  7.26e-16 &      \\ 
     &           &    9 &  5.23e-08 &  7.25e-16 &      \\ 
     &           &   10 &  2.54e-08 &  1.49e-15 &      \\ 
2751 &  2.75e+04 &   10 &           &           & iters  \\ 
 \hdashline 
     &           &    1 &  7.21e-09 &  1.95e-09 &      \\ 
     &           &    2 &  7.20e-09 &  2.06e-16 &      \\ 
     &           &    3 &  7.19e-09 &  2.05e-16 &      \\ 
     &           &    4 &  7.18e-09 &  2.05e-16 &      \\ 
     &           &    5 &  7.17e-09 &  2.05e-16 &      \\ 
     &           &    6 &  7.15e-09 &  2.04e-16 &      \\ 
     &           &    7 &  7.05e-08 &  2.04e-16 &      \\ 
     &           &    8 &  8.49e-08 &  2.01e-15 &      \\ 
     &           &    9 &  7.06e-08 &  2.42e-15 &      \\ 
     &           &   10 &  7.23e-09 &  2.01e-15 &      \\ 
2752 &  2.75e+04 &   10 &           &           & iters  \\ 
 \hdashline 
     &           &    1 &  3.58e-09 &  1.04e-09 &      \\ 
     &           &    2 &  3.58e-09 &  1.02e-16 &      \\ 
     &           &    3 &  3.57e-09 &  1.02e-16 &      \\ 
     &           &    4 &  7.41e-08 &  1.02e-16 &      \\ 
     &           &    5 &  8.14e-08 &  2.12e-15 &      \\ 
     &           &    6 &  7.41e-08 &  2.32e-15 &      \\ 
     &           &    7 &  8.14e-08 &  2.12e-15 &      \\ 
     &           &    8 &  7.41e-08 &  2.32e-15 &      \\ 
     &           &    9 &  3.65e-09 &  2.12e-15 &      \\ 
     &           &   10 &  3.65e-09 &  1.04e-16 &      \\ 
2753 &  2.75e+04 &   10 &           &           & iters  \\ 
 \hdashline 
     &           &    1 &  9.72e-09 &  2.73e-10 &      \\ 
     &           &    2 &  9.70e-09 &  2.77e-16 &      \\ 
     &           &    3 &  9.69e-09 &  2.77e-16 &      \\ 
     &           &    4 &  6.80e-08 &  2.77e-16 &      \\ 
     &           &    5 &  8.75e-08 &  1.94e-15 &      \\ 
     &           &    6 &  6.80e-08 &  2.50e-15 &      \\ 
     &           &    7 &  9.75e-09 &  1.94e-15 &      \\ 
     &           &    8 &  9.73e-09 &  2.78e-16 &      \\ 
     &           &    9 &  9.72e-09 &  2.78e-16 &      \\ 
     &           &   10 &  9.70e-09 &  2.77e-16 &      \\ 
2754 &  2.75e+04 &   10 &           &           & iters  \\ 
 \hdashline 
     &           &    1 &  3.38e-08 &  1.08e-09 &      \\ 
     &           &    2 &  4.39e-08 &  9.65e-16 &      \\ 
     &           &    3 &  3.38e-08 &  1.25e-15 &      \\ 
     &           &    4 &  3.38e-08 &  9.66e-16 &      \\ 
     &           &    5 &  4.39e-08 &  9.64e-16 &      \\ 
     &           &    6 &  3.38e-08 &  1.25e-15 &      \\ 
     &           &    7 &  4.39e-08 &  9.65e-16 &      \\ 
     &           &    8 &  3.38e-08 &  1.25e-15 &      \\ 
     &           &    9 &  4.39e-08 &  9.65e-16 &      \\ 
     &           &   10 &  3.38e-08 &  1.25e-15 &      \\ 
2755 &  2.75e+04 &   10 &           &           & iters  \\ 
 \hdashline 
     &           &    1 &  9.41e-09 &  7.51e-10 &      \\ 
     &           &    2 &  9.40e-09 &  2.69e-16 &      \\ 
     &           &    3 &  9.39e-09 &  2.68e-16 &      \\ 
     &           &    4 &  9.37e-09 &  2.68e-16 &      \\ 
     &           &    5 &  9.36e-09 &  2.67e-16 &      \\ 
     &           &    6 &  9.35e-09 &  2.67e-16 &      \\ 
     &           &    7 &  6.84e-08 &  2.67e-16 &      \\ 
     &           &    8 &  8.71e-08 &  1.95e-15 &      \\ 
     &           &    9 &  6.84e-08 &  2.49e-15 &      \\ 
     &           &   10 &  9.40e-09 &  1.95e-15 &      \\ 
2756 &  2.76e+04 &   10 &           &           & iters  \\ 
 \hdashline 
     &           &    1 &  1.36e-08 &  1.56e-10 &      \\ 
     &           &    2 &  1.36e-08 &  3.89e-16 &      \\ 
     &           &    3 &  1.36e-08 &  3.88e-16 &      \\ 
     &           &    4 &  6.41e-08 &  3.88e-16 &      \\ 
     &           &    5 &  1.37e-08 &  1.83e-15 &      \\ 
     &           &    6 &  1.36e-08 &  3.90e-16 &      \\ 
     &           &    7 &  1.36e-08 &  3.89e-16 &      \\ 
     &           &    8 &  1.36e-08 &  3.89e-16 &      \\ 
     &           &    9 &  1.36e-08 &  3.88e-16 &      \\ 
     &           &   10 &  6.41e-08 &  3.88e-16 &      \\ 
2757 &  2.76e+04 &   10 &           &           & iters  \\ 
 \hdashline 
     &           &    1 &  9.59e-09 &  4.27e-10 &      \\ 
     &           &    2 &  6.81e-08 &  2.74e-16 &      \\ 
     &           &    3 &  9.67e-09 &  1.94e-15 &      \\ 
     &           &    4 &  9.66e-09 &  2.76e-16 &      \\ 
     &           &    5 &  9.65e-09 &  2.76e-16 &      \\ 
     &           &    6 &  9.63e-09 &  2.75e-16 &      \\ 
     &           &    7 &  9.62e-09 &  2.75e-16 &      \\ 
     &           &    8 &  9.60e-09 &  2.74e-16 &      \\ 
     &           &    9 &  9.59e-09 &  2.74e-16 &      \\ 
     &           &   10 &  6.81e-08 &  2.74e-16 &      \\ 
2758 &  2.76e+04 &   10 &           &           & iters  \\ 
 \hdashline 
     &           &    1 &  4.37e-08 &  5.11e-10 &      \\ 
     &           &    2 &  3.41e-08 &  1.25e-15 &      \\ 
     &           &    3 &  4.37e-08 &  9.73e-16 &      \\ 
     &           &    4 &  3.41e-08 &  1.25e-15 &      \\ 
     &           &    5 &  4.36e-08 &  9.73e-16 &      \\ 
     &           &    6 &  3.41e-08 &  1.25e-15 &      \\ 
     &           &    7 &  4.36e-08 &  9.73e-16 &      \\ 
     &           &    8 &  3.41e-08 &  1.25e-15 &      \\ 
     &           &    9 &  4.36e-08 &  9.74e-16 &      \\ 
     &           &   10 &  3.41e-08 &  1.24e-15 &      \\ 
2759 &  2.76e+04 &   10 &           &           & iters  \\ 
 \hdashline 
     &           &    1 &  1.57e-08 &  2.09e-09 &      \\ 
     &           &    2 &  1.57e-08 &  4.48e-16 &      \\ 
     &           &    3 &  1.56e-08 &  4.47e-16 &      \\ 
     &           &    4 &  6.21e-08 &  4.46e-16 &      \\ 
     &           &    5 &  1.57e-08 &  1.77e-15 &      \\ 
     &           &    6 &  1.57e-08 &  4.48e-16 &      \\ 
     &           &    7 &  1.57e-08 &  4.48e-16 &      \\ 
     &           &    8 &  1.56e-08 &  4.47e-16 &      \\ 
     &           &    9 &  6.21e-08 &  4.46e-16 &      \\ 
     &           &   10 &  1.57e-08 &  1.77e-15 &      \\ 
2760 &  2.76e+04 &   10 &           &           & iters  \\ 
 \hdashline 
     &           &    1 &  7.11e-09 &  2.38e-09 &      \\ 
     &           &    2 &  7.10e-09 &  2.03e-16 &      \\ 
     &           &    3 &  7.09e-09 &  2.03e-16 &      \\ 
     &           &    4 &  7.08e-09 &  2.02e-16 &      \\ 
     &           &    5 &  3.18e-08 &  2.02e-16 &      \\ 
     &           &    6 &  7.12e-09 &  9.07e-16 &      \\ 
     &           &    7 &  7.11e-09 &  2.03e-16 &      \\ 
     &           &    8 &  7.10e-09 &  2.03e-16 &      \\ 
     &           &    9 &  7.09e-09 &  2.03e-16 &      \\ 
     &           &   10 &  3.18e-08 &  2.02e-16 &      \\ 
2761 &  2.76e+04 &   10 &           &           & iters  \\ 
 \hdashline 
     &           &    1 &  5.01e-08 &  1.22e-09 &      \\ 
     &           &    2 &  2.76e-08 &  1.43e-15 &      \\ 
     &           &    3 &  2.76e-08 &  7.89e-16 &      \\ 
     &           &    4 &  5.01e-08 &  7.87e-16 &      \\ 
     &           &    5 &  2.76e-08 &  1.43e-15 &      \\ 
     &           &    6 &  2.76e-08 &  7.88e-16 &      \\ 
     &           &    7 &  5.01e-08 &  7.87e-16 &      \\ 
     &           &    8 &  2.76e-08 &  1.43e-15 &      \\ 
     &           &    9 &  2.76e-08 &  7.88e-16 &      \\ 
     &           &   10 &  5.02e-08 &  7.87e-16 &      \\ 
2762 &  2.76e+04 &   10 &           &           & iters  \\ 
 \hdashline 
     &           &    1 &  2.19e-08 &  5.67e-10 &      \\ 
     &           &    2 &  5.58e-08 &  6.24e-16 &      \\ 
     &           &    3 &  2.19e-08 &  1.59e-15 &      \\ 
     &           &    4 &  2.19e-08 &  6.26e-16 &      \\ 
     &           &    5 &  5.58e-08 &  6.25e-16 &      \\ 
     &           &    6 &  2.19e-08 &  1.59e-15 &      \\ 
     &           &    7 &  2.19e-08 &  6.26e-16 &      \\ 
     &           &    8 &  2.19e-08 &  6.25e-16 &      \\ 
     &           &    9 &  5.58e-08 &  6.25e-16 &      \\ 
     &           &   10 &  2.19e-08 &  1.59e-15 &      \\ 
2763 &  2.76e+04 &   10 &           &           & iters  \\ 
 \hdashline 
     &           &    1 &  1.58e-08 &  2.70e-10 &      \\ 
     &           &    2 &  6.20e-08 &  4.50e-16 &      \\ 
     &           &    3 &  1.58e-08 &  1.77e-15 &      \\ 
     &           &    4 &  1.58e-08 &  4.51e-16 &      \\ 
     &           &    5 &  1.58e-08 &  4.51e-16 &      \\ 
     &           &    6 &  1.57e-08 &  4.50e-16 &      \\ 
     &           &    7 &  6.20e-08 &  4.49e-16 &      \\ 
     &           &    8 &  1.58e-08 &  1.77e-15 &      \\ 
     &           &    9 &  1.58e-08 &  4.51e-16 &      \\ 
     &           &   10 &  1.58e-08 &  4.51e-16 &      \\ 
2764 &  2.76e+04 &   10 &           &           & iters  \\ 
 \hdashline 
     &           &    1 &  3.69e-08 &  4.55e-10 &      \\ 
     &           &    2 &  4.09e-08 &  1.05e-15 &      \\ 
     &           &    3 &  3.69e-08 &  1.17e-15 &      \\ 
     &           &    4 &  4.09e-08 &  1.05e-15 &      \\ 
     &           &    5 &  3.69e-08 &  1.17e-15 &      \\ 
     &           &    6 &  4.09e-08 &  1.05e-15 &      \\ 
     &           &    7 &  3.69e-08 &  1.17e-15 &      \\ 
     &           &    8 &  4.09e-08 &  1.05e-15 &      \\ 
     &           &    9 &  3.69e-08 &  1.17e-15 &      \\ 
     &           &   10 &  3.68e-08 &  1.05e-15 &      \\ 
2765 &  2.76e+04 &   10 &           &           & iters  \\ 
 \hdashline 
     &           &    1 &  4.43e-08 &  1.02e-09 &      \\ 
     &           &    2 &  5.38e-09 &  1.26e-15 &      \\ 
     &           &    3 &  5.37e-09 &  1.54e-16 &      \\ 
     &           &    4 &  5.37e-09 &  1.53e-16 &      \\ 
     &           &    5 &  5.36e-09 &  1.53e-16 &      \\ 
     &           &    6 &  5.35e-09 &  1.53e-16 &      \\ 
     &           &    7 &  5.34e-09 &  1.53e-16 &      \\ 
     &           &    8 &  5.34e-09 &  1.52e-16 &      \\ 
     &           &    9 &  5.33e-09 &  1.52e-16 &      \\ 
     &           &   10 &  5.32e-09 &  1.52e-16 &      \\ 
2766 &  2.76e+04 &   10 &           &           & iters  \\ 
 \hdashline 
     &           &    1 &  4.72e-08 &  1.35e-09 &      \\ 
     &           &    2 &  8.25e-09 &  1.35e-15 &      \\ 
     &           &    3 &  8.24e-09 &  2.35e-16 &      \\ 
     &           &    4 &  8.22e-09 &  2.35e-16 &      \\ 
     &           &    5 &  8.21e-09 &  2.35e-16 &      \\ 
     &           &    6 &  8.20e-09 &  2.34e-16 &      \\ 
     &           &    7 &  6.95e-08 &  2.34e-16 &      \\ 
     &           &    8 &  4.71e-08 &  1.98e-15 &      \\ 
     &           &    9 &  8.22e-09 &  1.35e-15 &      \\ 
     &           &   10 &  8.21e-09 &  2.35e-16 &      \\ 
2767 &  2.77e+04 &   10 &           &           & iters  \\ 
 \hdashline 
     &           &    1 &  7.26e-08 &  1.33e-09 &      \\ 
     &           &    2 &  4.40e-08 &  2.07e-15 &      \\ 
     &           &    3 &  5.13e-09 &  1.26e-15 &      \\ 
     &           &    4 &  5.12e-09 &  1.46e-16 &      \\ 
     &           &    5 &  5.11e-09 &  1.46e-16 &      \\ 
     &           &    6 &  5.11e-09 &  1.46e-16 &      \\ 
     &           &    7 &  5.10e-09 &  1.46e-16 &      \\ 
     &           &    8 &  5.09e-09 &  1.46e-16 &      \\ 
     &           &    9 &  7.26e-08 &  1.45e-16 &      \\ 
     &           &   10 &  4.40e-08 &  2.07e-15 &      \\ 
2768 &  2.77e+04 &   10 &           &           & iters  \\ 
 \hdashline 
     &           &    1 &  1.10e-08 &  1.13e-09 &      \\ 
     &           &    2 &  1.10e-08 &  3.14e-16 &      \\ 
     &           &    3 &  1.10e-08 &  3.13e-16 &      \\ 
     &           &    4 &  6.67e-08 &  3.13e-16 &      \\ 
     &           &    5 &  8.87e-08 &  1.90e-15 &      \\ 
     &           &    6 &  6.68e-08 &  2.53e-15 &      \\ 
     &           &    7 &  1.10e-08 &  1.91e-15 &      \\ 
     &           &    8 &  1.10e-08 &  3.14e-16 &      \\ 
     &           &    9 &  1.10e-08 &  3.14e-16 &      \\ 
     &           &   10 &  1.10e-08 &  3.13e-16 &      \\ 
2769 &  2.77e+04 &   10 &           &           & iters  \\ 
 \hdashline 
     &           &    1 &  3.07e-08 &  9.38e-10 &      \\ 
     &           &    2 &  3.07e-08 &  8.77e-16 &      \\ 
     &           &    3 &  4.71e-08 &  8.75e-16 &      \\ 
     &           &    4 &  3.07e-08 &  1.34e-15 &      \\ 
     &           &    5 &  3.07e-08 &  8.76e-16 &      \\ 
     &           &    6 &  4.71e-08 &  8.75e-16 &      \\ 
     &           &    7 &  3.07e-08 &  1.34e-15 &      \\ 
     &           &    8 &  4.71e-08 &  8.75e-16 &      \\ 
     &           &    9 &  3.07e-08 &  1.34e-15 &      \\ 
     &           &   10 &  3.07e-08 &  8.76e-16 &      \\ 
2770 &  2.77e+04 &   10 &           &           & iters  \\ 
 \hdashline 
     &           &    1 &  7.22e-08 &  9.22e-10 &      \\ 
     &           &    2 &  5.62e-09 &  2.06e-15 &      \\ 
     &           &    3 &  5.61e-09 &  1.60e-16 &      \\ 
     &           &    4 &  5.61e-09 &  1.60e-16 &      \\ 
     &           &    5 &  5.60e-09 &  1.60e-16 &      \\ 
     &           &    6 &  5.59e-09 &  1.60e-16 &      \\ 
     &           &    7 &  5.58e-09 &  1.60e-16 &      \\ 
     &           &    8 &  7.21e-08 &  1.59e-16 &      \\ 
     &           &    9 &  8.34e-08 &  2.06e-15 &      \\ 
     &           &   10 &  7.21e-08 &  2.38e-15 &      \\ 
2771 &  2.77e+04 &   10 &           &           & iters  \\ 
 \hdashline 
     &           &    1 &  1.51e-08 &  5.89e-10 &      \\ 
     &           &    2 &  1.51e-08 &  4.32e-16 &      \\ 
     &           &    3 &  1.51e-08 &  4.32e-16 &      \\ 
     &           &    4 &  1.51e-08 &  4.31e-16 &      \\ 
     &           &    5 &  6.26e-08 &  4.31e-16 &      \\ 
     &           &    6 &  1.52e-08 &  1.79e-15 &      \\ 
     &           &    7 &  1.51e-08 &  4.32e-16 &      \\ 
     &           &    8 &  1.51e-08 &  4.32e-16 &      \\ 
     &           &    9 &  1.51e-08 &  4.31e-16 &      \\ 
     &           &   10 &  6.26e-08 &  4.31e-16 &      \\ 
2772 &  2.77e+04 &   10 &           &           & iters  \\ 
 \hdashline 
     &           &    1 &  5.26e-08 &  2.72e-10 &      \\ 
     &           &    2 &  2.51e-08 &  1.50e-15 &      \\ 
     &           &    3 &  2.51e-08 &  7.17e-16 &      \\ 
     &           &    4 &  5.26e-08 &  7.16e-16 &      \\ 
     &           &    5 &  2.51e-08 &  1.50e-15 &      \\ 
     &           &    6 &  2.51e-08 &  7.17e-16 &      \\ 
     &           &    7 &  5.26e-08 &  7.16e-16 &      \\ 
     &           &    8 &  2.51e-08 &  1.50e-15 &      \\ 
     &           &    9 &  2.51e-08 &  7.17e-16 &      \\ 
     &           &   10 &  5.26e-08 &  7.16e-16 &      \\ 
2773 &  2.77e+04 &   10 &           &           & iters  \\ 
 \hdashline 
     &           &    1 &  2.43e-08 &  1.88e-10 &      \\ 
     &           &    2 &  2.43e-08 &  6.94e-16 &      \\ 
     &           &    3 &  5.35e-08 &  6.93e-16 &      \\ 
     &           &    4 &  2.43e-08 &  1.53e-15 &      \\ 
     &           &    5 &  2.43e-08 &  6.94e-16 &      \\ 
     &           &    6 &  5.35e-08 &  6.93e-16 &      \\ 
     &           &    7 &  2.43e-08 &  1.53e-15 &      \\ 
     &           &    8 &  2.43e-08 &  6.94e-16 &      \\ 
     &           &    9 &  5.34e-08 &  6.93e-16 &      \\ 
     &           &   10 &  2.43e-08 &  1.53e-15 &      \\ 
2774 &  2.77e+04 &   10 &           &           & iters  \\ 
 \hdashline 
     &           &    1 &  2.06e-08 &  4.80e-10 &      \\ 
     &           &    2 &  2.06e-08 &  5.88e-16 &      \\ 
     &           &    3 &  5.71e-08 &  5.87e-16 &      \\ 
     &           &    4 &  2.06e-08 &  1.63e-15 &      \\ 
     &           &    5 &  2.06e-08 &  5.89e-16 &      \\ 
     &           &    6 &  5.71e-08 &  5.88e-16 &      \\ 
     &           &    7 &  2.06e-08 &  1.63e-15 &      \\ 
     &           &    8 &  2.06e-08 &  5.89e-16 &      \\ 
     &           &    9 &  2.06e-08 &  5.88e-16 &      \\ 
     &           &   10 &  5.71e-08 &  5.88e-16 &      \\ 
2775 &  2.77e+04 &   10 &           &           & iters  \\ 
 \hdashline 
     &           &    1 &  7.02e-08 &  7.13e-11 &      \\ 
     &           &    2 &  7.58e-09 &  2.00e-15 &      \\ 
     &           &    3 &  7.57e-09 &  2.16e-16 &      \\ 
     &           &    4 &  7.56e-09 &  2.16e-16 &      \\ 
     &           &    5 &  7.55e-09 &  2.16e-16 &      \\ 
     &           &    6 &  7.54e-09 &  2.15e-16 &      \\ 
     &           &    7 &  7.53e-09 &  2.15e-16 &      \\ 
     &           &    8 &  7.52e-09 &  2.15e-16 &      \\ 
     &           &    9 &  7.51e-09 &  2.15e-16 &      \\ 
     &           &   10 &  7.02e-08 &  2.14e-16 &      \\ 
2776 &  2.78e+04 &   10 &           &           & iters  \\ 
 \hdashline 
     &           &    1 &  5.57e-08 &  5.94e-10 &      \\ 
     &           &    2 &  5.57e-08 &  1.59e-15 &      \\ 
     &           &    3 &  2.21e-08 &  1.59e-15 &      \\ 
     &           &    4 &  2.21e-08 &  6.31e-16 &      \\ 
     &           &    5 &  2.20e-08 &  6.30e-16 &      \\ 
     &           &    6 &  5.57e-08 &  6.29e-16 &      \\ 
     &           &    7 &  2.21e-08 &  1.59e-15 &      \\ 
     &           &    8 &  2.21e-08 &  6.30e-16 &      \\ 
     &           &    9 &  5.57e-08 &  6.30e-16 &      \\ 
     &           &   10 &  2.21e-08 &  1.59e-15 &      \\ 
2777 &  2.78e+04 &   10 &           &           & iters  \\ 
 \hdashline 
     &           &    1 &  5.80e-08 &  4.32e-11 &      \\ 
     &           &    2 &  1.98e-08 &  1.66e-15 &      \\ 
     &           &    3 &  1.97e-08 &  5.64e-16 &      \\ 
     &           &    4 &  1.97e-08 &  5.63e-16 &      \\ 
     &           &    5 &  5.80e-08 &  5.62e-16 &      \\ 
     &           &    6 &  1.97e-08 &  1.66e-15 &      \\ 
     &           &    7 &  1.97e-08 &  5.64e-16 &      \\ 
     &           &    8 &  1.97e-08 &  5.63e-16 &      \\ 
     &           &    9 &  5.80e-08 &  5.62e-16 &      \\ 
     &           &   10 &  1.97e-08 &  1.66e-15 &      \\ 
2778 &  2.78e+04 &   10 &           &           & iters  \\ 
 \hdashline 
     &           &    1 &  2.10e-08 &  5.85e-10 &      \\ 
     &           &    2 &  2.09e-08 &  5.99e-16 &      \\ 
     &           &    3 &  5.68e-08 &  5.98e-16 &      \\ 
     &           &    4 &  2.10e-08 &  1.62e-15 &      \\ 
     &           &    5 &  2.10e-08 &  5.99e-16 &      \\ 
     &           &    6 &  5.68e-08 &  5.98e-16 &      \\ 
     &           &    7 &  2.10e-08 &  1.62e-15 &      \\ 
     &           &    8 &  2.10e-08 &  6.00e-16 &      \\ 
     &           &    9 &  2.10e-08 &  5.99e-16 &      \\ 
     &           &   10 &  5.68e-08 &  5.98e-16 &      \\ 
2779 &  2.78e+04 &   10 &           &           & iters  \\ 
 \hdashline 
     &           &    1 &  5.02e-08 &  5.82e-10 &      \\ 
     &           &    2 &  2.75e-08 &  1.43e-15 &      \\ 
     &           &    3 &  2.75e-08 &  7.86e-16 &      \\ 
     &           &    4 &  5.02e-08 &  7.85e-16 &      \\ 
     &           &    5 &  2.75e-08 &  1.43e-15 &      \\ 
     &           &    6 &  2.75e-08 &  7.86e-16 &      \\ 
     &           &    7 &  5.02e-08 &  7.85e-16 &      \\ 
     &           &    8 &  2.75e-08 &  1.43e-15 &      \\ 
     &           &    9 &  2.75e-08 &  7.86e-16 &      \\ 
     &           &   10 &  5.02e-08 &  7.84e-16 &      \\ 
2780 &  2.78e+04 &   10 &           &           & iters  \\ 
 \hdashline 
     &           &    1 &  3.25e-08 &  1.34e-09 &      \\ 
     &           &    2 &  3.25e-08 &  9.28e-16 &      \\ 
     &           &    3 &  4.53e-08 &  9.27e-16 &      \\ 
     &           &    4 &  3.25e-08 &  1.29e-15 &      \\ 
     &           &    5 &  4.52e-08 &  9.27e-16 &      \\ 
     &           &    6 &  3.25e-08 &  1.29e-15 &      \\ 
     &           &    7 &  3.25e-08 &  9.28e-16 &      \\ 
     &           &    8 &  4.53e-08 &  9.27e-16 &      \\ 
     &           &    9 &  3.25e-08 &  1.29e-15 &      \\ 
     &           &   10 &  4.53e-08 &  9.27e-16 &      \\ 
2781 &  2.78e+04 &   10 &           &           & iters  \\ 
 \hdashline 
     &           &    1 &  3.08e-08 &  6.99e-10 &      \\ 
     &           &    2 &  4.69e-08 &  8.79e-16 &      \\ 
     &           &    3 &  3.08e-08 &  1.34e-15 &      \\ 
     &           &    4 &  3.08e-08 &  8.80e-16 &      \\ 
     &           &    5 &  4.69e-08 &  8.79e-16 &      \\ 
     &           &    6 &  3.08e-08 &  1.34e-15 &      \\ 
     &           &    7 &  4.69e-08 &  8.80e-16 &      \\ 
     &           &    8 &  3.08e-08 &  1.34e-15 &      \\ 
     &           &    9 &  3.08e-08 &  8.80e-16 &      \\ 
     &           &   10 &  4.69e-08 &  8.79e-16 &      \\ 
2782 &  2.78e+04 &   10 &           &           & iters  \\ 
 \hdashline 
     &           &    1 &  3.71e-08 &  6.13e-10 &      \\ 
     &           &    2 &  4.07e-08 &  1.06e-15 &      \\ 
     &           &    3 &  3.71e-08 &  1.16e-15 &      \\ 
     &           &    4 &  4.07e-08 &  1.06e-15 &      \\ 
     &           &    5 &  3.71e-08 &  1.16e-15 &      \\ 
     &           &    6 &  4.07e-08 &  1.06e-15 &      \\ 
     &           &    7 &  3.71e-08 &  1.16e-15 &      \\ 
     &           &    8 &  4.07e-08 &  1.06e-15 &      \\ 
     &           &    9 &  3.71e-08 &  1.16e-15 &      \\ 
     &           &   10 &  4.07e-08 &  1.06e-15 &      \\ 
2783 &  2.78e+04 &   10 &           &           & iters  \\ 
 \hdashline 
     &           &    1 &  5.39e-08 &  9.10e-10 &      \\ 
     &           &    2 &  2.39e-08 &  1.54e-15 &      \\ 
     &           &    3 &  2.38e-08 &  6.81e-16 &      \\ 
     &           &    4 &  5.39e-08 &  6.80e-16 &      \\ 
     &           &    5 &  2.39e-08 &  1.54e-15 &      \\ 
     &           &    6 &  2.38e-08 &  6.82e-16 &      \\ 
     &           &    7 &  2.38e-08 &  6.81e-16 &      \\ 
     &           &    8 &  5.39e-08 &  6.80e-16 &      \\ 
     &           &    9 &  2.39e-08 &  1.54e-15 &      \\ 
     &           &   10 &  2.38e-08 &  6.81e-16 &      \\ 
2784 &  2.78e+04 &   10 &           &           & iters  \\ 
 \hdashline 
     &           &    1 &  2.92e-08 &  8.67e-10 &      \\ 
     &           &    2 &  2.92e-08 &  8.34e-16 &      \\ 
     &           &    3 &  4.85e-08 &  8.33e-16 &      \\ 
     &           &    4 &  2.92e-08 &  1.39e-15 &      \\ 
     &           &    5 &  2.92e-08 &  8.34e-16 &      \\ 
     &           &    6 &  4.86e-08 &  8.32e-16 &      \\ 
     &           &    7 &  2.92e-08 &  1.39e-15 &      \\ 
     &           &    8 &  4.85e-08 &  8.33e-16 &      \\ 
     &           &    9 &  2.92e-08 &  1.39e-15 &      \\ 
     &           &   10 &  2.92e-08 &  8.34e-16 &      \\ 
2785 &  2.78e+04 &   10 &           &           & iters  \\ 
 \hdashline 
     &           &    1 &  4.92e-08 &  6.64e-10 &      \\ 
     &           &    2 &  2.86e-08 &  1.40e-15 &      \\ 
     &           &    3 &  4.92e-08 &  8.15e-16 &      \\ 
     &           &    4 &  2.86e-08 &  1.40e-15 &      \\ 
     &           &    5 &  2.85e-08 &  8.16e-16 &      \\ 
     &           &    6 &  4.92e-08 &  8.15e-16 &      \\ 
     &           &    7 &  2.86e-08 &  1.40e-15 &      \\ 
     &           &    8 &  2.85e-08 &  8.16e-16 &      \\ 
     &           &    9 &  4.92e-08 &  8.14e-16 &      \\ 
     &           &   10 &  2.86e-08 &  1.40e-15 &      \\ 
2786 &  2.78e+04 &   10 &           &           & iters  \\ 
 \hdashline 
     &           &    1 &  3.35e-08 &  2.47e-10 &      \\ 
     &           &    2 &  4.42e-08 &  9.57e-16 &      \\ 
     &           &    3 &  3.36e-08 &  1.26e-15 &      \\ 
     &           &    4 &  4.42e-08 &  9.58e-16 &      \\ 
     &           &    5 &  3.36e-08 &  1.26e-15 &      \\ 
     &           &    6 &  4.42e-08 &  9.58e-16 &      \\ 
     &           &    7 &  3.36e-08 &  1.26e-15 &      \\ 
     &           &    8 &  3.35e-08 &  9.58e-16 &      \\ 
     &           &    9 &  4.42e-08 &  9.57e-16 &      \\ 
     &           &   10 &  3.36e-08 &  1.26e-15 &      \\ 
2787 &  2.79e+04 &   10 &           &           & iters  \\ 
 \hdashline 
     &           &    1 &  4.63e-08 &  9.21e-10 &      \\ 
     &           &    2 &  7.41e-09 &  1.32e-15 &      \\ 
     &           &    3 &  7.40e-09 &  2.11e-16 &      \\ 
     &           &    4 &  7.39e-09 &  2.11e-16 &      \\ 
     &           &    5 &  7.03e-08 &  2.11e-16 &      \\ 
     &           &    6 &  4.63e-08 &  2.01e-15 &      \\ 
     &           &    7 &  7.41e-09 &  1.32e-15 &      \\ 
     &           &    8 &  7.40e-09 &  2.12e-16 &      \\ 
     &           &    9 &  7.39e-09 &  2.11e-16 &      \\ 
     &           &   10 &  7.03e-08 &  2.11e-16 &      \\ 
2788 &  2.79e+04 &   10 &           &           & iters  \\ 
 \hdashline 
     &           &    1 &  7.85e-09 &  6.15e-10 &      \\ 
     &           &    2 &  7.84e-09 &  2.24e-16 &      \\ 
     &           &    3 &  7.83e-09 &  2.24e-16 &      \\ 
     &           &    4 &  7.82e-09 &  2.23e-16 &      \\ 
     &           &    5 &  7.81e-09 &  2.23e-16 &      \\ 
     &           &    6 &  7.80e-09 &  2.23e-16 &      \\ 
     &           &    7 &  7.78e-09 &  2.23e-16 &      \\ 
     &           &    8 &  3.11e-08 &  2.22e-16 &      \\ 
     &           &    9 &  7.82e-09 &  8.87e-16 &      \\ 
     &           &   10 &  7.81e-09 &  2.23e-16 &      \\ 
2789 &  2.79e+04 &   10 &           &           & iters  \\ 
 \hdashline 
     &           &    1 &  3.01e-08 &  6.32e-10 &      \\ 
     &           &    2 &  4.76e-08 &  8.60e-16 &      \\ 
     &           &    3 &  3.02e-08 &  1.36e-15 &      \\ 
     &           &    4 &  4.76e-08 &  8.61e-16 &      \\ 
     &           &    5 &  3.02e-08 &  1.36e-15 &      \\ 
     &           &    6 &  3.02e-08 &  8.62e-16 &      \\ 
     &           &    7 &  4.76e-08 &  8.61e-16 &      \\ 
     &           &    8 &  3.02e-08 &  1.36e-15 &      \\ 
     &           &    9 &  3.01e-08 &  8.61e-16 &      \\ 
     &           &   10 &  4.76e-08 &  8.60e-16 &      \\ 
2790 &  2.79e+04 &   10 &           &           & iters  \\ 
 \hdashline 
     &           &    1 &  2.74e-09 &  8.87e-10 &      \\ 
     &           &    2 &  2.74e-09 &  7.83e-17 &      \\ 
     &           &    3 &  2.74e-09 &  7.82e-17 &      \\ 
     &           &    4 &  2.73e-09 &  7.81e-17 &      \\ 
     &           &    5 &  2.73e-09 &  7.80e-17 &      \\ 
     &           &    6 &  2.72e-09 &  7.78e-17 &      \\ 
     &           &    7 &  2.72e-09 &  7.77e-17 &      \\ 
     &           &    8 &  2.72e-09 &  7.76e-17 &      \\ 
     &           &    9 &  2.71e-09 &  7.75e-17 &      \\ 
     &           &   10 &  2.71e-09 &  7.74e-17 &      \\ 
2791 &  2.79e+04 &   10 &           &           & iters  \\ 
 \hdashline 
     &           &    1 &  1.79e-08 &  7.47e-10 &      \\ 
     &           &    2 &  5.98e-08 &  5.11e-16 &      \\ 
     &           &    3 &  1.79e-08 &  1.71e-15 &      \\ 
     &           &    4 &  1.79e-08 &  5.12e-16 &      \\ 
     &           &    5 &  1.79e-08 &  5.12e-16 &      \\ 
     &           &    6 &  1.79e-08 &  5.11e-16 &      \\ 
     &           &    7 &  5.98e-08 &  5.10e-16 &      \\ 
     &           &    8 &  1.79e-08 &  1.71e-15 &      \\ 
     &           &    9 &  1.79e-08 &  5.12e-16 &      \\ 
     &           &   10 &  1.79e-08 &  5.11e-16 &      \\ 
2792 &  2.79e+04 &   10 &           &           & iters  \\ 
 \hdashline 
     &           &    1 &  9.73e-08 &  3.75e-10 &      \\ 
     &           &    2 &  5.82e-08 &  2.78e-15 &      \\ 
     &           &    3 &  1.96e-08 &  1.66e-15 &      \\ 
     &           &    4 &  1.95e-08 &  5.59e-16 &      \\ 
     &           &    5 &  1.95e-08 &  5.58e-16 &      \\ 
     &           &    6 &  5.82e-08 &  5.57e-16 &      \\ 
     &           &    7 &  1.96e-08 &  1.66e-15 &      \\ 
     &           &    8 &  1.95e-08 &  5.59e-16 &      \\ 
     &           &    9 &  1.95e-08 &  5.58e-16 &      \\ 
     &           &   10 &  5.82e-08 &  5.57e-16 &      \\ 
2793 &  2.79e+04 &   10 &           &           & iters  \\ 
 \hdashline 
     &           &    1 &  2.17e-08 &  9.65e-11 &      \\ 
     &           &    2 &  5.60e-08 &  6.20e-16 &      \\ 
     &           &    3 &  2.18e-08 &  1.60e-15 &      \\ 
     &           &    4 &  2.17e-08 &  6.21e-16 &      \\ 
     &           &    5 &  5.60e-08 &  6.20e-16 &      \\ 
     &           &    6 &  2.18e-08 &  1.60e-15 &      \\ 
     &           &    7 &  2.18e-08 &  6.22e-16 &      \\ 
     &           &    8 &  2.17e-08 &  6.21e-16 &      \\ 
     &           &    9 &  5.60e-08 &  6.20e-16 &      \\ 
     &           &   10 &  2.18e-08 &  1.60e-15 &      \\ 
2794 &  2.79e+04 &   10 &           &           & iters  \\ 
 \hdashline 
     &           &    1 &  2.46e-08 &  2.01e-10 &      \\ 
     &           &    2 &  2.45e-08 &  7.01e-16 &      \\ 
     &           &    3 &  5.32e-08 &  7.00e-16 &      \\ 
     &           &    4 &  2.46e-08 &  1.52e-15 &      \\ 
     &           &    5 &  2.45e-08 &  7.02e-16 &      \\ 
     &           &    6 &  5.32e-08 &  7.01e-16 &      \\ 
     &           &    7 &  2.46e-08 &  1.52e-15 &      \\ 
     &           &    8 &  2.45e-08 &  7.02e-16 &      \\ 
     &           &    9 &  5.32e-08 &  7.01e-16 &      \\ 
     &           &   10 &  2.46e-08 &  1.52e-15 &      \\ 
2795 &  2.79e+04 &   10 &           &           & iters  \\ 
 \hdashline 
     &           &    1 &  1.19e-08 &  3.35e-10 &      \\ 
     &           &    2 &  1.19e-08 &  3.41e-16 &      \\ 
     &           &    3 &  1.19e-08 &  3.40e-16 &      \\ 
     &           &    4 &  1.19e-08 &  3.40e-16 &      \\ 
     &           &    5 &  6.58e-08 &  3.39e-16 &      \\ 
     &           &    6 &  8.96e-08 &  1.88e-15 &      \\ 
     &           &    7 &  6.59e-08 &  2.56e-15 &      \\ 
     &           &    8 &  1.19e-08 &  1.88e-15 &      \\ 
     &           &    9 &  1.19e-08 &  3.40e-16 &      \\ 
     &           &   10 &  1.19e-08 &  3.40e-16 &      \\ 
2796 &  2.80e+04 &   10 &           &           & iters  \\ 
 \hdashline 
     &           &    1 &  3.09e-08 &  4.43e-10 &      \\ 
     &           &    2 &  4.69e-08 &  8.81e-16 &      \\ 
     &           &    3 &  3.09e-08 &  1.34e-15 &      \\ 
     &           &    4 &  4.68e-08 &  8.82e-16 &      \\ 
     &           &    5 &  3.09e-08 &  1.34e-15 &      \\ 
     &           &    6 &  3.09e-08 &  8.82e-16 &      \\ 
     &           &    7 &  4.69e-08 &  8.81e-16 &      \\ 
     &           &    8 &  3.09e-08 &  1.34e-15 &      \\ 
     &           &    9 &  4.68e-08 &  8.82e-16 &      \\ 
     &           &   10 &  3.09e-08 &  1.34e-15 &      \\ 
2797 &  2.80e+04 &   10 &           &           & iters  \\ 
 \hdashline 
     &           &    1 &  2.99e-08 &  5.91e-10 &      \\ 
     &           &    2 &  4.78e-08 &  8.55e-16 &      \\ 
     &           &    3 &  3.00e-08 &  1.36e-15 &      \\ 
     &           &    4 &  4.78e-08 &  8.55e-16 &      \\ 
     &           &    5 &  3.00e-08 &  1.36e-15 &      \\ 
     &           &    6 &  2.99e-08 &  8.56e-16 &      \\ 
     &           &    7 &  4.78e-08 &  8.55e-16 &      \\ 
     &           &    8 &  3.00e-08 &  1.36e-15 &      \\ 
     &           &    9 &  4.78e-08 &  8.56e-16 &      \\ 
     &           &   10 &  3.00e-08 &  1.36e-15 &      \\ 
2798 &  2.80e+04 &   10 &           &           & iters  \\ 
 \hdashline 
     &           &    1 &  1.77e-10 &  5.62e-10 &      \\ 
     &           &    2 &  1.77e-10 &  5.06e-18 &      \\ 
     &           &    3 &  1.77e-10 &  5.05e-18 &      \\ 
     &           &    4 &  1.76e-10 &  5.04e-18 &      \\ 
     &           &    5 &  1.76e-10 &  5.04e-18 &      \\ 
     &           &    6 &  1.76e-10 &  5.03e-18 &      \\ 
     &           &    7 &  1.76e-10 &  5.02e-18 &      \\ 
     &           &    8 &  1.75e-10 &  5.01e-18 &      \\ 
     &           &    9 &  1.75e-10 &  5.01e-18 &      \\ 
     &           &   10 &  1.75e-10 &  5.00e-18 &      \\ 
2799 &  2.80e+04 &   10 &           &           & iters  \\ 
 \hdashline 
     &           &    1 &  3.38e-08 &  3.95e-11 &      \\ 
     &           &    2 &  3.38e-08 &  9.65e-16 &      \\ 
     &           &    3 &  4.40e-08 &  9.64e-16 &      \\ 
     &           &    4 &  3.38e-08 &  1.25e-15 &      \\ 
     &           &    5 &  4.40e-08 &  9.64e-16 &      \\ 
     &           &    6 &  3.38e-08 &  1.25e-15 &      \\ 
     &           &    7 &  4.39e-08 &  9.65e-16 &      \\ 
     &           &    8 &  3.38e-08 &  1.25e-15 &      \\ 
     &           &    9 &  4.39e-08 &  9.65e-16 &      \\ 
     &           &   10 &  3.38e-08 &  1.25e-15 &      \\ 
2800 &  2.80e+04 &   10 &           &           & iters  \\ 
 \hdashline 
     &           &    1 &  5.52e-08 &  7.87e-10 &      \\ 
     &           &    2 &  2.25e-08 &  1.58e-15 &      \\ 
     &           &    3 &  2.25e-08 &  6.43e-16 &      \\ 
     &           &    4 &  2.25e-08 &  6.42e-16 &      \\ 
     &           &    5 &  5.53e-08 &  6.41e-16 &      \\ 
     &           &    6 &  2.25e-08 &  1.58e-15 &      \\ 
     &           &    7 &  2.25e-08 &  6.42e-16 &      \\ 
     &           &    8 &  5.52e-08 &  6.41e-16 &      \\ 
     &           &    9 &  2.25e-08 &  1.58e-15 &      \\ 
     &           &   10 &  2.25e-08 &  6.43e-16 &      \\ 
2801 &  2.80e+04 &   10 &           &           & iters  \\ 
 \hdashline 
     &           &    1 &  1.40e-08 &  6.05e-10 &      \\ 
     &           &    2 &  1.40e-08 &  3.99e-16 &      \\ 
     &           &    3 &  6.37e-08 &  3.99e-16 &      \\ 
     &           &    4 &  1.40e-08 &  1.82e-15 &      \\ 
     &           &    5 &  1.40e-08 &  4.01e-16 &      \\ 
     &           &    6 &  1.40e-08 &  4.00e-16 &      \\ 
     &           &    7 &  1.40e-08 &  4.00e-16 &      \\ 
     &           &    8 &  6.37e-08 &  3.99e-16 &      \\ 
     &           &    9 &  9.17e-08 &  1.82e-15 &      \\ 
     &           &   10 &  6.38e-08 &  2.62e-15 &      \\ 
2802 &  2.80e+04 &   10 &           &           & iters  \\ 
 \hdashline 
     &           &    1 &  5.21e-09 &  2.82e-10 &      \\ 
     &           &    2 &  5.20e-09 &  1.49e-16 &      \\ 
     &           &    3 &  5.20e-09 &  1.49e-16 &      \\ 
     &           &    4 &  5.19e-09 &  1.48e-16 &      \\ 
     &           &    5 &  5.18e-09 &  1.48e-16 &      \\ 
     &           &    6 &  5.17e-09 &  1.48e-16 &      \\ 
     &           &    7 &  5.17e-09 &  1.48e-16 &      \\ 
     &           &    8 &  5.16e-09 &  1.47e-16 &      \\ 
     &           &    9 &  5.15e-09 &  1.47e-16 &      \\ 
     &           &   10 &  7.25e-08 &  1.47e-16 &      \\ 
2803 &  2.80e+04 &   10 &           &           & iters  \\ 
 \hdashline 
     &           &    1 &  5.06e-08 &  5.10e-10 &      \\ 
     &           &    2 &  2.71e-08 &  1.45e-15 &      \\ 
     &           &    3 &  5.06e-08 &  7.74e-16 &      \\ 
     &           &    4 &  2.72e-08 &  1.44e-15 &      \\ 
     &           &    5 &  2.71e-08 &  7.75e-16 &      \\ 
     &           &    6 &  5.06e-08 &  7.74e-16 &      \\ 
     &           &    7 &  2.72e-08 &  1.44e-15 &      \\ 
     &           &    8 &  2.71e-08 &  7.75e-16 &      \\ 
     &           &    9 &  5.06e-08 &  7.74e-16 &      \\ 
     &           &   10 &  2.71e-08 &  1.44e-15 &      \\ 
2804 &  2.80e+04 &   10 &           &           & iters  \\ 
 \hdashline 
     &           &    1 &  1.07e-09 &  6.21e-10 &      \\ 
     &           &    2 &  1.07e-09 &  3.06e-17 &      \\ 
     &           &    3 &  1.07e-09 &  3.05e-17 &      \\ 
     &           &    4 &  1.07e-09 &  3.05e-17 &      \\ 
     &           &    5 &  1.06e-09 &  3.04e-17 &      \\ 
     &           &    6 &  1.06e-09 &  3.04e-17 &      \\ 
     &           &    7 &  1.06e-09 &  3.03e-17 &      \\ 
     &           &    8 &  1.06e-09 &  3.03e-17 &      \\ 
     &           &    9 &  1.06e-09 &  3.02e-17 &      \\ 
     &           &   10 &  1.06e-09 &  3.02e-17 &      \\ 
2805 &  2.80e+04 &   10 &           &           & iters  \\ 
 \hdashline 
     &           &    1 &  2.94e-08 &  1.80e-10 &      \\ 
     &           &    2 &  4.84e-08 &  8.38e-16 &      \\ 
     &           &    3 &  2.94e-08 &  1.38e-15 &      \\ 
     &           &    4 &  4.83e-08 &  8.39e-16 &      \\ 
     &           &    5 &  2.94e-08 &  1.38e-15 &      \\ 
     &           &    6 &  2.94e-08 &  8.40e-16 &      \\ 
     &           &    7 &  4.83e-08 &  8.39e-16 &      \\ 
     &           &    8 &  2.94e-08 &  1.38e-15 &      \\ 
     &           &    9 &  4.83e-08 &  8.39e-16 &      \\ 
     &           &   10 &  2.94e-08 &  1.38e-15 &      \\ 
2806 &  2.80e+04 &   10 &           &           & iters  \\ 
 \hdashline 
     &           &    1 &  7.35e-08 &  4.52e-10 &      \\ 
     &           &    2 &  8.20e-08 &  2.10e-15 &      \\ 
     &           &    3 &  7.35e-08 &  2.34e-15 &      \\ 
     &           &    4 &  4.25e-09 &  2.10e-15 &      \\ 
     &           &    5 &  4.25e-09 &  1.21e-16 &      \\ 
     &           &    6 &  4.24e-09 &  1.21e-16 &      \\ 
     &           &    7 &  4.24e-09 &  1.21e-16 &      \\ 
     &           &    8 &  4.23e-09 &  1.21e-16 &      \\ 
     &           &    9 &  4.22e-09 &  1.21e-16 &      \\ 
     &           &   10 &  4.22e-09 &  1.21e-16 &      \\ 
2807 &  2.81e+04 &   10 &           &           & iters  \\ 
 \hdashline 
     &           &    1 &  1.70e-08 &  3.24e-10 &      \\ 
     &           &    2 &  1.70e-08 &  4.84e-16 &      \\ 
     &           &    3 &  1.69e-08 &  4.84e-16 &      \\ 
     &           &    4 &  6.08e-08 &  4.83e-16 &      \\ 
     &           &    5 &  1.70e-08 &  1.73e-15 &      \\ 
     &           &    6 &  1.70e-08 &  4.85e-16 &      \\ 
     &           &    7 &  1.69e-08 &  4.84e-16 &      \\ 
     &           &    8 &  6.08e-08 &  4.83e-16 &      \\ 
     &           &    9 &  1.70e-08 &  1.73e-15 &      \\ 
     &           &   10 &  1.70e-08 &  4.85e-16 &      \\ 
2808 &  2.81e+04 &   10 &           &           & iters  \\ 
 \hdashline 
     &           &    1 &  3.49e-09 &  8.36e-11 &      \\ 
     &           &    2 &  3.48e-09 &  9.96e-17 &      \\ 
     &           &    3 &  3.48e-09 &  9.95e-17 &      \\ 
     &           &    4 &  3.47e-09 &  9.93e-17 &      \\ 
     &           &    5 &  3.47e-09 &  9.92e-17 &      \\ 
     &           &    6 &  3.47e-09 &  9.90e-17 &      \\ 
     &           &    7 &  3.46e-09 &  9.89e-17 &      \\ 
     &           &    8 &  3.46e-09 &  9.88e-17 &      \\ 
     &           &    9 &  3.45e-09 &  9.86e-17 &      \\ 
     &           &   10 &  3.45e-09 &  9.85e-17 &      \\ 
2809 &  2.81e+04 &   10 &           &           & iters  \\ 
 \hdashline 
     &           &    1 &  1.63e-09 &  9.91e-11 &      \\ 
     &           &    2 &  1.62e-09 &  4.64e-17 &      \\ 
     &           &    3 &  1.62e-09 &  4.64e-17 &      \\ 
     &           &    4 &  1.62e-09 &  4.63e-17 &      \\ 
     &           &    5 &  1.62e-09 &  4.62e-17 &      \\ 
     &           &    6 &  1.61e-09 &  4.62e-17 &      \\ 
     &           &    7 &  1.61e-09 &  4.61e-17 &      \\ 
     &           &    8 &  1.61e-09 &  4.60e-17 &      \\ 
     &           &    9 &  1.61e-09 &  4.60e-17 &      \\ 
     &           &   10 &  1.61e-09 &  4.59e-17 &      \\ 
2810 &  2.81e+04 &   10 &           &           & iters  \\ 
 \hdashline 
     &           &    1 &  2.83e-08 &  3.71e-10 &      \\ 
     &           &    2 &  2.83e-08 &  8.08e-16 &      \\ 
     &           &    3 &  4.95e-08 &  8.07e-16 &      \\ 
     &           &    4 &  2.83e-08 &  1.41e-15 &      \\ 
     &           &    5 &  2.83e-08 &  8.08e-16 &      \\ 
     &           &    6 &  4.95e-08 &  8.07e-16 &      \\ 
     &           &    7 &  2.83e-08 &  1.41e-15 &      \\ 
     &           &    8 &  2.83e-08 &  8.08e-16 &      \\ 
     &           &    9 &  4.95e-08 &  8.06e-16 &      \\ 
     &           &   10 &  2.83e-08 &  1.41e-15 &      \\ 
2811 &  2.81e+04 &   10 &           &           & iters  \\ 
 \hdashline 
     &           &    1 &  5.61e-08 &  4.26e-10 &      \\ 
     &           &    2 &  2.16e-08 &  1.60e-15 &      \\ 
     &           &    3 &  2.16e-08 &  6.17e-16 &      \\ 
     &           &    4 &  2.16e-08 &  6.16e-16 &      \\ 
     &           &    5 &  5.62e-08 &  6.16e-16 &      \\ 
     &           &    6 &  2.16e-08 &  1.60e-15 &      \\ 
     &           &    7 &  2.16e-08 &  6.17e-16 &      \\ 
     &           &    8 &  5.61e-08 &  6.16e-16 &      \\ 
     &           &    9 &  2.16e-08 &  1.60e-15 &      \\ 
     &           &   10 &  2.16e-08 &  6.17e-16 &      \\ 
2812 &  2.81e+04 &   10 &           &           & iters  \\ 
 \hdashline 
     &           &    1 &  3.76e-08 &  5.20e-10 &      \\ 
     &           &    2 &  4.02e-08 &  1.07e-15 &      \\ 
     &           &    3 &  3.76e-08 &  1.15e-15 &      \\ 
     &           &    4 &  4.02e-08 &  1.07e-15 &      \\ 
     &           &    5 &  3.76e-08 &  1.15e-15 &      \\ 
     &           &    6 &  4.02e-08 &  1.07e-15 &      \\ 
     &           &    7 &  3.76e-08 &  1.15e-15 &      \\ 
     &           &    8 &  4.02e-08 &  1.07e-15 &      \\ 
     &           &    9 &  3.76e-08 &  1.15e-15 &      \\ 
     &           &   10 &  4.02e-08 &  1.07e-15 &      \\ 
2813 &  2.81e+04 &   10 &           &           & iters  \\ 
 \hdashline 
     &           &    1 &  3.43e-08 &  2.90e-10 &      \\ 
     &           &    2 &  4.35e-08 &  9.78e-16 &      \\ 
     &           &    3 &  3.43e-08 &  1.24e-15 &      \\ 
     &           &    4 &  4.34e-08 &  9.79e-16 &      \\ 
     &           &    5 &  3.43e-08 &  1.24e-15 &      \\ 
     &           &    6 &  4.34e-08 &  9.79e-16 &      \\ 
     &           &    7 &  3.43e-08 &  1.24e-15 &      \\ 
     &           &    8 &  3.43e-08 &  9.80e-16 &      \\ 
     &           &    9 &  4.35e-08 &  9.78e-16 &      \\ 
     &           &   10 &  3.43e-08 &  1.24e-15 &      \\ 
2814 &  2.81e+04 &   10 &           &           & iters  \\ 
 \hdashline 
     &           &    1 &  3.94e-08 &  1.89e-09 &      \\ 
     &           &    2 &  3.84e-08 &  1.12e-15 &      \\ 
     &           &    3 &  3.94e-08 &  1.09e-15 &      \\ 
     &           &    4 &  3.84e-08 &  1.12e-15 &      \\ 
     &           &    5 &  3.94e-08 &  1.09e-15 &      \\ 
     &           &    6 &  3.84e-08 &  1.12e-15 &      \\ 
     &           &    7 &  3.94e-08 &  1.09e-15 &      \\ 
     &           &    8 &  3.84e-08 &  1.12e-15 &      \\ 
     &           &    9 &  3.94e-08 &  1.10e-15 &      \\ 
     &           &   10 &  3.84e-08 &  1.12e-15 &      \\ 
2815 &  2.81e+04 &   10 &           &           & iters  \\ 
 \hdashline 
     &           &    1 &  2.36e-08 &  1.90e-09 &      \\ 
     &           &    2 &  5.41e-08 &  6.74e-16 &      \\ 
     &           &    3 &  2.37e-08 &  1.54e-15 &      \\ 
     &           &    4 &  2.36e-08 &  6.76e-16 &      \\ 
     &           &    5 &  5.41e-08 &  6.75e-16 &      \\ 
     &           &    6 &  2.37e-08 &  1.54e-15 &      \\ 
     &           &    7 &  2.36e-08 &  6.76e-16 &      \\ 
     &           &    8 &  2.36e-08 &  6.75e-16 &      \\ 
     &           &    9 &  5.41e-08 &  6.74e-16 &      \\ 
     &           &   10 &  2.37e-08 &  1.54e-15 &      \\ 
2816 &  2.82e+04 &   10 &           &           & iters  \\ 
 \hdashline 
     &           &    1 &  5.54e-09 &  2.90e-10 &      \\ 
     &           &    2 &  5.53e-09 &  1.58e-16 &      \\ 
     &           &    3 &  5.52e-09 &  1.58e-16 &      \\ 
     &           &    4 &  5.51e-09 &  1.58e-16 &      \\ 
     &           &    5 &  5.50e-09 &  1.57e-16 &      \\ 
     &           &    6 &  5.50e-09 &  1.57e-16 &      \\ 
     &           &    7 &  5.49e-09 &  1.57e-16 &      \\ 
     &           &    8 &  7.22e-08 &  1.57e-16 &      \\ 
     &           &    9 &  4.44e-08 &  2.06e-15 &      \\ 
     &           &   10 &  5.52e-09 &  1.27e-15 &      \\ 
2817 &  2.82e+04 &   10 &           &           & iters  \\ 
 \hdashline 
     &           &    1 &  2.99e-08 &  7.67e-10 &      \\ 
     &           &    2 &  4.79e-08 &  8.52e-16 &      \\ 
     &           &    3 &  2.99e-08 &  1.37e-15 &      \\ 
     &           &    4 &  2.98e-08 &  8.53e-16 &      \\ 
     &           &    5 &  4.79e-08 &  8.52e-16 &      \\ 
     &           &    6 &  2.99e-08 &  1.37e-15 &      \\ 
     &           &    7 &  4.79e-08 &  8.52e-16 &      \\ 
     &           &    8 &  2.99e-08 &  1.37e-15 &      \\ 
     &           &    9 &  2.98e-08 &  8.53e-16 &      \\ 
     &           &   10 &  4.79e-08 &  8.52e-16 &      \\ 
2818 &  2.82e+04 &   10 &           &           & iters  \\ 
 \hdashline 
     &           &    1 &  4.08e-08 &  5.64e-10 &      \\ 
     &           &    2 &  3.70e-08 &  1.16e-15 &      \\ 
     &           &    3 &  4.08e-08 &  1.06e-15 &      \\ 
     &           &    4 &  3.70e-08 &  1.16e-15 &      \\ 
     &           &    5 &  4.08e-08 &  1.06e-15 &      \\ 
     &           &    6 &  3.70e-08 &  1.16e-15 &      \\ 
     &           &    7 &  4.07e-08 &  1.06e-15 &      \\ 
     &           &    8 &  3.70e-08 &  1.16e-15 &      \\ 
     &           &    9 &  4.07e-08 &  1.06e-15 &      \\ 
     &           &   10 &  3.70e-08 &  1.16e-15 &      \\ 
2819 &  2.82e+04 &   10 &           &           & iters  \\ 
 \hdashline 
     &           &    1 &  1.97e-08 &  1.92e-10 &      \\ 
     &           &    2 &  5.80e-08 &  5.64e-16 &      \\ 
     &           &    3 &  1.98e-08 &  1.65e-15 &      \\ 
     &           &    4 &  1.98e-08 &  5.65e-16 &      \\ 
     &           &    5 &  1.97e-08 &  5.64e-16 &      \\ 
     &           &    6 &  5.80e-08 &  5.64e-16 &      \\ 
     &           &    7 &  1.98e-08 &  1.65e-15 &      \\ 
     &           &    8 &  1.98e-08 &  5.65e-16 &      \\ 
     &           &    9 &  1.97e-08 &  5.64e-16 &      \\ 
     &           &   10 &  5.80e-08 &  5.64e-16 &      \\ 
2820 &  2.82e+04 &   10 &           &           & iters  \\ 
 \hdashline 
     &           &    1 &  5.98e-08 &  1.81e-10 &      \\ 
     &           &    2 &  1.80e-08 &  1.71e-15 &      \\ 
     &           &    3 &  1.80e-08 &  5.14e-16 &      \\ 
     &           &    4 &  1.80e-08 &  5.13e-16 &      \\ 
     &           &    5 &  5.98e-08 &  5.13e-16 &      \\ 
     &           &    6 &  1.80e-08 &  1.71e-15 &      \\ 
     &           &    7 &  1.80e-08 &  5.14e-16 &      \\ 
     &           &    8 &  1.80e-08 &  5.13e-16 &      \\ 
     &           &    9 &  5.97e-08 &  5.13e-16 &      \\ 
     &           &   10 &  1.80e-08 &  1.71e-15 &      \\ 
2821 &  2.82e+04 &   10 &           &           & iters  \\ 
 \hdashline 
     &           &    1 &  5.65e-08 &  1.98e-10 &      \\ 
     &           &    2 &  2.12e-08 &  1.61e-15 &      \\ 
     &           &    3 &  2.12e-08 &  6.06e-16 &      \\ 
     &           &    4 &  5.65e-08 &  6.05e-16 &      \\ 
     &           &    5 &  2.13e-08 &  1.61e-15 &      \\ 
     &           &    6 &  2.12e-08 &  6.07e-16 &      \\ 
     &           &    7 &  2.12e-08 &  6.06e-16 &      \\ 
     &           &    8 &  5.65e-08 &  6.05e-16 &      \\ 
     &           &    9 &  2.13e-08 &  1.61e-15 &      \\ 
     &           &   10 &  2.12e-08 &  6.07e-16 &      \\ 
2822 &  2.82e+04 &   10 &           &           & iters  \\ 
 \hdashline 
     &           &    1 &  6.37e-09 &  5.85e-10 &      \\ 
     &           &    2 &  6.36e-09 &  1.82e-16 &      \\ 
     &           &    3 &  6.35e-09 &  1.81e-16 &      \\ 
     &           &    4 &  6.34e-09 &  1.81e-16 &      \\ 
     &           &    5 &  7.14e-08 &  1.81e-16 &      \\ 
     &           &    6 &  8.41e-08 &  2.04e-15 &      \\ 
     &           &    7 &  7.14e-08 &  2.40e-15 &      \\ 
     &           &    8 &  6.41e-09 &  2.04e-15 &      \\ 
     &           &    9 &  6.40e-09 &  1.83e-16 &      \\ 
     &           &   10 &  6.39e-09 &  1.83e-16 &      \\ 
2823 &  2.82e+04 &   10 &           &           & iters  \\ 
 \hdashline 
     &           &    1 &  3.95e-09 &  7.86e-10 &      \\ 
     &           &    2 &  3.94e-09 &  1.13e-16 &      \\ 
     &           &    3 &  3.94e-09 &  1.13e-16 &      \\ 
     &           &    4 &  3.93e-09 &  1.12e-16 &      \\ 
     &           &    5 &  3.93e-09 &  1.12e-16 &      \\ 
     &           &    6 &  3.92e-09 &  1.12e-16 &      \\ 
     &           &    7 &  3.92e-09 &  1.12e-16 &      \\ 
     &           &    8 &  3.91e-09 &  1.12e-16 &      \\ 
     &           &    9 &  3.91e-09 &  1.12e-16 &      \\ 
     &           &   10 &  7.38e-08 &  1.11e-16 &      \\ 
2824 &  2.82e+04 &   10 &           &           & iters  \\ 
 \hdashline 
     &           &    1 &  1.96e-08 &  5.95e-10 &      \\ 
     &           &    2 &  1.96e-08 &  5.59e-16 &      \\ 
     &           &    3 &  1.95e-08 &  5.59e-16 &      \\ 
     &           &    4 &  5.82e-08 &  5.58e-16 &      \\ 
     &           &    5 &  1.96e-08 &  1.66e-15 &      \\ 
     &           &    6 &  1.96e-08 &  5.59e-16 &      \\ 
     &           &    7 &  1.95e-08 &  5.59e-16 &      \\ 
     &           &    8 &  5.82e-08 &  5.58e-16 &      \\ 
     &           &    9 &  1.96e-08 &  1.66e-15 &      \\ 
     &           &   10 &  1.96e-08 &  5.59e-16 &      \\ 
2825 &  2.82e+04 &   10 &           &           & iters  \\ 
 \hdashline 
     &           &    1 &  3.28e-08 &  5.45e-10 &      \\ 
     &           &    2 &  4.49e-08 &  9.36e-16 &      \\ 
     &           &    3 &  3.28e-08 &  1.28e-15 &      \\ 
     &           &    4 &  4.49e-08 &  9.36e-16 &      \\ 
     &           &    5 &  3.28e-08 &  1.28e-15 &      \\ 
     &           &    6 &  3.28e-08 &  9.37e-16 &      \\ 
     &           &    7 &  4.50e-08 &  9.35e-16 &      \\ 
     &           &    8 &  3.28e-08 &  1.28e-15 &      \\ 
     &           &    9 &  4.49e-08 &  9.36e-16 &      \\ 
     &           &   10 &  3.28e-08 &  1.28e-15 &      \\ 
2826 &  2.82e+04 &   10 &           &           & iters  \\ 
 \hdashline 
     &           &    1 &  1.79e-08 &  5.41e-10 &      \\ 
     &           &    2 &  5.98e-08 &  5.11e-16 &      \\ 
     &           &    3 &  1.79e-08 &  1.71e-15 &      \\ 
     &           &    4 &  1.79e-08 &  5.12e-16 &      \\ 
     &           &    5 &  1.79e-08 &  5.12e-16 &      \\ 
     &           &    6 &  5.98e-08 &  5.11e-16 &      \\ 
     &           &    7 &  1.80e-08 &  1.71e-15 &      \\ 
     &           &    8 &  1.79e-08 &  5.13e-16 &      \\ 
     &           &    9 &  1.79e-08 &  5.12e-16 &      \\ 
     &           &   10 &  5.98e-08 &  5.11e-16 &      \\ 
2827 &  2.83e+04 &   10 &           &           & iters  \\ 
 \hdashline 
     &           &    1 &  3.29e-08 &  1.49e-09 &      \\ 
     &           &    2 &  5.98e-09 &  9.39e-16 &      \\ 
     &           &    3 &  5.98e-09 &  1.71e-16 &      \\ 
     &           &    4 &  5.97e-09 &  1.71e-16 &      \\ 
     &           &    5 &  5.96e-09 &  1.70e-16 &      \\ 
     &           &    6 &  5.95e-09 &  1.70e-16 &      \\ 
     &           &    7 &  3.29e-08 &  1.70e-16 &      \\ 
     &           &    8 &  5.99e-09 &  9.39e-16 &      \\ 
     &           &    9 &  5.98e-09 &  1.71e-16 &      \\ 
     &           &   10 &  5.97e-09 &  1.71e-16 &      \\ 
2828 &  2.83e+04 &   10 &           &           & iters  \\ 
 \hdashline 
     &           &    1 &  2.76e-08 &  1.05e-09 &      \\ 
     &           &    2 &  5.01e-08 &  7.88e-16 &      \\ 
     &           &    3 &  2.76e-08 &  1.43e-15 &      \\ 
     &           &    4 &  5.01e-08 &  7.89e-16 &      \\ 
     &           &    5 &  2.77e-08 &  1.43e-15 &      \\ 
     &           &    6 &  2.76e-08 &  7.90e-16 &      \\ 
     &           &    7 &  5.01e-08 &  7.89e-16 &      \\ 
     &           &    8 &  2.77e-08 &  1.43e-15 &      \\ 
     &           &    9 &  2.76e-08 &  7.90e-16 &      \\ 
     &           &   10 &  5.01e-08 &  7.89e-16 &      \\ 
2829 &  2.83e+04 &   10 &           &           & iters  \\ 
 \hdashline 
     &           &    1 &  5.92e-08 &  8.43e-12 &      \\ 
     &           &    2 &  1.86e-08 &  1.69e-15 &      \\ 
     &           &    3 &  1.86e-08 &  5.30e-16 &      \\ 
     &           &    4 &  1.85e-08 &  5.29e-16 &      \\ 
     &           &    5 &  5.92e-08 &  5.29e-16 &      \\ 
     &           &    6 &  1.86e-08 &  1.69e-15 &      \\ 
     &           &    7 &  1.86e-08 &  5.30e-16 &      \\ 
     &           &    8 &  1.85e-08 &  5.30e-16 &      \\ 
     &           &    9 &  5.92e-08 &  5.29e-16 &      \\ 
     &           &   10 &  1.86e-08 &  1.69e-15 &      \\ 
2830 &  2.83e+04 &   10 &           &           & iters  \\ 
 \hdashline 
     &           &    1 &  1.13e-08 &  1.01e-09 &      \\ 
     &           &    2 &  1.13e-08 &  3.23e-16 &      \\ 
     &           &    3 &  1.13e-08 &  3.23e-16 &      \\ 
     &           &    4 &  1.13e-08 &  3.22e-16 &      \\ 
     &           &    5 &  1.13e-08 &  3.22e-16 &      \\ 
     &           &    6 &  6.64e-08 &  3.21e-16 &      \\ 
     &           &    7 &  5.02e-08 &  1.90e-15 &      \\ 
     &           &    8 &  1.13e-08 &  1.43e-15 &      \\ 
     &           &    9 &  1.13e-08 &  3.22e-16 &      \\ 
     &           &   10 &  6.64e-08 &  3.21e-16 &      \\ 
2831 &  2.83e+04 &   10 &           &           & iters  \\ 
 \hdashline 
     &           &    1 &  2.64e-08 &  9.44e-10 &      \\ 
     &           &    2 &  5.13e-08 &  7.53e-16 &      \\ 
     &           &    3 &  2.64e-08 &  1.46e-15 &      \\ 
     &           &    4 &  2.64e-08 &  7.54e-16 &      \\ 
     &           &    5 &  5.13e-08 &  7.53e-16 &      \\ 
     &           &    6 &  2.64e-08 &  1.46e-15 &      \\ 
     &           &    7 &  2.64e-08 &  7.54e-16 &      \\ 
     &           &    8 &  5.13e-08 &  7.53e-16 &      \\ 
     &           &    9 &  2.64e-08 &  1.47e-15 &      \\ 
     &           &   10 &  2.64e-08 &  7.54e-16 &      \\ 
2832 &  2.83e+04 &   10 &           &           & iters  \\ 
 \hdashline 
     &           &    1 &  1.96e-08 &  4.60e-10 &      \\ 
     &           &    2 &  1.96e-08 &  5.60e-16 &      \\ 
     &           &    3 &  5.81e-08 &  5.59e-16 &      \\ 
     &           &    4 &  1.96e-08 &  1.66e-15 &      \\ 
     &           &    5 &  1.96e-08 &  5.60e-16 &      \\ 
     &           &    6 &  1.96e-08 &  5.60e-16 &      \\ 
     &           &    7 &  5.81e-08 &  5.59e-16 &      \\ 
     &           &    8 &  1.96e-08 &  1.66e-15 &      \\ 
     &           &    9 &  1.96e-08 &  5.60e-16 &      \\ 
     &           &   10 &  1.96e-08 &  5.59e-16 &      \\ 
2833 &  2.83e+04 &   10 &           &           & iters  \\ 
 \hdashline 
     &           &    1 &  6.13e-08 &  5.73e-10 &      \\ 
     &           &    2 &  1.65e-08 &  1.75e-15 &      \\ 
     &           &    3 &  1.64e-08 &  4.70e-16 &      \\ 
     &           &    4 &  1.64e-08 &  4.69e-16 &      \\ 
     &           &    5 &  1.64e-08 &  4.68e-16 &      \\ 
     &           &    6 &  6.13e-08 &  4.68e-16 &      \\ 
     &           &    7 &  1.64e-08 &  1.75e-15 &      \\ 
     &           &    8 &  1.64e-08 &  4.69e-16 &      \\ 
     &           &    9 &  1.64e-08 &  4.69e-16 &      \\ 
     &           &   10 &  1.64e-08 &  4.68e-16 &      \\ 
2834 &  2.83e+04 &   10 &           &           & iters  \\ 
 \hdashline 
     &           &    1 &  7.53e-08 &  6.20e-10 &      \\ 
     &           &    2 &  2.49e-09 &  2.15e-15 &      \\ 
     &           &    3 &  2.49e-09 &  7.11e-17 &      \\ 
     &           &    4 &  2.48e-09 &  7.10e-17 &      \\ 
     &           &    5 &  2.48e-09 &  7.09e-17 &      \\ 
     &           &    6 &  2.48e-09 &  7.08e-17 &      \\ 
     &           &    7 &  2.47e-09 &  7.07e-17 &      \\ 
     &           &    8 &  2.47e-09 &  7.06e-17 &      \\ 
     &           &    9 &  2.47e-09 &  7.05e-17 &      \\ 
     &           &   10 &  2.46e-09 &  7.04e-17 &      \\ 
2835 &  2.83e+04 &   10 &           &           & iters  \\ 
 \hdashline 
     &           &    1 &  5.45e-08 &  6.60e-10 &      \\ 
     &           &    2 &  2.33e-08 &  1.56e-15 &      \\ 
     &           &    3 &  2.32e-08 &  6.64e-16 &      \\ 
     &           &    4 &  2.32e-08 &  6.63e-16 &      \\ 
     &           &    5 &  5.45e-08 &  6.62e-16 &      \\ 
     &           &    6 &  2.32e-08 &  1.56e-15 &      \\ 
     &           &    7 &  2.32e-08 &  6.63e-16 &      \\ 
     &           &    8 &  5.45e-08 &  6.62e-16 &      \\ 
     &           &    9 &  2.33e-08 &  1.56e-15 &      \\ 
     &           &   10 &  2.32e-08 &  6.64e-16 &      \\ 
2836 &  2.84e+04 &   10 &           &           & iters  \\ 
 \hdashline 
     &           &    1 &  2.12e-08 &  1.06e-09 &      \\ 
     &           &    2 &  2.11e-08 &  6.04e-16 &      \\ 
     &           &    3 &  5.66e-08 &  6.03e-16 &      \\ 
     &           &    4 &  2.12e-08 &  1.62e-15 &      \\ 
     &           &    5 &  2.11e-08 &  6.04e-16 &      \\ 
     &           &    6 &  2.11e-08 &  6.04e-16 &      \\ 
     &           &    7 &  5.66e-08 &  6.03e-16 &      \\ 
     &           &    8 &  2.12e-08 &  1.62e-15 &      \\ 
     &           &    9 &  2.11e-08 &  6.04e-16 &      \\ 
     &           &   10 &  5.66e-08 &  6.03e-16 &      \\ 
2837 &  2.84e+04 &   10 &           &           & iters  \\ 
 \hdashline 
     &           &    1 &  1.10e-10 &  5.70e-10 &      \\ 
     &           &    2 &  1.10e-10 &  3.14e-18 &      \\ 
     &           &    3 &  1.10e-10 &  3.14e-18 &      \\ 
     &           &    4 &  1.10e-10 &  3.13e-18 &      \\ 
     &           &    5 &  1.09e-10 &  3.13e-18 &      \\ 
     &           &    6 &  1.09e-10 &  3.12e-18 &      \\ 
     &           &    7 &  1.09e-10 &  3.12e-18 &      \\ 
     &           &    8 &  1.09e-10 &  3.11e-18 &      \\ 
     &           &    9 &  1.09e-10 &  3.11e-18 &      \\ 
     &           &   10 &  1.09e-10 &  3.11e-18 &      \\ 
2838 &  2.84e+04 &   10 &           &           & iters  \\ 
 \hdashline 
     &           &    1 &  6.41e-08 &  1.23e-09 &      \\ 
     &           &    2 &  1.37e-08 &  1.83e-15 &      \\ 
     &           &    3 &  1.37e-08 &  3.91e-16 &      \\ 
     &           &    4 &  1.37e-08 &  3.90e-16 &      \\ 
     &           &    5 &  6.40e-08 &  3.90e-16 &      \\ 
     &           &    6 &  1.37e-08 &  1.83e-15 &      \\ 
     &           &    7 &  1.37e-08 &  3.92e-16 &      \\ 
     &           &    8 &  1.37e-08 &  3.91e-16 &      \\ 
     &           &    9 &  1.37e-08 &  3.91e-16 &      \\ 
     &           &   10 &  1.37e-08 &  3.90e-16 &      \\ 
2839 &  2.84e+04 &   10 &           &           & iters  \\ 
 \hdashline 
     &           &    1 &  5.19e-08 &  1.67e-09 &      \\ 
     &           &    2 &  2.59e-08 &  1.48e-15 &      \\ 
     &           &    3 &  2.58e-08 &  7.38e-16 &      \\ 
     &           &    4 &  5.19e-08 &  7.37e-16 &      \\ 
     &           &    5 &  2.59e-08 &  1.48e-15 &      \\ 
     &           &    6 &  2.58e-08 &  7.38e-16 &      \\ 
     &           &    7 &  5.19e-08 &  7.37e-16 &      \\ 
     &           &    8 &  2.59e-08 &  1.48e-15 &      \\ 
     &           &    9 &  2.58e-08 &  7.38e-16 &      \\ 
     &           &   10 &  5.19e-08 &  7.37e-16 &      \\ 
2840 &  2.84e+04 &   10 &           &           & iters  \\ 
 \hdashline 
     &           &    1 &  1.04e-08 &  6.78e-10 &      \\ 
     &           &    2 &  1.03e-08 &  2.95e-16 &      \\ 
     &           &    3 &  1.03e-08 &  2.95e-16 &      \\ 
     &           &    4 &  1.03e-08 &  2.95e-16 &      \\ 
     &           &    5 &  1.03e-08 &  2.94e-16 &      \\ 
     &           &    6 &  6.74e-08 &  2.94e-16 &      \\ 
     &           &    7 &  8.81e-08 &  1.92e-15 &      \\ 
     &           &    8 &  6.74e-08 &  2.51e-15 &      \\ 
     &           &    9 &  1.03e-08 &  1.92e-15 &      \\ 
     &           &   10 &  1.03e-08 &  2.95e-16 &      \\ 
2841 &  2.84e+04 &   10 &           &           & iters  \\ 
 \hdashline 
     &           &    1 &  5.76e-09 &  7.39e-10 &      \\ 
     &           &    2 &  5.76e-09 &  1.65e-16 &      \\ 
     &           &    3 &  7.19e-08 &  1.64e-16 &      \\ 
     &           &    4 &  8.35e-08 &  2.05e-15 &      \\ 
     &           &    5 &  7.20e-08 &  2.38e-15 &      \\ 
     &           &    6 &  5.83e-09 &  2.05e-15 &      \\ 
     &           &    7 &  5.83e-09 &  1.67e-16 &      \\ 
     &           &    8 &  5.82e-09 &  1.66e-16 &      \\ 
     &           &    9 &  5.81e-09 &  1.66e-16 &      \\ 
     &           &   10 &  5.80e-09 &  1.66e-16 &      \\ 
2842 &  2.84e+04 &   10 &           &           & iters  \\ 
 \hdashline 
     &           &    1 &  4.05e-09 &  1.39e-09 &      \\ 
     &           &    2 &  4.04e-09 &  1.16e-16 &      \\ 
     &           &    3 &  4.04e-09 &  1.15e-16 &      \\ 
     &           &    4 &  4.03e-09 &  1.15e-16 &      \\ 
     &           &    5 &  3.48e-08 &  1.15e-16 &      \\ 
     &           &    6 &  4.08e-09 &  9.94e-16 &      \\ 
     &           &    7 &  4.07e-09 &  1.16e-16 &      \\ 
     &           &    8 &  4.06e-09 &  1.16e-16 &      \\ 
     &           &    9 &  4.06e-09 &  1.16e-16 &      \\ 
     &           &   10 &  4.05e-09 &  1.16e-16 &      \\ 
2843 &  2.84e+04 &   10 &           &           & iters  \\ 
 \hdashline 
     &           &    1 &  2.55e-08 &  1.37e-09 &      \\ 
     &           &    2 &  1.34e-08 &  7.28e-16 &      \\ 
     &           &    3 &  1.33e-08 &  3.81e-16 &      \\ 
     &           &    4 &  2.55e-08 &  3.81e-16 &      \\ 
     &           &    5 &  1.34e-08 &  7.29e-16 &      \\ 
     &           &    6 &  1.33e-08 &  3.81e-16 &      \\ 
     &           &    7 &  2.55e-08 &  3.81e-16 &      \\ 
     &           &    8 &  1.34e-08 &  7.29e-16 &      \\ 
     &           &    9 &  1.33e-08 &  3.81e-16 &      \\ 
     &           &   10 &  2.55e-08 &  3.81e-16 &      \\ 
2844 &  2.84e+04 &   10 &           &           & iters  \\ 
 \hdashline 
     &           &    1 &  2.32e-08 &  1.82e-10 &      \\ 
     &           &    2 &  2.31e-08 &  6.62e-16 &      \\ 
     &           &    3 &  5.46e-08 &  6.61e-16 &      \\ 
     &           &    4 &  2.32e-08 &  1.56e-15 &      \\ 
     &           &    5 &  2.32e-08 &  6.62e-16 &      \\ 
     &           &    6 &  2.31e-08 &  6.61e-16 &      \\ 
     &           &    7 &  5.46e-08 &  6.60e-16 &      \\ 
     &           &    8 &  2.32e-08 &  1.56e-15 &      \\ 
     &           &    9 &  2.31e-08 &  6.61e-16 &      \\ 
     &           &   10 &  5.46e-08 &  6.60e-16 &      \\ 
2845 &  2.84e+04 &   10 &           &           & iters  \\ 
 \hdashline 
     &           &    1 &  9.42e-09 &  9.44e-10 &      \\ 
     &           &    2 &  9.41e-09 &  2.69e-16 &      \\ 
     &           &    3 &  9.40e-09 &  2.69e-16 &      \\ 
     &           &    4 &  9.38e-09 &  2.68e-16 &      \\ 
     &           &    5 &  9.37e-09 &  2.68e-16 &      \\ 
     &           &    6 &  6.83e-08 &  2.67e-16 &      \\ 
     &           &    7 &  9.45e-09 &  1.95e-15 &      \\ 
     &           &    8 &  9.44e-09 &  2.70e-16 &      \\ 
     &           &    9 &  9.43e-09 &  2.69e-16 &      \\ 
     &           &   10 &  9.41e-09 &  2.69e-16 &      \\ 
2846 &  2.84e+04 &   10 &           &           & iters  \\ 
 \hdashline 
     &           &    1 &  1.51e-08 &  4.56e-10 &      \\ 
     &           &    2 &  1.51e-08 &  4.31e-16 &      \\ 
     &           &    3 &  6.26e-08 &  4.30e-16 &      \\ 
     &           &    4 &  5.40e-08 &  1.79e-15 &      \\ 
     &           &    5 &  1.51e-08 &  1.54e-15 &      \\ 
     &           &    6 &  6.26e-08 &  4.30e-16 &      \\ 
     &           &    7 &  1.51e-08 &  1.79e-15 &      \\ 
     &           &    8 &  1.51e-08 &  4.32e-16 &      \\ 
     &           &    9 &  1.51e-08 &  4.31e-16 &      \\ 
     &           &   10 &  1.51e-08 &  4.31e-16 &      \\ 
2847 &  2.85e+04 &   10 &           &           & iters  \\ 
 \hdashline 
     &           &    1 &  8.49e-09 &  2.03e-11 &      \\ 
     &           &    2 &  8.48e-09 &  2.42e-16 &      \\ 
     &           &    3 &  8.46e-09 &  2.42e-16 &      \\ 
     &           &    4 &  8.45e-09 &  2.42e-16 &      \\ 
     &           &    5 &  8.44e-09 &  2.41e-16 &      \\ 
     &           &    6 &  8.43e-09 &  2.41e-16 &      \\ 
     &           &    7 &  8.41e-09 &  2.41e-16 &      \\ 
     &           &    8 &  3.04e-08 &  2.40e-16 &      \\ 
     &           &    9 &  8.45e-09 &  8.69e-16 &      \\ 
     &           &   10 &  8.43e-09 &  2.41e-16 &      \\ 
2848 &  2.85e+04 &   10 &           &           & iters  \\ 
 \hdashline 
     &           &    1 &  1.25e-08 &  4.26e-10 &      \\ 
     &           &    2 &  2.64e-08 &  3.57e-16 &      \\ 
     &           &    3 &  1.25e-08 &  7.52e-16 &      \\ 
     &           &    4 &  1.25e-08 &  3.58e-16 &      \\ 
     &           &    5 &  1.25e-08 &  3.57e-16 &      \\ 
     &           &    6 &  2.64e-08 &  3.57e-16 &      \\ 
     &           &    7 &  1.25e-08 &  7.53e-16 &      \\ 
     &           &    8 &  1.25e-08 &  3.57e-16 &      \\ 
     &           &    9 &  2.64e-08 &  3.57e-16 &      \\ 
     &           &   10 &  1.25e-08 &  7.53e-16 &      \\ 
2849 &  2.85e+04 &   10 &           &           & iters  \\ 
 \hdashline 
     &           &    1 &  9.81e-09 &  6.55e-10 &      \\ 
     &           &    2 &  2.91e-08 &  2.80e-16 &      \\ 
     &           &    3 &  9.83e-09 &  8.29e-16 &      \\ 
     &           &    4 &  9.82e-09 &  2.81e-16 &      \\ 
     &           &    5 &  9.80e-09 &  2.80e-16 &      \\ 
     &           &    6 &  2.91e-08 &  2.80e-16 &      \\ 
     &           &    7 &  9.83e-09 &  8.29e-16 &      \\ 
     &           &    8 &  9.82e-09 &  2.81e-16 &      \\ 
     &           &    9 &  9.80e-09 &  2.80e-16 &      \\ 
     &           &   10 &  2.91e-08 &  2.80e-16 &      \\ 
2850 &  2.85e+04 &   10 &           &           & iters  \\ 
 \hdashline 
     &           &    1 &  5.99e-09 &  7.23e-10 &      \\ 
     &           &    2 &  5.98e-09 &  1.71e-16 &      \\ 
     &           &    3 &  5.97e-09 &  1.71e-16 &      \\ 
     &           &    4 &  5.96e-09 &  1.70e-16 &      \\ 
     &           &    5 &  5.96e-09 &  1.70e-16 &      \\ 
     &           &    6 &  5.95e-09 &  1.70e-16 &      \\ 
     &           &    7 &  5.94e-09 &  1.70e-16 &      \\ 
     &           &    8 &  5.93e-09 &  1.70e-16 &      \\ 
     &           &    9 &  7.18e-08 &  1.69e-16 &      \\ 
     &           &   10 &  8.37e-08 &  2.05e-15 &      \\ 
2851 &  2.85e+04 &   10 &           &           & iters  \\ 
 \hdashline 
     &           &    1 &  4.57e-08 &  7.99e-10 &      \\ 
     &           &    2 &  3.21e-08 &  1.30e-15 &      \\ 
     &           &    3 &  4.57e-08 &  9.15e-16 &      \\ 
     &           &    4 &  3.21e-08 &  1.30e-15 &      \\ 
     &           &    5 &  4.57e-08 &  9.16e-16 &      \\ 
     &           &    6 &  3.21e-08 &  1.30e-15 &      \\ 
     &           &    7 &  3.21e-08 &  9.16e-16 &      \\ 
     &           &    8 &  4.57e-08 &  9.15e-16 &      \\ 
     &           &    9 &  3.21e-08 &  1.30e-15 &      \\ 
     &           &   10 &  4.57e-08 &  9.15e-16 &      \\ 
2852 &  2.85e+04 &   10 &           &           & iters  \\ 
 \hdashline 
     &           &    1 &  2.09e-09 &  2.76e-10 &      \\ 
     &           &    2 &  2.09e-09 &  5.96e-17 &      \\ 
     &           &    3 &  7.56e-08 &  5.95e-17 &      \\ 
     &           &    4 &  7.99e-08 &  2.16e-15 &      \\ 
     &           &    5 &  7.56e-08 &  2.28e-15 &      \\ 
     &           &    6 &  7.99e-08 &  2.16e-15 &      \\ 
     &           &    7 &  7.56e-08 &  2.28e-15 &      \\ 
     &           &    8 &  7.99e-08 &  2.16e-15 &      \\ 
     &           &    9 &  7.56e-08 &  2.28e-15 &      \\ 
     &           &   10 &  2.17e-09 &  2.16e-15 &      \\ 
2853 &  2.85e+04 &   10 &           &           & iters  \\ 
 \hdashline 
     &           &    1 &  1.11e-08 &  6.67e-10 &      \\ 
     &           &    2 &  1.10e-08 &  3.15e-16 &      \\ 
     &           &    3 &  1.10e-08 &  3.15e-16 &      \\ 
     &           &    4 &  1.10e-08 &  3.14e-16 &      \\ 
     &           &    5 &  6.67e-08 &  3.14e-16 &      \\ 
     &           &    6 &  1.11e-08 &  1.90e-15 &      \\ 
     &           &    7 &  1.11e-08 &  3.16e-16 &      \\ 
     &           &    8 &  1.11e-08 &  3.16e-16 &      \\ 
     &           &    9 &  1.10e-08 &  3.15e-16 &      \\ 
     &           &   10 &  1.10e-08 &  3.15e-16 &      \\ 
2854 &  2.85e+04 &   10 &           &           & iters  \\ 
 \hdashline 
     &           &    1 &  1.24e-08 &  7.61e-10 &      \\ 
     &           &    2 &  1.24e-08 &  3.54e-16 &      \\ 
     &           &    3 &  2.65e-08 &  3.54e-16 &      \\ 
     &           &    4 &  1.24e-08 &  7.55e-16 &      \\ 
     &           &    5 &  1.24e-08 &  3.54e-16 &      \\ 
     &           &    6 &  2.65e-08 &  3.54e-16 &      \\ 
     &           &    7 &  1.24e-08 &  7.55e-16 &      \\ 
     &           &    8 &  1.24e-08 &  3.55e-16 &      \\ 
     &           &    9 &  2.65e-08 &  3.54e-16 &      \\ 
     &           &   10 &  1.24e-08 &  7.55e-16 &      \\ 
2855 &  2.85e+04 &   10 &           &           & iters  \\ 
 \hdashline 
     &           &    1 &  2.76e-08 &  4.30e-10 &      \\ 
     &           &    2 &  5.01e-08 &  7.89e-16 &      \\ 
     &           &    3 &  2.77e-08 &  1.43e-15 &      \\ 
     &           &    4 &  2.76e-08 &  7.90e-16 &      \\ 
     &           &    5 &  5.01e-08 &  7.88e-16 &      \\ 
     &           &    6 &  2.77e-08 &  1.43e-15 &      \\ 
     &           &    7 &  5.01e-08 &  7.89e-16 &      \\ 
     &           &    8 &  2.77e-08 &  1.43e-15 &      \\ 
     &           &    9 &  2.76e-08 &  7.90e-16 &      \\ 
     &           &   10 &  5.01e-08 &  7.89e-16 &      \\ 
2856 &  2.86e+04 &   10 &           &           & iters  \\ 
 \hdashline 
     &           &    1 &  6.01e-08 &  7.21e-12 &      \\ 
     &           &    2 &  1.76e-08 &  1.72e-15 &      \\ 
     &           &    3 &  1.76e-08 &  5.03e-16 &      \\ 
     &           &    4 &  1.76e-08 &  5.03e-16 &      \\ 
     &           &    5 &  6.01e-08 &  5.02e-16 &      \\ 
     &           &    6 &  1.76e-08 &  1.72e-15 &      \\ 
     &           &    7 &  1.76e-08 &  5.04e-16 &      \\ 
     &           &    8 &  1.76e-08 &  5.03e-16 &      \\ 
     &           &    9 &  1.76e-08 &  5.02e-16 &      \\ 
     &           &   10 &  6.01e-08 &  5.02e-16 &      \\ 
2857 &  2.86e+04 &   10 &           &           & iters  \\ 
 \hdashline 
     &           &    1 &  3.97e-08 &  5.56e-10 &      \\ 
     &           &    2 &  3.80e-08 &  1.13e-15 &      \\ 
     &           &    3 &  3.97e-08 &  1.09e-15 &      \\ 
     &           &    4 &  3.80e-08 &  1.13e-15 &      \\ 
     &           &    5 &  3.97e-08 &  1.09e-15 &      \\ 
     &           &    6 &  3.80e-08 &  1.13e-15 &      \\ 
     &           &    7 &  3.97e-08 &  1.09e-15 &      \\ 
     &           &    8 &  3.80e-08 &  1.13e-15 &      \\ 
     &           &    9 &  3.80e-08 &  1.09e-15 &      \\ 
     &           &   10 &  3.98e-08 &  1.08e-15 &      \\ 
2858 &  2.86e+04 &   10 &           &           & iters  \\ 
 \hdashline 
     &           &    1 &  1.67e-08 &  2.85e-10 &      \\ 
     &           &    2 &  1.67e-08 &  4.77e-16 &      \\ 
     &           &    3 &  6.10e-08 &  4.77e-16 &      \\ 
     &           &    4 &  1.68e-08 &  1.74e-15 &      \\ 
     &           &    5 &  1.67e-08 &  4.79e-16 &      \\ 
     &           &    6 &  1.67e-08 &  4.78e-16 &      \\ 
     &           &    7 &  1.67e-08 &  4.77e-16 &      \\ 
     &           &    8 &  6.10e-08 &  4.77e-16 &      \\ 
     &           &    9 &  1.68e-08 &  1.74e-15 &      \\ 
     &           &   10 &  1.67e-08 &  4.78e-16 &      \\ 
2859 &  2.86e+04 &   10 &           &           & iters  \\ 
 \hdashline 
     &           &    1 &  1.05e-08 &  2.45e-11 &      \\ 
     &           &    2 &  1.05e-08 &  2.99e-16 &      \\ 
     &           &    3 &  6.72e-08 &  2.99e-16 &      \\ 
     &           &    4 &  8.82e-08 &  1.92e-15 &      \\ 
     &           &    5 &  6.73e-08 &  2.52e-15 &      \\ 
     &           &    6 &  1.05e-08 &  1.92e-15 &      \\ 
     &           &    7 &  1.05e-08 &  3.00e-16 &      \\ 
     &           &    8 &  1.05e-08 &  3.00e-16 &      \\ 
     &           &    9 &  1.05e-08 &  3.00e-16 &      \\ 
     &           &   10 &  1.05e-08 &  2.99e-16 &      \\ 
2860 &  2.86e+04 &   10 &           &           & iters  \\ 
 \hdashline 
     &           &    1 &  1.56e-08 &  1.38e-10 &      \\ 
     &           &    2 &  1.56e-08 &  4.46e-16 &      \\ 
     &           &    3 &  6.21e-08 &  4.46e-16 &      \\ 
     &           &    4 &  1.57e-08 &  1.77e-15 &      \\ 
     &           &    5 &  1.57e-08 &  4.48e-16 &      \\ 
     &           &    6 &  1.56e-08 &  4.47e-16 &      \\ 
     &           &    7 &  1.56e-08 &  4.46e-16 &      \\ 
     &           &    8 &  6.21e-08 &  4.46e-16 &      \\ 
     &           &    9 &  1.57e-08 &  1.77e-15 &      \\ 
     &           &   10 &  1.57e-08 &  4.48e-16 &      \\ 
2861 &  2.86e+04 &   10 &           &           & iters  \\ 
 \hdashline 
     &           &    1 &  2.57e-08 &  1.22e-10 &      \\ 
     &           &    2 &  5.20e-08 &  7.34e-16 &      \\ 
     &           &    3 &  2.58e-08 &  1.48e-15 &      \\ 
     &           &    4 &  2.57e-08 &  7.35e-16 &      \\ 
     &           &    5 &  5.20e-08 &  7.34e-16 &      \\ 
     &           &    6 &  2.58e-08 &  1.48e-15 &      \\ 
     &           &    7 &  2.57e-08 &  7.35e-16 &      \\ 
     &           &    8 &  5.20e-08 &  7.34e-16 &      \\ 
     &           &    9 &  2.58e-08 &  1.48e-15 &      \\ 
     &           &   10 &  2.57e-08 &  7.35e-16 &      \\ 
2862 &  2.86e+04 &   10 &           &           & iters  \\ 
 \hdashline 
     &           &    1 &  3.55e-08 &  4.20e-10 &      \\ 
     &           &    2 &  3.55e-08 &  1.01e-15 &      \\ 
     &           &    3 &  4.22e-08 &  1.01e-15 &      \\ 
     &           &    4 &  3.55e-08 &  1.21e-15 &      \\ 
     &           &    5 &  4.22e-08 &  1.01e-15 &      \\ 
     &           &    6 &  3.55e-08 &  1.21e-15 &      \\ 
     &           &    7 &  4.22e-08 &  1.01e-15 &      \\ 
     &           &    8 &  3.55e-08 &  1.21e-15 &      \\ 
     &           &    9 &  4.22e-08 &  1.01e-15 &      \\ 
     &           &   10 &  3.55e-08 &  1.20e-15 &      \\ 
2863 &  2.86e+04 &   10 &           &           & iters  \\ 
 \hdashline 
     &           &    1 &  4.59e-09 &  3.18e-10 &      \\ 
     &           &    2 &  4.58e-09 &  1.31e-16 &      \\ 
     &           &    3 &  4.58e-09 &  1.31e-16 &      \\ 
     &           &    4 &  4.57e-09 &  1.31e-16 &      \\ 
     &           &    5 &  4.56e-09 &  1.30e-16 &      \\ 
     &           &    6 &  7.31e-08 &  1.30e-16 &      \\ 
     &           &    7 &  8.24e-08 &  2.09e-15 &      \\ 
     &           &    8 &  7.31e-08 &  2.35e-15 &      \\ 
     &           &    9 &  4.65e-09 &  2.09e-15 &      \\ 
     &           &   10 &  4.64e-09 &  1.33e-16 &      \\ 
2864 &  2.86e+04 &   10 &           &           & iters  \\ 
 \hdashline 
     &           &    1 &  1.76e-08 &  3.00e-10 &      \\ 
     &           &    2 &  1.76e-08 &  5.02e-16 &      \\ 
     &           &    3 &  1.75e-08 &  5.01e-16 &      \\ 
     &           &    4 &  6.02e-08 &  5.00e-16 &      \\ 
     &           &    5 &  1.76e-08 &  1.72e-15 &      \\ 
     &           &    6 &  1.76e-08 &  5.02e-16 &      \\ 
     &           &    7 &  1.75e-08 &  5.01e-16 &      \\ 
     &           &    8 &  1.75e-08 &  5.01e-16 &      \\ 
     &           &    9 &  6.02e-08 &  5.00e-16 &      \\ 
     &           &   10 &  1.76e-08 &  1.72e-15 &      \\ 
2865 &  2.86e+04 &   10 &           &           & iters  \\ 
 \hdashline 
     &           &    1 &  1.72e-08 &  2.16e-11 &      \\ 
     &           &    2 &  1.72e-08 &  4.91e-16 &      \\ 
     &           &    3 &  6.05e-08 &  4.90e-16 &      \\ 
     &           &    4 &  1.72e-08 &  1.73e-15 &      \\ 
     &           &    5 &  1.72e-08 &  4.92e-16 &      \\ 
     &           &    6 &  1.72e-08 &  4.91e-16 &      \\ 
     &           &    7 &  6.05e-08 &  4.90e-16 &      \\ 
     &           &    8 &  1.72e-08 &  1.73e-15 &      \\ 
     &           &    9 &  1.72e-08 &  4.92e-16 &      \\ 
     &           &   10 &  1.72e-08 &  4.92e-16 &      \\ 
2866 &  2.86e+04 &   10 &           &           & iters  \\ 
 \hdashline 
     &           &    1 &  3.43e-08 &  4.76e-10 &      \\ 
     &           &    2 &  4.34e-08 &  9.80e-16 &      \\ 
     &           &    3 &  3.43e-08 &  1.24e-15 &      \\ 
     &           &    4 &  4.34e-08 &  9.80e-16 &      \\ 
     &           &    5 &  3.44e-08 &  1.24e-15 &      \\ 
     &           &    6 &  4.34e-08 &  9.80e-16 &      \\ 
     &           &    7 &  3.44e-08 &  1.24e-15 &      \\ 
     &           &    8 &  3.43e-08 &  9.81e-16 &      \\ 
     &           &    9 &  4.34e-08 &  9.79e-16 &      \\ 
     &           &   10 &  3.43e-08 &  1.24e-15 &      \\ 
2867 &  2.87e+04 &   10 &           &           & iters  \\ 
 \hdashline 
     &           &    1 &  2.34e-08 &  2.19e-11 &      \\ 
     &           &    2 &  1.55e-08 &  6.68e-16 &      \\ 
     &           &    3 &  2.34e-08 &  4.42e-16 &      \\ 
     &           &    4 &  1.55e-08 &  6.67e-16 &      \\ 
     &           &    5 &  1.55e-08 &  4.42e-16 &      \\ 
     &           &    6 &  2.34e-08 &  4.42e-16 &      \\ 
     &           &    7 &  1.55e-08 &  6.68e-16 &      \\ 
     &           &    8 &  2.34e-08 &  4.42e-16 &      \\ 
     &           &    9 &  1.55e-08 &  6.67e-16 &      \\ 
     &           &   10 &  1.55e-08 &  4.42e-16 &      \\ 
2868 &  2.87e+04 &   10 &           &           & iters  \\ 
 \hdashline 
     &           &    1 &  3.48e-08 &  7.55e-10 &      \\ 
     &           &    2 &  4.29e-08 &  9.94e-16 &      \\ 
     &           &    3 &  3.48e-08 &  1.22e-15 &      \\ 
     &           &    4 &  4.29e-08 &  9.94e-16 &      \\ 
     &           &    5 &  3.48e-08 &  1.22e-15 &      \\ 
     &           &    6 &  3.48e-08 &  9.94e-16 &      \\ 
     &           &    7 &  4.29e-08 &  9.93e-16 &      \\ 
     &           &    8 &  3.48e-08 &  1.23e-15 &      \\ 
     &           &    9 &  4.29e-08 &  9.93e-16 &      \\ 
     &           &   10 &  3.48e-08 &  1.23e-15 &      \\ 
2869 &  2.87e+04 &   10 &           &           & iters  \\ 
 \hdashline 
     &           &    1 &  1.48e-09 &  7.13e-10 &      \\ 
     &           &    2 &  1.47e-09 &  4.21e-17 &      \\ 
     &           &    3 &  1.47e-09 &  4.21e-17 &      \\ 
     &           &    4 &  1.47e-09 &  4.20e-17 &      \\ 
     &           &    5 &  1.47e-09 &  4.20e-17 &      \\ 
     &           &    6 &  1.47e-09 &  4.19e-17 &      \\ 
     &           &    7 &  1.46e-09 &  4.18e-17 &      \\ 
     &           &    8 &  1.46e-09 &  4.18e-17 &      \\ 
     &           &    9 &  1.46e-09 &  4.17e-17 &      \\ 
     &           &   10 &  1.46e-09 &  4.17e-17 &      \\ 
2870 &  2.87e+04 &   10 &           &           & iters  \\ 
 \hdashline 
     &           &    1 &  1.26e-08 &  9.21e-12 &      \\ 
     &           &    2 &  1.26e-08 &  3.60e-16 &      \\ 
     &           &    3 &  6.51e-08 &  3.60e-16 &      \\ 
     &           &    4 &  1.27e-08 &  1.86e-15 &      \\ 
     &           &    5 &  1.27e-08 &  3.62e-16 &      \\ 
     &           &    6 &  1.26e-08 &  3.61e-16 &      \\ 
     &           &    7 &  1.26e-08 &  3.61e-16 &      \\ 
     &           &    8 &  1.26e-08 &  3.60e-16 &      \\ 
     &           &    9 &  6.51e-08 &  3.60e-16 &      \\ 
     &           &   10 &  9.04e-08 &  1.86e-15 &      \\ 
2871 &  2.87e+04 &   10 &           &           & iters  \\ 
 \hdashline 
     &           &    1 &  5.98e-08 &  5.47e-10 &      \\ 
     &           &    2 &  1.80e-08 &  1.71e-15 &      \\ 
     &           &    3 &  1.80e-08 &  5.14e-16 &      \\ 
     &           &    4 &  1.80e-08 &  5.13e-16 &      \\ 
     &           &    5 &  5.98e-08 &  5.13e-16 &      \\ 
     &           &    6 &  1.80e-08 &  1.71e-15 &      \\ 
     &           &    7 &  1.80e-08 &  5.14e-16 &      \\ 
     &           &    8 &  1.80e-08 &  5.14e-16 &      \\ 
     &           &    9 &  5.97e-08 &  5.13e-16 &      \\ 
     &           &   10 &  1.80e-08 &  1.71e-15 &      \\ 
2872 &  2.87e+04 &   10 &           &           & iters  \\ 
 \hdashline 
     &           &    1 &  1.60e-09 &  2.68e-10 &      \\ 
     &           &    2 &  1.60e-09 &  4.58e-17 &      \\ 
     &           &    3 &  1.60e-09 &  4.57e-17 &      \\ 
     &           &    4 &  1.60e-09 &  4.56e-17 &      \\ 
     &           &    5 &  1.59e-09 &  4.56e-17 &      \\ 
     &           &    6 &  1.59e-09 &  4.55e-17 &      \\ 
     &           &    7 &  1.59e-09 &  4.54e-17 &      \\ 
     &           &    8 &  1.59e-09 &  4.54e-17 &      \\ 
     &           &    9 &  1.59e-09 &  4.53e-17 &      \\ 
     &           &   10 &  1.58e-09 &  4.53e-17 &      \\ 
2873 &  2.87e+04 &   10 &           &           & iters  \\ 
 \hdashline 
     &           &    1 &  1.25e-08 &  1.85e-10 &      \\ 
     &           &    2 &  1.25e-08 &  3.57e-16 &      \\ 
     &           &    3 &  1.25e-08 &  3.56e-16 &      \\ 
     &           &    4 &  6.52e-08 &  3.56e-16 &      \\ 
     &           &    5 &  1.25e-08 &  1.86e-15 &      \\ 
     &           &    6 &  1.25e-08 &  3.58e-16 &      \\ 
     &           &    7 &  1.25e-08 &  3.57e-16 &      \\ 
     &           &    8 &  1.25e-08 &  3.57e-16 &      \\ 
     &           &    9 &  1.25e-08 &  3.56e-16 &      \\ 
     &           &   10 &  6.52e-08 &  3.56e-16 &      \\ 
2874 &  2.87e+04 &   10 &           &           & iters  \\ 
 \hdashline 
     &           &    1 &  2.13e-08 &  6.99e-10 &      \\ 
     &           &    2 &  2.13e-08 &  6.09e-16 &      \\ 
     &           &    3 &  2.13e-08 &  6.08e-16 &      \\ 
     &           &    4 &  5.65e-08 &  6.07e-16 &      \\ 
     &           &    5 &  2.13e-08 &  1.61e-15 &      \\ 
     &           &    6 &  2.13e-08 &  6.08e-16 &      \\ 
     &           &    7 &  2.13e-08 &  6.07e-16 &      \\ 
     &           &    8 &  5.65e-08 &  6.07e-16 &      \\ 
     &           &    9 &  2.13e-08 &  1.61e-15 &      \\ 
     &           &   10 &  2.13e-08 &  6.08e-16 &      \\ 
2875 &  2.87e+04 &   10 &           &           & iters  \\ 
 \hdashline 
     &           &    1 &  2.78e-08 &  7.70e-10 &      \\ 
     &           &    2 &  2.78e-08 &  7.94e-16 &      \\ 
     &           &    3 &  4.99e-08 &  7.93e-16 &      \\ 
     &           &    4 &  2.78e-08 &  1.43e-15 &      \\ 
     &           &    5 &  2.78e-08 &  7.94e-16 &      \\ 
     &           &    6 &  4.99e-08 &  7.93e-16 &      \\ 
     &           &    7 &  2.78e-08 &  1.43e-15 &      \\ 
     &           &    8 &  2.78e-08 &  7.94e-16 &      \\ 
     &           &    9 &  5.00e-08 &  7.93e-16 &      \\ 
     &           &   10 &  2.78e-08 &  1.43e-15 &      \\ 
2876 &  2.88e+04 &   10 &           &           & iters  \\ 
 \hdashline 
     &           &    1 &  3.88e-08 &  1.80e-10 &      \\ 
     &           &    2 &  3.90e-08 &  1.11e-15 &      \\ 
     &           &    3 &  3.88e-08 &  1.11e-15 &      \\ 
     &           &    4 &  3.90e-08 &  1.11e-15 &      \\ 
     &           &    5 &  3.88e-08 &  1.11e-15 &      \\ 
     &           &    6 &  3.90e-08 &  1.11e-15 &      \\ 
     &           &    7 &  3.88e-08 &  1.11e-15 &      \\ 
     &           &    8 &  3.90e-08 &  1.11e-15 &      \\ 
     &           &    9 &  3.88e-08 &  1.11e-15 &      \\ 
     &           &   10 &  3.90e-08 &  1.11e-15 &      \\ 
2877 &  2.88e+04 &   10 &           &           & iters  \\ 
 \hdashline 
     &           &    1 &  4.63e-08 &  2.61e-11 &      \\ 
     &           &    2 &  3.14e-08 &  1.32e-15 &      \\ 
     &           &    3 &  3.14e-08 &  8.98e-16 &      \\ 
     &           &    4 &  4.63e-08 &  8.96e-16 &      \\ 
     &           &    5 &  3.14e-08 &  1.32e-15 &      \\ 
     &           &    6 &  4.63e-08 &  8.97e-16 &      \\ 
     &           &    7 &  3.14e-08 &  1.32e-15 &      \\ 
     &           &    8 &  3.14e-08 &  8.98e-16 &      \\ 
     &           &    9 &  4.63e-08 &  8.96e-16 &      \\ 
     &           &   10 &  3.14e-08 &  1.32e-15 &      \\ 
2878 &  2.88e+04 &   10 &           &           & iters  \\ 
 \hdashline 
     &           &    1 &  1.15e-08 &  2.73e-10 &      \\ 
     &           &    2 &  1.14e-08 &  3.27e-16 &      \\ 
     &           &    3 &  1.14e-08 &  3.27e-16 &      \\ 
     &           &    4 &  1.14e-08 &  3.26e-16 &      \\ 
     &           &    5 &  1.14e-08 &  3.26e-16 &      \\ 
     &           &    6 &  6.63e-08 &  3.25e-16 &      \\ 
     &           &    7 &  1.15e-08 &  1.89e-15 &      \\ 
     &           &    8 &  1.15e-08 &  3.27e-16 &      \\ 
     &           &    9 &  1.14e-08 &  3.27e-16 &      \\ 
     &           &   10 &  1.14e-08 &  3.27e-16 &      \\ 
2879 &  2.88e+04 &   10 &           &           & iters  \\ 
 \hdashline 
     &           &    1 &  5.98e-08 &  3.66e-11 &      \\ 
     &           &    2 &  1.80e-08 &  1.71e-15 &      \\ 
     &           &    3 &  1.80e-08 &  5.14e-16 &      \\ 
     &           &    4 &  5.97e-08 &  5.13e-16 &      \\ 
     &           &    5 &  1.80e-08 &  1.70e-15 &      \\ 
     &           &    6 &  1.80e-08 &  5.15e-16 &      \\ 
     &           &    7 &  1.80e-08 &  5.14e-16 &      \\ 
     &           &    8 &  1.80e-08 &  5.13e-16 &      \\ 
     &           &    9 &  5.98e-08 &  5.13e-16 &      \\ 
     &           &   10 &  1.80e-08 &  1.71e-15 &      \\ 
2880 &  2.88e+04 &   10 &           &           & iters  \\ 
 \hdashline 
     &           &    1 &  1.80e-08 &  3.52e-10 &      \\ 
     &           &    2 &  1.80e-08 &  5.14e-16 &      \\ 
     &           &    3 &  5.97e-08 &  5.13e-16 &      \\ 
     &           &    4 &  1.80e-08 &  1.71e-15 &      \\ 
     &           &    5 &  1.80e-08 &  5.15e-16 &      \\ 
     &           &    6 &  1.80e-08 &  5.14e-16 &      \\ 
     &           &    7 &  5.97e-08 &  5.13e-16 &      \\ 
     &           &    8 &  1.80e-08 &  1.70e-15 &      \\ 
     &           &    9 &  1.80e-08 &  5.15e-16 &      \\ 
     &           &   10 &  1.80e-08 &  5.14e-16 &      \\ 
2881 &  2.88e+04 &   10 &           &           & iters  \\ 
 \hdashline 
     &           &    1 &  4.17e-08 &  2.19e-10 &      \\ 
     &           &    2 &  3.61e-08 &  1.19e-15 &      \\ 
     &           &    3 &  4.16e-08 &  1.03e-15 &      \\ 
     &           &    4 &  3.61e-08 &  1.19e-15 &      \\ 
     &           &    5 &  4.16e-08 &  1.03e-15 &      \\ 
     &           &    6 &  3.61e-08 &  1.19e-15 &      \\ 
     &           &    7 &  4.16e-08 &  1.03e-15 &      \\ 
     &           &    8 &  3.61e-08 &  1.19e-15 &      \\ 
     &           &    9 &  4.16e-08 &  1.03e-15 &      \\ 
     &           &   10 &  3.61e-08 &  1.19e-15 &      \\ 
2882 &  2.88e+04 &   10 &           &           & iters  \\ 
 \hdashline 
     &           &    1 &  5.09e-08 &  3.30e-10 &      \\ 
     &           &    2 &  2.68e-08 &  1.45e-15 &      \\ 
     &           &    3 &  5.09e-08 &  7.66e-16 &      \\ 
     &           &    4 &  2.69e-08 &  1.45e-15 &      \\ 
     &           &    5 &  2.68e-08 &  7.67e-16 &      \\ 
     &           &    6 &  5.09e-08 &  7.66e-16 &      \\ 
     &           &    7 &  2.69e-08 &  1.45e-15 &      \\ 
     &           &    8 &  2.68e-08 &  7.67e-16 &      \\ 
     &           &    9 &  5.09e-08 &  7.66e-16 &      \\ 
     &           &   10 &  2.69e-08 &  1.45e-15 &      \\ 
2883 &  2.88e+04 &   10 &           &           & iters  \\ 
 \hdashline 
     &           &    1 &  4.70e-08 &  5.91e-10 &      \\ 
     &           &    2 &  3.07e-08 &  1.34e-15 &      \\ 
     &           &    3 &  3.07e-08 &  8.77e-16 &      \\ 
     &           &    4 &  4.71e-08 &  8.75e-16 &      \\ 
     &           &    5 &  3.07e-08 &  1.34e-15 &      \\ 
     &           &    6 &  4.70e-08 &  8.76e-16 &      \\ 
     &           &    7 &  3.07e-08 &  1.34e-15 &      \\ 
     &           &    8 &  3.07e-08 &  8.77e-16 &      \\ 
     &           &    9 &  4.71e-08 &  8.76e-16 &      \\ 
     &           &   10 &  3.07e-08 &  1.34e-15 &      \\ 
2884 &  2.88e+04 &   10 &           &           & iters  \\ 
 \hdashline 
     &           &    1 &  5.74e-09 &  1.91e-10 &      \\ 
     &           &    2 &  5.73e-09 &  1.64e-16 &      \\ 
     &           &    3 &  5.72e-09 &  1.64e-16 &      \\ 
     &           &    4 &  7.20e-08 &  1.63e-16 &      \\ 
     &           &    5 &  4.47e-08 &  2.05e-15 &      \\ 
     &           &    6 &  5.75e-09 &  1.27e-15 &      \\ 
     &           &    7 &  5.75e-09 &  1.64e-16 &      \\ 
     &           &    8 &  5.74e-09 &  1.64e-16 &      \\ 
     &           &    9 &  5.73e-09 &  1.64e-16 &      \\ 
     &           &   10 &  5.72e-09 &  1.64e-16 &      \\ 
2885 &  2.88e+04 &   10 &           &           & iters  \\ 
 \hdashline 
     &           &    1 &  4.81e-08 &  3.83e-10 &      \\ 
     &           &    2 &  2.97e-08 &  1.37e-15 &      \\ 
     &           &    3 &  4.81e-08 &  8.46e-16 &      \\ 
     &           &    4 &  2.97e-08 &  1.37e-15 &      \\ 
     &           &    5 &  2.96e-08 &  8.47e-16 &      \\ 
     &           &    6 &  4.81e-08 &  8.46e-16 &      \\ 
     &           &    7 &  2.97e-08 &  1.37e-15 &      \\ 
     &           &    8 &  4.81e-08 &  8.47e-16 &      \\ 
     &           &    9 &  2.97e-08 &  1.37e-15 &      \\ 
     &           &   10 &  2.97e-08 &  8.47e-16 &      \\ 
2886 &  2.88e+04 &   10 &           &           & iters  \\ 
 \hdashline 
     &           &    1 &  2.25e-08 &  8.00e-11 &      \\ 
     &           &    2 &  5.53e-08 &  6.41e-16 &      \\ 
     &           &    3 &  2.25e-08 &  1.58e-15 &      \\ 
     &           &    4 &  2.25e-08 &  6.43e-16 &      \\ 
     &           &    5 &  5.52e-08 &  6.42e-16 &      \\ 
     &           &    6 &  2.25e-08 &  1.58e-15 &      \\ 
     &           &    7 &  2.25e-08 &  6.43e-16 &      \\ 
     &           &    8 &  2.25e-08 &  6.42e-16 &      \\ 
     &           &    9 &  5.53e-08 &  6.41e-16 &      \\ 
     &           &   10 &  2.25e-08 &  1.58e-15 &      \\ 
2887 &  2.89e+04 &   10 &           &           & iters  \\ 
 \hdashline 
     &           &    1 &  5.68e-09 &  5.39e-10 &      \\ 
     &           &    2 &  5.68e-09 &  1.62e-16 &      \\ 
     &           &    3 &  5.67e-09 &  1.62e-16 &      \\ 
     &           &    4 &  5.66e-09 &  1.62e-16 &      \\ 
     &           &    5 &  7.20e-08 &  1.62e-16 &      \\ 
     &           &    6 &  4.46e-08 &  2.06e-15 &      \\ 
     &           &    7 &  5.69e-09 &  1.27e-15 &      \\ 
     &           &    8 &  5.68e-09 &  1.62e-16 &      \\ 
     &           &    9 &  5.68e-09 &  1.62e-16 &      \\ 
     &           &   10 &  5.67e-09 &  1.62e-16 &      \\ 
2888 &  2.89e+04 &   10 &           &           & iters  \\ 
 \hdashline 
     &           &    1 &  3.38e-08 &  6.87e-10 &      \\ 
     &           &    2 &  5.08e-09 &  9.65e-16 &      \\ 
     &           &    3 &  5.07e-09 &  1.45e-16 &      \\ 
     &           &    4 &  5.06e-09 &  1.45e-16 &      \\ 
     &           &    5 &  5.05e-09 &  1.44e-16 &      \\ 
     &           &    6 &  5.05e-09 &  1.44e-16 &      \\ 
     &           &    7 &  3.38e-08 &  1.44e-16 &      \\ 
     &           &    8 &  5.09e-09 &  9.65e-16 &      \\ 
     &           &    9 &  5.08e-09 &  1.45e-16 &      \\ 
     &           &   10 &  5.07e-09 &  1.45e-16 &      \\ 
2889 &  2.89e+04 &   10 &           &           & iters  \\ 
 \hdashline 
     &           &    1 &  3.68e-08 &  5.36e-10 &      \\ 
     &           &    2 &  4.09e-08 &  1.05e-15 &      \\ 
     &           &    3 &  3.68e-08 &  1.17e-15 &      \\ 
     &           &    4 &  4.09e-08 &  1.05e-15 &      \\ 
     &           &    5 &  3.68e-08 &  1.17e-15 &      \\ 
     &           &    6 &  4.09e-08 &  1.05e-15 &      \\ 
     &           &    7 &  3.68e-08 &  1.17e-15 &      \\ 
     &           &    8 &  4.09e-08 &  1.05e-15 &      \\ 
     &           &    9 &  3.68e-08 &  1.17e-15 &      \\ 
     &           &   10 &  4.09e-08 &  1.05e-15 &      \\ 
2890 &  2.89e+04 &   10 &           &           & iters  \\ 
 \hdashline 
     &           &    1 &  2.18e-08 &  2.63e-10 &      \\ 
     &           &    2 &  5.59e-08 &  6.23e-16 &      \\ 
     &           &    3 &  2.19e-08 &  1.60e-15 &      \\ 
     &           &    4 &  2.18e-08 &  6.24e-16 &      \\ 
     &           &    5 &  5.59e-08 &  6.23e-16 &      \\ 
     &           &    6 &  2.19e-08 &  1.60e-15 &      \\ 
     &           &    7 &  2.18e-08 &  6.24e-16 &      \\ 
     &           &    8 &  2.18e-08 &  6.24e-16 &      \\ 
     &           &    9 &  5.59e-08 &  6.23e-16 &      \\ 
     &           &   10 &  2.19e-08 &  1.60e-15 &      \\ 
2891 &  2.89e+04 &   10 &           &           & iters  \\ 
 \hdashline 
     &           &    1 &  1.89e-08 &  3.67e-11 &      \\ 
     &           &    2 &  1.89e-08 &  5.40e-16 &      \\ 
     &           &    3 &  5.88e-08 &  5.40e-16 &      \\ 
     &           &    4 &  1.90e-08 &  1.68e-15 &      \\ 
     &           &    5 &  1.89e-08 &  5.41e-16 &      \\ 
     &           &    6 &  1.89e-08 &  5.40e-16 &      \\ 
     &           &    7 &  5.88e-08 &  5.40e-16 &      \\ 
     &           &    8 &  1.90e-08 &  1.68e-15 &      \\ 
     &           &    9 &  1.89e-08 &  5.41e-16 &      \\ 
     &           &   10 &  1.89e-08 &  5.41e-16 &      \\ 
2892 &  2.89e+04 &   10 &           &           & iters  \\ 
 \hdashline 
     &           &    1 &  3.03e-08 &  3.12e-10 &      \\ 
     &           &    2 &  3.02e-08 &  8.64e-16 &      \\ 
     &           &    3 &  4.75e-08 &  8.63e-16 &      \\ 
     &           &    4 &  3.03e-08 &  1.36e-15 &      \\ 
     &           &    5 &  4.75e-08 &  8.63e-16 &      \\ 
     &           &    6 &  3.03e-08 &  1.36e-15 &      \\ 
     &           &    7 &  3.02e-08 &  8.64e-16 &      \\ 
     &           &    8 &  4.75e-08 &  8.63e-16 &      \\ 
     &           &    9 &  3.03e-08 &  1.36e-15 &      \\ 
     &           &   10 &  3.02e-08 &  8.64e-16 &      \\ 
2893 &  2.89e+04 &   10 &           &           & iters  \\ 
 \hdashline 
     &           &    1 &  6.58e-08 &  7.74e-10 &      \\ 
     &           &    2 &  1.20e-08 &  1.88e-15 &      \\ 
     &           &    3 &  1.20e-08 &  3.43e-16 &      \\ 
     &           &    4 &  1.20e-08 &  3.43e-16 &      \\ 
     &           &    5 &  1.20e-08 &  3.42e-16 &      \\ 
     &           &    6 &  6.57e-08 &  3.42e-16 &      \\ 
     &           &    7 &  8.97e-08 &  1.88e-15 &      \\ 
     &           &    8 &  6.58e-08 &  2.56e-15 &      \\ 
     &           &    9 &  1.20e-08 &  1.88e-15 &      \\ 
     &           &   10 &  1.20e-08 &  3.43e-16 &      \\ 
2894 &  2.89e+04 &   10 &           &           & iters  \\ 
 \hdashline 
     &           &    1 &  1.92e-09 &  6.94e-10 &      \\ 
     &           &    2 &  1.92e-09 &  5.48e-17 &      \\ 
     &           &    3 &  1.92e-09 &  5.48e-17 &      \\ 
     &           &    4 &  1.91e-09 &  5.47e-17 &      \\ 
     &           &    5 &  1.91e-09 &  5.46e-17 &      \\ 
     &           &    6 &  1.91e-09 &  5.45e-17 &      \\ 
     &           &    7 &  1.90e-09 &  5.44e-17 &      \\ 
     &           &    8 &  1.90e-09 &  5.44e-17 &      \\ 
     &           &    9 &  1.90e-09 &  5.43e-17 &      \\ 
     &           &   10 &  1.90e-09 &  5.42e-17 &      \\ 
2895 &  2.89e+04 &   10 &           &           & iters  \\ 
 \hdashline 
     &           &    1 &  1.75e-08 &  2.28e-10 &      \\ 
     &           &    2 &  6.02e-08 &  5.00e-16 &      \\ 
     &           &    3 &  1.76e-08 &  1.72e-15 &      \\ 
     &           &    4 &  1.76e-08 &  5.02e-16 &      \\ 
     &           &    5 &  1.75e-08 &  5.01e-16 &      \\ 
     &           &    6 &  6.02e-08 &  5.00e-16 &      \\ 
     &           &    7 &  1.76e-08 &  1.72e-15 &      \\ 
     &           &    8 &  1.76e-08 &  5.02e-16 &      \\ 
     &           &    9 &  1.75e-08 &  5.01e-16 &      \\ 
     &           &   10 &  1.75e-08 &  5.01e-16 &      \\ 
2896 &  2.90e+04 &   10 &           &           & iters  \\ 
 \hdashline 
     &           &    1 &  3.37e-08 &  4.64e-10 &      \\ 
     &           &    2 &  3.37e-08 &  9.63e-16 &      \\ 
     &           &    3 &  4.40e-08 &  9.62e-16 &      \\ 
     &           &    4 &  3.37e-08 &  1.26e-15 &      \\ 
     &           &    5 &  4.40e-08 &  9.62e-16 &      \\ 
     &           &    6 &  3.37e-08 &  1.26e-15 &      \\ 
     &           &    7 &  4.40e-08 &  9.63e-16 &      \\ 
     &           &    8 &  3.37e-08 &  1.26e-15 &      \\ 
     &           &    9 &  3.37e-08 &  9.63e-16 &      \\ 
     &           &   10 &  4.40e-08 &  9.62e-16 &      \\ 
2897 &  2.90e+04 &   10 &           &           & iters  \\ 
 \hdashline 
     &           &    1 &  2.28e-08 &  1.04e-09 &      \\ 
     &           &    2 &  2.27e-08 &  6.50e-16 &      \\ 
     &           &    3 &  5.50e-08 &  6.49e-16 &      \\ 
     &           &    4 &  2.28e-08 &  1.57e-15 &      \\ 
     &           &    5 &  2.28e-08 &  6.50e-16 &      \\ 
     &           &    6 &  5.50e-08 &  6.49e-16 &      \\ 
     &           &    7 &  2.28e-08 &  1.57e-15 &      \\ 
     &           &    8 &  2.28e-08 &  6.51e-16 &      \\ 
     &           &    9 &  5.50e-08 &  6.50e-16 &      \\ 
     &           &   10 &  2.28e-08 &  1.57e-15 &      \\ 
2898 &  2.90e+04 &   10 &           &           & iters  \\ 
 \hdashline 
     &           &    1 &  3.68e-08 &  8.26e-10 &      \\ 
     &           &    2 &  4.09e-08 &  1.05e-15 &      \\ 
     &           &    3 &  3.68e-08 &  1.17e-15 &      \\ 
     &           &    4 &  4.09e-08 &  1.05e-15 &      \\ 
     &           &    5 &  3.68e-08 &  1.17e-15 &      \\ 
     &           &    6 &  4.09e-08 &  1.05e-15 &      \\ 
     &           &    7 &  3.69e-08 &  1.17e-15 &      \\ 
     &           &    8 &  4.09e-08 &  1.05e-15 &      \\ 
     &           &    9 &  3.69e-08 &  1.17e-15 &      \\ 
     &           &   10 &  4.09e-08 &  1.05e-15 &      \\ 
2899 &  2.90e+04 &   10 &           &           & iters  \\ 
 \hdashline 
     &           &    1 &  2.94e-08 &  6.06e-10 &      \\ 
     &           &    2 &  2.94e-08 &  8.39e-16 &      \\ 
     &           &    3 &  4.84e-08 &  8.38e-16 &      \\ 
     &           &    4 &  2.94e-08 &  1.38e-15 &      \\ 
     &           &    5 &  2.94e-08 &  8.39e-16 &      \\ 
     &           &    6 &  4.84e-08 &  8.38e-16 &      \\ 
     &           &    7 &  2.94e-08 &  1.38e-15 &      \\ 
     &           &    8 &  4.84e-08 &  8.38e-16 &      \\ 
     &           &    9 &  2.94e-08 &  1.38e-15 &      \\ 
     &           &   10 &  2.94e-08 &  8.39e-16 &      \\ 
2900 &  2.90e+04 &   10 &           &           & iters  \\ 
 \hdashline 
     &           &    1 &  4.30e-08 &  2.63e-10 &      \\ 
     &           &    2 &  3.47e-08 &  1.23e-15 &      \\ 
     &           &    3 &  4.30e-08 &  9.92e-16 &      \\ 
     &           &    4 &  3.48e-08 &  1.23e-15 &      \\ 
     &           &    5 &  4.30e-08 &  9.92e-16 &      \\ 
     &           &    6 &  3.48e-08 &  1.23e-15 &      \\ 
     &           &    7 &  3.47e-08 &  9.92e-16 &      \\ 
     &           &    8 &  4.30e-08 &  9.91e-16 &      \\ 
     &           &    9 &  3.47e-08 &  1.23e-15 &      \\ 
     &           &   10 &  4.30e-08 &  9.91e-16 &      \\ 
2901 &  2.90e+04 &   10 &           &           & iters  \\ 
 \hdashline 
     &           &    1 &  3.61e-08 &  9.35e-10 &      \\ 
     &           &    2 &  4.16e-08 &  1.03e-15 &      \\ 
     &           &    3 &  3.61e-08 &  1.19e-15 &      \\ 
     &           &    4 &  3.61e-08 &  1.03e-15 &      \\ 
     &           &    5 &  4.17e-08 &  1.03e-15 &      \\ 
     &           &    6 &  3.61e-08 &  1.19e-15 &      \\ 
     &           &    7 &  4.16e-08 &  1.03e-15 &      \\ 
     &           &    8 &  3.61e-08 &  1.19e-15 &      \\ 
     &           &    9 &  4.16e-08 &  1.03e-15 &      \\ 
     &           &   10 &  3.61e-08 &  1.19e-15 &      \\ 
2902 &  2.90e+04 &   10 &           &           & iters  \\ 
 \hdashline 
     &           &    1 &  2.14e-08 &  1.40e-09 &      \\ 
     &           &    2 &  2.13e-08 &  6.09e-16 &      \\ 
     &           &    3 &  2.13e-08 &  6.09e-16 &      \\ 
     &           &    4 &  5.64e-08 &  6.08e-16 &      \\ 
     &           &    5 &  2.13e-08 &  1.61e-15 &      \\ 
     &           &    6 &  2.13e-08 &  6.09e-16 &      \\ 
     &           &    7 &  2.13e-08 &  6.08e-16 &      \\ 
     &           &    8 &  5.64e-08 &  6.07e-16 &      \\ 
     &           &    9 &  2.13e-08 &  1.61e-15 &      \\ 
     &           &   10 &  2.13e-08 &  6.09e-16 &      \\ 
2903 &  2.90e+04 &   10 &           &           & iters  \\ 
 \hdashline 
     &           &    1 &  4.52e-08 &  8.24e-10 &      \\ 
     &           &    2 &  3.26e-08 &  1.29e-15 &      \\ 
     &           &    3 &  3.25e-08 &  9.30e-16 &      \\ 
     &           &    4 &  4.52e-08 &  9.29e-16 &      \\ 
     &           &    5 &  3.26e-08 &  1.29e-15 &      \\ 
     &           &    6 &  4.52e-08 &  9.29e-16 &      \\ 
     &           &    7 &  3.26e-08 &  1.29e-15 &      \\ 
     &           &    8 &  3.25e-08 &  9.30e-16 &      \\ 
     &           &    9 &  4.52e-08 &  9.29e-16 &      \\ 
     &           &   10 &  3.26e-08 &  1.29e-15 &      \\ 
2904 &  2.90e+04 &   10 &           &           & iters  \\ 
 \hdashline 
     &           &    1 &  2.66e-08 &  8.32e-11 &      \\ 
     &           &    2 &  2.65e-08 &  7.59e-16 &      \\ 
     &           &    3 &  5.12e-08 &  7.58e-16 &      \\ 
     &           &    4 &  2.66e-08 &  1.46e-15 &      \\ 
     &           &    5 &  2.65e-08 &  7.59e-16 &      \\ 
     &           &    6 &  5.12e-08 &  7.58e-16 &      \\ 
     &           &    7 &  2.66e-08 &  1.46e-15 &      \\ 
     &           &    8 &  2.65e-08 &  7.59e-16 &      \\ 
     &           &    9 &  5.12e-08 &  7.57e-16 &      \\ 
     &           &   10 &  2.66e-08 &  1.46e-15 &      \\ 
2905 &  2.90e+04 &   10 &           &           & iters  \\ 
 \hdashline 
     &           &    1 &  3.11e-09 &  1.11e-09 &      \\ 
     &           &    2 &  3.10e-09 &  8.86e-17 &      \\ 
     &           &    3 &  3.10e-09 &  8.85e-17 &      \\ 
     &           &    4 &  3.09e-09 &  8.84e-17 &      \\ 
     &           &    5 &  3.09e-09 &  8.82e-17 &      \\ 
     &           &    6 &  3.08e-09 &  8.81e-17 &      \\ 
     &           &    7 &  3.08e-09 &  8.80e-17 &      \\ 
     &           &    8 &  3.07e-09 &  8.79e-17 &      \\ 
     &           &    9 &  3.07e-09 &  8.77e-17 &      \\ 
     &           &   10 &  3.07e-09 &  8.76e-17 &      \\ 
2906 &  2.90e+04 &   10 &           &           & iters  \\ 
 \hdashline 
     &           &    1 &  2.53e-08 &  3.05e-10 &      \\ 
     &           &    2 &  5.24e-08 &  7.21e-16 &      \\ 
     &           &    3 &  2.53e-08 &  1.50e-15 &      \\ 
     &           &    4 &  2.53e-08 &  7.22e-16 &      \\ 
     &           &    5 &  5.24e-08 &  7.21e-16 &      \\ 
     &           &    6 &  2.53e-08 &  1.50e-15 &      \\ 
     &           &    7 &  2.53e-08 &  7.23e-16 &      \\ 
     &           &    8 &  5.24e-08 &  7.22e-16 &      \\ 
     &           &    9 &  2.53e-08 &  1.50e-15 &      \\ 
     &           &   10 &  2.53e-08 &  7.23e-16 &      \\ 
2907 &  2.91e+04 &   10 &           &           & iters  \\ 
 \hdashline 
     &           &    1 &  6.67e-08 &  1.67e-09 &      \\ 
     &           &    2 &  1.10e-08 &  1.91e-15 &      \\ 
     &           &    3 &  1.10e-08 &  3.15e-16 &      \\ 
     &           &    4 &  1.10e-08 &  3.14e-16 &      \\ 
     &           &    5 &  1.10e-08 &  3.14e-16 &      \\ 
     &           &    6 &  6.67e-08 &  3.14e-16 &      \\ 
     &           &    7 &  1.11e-08 &  1.90e-15 &      \\ 
     &           &    8 &  1.11e-08 &  3.16e-16 &      \\ 
     &           &    9 &  1.10e-08 &  3.15e-16 &      \\ 
     &           &   10 &  1.10e-08 &  3.15e-16 &      \\ 
2908 &  2.91e+04 &   10 &           &           & iters  \\ 
 \hdashline 
     &           &    1 &  1.25e-10 &  3.08e-09 &      \\ 
     &           &    2 &  1.24e-10 &  3.55e-18 &      \\ 
     &           &    3 &  1.24e-10 &  3.55e-18 &      \\ 
     &           &    4 &  1.24e-10 &  3.54e-18 &      \\ 
     &           &    5 &  1.24e-10 &  3.54e-18 &      \\ 
     &           &    6 &  1.24e-10 &  3.53e-18 &      \\ 
     &           &    7 &  1.23e-10 &  3.53e-18 &      \\ 
     &           &    8 &  1.23e-10 &  3.52e-18 &      \\ 
     &           &    9 &  1.23e-10 &  3.52e-18 &      \\ 
     &           &   10 &  1.23e-10 &  3.51e-18 &      \\ 
2909 &  2.91e+04 &   10 &           &           & iters  \\ 
 \hdashline 
     &           &    1 &  3.45e-08 &  2.44e-09 &      \\ 
     &           &    2 &  4.33e-08 &  9.83e-16 &      \\ 
     &           &    3 &  3.45e-08 &  1.24e-15 &      \\ 
     &           &    4 &  4.33e-08 &  9.84e-16 &      \\ 
     &           &    5 &  3.45e-08 &  1.24e-15 &      \\ 
     &           &    6 &  3.44e-08 &  9.84e-16 &      \\ 
     &           &    7 &  4.33e-08 &  9.83e-16 &      \\ 
     &           &    8 &  3.44e-08 &  1.24e-15 &      \\ 
     &           &    9 &  4.33e-08 &  9.83e-16 &      \\ 
     &           &   10 &  3.45e-08 &  1.24e-15 &      \\ 
2910 &  2.91e+04 &   10 &           &           & iters  \\ 
 \hdashline 
     &           &    1 &  8.02e-10 &  1.86e-09 &      \\ 
     &           &    2 &  8.01e-10 &  2.29e-17 &      \\ 
     &           &    3 &  8.00e-10 &  2.29e-17 &      \\ 
     &           &    4 &  7.99e-10 &  2.28e-17 &      \\ 
     &           &    5 &  7.69e-08 &  2.28e-17 &      \\ 
     &           &    6 &  7.86e-08 &  2.19e-15 &      \\ 
     &           &    7 &  7.69e-08 &  2.24e-15 &      \\ 
     &           &    8 &  7.86e-08 &  2.19e-15 &      \\ 
     &           &    9 &  7.69e-08 &  2.24e-15 &      \\ 
     &           &   10 &  7.86e-08 &  2.19e-15 &      \\ 
2911 &  2.91e+04 &   10 &           &           & iters  \\ 
 \hdashline 
     &           &    1 &  4.12e-08 &  4.84e-09 &      \\ 
     &           &    2 &  4.11e-08 &  1.18e-15 &      \\ 
     &           &    3 &  3.66e-08 &  1.17e-15 &      \\ 
     &           &    4 &  4.11e-08 &  1.05e-15 &      \\ 
     &           &    5 &  3.66e-08 &  1.17e-15 &      \\ 
     &           &    6 &  4.11e-08 &  1.05e-15 &      \\ 
     &           &    7 &  3.66e-08 &  1.17e-15 &      \\ 
     &           &    8 &  4.11e-08 &  1.05e-15 &      \\ 
     &           &    9 &  3.66e-08 &  1.17e-15 &      \\ 
     &           &   10 &  3.66e-08 &  1.05e-15 &      \\ 
2912 &  2.91e+04 &   10 &           &           & iters  \\ 
 \hdashline 
     &           &    1 &  3.57e-08 &  2.76e-09 &      \\ 
     &           &    2 &  4.20e-08 &  1.02e-15 &      \\ 
     &           &    3 &  3.57e-08 &  1.20e-15 &      \\ 
     &           &    4 &  4.20e-08 &  1.02e-15 &      \\ 
     &           &    5 &  3.57e-08 &  1.20e-15 &      \\ 
     &           &    6 &  4.20e-08 &  1.02e-15 &      \\ 
     &           &    7 &  3.58e-08 &  1.20e-15 &      \\ 
     &           &    8 &  4.20e-08 &  1.02e-15 &      \\ 
     &           &    9 &  3.58e-08 &  1.20e-15 &      \\ 
     &           &   10 &  3.57e-08 &  1.02e-15 &      \\ 
2913 &  2.91e+04 &   10 &           &           & iters  \\ 
 \hdashline 
     &           &    1 &  1.82e-08 &  4.85e-10 &      \\ 
     &           &    2 &  1.81e-08 &  5.18e-16 &      \\ 
     &           &    3 &  1.81e-08 &  5.17e-16 &      \\ 
     &           &    4 &  1.81e-08 &  5.17e-16 &      \\ 
     &           &    5 &  1.81e-08 &  5.16e-16 &      \\ 
     &           &    6 &  1.80e-08 &  5.15e-16 &      \\ 
     &           &    7 &  5.97e-08 &  5.14e-16 &      \\ 
     &           &    8 &  1.81e-08 &  1.70e-15 &      \\ 
     &           &    9 &  1.81e-08 &  5.16e-16 &      \\ 
     &           &   10 &  1.80e-08 &  5.15e-16 &      \\ 
2914 &  2.91e+04 &   10 &           &           & iters  \\ 
 \hdashline 
     &           &    1 &  3.58e-08 &  1.68e-09 &      \\ 
     &           &    2 &  4.19e-08 &  1.02e-15 &      \\ 
     &           &    3 &  3.58e-08 &  1.20e-15 &      \\ 
     &           &    4 &  4.19e-08 &  1.02e-15 &      \\ 
     &           &    5 &  3.59e-08 &  1.20e-15 &      \\ 
     &           &    6 &  4.19e-08 &  1.02e-15 &      \\ 
     &           &    7 &  3.59e-08 &  1.20e-15 &      \\ 
     &           &    8 &  3.58e-08 &  1.02e-15 &      \\ 
     &           &    9 &  4.19e-08 &  1.02e-15 &      \\ 
     &           &   10 &  3.58e-08 &  1.20e-15 &      \\ 
2915 &  2.91e+04 &   10 &           &           & iters  \\ 
 \hdashline 
     &           &    1 &  2.01e-08 &  8.67e-10 &      \\ 
     &           &    2 &  5.76e-08 &  5.75e-16 &      \\ 
     &           &    3 &  2.02e-08 &  1.64e-15 &      \\ 
     &           &    4 &  2.02e-08 &  5.76e-16 &      \\ 
     &           &    5 &  2.01e-08 &  5.76e-16 &      \\ 
     &           &    6 &  5.76e-08 &  5.75e-16 &      \\ 
     &           &    7 &  2.02e-08 &  1.64e-15 &      \\ 
     &           &    8 &  2.02e-08 &  5.76e-16 &      \\ 
     &           &    9 &  2.01e-08 &  5.75e-16 &      \\ 
     &           &   10 &  5.76e-08 &  5.75e-16 &      \\ 
2916 &  2.92e+04 &   10 &           &           & iters  \\ 
 \hdashline 
     &           &    1 &  2.62e-08 &  9.99e-11 &      \\ 
     &           &    2 &  5.15e-08 &  7.48e-16 &      \\ 
     &           &    3 &  2.62e-08 &  1.47e-15 &      \\ 
     &           &    4 &  2.62e-08 &  7.49e-16 &      \\ 
     &           &    5 &  5.15e-08 &  7.48e-16 &      \\ 
     &           &    6 &  2.62e-08 &  1.47e-15 &      \\ 
     &           &    7 &  2.62e-08 &  7.49e-16 &      \\ 
     &           &    8 &  5.15e-08 &  7.48e-16 &      \\ 
     &           &    9 &  2.62e-08 &  1.47e-15 &      \\ 
     &           &   10 &  2.62e-08 &  7.49e-16 &      \\ 
2917 &  2.92e+04 &   10 &           &           & iters  \\ 
 \hdashline 
     &           &    1 &  5.60e-09 &  2.38e-10 &      \\ 
     &           &    2 &  5.59e-09 &  1.60e-16 &      \\ 
     &           &    3 &  5.58e-09 &  1.59e-16 &      \\ 
     &           &    4 &  5.57e-09 &  1.59e-16 &      \\ 
     &           &    5 &  5.56e-09 &  1.59e-16 &      \\ 
     &           &    6 &  5.56e-09 &  1.59e-16 &      \\ 
     &           &    7 &  5.55e-09 &  1.59e-16 &      \\ 
     &           &    8 &  7.21e-08 &  1.58e-16 &      \\ 
     &           &    9 &  8.33e-08 &  2.06e-15 &      \\ 
     &           &   10 &  7.22e-08 &  2.38e-15 &      \\ 
2918 &  2.92e+04 &   10 &           &           & iters  \\ 
 \hdashline 
     &           &    1 &  6.03e-08 &  4.87e-10 &      \\ 
     &           &    2 &  1.75e-08 &  1.72e-15 &      \\ 
     &           &    3 &  1.75e-08 &  5.00e-16 &      \\ 
     &           &    4 &  1.75e-08 &  4.99e-16 &      \\ 
     &           &    5 &  6.03e-08 &  4.98e-16 &      \\ 
     &           &    6 &  1.75e-08 &  1.72e-15 &      \\ 
     &           &    7 &  1.75e-08 &  5.00e-16 &      \\ 
     &           &    8 &  1.75e-08 &  4.99e-16 &      \\ 
     &           &    9 &  6.02e-08 &  4.99e-16 &      \\ 
     &           &   10 &  1.75e-08 &  1.72e-15 &      \\ 
2919 &  2.92e+04 &   10 &           &           & iters  \\ 
 \hdashline 
     &           &    1 &  4.92e-08 &  6.80e-12 &      \\ 
     &           &    2 &  2.86e-08 &  1.40e-15 &      \\ 
     &           &    3 &  4.92e-08 &  8.15e-16 &      \\ 
     &           &    4 &  2.86e-08 &  1.40e-15 &      \\ 
     &           &    5 &  2.85e-08 &  8.16e-16 &      \\ 
     &           &    6 &  4.92e-08 &  8.15e-16 &      \\ 
     &           &    7 &  2.86e-08 &  1.40e-15 &      \\ 
     &           &    8 &  2.85e-08 &  8.15e-16 &      \\ 
     &           &    9 &  4.92e-08 &  8.14e-16 &      \\ 
     &           &   10 &  2.86e-08 &  1.40e-15 &      \\ 
2920 &  2.92e+04 &   10 &           &           & iters  \\ 
 \hdashline 
     &           &    1 &  2.13e-08 &  4.25e-10 &      \\ 
     &           &    2 &  2.13e-08 &  6.08e-16 &      \\ 
     &           &    3 &  2.13e-08 &  6.07e-16 &      \\ 
     &           &    4 &  5.65e-08 &  6.07e-16 &      \\ 
     &           &    5 &  2.13e-08 &  1.61e-15 &      \\ 
     &           &    6 &  2.13e-08 &  6.08e-16 &      \\ 
     &           &    7 &  2.12e-08 &  6.07e-16 &      \\ 
     &           &    8 &  5.65e-08 &  6.06e-16 &      \\ 
     &           &    9 &  2.13e-08 &  1.61e-15 &      \\ 
     &           &   10 &  2.13e-08 &  6.08e-16 &      \\ 
2921 &  2.92e+04 &   10 &           &           & iters  \\ 
 \hdashline 
     &           &    1 &  6.80e-08 &  3.39e-10 &      \\ 
     &           &    2 &  9.74e-09 &  1.94e-15 &      \\ 
     &           &    3 &  9.73e-09 &  2.78e-16 &      \\ 
     &           &    4 &  9.72e-09 &  2.78e-16 &      \\ 
     &           &    5 &  9.70e-09 &  2.77e-16 &      \\ 
     &           &    6 &  9.69e-09 &  2.77e-16 &      \\ 
     &           &    7 &  9.68e-09 &  2.77e-16 &      \\ 
     &           &    8 &  6.80e-08 &  2.76e-16 &      \\ 
     &           &    9 &  9.76e-09 &  1.94e-15 &      \\ 
     &           &   10 &  9.74e-09 &  2.79e-16 &      \\ 
2922 &  2.92e+04 &   10 &           &           & iters  \\ 
 \hdashline 
     &           &    1 &  7.19e-09 &  9.79e-10 &      \\ 
     &           &    2 &  7.18e-09 &  2.05e-16 &      \\ 
     &           &    3 &  7.17e-09 &  2.05e-16 &      \\ 
     &           &    4 &  7.16e-09 &  2.05e-16 &      \\ 
     &           &    5 &  7.15e-09 &  2.04e-16 &      \\ 
     &           &    6 &  7.14e-09 &  2.04e-16 &      \\ 
     &           &    7 &  7.13e-09 &  2.04e-16 &      \\ 
     &           &    8 &  7.12e-09 &  2.03e-16 &      \\ 
     &           &    9 &  7.11e-09 &  2.03e-16 &      \\ 
     &           &   10 &  7.06e-08 &  2.03e-16 &      \\ 
2923 &  2.92e+04 &   10 &           &           & iters  \\ 
 \hdashline 
     &           &    1 &  6.05e-08 &  1.09e-09 &      \\ 
     &           &    2 &  1.73e-08 &  1.73e-15 &      \\ 
     &           &    3 &  1.73e-08 &  4.94e-16 &      \\ 
     &           &    4 &  1.73e-08 &  4.93e-16 &      \\ 
     &           &    5 &  1.72e-08 &  4.92e-16 &      \\ 
     &           &    6 &  6.05e-08 &  4.92e-16 &      \\ 
     &           &    7 &  1.73e-08 &  1.73e-15 &      \\ 
     &           &    8 &  1.73e-08 &  4.93e-16 &      \\ 
     &           &    9 &  1.72e-08 &  4.93e-16 &      \\ 
     &           &   10 &  6.05e-08 &  4.92e-16 &      \\ 
2924 &  2.92e+04 &   10 &           &           & iters  \\ 
 \hdashline 
     &           &    1 &  5.01e-08 &  1.35e-09 &      \\ 
     &           &    2 &  2.77e-08 &  1.43e-15 &      \\ 
     &           &    3 &  5.00e-08 &  7.91e-16 &      \\ 
     &           &    4 &  2.77e-08 &  1.43e-15 &      \\ 
     &           &    5 &  5.00e-08 &  7.92e-16 &      \\ 
     &           &    6 &  2.78e-08 &  1.43e-15 &      \\ 
     &           &    7 &  2.77e-08 &  7.93e-16 &      \\ 
     &           &    8 &  5.00e-08 &  7.91e-16 &      \\ 
     &           &    9 &  2.78e-08 &  1.43e-15 &      \\ 
     &           &   10 &  2.77e-08 &  7.92e-16 &      \\ 
2925 &  2.92e+04 &   10 &           &           & iters  \\ 
 \hdashline 
     &           &    1 &  3.18e-08 &  2.10e-09 &      \\ 
     &           &    2 &  3.17e-08 &  9.07e-16 &      \\ 
     &           &    3 &  4.60e-08 &  9.06e-16 &      \\ 
     &           &    4 &  3.17e-08 &  1.31e-15 &      \\ 
     &           &    5 &  4.60e-08 &  9.06e-16 &      \\ 
     &           &    6 &  3.18e-08 &  1.31e-15 &      \\ 
     &           &    7 &  4.60e-08 &  9.07e-16 &      \\ 
     &           &    8 &  3.18e-08 &  1.31e-15 &      \\ 
     &           &    9 &  3.17e-08 &  9.07e-16 &      \\ 
     &           &   10 &  4.60e-08 &  9.06e-16 &      \\ 
2926 &  2.92e+04 &   10 &           &           & iters  \\ 
 \hdashline 
     &           &    1 &  3.37e-08 &  3.60e-10 &      \\ 
     &           &    2 &  4.40e-08 &  9.63e-16 &      \\ 
     &           &    3 &  3.38e-08 &  1.26e-15 &      \\ 
     &           &    4 &  4.40e-08 &  9.63e-16 &      \\ 
     &           &    5 &  3.38e-08 &  1.26e-15 &      \\ 
     &           &    6 &  4.40e-08 &  9.64e-16 &      \\ 
     &           &    7 &  3.38e-08 &  1.25e-15 &      \\ 
     &           &    8 &  3.37e-08 &  9.64e-16 &      \\ 
     &           &    9 &  4.40e-08 &  9.63e-16 &      \\ 
     &           &   10 &  3.38e-08 &  1.26e-15 &      \\ 
2927 &  2.93e+04 &   10 &           &           & iters  \\ 
 \hdashline 
     &           &    1 &  5.41e-08 &  1.36e-09 &      \\ 
     &           &    2 &  2.36e-08 &  1.55e-15 &      \\ 
     &           &    3 &  2.36e-08 &  6.74e-16 &      \\ 
     &           &    4 &  5.41e-08 &  6.73e-16 &      \\ 
     &           &    5 &  2.36e-08 &  1.54e-15 &      \\ 
     &           &    6 &  2.36e-08 &  6.74e-16 &      \\ 
     &           &    7 &  5.41e-08 &  6.74e-16 &      \\ 
     &           &    8 &  2.36e-08 &  1.54e-15 &      \\ 
     &           &    9 &  2.36e-08 &  6.75e-16 &      \\ 
     &           &   10 &  5.41e-08 &  6.74e-16 &      \\ 
2928 &  2.93e+04 &   10 &           &           & iters  \\ 
 \hdashline 
     &           &    1 &  6.66e-08 &  2.08e-09 &      \\ 
     &           &    2 &  1.12e-08 &  1.90e-15 &      \\ 
     &           &    3 &  1.12e-08 &  3.19e-16 &      \\ 
     &           &    4 &  1.12e-08 &  3.19e-16 &      \\ 
     &           &    5 &  1.11e-08 &  3.18e-16 &      \\ 
     &           &    6 &  6.66e-08 &  3.18e-16 &      \\ 
     &           &    7 &  8.89e-08 &  1.90e-15 &      \\ 
     &           &    8 &  6.66e-08 &  2.54e-15 &      \\ 
     &           &    9 &  1.12e-08 &  1.90e-15 &      \\ 
     &           &   10 &  1.12e-08 &  3.19e-16 &      \\ 
2929 &  2.93e+04 &   10 &           &           & iters  \\ 
 \hdashline 
     &           &    1 &  8.77e-09 &  1.99e-09 &      \\ 
     &           &    2 &  3.01e-08 &  2.50e-16 &      \\ 
     &           &    3 &  8.80e-09 &  8.59e-16 &      \\ 
     &           &    4 &  8.79e-09 &  2.51e-16 &      \\ 
     &           &    5 &  8.78e-09 &  2.51e-16 &      \\ 
     &           &    6 &  8.77e-09 &  2.51e-16 &      \\ 
     &           &    7 &  3.01e-08 &  2.50e-16 &      \\ 
     &           &    8 &  8.80e-09 &  8.59e-16 &      \\ 
     &           &    9 &  8.78e-09 &  2.51e-16 &      \\ 
     &           &   10 &  8.77e-09 &  2.51e-16 &      \\ 
2930 &  2.93e+04 &   10 &           &           & iters  \\ 
 \hdashline 
     &           &    1 &  1.04e-08 &  7.53e-10 &      \\ 
     &           &    2 &  1.03e-08 &  2.96e-16 &      \\ 
     &           &    3 &  1.03e-08 &  2.95e-16 &      \\ 
     &           &    4 &  2.85e-08 &  2.95e-16 &      \\ 
     &           &    5 &  1.04e-08 &  8.14e-16 &      \\ 
     &           &    6 &  1.03e-08 &  2.96e-16 &      \\ 
     &           &    7 &  2.85e-08 &  2.95e-16 &      \\ 
     &           &    8 &  1.04e-08 &  8.14e-16 &      \\ 
     &           &    9 &  1.04e-08 &  2.96e-16 &      \\ 
     &           &   10 &  1.03e-08 &  2.96e-16 &      \\ 
2931 &  2.93e+04 &   10 &           &           & iters  \\ 
 \hdashline 
     &           &    1 &  2.11e-08 &  1.42e-09 &      \\ 
     &           &    2 &  5.66e-08 &  6.03e-16 &      \\ 
     &           &    3 &  2.12e-08 &  1.61e-15 &      \\ 
     &           &    4 &  2.12e-08 &  6.05e-16 &      \\ 
     &           &    5 &  5.66e-08 &  6.04e-16 &      \\ 
     &           &    6 &  2.12e-08 &  1.61e-15 &      \\ 
     &           &    7 &  2.12e-08 &  6.05e-16 &      \\ 
     &           &    8 &  2.11e-08 &  6.04e-16 &      \\ 
     &           &    9 &  5.66e-08 &  6.03e-16 &      \\ 
     &           &   10 &  2.12e-08 &  1.61e-15 &      \\ 
2932 &  2.93e+04 &   10 &           &           & iters  \\ 
 \hdashline 
     &           &    1 &  3.25e-08 &  8.17e-10 &      \\ 
     &           &    2 &  4.53e-08 &  9.26e-16 &      \\ 
     &           &    3 &  3.25e-08 &  1.29e-15 &      \\ 
     &           &    4 &  4.53e-08 &  9.27e-16 &      \\ 
     &           &    5 &  3.25e-08 &  1.29e-15 &      \\ 
     &           &    6 &  4.52e-08 &  9.27e-16 &      \\ 
     &           &    7 &  3.25e-08 &  1.29e-15 &      \\ 
     &           &    8 &  3.25e-08 &  9.28e-16 &      \\ 
     &           &    9 &  4.53e-08 &  9.27e-16 &      \\ 
     &           &   10 &  3.25e-08 &  1.29e-15 &      \\ 
2933 &  2.93e+04 &   10 &           &           & iters  \\ 
 \hdashline 
     &           &    1 &  3.86e-08 &  1.56e-09 &      \\ 
     &           &    2 &  3.91e-08 &  1.10e-15 &      \\ 
     &           &    3 &  3.86e-08 &  1.12e-15 &      \\ 
     &           &    4 &  3.91e-08 &  1.10e-15 &      \\ 
     &           &    5 &  3.86e-08 &  1.12e-15 &      \\ 
     &           &    6 &  3.91e-08 &  1.10e-15 &      \\ 
     &           &    7 &  3.86e-08 &  1.12e-15 &      \\ 
     &           &    8 &  3.91e-08 &  1.10e-15 &      \\ 
     &           &    9 &  3.86e-08 &  1.12e-15 &      \\ 
     &           &   10 &  3.91e-08 &  1.10e-15 &      \\ 
2934 &  2.93e+04 &   10 &           &           & iters  \\ 
 \hdashline 
     &           &    1 &  7.30e-09 &  9.62e-10 &      \\ 
     &           &    2 &  7.29e-09 &  2.08e-16 &      \\ 
     &           &    3 &  7.28e-09 &  2.08e-16 &      \\ 
     &           &    4 &  7.27e-09 &  2.08e-16 &      \\ 
     &           &    5 &  7.26e-09 &  2.07e-16 &      \\ 
     &           &    6 &  7.04e-08 &  2.07e-16 &      \\ 
     &           &    7 &  8.50e-08 &  2.01e-15 &      \\ 
     &           &    8 &  7.05e-08 &  2.43e-15 &      \\ 
     &           &    9 &  7.33e-09 &  2.01e-15 &      \\ 
     &           &   10 &  7.32e-09 &  2.09e-16 &      \\ 
2935 &  2.93e+04 &   10 &           &           & iters  \\ 
 \hdashline 
     &           &    1 &  1.20e-08 &  1.18e-09 &      \\ 
     &           &    2 &  1.20e-08 &  3.43e-16 &      \\ 
     &           &    3 &  1.20e-08 &  3.43e-16 &      \\ 
     &           &    4 &  1.20e-08 &  3.42e-16 &      \\ 
     &           &    5 &  6.57e-08 &  3.42e-16 &      \\ 
     &           &    6 &  1.20e-08 &  1.88e-15 &      \\ 
     &           &    7 &  1.20e-08 &  3.44e-16 &      \\ 
     &           &    8 &  1.20e-08 &  3.43e-16 &      \\ 
     &           &    9 &  1.20e-08 &  3.43e-16 &      \\ 
     &           &   10 &  1.20e-08 &  3.42e-16 &      \\ 
2936 &  2.94e+04 &   10 &           &           & iters  \\ 
 \hdashline 
     &           &    1 &  1.84e-08 &  2.31e-09 &      \\ 
     &           &    2 &  1.83e-08 &  5.24e-16 &      \\ 
     &           &    3 &  2.05e-08 &  5.24e-16 &      \\ 
     &           &    4 &  1.83e-08 &  5.86e-16 &      \\ 
     &           &    5 &  2.05e-08 &  5.24e-16 &      \\ 
     &           &    6 &  1.84e-08 &  5.86e-16 &      \\ 
     &           &    7 &  2.05e-08 &  5.24e-16 &      \\ 
     &           &    8 &  1.84e-08 &  5.86e-16 &      \\ 
     &           &    9 &  1.83e-08 &  5.24e-16 &      \\ 
     &           &   10 &  2.05e-08 &  5.23e-16 &      \\ 
2937 &  2.94e+04 &   10 &           &           & iters  \\ 
 \hdashline 
     &           &    1 &  4.68e-08 &  2.19e-09 &      \\ 
     &           &    2 &  3.10e-08 &  1.34e-15 &      \\ 
     &           &    3 &  4.68e-08 &  8.84e-16 &      \\ 
     &           &    4 &  3.10e-08 &  1.33e-15 &      \\ 
     &           &    5 &  3.10e-08 &  8.85e-16 &      \\ 
     &           &    6 &  4.68e-08 &  8.83e-16 &      \\ 
     &           &    7 &  3.10e-08 &  1.34e-15 &      \\ 
     &           &    8 &  3.09e-08 &  8.84e-16 &      \\ 
     &           &    9 &  4.68e-08 &  8.83e-16 &      \\ 
     &           &   10 &  3.10e-08 &  1.34e-15 &      \\ 
2938 &  2.94e+04 &   10 &           &           & iters  \\ 
 \hdashline 
     &           &    1 &  3.76e-08 &  5.64e-10 &      \\ 
     &           &    2 &  4.02e-08 &  1.07e-15 &      \\ 
     &           &    3 &  3.76e-08 &  1.15e-15 &      \\ 
     &           &    4 &  4.02e-08 &  1.07e-15 &      \\ 
     &           &    5 &  3.76e-08 &  1.15e-15 &      \\ 
     &           &    6 &  4.02e-08 &  1.07e-15 &      \\ 
     &           &    7 &  3.76e-08 &  1.15e-15 &      \\ 
     &           &    8 &  4.02e-08 &  1.07e-15 &      \\ 
     &           &    9 &  3.76e-08 &  1.15e-15 &      \\ 
     &           &   10 &  4.02e-08 &  1.07e-15 &      \\ 
2939 &  2.94e+04 &   10 &           &           & iters  \\ 
 \hdashline 
     &           &    1 &  5.57e-08 &  8.91e-10 &      \\ 
     &           &    2 &  2.21e-08 &  1.59e-15 &      \\ 
     &           &    3 &  2.21e-08 &  6.30e-16 &      \\ 
     &           &    4 &  5.57e-08 &  6.30e-16 &      \\ 
     &           &    5 &  2.21e-08 &  1.59e-15 &      \\ 
     &           &    6 &  2.21e-08 &  6.31e-16 &      \\ 
     &           &    7 &  2.20e-08 &  6.30e-16 &      \\ 
     &           &    8 &  5.57e-08 &  6.29e-16 &      \\ 
     &           &    9 &  2.21e-08 &  1.59e-15 &      \\ 
     &           &   10 &  2.21e-08 &  6.30e-16 &      \\ 
2940 &  2.94e+04 &   10 &           &           & iters  \\ 
 \hdashline 
     &           &    1 &  3.13e-09 &  1.03e-09 &      \\ 
     &           &    2 &  3.12e-09 &  8.92e-17 &      \\ 
     &           &    3 &  3.12e-09 &  8.91e-17 &      \\ 
     &           &    4 &  3.11e-09 &  8.90e-17 &      \\ 
     &           &    5 &  3.11e-09 &  8.88e-17 &      \\ 
     &           &    6 &  3.10e-09 &  8.87e-17 &      \\ 
     &           &    7 &  3.10e-09 &  8.86e-17 &      \\ 
     &           &    8 &  3.09e-09 &  8.85e-17 &      \\ 
     &           &    9 &  3.09e-09 &  8.83e-17 &      \\ 
     &           &   10 &  3.09e-09 &  8.82e-17 &      \\ 
2941 &  2.94e+04 &   10 &           &           & iters  \\ 
 \hdashline 
     &           &    1 &  8.19e-09 &  2.38e-10 &      \\ 
     &           &    2 &  8.18e-09 &  2.34e-16 &      \\ 
     &           &    3 &  8.17e-09 &  2.33e-16 &      \\ 
     &           &    4 &  8.16e-09 &  2.33e-16 &      \\ 
     &           &    5 &  8.14e-09 &  2.33e-16 &      \\ 
     &           &    6 &  8.13e-09 &  2.32e-16 &      \\ 
     &           &    7 &  8.12e-09 &  2.32e-16 &      \\ 
     &           &    8 &  6.96e-08 &  2.32e-16 &      \\ 
     &           &    9 &  8.59e-08 &  1.99e-15 &      \\ 
     &           &   10 &  6.96e-08 &  2.45e-15 &      \\ 
2942 &  2.94e+04 &   10 &           &           & iters  \\ 
 \hdashline 
     &           &    1 &  6.33e-08 &  5.58e-10 &      \\ 
     &           &    2 &  1.45e-08 &  1.81e-15 &      \\ 
     &           &    3 &  1.45e-08 &  4.14e-16 &      \\ 
     &           &    4 &  1.45e-08 &  4.13e-16 &      \\ 
     &           &    5 &  6.32e-08 &  4.13e-16 &      \\ 
     &           &    6 &  1.45e-08 &  1.81e-15 &      \\ 
     &           &    7 &  1.45e-08 &  4.15e-16 &      \\ 
     &           &    8 &  1.45e-08 &  4.14e-16 &      \\ 
     &           &    9 &  1.45e-08 &  4.13e-16 &      \\ 
     &           &   10 &  6.32e-08 &  4.13e-16 &      \\ 
2943 &  2.94e+04 &   10 &           &           & iters  \\ 
 \hdashline 
     &           &    1 &  5.55e-08 &  1.34e-10 &      \\ 
     &           &    2 &  2.23e-08 &  1.58e-15 &      \\ 
     &           &    3 &  5.55e-08 &  6.35e-16 &      \\ 
     &           &    4 &  2.23e-08 &  1.58e-15 &      \\ 
     &           &    5 &  2.23e-08 &  6.37e-16 &      \\ 
     &           &    6 &  5.54e-08 &  6.36e-16 &      \\ 
     &           &    7 &  2.23e-08 &  1.58e-15 &      \\ 
     &           &    8 &  2.23e-08 &  6.37e-16 &      \\ 
     &           &    9 &  2.23e-08 &  6.36e-16 &      \\ 
     &           &   10 &  5.55e-08 &  6.35e-16 &      \\ 
2944 &  2.94e+04 &   10 &           &           & iters  \\ 
 \hdashline 
     &           &    1 &  2.71e-08 &  7.40e-10 &      \\ 
     &           &    2 &  2.71e-08 &  7.74e-16 &      \\ 
     &           &    3 &  5.06e-08 &  7.73e-16 &      \\ 
     &           &    4 &  2.71e-08 &  1.45e-15 &      \\ 
     &           &    5 &  2.71e-08 &  7.74e-16 &      \\ 
     &           &    6 &  5.06e-08 &  7.73e-16 &      \\ 
     &           &    7 &  2.71e-08 &  1.45e-15 &      \\ 
     &           &    8 &  2.71e-08 &  7.74e-16 &      \\ 
     &           &    9 &  5.07e-08 &  7.73e-16 &      \\ 
     &           &   10 &  2.71e-08 &  1.45e-15 &      \\ 
2945 &  2.94e+04 &   10 &           &           & iters  \\ 
 \hdashline 
     &           &    1 &  5.69e-08 &  9.93e-12 &      \\ 
     &           &    2 &  2.09e-08 &  1.62e-15 &      \\ 
     &           &    3 &  2.09e-08 &  5.96e-16 &      \\ 
     &           &    4 &  2.08e-08 &  5.95e-16 &      \\ 
     &           &    5 &  5.69e-08 &  5.95e-16 &      \\ 
     &           &    6 &  2.09e-08 &  1.62e-15 &      \\ 
     &           &    7 &  2.09e-08 &  5.96e-16 &      \\ 
     &           &    8 &  2.08e-08 &  5.95e-16 &      \\ 
     &           &    9 &  5.69e-08 &  5.94e-16 &      \\ 
     &           &   10 &  2.09e-08 &  1.62e-15 &      \\ 
2946 &  2.94e+04 &   10 &           &           & iters  \\ 
 \hdashline 
     &           &    1 &  2.09e-08 &  5.55e-10 &      \\ 
     &           &    2 &  2.09e-08 &  5.98e-16 &      \\ 
     &           &    3 &  2.09e-08 &  5.97e-16 &      \\ 
     &           &    4 &  5.68e-08 &  5.96e-16 &      \\ 
     &           &    5 &  2.09e-08 &  1.62e-15 &      \\ 
     &           &    6 &  2.09e-08 &  5.97e-16 &      \\ 
     &           &    7 &  2.09e-08 &  5.96e-16 &      \\ 
     &           &    8 &  5.69e-08 &  5.96e-16 &      \\ 
     &           &    9 &  2.09e-08 &  1.62e-15 &      \\ 
     &           &   10 &  2.09e-08 &  5.97e-16 &      \\ 
2947 &  2.95e+04 &   10 &           &           & iters  \\ 
 \hdashline 
     &           &    1 &  7.42e-08 &  5.96e-10 &      \\ 
     &           &    2 &  3.62e-09 &  2.12e-15 &      \\ 
     &           &    3 &  3.61e-09 &  1.03e-16 &      \\ 
     &           &    4 &  3.61e-09 &  1.03e-16 &      \\ 
     &           &    5 &  3.60e-09 &  1.03e-16 &      \\ 
     &           &    6 &  3.60e-09 &  1.03e-16 &      \\ 
     &           &    7 &  3.59e-09 &  1.03e-16 &      \\ 
     &           &    8 &  3.59e-09 &  1.03e-16 &      \\ 
     &           &    9 &  3.58e-09 &  1.02e-16 &      \\ 
     &           &   10 &  3.58e-09 &  1.02e-16 &      \\ 
2948 &  2.95e+04 &   10 &           &           & iters  \\ 
 \hdashline 
     &           &    1 &  2.19e-08 &  6.18e-10 &      \\ 
     &           &    2 &  2.19e-08 &  6.25e-16 &      \\ 
     &           &    3 &  5.59e-08 &  6.24e-16 &      \\ 
     &           &    4 &  2.19e-08 &  1.59e-15 &      \\ 
     &           &    5 &  2.19e-08 &  6.25e-16 &      \\ 
     &           &    6 &  2.18e-08 &  6.24e-16 &      \\ 
     &           &    7 &  5.59e-08 &  6.23e-16 &      \\ 
     &           &    8 &  2.19e-08 &  1.59e-15 &      \\ 
     &           &    9 &  2.19e-08 &  6.25e-16 &      \\ 
     &           &   10 &  2.18e-08 &  6.24e-16 &      \\ 
2949 &  2.95e+04 &   10 &           &           & iters  \\ 
 \hdashline 
     &           &    1 &  6.50e-08 &  4.33e-10 &      \\ 
     &           &    2 &  1.28e-08 &  1.86e-15 &      \\ 
     &           &    3 &  1.28e-08 &  3.65e-16 &      \\ 
     &           &    4 &  1.27e-08 &  3.64e-16 &      \\ 
     &           &    5 &  1.27e-08 &  3.64e-16 &      \\ 
     &           &    6 &  1.27e-08 &  3.63e-16 &      \\ 
     &           &    7 &  6.50e-08 &  3.62e-16 &      \\ 
     &           &    8 &  1.28e-08 &  1.86e-15 &      \\ 
     &           &    9 &  1.28e-08 &  3.65e-16 &      \\ 
     &           &   10 &  1.27e-08 &  3.64e-16 &      \\ 
2950 &  2.95e+04 &   10 &           &           & iters  \\ 
 \hdashline 
     &           &    1 &  4.05e-08 &  2.78e-10 &      \\ 
     &           &    2 &  3.73e-08 &  1.16e-15 &      \\ 
     &           &    3 &  4.05e-08 &  1.06e-15 &      \\ 
     &           &    4 &  3.73e-08 &  1.16e-15 &      \\ 
     &           &    5 &  4.05e-08 &  1.06e-15 &      \\ 
     &           &    6 &  3.73e-08 &  1.16e-15 &      \\ 
     &           &    7 &  3.72e-08 &  1.06e-15 &      \\ 
     &           &    8 &  4.05e-08 &  1.06e-15 &      \\ 
     &           &    9 &  3.72e-08 &  1.16e-15 &      \\ 
     &           &   10 &  4.05e-08 &  1.06e-15 &      \\ 
2951 &  2.95e+04 &   10 &           &           & iters  \\ 
 \hdashline 
     &           &    1 &  1.27e-09 &  9.84e-10 &      \\ 
     &           &    2 &  1.27e-09 &  3.63e-17 &      \\ 
     &           &    3 &  1.27e-09 &  3.62e-17 &      \\ 
     &           &    4 &  1.27e-09 &  3.62e-17 &      \\ 
     &           &    5 &  1.26e-09 &  3.61e-17 &      \\ 
     &           &    6 &  1.26e-09 &  3.61e-17 &      \\ 
     &           &    7 &  1.26e-09 &  3.60e-17 &      \\ 
     &           &    8 &  1.26e-09 &  3.60e-17 &      \\ 
     &           &    9 &  1.26e-09 &  3.59e-17 &      \\ 
     &           &   10 &  1.26e-09 &  3.59e-17 &      \\ 
2952 &  2.95e+04 &   10 &           &           & iters  \\ 
 \hdashline 
     &           &    1 &  3.53e-08 &  8.90e-10 &      \\ 
     &           &    2 &  4.25e-08 &  1.01e-15 &      \\ 
     &           &    3 &  3.53e-08 &  1.21e-15 &      \\ 
     &           &    4 &  4.25e-08 &  1.01e-15 &      \\ 
     &           &    5 &  3.53e-08 &  1.21e-15 &      \\ 
     &           &    6 &  4.25e-08 &  1.01e-15 &      \\ 
     &           &    7 &  3.53e-08 &  1.21e-15 &      \\ 
     &           &    8 &  4.25e-08 &  1.01e-15 &      \\ 
     &           &    9 &  3.53e-08 &  1.21e-15 &      \\ 
     &           &   10 &  3.52e-08 &  1.01e-15 &      \\ 
2953 &  2.95e+04 &   10 &           &           & iters  \\ 
 \hdashline 
     &           &    1 &  4.28e-08 &  3.23e-10 &      \\ 
     &           &    2 &  3.86e-09 &  1.22e-15 &      \\ 
     &           &    3 &  3.85e-09 &  1.10e-16 &      \\ 
     &           &    4 &  3.85e-09 &  1.10e-16 &      \\ 
     &           &    5 &  3.84e-09 &  1.10e-16 &      \\ 
     &           &    6 &  3.84e-09 &  1.10e-16 &      \\ 
     &           &    7 &  3.83e-09 &  1.09e-16 &      \\ 
     &           &    8 &  3.83e-09 &  1.09e-16 &      \\ 
     &           &    9 &  3.82e-09 &  1.09e-16 &      \\ 
     &           &   10 &  3.81e-09 &  1.09e-16 &      \\ 
2954 &  2.95e+04 &   10 &           &           & iters  \\ 
 \hdashline 
     &           &    1 &  3.29e-08 &  4.98e-11 &      \\ 
     &           &    2 &  3.28e-08 &  9.39e-16 &      \\ 
     &           &    3 &  4.49e-08 &  9.37e-16 &      \\ 
     &           &    4 &  3.29e-08 &  1.28e-15 &      \\ 
     &           &    5 &  4.49e-08 &  9.38e-16 &      \\ 
     &           &    6 &  3.29e-08 &  1.28e-15 &      \\ 
     &           &    7 &  3.28e-08 &  9.38e-16 &      \\ 
     &           &    8 &  4.49e-08 &  9.37e-16 &      \\ 
     &           &    9 &  3.28e-08 &  1.28e-15 &      \\ 
     &           &   10 &  4.49e-08 &  9.37e-16 &      \\ 
2955 &  2.95e+04 &   10 &           &           & iters  \\ 
 \hdashline 
     &           &    1 &  8.79e-09 &  2.33e-10 &      \\ 
     &           &    2 &  3.01e-08 &  2.51e-16 &      \\ 
     &           &    3 &  8.82e-09 &  8.58e-16 &      \\ 
     &           &    4 &  8.80e-09 &  2.52e-16 &      \\ 
     &           &    5 &  8.79e-09 &  2.51e-16 &      \\ 
     &           &    6 &  3.01e-08 &  2.51e-16 &      \\ 
     &           &    7 &  8.82e-09 &  8.58e-16 &      \\ 
     &           &    8 &  8.81e-09 &  2.52e-16 &      \\ 
     &           &    9 &  8.80e-09 &  2.51e-16 &      \\ 
     &           &   10 &  3.01e-08 &  2.51e-16 &      \\ 
2956 &  2.96e+04 &   10 &           &           & iters  \\ 
 \hdashline 
     &           &    1 &  8.83e-09 &  8.69e-10 &      \\ 
     &           &    2 &  8.82e-09 &  2.52e-16 &      \\ 
     &           &    3 &  8.80e-09 &  2.52e-16 &      \\ 
     &           &    4 &  3.01e-08 &  2.51e-16 &      \\ 
     &           &    5 &  8.83e-09 &  8.58e-16 &      \\ 
     &           &    6 &  8.82e-09 &  2.52e-16 &      \\ 
     &           &    7 &  8.81e-09 &  2.52e-16 &      \\ 
     &           &    8 &  3.00e-08 &  2.51e-16 &      \\ 
     &           &    9 &  8.84e-09 &  8.58e-16 &      \\ 
     &           &   10 &  8.83e-09 &  2.52e-16 &      \\ 
2957 &  2.96e+04 &   10 &           &           & iters  \\ 
 \hdashline 
     &           &    1 &  1.89e-08 &  9.82e-10 &      \\ 
     &           &    2 &  5.88e-08 &  5.39e-16 &      \\ 
     &           &    3 &  1.90e-08 &  1.68e-15 &      \\ 
     &           &    4 &  1.89e-08 &  5.41e-16 &      \\ 
     &           &    5 &  1.89e-08 &  5.40e-16 &      \\ 
     &           &    6 &  5.88e-08 &  5.39e-16 &      \\ 
     &           &    7 &  1.90e-08 &  1.68e-15 &      \\ 
     &           &    8 &  1.89e-08 &  5.41e-16 &      \\ 
     &           &    9 &  1.89e-08 &  5.40e-16 &      \\ 
     &           &   10 &  5.88e-08 &  5.39e-16 &      \\ 
2958 &  2.96e+04 &   10 &           &           & iters  \\ 
 \hdashline 
     &           &    1 &  6.19e-08 &  5.93e-10 &      \\ 
     &           &    2 &  1.58e-08 &  1.77e-15 &      \\ 
     &           &    3 &  1.58e-08 &  4.52e-16 &      \\ 
     &           &    4 &  1.58e-08 &  4.51e-16 &      \\ 
     &           &    5 &  6.19e-08 &  4.51e-16 &      \\ 
     &           &    6 &  1.59e-08 &  1.77e-15 &      \\ 
     &           &    7 &  1.58e-08 &  4.53e-16 &      \\ 
     &           &    8 &  1.58e-08 &  4.52e-16 &      \\ 
     &           &    9 &  1.58e-08 &  4.51e-16 &      \\ 
     &           &   10 &  6.19e-08 &  4.51e-16 &      \\ 
2959 &  2.96e+04 &   10 &           &           & iters  \\ 
 \hdashline 
     &           &    1 &  4.35e-08 &  2.66e-10 &      \\ 
     &           &    2 &  3.42e-08 &  1.24e-15 &      \\ 
     &           &    3 &  4.35e-08 &  9.76e-16 &      \\ 
     &           &    4 &  3.42e-08 &  1.24e-15 &      \\ 
     &           &    5 &  4.35e-08 &  9.77e-16 &      \\ 
     &           &    6 &  3.42e-08 &  1.24e-15 &      \\ 
     &           &    7 &  3.42e-08 &  9.77e-16 &      \\ 
     &           &    8 &  4.36e-08 &  9.76e-16 &      \\ 
     &           &    9 &  3.42e-08 &  1.24e-15 &      \\ 
     &           &   10 &  4.35e-08 &  9.76e-16 &      \\ 
2960 &  2.96e+04 &   10 &           &           & iters  \\ 
 \hdashline 
     &           &    1 &  3.22e-08 &  1.24e-09 &      \\ 
     &           &    2 &  4.55e-08 &  9.19e-16 &      \\ 
     &           &    3 &  3.22e-08 &  1.30e-15 &      \\ 
     &           &    4 &  4.55e-08 &  9.19e-16 &      \\ 
     &           &    5 &  3.22e-08 &  1.30e-15 &      \\ 
     &           &    6 &  3.22e-08 &  9.20e-16 &      \\ 
     &           &    7 &  4.56e-08 &  9.18e-16 &      \\ 
     &           &    8 &  3.22e-08 &  1.30e-15 &      \\ 
     &           &    9 &  4.55e-08 &  9.19e-16 &      \\ 
     &           &   10 &  3.22e-08 &  1.30e-15 &      \\ 
2961 &  2.96e+04 &   10 &           &           & iters  \\ 
 \hdashline 
     &           &    1 &  2.58e-10 &  8.04e-10 &      \\ 
     &           &    2 &  2.58e-10 &  7.38e-18 &      \\ 
     &           &    3 &  2.58e-10 &  7.36e-18 &      \\ 
     &           &    4 &  2.57e-10 &  7.35e-18 &      \\ 
     &           &    5 &  2.57e-10 &  7.34e-18 &      \\ 
     &           &    6 &  2.57e-10 &  7.33e-18 &      \\ 
     &           &    7 &  2.56e-10 &  7.32e-18 &      \\ 
     &           &    8 &  2.56e-10 &  7.31e-18 &      \\ 
     &           &    9 &  2.55e-10 &  7.30e-18 &      \\ 
     &           &   10 &  2.55e-10 &  7.29e-18 &      \\ 
2962 &  2.96e+04 &   10 &           &           & iters  \\ 
 \hdashline 
     &           &    1 &  3.46e-08 &  4.09e-10 &      \\ 
     &           &    2 &  4.32e-08 &  9.86e-16 &      \\ 
     &           &    3 &  3.46e-08 &  1.23e-15 &      \\ 
     &           &    4 &  4.32e-08 &  9.87e-16 &      \\ 
     &           &    5 &  3.46e-08 &  1.23e-15 &      \\ 
     &           &    6 &  4.32e-08 &  9.87e-16 &      \\ 
     &           &    7 &  3.46e-08 &  1.23e-15 &      \\ 
     &           &    8 &  3.45e-08 &  9.87e-16 &      \\ 
     &           &    9 &  4.32e-08 &  9.86e-16 &      \\ 
     &           &   10 &  3.46e-08 &  1.23e-15 &      \\ 
2963 &  2.96e+04 &   10 &           &           & iters  \\ 
 \hdashline 
     &           &    1 &  9.46e-09 &  1.59e-10 &      \\ 
     &           &    2 &  9.45e-09 &  2.70e-16 &      \\ 
     &           &    3 &  9.44e-09 &  2.70e-16 &      \\ 
     &           &    4 &  9.42e-09 &  2.69e-16 &      \\ 
     &           &    5 &  9.41e-09 &  2.69e-16 &      \\ 
     &           &    6 &  2.94e-08 &  2.69e-16 &      \\ 
     &           &    7 &  9.44e-09 &  8.40e-16 &      \\ 
     &           &    8 &  9.42e-09 &  2.69e-16 &      \\ 
     &           &    9 &  9.41e-09 &  2.69e-16 &      \\ 
     &           &   10 &  2.94e-08 &  2.69e-16 &      \\ 
2964 &  2.96e+04 &   10 &           &           & iters  \\ 
 \hdashline 
     &           &    1 &  1.07e-08 &  6.77e-10 &      \\ 
     &           &    2 &  1.07e-08 &  3.06e-16 &      \\ 
     &           &    3 &  6.70e-08 &  3.06e-16 &      \\ 
     &           &    4 &  1.08e-08 &  1.91e-15 &      \\ 
     &           &    5 &  1.08e-08 &  3.08e-16 &      \\ 
     &           &    6 &  1.08e-08 &  3.08e-16 &      \\ 
     &           &    7 &  1.07e-08 &  3.07e-16 &      \\ 
     &           &    8 &  1.07e-08 &  3.07e-16 &      \\ 
     &           &    9 &  1.07e-08 &  3.06e-16 &      \\ 
     &           &   10 &  6.70e-08 &  3.06e-16 &      \\ 
2965 &  2.96e+04 &   10 &           &           & iters  \\ 
 \hdashline 
     &           &    1 &  1.85e-08 &  3.26e-10 &      \\ 
     &           &    2 &  1.85e-08 &  5.29e-16 &      \\ 
     &           &    3 &  2.04e-08 &  5.28e-16 &      \\ 
     &           &    4 &  1.85e-08 &  5.81e-16 &      \\ 
     &           &    5 &  2.04e-08 &  5.28e-16 &      \\ 
     &           &    6 &  1.85e-08 &  5.81e-16 &      \\ 
     &           &    7 &  2.04e-08 &  5.28e-16 &      \\ 
     &           &    8 &  1.85e-08 &  5.81e-16 &      \\ 
     &           &    9 &  2.04e-08 &  5.28e-16 &      \\ 
     &           &   10 &  1.85e-08 &  5.81e-16 &      \\ 
2966 &  2.96e+04 &   10 &           &           & iters  \\ 
 \hdashline 
     &           &    1 &  1.14e-09 &  2.95e-10 &      \\ 
     &           &    2 &  1.14e-09 &  3.25e-17 &      \\ 
     &           &    3 &  1.14e-09 &  3.25e-17 &      \\ 
     &           &    4 &  1.14e-09 &  3.24e-17 &      \\ 
     &           &    5 &  1.13e-09 &  3.24e-17 &      \\ 
     &           &    6 &  1.13e-09 &  3.23e-17 &      \\ 
     &           &    7 &  1.13e-09 &  3.23e-17 &      \\ 
     &           &    8 &  1.13e-09 &  3.23e-17 &      \\ 
     &           &    9 &  1.13e-09 &  3.22e-17 &      \\ 
     &           &   10 &  1.13e-09 &  3.22e-17 &      \\ 
2967 &  2.97e+04 &   10 &           &           & iters  \\ 
 \hdashline 
     &           &    1 &  4.99e-08 &  5.36e-10 &      \\ 
     &           &    2 &  2.79e-08 &  1.42e-15 &      \\ 
     &           &    3 &  2.79e-08 &  7.96e-16 &      \\ 
     &           &    4 &  4.99e-08 &  7.95e-16 &      \\ 
     &           &    5 &  2.79e-08 &  1.42e-15 &      \\ 
     &           &    6 &  2.78e-08 &  7.96e-16 &      \\ 
     &           &    7 &  4.99e-08 &  7.95e-16 &      \\ 
     &           &    8 &  2.79e-08 &  1.42e-15 &      \\ 
     &           &    9 &  4.98e-08 &  7.96e-16 &      \\ 
     &           &   10 &  2.79e-08 &  1.42e-15 &      \\ 
2968 &  2.97e+04 &   10 &           &           & iters  \\ 
 \hdashline 
     &           &    1 &  2.01e-08 &  8.43e-10 &      \\ 
     &           &    2 &  1.87e-08 &  5.75e-16 &      \\ 
     &           &    3 &  2.01e-08 &  5.34e-16 &      \\ 
     &           &    4 &  1.87e-08 &  5.75e-16 &      \\ 
     &           &    5 &  2.01e-08 &  5.35e-16 &      \\ 
     &           &    6 &  1.87e-08 &  5.75e-16 &      \\ 
     &           &    7 &  2.01e-08 &  5.35e-16 &      \\ 
     &           &    8 &  1.87e-08 &  5.75e-16 &      \\ 
     &           &    9 &  2.01e-08 &  5.35e-16 &      \\ 
     &           &   10 &  1.87e-08 &  5.75e-16 &      \\ 
2969 &  2.97e+04 &   10 &           &           & iters  \\ 
 \hdashline 
     &           &    1 &  3.36e-08 &  1.38e-10 &      \\ 
     &           &    2 &  4.42e-08 &  9.58e-16 &      \\ 
     &           &    3 &  3.36e-08 &  1.26e-15 &      \\ 
     &           &    4 &  4.42e-08 &  9.58e-16 &      \\ 
     &           &    5 &  3.36e-08 &  1.26e-15 &      \\ 
     &           &    6 &  3.35e-08 &  9.59e-16 &      \\ 
     &           &    7 &  4.42e-08 &  9.57e-16 &      \\ 
     &           &    8 &  3.36e-08 &  1.26e-15 &      \\ 
     &           &    9 &  4.42e-08 &  9.58e-16 &      \\ 
     &           &   10 &  3.36e-08 &  1.26e-15 &      \\ 
2970 &  2.97e+04 &   10 &           &           & iters  \\ 
 \hdashline 
     &           &    1 &  2.38e-08 &  9.07e-10 &      \\ 
     &           &    2 &  2.37e-08 &  6.79e-16 &      \\ 
     &           &    3 &  5.40e-08 &  6.78e-16 &      \\ 
     &           &    4 &  2.38e-08 &  1.54e-15 &      \\ 
     &           &    5 &  2.37e-08 &  6.79e-16 &      \\ 
     &           &    6 &  5.40e-08 &  6.78e-16 &      \\ 
     &           &    7 &  2.38e-08 &  1.54e-15 &      \\ 
     &           &    8 &  2.38e-08 &  6.79e-16 &      \\ 
     &           &    9 &  2.37e-08 &  6.78e-16 &      \\ 
     &           &   10 &  5.40e-08 &  6.77e-16 &      \\ 
2971 &  2.97e+04 &   10 &           &           & iters  \\ 
 \hdashline 
     &           &    1 &  4.35e-09 &  4.18e-10 &      \\ 
     &           &    2 &  4.34e-09 &  1.24e-16 &      \\ 
     &           &    3 &  4.34e-09 &  1.24e-16 &      \\ 
     &           &    4 &  4.33e-09 &  1.24e-16 &      \\ 
     &           &    5 &  4.32e-09 &  1.24e-16 &      \\ 
     &           &    6 &  4.32e-09 &  1.23e-16 &      \\ 
     &           &    7 &  7.34e-08 &  1.23e-16 &      \\ 
     &           &    8 &  4.33e-08 &  2.09e-15 &      \\ 
     &           &    9 &  4.35e-09 &  1.23e-15 &      \\ 
     &           &   10 &  4.35e-09 &  1.24e-16 &      \\ 
2972 &  2.97e+04 &   10 &           &           & iters  \\ 
 \hdashline 
     &           &    1 &  7.65e-09 &  2.59e-10 &      \\ 
     &           &    2 &  7.64e-09 &  2.18e-16 &      \\ 
     &           &    3 &  7.63e-09 &  2.18e-16 &      \\ 
     &           &    4 &  7.62e-09 &  2.18e-16 &      \\ 
     &           &    5 &  7.61e-09 &  2.17e-16 &      \\ 
     &           &    6 &  7.01e-08 &  2.17e-16 &      \\ 
     &           &    7 &  8.54e-08 &  2.00e-15 &      \\ 
     &           &    8 &  7.01e-08 &  2.44e-15 &      \\ 
     &           &    9 &  7.68e-09 &  2.00e-15 &      \\ 
     &           &   10 &  7.66e-09 &  2.19e-16 &      \\ 
2973 &  2.97e+04 &   10 &           &           & iters  \\ 
 \hdashline 
     &           &    1 &  7.08e-09 &  7.75e-10 &      \\ 
     &           &    2 &  7.07e-09 &  2.02e-16 &      \\ 
     &           &    3 &  7.06e-09 &  2.02e-16 &      \\ 
     &           &    4 &  7.05e-09 &  2.01e-16 &      \\ 
     &           &    5 &  7.04e-09 &  2.01e-16 &      \\ 
     &           &    6 &  7.03e-09 &  2.01e-16 &      \\ 
     &           &    7 &  7.02e-09 &  2.01e-16 &      \\ 
     &           &    8 &  7.01e-09 &  2.00e-16 &      \\ 
     &           &    9 &  7.00e-09 &  2.00e-16 &      \\ 
     &           &   10 &  6.99e-09 &  2.00e-16 &      \\ 
2974 &  2.97e+04 &   10 &           &           & iters  \\ 
 \hdashline 
     &           &    1 &  5.98e-09 &  3.00e-10 &      \\ 
     &           &    2 &  7.17e-08 &  1.71e-16 &      \\ 
     &           &    3 &  6.08e-09 &  2.05e-15 &      \\ 
     &           &    4 &  6.07e-09 &  1.73e-16 &      \\ 
     &           &    5 &  6.06e-09 &  1.73e-16 &      \\ 
     &           &    6 &  6.05e-09 &  1.73e-16 &      \\ 
     &           &    7 &  6.04e-09 &  1.73e-16 &      \\ 
     &           &    8 &  6.03e-09 &  1.72e-16 &      \\ 
     &           &    9 &  6.03e-09 &  1.72e-16 &      \\ 
     &           &   10 &  6.02e-09 &  1.72e-16 &      \\ 
2975 &  2.97e+04 &   10 &           &           & iters  \\ 
 \hdashline 
     &           &    1 &  3.48e-08 &  1.75e-10 &      \\ 
     &           &    2 &  3.47e-08 &  9.92e-16 &      \\ 
     &           &    3 &  4.30e-08 &  9.91e-16 &      \\ 
     &           &    4 &  3.47e-08 &  1.23e-15 &      \\ 
     &           &    5 &  3.47e-08 &  9.91e-16 &      \\ 
     &           &    6 &  4.31e-08 &  9.90e-16 &      \\ 
     &           &    7 &  3.47e-08 &  1.23e-15 &      \\ 
     &           &    8 &  4.31e-08 &  9.90e-16 &      \\ 
     &           &    9 &  3.47e-08 &  1.23e-15 &      \\ 
     &           &   10 &  4.30e-08 &  9.90e-16 &      \\ 
2976 &  2.98e+04 &   10 &           &           & iters  \\ 
 \hdashline 
     &           &    1 &  2.11e-09 &  8.96e-11 &      \\ 
     &           &    2 &  2.10e-09 &  6.02e-17 &      \\ 
     &           &    3 &  2.10e-09 &  6.01e-17 &      \\ 
     &           &    4 &  2.10e-09 &  6.00e-17 &      \\ 
     &           &    5 &  2.10e-09 &  5.99e-17 &      \\ 
     &           &    6 &  2.09e-09 &  5.98e-17 &      \\ 
     &           &    7 &  2.09e-09 &  5.97e-17 &      \\ 
     &           &    8 &  2.09e-09 &  5.96e-17 &      \\ 
     &           &    9 &  2.08e-09 &  5.95e-17 &      \\ 
     &           &   10 &  2.08e-09 &  5.95e-17 &      \\ 
2977 &  2.98e+04 &   10 &           &           & iters  \\ 
 \hdashline 
     &           &    1 &  6.77e-08 &  1.92e-10 &      \\ 
     &           &    2 &  1.00e-08 &  1.93e-15 &      \\ 
     &           &    3 &  1.00e-08 &  2.87e-16 &      \\ 
     &           &    4 &  1.00e-08 &  2.86e-16 &      \\ 
     &           &    5 &  1.00e-08 &  2.86e-16 &      \\ 
     &           &    6 &  9.99e-09 &  2.86e-16 &      \\ 
     &           &    7 &  9.98e-09 &  2.85e-16 &      \\ 
     &           &    8 &  6.77e-08 &  2.85e-16 &      \\ 
     &           &    9 &  1.01e-08 &  1.93e-15 &      \\ 
     &           &   10 &  1.00e-08 &  2.87e-16 &      \\ 
2978 &  2.98e+04 &   10 &           &           & iters  \\ 
 \hdashline 
     &           &    1 &  6.09e-08 &  7.60e-11 &      \\ 
     &           &    2 &  1.69e-08 &  1.74e-15 &      \\ 
     &           &    3 &  1.68e-08 &  4.81e-16 &      \\ 
     &           &    4 &  1.68e-08 &  4.81e-16 &      \\ 
     &           &    5 &  1.68e-08 &  4.80e-16 &      \\ 
     &           &    6 &  6.09e-08 &  4.79e-16 &      \\ 
     &           &    7 &  1.69e-08 &  1.74e-15 &      \\ 
     &           &    8 &  1.68e-08 &  4.81e-16 &      \\ 
     &           &    9 &  1.68e-08 &  4.80e-16 &      \\ 
     &           &   10 &  1.68e-08 &  4.80e-16 &      \\ 
2979 &  2.98e+04 &   10 &           &           & iters  \\ 
 \hdashline 
     &           &    1 &  2.76e-08 &  6.24e-11 &      \\ 
     &           &    2 &  2.75e-08 &  7.87e-16 &      \\ 
     &           &    3 &  2.75e-08 &  7.85e-16 &      \\ 
     &           &    4 &  5.02e-08 &  7.84e-16 &      \\ 
     &           &    5 &  2.75e-08 &  1.43e-15 &      \\ 
     &           &    6 &  2.75e-08 &  7.85e-16 &      \\ 
     &           &    7 &  5.03e-08 &  7.84e-16 &      \\ 
     &           &    8 &  2.75e-08 &  1.43e-15 &      \\ 
     &           &    9 &  2.75e-08 &  7.85e-16 &      \\ 
     &           &   10 &  5.03e-08 &  7.84e-16 &      \\ 
2980 &  2.98e+04 &   10 &           &           & iters  \\ 
 \hdashline 
     &           &    1 &  1.21e-08 &  5.37e-10 &      \\ 
     &           &    2 &  1.21e-08 &  3.46e-16 &      \\ 
     &           &    3 &  1.21e-08 &  3.46e-16 &      \\ 
     &           &    4 &  6.56e-08 &  3.45e-16 &      \\ 
     &           &    5 &  1.22e-08 &  1.87e-15 &      \\ 
     &           &    6 &  1.22e-08 &  3.47e-16 &      \\ 
     &           &    7 &  1.21e-08 &  3.47e-16 &      \\ 
     &           &    8 &  1.21e-08 &  3.46e-16 &      \\ 
     &           &    9 &  1.21e-08 &  3.46e-16 &      \\ 
     &           &   10 &  6.56e-08 &  3.45e-16 &      \\ 
2981 &  2.98e+04 &   10 &           &           & iters  \\ 
 \hdashline 
     &           &    1 &  4.71e-08 &  1.40e-09 &      \\ 
     &           &    2 &  8.23e-09 &  1.35e-15 &      \\ 
     &           &    3 &  8.21e-09 &  2.35e-16 &      \\ 
     &           &    4 &  8.20e-09 &  2.34e-16 &      \\ 
     &           &    5 &  6.95e-08 &  2.34e-16 &      \\ 
     &           &    6 &  4.71e-08 &  1.98e-15 &      \\ 
     &           &    7 &  8.22e-09 &  1.35e-15 &      \\ 
     &           &    8 &  8.21e-09 &  2.35e-16 &      \\ 
     &           &    9 &  8.20e-09 &  2.34e-16 &      \\ 
     &           &   10 &  6.95e-08 &  2.34e-16 &      \\ 
2982 &  2.98e+04 &   10 &           &           & iters  \\ 
 \hdashline 
     &           &    1 &  5.69e-08 &  1.12e-09 &      \\ 
     &           &    2 &  2.09e-08 &  1.62e-15 &      \\ 
     &           &    3 &  2.08e-08 &  5.96e-16 &      \\ 
     &           &    4 &  2.08e-08 &  5.95e-16 &      \\ 
     &           &    5 &  5.69e-08 &  5.94e-16 &      \\ 
     &           &    6 &  2.09e-08 &  1.62e-15 &      \\ 
     &           &    7 &  2.08e-08 &  5.95e-16 &      \\ 
     &           &    8 &  2.08e-08 &  5.95e-16 &      \\ 
     &           &    9 &  5.69e-08 &  5.94e-16 &      \\ 
     &           &   10 &  2.09e-08 &  1.62e-15 &      \\ 
2983 &  2.98e+04 &   10 &           &           & iters  \\ 
 \hdashline 
     &           &    1 &  1.36e-08 &  6.70e-11 &      \\ 
     &           &    2 &  2.52e-08 &  3.89e-16 &      \\ 
     &           &    3 &  1.37e-08 &  7.20e-16 &      \\ 
     &           &    4 &  1.36e-08 &  3.90e-16 &      \\ 
     &           &    5 &  2.52e-08 &  3.89e-16 &      \\ 
     &           &    6 &  1.37e-08 &  7.20e-16 &      \\ 
     &           &    7 &  1.36e-08 &  3.90e-16 &      \\ 
     &           &    8 &  2.52e-08 &  3.89e-16 &      \\ 
     &           &    9 &  1.36e-08 &  7.20e-16 &      \\ 
     &           &   10 &  1.36e-08 &  3.90e-16 &      \\ 
2984 &  2.98e+04 &   10 &           &           & iters  \\ 
 \hdashline 
     &           &    1 &  1.65e-08 &  3.43e-10 &      \\ 
     &           &    2 &  6.12e-08 &  4.71e-16 &      \\ 
     &           &    3 &  1.66e-08 &  1.75e-15 &      \\ 
     &           &    4 &  1.65e-08 &  4.72e-16 &      \\ 
     &           &    5 &  1.65e-08 &  4.72e-16 &      \\ 
     &           &    6 &  6.12e-08 &  4.71e-16 &      \\ 
     &           &    7 &  1.66e-08 &  1.75e-15 &      \\ 
     &           &    8 &  1.65e-08 &  4.73e-16 &      \\ 
     &           &    9 &  1.65e-08 &  4.72e-16 &      \\ 
     &           &   10 &  1.65e-08 &  4.72e-16 &      \\ 
2985 &  2.98e+04 &   10 &           &           & iters  \\ 
 \hdashline 
     &           &    1 &  1.84e-08 &  2.66e-10 &      \\ 
     &           &    2 &  1.84e-08 &  5.25e-16 &      \\ 
     &           &    3 &  1.83e-08 &  5.24e-16 &      \\ 
     &           &    4 &  5.94e-08 &  5.23e-16 &      \\ 
     &           &    5 &  1.84e-08 &  1.69e-15 &      \\ 
     &           &    6 &  1.84e-08 &  5.25e-16 &      \\ 
     &           &    7 &  1.83e-08 &  5.24e-16 &      \\ 
     &           &    8 &  5.94e-08 &  5.23e-16 &      \\ 
     &           &    9 &  1.84e-08 &  1.69e-15 &      \\ 
     &           &   10 &  1.84e-08 &  5.25e-16 &      \\ 
2986 &  2.98e+04 &   10 &           &           & iters  \\ 
 \hdashline 
     &           &    1 &  1.72e-08 &  6.11e-10 &      \\ 
     &           &    2 &  1.71e-08 &  4.90e-16 &      \\ 
     &           &    3 &  6.06e-08 &  4.89e-16 &      \\ 
     &           &    4 &  1.72e-08 &  1.73e-15 &      \\ 
     &           &    5 &  1.72e-08 &  4.91e-16 &      \\ 
     &           &    6 &  1.72e-08 &  4.90e-16 &      \\ 
     &           &    7 &  6.06e-08 &  4.90e-16 &      \\ 
     &           &    8 &  1.72e-08 &  1.73e-15 &      \\ 
     &           &    9 &  1.72e-08 &  4.91e-16 &      \\ 
     &           &   10 &  1.72e-08 &  4.91e-16 &      \\ 
2987 &  2.99e+04 &   10 &           &           & iters  \\ 
 \hdashline 
     &           &    1 &  1.43e-08 &  1.03e-09 &      \\ 
     &           &    2 &  1.43e-08 &  4.09e-16 &      \\ 
     &           &    3 &  1.43e-08 &  4.09e-16 &      \\ 
     &           &    4 &  1.43e-08 &  4.08e-16 &      \\ 
     &           &    5 &  1.43e-08 &  4.08e-16 &      \\ 
     &           &    6 &  6.34e-08 &  4.07e-16 &      \\ 
     &           &    7 &  1.43e-08 &  1.81e-15 &      \\ 
     &           &    8 &  1.43e-08 &  4.09e-16 &      \\ 
     &           &    9 &  1.43e-08 &  4.09e-16 &      \\ 
     &           &   10 &  1.43e-08 &  4.08e-16 &      \\ 
2988 &  2.99e+04 &   10 &           &           & iters  \\ 
 \hdashline 
     &           &    1 &  1.56e-08 &  1.10e-09 &      \\ 
     &           &    2 &  1.55e-08 &  4.44e-16 &      \\ 
     &           &    3 &  1.55e-08 &  4.43e-16 &      \\ 
     &           &    4 &  1.55e-08 &  4.43e-16 &      \\ 
     &           &    5 &  1.55e-08 &  4.42e-16 &      \\ 
     &           &    6 &  1.54e-08 &  4.42e-16 &      \\ 
     &           &    7 &  1.54e-08 &  4.41e-16 &      \\ 
     &           &    8 &  6.23e-08 &  4.40e-16 &      \\ 
     &           &    9 &  1.55e-08 &  1.78e-15 &      \\ 
     &           &   10 &  1.55e-08 &  4.42e-16 &      \\ 
2989 &  2.99e+04 &   10 &           &           & iters  \\ 
 \hdashline 
     &           &    1 &  8.65e-09 &  5.63e-10 &      \\ 
     &           &    2 &  8.64e-09 &  2.47e-16 &      \\ 
     &           &    3 &  8.63e-09 &  2.47e-16 &      \\ 
     &           &    4 &  8.61e-09 &  2.46e-16 &      \\ 
     &           &    5 &  8.60e-09 &  2.46e-16 &      \\ 
     &           &    6 &  8.59e-09 &  2.46e-16 &      \\ 
     &           &    7 &  8.58e-09 &  2.45e-16 &      \\ 
     &           &    8 &  8.57e-09 &  2.45e-16 &      \\ 
     &           &    9 &  8.55e-09 &  2.44e-16 &      \\ 
     &           &   10 &  8.54e-09 &  2.44e-16 &      \\ 
2990 &  2.99e+04 &   10 &           &           & iters  \\ 
 \hdashline 
     &           &    1 &  5.35e-08 &  5.17e-10 &      \\ 
     &           &    2 &  2.43e-08 &  1.53e-15 &      \\ 
     &           &    3 &  5.34e-08 &  6.93e-16 &      \\ 
     &           &    4 &  2.43e-08 &  1.53e-15 &      \\ 
     &           &    5 &  2.43e-08 &  6.94e-16 &      \\ 
     &           &    6 &  5.34e-08 &  6.93e-16 &      \\ 
     &           &    7 &  2.43e-08 &  1.53e-15 &      \\ 
     &           &    8 &  2.43e-08 &  6.94e-16 &      \\ 
     &           &    9 &  5.34e-08 &  6.93e-16 &      \\ 
     &           &   10 &  2.43e-08 &  1.53e-15 &      \\ 
2991 &  2.99e+04 &   10 &           &           & iters  \\ 
 \hdashline 
     &           &    1 &  8.20e-09 &  7.91e-10 &      \\ 
     &           &    2 &  8.19e-09 &  2.34e-16 &      \\ 
     &           &    3 &  8.18e-09 &  2.34e-16 &      \\ 
     &           &    4 &  8.16e-09 &  2.33e-16 &      \\ 
     &           &    5 &  8.15e-09 &  2.33e-16 &      \\ 
     &           &    6 &  8.14e-09 &  2.33e-16 &      \\ 
     &           &    7 &  6.96e-08 &  2.32e-16 &      \\ 
     &           &    8 &  8.59e-08 &  1.99e-15 &      \\ 
     &           &    9 &  6.96e-08 &  2.45e-15 &      \\ 
     &           &   10 &  8.21e-09 &  1.99e-15 &      \\ 
2992 &  2.99e+04 &   10 &           &           & iters  \\ 
 \hdashline 
     &           &    1 &  3.02e-08 &  5.14e-10 &      \\ 
     &           &    2 &  4.75e-08 &  8.62e-16 &      \\ 
     &           &    3 &  3.02e-08 &  1.36e-15 &      \\ 
     &           &    4 &  4.75e-08 &  8.63e-16 &      \\ 
     &           &    5 &  3.03e-08 &  1.36e-15 &      \\ 
     &           &    6 &  3.02e-08 &  8.64e-16 &      \\ 
     &           &    7 &  4.75e-08 &  8.62e-16 &      \\ 
     &           &    8 &  3.02e-08 &  1.36e-15 &      \\ 
     &           &    9 &  4.75e-08 &  8.63e-16 &      \\ 
     &           &   10 &  3.03e-08 &  1.36e-15 &      \\ 
2993 &  2.99e+04 &   10 &           &           & iters  \\ 
 \hdashline 
     &           &    1 &  1.95e-08 &  7.94e-10 &      \\ 
     &           &    2 &  1.95e-08 &  5.58e-16 &      \\ 
     &           &    3 &  5.82e-08 &  5.57e-16 &      \\ 
     &           &    4 &  1.96e-08 &  1.66e-15 &      \\ 
     &           &    5 &  1.95e-08 &  5.59e-16 &      \\ 
     &           &    6 &  1.95e-08 &  5.58e-16 &      \\ 
     &           &    7 &  5.82e-08 &  5.57e-16 &      \\ 
     &           &    8 &  1.96e-08 &  1.66e-15 &      \\ 
     &           &    9 &  1.95e-08 &  5.59e-16 &      \\ 
     &           &   10 &  1.95e-08 &  5.58e-16 &      \\ 
2994 &  2.99e+04 &   10 &           &           & iters  \\ 
 \hdashline 
     &           &    1 &  3.74e-08 &  8.23e-10 &      \\ 
     &           &    2 &  1.48e-09 &  1.07e-15 &      \\ 
     &           &    3 &  1.47e-09 &  4.22e-17 &      \\ 
     &           &    4 &  1.47e-09 &  4.21e-17 &      \\ 
     &           &    5 &  1.47e-09 &  4.20e-17 &      \\ 
     &           &    6 &  1.47e-09 &  4.20e-17 &      \\ 
     &           &    7 &  1.47e-09 &  4.19e-17 &      \\ 
     &           &    8 &  3.74e-08 &  4.19e-17 &      \\ 
     &           &    9 &  1.52e-09 &  1.07e-15 &      \\ 
     &           &   10 &  1.52e-09 &  4.33e-17 &      \\ 
2995 &  2.99e+04 &   10 &           &           & iters  \\ 
 \hdashline 
     &           &    1 &  4.85e-09 &  1.91e-10 &      \\ 
     &           &    2 &  4.84e-09 &  1.38e-16 &      \\ 
     &           &    3 &  4.83e-09 &  1.38e-16 &      \\ 
     &           &    4 &  4.83e-09 &  1.38e-16 &      \\ 
     &           &    5 &  4.82e-09 &  1.38e-16 &      \\ 
     &           &    6 &  4.81e-09 &  1.38e-16 &      \\ 
     &           &    7 &  4.81e-09 &  1.37e-16 &      \\ 
     &           &    8 &  4.80e-09 &  1.37e-16 &      \\ 
     &           &    9 &  4.79e-09 &  1.37e-16 &      \\ 
     &           &   10 &  4.78e-09 &  1.37e-16 &      \\ 
2996 &  3.00e+04 &   10 &           &           & iters  \\ 
 \hdashline 
     &           &    1 &  3.21e-08 &  3.09e-10 &      \\ 
     &           &    2 &  3.20e-08 &  9.15e-16 &      \\ 
     &           &    3 &  4.57e-08 &  9.14e-16 &      \\ 
     &           &    4 &  3.20e-08 &  1.30e-15 &      \\ 
     &           &    5 &  4.57e-08 &  9.14e-16 &      \\ 
     &           &    6 &  3.21e-08 &  1.30e-15 &      \\ 
     &           &    7 &  3.20e-08 &  9.15e-16 &      \\ 
     &           &    8 &  4.57e-08 &  9.14e-16 &      \\ 
     &           &    9 &  3.20e-08 &  1.30e-15 &      \\ 
     &           &   10 &  4.57e-08 &  9.14e-16 &      \\ 
2997 &  3.00e+04 &   10 &           &           & iters  \\ 
 \hdashline 
     &           &    1 &  2.15e-09 &  4.77e-10 &      \\ 
     &           &    2 &  2.15e-09 &  6.14e-17 &      \\ 
     &           &    3 &  2.15e-09 &  6.14e-17 &      \\ 
     &           &    4 &  2.14e-09 &  6.13e-17 &      \\ 
     &           &    5 &  2.14e-09 &  6.12e-17 &      \\ 
     &           &    6 &  2.14e-09 &  6.11e-17 &      \\ 
     &           &    7 &  2.13e-09 &  6.10e-17 &      \\ 
     &           &    8 &  2.13e-09 &  6.09e-17 &      \\ 
     &           &    9 &  2.13e-09 &  6.08e-17 &      \\ 
     &           &   10 &  2.13e-09 &  6.07e-17 &      \\ 
2998 &  3.00e+04 &   10 &           &           & iters  \\ 
 \hdashline 
     &           &    1 &  5.30e-08 &  6.60e-10 &      \\ 
     &           &    2 &  2.48e-08 &  1.51e-15 &      \\ 
     &           &    3 &  2.47e-08 &  7.07e-16 &      \\ 
     &           &    4 &  5.30e-08 &  7.06e-16 &      \\ 
     &           &    5 &  2.48e-08 &  1.51e-15 &      \\ 
     &           &    6 &  2.47e-08 &  7.07e-16 &      \\ 
     &           &    7 &  5.30e-08 &  7.06e-16 &      \\ 
     &           &    8 &  2.48e-08 &  1.51e-15 &      \\ 
     &           &    9 &  2.47e-08 &  7.07e-16 &      \\ 
     &           &   10 &  5.30e-08 &  7.06e-16 &      \\ 
2999 &  3.00e+04 &   10 &           &           & iters  \\ 
 \hdashline 
     &           &    1 &  5.99e-08 &  5.99e-10 &      \\ 
     &           &    2 &  1.79e-08 &  1.71e-15 &      \\ 
     &           &    3 &  1.79e-08 &  5.11e-16 &      \\ 
     &           &    4 &  5.98e-08 &  5.10e-16 &      \\ 
     &           &    5 &  1.79e-08 &  1.71e-15 &      \\ 
     &           &    6 &  1.79e-08 &  5.12e-16 &      \\ 
     &           &    7 &  1.79e-08 &  5.11e-16 &      \\ 
     &           &    8 &  1.79e-08 &  5.11e-16 &      \\ 
     &           &    9 &  5.99e-08 &  5.10e-16 &      \\ 
     &           &   10 &  1.79e-08 &  1.71e-15 &      \\ 
3000 &  3.00e+04 &   10 &           &           & iters  \\ 
 \hdashline 
     &           &    1 &  6.30e-08 &  7.18e-11 &      \\ 
     &           &    2 &  1.48e-08 &  1.80e-15 &      \\ 
     &           &    3 &  1.48e-08 &  4.23e-16 &      \\ 
     &           &    4 &  1.48e-08 &  4.22e-16 &      \\ 
     &           &    5 &  1.48e-08 &  4.22e-16 &      \\ 
     &           &    6 &  6.30e-08 &  4.21e-16 &      \\ 
     &           &    7 &  1.48e-08 &  1.80e-15 &      \\ 
     &           &    8 &  1.48e-08 &  4.23e-16 &      \\ 
     &           &    9 &  1.48e-08 &  4.23e-16 &      \\ 
     &           &   10 &  1.48e-08 &  4.22e-16 &      \\ 
3001 &  3.00e+04 &   10 &           &           & iters  \\ 
 \hdashline 
     &           &    1 &  2.82e-10 &  5.37e-10 &      \\ 
     &           &    2 &  2.82e-10 &  8.05e-18 &      \\ 
     &           &    3 &  2.81e-10 &  8.04e-18 &      \\ 
     &           &    4 &  2.81e-10 &  8.02e-18 &      \\ 
     &           &    5 &  2.80e-10 &  8.01e-18 &      \\ 
     &           &    6 &  2.80e-10 &  8.00e-18 &      \\ 
     &           &    7 &  2.80e-10 &  7.99e-18 &      \\ 
     &           &    8 &  2.79e-10 &  7.98e-18 &      \\ 
     &           &    9 &  2.79e-10 &  7.97e-18 &      \\ 
     &           &   10 &  2.78e-10 &  7.96e-18 &      \\ 
3002 &  3.00e+04 &   10 &           &           & iters  \\ 
 \hdashline 
     &           &    1 &  3.59e-08 &  5.85e-10 &      \\ 
     &           &    2 &  4.19e-08 &  1.02e-15 &      \\ 
     &           &    3 &  3.59e-08 &  1.19e-15 &      \\ 
     &           &    4 &  3.58e-08 &  1.02e-15 &      \\ 
     &           &    5 &  4.19e-08 &  1.02e-15 &      \\ 
     &           &    6 &  3.58e-08 &  1.20e-15 &      \\ 
     &           &    7 &  4.19e-08 &  1.02e-15 &      \\ 
     &           &    8 &  3.58e-08 &  1.20e-15 &      \\ 
     &           &    9 &  4.19e-08 &  1.02e-15 &      \\ 
     &           &   10 &  3.59e-08 &  1.20e-15 &      \\ 
3003 &  3.00e+04 &   10 &           &           & iters  \\ 
 \hdashline 
     &           &    1 &  6.03e-08 &  9.72e-10 &      \\ 
     &           &    2 &  1.75e-08 &  1.72e-15 &      \\ 
     &           &    3 &  1.75e-08 &  5.00e-16 &      \\ 
     &           &    4 &  1.75e-08 &  4.99e-16 &      \\ 
     &           &    5 &  6.02e-08 &  4.99e-16 &      \\ 
     &           &    6 &  1.75e-08 &  1.72e-15 &      \\ 
     &           &    7 &  1.75e-08 &  5.00e-16 &      \\ 
     &           &    8 &  1.75e-08 &  5.00e-16 &      \\ 
     &           &    9 &  1.75e-08 &  4.99e-16 &      \\ 
     &           &   10 &  6.03e-08 &  4.98e-16 &      \\ 
3004 &  3.00e+04 &   10 &           &           & iters  \\ 
 \hdashline 
     &           &    1 &  3.07e-08 &  9.81e-10 &      \\ 
     &           &    2 &  3.06e-08 &  8.76e-16 &      \\ 
     &           &    3 &  4.71e-08 &  8.75e-16 &      \\ 
     &           &    4 &  3.07e-08 &  1.34e-15 &      \\ 
     &           &    5 &  3.06e-08 &  8.75e-16 &      \\ 
     &           &    6 &  4.71e-08 &  8.74e-16 &      \\ 
     &           &    7 &  3.07e-08 &  1.34e-15 &      \\ 
     &           &    8 &  4.71e-08 &  8.75e-16 &      \\ 
     &           &    9 &  3.07e-08 &  1.34e-15 &      \\ 
     &           &   10 &  3.06e-08 &  8.75e-16 &      \\ 
3005 &  3.00e+04 &   10 &           &           & iters  \\ 
 \hdashline 
     &           &    1 &  8.73e-10 &  2.19e-10 &      \\ 
     &           &    2 &  8.72e-10 &  2.49e-17 &      \\ 
     &           &    3 &  8.70e-10 &  2.49e-17 &      \\ 
     &           &    4 &  8.69e-10 &  2.48e-17 &      \\ 
     &           &    5 &  8.68e-10 &  2.48e-17 &      \\ 
     &           &    6 &  8.67e-10 &  2.48e-17 &      \\ 
     &           &    7 &  8.66e-10 &  2.47e-17 &      \\ 
     &           &    8 &  8.64e-10 &  2.47e-17 &      \\ 
     &           &    9 &  8.63e-10 &  2.47e-17 &      \\ 
     &           &   10 &  8.62e-10 &  2.46e-17 &      \\ 
3006 &  3.00e+04 &   10 &           &           & iters  \\ 
 \hdashline 
     &           &    1 &  4.03e-08 &  1.01e-09 &      \\ 
     &           &    2 &  3.75e-08 &  1.15e-15 &      \\ 
     &           &    3 &  4.03e-08 &  1.07e-15 &      \\ 
     &           &    4 &  3.75e-08 &  1.15e-15 &      \\ 
     &           &    5 &  4.03e-08 &  1.07e-15 &      \\ 
     &           &    6 &  3.75e-08 &  1.15e-15 &      \\ 
     &           &    7 &  4.03e-08 &  1.07e-15 &      \\ 
     &           &    8 &  3.75e-08 &  1.15e-15 &      \\ 
     &           &    9 &  4.03e-08 &  1.07e-15 &      \\ 
     &           &   10 &  3.75e-08 &  1.15e-15 &      \\ 
3007 &  3.01e+04 &   10 &           &           & iters  \\ 
 \hdashline 
     &           &    1 &  3.89e-08 &  2.07e-10 &      \\ 
     &           &    2 &  3.88e-08 &  1.11e-15 &      \\ 
     &           &    3 &  3.89e-08 &  1.11e-15 &      \\ 
     &           &    4 &  3.88e-08 &  1.11e-15 &      \\ 
     &           &    5 &  3.89e-08 &  1.11e-15 &      \\ 
     &           &    6 &  3.88e-08 &  1.11e-15 &      \\ 
     &           &    7 &  3.89e-08 &  1.11e-15 &      \\ 
     &           &    8 &  3.88e-08 &  1.11e-15 &      \\ 
     &           &    9 &  3.89e-08 &  1.11e-15 &      \\ 
     &           &   10 &  3.88e-08 &  1.11e-15 &      \\ 
3008 &  3.01e+04 &   10 &           &           & iters  \\ 
 \hdashline 
     &           &    1 &  4.74e-08 &  1.91e-11 &      \\ 
     &           &    2 &  3.04e-08 &  1.35e-15 &      \\ 
     &           &    3 &  3.03e-08 &  8.67e-16 &      \\ 
     &           &    4 &  4.74e-08 &  8.65e-16 &      \\ 
     &           &    5 &  3.03e-08 &  1.35e-15 &      \\ 
     &           &    6 &  4.74e-08 &  8.66e-16 &      \\ 
     &           &    7 &  3.04e-08 &  1.35e-15 &      \\ 
     &           &    8 &  3.03e-08 &  8.67e-16 &      \\ 
     &           &    9 &  4.74e-08 &  8.66e-16 &      \\ 
     &           &   10 &  3.04e-08 &  1.35e-15 &      \\ 
3009 &  3.01e+04 &   10 &           &           & iters  \\ 
 \hdashline 
     &           &    1 &  1.33e-08 &  7.17e-10 &      \\ 
     &           &    2 &  1.33e-08 &  3.79e-16 &      \\ 
     &           &    3 &  2.56e-08 &  3.78e-16 &      \\ 
     &           &    4 &  1.33e-08 &  7.31e-16 &      \\ 
     &           &    5 &  2.56e-08 &  3.79e-16 &      \\ 
     &           &    6 &  1.33e-08 &  7.30e-16 &      \\ 
     &           &    7 &  1.33e-08 &  3.79e-16 &      \\ 
     &           &    8 &  2.56e-08 &  3.79e-16 &      \\ 
     &           &    9 &  1.33e-08 &  7.30e-16 &      \\ 
     &           &   10 &  1.33e-08 &  3.79e-16 &      \\ 
3010 &  3.01e+04 &   10 &           &           & iters  \\ 
 \hdashline 
     &           &    1 &  2.01e-08 &  7.00e-10 &      \\ 
     &           &    2 &  1.88e-08 &  5.73e-16 &      \\ 
     &           &    3 &  1.88e-08 &  5.37e-16 &      \\ 
     &           &    4 &  2.01e-08 &  5.36e-16 &      \\ 
     &           &    5 &  1.88e-08 &  5.73e-16 &      \\ 
     &           &    6 &  2.01e-08 &  5.36e-16 &      \\ 
     &           &    7 &  1.88e-08 &  5.73e-16 &      \\ 
     &           &    8 &  2.01e-08 &  5.36e-16 &      \\ 
     &           &    9 &  1.88e-08 &  5.73e-16 &      \\ 
     &           &   10 &  2.01e-08 &  5.36e-16 &      \\ 
3011 &  3.01e+04 &   10 &           &           & iters  \\ 
 \hdashline 
     &           &    1 &  2.95e-08 &  1.39e-10 &      \\ 
     &           &    2 &  4.82e-08 &  8.42e-16 &      \\ 
     &           &    3 &  2.95e-08 &  1.38e-15 &      \\ 
     &           &    4 &  2.95e-08 &  8.42e-16 &      \\ 
     &           &    5 &  4.83e-08 &  8.41e-16 &      \\ 
     &           &    6 &  2.95e-08 &  1.38e-15 &      \\ 
     &           &    7 &  4.82e-08 &  8.42e-16 &      \\ 
     &           &    8 &  2.95e-08 &  1.38e-15 &      \\ 
     &           &    9 &  2.95e-08 &  8.43e-16 &      \\ 
     &           &   10 &  4.82e-08 &  8.41e-16 &      \\ 
3012 &  3.01e+04 &   10 &           &           & iters  \\ 
 \hdashline 
     &           &    1 &  1.12e-08 &  4.32e-10 &      \\ 
     &           &    2 &  6.65e-08 &  3.20e-16 &      \\ 
     &           &    3 &  1.13e-08 &  1.90e-15 &      \\ 
     &           &    4 &  1.13e-08 &  3.23e-16 &      \\ 
     &           &    5 &  1.13e-08 &  3.22e-16 &      \\ 
     &           &    6 &  1.13e-08 &  3.22e-16 &      \\ 
     &           &    7 &  1.12e-08 &  3.21e-16 &      \\ 
     &           &    8 &  1.12e-08 &  3.21e-16 &      \\ 
     &           &    9 &  6.65e-08 &  3.20e-16 &      \\ 
     &           &   10 &  1.13e-08 &  1.90e-15 &      \\ 
3013 &  3.01e+04 &   10 &           &           & iters  \\ 
 \hdashline 
     &           &    1 &  3.06e-08 &  3.63e-10 &      \\ 
     &           &    2 &  3.05e-08 &  8.72e-16 &      \\ 
     &           &    3 &  4.72e-08 &  8.71e-16 &      \\ 
     &           &    4 &  3.05e-08 &  1.35e-15 &      \\ 
     &           &    5 &  4.72e-08 &  8.72e-16 &      \\ 
     &           &    6 &  3.06e-08 &  1.35e-15 &      \\ 
     &           &    7 &  3.05e-08 &  8.72e-16 &      \\ 
     &           &    8 &  4.72e-08 &  8.71e-16 &      \\ 
     &           &    9 &  3.05e-08 &  1.35e-15 &      \\ 
     &           &   10 &  4.72e-08 &  8.72e-16 &      \\ 
3014 &  3.01e+04 &   10 &           &           & iters  \\ 
 \hdashline 
     &           &    1 &  2.12e-08 &  2.38e-10 &      \\ 
     &           &    2 &  2.12e-08 &  6.05e-16 &      \\ 
     &           &    3 &  1.77e-08 &  6.04e-16 &      \\ 
     &           &    4 &  2.12e-08 &  5.06e-16 &      \\ 
     &           &    5 &  1.77e-08 &  6.04e-16 &      \\ 
     &           &    6 &  2.12e-08 &  5.06e-16 &      \\ 
     &           &    7 &  1.77e-08 &  6.04e-16 &      \\ 
     &           &    8 &  2.11e-08 &  5.06e-16 &      \\ 
     &           &    9 &  1.77e-08 &  6.04e-16 &      \\ 
     &           &   10 &  1.77e-08 &  5.06e-16 &      \\ 
3015 &  3.01e+04 &   10 &           &           & iters  \\ 
 \hdashline 
     &           &    1 &  2.73e-09 &  1.21e-10 &      \\ 
     &           &    2 &  2.73e-09 &  7.79e-17 &      \\ 
     &           &    3 &  2.72e-09 &  7.78e-17 &      \\ 
     &           &    4 &  2.72e-09 &  7.77e-17 &      \\ 
     &           &    5 &  2.72e-09 &  7.76e-17 &      \\ 
     &           &    6 &  2.71e-09 &  7.75e-17 &      \\ 
     &           &    7 &  7.50e-08 &  7.74e-17 &      \\ 
     &           &    8 &  8.05e-08 &  2.14e-15 &      \\ 
     &           &    9 &  7.50e-08 &  2.30e-15 &      \\ 
     &           &   10 &  8.05e-08 &  2.14e-15 &      \\ 
3016 &  3.02e+04 &   10 &           &           & iters  \\ 
 \hdashline 
     &           &    1 &  7.47e-08 &  3.93e-10 &      \\ 
     &           &    2 &  3.12e-09 &  2.13e-15 &      \\ 
     &           &    3 &  3.11e-09 &  8.90e-17 &      \\ 
     &           &    4 &  3.11e-09 &  8.88e-17 &      \\ 
     &           &    5 &  3.10e-09 &  8.87e-17 &      \\ 
     &           &    6 &  3.10e-09 &  8.86e-17 &      \\ 
     &           &    7 &  3.10e-09 &  8.85e-17 &      \\ 
     &           &    8 &  3.09e-09 &  8.83e-17 &      \\ 
     &           &    9 &  3.09e-09 &  8.82e-17 &      \\ 
     &           &   10 &  3.08e-09 &  8.81e-17 &      \\ 
3017 &  3.02e+04 &   10 &           &           & iters  \\ 
 \hdashline 
     &           &    1 &  3.04e-08 &  2.87e-10 &      \\ 
     &           &    2 &  4.73e-08 &  8.68e-16 &      \\ 
     &           &    3 &  3.04e-08 &  1.35e-15 &      \\ 
     &           &    4 &  3.04e-08 &  8.68e-16 &      \\ 
     &           &    5 &  4.73e-08 &  8.67e-16 &      \\ 
     &           &    6 &  3.04e-08 &  1.35e-15 &      \\ 
     &           &    7 &  4.73e-08 &  8.68e-16 &      \\ 
     &           &    8 &  3.04e-08 &  1.35e-15 &      \\ 
     &           &    9 &  3.04e-08 &  8.69e-16 &      \\ 
     &           &   10 &  4.73e-08 &  8.67e-16 &      \\ 
3018 &  3.02e+04 &   10 &           &           & iters  \\ 
 \hdashline 
     &           &    1 &  2.26e-08 &  2.15e-10 &      \\ 
     &           &    2 &  5.52e-08 &  6.44e-16 &      \\ 
     &           &    3 &  2.26e-08 &  1.57e-15 &      \\ 
     &           &    4 &  2.26e-08 &  6.45e-16 &      \\ 
     &           &    5 &  5.52e-08 &  6.44e-16 &      \\ 
     &           &    6 &  2.26e-08 &  1.57e-15 &      \\ 
     &           &    7 &  2.26e-08 &  6.45e-16 &      \\ 
     &           &    8 &  2.26e-08 &  6.45e-16 &      \\ 
     &           &    9 &  5.52e-08 &  6.44e-16 &      \\ 
     &           &   10 &  2.26e-08 &  1.57e-15 &      \\ 
3019 &  3.02e+04 &   10 &           &           & iters  \\ 
 \hdashline 
     &           &    1 &  3.50e-08 &  3.93e-10 &      \\ 
     &           &    2 &  4.27e-08 &  1.00e-15 &      \\ 
     &           &    3 &  3.51e-08 &  1.22e-15 &      \\ 
     &           &    4 &  4.27e-08 &  1.00e-15 &      \\ 
     &           &    5 &  3.51e-08 &  1.22e-15 &      \\ 
     &           &    6 &  4.27e-08 &  1.00e-15 &      \\ 
     &           &    7 &  3.51e-08 &  1.22e-15 &      \\ 
     &           &    8 &  4.27e-08 &  1.00e-15 &      \\ 
     &           &    9 &  3.51e-08 &  1.22e-15 &      \\ 
     &           &   10 &  3.50e-08 &  1.00e-15 &      \\ 
3020 &  3.02e+04 &   10 &           &           & iters  \\ 
 \hdashline 
     &           &    1 &  4.07e-08 &  4.14e-10 &      \\ 
     &           &    2 &  4.06e-08 &  1.16e-15 &      \\ 
     &           &    3 &  3.71e-08 &  1.16e-15 &      \\ 
     &           &    4 &  3.71e-08 &  1.06e-15 &      \\ 
     &           &    5 &  4.07e-08 &  1.06e-15 &      \\ 
     &           &    6 &  3.71e-08 &  1.16e-15 &      \\ 
     &           &    7 &  4.07e-08 &  1.06e-15 &      \\ 
     &           &    8 &  3.71e-08 &  1.16e-15 &      \\ 
     &           &    9 &  4.06e-08 &  1.06e-15 &      \\ 
     &           &   10 &  3.71e-08 &  1.16e-15 &      \\ 
3021 &  3.02e+04 &   10 &           &           & iters  \\ 
 \hdashline 
     &           &    1 &  4.86e-08 &  7.63e-10 &      \\ 
     &           &    2 &  2.92e-08 &  1.39e-15 &      \\ 
     &           &    3 &  4.86e-08 &  8.32e-16 &      \\ 
     &           &    4 &  2.92e-08 &  1.39e-15 &      \\ 
     &           &    5 &  2.91e-08 &  8.33e-16 &      \\ 
     &           &    6 &  4.86e-08 &  8.32e-16 &      \\ 
     &           &    7 &  2.92e-08 &  1.39e-15 &      \\ 
     &           &    8 &  2.91e-08 &  8.33e-16 &      \\ 
     &           &    9 &  4.86e-08 &  8.31e-16 &      \\ 
     &           &   10 &  2.92e-08 &  1.39e-15 &      \\ 
3022 &  3.02e+04 &   10 &           &           & iters  \\ 
 \hdashline 
     &           &    1 &  2.60e-08 &  8.82e-10 &      \\ 
     &           &    2 &  5.17e-08 &  7.43e-16 &      \\ 
     &           &    3 &  2.61e-08 &  1.48e-15 &      \\ 
     &           &    4 &  2.60e-08 &  7.44e-16 &      \\ 
     &           &    5 &  5.17e-08 &  7.43e-16 &      \\ 
     &           &    6 &  2.61e-08 &  1.48e-15 &      \\ 
     &           &    7 &  2.60e-08 &  7.44e-16 &      \\ 
     &           &    8 &  5.17e-08 &  7.43e-16 &      \\ 
     &           &    9 &  2.61e-08 &  1.48e-15 &      \\ 
     &           &   10 &  2.60e-08 &  7.44e-16 &      \\ 
3023 &  3.02e+04 &   10 &           &           & iters  \\ 
 \hdashline 
     &           &    1 &  1.23e-08 &  7.62e-10 &      \\ 
     &           &    2 &  1.23e-08 &  3.50e-16 &      \\ 
     &           &    3 &  1.22e-08 &  3.50e-16 &      \\ 
     &           &    4 &  1.22e-08 &  3.49e-16 &      \\ 
     &           &    5 &  1.22e-08 &  3.49e-16 &      \\ 
     &           &    6 &  1.22e-08 &  3.48e-16 &      \\ 
     &           &    7 &  1.22e-08 &  3.48e-16 &      \\ 
     &           &    8 &  6.55e-08 &  3.47e-16 &      \\ 
     &           &    9 &  1.22e-08 &  1.87e-15 &      \\ 
     &           &   10 &  1.22e-08 &  3.49e-16 &      \\ 
3024 &  3.02e+04 &   10 &           &           & iters  \\ 
 \hdashline 
     &           &    1 &  2.49e-09 &  4.22e-10 &      \\ 
     &           &    2 &  2.49e-09 &  7.10e-17 &      \\ 
     &           &    3 &  2.48e-09 &  7.09e-17 &      \\ 
     &           &    4 &  2.48e-09 &  7.08e-17 &      \\ 
     &           &    5 &  2.47e-09 &  7.07e-17 &      \\ 
     &           &    6 &  2.47e-09 &  7.06e-17 &      \\ 
     &           &    7 &  2.47e-09 &  7.05e-17 &      \\ 
     &           &    8 &  2.46e-09 &  7.04e-17 &      \\ 
     &           &    9 &  2.46e-09 &  7.03e-17 &      \\ 
     &           &   10 &  2.46e-09 &  7.02e-17 &      \\ 
3025 &  3.02e+04 &   10 &           &           & iters  \\ 
 \hdashline 
     &           &    1 &  5.04e-08 &  4.11e-10 &      \\ 
     &           &    2 &  2.73e-08 &  1.44e-15 &      \\ 
     &           &    3 &  5.04e-08 &  7.79e-16 &      \\ 
     &           &    4 &  2.73e-08 &  1.44e-15 &      \\ 
     &           &    5 &  2.73e-08 &  7.80e-16 &      \\ 
     &           &    6 &  5.04e-08 &  7.79e-16 &      \\ 
     &           &    7 &  2.73e-08 &  1.44e-15 &      \\ 
     &           &    8 &  2.73e-08 &  7.80e-16 &      \\ 
     &           &    9 &  5.04e-08 &  7.79e-16 &      \\ 
     &           &   10 &  2.73e-08 &  1.44e-15 &      \\ 
3026 &  3.02e+04 &   10 &           &           & iters  \\ 
 \hdashline 
     &           &    1 &  2.33e-08 &  6.76e-10 &      \\ 
     &           &    2 &  2.33e-08 &  6.66e-16 &      \\ 
     &           &    3 &  2.33e-08 &  6.65e-16 &      \\ 
     &           &    4 &  5.44e-08 &  6.64e-16 &      \\ 
     &           &    5 &  2.33e-08 &  1.55e-15 &      \\ 
     &           &    6 &  2.33e-08 &  6.66e-16 &      \\ 
     &           &    7 &  5.44e-08 &  6.65e-16 &      \\ 
     &           &    8 &  2.33e-08 &  1.55e-15 &      \\ 
     &           &    9 &  2.33e-08 &  6.66e-16 &      \\ 
     &           &   10 &  2.33e-08 &  6.65e-16 &      \\ 
3027 &  3.03e+04 &   10 &           &           & iters  \\ 
 \hdashline 
     &           &    1 &  6.97e-08 &  6.60e-10 &      \\ 
     &           &    2 &  8.11e-09 &  1.99e-15 &      \\ 
     &           &    3 &  8.10e-09 &  2.31e-16 &      \\ 
     &           &    4 &  8.08e-09 &  2.31e-16 &      \\ 
     &           &    5 &  6.96e-08 &  2.31e-16 &      \\ 
     &           &    6 &  8.59e-08 &  1.99e-15 &      \\ 
     &           &    7 &  6.96e-08 &  2.45e-15 &      \\ 
     &           &    8 &  8.15e-09 &  1.99e-15 &      \\ 
     &           &    9 &  8.14e-09 &  2.33e-16 &      \\ 
     &           &   10 &  8.13e-09 &  2.32e-16 &      \\ 
3028 &  3.03e+04 &   10 &           &           & iters  \\ 
 \hdashline 
     &           &    1 &  8.74e-09 &  1.22e-09 &      \\ 
     &           &    2 &  8.73e-09 &  2.49e-16 &      \\ 
     &           &    3 &  8.71e-09 &  2.49e-16 &      \\ 
     &           &    4 &  8.70e-09 &  2.49e-16 &      \\ 
     &           &    5 &  8.69e-09 &  2.48e-16 &      \\ 
     &           &    6 &  6.90e-08 &  2.48e-16 &      \\ 
     &           &    7 &  8.65e-08 &  1.97e-15 &      \\ 
     &           &    8 &  6.90e-08 &  2.47e-15 &      \\ 
     &           &    9 &  8.75e-09 &  1.97e-15 &      \\ 
     &           &   10 &  8.74e-09 &  2.50e-16 &      \\ 
3029 &  3.03e+04 &   10 &           &           & iters  \\ 
 \hdashline 
     &           &    1 &  1.43e-09 &  1.44e-09 &      \\ 
     &           &    2 &  1.43e-09 &  4.09e-17 &      \\ 
     &           &    3 &  1.43e-09 &  4.09e-17 &      \\ 
     &           &    4 &  1.43e-09 &  4.08e-17 &      \\ 
     &           &    5 &  1.43e-09 &  4.08e-17 &      \\ 
     &           &    6 &  1.42e-09 &  4.07e-17 &      \\ 
     &           &    7 &  1.42e-09 &  4.06e-17 &      \\ 
     &           &    8 &  1.42e-09 &  4.06e-17 &      \\ 
     &           &    9 &  1.42e-09 &  4.05e-17 &      \\ 
     &           &   10 &  1.42e-09 &  4.05e-17 &      \\ 
3030 &  3.03e+04 &   10 &           &           & iters  \\ 
 \hdashline 
     &           &    1 &  3.76e-08 &  1.28e-09 &      \\ 
     &           &    2 &  4.01e-08 &  1.07e-15 &      \\ 
     &           &    3 &  3.76e-08 &  1.14e-15 &      \\ 
     &           &    4 &  4.01e-08 &  1.07e-15 &      \\ 
     &           &    5 &  3.76e-08 &  1.14e-15 &      \\ 
     &           &    6 &  4.01e-08 &  1.07e-15 &      \\ 
     &           &    7 &  3.77e-08 &  1.14e-15 &      \\ 
     &           &    8 &  4.01e-08 &  1.07e-15 &      \\ 
     &           &    9 &  3.77e-08 &  1.14e-15 &      \\ 
     &           &   10 &  4.01e-08 &  1.07e-15 &      \\ 
3031 &  3.03e+04 &   10 &           &           & iters  \\ 
 \hdashline 
     &           &    1 &  2.25e-08 &  7.13e-10 &      \\ 
     &           &    2 &  5.52e-08 &  6.41e-16 &      \\ 
     &           &    3 &  2.25e-08 &  1.58e-15 &      \\ 
     &           &    4 &  2.25e-08 &  6.43e-16 &      \\ 
     &           &    5 &  5.52e-08 &  6.42e-16 &      \\ 
     &           &    6 &  2.25e-08 &  1.58e-15 &      \\ 
     &           &    7 &  2.25e-08 &  6.43e-16 &      \\ 
     &           &    8 &  2.25e-08 &  6.42e-16 &      \\ 
     &           &    9 &  5.53e-08 &  6.41e-16 &      \\ 
     &           &   10 &  2.25e-08 &  1.58e-15 &      \\ 
3032 &  3.03e+04 &   10 &           &           & iters  \\ 
 \hdashline 
     &           &    1 &  1.32e-09 &  5.43e-10 &      \\ 
     &           &    2 &  1.31e-09 &  3.76e-17 &      \\ 
     &           &    3 &  1.31e-09 &  3.75e-17 &      \\ 
     &           &    4 &  1.31e-09 &  3.74e-17 &      \\ 
     &           &    5 &  1.31e-09 &  3.74e-17 &      \\ 
     &           &    6 &  1.31e-09 &  3.73e-17 &      \\ 
     &           &    7 &  1.30e-09 &  3.73e-17 &      \\ 
     &           &    8 &  1.30e-09 &  3.72e-17 &      \\ 
     &           &    9 &  1.30e-09 &  3.72e-17 &      \\ 
     &           &   10 &  1.30e-09 &  3.71e-17 &      \\ 
3033 &  3.03e+04 &   10 &           &           & iters  \\ 
 \hdashline 
     &           &    1 &  3.46e-08 &  1.36e-09 &      \\ 
     &           &    2 &  4.32e-08 &  9.87e-16 &      \\ 
     &           &    3 &  3.46e-08 &  1.23e-15 &      \\ 
     &           &    4 &  4.31e-08 &  9.87e-16 &      \\ 
     &           &    5 &  3.46e-08 &  1.23e-15 &      \\ 
     &           &    6 &  3.46e-08 &  9.88e-16 &      \\ 
     &           &    7 &  4.32e-08 &  9.86e-16 &      \\ 
     &           &    8 &  3.46e-08 &  1.23e-15 &      \\ 
     &           &    9 &  4.32e-08 &  9.87e-16 &      \\ 
     &           &   10 &  3.46e-08 &  1.23e-15 &      \\ 
3034 &  3.03e+04 &   10 &           &           & iters  \\ 
 \hdashline 
     &           &    1 &  3.19e-08 &  1.31e-09 &      \\ 
     &           &    2 &  4.58e-08 &  9.10e-16 &      \\ 
     &           &    3 &  3.19e-08 &  1.31e-15 &      \\ 
     &           &    4 &  3.19e-08 &  9.11e-16 &      \\ 
     &           &    5 &  4.59e-08 &  9.10e-16 &      \\ 
     &           &    6 &  3.19e-08 &  1.31e-15 &      \\ 
     &           &    7 &  4.58e-08 &  9.10e-16 &      \\ 
     &           &    8 &  3.19e-08 &  1.31e-15 &      \\ 
     &           &    9 &  4.58e-08 &  9.11e-16 &      \\ 
     &           &   10 &  3.19e-08 &  1.31e-15 &      \\ 
3035 &  3.03e+04 &   10 &           &           & iters  \\ 
 \hdashline 
     &           &    1 &  2.63e-08 &  8.59e-10 &      \\ 
     &           &    2 &  5.14e-08 &  7.51e-16 &      \\ 
     &           &    3 &  2.63e-08 &  1.47e-15 &      \\ 
     &           &    4 &  2.63e-08 &  7.52e-16 &      \\ 
     &           &    5 &  5.14e-08 &  7.51e-16 &      \\ 
     &           &    6 &  2.63e-08 &  1.47e-15 &      \\ 
     &           &    7 &  2.63e-08 &  7.52e-16 &      \\ 
     &           &    8 &  5.14e-08 &  7.50e-16 &      \\ 
     &           &    9 &  2.63e-08 &  1.47e-15 &      \\ 
     &           &   10 &  2.63e-08 &  7.51e-16 &      \\ 
3036 &  3.04e+04 &   10 &           &           & iters  \\ 
 \hdashline 
     &           &    1 &  7.79e-09 &  3.18e-10 &      \\ 
     &           &    2 &  7.78e-09 &  2.22e-16 &      \\ 
     &           &    3 &  7.77e-09 &  2.22e-16 &      \\ 
     &           &    4 &  7.76e-09 &  2.22e-16 &      \\ 
     &           &    5 &  7.75e-09 &  2.21e-16 &      \\ 
     &           &    6 &  7.74e-09 &  2.21e-16 &      \\ 
     &           &    7 &  7.72e-09 &  2.21e-16 &      \\ 
     &           &    8 &  7.71e-09 &  2.20e-16 &      \\ 
     &           &    9 &  7.70e-09 &  2.20e-16 &      \\ 
     &           &   10 &  7.69e-09 &  2.20e-16 &      \\ 
3037 &  3.04e+04 &   10 &           &           & iters  \\ 
 \hdashline 
     &           &    1 &  7.69e-08 &  2.99e-11 &      \\ 
     &           &    2 &  8.52e-10 &  2.20e-15 &      \\ 
     &           &    3 &  8.50e-10 &  2.43e-17 &      \\ 
     &           &    4 &  8.49e-10 &  2.43e-17 &      \\ 
     &           &    5 &  8.48e-10 &  2.42e-17 &      \\ 
     &           &    6 &  8.47e-10 &  2.42e-17 &      \\ 
     &           &    7 &  8.45e-10 &  2.42e-17 &      \\ 
     &           &    8 &  8.44e-10 &  2.41e-17 &      \\ 
     &           &    9 &  8.43e-10 &  2.41e-17 &      \\ 
     &           &   10 &  8.42e-10 &  2.41e-17 &      \\ 
3038 &  3.04e+04 &   10 &           &           & iters  \\ 
 \hdashline 
     &           &    1 &  5.94e-09 &  5.98e-10 &      \\ 
     &           &    2 &  5.93e-09 &  1.69e-16 &      \\ 
     &           &    3 &  5.92e-09 &  1.69e-16 &      \\ 
     &           &    4 &  5.91e-09 &  1.69e-16 &      \\ 
     &           &    5 &  5.90e-09 &  1.69e-16 &      \\ 
     &           &    6 &  5.90e-09 &  1.69e-16 &      \\ 
     &           &    7 &  5.89e-09 &  1.68e-16 &      \\ 
     &           &    8 &  5.88e-09 &  1.68e-16 &      \\ 
     &           &    9 &  7.18e-08 &  1.68e-16 &      \\ 
     &           &   10 &  8.37e-08 &  2.05e-15 &      \\ 
3039 &  3.04e+04 &   10 &           &           & iters  \\ 
 \hdashline 
     &           &    1 &  1.92e-08 &  1.71e-09 &      \\ 
     &           &    2 &  1.91e-08 &  5.47e-16 &      \\ 
     &           &    3 &  5.86e-08 &  5.47e-16 &      \\ 
     &           &    4 &  1.92e-08 &  1.67e-15 &      \\ 
     &           &    5 &  1.92e-08 &  5.48e-16 &      \\ 
     &           &    6 &  1.92e-08 &  5.47e-16 &      \\ 
     &           &    7 &  5.86e-08 &  5.47e-16 &      \\ 
     &           &    8 &  1.92e-08 &  1.67e-15 &      \\ 
     &           &    9 &  1.92e-08 &  5.48e-16 &      \\ 
     &           &   10 &  1.92e-08 &  5.47e-16 &      \\ 
3040 &  3.04e+04 &   10 &           &           & iters  \\ 
 \hdashline 
     &           &    1 &  2.90e-08 &  8.43e-10 &      \\ 
     &           &    2 &  2.89e-08 &  8.27e-16 &      \\ 
     &           &    3 &  4.88e-08 &  8.26e-16 &      \\ 
     &           &    4 &  2.90e-08 &  1.39e-15 &      \\ 
     &           &    5 &  2.89e-08 &  8.26e-16 &      \\ 
     &           &    6 &  4.88e-08 &  8.25e-16 &      \\ 
     &           &    7 &  2.89e-08 &  1.39e-15 &      \\ 
     &           &    8 &  4.88e-08 &  8.26e-16 &      \\ 
     &           &    9 &  2.90e-08 &  1.39e-15 &      \\ 
     &           &   10 &  2.89e-08 &  8.27e-16 &      \\ 
3041 &  3.04e+04 &   10 &           &           & iters  \\ 
 \hdashline 
     &           &    1 &  4.70e-08 &  1.43e-09 &      \\ 
     &           &    2 &  3.07e-08 &  1.34e-15 &      \\ 
     &           &    3 &  3.07e-08 &  8.77e-16 &      \\ 
     &           &    4 &  4.71e-08 &  8.75e-16 &      \\ 
     &           &    5 &  3.07e-08 &  1.34e-15 &      \\ 
     &           &    6 &  4.70e-08 &  8.76e-16 &      \\ 
     &           &    7 &  3.07e-08 &  1.34e-15 &      \\ 
     &           &    8 &  3.07e-08 &  8.77e-16 &      \\ 
     &           &    9 &  4.71e-08 &  8.76e-16 &      \\ 
     &           &   10 &  3.07e-08 &  1.34e-15 &      \\ 
3042 &  3.04e+04 &   10 &           &           & iters  \\ 
 \hdashline 
     &           &    1 &  2.93e-08 &  1.91e-09 &      \\ 
     &           &    2 &  4.84e-08 &  8.36e-16 &      \\ 
     &           &    3 &  2.93e-08 &  1.38e-15 &      \\ 
     &           &    4 &  2.93e-08 &  8.36e-16 &      \\ 
     &           &    5 &  4.85e-08 &  8.35e-16 &      \\ 
     &           &    6 &  2.93e-08 &  1.38e-15 &      \\ 
     &           &    7 &  2.93e-08 &  8.36e-16 &      \\ 
     &           &    8 &  4.85e-08 &  8.35e-16 &      \\ 
     &           &    9 &  2.93e-08 &  1.38e-15 &      \\ 
     &           &   10 &  4.85e-08 &  8.36e-16 &      \\ 
3043 &  3.04e+04 &   10 &           &           & iters  \\ 
 \hdashline 
     &           &    1 &  5.69e-08 &  7.01e-10 &      \\ 
     &           &    2 &  2.09e-08 &  1.62e-15 &      \\ 
     &           &    3 &  2.08e-08 &  5.96e-16 &      \\ 
     &           &    4 &  5.69e-08 &  5.95e-16 &      \\ 
     &           &    5 &  2.09e-08 &  1.62e-15 &      \\ 
     &           &    6 &  2.09e-08 &  5.96e-16 &      \\ 
     &           &    7 &  5.69e-08 &  5.96e-16 &      \\ 
     &           &    8 &  2.09e-08 &  1.62e-15 &      \\ 
     &           &    9 &  2.09e-08 &  5.97e-16 &      \\ 
     &           &   10 &  2.09e-08 &  5.96e-16 &      \\ 
3044 &  3.04e+04 &   10 &           &           & iters  \\ 
 \hdashline 
     &           &    1 &  5.77e-08 &  9.19e-11 &      \\ 
     &           &    2 &  2.01e-08 &  1.65e-15 &      \\ 
     &           &    3 &  2.01e-08 &  5.74e-16 &      \\ 
     &           &    4 &  2.01e-08 &  5.73e-16 &      \\ 
     &           &    5 &  5.77e-08 &  5.72e-16 &      \\ 
     &           &    6 &  2.01e-08 &  1.65e-15 &      \\ 
     &           &    7 &  2.01e-08 &  5.74e-16 &      \\ 
     &           &    8 &  2.00e-08 &  5.73e-16 &      \\ 
     &           &    9 &  5.77e-08 &  5.72e-16 &      \\ 
     &           &   10 &  2.01e-08 &  1.65e-15 &      \\ 
3045 &  3.04e+04 &   10 &           &           & iters  \\ 
 \hdashline 
     &           &    1 &  1.80e-08 &  1.57e-10 &      \\ 
     &           &    2 &  2.09e-08 &  5.13e-16 &      \\ 
     &           &    3 &  1.80e-08 &  5.96e-16 &      \\ 
     &           &    4 &  2.09e-08 &  5.14e-16 &      \\ 
     &           &    5 &  1.80e-08 &  5.96e-16 &      \\ 
     &           &    6 &  2.09e-08 &  5.14e-16 &      \\ 
     &           &    7 &  1.80e-08 &  5.96e-16 &      \\ 
     &           &    8 &  2.09e-08 &  5.14e-16 &      \\ 
     &           &    9 &  1.80e-08 &  5.96e-16 &      \\ 
     &           &   10 &  2.09e-08 &  5.14e-16 &      \\ 
3046 &  3.04e+04 &   10 &           &           & iters  \\ 
 \hdashline 
     &           &    1 &  6.42e-11 &  7.18e-10 &      \\ 
     &           &    2 &  6.41e-11 &  1.83e-18 &      \\ 
     &           &    3 &  6.40e-11 &  1.83e-18 &      \\ 
     &           &    4 &  6.39e-11 &  1.83e-18 &      \\ 
     &           &    5 &  6.38e-11 &  1.82e-18 &      \\ 
     &           &    6 &  6.37e-11 &  1.82e-18 &      \\ 
     &           &    7 &  6.36e-11 &  1.82e-18 &      \\ 
     &           &    8 &  6.35e-11 &  1.82e-18 &      \\ 
     &           &    9 &  6.35e-11 &  1.81e-18 &      \\ 
     &           &   10 &  6.34e-11 &  1.81e-18 &      \\ 
3047 &  3.05e+04 &   10 &           &           & iters  \\ 
 \hdashline 
     &           &    1 &  1.01e-08 &  9.97e-10 &      \\ 
     &           &    2 &  1.01e-08 &  2.89e-16 &      \\ 
     &           &    3 &  1.01e-08 &  2.88e-16 &      \\ 
     &           &    4 &  1.01e-08 &  2.88e-16 &      \\ 
     &           &    5 &  6.76e-08 &  2.88e-16 &      \\ 
     &           &    6 &  1.02e-08 &  1.93e-15 &      \\ 
     &           &    7 &  1.01e-08 &  2.90e-16 &      \\ 
     &           &    8 &  1.01e-08 &  2.90e-16 &      \\ 
     &           &    9 &  1.01e-08 &  2.89e-16 &      \\ 
     &           &   10 &  1.01e-08 &  2.89e-16 &      \\ 
3048 &  3.05e+04 &   10 &           &           & iters  \\ 
 \hdashline 
     &           &    1 &  3.52e-08 &  6.58e-11 &      \\ 
     &           &    2 &  3.51e-08 &  1.00e-15 &      \\ 
     &           &    3 &  4.26e-08 &  1.00e-15 &      \\ 
     &           &    4 &  3.51e-08 &  1.22e-15 &      \\ 
     &           &    5 &  4.26e-08 &  1.00e-15 &      \\ 
     &           &    6 &  3.51e-08 &  1.22e-15 &      \\ 
     &           &    7 &  4.26e-08 &  1.00e-15 &      \\ 
     &           &    8 &  3.52e-08 &  1.22e-15 &      \\ 
     &           &    9 &  4.26e-08 &  1.00e-15 &      \\ 
     &           &   10 &  3.52e-08 &  1.22e-15 &      \\ 
3049 &  3.05e+04 &   10 &           &           & iters  \\ 
 \hdashline 
     &           &    1 &  3.16e-08 &  1.36e-09 &      \\ 
     &           &    2 &  3.16e-08 &  9.03e-16 &      \\ 
     &           &    3 &  4.61e-08 &  9.02e-16 &      \\ 
     &           &    4 &  3.16e-08 &  1.32e-15 &      \\ 
     &           &    5 &  3.16e-08 &  9.02e-16 &      \\ 
     &           &    6 &  4.62e-08 &  9.01e-16 &      \\ 
     &           &    7 &  3.16e-08 &  1.32e-15 &      \\ 
     &           &    8 &  4.61e-08 &  9.02e-16 &      \\ 
     &           &    9 &  3.16e-08 &  1.32e-15 &      \\ 
     &           &   10 &  3.16e-08 &  9.02e-16 &      \\ 
3050 &  3.05e+04 &   10 &           &           & iters  \\ 
 \hdashline 
     &           &    1 &  3.00e-08 &  2.30e-10 &      \\ 
     &           &    2 &  3.00e-08 &  8.57e-16 &      \\ 
     &           &    3 &  4.78e-08 &  8.56e-16 &      \\ 
     &           &    4 &  3.00e-08 &  1.36e-15 &      \\ 
     &           &    5 &  4.77e-08 &  8.56e-16 &      \\ 
     &           &    6 &  3.00e-08 &  1.36e-15 &      \\ 
     &           &    7 &  3.00e-08 &  8.57e-16 &      \\ 
     &           &    8 &  4.77e-08 &  8.56e-16 &      \\ 
     &           &    9 &  3.00e-08 &  1.36e-15 &      \\ 
     &           &   10 &  3.00e-08 &  8.56e-16 &      \\ 
3051 &  3.05e+04 &   10 &           &           & iters  \\ 
 \hdashline 
     &           &    1 &  4.98e-08 &  1.56e-09 &      \\ 
     &           &    2 &  2.80e-08 &  1.42e-15 &      \\ 
     &           &    3 &  2.79e-08 &  7.99e-16 &      \\ 
     &           &    4 &  4.98e-08 &  7.98e-16 &      \\ 
     &           &    5 &  2.80e-08 &  1.42e-15 &      \\ 
     &           &    6 &  2.79e-08 &  7.98e-16 &      \\ 
     &           &    7 &  4.98e-08 &  7.97e-16 &      \\ 
     &           &    8 &  2.80e-08 &  1.42e-15 &      \\ 
     &           &    9 &  4.98e-08 &  7.98e-16 &      \\ 
     &           &   10 &  2.80e-08 &  1.42e-15 &      \\ 
3052 &  3.05e+04 &   10 &           &           & iters  \\ 
 \hdashline 
     &           &    1 &  7.41e-09 &  1.04e-09 &      \\ 
     &           &    2 &  7.40e-09 &  2.12e-16 &      \\ 
     &           &    3 &  7.39e-09 &  2.11e-16 &      \\ 
     &           &    4 &  7.38e-09 &  2.11e-16 &      \\ 
     &           &    5 &  7.37e-09 &  2.11e-16 &      \\ 
     &           &    6 &  7.36e-09 &  2.10e-16 &      \\ 
     &           &    7 &  7.35e-09 &  2.10e-16 &      \\ 
     &           &    8 &  7.34e-09 &  2.10e-16 &      \\ 
     &           &    9 &  7.33e-09 &  2.09e-16 &      \\ 
     &           &   10 &  7.32e-09 &  2.09e-16 &      \\ 
3053 &  3.05e+04 &   10 &           &           & iters  \\ 
 \hdashline 
     &           &    1 &  7.11e-08 &  1.17e-11 &      \\ 
     &           &    2 &  6.68e-09 &  2.03e-15 &      \\ 
     &           &    3 &  6.67e-09 &  1.91e-16 &      \\ 
     &           &    4 &  6.66e-09 &  1.90e-16 &      \\ 
     &           &    5 &  6.65e-09 &  1.90e-16 &      \\ 
     &           &    6 &  6.64e-09 &  1.90e-16 &      \\ 
     &           &    7 &  6.63e-09 &  1.89e-16 &      \\ 
     &           &    8 &  6.62e-09 &  1.89e-16 &      \\ 
     &           &    9 &  6.61e-09 &  1.89e-16 &      \\ 
     &           &   10 &  6.60e-09 &  1.89e-16 &      \\ 
3054 &  3.05e+04 &   10 &           &           & iters  \\ 
 \hdashline 
     &           &    1 &  5.27e-08 &  1.44e-09 &      \\ 
     &           &    2 &  2.51e-08 &  1.50e-15 &      \\ 
     &           &    3 &  2.50e-08 &  7.16e-16 &      \\ 
     &           &    4 &  5.27e-08 &  7.15e-16 &      \\ 
     &           &    5 &  2.51e-08 &  1.50e-15 &      \\ 
     &           &    6 &  2.50e-08 &  7.16e-16 &      \\ 
     &           &    7 &  5.27e-08 &  7.15e-16 &      \\ 
     &           &    8 &  2.51e-08 &  1.50e-15 &      \\ 
     &           &    9 &  2.50e-08 &  7.16e-16 &      \\ 
     &           &   10 &  5.27e-08 &  7.15e-16 &      \\ 
3055 &  3.05e+04 &   10 &           &           & iters  \\ 
 \hdashline 
     &           &    1 &  2.11e-08 &  2.90e-09 &      \\ 
     &           &    2 &  2.11e-08 &  6.04e-16 &      \\ 
     &           &    3 &  5.66e-08 &  6.03e-16 &      \\ 
     &           &    4 &  2.12e-08 &  1.62e-15 &      \\ 
     &           &    5 &  2.11e-08 &  6.04e-16 &      \\ 
     &           &    6 &  2.11e-08 &  6.03e-16 &      \\ 
     &           &    7 &  5.66e-08 &  6.02e-16 &      \\ 
     &           &    8 &  2.12e-08 &  1.62e-15 &      \\ 
     &           &    9 &  2.11e-08 &  6.04e-16 &      \\ 
     &           &   10 &  2.11e-08 &  6.03e-16 &      \\ 
3056 &  3.06e+04 &   10 &           &           & iters  \\ 
 \hdashline 
     &           &    1 &  3.21e-08 &  2.13e-09 &      \\ 
     &           &    2 &  6.83e-09 &  9.15e-16 &      \\ 
     &           &    3 &  6.82e-09 &  1.95e-16 &      \\ 
     &           &    4 &  3.20e-08 &  1.95e-16 &      \\ 
     &           &    5 &  6.86e-09 &  9.14e-16 &      \\ 
     &           &    6 &  6.85e-09 &  1.96e-16 &      \\ 
     &           &    7 &  6.84e-09 &  1.95e-16 &      \\ 
     &           &    8 &  6.83e-09 &  1.95e-16 &      \\ 
     &           &    9 &  3.20e-08 &  1.95e-16 &      \\ 
     &           &   10 &  6.86e-09 &  9.14e-16 &      \\ 
3057 &  3.06e+04 &   10 &           &           & iters  \\ 
 \hdashline 
     &           &    1 &  1.47e-08 &  1.39e-10 &      \\ 
     &           &    2 &  1.47e-08 &  4.21e-16 &      \\ 
     &           &    3 &  1.47e-08 &  4.20e-16 &      \\ 
     &           &    4 &  6.30e-08 &  4.20e-16 &      \\ 
     &           &    5 &  1.48e-08 &  1.80e-15 &      \\ 
     &           &    6 &  1.47e-08 &  4.22e-16 &      \\ 
     &           &    7 &  1.47e-08 &  4.21e-16 &      \\ 
     &           &    8 &  1.47e-08 &  4.20e-16 &      \\ 
     &           &    9 &  6.30e-08 &  4.20e-16 &      \\ 
     &           &   10 &  1.48e-08 &  1.80e-15 &      \\ 
3058 &  3.06e+04 &   10 &           &           & iters  \\ 
 \hdashline 
     &           &    1 &  3.71e-08 &  1.69e-09 &      \\ 
     &           &    2 &  4.06e-08 &  1.06e-15 &      \\ 
     &           &    3 &  3.71e-08 &  1.16e-15 &      \\ 
     &           &    4 &  4.06e-08 &  1.06e-15 &      \\ 
     &           &    5 &  3.72e-08 &  1.16e-15 &      \\ 
     &           &    6 &  4.06e-08 &  1.06e-15 &      \\ 
     &           &    7 &  3.72e-08 &  1.16e-15 &      \\ 
     &           &    8 &  4.06e-08 &  1.06e-15 &      \\ 
     &           &    9 &  3.72e-08 &  1.16e-15 &      \\ 
     &           &   10 &  3.71e-08 &  1.06e-15 &      \\ 
3059 &  3.06e+04 &   10 &           &           & iters  \\ 
 \hdashline 
     &           &    1 &  2.61e-08 &  1.78e-09 &      \\ 
     &           &    2 &  2.61e-08 &  7.45e-16 &      \\ 
     &           &    3 &  1.28e-08 &  7.44e-16 &      \\ 
     &           &    4 &  2.61e-08 &  3.66e-16 &      \\ 
     &           &    5 &  1.28e-08 &  7.44e-16 &      \\ 
     &           &    6 &  1.28e-08 &  3.66e-16 &      \\ 
     &           &    7 &  2.61e-08 &  3.66e-16 &      \\ 
     &           &    8 &  1.28e-08 &  7.44e-16 &      \\ 
     &           &    9 &  1.28e-08 &  3.66e-16 &      \\ 
     &           &   10 &  2.61e-08 &  3.66e-16 &      \\ 
3060 &  3.06e+04 &   10 &           &           & iters  \\ 
 \hdashline 
     &           &    1 &  5.04e-08 &  5.23e-10 &      \\ 
     &           &    2 &  2.74e-08 &  1.44e-15 &      \\ 
     &           &    3 &  2.73e-08 &  7.82e-16 &      \\ 
     &           &    4 &  5.04e-08 &  7.81e-16 &      \\ 
     &           &    5 &  2.74e-08 &  1.44e-15 &      \\ 
     &           &    6 &  2.73e-08 &  7.81e-16 &      \\ 
     &           &    7 &  5.04e-08 &  7.80e-16 &      \\ 
     &           &    8 &  2.74e-08 &  1.44e-15 &      \\ 
     &           &    9 &  5.04e-08 &  7.81e-16 &      \\ 
     &           &   10 &  2.74e-08 &  1.44e-15 &      \\ 
3061 &  3.06e+04 &   10 &           &           & iters  \\ 
 \hdashline 
     &           &    1 &  1.52e-08 &  1.05e-10 &      \\ 
     &           &    2 &  1.52e-08 &  4.34e-16 &      \\ 
     &           &    3 &  1.52e-08 &  4.33e-16 &      \\ 
     &           &    4 &  6.26e-08 &  4.32e-16 &      \\ 
     &           &    5 &  1.52e-08 &  1.79e-15 &      \\ 
     &           &    6 &  1.52e-08 &  4.34e-16 &      \\ 
     &           &    7 &  1.52e-08 &  4.34e-16 &      \\ 
     &           &    8 &  1.52e-08 &  4.33e-16 &      \\ 
     &           &    9 &  6.26e-08 &  4.33e-16 &      \\ 
     &           &   10 &  1.52e-08 &  1.79e-15 &      \\ 
3062 &  3.06e+04 &   10 &           &           & iters  \\ 
 \hdashline 
     &           &    1 &  3.51e-08 &  4.58e-10 &      \\ 
     &           &    2 &  3.50e-08 &  1.00e-15 &      \\ 
     &           &    3 &  4.27e-08 &  1.00e-15 &      \\ 
     &           &    4 &  3.50e-08 &  1.22e-15 &      \\ 
     &           &    5 &  4.27e-08 &  1.00e-15 &      \\ 
     &           &    6 &  3.51e-08 &  1.22e-15 &      \\ 
     &           &    7 &  4.27e-08 &  1.00e-15 &      \\ 
     &           &    8 &  3.51e-08 &  1.22e-15 &      \\ 
     &           &    9 &  4.27e-08 &  1.00e-15 &      \\ 
     &           &   10 &  3.51e-08 &  1.22e-15 &      \\ 
3063 &  3.06e+04 &   10 &           &           & iters  \\ 
 \hdashline 
     &           &    1 &  5.82e-08 &  1.19e-09 &      \\ 
     &           &    2 &  1.95e-08 &  1.66e-15 &      \\ 
     &           &    3 &  1.95e-08 &  5.57e-16 &      \\ 
     &           &    4 &  1.95e-08 &  5.56e-16 &      \\ 
     &           &    5 &  5.82e-08 &  5.56e-16 &      \\ 
     &           &    6 &  1.95e-08 &  1.66e-15 &      \\ 
     &           &    7 &  1.95e-08 &  5.57e-16 &      \\ 
     &           &    8 &  1.95e-08 &  5.56e-16 &      \\ 
     &           &    9 &  5.82e-08 &  5.56e-16 &      \\ 
     &           &   10 &  1.95e-08 &  1.66e-15 &      \\ 
3064 &  3.06e+04 &   10 &           &           & iters  \\ 
 \hdashline 
     &           &    1 &  1.89e-08 &  1.98e-09 &      \\ 
     &           &    2 &  1.89e-08 &  5.40e-16 &      \\ 
     &           &    3 &  5.88e-08 &  5.39e-16 &      \\ 
     &           &    4 &  1.89e-08 &  1.68e-15 &      \\ 
     &           &    5 &  1.89e-08 &  5.41e-16 &      \\ 
     &           &    6 &  1.89e-08 &  5.40e-16 &      \\ 
     &           &    7 &  5.88e-08 &  5.39e-16 &      \\ 
     &           &    8 &  1.90e-08 &  1.68e-15 &      \\ 
     &           &    9 &  1.89e-08 &  5.41e-16 &      \\ 
     &           &   10 &  1.89e-08 &  5.40e-16 &      \\ 
3065 &  3.06e+04 &   10 &           &           & iters  \\ 
 \hdashline 
     &           &    1 &  3.31e-08 &  1.57e-09 &      \\ 
     &           &    2 &  3.30e-08 &  9.44e-16 &      \\ 
     &           &    3 &  4.47e-08 &  9.42e-16 &      \\ 
     &           &    4 &  3.30e-08 &  1.28e-15 &      \\ 
     &           &    5 &  4.47e-08 &  9.43e-16 &      \\ 
     &           &    6 &  3.30e-08 &  1.28e-15 &      \\ 
     &           &    7 &  3.30e-08 &  9.43e-16 &      \\ 
     &           &    8 &  4.47e-08 &  9.42e-16 &      \\ 
     &           &    9 &  3.30e-08 &  1.28e-15 &      \\ 
     &           &   10 &  4.47e-08 &  9.42e-16 &      \\ 
3066 &  3.06e+04 &   10 &           &           & iters  \\ 
 \hdashline 
     &           &    1 &  1.10e-08 &  9.80e-10 &      \\ 
     &           &    2 &  1.10e-08 &  3.15e-16 &      \\ 
     &           &    3 &  6.67e-08 &  3.14e-16 &      \\ 
     &           &    4 &  1.11e-08 &  1.90e-15 &      \\ 
     &           &    5 &  1.11e-08 &  3.17e-16 &      \\ 
     &           &    6 &  1.11e-08 &  3.16e-16 &      \\ 
     &           &    7 &  1.10e-08 &  3.16e-16 &      \\ 
     &           &    8 &  1.10e-08 &  3.15e-16 &      \\ 
     &           &    9 &  1.10e-08 &  3.15e-16 &      \\ 
     &           &   10 &  6.67e-08 &  3.14e-16 &      \\ 
3067 &  3.07e+04 &   10 &           &           & iters  \\ 
 \hdashline 
     &           &    1 &  8.16e-08 &  2.77e-09 &      \\ 
     &           &    2 &  7.38e-08 &  2.33e-15 &      \\ 
     &           &    3 &  3.95e-09 &  2.11e-15 &      \\ 
     &           &    4 &  3.94e-09 &  1.13e-16 &      \\ 
     &           &    5 &  3.93e-09 &  1.12e-16 &      \\ 
     &           &    6 &  3.93e-09 &  1.12e-16 &      \\ 
     &           &    7 &  3.92e-09 &  1.12e-16 &      \\ 
     &           &    8 &  3.92e-09 &  1.12e-16 &      \\ 
     &           &    9 &  3.91e-09 &  1.12e-16 &      \\ 
     &           &   10 &  3.91e-09 &  1.12e-16 &      \\ 
3068 &  3.07e+04 &   10 &           &           & iters  \\ 
 \hdashline 
     &           &    1 &  2.51e-08 &  1.86e-09 &      \\ 
     &           &    2 &  5.27e-08 &  7.15e-16 &      \\ 
     &           &    3 &  2.51e-08 &  1.50e-15 &      \\ 
     &           &    4 &  2.51e-08 &  7.16e-16 &      \\ 
     &           &    5 &  5.27e-08 &  7.15e-16 &      \\ 
     &           &    6 &  2.51e-08 &  1.50e-15 &      \\ 
     &           &    7 &  2.51e-08 &  7.17e-16 &      \\ 
     &           &    8 &  5.27e-08 &  7.16e-16 &      \\ 
     &           &    9 &  2.51e-08 &  1.50e-15 &      \\ 
     &           &   10 &  2.51e-08 &  7.17e-16 &      \\ 
3069 &  3.07e+04 &   10 &           &           & iters  \\ 
 \hdashline 
     &           &    1 &  2.30e-08 &  1.53e-09 &      \\ 
     &           &    2 &  1.59e-08 &  6.55e-16 &      \\ 
     &           &    3 &  2.29e-08 &  4.55e-16 &      \\ 
     &           &    4 &  1.59e-08 &  6.55e-16 &      \\ 
     &           &    5 &  1.59e-08 &  4.55e-16 &      \\ 
     &           &    6 &  2.30e-08 &  4.54e-16 &      \\ 
     &           &    7 &  1.59e-08 &  6.55e-16 &      \\ 
     &           &    8 &  2.29e-08 &  4.54e-16 &      \\ 
     &           &    9 &  1.59e-08 &  6.55e-16 &      \\ 
     &           &   10 &  1.59e-08 &  4.55e-16 &      \\ 
3070 &  3.07e+04 &   10 &           &           & iters  \\ 
 \hdashline 
     &           &    1 &  3.51e-08 &  1.77e-09 &      \\ 
     &           &    2 &  4.27e-08 &  1.00e-15 &      \\ 
     &           &    3 &  3.51e-08 &  1.22e-15 &      \\ 
     &           &    4 &  4.26e-08 &  1.00e-15 &      \\ 
     &           &    5 &  3.51e-08 &  1.22e-15 &      \\ 
     &           &    6 &  3.51e-08 &  1.00e-15 &      \\ 
     &           &    7 &  4.27e-08 &  1.00e-15 &      \\ 
     &           &    8 &  3.51e-08 &  1.22e-15 &      \\ 
     &           &    9 &  4.27e-08 &  1.00e-15 &      \\ 
     &           &   10 &  3.51e-08 &  1.22e-15 &      \\ 
3071 &  3.07e+04 &   10 &           &           & iters  \\ 
 \hdashline 
     &           &    1 &  1.38e-08 &  2.60e-10 &      \\ 
     &           &    2 &  1.38e-08 &  3.94e-16 &      \\ 
     &           &    3 &  6.39e-08 &  3.93e-16 &      \\ 
     &           &    4 &  1.39e-08 &  1.82e-15 &      \\ 
     &           &    5 &  1.38e-08 &  3.95e-16 &      \\ 
     &           &    6 &  1.38e-08 &  3.95e-16 &      \\ 
     &           &    7 &  1.38e-08 &  3.94e-16 &      \\ 
     &           &    8 &  6.39e-08 &  3.94e-16 &      \\ 
     &           &    9 &  1.39e-08 &  1.82e-15 &      \\ 
     &           &   10 &  1.38e-08 &  3.96e-16 &      \\ 
3072 &  3.07e+04 &   10 &           &           & iters  \\ 
 \hdashline 
     &           &    1 &  5.18e-10 &  1.29e-09 &      \\ 
     &           &    2 &  5.17e-10 &  1.48e-17 &      \\ 
     &           &    3 &  5.16e-10 &  1.48e-17 &      \\ 
     &           &    4 &  5.16e-10 &  1.47e-17 &      \\ 
     &           &    5 &  5.15e-10 &  1.47e-17 &      \\ 
     &           &    6 &  5.14e-10 &  1.47e-17 &      \\ 
     &           &    7 &  5.13e-10 &  1.47e-17 &      \\ 
     &           &    8 &  5.13e-10 &  1.47e-17 &      \\ 
     &           &    9 &  5.12e-10 &  1.46e-17 &      \\ 
     &           &   10 &  5.11e-10 &  1.46e-17 &      \\ 
3073 &  3.07e+04 &   10 &           &           & iters  \\ 
 \hdashline 
     &           &    1 &  2.68e-08 &  5.70e-10 &      \\ 
     &           &    2 &  1.21e-08 &  7.64e-16 &      \\ 
     &           &    3 &  1.21e-08 &  3.45e-16 &      \\ 
     &           &    4 &  1.21e-08 &  3.45e-16 &      \\ 
     &           &    5 &  2.68e-08 &  3.44e-16 &      \\ 
     &           &    6 &  1.21e-08 &  7.65e-16 &      \\ 
     &           &    7 &  1.21e-08 &  3.45e-16 &      \\ 
     &           &    8 &  2.68e-08 &  3.45e-16 &      \\ 
     &           &    9 &  1.21e-08 &  7.65e-16 &      \\ 
     &           &   10 &  1.21e-08 &  3.45e-16 &      \\ 
3074 &  3.07e+04 &   10 &           &           & iters  \\ 
 \hdashline 
     &           &    1 &  5.98e-09 &  4.86e-10 &      \\ 
     &           &    2 &  5.98e-09 &  1.71e-16 &      \\ 
     &           &    3 &  5.97e-09 &  1.71e-16 &      \\ 
     &           &    4 &  5.96e-09 &  1.70e-16 &      \\ 
     &           &    5 &  7.17e-08 &  1.70e-16 &      \\ 
     &           &    6 &  4.49e-08 &  2.05e-15 &      \\ 
     &           &    7 &  5.99e-09 &  1.28e-15 &      \\ 
     &           &    8 &  5.98e-09 &  1.71e-16 &      \\ 
     &           &    9 &  5.97e-09 &  1.71e-16 &      \\ 
     &           &   10 &  5.96e-09 &  1.70e-16 &      \\ 
3075 &  3.07e+04 &   10 &           &           & iters  \\ 
 \hdashline 
     &           &    1 &  2.39e-08 &  7.80e-10 &      \\ 
     &           &    2 &  2.39e-08 &  6.83e-16 &      \\ 
     &           &    3 &  2.39e-08 &  6.82e-16 &      \\ 
     &           &    4 &  5.39e-08 &  6.81e-16 &      \\ 
     &           &    5 &  2.39e-08 &  1.54e-15 &      \\ 
     &           &    6 &  2.39e-08 &  6.82e-16 &      \\ 
     &           &    7 &  5.38e-08 &  6.81e-16 &      \\ 
     &           &    8 &  2.39e-08 &  1.54e-15 &      \\ 
     &           &    9 &  2.39e-08 &  6.83e-16 &      \\ 
     &           &   10 &  5.38e-08 &  6.82e-16 &      \\ 
3076 &  3.08e+04 &   10 &           &           & iters  \\ 
 \hdashline 
     &           &    1 &  4.36e-08 &  1.64e-09 &      \\ 
     &           &    2 &  3.42e-08 &  1.24e-15 &      \\ 
     &           &    3 &  4.36e-08 &  9.76e-16 &      \\ 
     &           &    4 &  3.42e-08 &  1.24e-15 &      \\ 
     &           &    5 &  3.41e-08 &  9.76e-16 &      \\ 
     &           &    6 &  4.36e-08 &  9.75e-16 &      \\ 
     &           &    7 &  3.42e-08 &  1.24e-15 &      \\ 
     &           &    8 &  4.36e-08 &  9.75e-16 &      \\ 
     &           &    9 &  3.42e-08 &  1.24e-15 &      \\ 
     &           &   10 &  4.36e-08 &  9.75e-16 &      \\ 
3077 &  3.08e+04 &   10 &           &           & iters  \\ 
 \hdashline 
     &           &    1 &  4.46e-09 &  1.05e-09 &      \\ 
     &           &    2 &  4.46e-09 &  1.27e-16 &      \\ 
     &           &    3 &  4.45e-09 &  1.27e-16 &      \\ 
     &           &    4 &  4.44e-09 &  1.27e-16 &      \\ 
     &           &    5 &  4.44e-09 &  1.27e-16 &      \\ 
     &           &    6 &  7.33e-08 &  1.27e-16 &      \\ 
     &           &    7 &  8.22e-08 &  2.09e-15 &      \\ 
     &           &    8 &  7.33e-08 &  2.35e-15 &      \\ 
     &           &    9 &  4.52e-09 &  2.09e-15 &      \\ 
     &           &   10 &  4.52e-09 &  1.29e-16 &      \\ 
3078 &  3.08e+04 &   10 &           &           & iters  \\ 
 \hdashline 
     &           &    1 &  3.48e-08 &  1.61e-09 &      \\ 
     &           &    2 &  4.29e-08 &  9.95e-16 &      \\ 
     &           &    3 &  3.49e-08 &  1.22e-15 &      \\ 
     &           &    4 &  4.29e-08 &  9.95e-16 &      \\ 
     &           &    5 &  3.49e-08 &  1.22e-15 &      \\ 
     &           &    6 &  4.29e-08 &  9.95e-16 &      \\ 
     &           &    7 &  3.49e-08 &  1.22e-15 &      \\ 
     &           &    8 &  4.29e-08 &  9.96e-16 &      \\ 
     &           &    9 &  3.49e-08 &  1.22e-15 &      \\ 
     &           &   10 &  4.28e-08 &  9.96e-16 &      \\ 
3079 &  3.08e+04 &   10 &           &           & iters  \\ 
 \hdashline 
     &           &    1 &  1.39e-08 &  8.72e-10 &      \\ 
     &           &    2 &  1.39e-08 &  3.97e-16 &      \\ 
     &           &    3 &  1.39e-08 &  3.97e-16 &      \\ 
     &           &    4 &  1.39e-08 &  3.96e-16 &      \\ 
     &           &    5 &  1.38e-08 &  3.96e-16 &      \\ 
     &           &    6 &  1.38e-08 &  3.95e-16 &      \\ 
     &           &    7 &  6.39e-08 &  3.95e-16 &      \\ 
     &           &    8 &  1.39e-08 &  1.82e-15 &      \\ 
     &           &    9 &  1.39e-08 &  3.97e-16 &      \\ 
     &           &   10 &  1.39e-08 &  3.96e-16 &      \\ 
3080 &  3.08e+04 &   10 &           &           & iters  \\ 
 \hdashline 
     &           &    1 &  1.29e-08 &  2.49e-09 &      \\ 
     &           &    2 &  6.48e-08 &  3.68e-16 &      \\ 
     &           &    3 &  1.30e-08 &  1.85e-15 &      \\ 
     &           &    4 &  1.29e-08 &  3.70e-16 &      \\ 
     &           &    5 &  1.29e-08 &  3.70e-16 &      \\ 
     &           &    6 &  1.29e-08 &  3.69e-16 &      \\ 
     &           &    7 &  1.29e-08 &  3.68e-16 &      \\ 
     &           &    8 &  6.48e-08 &  3.68e-16 &      \\ 
     &           &    9 &  1.30e-08 &  1.85e-15 &      \\ 
     &           &   10 &  1.29e-08 &  3.70e-16 &      \\ 
3081 &  3.08e+04 &   10 &           &           & iters  \\ 
 \hdashline 
     &           &    1 &  3.49e-08 &  2.98e-09 &      \\ 
     &           &    2 &  4.28e-08 &  9.97e-16 &      \\ 
     &           &    3 &  3.50e-08 &  1.22e-15 &      \\ 
     &           &    4 &  4.28e-08 &  9.98e-16 &      \\ 
     &           &    5 &  3.50e-08 &  1.22e-15 &      \\ 
     &           &    6 &  4.28e-08 &  9.98e-16 &      \\ 
     &           &    7 &  3.50e-08 &  1.22e-15 &      \\ 
     &           &    8 &  4.28e-08 &  9.98e-16 &      \\ 
     &           &    9 &  3.50e-08 &  1.22e-15 &      \\ 
     &           &   10 &  3.49e-08 &  9.99e-16 &      \\ 
3082 &  3.08e+04 &   10 &           &           & iters  \\ 
 \hdashline 
     &           &    1 &  4.02e-08 &  1.77e-11 &      \\ 
     &           &    2 &  1.34e-09 &  1.15e-15 &      \\ 
     &           &    3 &  1.34e-09 &  3.83e-17 &      \\ 
     &           &    4 &  1.34e-09 &  3.83e-17 &      \\ 
     &           &    5 &  1.34e-09 &  3.82e-17 &      \\ 
     &           &    6 &  1.33e-09 &  3.81e-17 &      \\ 
     &           &    7 &  1.33e-09 &  3.81e-17 &      \\ 
     &           &    8 &  1.33e-09 &  3.80e-17 &      \\ 
     &           &    9 &  1.33e-09 &  3.80e-17 &      \\ 
     &           &   10 &  1.33e-09 &  3.79e-17 &      \\ 
3083 &  3.08e+04 &   10 &           &           & iters  \\ 
 \hdashline 
     &           &    1 &  2.27e-08 &  1.80e-09 &      \\ 
     &           &    2 &  1.62e-08 &  6.47e-16 &      \\ 
     &           &    3 &  2.26e-08 &  4.63e-16 &      \\ 
     &           &    4 &  1.62e-08 &  6.46e-16 &      \\ 
     &           &    5 &  1.62e-08 &  4.63e-16 &      \\ 
     &           &    6 &  2.27e-08 &  4.63e-16 &      \\ 
     &           &    7 &  1.62e-08 &  6.47e-16 &      \\ 
     &           &    8 &  2.26e-08 &  4.63e-16 &      \\ 
     &           &    9 &  1.62e-08 &  6.46e-16 &      \\ 
     &           &   10 &  1.62e-08 &  4.63e-16 &      \\ 
3084 &  3.08e+04 &   10 &           &           & iters  \\ 
 \hdashline 
     &           &    1 &  3.43e-09 &  1.99e-09 &      \\ 
     &           &    2 &  3.42e-09 &  9.79e-17 &      \\ 
     &           &    3 &  3.42e-09 &  9.77e-17 &      \\ 
     &           &    4 &  3.41e-09 &  9.76e-17 &      \\ 
     &           &    5 &  3.41e-09 &  9.74e-17 &      \\ 
     &           &    6 &  3.40e-09 &  9.73e-17 &      \\ 
     &           &    7 &  3.40e-09 &  9.72e-17 &      \\ 
     &           &    8 &  3.39e-09 &  9.70e-17 &      \\ 
     &           &    9 &  3.39e-09 &  9.69e-17 &      \\ 
     &           &   10 &  3.38e-09 &  9.67e-17 &      \\ 
3085 &  3.08e+04 &   10 &           &           & iters  \\ 
 \hdashline 
     &           &    1 &  9.92e-10 &  1.42e-09 &      \\ 
     &           &    2 &  9.91e-10 &  2.83e-17 &      \\ 
     &           &    3 &  9.89e-10 &  2.83e-17 &      \\ 
     &           &    4 &  9.88e-10 &  2.82e-17 &      \\ 
     &           &    5 &  9.87e-10 &  2.82e-17 &      \\ 
     &           &    6 &  9.85e-10 &  2.82e-17 &      \\ 
     &           &    7 &  9.84e-10 &  2.81e-17 &      \\ 
     &           &    8 &  9.82e-10 &  2.81e-17 &      \\ 
     &           &    9 &  9.81e-10 &  2.80e-17 &      \\ 
     &           &   10 &  9.80e-10 &  2.80e-17 &      \\ 
3086 &  3.08e+04 &   10 &           &           & iters  \\ 
 \hdashline 
     &           &    1 &  2.87e-08 &  3.18e-10 &      \\ 
     &           &    2 &  4.91e-08 &  8.18e-16 &      \\ 
     &           &    3 &  2.87e-08 &  1.40e-15 &      \\ 
     &           &    4 &  4.90e-08 &  8.19e-16 &      \\ 
     &           &    5 &  2.87e-08 &  1.40e-15 &      \\ 
     &           &    6 &  2.87e-08 &  8.20e-16 &      \\ 
     &           &    7 &  4.90e-08 &  8.19e-16 &      \\ 
     &           &    8 &  2.87e-08 &  1.40e-15 &      \\ 
     &           &    9 &  2.87e-08 &  8.20e-16 &      \\ 
     &           &   10 &  4.91e-08 &  8.18e-16 &      \\ 
3087 &  3.09e+04 &   10 &           &           & iters  \\ 
 \hdashline 
     &           &    1 &  2.83e-08 &  1.67e-09 &      \\ 
     &           &    2 &  4.94e-08 &  8.09e-16 &      \\ 
     &           &    3 &  2.84e-08 &  1.41e-15 &      \\ 
     &           &    4 &  2.83e-08 &  8.10e-16 &      \\ 
     &           &    5 &  4.94e-08 &  8.08e-16 &      \\ 
     &           &    6 &  2.84e-08 &  1.41e-15 &      \\ 
     &           &    7 &  2.83e-08 &  8.09e-16 &      \\ 
     &           &    8 &  4.94e-08 &  8.08e-16 &      \\ 
     &           &    9 &  2.83e-08 &  1.41e-15 &      \\ 
     &           &   10 &  4.94e-08 &  8.09e-16 &      \\ 
3088 &  3.09e+04 &   10 &           &           & iters  \\ 
 \hdashline 
     &           &    1 &  3.77e-08 &  8.05e-10 &      \\ 
     &           &    2 &  3.76e-08 &  1.08e-15 &      \\ 
     &           &    3 &  4.01e-08 &  1.07e-15 &      \\ 
     &           &    4 &  3.76e-08 &  1.14e-15 &      \\ 
     &           &    5 &  4.01e-08 &  1.07e-15 &      \\ 
     &           &    6 &  3.76e-08 &  1.14e-15 &      \\ 
     &           &    7 &  4.01e-08 &  1.07e-15 &      \\ 
     &           &    8 &  3.76e-08 &  1.14e-15 &      \\ 
     &           &    9 &  4.01e-08 &  1.07e-15 &      \\ 
     &           &   10 &  3.76e-08 &  1.14e-15 &      \\ 
3089 &  3.09e+04 &   10 &           &           & iters  \\ 
 \hdashline 
     &           &    1 &  3.45e-09 &  7.51e-10 &      \\ 
     &           &    2 &  3.45e-09 &  9.86e-17 &      \\ 
     &           &    3 &  3.44e-09 &  9.84e-17 &      \\ 
     &           &    4 &  3.44e-09 &  9.83e-17 &      \\ 
     &           &    5 &  3.43e-09 &  9.82e-17 &      \\ 
     &           &    6 &  3.43e-09 &  9.80e-17 &      \\ 
     &           &    7 &  3.42e-09 &  9.79e-17 &      \\ 
     &           &    8 &  3.42e-09 &  9.77e-17 &      \\ 
     &           &    9 &  3.41e-09 &  9.76e-17 &      \\ 
     &           &   10 &  3.41e-09 &  9.74e-17 &      \\ 
3090 &  3.09e+04 &   10 &           &           & iters  \\ 
 \hdashline 
     &           &    1 &  3.86e-08 &  6.84e-10 &      \\ 
     &           &    2 &  3.92e-08 &  1.10e-15 &      \\ 
     &           &    3 &  3.86e-08 &  1.12e-15 &      \\ 
     &           &    4 &  3.92e-08 &  1.10e-15 &      \\ 
     &           &    5 &  3.86e-08 &  1.12e-15 &      \\ 
     &           &    6 &  3.92e-08 &  1.10e-15 &      \\ 
     &           &    7 &  3.86e-08 &  1.12e-15 &      \\ 
     &           &    8 &  3.92e-08 &  1.10e-15 &      \\ 
     &           &    9 &  3.86e-08 &  1.12e-15 &      \\ 
     &           &   10 &  3.92e-08 &  1.10e-15 &      \\ 
3091 &  3.09e+04 &   10 &           &           & iters  \\ 
 \hdashline 
     &           &    1 &  2.61e-08 &  9.36e-10 &      \\ 
     &           &    2 &  2.60e-08 &  7.44e-16 &      \\ 
     &           &    3 &  5.17e-08 &  7.43e-16 &      \\ 
     &           &    4 &  2.61e-08 &  1.48e-15 &      \\ 
     &           &    5 &  2.60e-08 &  7.44e-16 &      \\ 
     &           &    6 &  5.17e-08 &  7.43e-16 &      \\ 
     &           &    7 &  2.61e-08 &  1.48e-15 &      \\ 
     &           &    8 &  2.60e-08 &  7.44e-16 &      \\ 
     &           &    9 &  5.17e-08 &  7.43e-16 &      \\ 
     &           &   10 &  2.61e-08 &  1.48e-15 &      \\ 
3092 &  3.09e+04 &   10 &           &           & iters  \\ 
 \hdashline 
     &           &    1 &  8.88e-09 &  1.96e-09 &      \\ 
     &           &    2 &  8.86e-09 &  2.53e-16 &      \\ 
     &           &    3 &  8.85e-09 &  2.53e-16 &      \\ 
     &           &    4 &  3.00e-08 &  2.53e-16 &      \\ 
     &           &    5 &  8.88e-09 &  8.56e-16 &      \\ 
     &           &    6 &  8.87e-09 &  2.53e-16 &      \\ 
     &           &    7 &  8.86e-09 &  2.53e-16 &      \\ 
     &           &    8 &  3.00e-08 &  2.53e-16 &      \\ 
     &           &    9 &  8.89e-09 &  8.56e-16 &      \\ 
     &           &   10 &  8.87e-09 &  2.54e-16 &      \\ 
3093 &  3.09e+04 &   10 &           &           & iters  \\ 
 \hdashline 
     &           &    1 &  1.44e-08 &  8.37e-10 &      \\ 
     &           &    2 &  2.45e-08 &  4.11e-16 &      \\ 
     &           &    3 &  1.44e-08 &  6.98e-16 &      \\ 
     &           &    4 &  1.44e-08 &  4.11e-16 &      \\ 
     &           &    5 &  2.45e-08 &  4.11e-16 &      \\ 
     &           &    6 &  1.44e-08 &  6.98e-16 &      \\ 
     &           &    7 &  2.45e-08 &  4.11e-16 &      \\ 
     &           &    8 &  1.44e-08 &  6.98e-16 &      \\ 
     &           &    9 &  1.44e-08 &  4.12e-16 &      \\ 
     &           &   10 &  2.45e-08 &  4.11e-16 &      \\ 
3094 &  3.09e+04 &   10 &           &           & iters  \\ 
 \hdashline 
     &           &    1 &  1.69e-08 &  7.72e-10 &      \\ 
     &           &    2 &  1.69e-08 &  4.82e-16 &      \\ 
     &           &    3 &  6.08e-08 &  4.82e-16 &      \\ 
     &           &    4 &  1.69e-08 &  1.74e-15 &      \\ 
     &           &    5 &  1.69e-08 &  4.83e-16 &      \\ 
     &           &    6 &  1.69e-08 &  4.83e-16 &      \\ 
     &           &    7 &  6.08e-08 &  4.82e-16 &      \\ 
     &           &    8 &  1.70e-08 &  1.74e-15 &      \\ 
     &           &    9 &  1.69e-08 &  4.84e-16 &      \\ 
     &           &   10 &  1.69e-08 &  4.83e-16 &      \\ 
3095 &  3.09e+04 &   10 &           &           & iters  \\ 
 \hdashline 
     &           &    1 &  1.66e-08 &  2.26e-09 &      \\ 
     &           &    2 &  1.66e-08 &  4.74e-16 &      \\ 
     &           &    3 &  6.11e-08 &  4.73e-16 &      \\ 
     &           &    4 &  1.66e-08 &  1.74e-15 &      \\ 
     &           &    5 &  1.66e-08 &  4.75e-16 &      \\ 
     &           &    6 &  1.66e-08 &  4.74e-16 &      \\ 
     &           &    7 &  6.11e-08 &  4.74e-16 &      \\ 
     &           &    8 &  1.67e-08 &  1.74e-15 &      \\ 
     &           &    9 &  1.66e-08 &  4.75e-16 &      \\ 
     &           &   10 &  1.66e-08 &  4.75e-16 &      \\ 
3096 &  3.10e+04 &   10 &           &           & iters  \\ 
 \hdashline 
     &           &    1 &  1.41e-08 &  4.93e-10 &      \\ 
     &           &    2 &  1.41e-08 &  4.02e-16 &      \\ 
     &           &    3 &  1.40e-08 &  4.01e-16 &      \\ 
     &           &    4 &  6.37e-08 &  4.01e-16 &      \\ 
     &           &    5 &  1.41e-08 &  1.82e-15 &      \\ 
     &           &    6 &  1.41e-08 &  4.03e-16 &      \\ 
     &           &    7 &  1.41e-08 &  4.02e-16 &      \\ 
     &           &    8 &  1.40e-08 &  4.01e-16 &      \\ 
     &           &    9 &  6.37e-08 &  4.01e-16 &      \\ 
     &           &   10 &  1.41e-08 &  1.82e-15 &      \\ 
3097 &  3.10e+04 &   10 &           &           & iters  \\ 
 \hdashline 
     &           &    1 &  4.32e-08 &  3.67e-09 &      \\ 
     &           &    2 &  4.27e-09 &  1.23e-15 &      \\ 
     &           &    3 &  4.26e-09 &  1.22e-16 &      \\ 
     &           &    4 &  4.26e-09 &  1.22e-16 &      \\ 
     &           &    5 &  4.25e-09 &  1.21e-16 &      \\ 
     &           &    6 &  4.24e-09 &  1.21e-16 &      \\ 
     &           &    7 &  4.24e-09 &  1.21e-16 &      \\ 
     &           &    8 &  4.23e-09 &  1.21e-16 &      \\ 
     &           &    9 &  7.35e-08 &  1.21e-16 &      \\ 
     &           &   10 &  4.32e-08 &  2.10e-15 &      \\ 
3098 &  3.10e+04 &   10 &           &           & iters  \\ 
 \hdashline 
     &           &    1 &  2.85e-08 &  3.84e-09 &      \\ 
     &           &    2 &  2.84e-08 &  8.13e-16 &      \\ 
     &           &    3 &  4.93e-08 &  8.12e-16 &      \\ 
     &           &    4 &  2.85e-08 &  1.41e-15 &      \\ 
     &           &    5 &  2.84e-08 &  8.12e-16 &      \\ 
     &           &    6 &  4.93e-08 &  8.11e-16 &      \\ 
     &           &    7 &  2.85e-08 &  1.41e-15 &      \\ 
     &           &    8 &  2.84e-08 &  8.12e-16 &      \\ 
     &           &    9 &  4.93e-08 &  8.11e-16 &      \\ 
     &           &   10 &  2.84e-08 &  1.41e-15 &      \\ 
3099 &  3.10e+04 &   10 &           &           & iters  \\ 
 \hdashline 
     &           &    1 &  5.11e-08 &  9.68e-11 &      \\ 
     &           &    2 &  2.67e-08 &  1.46e-15 &      \\ 
     &           &    3 &  2.66e-08 &  7.62e-16 &      \\ 
     &           &    4 &  5.11e-08 &  7.60e-16 &      \\ 
     &           &    5 &  2.67e-08 &  1.46e-15 &      \\ 
     &           &    6 &  2.66e-08 &  7.61e-16 &      \\ 
     &           &    7 &  5.11e-08 &  7.60e-16 &      \\ 
     &           &    8 &  2.67e-08 &  1.46e-15 &      \\ 
     &           &    9 &  2.66e-08 &  7.61e-16 &      \\ 
     &           &   10 &  5.11e-08 &  7.60e-16 &      \\ 
3100 &  3.10e+04 &   10 &           &           & iters  \\ 
 \hdashline 
     &           &    1 &  4.37e-08 &  2.08e-09 &      \\ 
     &           &    2 &  3.41e-08 &  1.25e-15 &      \\ 
     &           &    3 &  3.40e-08 &  9.73e-16 &      \\ 
     &           &    4 &  4.37e-08 &  9.72e-16 &      \\ 
     &           &    5 &  3.41e-08 &  1.25e-15 &      \\ 
     &           &    6 &  4.37e-08 &  9.72e-16 &      \\ 
     &           &    7 &  3.41e-08 &  1.25e-15 &      \\ 
     &           &    8 &  4.37e-08 &  9.72e-16 &      \\ 
     &           &    9 &  3.41e-08 &  1.25e-15 &      \\ 
     &           &   10 &  3.40e-08 &  9.73e-16 &      \\ 
3101 &  3.10e+04 &   10 &           &           & iters  \\ 
 \hdashline 
     &           &    1 &  9.79e-09 &  5.59e-10 &      \\ 
     &           &    2 &  6.79e-08 &  2.79e-16 &      \\ 
     &           &    3 &  9.87e-09 &  1.94e-15 &      \\ 
     &           &    4 &  9.86e-09 &  2.82e-16 &      \\ 
     &           &    5 &  9.84e-09 &  2.81e-16 &      \\ 
     &           &    6 &  9.83e-09 &  2.81e-16 &      \\ 
     &           &    7 &  9.81e-09 &  2.80e-16 &      \\ 
     &           &    8 &  9.80e-09 &  2.80e-16 &      \\ 
     &           &    9 &  9.78e-09 &  2.80e-16 &      \\ 
     &           &   10 &  6.79e-08 &  2.79e-16 &      \\ 
3102 &  3.10e+04 &   10 &           &           & iters  \\ 
 \hdashline 
     &           &    1 &  4.42e-08 &  1.54e-09 &      \\ 
     &           &    2 &  3.36e-08 &  1.26e-15 &      \\ 
     &           &    3 &  3.35e-08 &  9.59e-16 &      \\ 
     &           &    4 &  4.42e-08 &  9.57e-16 &      \\ 
     &           &    5 &  3.36e-08 &  1.26e-15 &      \\ 
     &           &    6 &  4.42e-08 &  9.58e-16 &      \\ 
     &           &    7 &  3.36e-08 &  1.26e-15 &      \\ 
     &           &    8 &  4.42e-08 &  9.58e-16 &      \\ 
     &           &    9 &  3.36e-08 &  1.26e-15 &      \\ 
     &           &   10 &  3.35e-08 &  9.59e-16 &      \\ 
3103 &  3.10e+04 &   10 &           &           & iters  \\ 
 \hdashline 
     &           &    1 &  2.67e-08 &  2.10e-09 &      \\ 
     &           &    2 &  2.67e-08 &  7.63e-16 &      \\ 
     &           &    3 &  2.67e-08 &  7.62e-16 &      \\ 
     &           &    4 &  2.66e-08 &  7.61e-16 &      \\ 
     &           &    5 &  5.11e-08 &  7.60e-16 &      \\ 
     &           &    6 &  2.66e-08 &  1.46e-15 &      \\ 
     &           &    7 &  5.11e-08 &  7.61e-16 &      \\ 
     &           &    8 &  2.67e-08 &  1.46e-15 &      \\ 
     &           &    9 &  2.66e-08 &  7.62e-16 &      \\ 
     &           &   10 &  5.11e-08 &  7.60e-16 &      \\ 
3104 &  3.10e+04 &   10 &           &           & iters  \\ 
 \hdashline 
     &           &    1 &  5.00e-08 &  1.80e-09 &      \\ 
     &           &    2 &  2.78e-08 &  1.43e-15 &      \\ 
     &           &    3 &  2.77e-08 &  7.93e-16 &      \\ 
     &           &    4 &  5.00e-08 &  7.92e-16 &      \\ 
     &           &    5 &  2.78e-08 &  1.43e-15 &      \\ 
     &           &    6 &  2.77e-08 &  7.93e-16 &      \\ 
     &           &    7 &  5.00e-08 &  7.92e-16 &      \\ 
     &           &    8 &  2.78e-08 &  1.43e-15 &      \\ 
     &           &    9 &  2.77e-08 &  7.93e-16 &      \\ 
     &           &   10 &  5.00e-08 &  7.91e-16 &      \\ 
3105 &  3.10e+04 &   10 &           &           & iters  \\ 
 \hdashline 
     &           &    1 &  3.23e-08 &  2.94e-11 &      \\ 
     &           &    2 &  4.55e-08 &  9.21e-16 &      \\ 
     &           &    3 &  3.23e-08 &  1.30e-15 &      \\ 
     &           &    4 &  4.54e-08 &  9.22e-16 &      \\ 
     &           &    5 &  3.23e-08 &  1.30e-15 &      \\ 
     &           &    6 &  3.23e-08 &  9.22e-16 &      \\ 
     &           &    7 &  4.55e-08 &  9.21e-16 &      \\ 
     &           &    8 &  3.23e-08 &  1.30e-15 &      \\ 
     &           &    9 &  4.54e-08 &  9.22e-16 &      \\ 
     &           &   10 &  3.23e-08 &  1.30e-15 &      \\ 
3106 &  3.10e+04 &   10 &           &           & iters  \\ 
 \hdashline 
     &           &    1 &  6.05e-08 &  4.51e-09 &      \\ 
     &           &    2 &  1.73e-08 &  1.73e-15 &      \\ 
     &           &    3 &  1.72e-08 &  4.93e-16 &      \\ 
     &           &    4 &  1.72e-08 &  4.92e-16 &      \\ 
     &           &    5 &  1.72e-08 &  4.91e-16 &      \\ 
     &           &    6 &  6.05e-08 &  4.91e-16 &      \\ 
     &           &    7 &  1.73e-08 &  1.73e-15 &      \\ 
     &           &    8 &  1.72e-08 &  4.92e-16 &      \\ 
     &           &    9 &  1.72e-08 &  4.92e-16 &      \\ 
     &           &   10 &  6.05e-08 &  4.91e-16 &      \\ 
3107 &  3.11e+04 &   10 &           &           & iters  \\ 
 \hdashline 
     &           &    1 &  5.22e-08 &  3.91e-09 &      \\ 
     &           &    2 &  2.56e-08 &  1.49e-15 &      \\ 
     &           &    3 &  2.55e-08 &  7.30e-16 &      \\ 
     &           &    4 &  5.22e-08 &  7.29e-16 &      \\ 
     &           &    5 &  2.56e-08 &  1.49e-15 &      \\ 
     &           &    6 &  2.55e-08 &  7.30e-16 &      \\ 
     &           &    7 &  5.22e-08 &  7.29e-16 &      \\ 
     &           &    8 &  2.56e-08 &  1.49e-15 &      \\ 
     &           &    9 &  2.55e-08 &  7.30e-16 &      \\ 
     &           &   10 &  5.22e-08 &  7.29e-16 &      \\ 
3108 &  3.11e+04 &   10 &           &           & iters  \\ 
 \hdashline 
     &           &    1 &  2.32e-08 &  2.30e-09 &      \\ 
     &           &    2 &  1.57e-08 &  6.61e-16 &      \\ 
     &           &    3 &  2.31e-08 &  4.49e-16 &      \\ 
     &           &    4 &  1.57e-08 &  6.61e-16 &      \\ 
     &           &    5 &  2.31e-08 &  4.49e-16 &      \\ 
     &           &    6 &  1.57e-08 &  6.60e-16 &      \\ 
     &           &    7 &  1.57e-08 &  4.49e-16 &      \\ 
     &           &    8 &  2.31e-08 &  4.49e-16 &      \\ 
     &           &    9 &  1.57e-08 &  6.61e-16 &      \\ 
     &           &   10 &  2.31e-08 &  4.49e-16 &      \\ 
3109 &  3.11e+04 &   10 &           &           & iters  \\ 
 \hdashline 
     &           &    1 &  9.88e-09 &  4.59e-09 &      \\ 
     &           &    2 &  9.87e-09 &  2.82e-16 &      \\ 
     &           &    3 &  9.86e-09 &  2.82e-16 &      \\ 
     &           &    4 &  9.84e-09 &  2.81e-16 &      \\ 
     &           &    5 &  9.83e-09 &  2.81e-16 &      \\ 
     &           &    6 &  6.79e-08 &  2.81e-16 &      \\ 
     &           &    7 &  8.76e-08 &  1.94e-15 &      \\ 
     &           &    8 &  6.79e-08 &  2.50e-15 &      \\ 
     &           &    9 &  9.88e-09 &  1.94e-15 &      \\ 
     &           &   10 &  9.87e-09 &  2.82e-16 &      \\ 
3110 &  3.11e+04 &   10 &           &           & iters  \\ 
 \hdashline 
     &           &    1 &  4.58e-08 &  2.58e-09 &      \\ 
     &           &    2 &  3.20e-08 &  1.31e-15 &      \\ 
     &           &    3 &  4.58e-08 &  9.13e-16 &      \\ 
     &           &    4 &  3.20e-08 &  1.31e-15 &      \\ 
     &           &    5 &  3.19e-08 &  9.13e-16 &      \\ 
     &           &    6 &  4.58e-08 &  9.12e-16 &      \\ 
     &           &    7 &  3.20e-08 &  1.31e-15 &      \\ 
     &           &    8 &  4.58e-08 &  9.12e-16 &      \\ 
     &           &    9 &  3.20e-08 &  1.31e-15 &      \\ 
     &           &   10 &  4.57e-08 &  9.13e-16 &      \\ 
3111 &  3.11e+04 &   10 &           &           & iters  \\ 
 \hdashline 
     &           &    1 &  3.12e-08 &  9.14e-10 &      \\ 
     &           &    2 &  3.11e-08 &  8.89e-16 &      \\ 
     &           &    3 &  3.11e-08 &  8.88e-16 &      \\ 
     &           &    4 &  4.67e-08 &  8.87e-16 &      \\ 
     &           &    5 &  3.11e-08 &  1.33e-15 &      \\ 
     &           &    6 &  3.11e-08 &  8.88e-16 &      \\ 
     &           &    7 &  4.67e-08 &  8.86e-16 &      \\ 
     &           &    8 &  3.11e-08 &  1.33e-15 &      \\ 
     &           &    9 &  4.67e-08 &  8.87e-16 &      \\ 
     &           &   10 &  3.11e-08 &  1.33e-15 &      \\ 
3112 &  3.11e+04 &   10 &           &           & iters  \\ 
 \hdashline 
     &           &    1 &  1.95e-08 &  2.94e-09 &      \\ 
     &           &    2 &  1.95e-08 &  5.58e-16 &      \\ 
     &           &    3 &  1.95e-08 &  5.57e-16 &      \\ 
     &           &    4 &  1.95e-08 &  5.56e-16 &      \\ 
     &           &    5 &  5.83e-08 &  5.55e-16 &      \\ 
     &           &    6 &  1.95e-08 &  1.66e-15 &      \\ 
     &           &    7 &  1.95e-08 &  5.57e-16 &      \\ 
     &           &    8 &  1.95e-08 &  5.56e-16 &      \\ 
     &           &    9 &  5.83e-08 &  5.55e-16 &      \\ 
     &           &   10 &  1.95e-08 &  1.66e-15 &      \\ 
3113 &  3.11e+04 &   10 &           &           & iters  \\ 
 \hdashline 
     &           &    1 &  2.25e-08 &  4.66e-09 &      \\ 
     &           &    2 &  2.24e-08 &  6.41e-16 &      \\ 
     &           &    3 &  5.53e-08 &  6.41e-16 &      \\ 
     &           &    4 &  2.25e-08 &  1.58e-15 &      \\ 
     &           &    5 &  2.25e-08 &  6.42e-16 &      \\ 
     &           &    6 &  2.24e-08 &  6.41e-16 &      \\ 
     &           &    7 &  5.53e-08 &  6.40e-16 &      \\ 
     &           &    8 &  2.25e-08 &  1.58e-15 &      \\ 
     &           &    9 &  2.24e-08 &  6.41e-16 &      \\ 
     &           &   10 &  5.53e-08 &  6.40e-16 &      \\ 
3114 &  3.11e+04 &   10 &           &           & iters  \\ 
 \hdashline 
     &           &    1 &  3.21e-08 &  6.11e-09 &      \\ 
     &           &    2 &  3.20e-08 &  9.16e-16 &      \\ 
     &           &    3 &  4.57e-08 &  9.15e-16 &      \\ 
     &           &    4 &  3.21e-08 &  1.30e-15 &      \\ 
     &           &    5 &  4.57e-08 &  9.15e-16 &      \\ 
     &           &    6 &  3.21e-08 &  1.30e-15 &      \\ 
     &           &    7 &  3.20e-08 &  9.16e-16 &      \\ 
     &           &    8 &  4.57e-08 &  9.14e-16 &      \\ 
     &           &    9 &  3.21e-08 &  1.30e-15 &      \\ 
     &           &   10 &  4.57e-08 &  9.15e-16 &      \\ 
3115 &  3.11e+04 &   10 &           &           & iters  \\ 
 \hdashline 
     &           &    1 &  4.66e-08 &  4.42e-10 &      \\ 
     &           &    2 &  3.11e-08 &  1.33e-15 &      \\ 
     &           &    3 &  3.11e-08 &  8.88e-16 &      \\ 
     &           &    4 &  4.67e-08 &  8.87e-16 &      \\ 
     &           &    5 &  3.11e-08 &  1.33e-15 &      \\ 
     &           &    6 &  4.66e-08 &  8.87e-16 &      \\ 
     &           &    7 &  3.11e-08 &  1.33e-15 &      \\ 
     &           &    8 &  3.11e-08 &  8.88e-16 &      \\ 
     &           &    9 &  4.67e-08 &  8.87e-16 &      \\ 
     &           &   10 &  3.11e-08 &  1.33e-15 &      \\ 
3116 &  3.12e+04 &   10 &           &           & iters  \\ 
 \hdashline 
     &           &    1 &  6.00e-08 &  6.08e-09 &      \\ 
     &           &    2 &  1.78e-08 &  1.71e-15 &      \\ 
     &           &    3 &  1.77e-08 &  5.07e-16 &      \\ 
     &           &    4 &  1.77e-08 &  5.07e-16 &      \\ 
     &           &    5 &  1.77e-08 &  5.06e-16 &      \\ 
     &           &    6 &  6.00e-08 &  5.05e-16 &      \\ 
     &           &    7 &  1.78e-08 &  1.71e-15 &      \\ 
     &           &    8 &  1.77e-08 &  5.07e-16 &      \\ 
     &           &    9 &  1.77e-08 &  5.06e-16 &      \\ 
     &           &   10 &  6.00e-08 &  5.05e-16 &      \\ 
3117 &  3.12e+04 &   10 &           &           & iters  \\ 
 \hdashline 
     &           &    1 &  2.18e-08 &  3.53e-09 &      \\ 
     &           &    2 &  5.59e-08 &  6.22e-16 &      \\ 
     &           &    3 &  2.18e-08 &  1.60e-15 &      \\ 
     &           &    4 &  2.18e-08 &  6.23e-16 &      \\ 
     &           &    5 &  5.59e-08 &  6.23e-16 &      \\ 
     &           &    6 &  2.19e-08 &  1.60e-15 &      \\ 
     &           &    7 &  2.18e-08 &  6.24e-16 &      \\ 
     &           &    8 &  2.18e-08 &  6.23e-16 &      \\ 
     &           &    9 &  5.59e-08 &  6.22e-16 &      \\ 
     &           &   10 &  2.18e-08 &  1.60e-15 &      \\ 
3118 &  3.12e+04 &   10 &           &           & iters  \\ 
 \hdashline 
     &           &    1 &  1.99e-08 &  2.28e-09 &      \\ 
     &           &    2 &  1.99e-08 &  5.69e-16 &      \\ 
     &           &    3 &  1.99e-08 &  5.69e-16 &      \\ 
     &           &    4 &  5.78e-08 &  5.68e-16 &      \\ 
     &           &    5 &  1.99e-08 &  1.65e-15 &      \\ 
     &           &    6 &  1.99e-08 &  5.69e-16 &      \\ 
     &           &    7 &  5.78e-08 &  5.68e-16 &      \\ 
     &           &    8 &  2.00e-08 &  1.65e-15 &      \\ 
     &           &    9 &  1.99e-08 &  5.70e-16 &      \\ 
     &           &   10 &  1.99e-08 &  5.69e-16 &      \\ 
3119 &  3.12e+04 &   10 &           &           & iters  \\ 
 \hdashline 
     &           &    1 &  3.37e-08 &  1.58e-09 &      \\ 
     &           &    2 &  4.40e-08 &  9.63e-16 &      \\ 
     &           &    3 &  3.38e-08 &  1.26e-15 &      \\ 
     &           &    4 &  4.40e-08 &  9.63e-16 &      \\ 
     &           &    5 &  3.38e-08 &  1.26e-15 &      \\ 
     &           &    6 &  3.37e-08 &  9.64e-16 &      \\ 
     &           &    7 &  4.40e-08 &  9.62e-16 &      \\ 
     &           &    8 &  3.37e-08 &  1.26e-15 &      \\ 
     &           &    9 &  4.40e-08 &  9.63e-16 &      \\ 
     &           &   10 &  3.38e-08 &  1.26e-15 &      \\ 
3120 &  3.12e+04 &   10 &           &           & iters  \\ 
 \hdashline 
     &           &    1 &  4.39e-10 &  3.47e-09 &      \\ 
     &           &    2 &  4.39e-10 &  1.25e-17 &      \\ 
     &           &    3 &  4.38e-10 &  1.25e-17 &      \\ 
     &           &    4 &  4.37e-10 &  1.25e-17 &      \\ 
     &           &    5 &  4.37e-10 &  1.25e-17 &      \\ 
     &           &    6 &  4.36e-10 &  1.25e-17 &      \\ 
     &           &    7 &  4.35e-10 &  1.24e-17 &      \\ 
     &           &    8 &  4.35e-10 &  1.24e-17 &      \\ 
     &           &    9 &  4.34e-10 &  1.24e-17 &      \\ 
     &           &   10 &  4.34e-10 &  1.24e-17 &      \\ 
3121 &  3.12e+04 &   10 &           &           & iters  \\ 
 \hdashline 
     &           &    1 &  2.24e-08 &  3.47e-09 &      \\ 
     &           &    2 &  2.24e-08 &  6.40e-16 &      \\ 
     &           &    3 &  5.53e-08 &  6.39e-16 &      \\ 
     &           &    4 &  2.24e-08 &  1.58e-15 &      \\ 
     &           &    5 &  2.24e-08 &  6.41e-16 &      \\ 
     &           &    6 &  5.53e-08 &  6.40e-16 &      \\ 
     &           &    7 &  2.25e-08 &  1.58e-15 &      \\ 
     &           &    8 &  2.24e-08 &  6.41e-16 &      \\ 
     &           &    9 &  2.24e-08 &  6.40e-16 &      \\ 
     &           &   10 &  5.53e-08 &  6.39e-16 &      \\ 
3122 &  3.12e+04 &   10 &           &           & iters  \\ 
 \hdashline 
     &           &    1 &  8.17e-10 &  3.76e-10 &      \\ 
     &           &    2 &  8.16e-10 &  2.33e-17 &      \\ 
     &           &    3 &  8.15e-10 &  2.33e-17 &      \\ 
     &           &    4 &  8.14e-10 &  2.33e-17 &      \\ 
     &           &    5 &  8.12e-10 &  2.32e-17 &      \\ 
     &           &    6 &  8.11e-10 &  2.32e-17 &      \\ 
     &           &    7 &  8.10e-10 &  2.32e-17 &      \\ 
     &           &    8 &  8.09e-10 &  2.31e-17 &      \\ 
     &           &    9 &  7.69e-08 &  2.31e-17 &      \\ 
     &           &   10 &  7.86e-08 &  2.19e-15 &      \\ 
3123 &  3.12e+04 &   10 &           &           & iters  \\ 
 \hdashline 
     &           &    1 &  6.54e-09 &  1.08e-09 &      \\ 
     &           &    2 &  6.53e-09 &  1.87e-16 &      \\ 
     &           &    3 &  7.12e-08 &  1.86e-16 &      \\ 
     &           &    4 &  8.43e-08 &  2.03e-15 &      \\ 
     &           &    5 &  7.12e-08 &  2.41e-15 &      \\ 
     &           &    6 &  6.60e-09 &  2.03e-15 &      \\ 
     &           &    7 &  6.60e-09 &  1.88e-16 &      \\ 
     &           &    8 &  6.59e-09 &  1.88e-16 &      \\ 
     &           &    9 &  6.58e-09 &  1.88e-16 &      \\ 
     &           &   10 &  6.57e-09 &  1.88e-16 &      \\ 
3124 &  3.12e+04 &   10 &           &           & iters  \\ 
 \hdashline 
     &           &    1 &  5.20e-10 &  1.66e-09 &      \\ 
     &           &    2 &  5.19e-10 &  1.48e-17 &      \\ 
     &           &    3 &  5.18e-10 &  1.48e-17 &      \\ 
     &           &    4 &  5.18e-10 &  1.48e-17 &      \\ 
     &           &    5 &  5.17e-10 &  1.48e-17 &      \\ 
     &           &    6 &  5.16e-10 &  1.48e-17 &      \\ 
     &           &    7 &  5.15e-10 &  1.47e-17 &      \\ 
     &           &    8 &  5.15e-10 &  1.47e-17 &      \\ 
     &           &    9 &  5.14e-10 &  1.47e-17 &      \\ 
     &           &   10 &  5.13e-10 &  1.47e-17 &      \\ 
3125 &  3.12e+04 &   10 &           &           & iters  \\ 
 \hdashline 
     &           &    1 &  1.74e-08 &  2.24e-09 &      \\ 
     &           &    2 &  1.74e-08 &  4.98e-16 &      \\ 
     &           &    3 &  6.03e-08 &  4.97e-16 &      \\ 
     &           &    4 &  1.75e-08 &  1.72e-15 &      \\ 
     &           &    5 &  1.75e-08 &  4.99e-16 &      \\ 
     &           &    6 &  1.74e-08 &  4.98e-16 &      \\ 
     &           &    7 &  6.03e-08 &  4.97e-16 &      \\ 
     &           &    8 &  1.75e-08 &  1.72e-15 &      \\ 
     &           &    9 &  1.75e-08 &  4.99e-16 &      \\ 
     &           &   10 &  1.74e-08 &  4.98e-16 &      \\ 
3126 &  3.12e+04 &   10 &           &           & iters  \\ 
 \hdashline 
     &           &    1 &  2.79e-08 &  4.75e-09 &      \\ 
     &           &    2 &  4.99e-08 &  7.95e-16 &      \\ 
     &           &    3 &  2.79e-08 &  1.42e-15 &      \\ 
     &           &    4 &  2.78e-08 &  7.96e-16 &      \\ 
     &           &    5 &  4.99e-08 &  7.95e-16 &      \\ 
     &           &    6 &  2.79e-08 &  1.42e-15 &      \\ 
     &           &    7 &  4.99e-08 &  7.96e-16 &      \\ 
     &           &    8 &  2.79e-08 &  1.42e-15 &      \\ 
     &           &    9 &  2.79e-08 &  7.96e-16 &      \\ 
     &           &   10 &  4.99e-08 &  7.95e-16 &      \\ 
3127 &  3.13e+04 &   10 &           &           & iters  \\ 
 \hdashline 
     &           &    1 &  1.86e-08 &  1.15e-09 &      \\ 
     &           &    2 &  1.86e-08 &  5.31e-16 &      \\ 
     &           &    3 &  2.03e-08 &  5.30e-16 &      \\ 
     &           &    4 &  1.86e-08 &  5.79e-16 &      \\ 
     &           &    5 &  2.03e-08 &  5.30e-16 &      \\ 
     &           &    6 &  1.86e-08 &  5.79e-16 &      \\ 
     &           &    7 &  2.03e-08 &  5.30e-16 &      \\ 
     &           &    8 &  1.86e-08 &  5.79e-16 &      \\ 
     &           &    9 &  2.03e-08 &  5.30e-16 &      \\ 
     &           &   10 &  1.86e-08 &  5.79e-16 &      \\ 
3128 &  3.13e+04 &   10 &           &           & iters  \\ 
 \hdashline 
     &           &    1 &  7.28e-09 &  1.62e-09 &      \\ 
     &           &    2 &  7.27e-09 &  2.08e-16 &      \\ 
     &           &    3 &  7.26e-09 &  2.08e-16 &      \\ 
     &           &    4 &  7.25e-09 &  2.07e-16 &      \\ 
     &           &    5 &  7.24e-09 &  2.07e-16 &      \\ 
     &           &    6 &  7.05e-08 &  2.07e-16 &      \\ 
     &           &    7 &  8.50e-08 &  2.01e-15 &      \\ 
     &           &    8 &  7.05e-08 &  2.43e-15 &      \\ 
     &           &    9 &  7.31e-09 &  2.01e-15 &      \\ 
     &           &   10 &  7.30e-09 &  2.09e-16 &      \\ 
3129 &  3.13e+04 &   10 &           &           & iters  \\ 
 \hdashline 
     &           &    1 &  1.88e-08 &  1.09e-09 &      \\ 
     &           &    2 &  5.89e-08 &  5.36e-16 &      \\ 
     &           &    3 &  1.88e-08 &  1.68e-15 &      \\ 
     &           &    4 &  1.88e-08 &  5.38e-16 &      \\ 
     &           &    5 &  1.88e-08 &  5.37e-16 &      \\ 
     &           &    6 &  5.89e-08 &  5.36e-16 &      \\ 
     &           &    7 &  1.88e-08 &  1.68e-15 &      \\ 
     &           &    8 &  1.88e-08 &  5.38e-16 &      \\ 
     &           &    9 &  1.88e-08 &  5.37e-16 &      \\ 
     &           &   10 &  5.89e-08 &  5.36e-16 &      \\ 
3130 &  3.13e+04 &   10 &           &           & iters  \\ 
 \hdashline 
     &           &    1 &  4.85e-08 &  1.49e-09 &      \\ 
     &           &    2 &  2.93e-08 &  1.38e-15 &      \\ 
     &           &    3 &  4.84e-08 &  8.36e-16 &      \\ 
     &           &    4 &  2.93e-08 &  1.38e-15 &      \\ 
     &           &    5 &  4.84e-08 &  8.37e-16 &      \\ 
     &           &    6 &  2.93e-08 &  1.38e-15 &      \\ 
     &           &    7 &  2.93e-08 &  8.38e-16 &      \\ 
     &           &    8 &  4.84e-08 &  8.36e-16 &      \\ 
     &           &    9 &  2.93e-08 &  1.38e-15 &      \\ 
     &           &   10 &  2.93e-08 &  8.37e-16 &      \\ 
3131 &  3.13e+04 &   10 &           &           & iters  \\ 
 \hdashline 
     &           &    1 &  3.86e-08 &  2.36e-10 &      \\ 
     &           &    2 &  3.92e-08 &  1.10e-15 &      \\ 
     &           &    3 &  3.86e-08 &  1.12e-15 &      \\ 
     &           &    4 &  3.92e-08 &  1.10e-15 &      \\ 
     &           &    5 &  3.86e-08 &  1.12e-15 &      \\ 
     &           &    6 &  3.92e-08 &  1.10e-15 &      \\ 
     &           &    7 &  3.86e-08 &  1.12e-15 &      \\ 
     &           &    8 &  3.92e-08 &  1.10e-15 &      \\ 
     &           &    9 &  3.86e-08 &  1.12e-15 &      \\ 
     &           &   10 &  3.92e-08 &  1.10e-15 &      \\ 
3132 &  3.13e+04 &   10 &           &           & iters  \\ 
 \hdashline 
     &           &    1 &  2.01e-08 &  2.66e-09 &      \\ 
     &           &    2 &  2.01e-08 &  5.75e-16 &      \\ 
     &           &    3 &  5.76e-08 &  5.74e-16 &      \\ 
     &           &    4 &  2.02e-08 &  1.64e-15 &      \\ 
     &           &    5 &  2.01e-08 &  5.76e-16 &      \\ 
     &           &    6 &  2.01e-08 &  5.75e-16 &      \\ 
     &           &    7 &  5.76e-08 &  5.74e-16 &      \\ 
     &           &    8 &  2.02e-08 &  1.64e-15 &      \\ 
     &           &    9 &  2.01e-08 &  5.76e-16 &      \\ 
     &           &   10 &  2.01e-08 &  5.75e-16 &      \\ 
3133 &  3.13e+04 &   10 &           &           & iters  \\ 
 \hdashline 
     &           &    1 &  7.05e-08 &  3.08e-09 &      \\ 
     &           &    2 &  7.24e-09 &  2.01e-15 &      \\ 
     &           &    3 &  7.23e-09 &  2.07e-16 &      \\ 
     &           &    4 &  7.22e-09 &  2.06e-16 &      \\ 
     &           &    5 &  7.21e-09 &  2.06e-16 &      \\ 
     &           &    6 &  7.20e-09 &  2.06e-16 &      \\ 
     &           &    7 &  7.19e-09 &  2.05e-16 &      \\ 
     &           &    8 &  7.18e-09 &  2.05e-16 &      \\ 
     &           &    9 &  7.17e-09 &  2.05e-16 &      \\ 
     &           &   10 &  7.16e-09 &  2.05e-16 &      \\ 
3134 &  3.13e+04 &   10 &           &           & iters  \\ 
 \hdashline 
     &           &    1 &  1.13e-08 &  4.29e-10 &      \\ 
     &           &    2 &  1.13e-08 &  3.22e-16 &      \\ 
     &           &    3 &  1.13e-08 &  3.22e-16 &      \\ 
     &           &    4 &  2.76e-08 &  3.21e-16 &      \\ 
     &           &    5 &  1.13e-08 &  7.88e-16 &      \\ 
     &           &    6 &  1.13e-08 &  3.22e-16 &      \\ 
     &           &    7 &  1.12e-08 &  3.21e-16 &      \\ 
     &           &    8 &  2.76e-08 &  3.21e-16 &      \\ 
     &           &    9 &  1.13e-08 &  7.88e-16 &      \\ 
     &           &   10 &  1.13e-08 &  3.22e-16 &      \\ 
3135 &  3.13e+04 &   10 &           &           & iters  \\ 
 \hdashline 
     &           &    1 &  2.32e-08 &  1.30e-09 &      \\ 
     &           &    2 &  5.45e-08 &  6.62e-16 &      \\ 
     &           &    3 &  2.32e-08 &  1.56e-15 &      \\ 
     &           &    4 &  2.32e-08 &  6.63e-16 &      \\ 
     &           &    5 &  2.32e-08 &  6.62e-16 &      \\ 
     &           &    6 &  5.45e-08 &  6.61e-16 &      \\ 
     &           &    7 &  2.32e-08 &  1.56e-15 &      \\ 
     &           &    8 &  2.32e-08 &  6.63e-16 &      \\ 
     &           &    9 &  5.45e-08 &  6.62e-16 &      \\ 
     &           &   10 &  2.32e-08 &  1.56e-15 &      \\ 
3136 &  3.14e+04 &   10 &           &           & iters  \\ 
 \hdashline 
     &           &    1 &  1.50e-08 &  3.47e-10 &      \\ 
     &           &    2 &  1.50e-08 &  4.28e-16 &      \\ 
     &           &    3 &  6.27e-08 &  4.28e-16 &      \\ 
     &           &    4 &  1.51e-08 &  1.79e-15 &      \\ 
     &           &    5 &  1.50e-08 &  4.30e-16 &      \\ 
     &           &    6 &  1.50e-08 &  4.29e-16 &      \\ 
     &           &    7 &  1.50e-08 &  4.28e-16 &      \\ 
     &           &    8 &  6.27e-08 &  4.28e-16 &      \\ 
     &           &    9 &  1.51e-08 &  1.79e-15 &      \\ 
     &           &   10 &  1.50e-08 &  4.30e-16 &      \\ 
3137 &  3.14e+04 &   10 &           &           & iters  \\ 
 \hdashline 
     &           &    1 &  5.08e-10 &  6.65e-11 &      \\ 
     &           &    2 &  5.08e-10 &  1.45e-17 &      \\ 
     &           &    3 &  5.07e-10 &  1.45e-17 &      \\ 
     &           &    4 &  5.06e-10 &  1.45e-17 &      \\ 
     &           &    5 &  5.06e-10 &  1.44e-17 &      \\ 
     &           &    6 &  5.05e-10 &  1.44e-17 &      \\ 
     &           &    7 &  5.04e-10 &  1.44e-17 &      \\ 
     &           &    8 &  5.03e-10 &  1.44e-17 &      \\ 
     &           &    9 &  5.03e-10 &  1.44e-17 &      \\ 
     &           &   10 &  5.02e-10 &  1.43e-17 &      \\ 
3138 &  3.14e+04 &   10 &           &           & iters  \\ 
 \hdashline 
     &           &    1 &  3.33e-08 &  2.43e-09 &      \\ 
     &           &    2 &  4.44e-08 &  9.51e-16 &      \\ 
     &           &    3 &  3.33e-08 &  1.27e-15 &      \\ 
     &           &    4 &  4.44e-08 &  9.51e-16 &      \\ 
     &           &    5 &  3.33e-08 &  1.27e-15 &      \\ 
     &           &    6 &  4.44e-08 &  9.52e-16 &      \\ 
     &           &    7 &  3.34e-08 &  1.27e-15 &      \\ 
     &           &    8 &  3.33e-08 &  9.52e-16 &      \\ 
     &           &    9 &  4.44e-08 &  9.51e-16 &      \\ 
     &           &   10 &  3.33e-08 &  1.27e-15 &      \\ 
3139 &  3.14e+04 &   10 &           &           & iters  \\ 
 \hdashline 
     &           &    1 &  3.06e-09 &  2.05e-09 &      \\ 
     &           &    2 &  3.05e-09 &  8.73e-17 &      \\ 
     &           &    3 &  3.05e-09 &  8.72e-17 &      \\ 
     &           &    4 &  3.04e-09 &  8.70e-17 &      \\ 
     &           &    5 &  3.04e-09 &  8.69e-17 &      \\ 
     &           &    6 &  3.04e-09 &  8.68e-17 &      \\ 
     &           &    7 &  3.03e-09 &  8.67e-17 &      \\ 
     &           &    8 &  3.03e-09 &  8.65e-17 &      \\ 
     &           &    9 &  3.02e-09 &  8.64e-17 &      \\ 
     &           &   10 &  3.02e-09 &  8.63e-17 &      \\ 
3140 &  3.14e+04 &   10 &           &           & iters  \\ 
 \hdashline 
     &           &    1 &  2.83e-08 &  1.15e-09 &      \\ 
     &           &    2 &  4.94e-08 &  8.07e-16 &      \\ 
     &           &    3 &  2.83e-08 &  1.41e-15 &      \\ 
     &           &    4 &  4.94e-08 &  8.08e-16 &      \\ 
     &           &    5 &  2.83e-08 &  1.41e-15 &      \\ 
     &           &    6 &  2.83e-08 &  8.09e-16 &      \\ 
     &           &    7 &  4.94e-08 &  8.08e-16 &      \\ 
     &           &    8 &  2.83e-08 &  1.41e-15 &      \\ 
     &           &    9 &  2.83e-08 &  8.09e-16 &      \\ 
     &           &   10 &  4.94e-08 &  8.08e-16 &      \\ 
3141 &  3.14e+04 &   10 &           &           & iters  \\ 
 \hdashline 
     &           &    1 &  6.76e-09 &  2.78e-09 &      \\ 
     &           &    2 &  6.75e-09 &  1.93e-16 &      \\ 
     &           &    3 &  6.74e-09 &  1.93e-16 &      \\ 
     &           &    4 &  7.10e-08 &  1.92e-16 &      \\ 
     &           &    5 &  4.57e-08 &  2.03e-15 &      \\ 
     &           &    6 &  6.76e-09 &  1.30e-15 &      \\ 
     &           &    7 &  6.75e-09 &  1.93e-16 &      \\ 
     &           &    8 &  6.74e-09 &  1.93e-16 &      \\ 
     &           &    9 &  6.73e-09 &  1.92e-16 &      \\ 
     &           &   10 &  7.10e-08 &  1.92e-16 &      \\ 
3142 &  3.14e+04 &   10 &           &           & iters  \\ 
 \hdashline 
     &           &    1 &  2.05e-08 &  3.76e-10 &      \\ 
     &           &    2 &  2.04e-08 &  5.84e-16 &      \\ 
     &           &    3 &  1.84e-08 &  5.84e-16 &      \\ 
     &           &    4 &  2.04e-08 &  5.26e-16 &      \\ 
     &           &    5 &  1.84e-08 &  5.84e-16 &      \\ 
     &           &    6 &  2.04e-08 &  5.26e-16 &      \\ 
     &           &    7 &  1.84e-08 &  5.83e-16 &      \\ 
     &           &    8 &  2.04e-08 &  5.26e-16 &      \\ 
     &           &    9 &  1.84e-08 &  5.83e-16 &      \\ 
     &           &   10 &  2.04e-08 &  5.26e-16 &      \\ 
3143 &  3.14e+04 &   10 &           &           & iters  \\ 
 \hdashline 
     &           &    1 &  9.26e-09 &  3.65e-09 &      \\ 
     &           &    2 &  9.24e-09 &  2.64e-16 &      \\ 
     &           &    3 &  9.23e-09 &  2.64e-16 &      \\ 
     &           &    4 &  6.85e-08 &  2.63e-16 &      \\ 
     &           &    5 &  8.70e-08 &  1.95e-15 &      \\ 
     &           &    6 &  6.85e-08 &  2.48e-15 &      \\ 
     &           &    7 &  9.29e-09 &  1.96e-15 &      \\ 
     &           &    8 &  9.27e-09 &  2.65e-16 &      \\ 
     &           &    9 &  9.26e-09 &  2.65e-16 &      \\ 
     &           &   10 &  9.25e-09 &  2.64e-16 &      \\ 
3144 &  3.14e+04 &   10 &           &           & iters  \\ 
 \hdashline 
     &           &    1 &  1.14e-08 &  5.41e-09 &      \\ 
     &           &    2 &  1.14e-08 &  3.24e-16 &      \\ 
     &           &    3 &  1.13e-08 &  3.24e-16 &      \\ 
     &           &    4 &  1.13e-08 &  3.23e-16 &      \\ 
     &           &    5 &  1.13e-08 &  3.23e-16 &      \\ 
     &           &    6 &  6.64e-08 &  3.23e-16 &      \\ 
     &           &    7 &  1.14e-08 &  1.90e-15 &      \\ 
     &           &    8 &  1.14e-08 &  3.25e-16 &      \\ 
     &           &    9 &  1.13e-08 &  3.24e-16 &      \\ 
     &           &   10 &  1.13e-08 &  3.24e-16 &      \\ 
3145 &  3.14e+04 &   10 &           &           & iters  \\ 
 \hdashline 
     &           &    1 &  4.10e-08 &  9.04e-10 &      \\ 
     &           &    2 &  3.68e-08 &  1.17e-15 &      \\ 
     &           &    3 &  4.10e-08 &  1.05e-15 &      \\ 
     &           &    4 &  3.68e-08 &  1.17e-15 &      \\ 
     &           &    5 &  4.10e-08 &  1.05e-15 &      \\ 
     &           &    6 &  3.68e-08 &  1.17e-15 &      \\ 
     &           &    7 &  4.10e-08 &  1.05e-15 &      \\ 
     &           &    8 &  3.68e-08 &  1.17e-15 &      \\ 
     &           &    9 &  4.09e-08 &  1.05e-15 &      \\ 
     &           &   10 &  3.68e-08 &  1.17e-15 &      \\ 
3146 &  3.14e+04 &   10 &           &           & iters  \\ 
 \hdashline 
     &           &    1 &  2.41e-08 &  4.56e-09 &      \\ 
     &           &    2 &  5.37e-08 &  6.87e-16 &      \\ 
     &           &    3 &  2.41e-08 &  1.53e-15 &      \\ 
     &           &    4 &  2.41e-08 &  6.88e-16 &      \\ 
     &           &    5 &  2.40e-08 &  6.87e-16 &      \\ 
     &           &    6 &  5.37e-08 &  6.86e-16 &      \\ 
     &           &    7 &  2.41e-08 &  1.53e-15 &      \\ 
     &           &    8 &  2.40e-08 &  6.87e-16 &      \\ 
     &           &    9 &  5.37e-08 &  6.86e-16 &      \\ 
     &           &   10 &  2.41e-08 &  1.53e-15 &      \\ 
3147 &  3.15e+04 &   10 &           &           & iters  \\ 
 \hdashline 
     &           &    1 &  7.22e-09 &  4.25e-09 &      \\ 
     &           &    2 &  7.21e-09 &  2.06e-16 &      \\ 
     &           &    3 &  7.19e-09 &  2.06e-16 &      \\ 
     &           &    4 &  7.18e-09 &  2.05e-16 &      \\ 
     &           &    5 &  7.17e-09 &  2.05e-16 &      \\ 
     &           &    6 &  7.05e-08 &  2.05e-16 &      \\ 
     &           &    7 &  4.61e-08 &  2.01e-15 &      \\ 
     &           &    8 &  7.20e-09 &  1.32e-15 &      \\ 
     &           &    9 &  7.19e-09 &  2.05e-16 &      \\ 
     &           &   10 &  7.18e-09 &  2.05e-16 &      \\ 
3148 &  3.15e+04 &   10 &           &           & iters  \\ 
 \hdashline 
     &           &    1 &  1.88e-08 &  1.49e-09 &      \\ 
     &           &    2 &  1.87e-08 &  5.36e-16 &      \\ 
     &           &    3 &  1.87e-08 &  5.35e-16 &      \\ 
     &           &    4 &  2.02e-08 &  5.34e-16 &      \\ 
     &           &    5 &  1.87e-08 &  5.75e-16 &      \\ 
     &           &    6 &  2.02e-08 &  5.34e-16 &      \\ 
     &           &    7 &  1.87e-08 &  5.75e-16 &      \\ 
     &           &    8 &  2.02e-08 &  5.34e-16 &      \\ 
     &           &    9 &  1.87e-08 &  5.75e-16 &      \\ 
     &           &   10 &  2.02e-08 &  5.34e-16 &      \\ 
3149 &  3.15e+04 &   10 &           &           & iters  \\ 
 \hdashline 
     &           &    1 &  2.01e-09 &  7.11e-10 &      \\ 
     &           &    2 &  2.01e-09 &  5.74e-17 &      \\ 
     &           &    3 &  2.01e-09 &  5.73e-17 &      \\ 
     &           &    4 &  2.00e-09 &  5.73e-17 &      \\ 
     &           &    5 &  3.68e-08 &  5.72e-17 &      \\ 
     &           &    6 &  2.05e-09 &  1.05e-15 &      \\ 
     &           &    7 &  2.05e-09 &  5.86e-17 &      \\ 
     &           &    8 &  2.05e-09 &  5.85e-17 &      \\ 
     &           &    9 &  2.04e-09 &  5.84e-17 &      \\ 
     &           &   10 &  2.04e-09 &  5.83e-17 &      \\ 
3150 &  3.15e+04 &   10 &           &           & iters  \\ 
 \hdashline 
     &           &    1 &  2.58e-08 &  1.19e-09 &      \\ 
     &           &    2 &  2.57e-08 &  7.35e-16 &      \\ 
     &           &    3 &  5.20e-08 &  7.34e-16 &      \\ 
     &           &    4 &  2.58e-08 &  1.48e-15 &      \\ 
     &           &    5 &  2.57e-08 &  7.35e-16 &      \\ 
     &           &    6 &  5.20e-08 &  7.34e-16 &      \\ 
     &           &    7 &  2.58e-08 &  1.48e-15 &      \\ 
     &           &    8 &  2.57e-08 &  7.35e-16 &      \\ 
     &           &    9 &  5.20e-08 &  7.34e-16 &      \\ 
     &           &   10 &  2.58e-08 &  1.48e-15 &      \\ 
3151 &  3.15e+04 &   10 &           &           & iters  \\ 
 \hdashline 
     &           &    1 &  1.97e-08 &  9.44e-10 &      \\ 
     &           &    2 &  5.80e-08 &  5.62e-16 &      \\ 
     &           &    3 &  1.97e-08 &  1.66e-15 &      \\ 
     &           &    4 &  1.97e-08 &  5.64e-16 &      \\ 
     &           &    5 &  1.97e-08 &  5.63e-16 &      \\ 
     &           &    6 &  5.80e-08 &  5.62e-16 &      \\ 
     &           &    7 &  1.97e-08 &  1.66e-15 &      \\ 
     &           &    8 &  1.97e-08 &  5.63e-16 &      \\ 
     &           &    9 &  1.97e-08 &  5.63e-16 &      \\ 
     &           &   10 &  5.80e-08 &  5.62e-16 &      \\ 
3152 &  3.15e+04 &   10 &           &           & iters  \\ 
 \hdashline 
     &           &    1 &  4.82e-08 &  1.35e-09 &      \\ 
     &           &    2 &  2.96e-08 &  1.38e-15 &      \\ 
     &           &    3 &  4.82e-08 &  8.44e-16 &      \\ 
     &           &    4 &  2.96e-08 &  1.37e-15 &      \\ 
     &           &    5 &  2.95e-08 &  8.44e-16 &      \\ 
     &           &    6 &  4.82e-08 &  8.43e-16 &      \\ 
     &           &    7 &  2.96e-08 &  1.38e-15 &      \\ 
     &           &    8 &  4.82e-08 &  8.44e-16 &      \\ 
     &           &    9 &  2.96e-08 &  1.37e-15 &      \\ 
     &           &   10 &  2.96e-08 &  8.45e-16 &      \\ 
3153 &  3.15e+04 &   10 &           &           & iters  \\ 
 \hdashline 
     &           &    1 &  9.18e-09 &  1.67e-09 &      \\ 
     &           &    2 &  9.16e-09 &  2.62e-16 &      \\ 
     &           &    3 &  9.15e-09 &  2.62e-16 &      \\ 
     &           &    4 &  6.85e-08 &  2.61e-16 &      \\ 
     &           &    5 &  9.24e-09 &  1.96e-15 &      \\ 
     &           &    6 &  9.22e-09 &  2.64e-16 &      \\ 
     &           &    7 &  9.21e-09 &  2.63e-16 &      \\ 
     &           &    8 &  9.20e-09 &  2.63e-16 &      \\ 
     &           &    9 &  9.18e-09 &  2.62e-16 &      \\ 
     &           &   10 &  9.17e-09 &  2.62e-16 &      \\ 
3154 &  3.15e+04 &   10 &           &           & iters  \\ 
 \hdashline 
     &           &    1 &  5.64e-08 &  6.83e-09 &      \\ 
     &           &    2 &  2.14e-08 &  1.61e-15 &      \\ 
     &           &    3 &  2.13e-08 &  6.10e-16 &      \\ 
     &           &    4 &  5.64e-08 &  6.09e-16 &      \\ 
     &           &    5 &  2.14e-08 &  1.61e-15 &      \\ 
     &           &    6 &  2.13e-08 &  6.10e-16 &      \\ 
     &           &    7 &  2.13e-08 &  6.09e-16 &      \\ 
     &           &    8 &  5.64e-08 &  6.08e-16 &      \\ 
     &           &    9 &  2.14e-08 &  1.61e-15 &      \\ 
     &           &   10 &  2.13e-08 &  6.10e-16 &      \\ 
3155 &  3.15e+04 &   10 &           &           & iters  \\ 
 \hdashline 
     &           &    1 &  4.29e-08 &  3.63e-09 &      \\ 
     &           &    2 &  3.48e-08 &  1.22e-15 &      \\ 
     &           &    3 &  4.29e-08 &  9.94e-16 &      \\ 
     &           &    4 &  3.48e-08 &  1.22e-15 &      \\ 
     &           &    5 &  4.29e-08 &  9.95e-16 &      \\ 
     &           &    6 &  3.49e-08 &  1.22e-15 &      \\ 
     &           &    7 &  4.29e-08 &  9.95e-16 &      \\ 
     &           &    8 &  3.49e-08 &  1.22e-15 &      \\ 
     &           &    9 &  3.48e-08 &  9.95e-16 &      \\ 
     &           &   10 &  4.29e-08 &  9.94e-16 &      \\ 
3156 &  3.16e+04 &   10 &           &           & iters  \\ 
 \hdashline 
     &           &    1 &  7.68e-10 &  6.88e-09 &      \\ 
     &           &    2 &  7.67e-10 &  2.19e-17 &      \\ 
     &           &    3 &  7.66e-10 &  2.19e-17 &      \\ 
     &           &    4 &  7.65e-10 &  2.19e-17 &      \\ 
     &           &    5 &  7.63e-10 &  2.18e-17 &      \\ 
     &           &    6 &  7.62e-10 &  2.18e-17 &      \\ 
     &           &    7 &  7.61e-10 &  2.18e-17 &      \\ 
     &           &    8 &  7.60e-10 &  2.17e-17 &      \\ 
     &           &    9 &  7.59e-10 &  2.17e-17 &      \\ 
     &           &   10 &  7.58e-10 &  2.17e-17 &      \\ 
3157 &  3.16e+04 &   10 &           &           & iters  \\ 
 \hdashline 
     &           &    1 &  3.97e-08 &  1.00e-08 &      \\ 
     &           &    2 &  3.80e-08 &  1.13e-15 &      \\ 
     &           &    3 &  3.97e-08 &  1.09e-15 &      \\ 
     &           &    4 &  3.80e-08 &  1.13e-15 &      \\ 
     &           &    5 &  3.97e-08 &  1.09e-15 &      \\ 
     &           &    6 &  3.80e-08 &  1.13e-15 &      \\ 
     &           &    7 &  3.97e-08 &  1.09e-15 &      \\ 
     &           &    8 &  3.80e-08 &  1.13e-15 &      \\ 
     &           &    9 &  3.97e-08 &  1.09e-15 &      \\ 
     &           &   10 &  3.81e-08 &  1.13e-15 &      \\ 
3158 &  3.16e+04 &   10 &           &           & iters  \\ 
 \hdashline 
     &           &    1 &  2.68e-09 &  2.53e-09 &      \\ 
     &           &    2 &  2.67e-09 &  7.64e-17 &      \\ 
     &           &    3 &  2.67e-09 &  7.63e-17 &      \\ 
     &           &    4 &  2.67e-09 &  7.62e-17 &      \\ 
     &           &    5 &  2.66e-09 &  7.61e-17 &      \\ 
     &           &    6 &  2.66e-09 &  7.60e-17 &      \\ 
     &           &    7 &  2.65e-09 &  7.59e-17 &      \\ 
     &           &    8 &  2.65e-09 &  7.58e-17 &      \\ 
     &           &    9 &  2.65e-09 &  7.56e-17 &      \\ 
     &           &   10 &  2.64e-09 &  7.55e-17 &      \\ 
3159 &  3.16e+04 &   10 &           &           & iters  \\ 
 \hdashline 
     &           &    1 &  2.48e-08 &  6.45e-09 &      \\ 
     &           &    2 &  5.30e-08 &  7.07e-16 &      \\ 
     &           &    3 &  2.48e-08 &  1.51e-15 &      \\ 
     &           &    4 &  2.48e-08 &  7.08e-16 &      \\ 
     &           &    5 &  5.30e-08 &  7.07e-16 &      \\ 
     &           &    6 &  2.48e-08 &  1.51e-15 &      \\ 
     &           &    7 &  2.48e-08 &  7.08e-16 &      \\ 
     &           &    8 &  2.47e-08 &  7.07e-16 &      \\ 
     &           &    9 &  5.30e-08 &  7.06e-16 &      \\ 
     &           &   10 &  2.48e-08 &  1.51e-15 &      \\ 
3160 &  3.16e+04 &   10 &           &           & iters  \\ 
 \hdashline 
     &           &    1 &  2.11e-08 &  8.03e-09 &      \\ 
     &           &    2 &  2.11e-08 &  6.03e-16 &      \\ 
     &           &    3 &  5.66e-08 &  6.02e-16 &      \\ 
     &           &    4 &  2.11e-08 &  1.62e-15 &      \\ 
     &           &    5 &  2.11e-08 &  6.04e-16 &      \\ 
     &           &    6 &  5.66e-08 &  6.03e-16 &      \\ 
     &           &    7 &  2.12e-08 &  1.62e-15 &      \\ 
     &           &    8 &  2.11e-08 &  6.04e-16 &      \\ 
     &           &    9 &  2.11e-08 &  6.03e-16 &      \\ 
     &           &   10 &  5.66e-08 &  6.02e-16 &      \\ 
3161 &  3.16e+04 &   10 &           &           & iters  \\ 
 \hdashline 
     &           &    1 &  7.38e-08 &  9.05e-10 &      \\ 
     &           &    2 &  8.17e-08 &  2.11e-15 &      \\ 
     &           &    3 &  7.38e-08 &  2.33e-15 &      \\ 
     &           &    4 &  4.02e-09 &  2.11e-15 &      \\ 
     &           &    5 &  4.02e-09 &  1.15e-16 &      \\ 
     &           &    6 &  4.01e-09 &  1.15e-16 &      \\ 
     &           &    7 &  4.00e-09 &  1.14e-16 &      \\ 
     &           &    8 &  4.00e-09 &  1.14e-16 &      \\ 
     &           &    9 &  3.99e-09 &  1.14e-16 &      \\ 
     &           &   10 &  3.99e-09 &  1.14e-16 &      \\ 
3162 &  3.16e+04 &   10 &           &           & iters  \\ 
 \hdashline 
     &           &    1 &  1.88e-08 &  8.45e-09 &      \\ 
     &           &    2 &  1.87e-08 &  5.35e-16 &      \\ 
     &           &    3 &  1.87e-08 &  5.35e-16 &      \\ 
     &           &    4 &  5.90e-08 &  5.34e-16 &      \\ 
     &           &    5 &  1.88e-08 &  1.68e-15 &      \\ 
     &           &    6 &  1.87e-08 &  5.36e-16 &      \\ 
     &           &    7 &  1.87e-08 &  5.35e-16 &      \\ 
     &           &    8 &  5.90e-08 &  5.34e-16 &      \\ 
     &           &    9 &  1.88e-08 &  1.68e-15 &      \\ 
     &           &   10 &  1.87e-08 &  5.36e-16 &      \\ 
3163 &  3.16e+04 &   10 &           &           & iters  \\ 
 \hdashline 
     &           &    1 &  1.00e-08 &  6.43e-09 &      \\ 
     &           &    2 &  6.77e-08 &  2.87e-16 &      \\ 
     &           &    3 &  8.78e-08 &  1.93e-15 &      \\ 
     &           &    4 &  6.77e-08 &  2.51e-15 &      \\ 
     &           &    5 &  1.01e-08 &  1.93e-15 &      \\ 
     &           &    6 &  1.01e-08 &  2.88e-16 &      \\ 
     &           &    7 &  1.01e-08 &  2.88e-16 &      \\ 
     &           &    8 &  1.01e-08 &  2.87e-16 &      \\ 
     &           &    9 &  1.00e-08 &  2.87e-16 &      \\ 
     &           &   10 &  6.77e-08 &  2.87e-16 &      \\ 
3164 &  3.16e+04 &   10 &           &           & iters  \\ 
 \hdashline 
     &           &    1 &  1.42e-08 &  5.49e-10 &      \\ 
     &           &    2 &  1.41e-08 &  4.04e-16 &      \\ 
     &           &    3 &  6.36e-08 &  4.04e-16 &      \\ 
     &           &    4 &  1.42e-08 &  1.81e-15 &      \\ 
     &           &    5 &  1.42e-08 &  4.06e-16 &      \\ 
     &           &    6 &  1.42e-08 &  4.05e-16 &      \\ 
     &           &    7 &  1.42e-08 &  4.05e-16 &      \\ 
     &           &    8 &  1.41e-08 &  4.04e-16 &      \\ 
     &           &    9 &  6.36e-08 &  4.03e-16 &      \\ 
     &           &   10 &  1.42e-08 &  1.81e-15 &      \\ 
3165 &  3.16e+04 &   10 &           &           & iters  \\ 
 \hdashline 
     &           &    1 &  2.76e-08 &  8.99e-09 &      \\ 
     &           &    2 &  2.75e-08 &  7.87e-16 &      \\ 
     &           &    3 &  5.02e-08 &  7.86e-16 &      \\ 
     &           &    4 &  2.76e-08 &  1.43e-15 &      \\ 
     &           &    5 &  2.75e-08 &  7.87e-16 &      \\ 
     &           &    6 &  5.02e-08 &  7.86e-16 &      \\ 
     &           &    7 &  2.76e-08 &  1.43e-15 &      \\ 
     &           &    8 &  2.75e-08 &  7.87e-16 &      \\ 
     &           &    9 &  5.02e-08 &  7.86e-16 &      \\ 
     &           &   10 &  2.76e-08 &  1.43e-15 &      \\ 
3166 &  3.16e+04 &   10 &           &           & iters  \\ 
 \hdashline 
     &           &    1 &  7.09e-09 &  1.14e-08 &      \\ 
     &           &    2 &  7.06e-08 &  2.02e-16 &      \\ 
     &           &    3 &  4.60e-08 &  2.02e-15 &      \\ 
     &           &    4 &  7.12e-09 &  1.31e-15 &      \\ 
     &           &    5 &  7.11e-09 &  2.03e-16 &      \\ 
     &           &    6 &  7.10e-09 &  2.03e-16 &      \\ 
     &           &    7 &  7.09e-09 &  2.03e-16 &      \\ 
     &           &    8 &  7.06e-08 &  2.02e-16 &      \\ 
     &           &    9 &  4.60e-08 &  2.02e-15 &      \\ 
     &           &   10 &  7.11e-09 &  1.31e-15 &      \\ 
3167 &  3.17e+04 &   10 &           &           & iters  \\ 
 \hdashline 
     &           &    1 &  1.02e-08 &  2.27e-09 &      \\ 
     &           &    2 &  1.02e-08 &  2.91e-16 &      \\ 
     &           &    3 &  1.02e-08 &  2.91e-16 &      \\ 
     &           &    4 &  1.02e-08 &  2.91e-16 &      \\ 
     &           &    5 &  2.87e-08 &  2.90e-16 &      \\ 
     &           &    6 &  1.02e-08 &  8.19e-16 &      \\ 
     &           &    7 &  1.02e-08 &  2.91e-16 &      \\ 
     &           &    8 &  1.02e-08 &  2.91e-16 &      \\ 
     &           &    9 &  2.87e-08 &  2.90e-16 &      \\ 
     &           &   10 &  1.02e-08 &  8.19e-16 &      \\ 
3168 &  3.17e+04 &   10 &           &           & iters  \\ 
 \hdashline 
     &           &    1 &  3.58e-08 &  1.07e-08 &      \\ 
     &           &    2 &  4.20e-08 &  1.02e-15 &      \\ 
     &           &    3 &  3.58e-08 &  1.20e-15 &      \\ 
     &           &    4 &  4.20e-08 &  1.02e-15 &      \\ 
     &           &    5 &  3.58e-08 &  1.20e-15 &      \\ 
     &           &    6 &  4.19e-08 &  1.02e-15 &      \\ 
     &           &    7 &  3.58e-08 &  1.20e-15 &      \\ 
     &           &    8 &  4.19e-08 &  1.02e-15 &      \\ 
     &           &    9 &  3.58e-08 &  1.20e-15 &      \\ 
     &           &   10 &  4.19e-08 &  1.02e-15 &      \\ 
3169 &  3.17e+04 &   10 &           &           & iters  \\ 
 \hdashline 
     &           &    1 &  2.36e-08 &  1.15e-08 &      \\ 
     &           &    2 &  2.35e-08 &  6.73e-16 &      \\ 
     &           &    3 &  2.35e-08 &  6.72e-16 &      \\ 
     &           &    4 &  5.42e-08 &  6.71e-16 &      \\ 
     &           &    5 &  2.35e-08 &  1.55e-15 &      \\ 
     &           &    6 &  2.35e-08 &  6.72e-16 &      \\ 
     &           &    7 &  2.35e-08 &  6.71e-16 &      \\ 
     &           &    8 &  5.42e-08 &  6.70e-16 &      \\ 
     &           &    9 &  2.35e-08 &  1.55e-15 &      \\ 
     &           &   10 &  2.35e-08 &  6.71e-16 &      \\ 
3170 &  3.17e+04 &   10 &           &           & iters  \\ 
 \hdashline 
     &           &    1 &  6.82e-08 &  1.02e-09 &      \\ 
     &           &    2 &  9.63e-09 &  1.95e-15 &      \\ 
     &           &    3 &  9.62e-09 &  2.75e-16 &      \\ 
     &           &    4 &  9.60e-09 &  2.74e-16 &      \\ 
     &           &    5 &  6.81e-08 &  2.74e-16 &      \\ 
     &           &    6 &  9.69e-09 &  1.94e-15 &      \\ 
     &           &    7 &  9.67e-09 &  2.76e-16 &      \\ 
     &           &    8 &  9.66e-09 &  2.76e-16 &      \\ 
     &           &    9 &  9.65e-09 &  2.76e-16 &      \\ 
     &           &   10 &  9.63e-09 &  2.75e-16 &      \\ 
3171 &  3.17e+04 &   10 &           &           & iters  \\ 
 \hdashline 
     &           &    1 &  3.85e-08 &  7.64e-09 &      \\ 
     &           &    2 &  3.92e-08 &  1.10e-15 &      \\ 
     &           &    3 &  3.85e-08 &  1.12e-15 &      \\ 
     &           &    4 &  3.92e-08 &  1.10e-15 &      \\ 
     &           &    5 &  3.85e-08 &  1.12e-15 &      \\ 
     &           &    6 &  3.92e-08 &  1.10e-15 &      \\ 
     &           &    7 &  3.85e-08 &  1.12e-15 &      \\ 
     &           &    8 &  3.92e-08 &  1.10e-15 &      \\ 
     &           &    9 &  3.85e-08 &  1.12e-15 &      \\ 
     &           &   10 &  3.92e-08 &  1.10e-15 &      \\ 
3172 &  3.17e+04 &   10 &           &           & iters  \\ 
 \hdashline 
     &           &    1 &  3.23e-09 &  1.26e-08 &      \\ 
     &           &    2 &  3.56e-08 &  9.23e-17 &      \\ 
     &           &    3 &  3.28e-09 &  1.02e-15 &      \\ 
     &           &    4 &  3.28e-09 &  9.36e-17 &      \\ 
     &           &    5 &  3.27e-09 &  9.35e-17 &      \\ 
     &           &    6 &  3.27e-09 &  9.34e-17 &      \\ 
     &           &    7 &  3.26e-09 &  9.32e-17 &      \\ 
     &           &    8 &  3.26e-09 &  9.31e-17 &      \\ 
     &           &    9 &  3.25e-09 &  9.30e-17 &      \\ 
     &           &   10 &  3.25e-09 &  9.28e-17 &      \\ 
3173 &  3.17e+04 &   10 &           &           & iters  \\ 
 \hdashline 
     &           &    1 &  1.40e-08 &  6.41e-09 &      \\ 
     &           &    2 &  1.40e-08 &  4.01e-16 &      \\ 
     &           &    3 &  6.37e-08 &  4.00e-16 &      \\ 
     &           &    4 &  1.41e-08 &  1.82e-15 &      \\ 
     &           &    5 &  1.41e-08 &  4.02e-16 &      \\ 
     &           &    6 &  1.41e-08 &  4.02e-16 &      \\ 
     &           &    7 &  1.40e-08 &  4.01e-16 &      \\ 
     &           &    8 &  1.40e-08 &  4.01e-16 &      \\ 
     &           &    9 &  6.37e-08 &  4.00e-16 &      \\ 
     &           &   10 &  1.41e-08 &  1.82e-15 &      \\ 
3174 &  3.17e+04 &   10 &           &           & iters  \\ 
 \hdashline 
     &           &    1 &  1.28e-09 &  9.33e-09 &      \\ 
     &           &    2 &  1.28e-09 &  3.67e-17 &      \\ 
     &           &    3 &  1.28e-09 &  3.66e-17 &      \\ 
     &           &    4 &  1.28e-09 &  3.66e-17 &      \\ 
     &           &    5 &  1.28e-09 &  3.65e-17 &      \\ 
     &           &    6 &  1.28e-09 &  3.65e-17 &      \\ 
     &           &    7 &  1.27e-09 &  3.64e-17 &      \\ 
     &           &    8 &  1.27e-09 &  3.63e-17 &      \\ 
     &           &    9 &  1.27e-09 &  3.63e-17 &      \\ 
     &           &   10 &  1.27e-09 &  3.62e-17 &      \\ 
3175 &  3.17e+04 &   10 &           &           & iters  \\ 
 \hdashline 
     &           &    1 &  5.15e-09 &  1.24e-08 &      \\ 
     &           &    2 &  7.25e-08 &  1.47e-16 &      \\ 
     &           &    3 &  8.29e-08 &  2.07e-15 &      \\ 
     &           &    4 &  7.26e-08 &  2.37e-15 &      \\ 
     &           &    5 &  5.24e-09 &  2.07e-15 &      \\ 
     &           &    6 &  5.23e-09 &  1.49e-16 &      \\ 
     &           &    7 &  5.22e-09 &  1.49e-16 &      \\ 
     &           &    8 &  5.21e-09 &  1.49e-16 &      \\ 
     &           &    9 &  5.21e-09 &  1.49e-16 &      \\ 
     &           &   10 &  5.20e-09 &  1.49e-16 &      \\ 
3176 &  3.18e+04 &   10 &           &           & iters  \\ 
 \hdashline 
     &           &    1 &  1.25e-08 &  4.87e-09 &      \\ 
     &           &    2 &  1.24e-08 &  3.55e-16 &      \\ 
     &           &    3 &  6.53e-08 &  3.55e-16 &      \\ 
     &           &    4 &  1.25e-08 &  1.86e-15 &      \\ 
     &           &    5 &  1.25e-08 &  3.57e-16 &      \\ 
     &           &    6 &  1.25e-08 &  3.57e-16 &      \\ 
     &           &    7 &  1.25e-08 &  3.56e-16 &      \\ 
     &           &    8 &  1.24e-08 &  3.56e-16 &      \\ 
     &           &    9 &  6.53e-08 &  3.55e-16 &      \\ 
     &           &   10 &  1.25e-08 &  1.86e-15 &      \\ 
3177 &  3.18e+04 &   10 &           &           & iters  \\ 
 \hdashline 
     &           &    1 &  3.28e-08 &  1.29e-09 &      \\ 
     &           &    2 &  4.50e-08 &  9.36e-16 &      \\ 
     &           &    3 &  3.28e-08 &  1.28e-15 &      \\ 
     &           &    4 &  4.49e-08 &  9.36e-16 &      \\ 
     &           &    5 &  3.28e-08 &  1.28e-15 &      \\ 
     &           &    6 &  3.28e-08 &  9.37e-16 &      \\ 
     &           &    7 &  4.50e-08 &  9.35e-16 &      \\ 
     &           &    8 &  3.28e-08 &  1.28e-15 &      \\ 
     &           &    9 &  4.49e-08 &  9.36e-16 &      \\ 
     &           &   10 &  3.28e-08 &  1.28e-15 &      \\ 
3178 &  3.18e+04 &   10 &           &           & iters  \\ 
 \hdashline 
     &           &    1 &  2.20e-08 &  7.94e-09 &      \\ 
     &           &    2 &  2.20e-08 &  6.29e-16 &      \\ 
     &           &    3 &  5.57e-08 &  6.28e-16 &      \\ 
     &           &    4 &  2.21e-08 &  1.59e-15 &      \\ 
     &           &    5 &  2.20e-08 &  6.30e-16 &      \\ 
     &           &    6 &  2.20e-08 &  6.29e-16 &      \\ 
     &           &    7 &  5.57e-08 &  6.28e-16 &      \\ 
     &           &    8 &  2.20e-08 &  1.59e-15 &      \\ 
     &           &    9 &  2.20e-08 &  6.29e-16 &      \\ 
     &           &   10 &  5.57e-08 &  6.28e-16 &      \\ 
3179 &  3.18e+04 &   10 &           &           & iters  \\ 
 \hdashline 
     &           &    1 &  2.97e-09 &  1.69e-08 &      \\ 
     &           &    2 &  2.96e-09 &  8.47e-17 &      \\ 
     &           &    3 &  3.59e-08 &  8.46e-17 &      \\ 
     &           &    4 &  3.01e-09 &  1.02e-15 &      \\ 
     &           &    5 &  3.01e-09 &  8.59e-17 &      \\ 
     &           &    6 &  3.00e-09 &  8.58e-17 &      \\ 
     &           &    7 &  3.00e-09 &  8.57e-17 &      \\ 
     &           &    8 &  2.99e-09 &  8.55e-17 &      \\ 
     &           &    9 &  2.99e-09 &  8.54e-17 &      \\ 
     &           &   10 &  2.98e-09 &  8.53e-17 &      \\ 
3180 &  3.18e+04 &   10 &           &           & iters  \\ 
 \hdashline 
     &           &    1 &  1.10e-08 &  4.95e-09 &      \\ 
     &           &    2 &  1.10e-08 &  3.14e-16 &      \\ 
     &           &    3 &  1.10e-08 &  3.13e-16 &      \\ 
     &           &    4 &  6.67e-08 &  3.13e-16 &      \\ 
     &           &    5 &  1.10e-08 &  1.90e-15 &      \\ 
     &           &    6 &  1.10e-08 &  3.15e-16 &      \\ 
     &           &    7 &  1.10e-08 &  3.15e-16 &      \\ 
     &           &    8 &  1.10e-08 &  3.14e-16 &      \\ 
     &           &    9 &  1.10e-08 &  3.14e-16 &      \\ 
     &           &   10 &  1.10e-08 &  3.13e-16 &      \\ 
3181 &  3.18e+04 &   10 &           &           & iters  \\ 
 \hdashline 
     &           &    1 &  5.61e-08 &  1.06e-08 &      \\ 
     &           &    2 &  2.17e-08 &  1.60e-15 &      \\ 
     &           &    3 &  2.17e-08 &  6.19e-16 &      \\ 
     &           &    4 &  5.61e-08 &  6.18e-16 &      \\ 
     &           &    5 &  2.17e-08 &  1.60e-15 &      \\ 
     &           &    6 &  2.17e-08 &  6.20e-16 &      \\ 
     &           &    7 &  2.17e-08 &  6.19e-16 &      \\ 
     &           &    8 &  5.61e-08 &  6.18e-16 &      \\ 
     &           &    9 &  2.17e-08 &  1.60e-15 &      \\ 
     &           &   10 &  2.17e-08 &  6.19e-16 &      \\ 
3182 &  3.18e+04 &   10 &           &           & iters  \\ 
 \hdashline 
     &           &    1 &  3.72e-08 &  1.09e-08 &      \\ 
     &           &    2 &  4.05e-08 &  1.06e-15 &      \\ 
     &           &    3 &  3.72e-08 &  1.16e-15 &      \\ 
     &           &    4 &  4.05e-08 &  1.06e-15 &      \\ 
     &           &    5 &  3.72e-08 &  1.16e-15 &      \\ 
     &           &    6 &  4.05e-08 &  1.06e-15 &      \\ 
     &           &    7 &  3.72e-08 &  1.16e-15 &      \\ 
     &           &    8 &  4.05e-08 &  1.06e-15 &      \\ 
     &           &    9 &  3.73e-08 &  1.16e-15 &      \\ 
     &           &   10 &  4.05e-08 &  1.06e-15 &      \\ 
3183 &  3.18e+04 &   10 &           &           & iters  \\ 
 \hdashline 
     &           &    1 &  1.30e-08 &  8.12e-09 &      \\ 
     &           &    2 &  1.30e-08 &  3.72e-16 &      \\ 
     &           &    3 &  6.47e-08 &  3.71e-16 &      \\ 
     &           &    4 &  1.31e-08 &  1.85e-15 &      \\ 
     &           &    5 &  1.31e-08 &  3.73e-16 &      \\ 
     &           &    6 &  1.30e-08 &  3.73e-16 &      \\ 
     &           &    7 &  1.30e-08 &  3.72e-16 &      \\ 
     &           &    8 &  1.30e-08 &  3.72e-16 &      \\ 
     &           &    9 &  6.47e-08 &  3.71e-16 &      \\ 
     &           &   10 &  1.31e-08 &  1.85e-15 &      \\ 
3184 &  3.18e+04 &   10 &           &           & iters  \\ 
 \hdashline 
     &           &    1 &  2.60e-08 &  4.16e-09 &      \\ 
     &           &    2 &  5.17e-08 &  7.42e-16 &      \\ 
     &           &    3 &  2.60e-08 &  1.48e-15 &      \\ 
     &           &    4 &  2.60e-08 &  7.43e-16 &      \\ 
     &           &    5 &  5.17e-08 &  7.42e-16 &      \\ 
     &           &    6 &  2.60e-08 &  1.48e-15 &      \\ 
     &           &    7 &  2.60e-08 &  7.43e-16 &      \\ 
     &           &    8 &  5.17e-08 &  7.42e-16 &      \\ 
     &           &    9 &  2.60e-08 &  1.48e-15 &      \\ 
     &           &   10 &  2.60e-08 &  7.43e-16 &      \\ 
3185 &  3.18e+04 &   10 &           &           & iters  \\ 
 \hdashline 
     &           &    1 &  6.67e-08 &  9.65e-09 &      \\ 
     &           &    2 &  1.11e-08 &  1.90e-15 &      \\ 
     &           &    3 &  1.11e-08 &  3.16e-16 &      \\ 
     &           &    4 &  6.66e-08 &  3.16e-16 &      \\ 
     &           &    5 &  8.88e-08 &  1.90e-15 &      \\ 
     &           &    6 &  6.67e-08 &  2.54e-15 &      \\ 
     &           &    7 &  1.11e-08 &  1.90e-15 &      \\ 
     &           &    8 &  1.11e-08 &  3.17e-16 &      \\ 
     &           &    9 &  1.11e-08 &  3.17e-16 &      \\ 
     &           &   10 &  1.11e-08 &  3.16e-16 &      \\ 
3186 &  3.18e+04 &   10 &           &           & iters  \\ 
 \hdashline 
     &           &    1 &  8.32e-09 &  1.72e-08 &      \\ 
     &           &    2 &  8.31e-09 &  2.38e-16 &      \\ 
     &           &    3 &  3.05e-08 &  2.37e-16 &      \\ 
     &           &    4 &  8.34e-09 &  8.72e-16 &      \\ 
     &           &    5 &  8.33e-09 &  2.38e-16 &      \\ 
     &           &    6 &  8.32e-09 &  2.38e-16 &      \\ 
     &           &    7 &  8.31e-09 &  2.37e-16 &      \\ 
     &           &    8 &  3.06e-08 &  2.37e-16 &      \\ 
     &           &    9 &  8.34e-09 &  8.72e-16 &      \\ 
     &           &   10 &  8.33e-09 &  2.38e-16 &      \\ 
3187 &  3.19e+04 &   10 &           &           & iters  \\ 
 \hdashline 
     &           &    1 &  2.52e-08 &  1.25e-09 &      \\ 
     &           &    2 &  5.25e-08 &  7.20e-16 &      \\ 
     &           &    3 &  2.53e-08 &  1.50e-15 &      \\ 
     &           &    4 &  2.52e-08 &  7.21e-16 &      \\ 
     &           &    5 &  2.52e-08 &  7.20e-16 &      \\ 
     &           &    6 &  5.25e-08 &  7.19e-16 &      \\ 
     &           &    7 &  2.52e-08 &  1.50e-15 &      \\ 
     &           &    8 &  2.52e-08 &  7.20e-16 &      \\ 
     &           &    9 &  5.25e-08 &  7.19e-16 &      \\ 
     &           &   10 &  2.52e-08 &  1.50e-15 &      \\ 
3188 &  3.19e+04 &   10 &           &           & iters  \\ 
 \hdashline 
     &           &    1 &  1.41e-08 &  1.34e-08 &      \\ 
     &           &    2 &  6.36e-08 &  4.03e-16 &      \\ 
     &           &    3 &  1.42e-08 &  1.81e-15 &      \\ 
     &           &    4 &  1.42e-08 &  4.05e-16 &      \\ 
     &           &    5 &  1.42e-08 &  4.05e-16 &      \\ 
     &           &    6 &  1.41e-08 &  4.04e-16 &      \\ 
     &           &    7 &  6.36e-08 &  4.04e-16 &      \\ 
     &           &    8 &  1.42e-08 &  1.81e-15 &      \\ 
     &           &    9 &  1.42e-08 &  4.06e-16 &      \\ 
     &           &   10 &  1.42e-08 &  4.05e-16 &      \\ 
3189 &  3.19e+04 &   10 &           &           & iters  \\ 
 \hdashline 
     &           &    1 &  6.07e-08 &  8.71e-09 &      \\ 
     &           &    2 &  1.71e-08 &  1.73e-15 &      \\ 
     &           &    3 &  1.70e-08 &  4.87e-16 &      \\ 
     &           &    4 &  1.70e-08 &  4.86e-16 &      \\ 
     &           &    5 &  1.70e-08 &  4.86e-16 &      \\ 
     &           &    6 &  6.07e-08 &  4.85e-16 &      \\ 
     &           &    7 &  1.71e-08 &  1.73e-15 &      \\ 
     &           &    8 &  1.70e-08 &  4.87e-16 &      \\ 
     &           &    9 &  1.70e-08 &  4.86e-16 &      \\ 
     &           &   10 &  6.07e-08 &  4.85e-16 &      \\ 
3190 &  3.19e+04 &   10 &           &           & iters  \\ 
 \hdashline 
     &           &    1 &  4.21e-08 &  3.04e-09 &      \\ 
     &           &    2 &  4.20e-08 &  1.20e-15 &      \\ 
     &           &    3 &  3.57e-08 &  1.20e-15 &      \\ 
     &           &    4 &  4.20e-08 &  1.02e-15 &      \\ 
     &           &    5 &  3.57e-08 &  1.20e-15 &      \\ 
     &           &    6 &  3.57e-08 &  1.02e-15 &      \\ 
     &           &    7 &  4.21e-08 &  1.02e-15 &      \\ 
     &           &    8 &  3.57e-08 &  1.20e-15 &      \\ 
     &           &    9 &  4.20e-08 &  1.02e-15 &      \\ 
     &           &   10 &  3.57e-08 &  1.20e-15 &      \\ 
3191 &  3.19e+04 &   10 &           &           & iters  \\ 
 \hdashline 
     &           &    1 &  1.14e-08 &  1.75e-09 &      \\ 
     &           &    2 &  1.13e-08 &  3.24e-16 &      \\ 
     &           &    3 &  1.13e-08 &  3.23e-16 &      \\ 
     &           &    4 &  1.13e-08 &  3.23e-16 &      \\ 
     &           &    5 &  1.13e-08 &  3.23e-16 &      \\ 
     &           &    6 &  6.64e-08 &  3.22e-16 &      \\ 
     &           &    7 &  1.14e-08 &  1.90e-15 &      \\ 
     &           &    8 &  1.13e-08 &  3.24e-16 &      \\ 
     &           &    9 &  1.13e-08 &  3.24e-16 &      \\ 
     &           &   10 &  1.13e-08 &  3.23e-16 &      \\ 
3192 &  3.19e+04 &   10 &           &           & iters  \\ 
 \hdashline 
     &           &    1 &  4.30e-09 &  1.15e-08 &      \\ 
     &           &    2 &  4.30e-09 &  1.23e-16 &      \\ 
     &           &    3 &  4.29e-09 &  1.23e-16 &      \\ 
     &           &    4 &  4.28e-09 &  1.22e-16 &      \\ 
     &           &    5 &  4.28e-09 &  1.22e-16 &      \\ 
     &           &    6 &  4.27e-09 &  1.22e-16 &      \\ 
     &           &    7 &  4.26e-09 &  1.22e-16 &      \\ 
     &           &    8 &  4.26e-09 &  1.22e-16 &      \\ 
     &           &    9 &  7.34e-08 &  1.22e-16 &      \\ 
     &           &   10 &  8.20e-08 &  2.10e-15 &      \\ 
3193 &  3.19e+04 &   10 &           &           & iters  \\ 
 \hdashline 
     &           &    1 &  1.42e-08 &  1.38e-08 &      \\ 
     &           &    2 &  1.42e-08 &  4.06e-16 &      \\ 
     &           &    3 &  1.42e-08 &  4.06e-16 &      \\ 
     &           &    4 &  2.47e-08 &  4.05e-16 &      \\ 
     &           &    5 &  1.42e-08 &  7.04e-16 &      \\ 
     &           &    6 &  1.42e-08 &  4.06e-16 &      \\ 
     &           &    7 &  2.47e-08 &  4.05e-16 &      \\ 
     &           &    8 &  1.42e-08 &  7.04e-16 &      \\ 
     &           &    9 &  1.42e-08 &  4.05e-16 &      \\ 
     &           &   10 &  2.47e-08 &  4.05e-16 &      \\ 
3194 &  3.19e+04 &   10 &           &           & iters  \\ 
 \hdashline 
     &           &    1 &  1.30e-08 &  1.78e-09 &      \\ 
     &           &    2 &  2.58e-08 &  3.71e-16 &      \\ 
     &           &    3 &  1.30e-08 &  7.38e-16 &      \\ 
     &           &    4 &  1.30e-08 &  3.72e-16 &      \\ 
     &           &    5 &  2.59e-08 &  3.71e-16 &      \\ 
     &           &    6 &  1.30e-08 &  7.38e-16 &      \\ 
     &           &    7 &  1.30e-08 &  3.72e-16 &      \\ 
     &           &    8 &  2.59e-08 &  3.71e-16 &      \\ 
     &           &    9 &  1.30e-08 &  7.38e-16 &      \\ 
     &           &   10 &  2.58e-08 &  3.72e-16 &      \\ 
3195 &  3.19e+04 &   10 &           &           & iters  \\ 
 \hdashline 
     &           &    1 &  1.06e-08 &  1.11e-08 &      \\ 
     &           &    2 &  1.06e-08 &  3.04e-16 &      \\ 
     &           &    3 &  6.71e-08 &  3.03e-16 &      \\ 
     &           &    4 &  4.96e-08 &  1.91e-15 &      \\ 
     &           &    5 &  1.06e-08 &  1.41e-15 &      \\ 
     &           &    6 &  1.06e-08 &  3.04e-16 &      \\ 
     &           &    7 &  6.71e-08 &  3.03e-16 &      \\ 
     &           &    8 &  4.95e-08 &  1.91e-15 &      \\ 
     &           &    9 &  1.06e-08 &  1.41e-15 &      \\ 
     &           &   10 &  1.06e-08 &  3.03e-16 &      \\ 
3196 &  3.20e+04 &   10 &           &           & iters  \\ 
 \hdashline 
     &           &    1 &  5.53e-08 &  7.52e-09 &      \\ 
     &           &    2 &  2.25e-08 &  1.58e-15 &      \\ 
     &           &    3 &  2.25e-08 &  6.42e-16 &      \\ 
     &           &    4 &  5.52e-08 &  6.41e-16 &      \\ 
     &           &    5 &  2.25e-08 &  1.58e-15 &      \\ 
     &           &    6 &  2.25e-08 &  6.43e-16 &      \\ 
     &           &    7 &  2.25e-08 &  6.42e-16 &      \\ 
     &           &    8 &  5.53e-08 &  6.41e-16 &      \\ 
     &           &    9 &  2.25e-08 &  1.58e-15 &      \\ 
     &           &   10 &  2.25e-08 &  6.42e-16 &      \\ 
3197 &  3.20e+04 &   10 &           &           & iters  \\ 
 \hdashline 
     &           &    1 &  2.64e-08 &  7.35e-09 &      \\ 
     &           &    2 &  2.63e-08 &  7.53e-16 &      \\ 
     &           &    3 &  5.14e-08 &  7.52e-16 &      \\ 
     &           &    4 &  2.64e-08 &  1.47e-15 &      \\ 
     &           &    5 &  2.63e-08 &  7.53e-16 &      \\ 
     &           &    6 &  5.14e-08 &  7.52e-16 &      \\ 
     &           &    7 &  2.64e-08 &  1.47e-15 &      \\ 
     &           &    8 &  2.63e-08 &  7.53e-16 &      \\ 
     &           &    9 &  5.14e-08 &  7.52e-16 &      \\ 
     &           &   10 &  2.64e-08 &  1.47e-15 &      \\ 
3198 &  3.20e+04 &   10 &           &           & iters  \\ 
 \hdashline 
     &           &    1 &  7.07e-09 &  1.21e-08 &      \\ 
     &           &    2 &  3.18e-08 &  2.02e-16 &      \\ 
     &           &    3 &  7.10e-09 &  9.07e-16 &      \\ 
     &           &    4 &  7.09e-09 &  2.03e-16 &      \\ 
     &           &    5 &  7.08e-09 &  2.02e-16 &      \\ 
     &           &    6 &  7.07e-09 &  2.02e-16 &      \\ 
     &           &    7 &  3.18e-08 &  2.02e-16 &      \\ 
     &           &    8 &  7.11e-09 &  9.07e-16 &      \\ 
     &           &    9 &  7.10e-09 &  2.03e-16 &      \\ 
     &           &   10 &  7.09e-09 &  2.03e-16 &      \\ 
3199 &  3.20e+04 &   10 &           &           & iters  \\ 
 \hdashline 
     &           &    1 &  2.65e-09 &  7.14e-09 &      \\ 
     &           &    2 &  2.64e-09 &  7.55e-17 &      \\ 
     &           &    3 &  2.64e-09 &  7.54e-17 &      \\ 
     &           &    4 &  2.63e-09 &  7.53e-17 &      \\ 
     &           &    5 &  2.63e-09 &  7.52e-17 &      \\ 
     &           &    6 &  2.63e-09 &  7.51e-17 &      \\ 
     &           &    7 &  2.62e-09 &  7.50e-17 &      \\ 
     &           &    8 &  2.62e-09 &  7.49e-17 &      \\ 
     &           &    9 &  2.62e-09 &  7.48e-17 &      \\ 
     &           &   10 &  2.61e-09 &  7.47e-17 &      \\ 
3200 &  3.20e+04 &   10 &           &           & iters  \\ 
 \hdashline 
     &           &    1 &  1.87e-10 &  1.78e-09 &      \\ 
     &           &    2 &  1.87e-10 &  5.35e-18 &      \\ 
     &           &    3 &  1.87e-10 &  5.34e-18 &      \\ 
     &           &    4 &  1.87e-10 &  5.33e-18 &      \\ 
     &           &    5 &  1.86e-10 &  5.33e-18 &      \\ 
     &           &    6 &  1.86e-10 &  5.32e-18 &      \\ 
     &           &    7 &  1.86e-10 &  5.31e-18 &      \\ 
     &           &    8 &  1.86e-10 &  5.30e-18 &      \\ 
     &           &    9 &  1.85e-10 &  5.29e-18 &      \\ 
     &           &   10 &  1.85e-10 &  5.29e-18 &      \\ 
3201 &  3.20e+04 &   10 &           &           & iters  \\ 
 \hdashline 
     &           &    1 &  3.68e-08 &  9.85e-09 &      \\ 
     &           &    2 &  2.07e-09 &  1.05e-15 &      \\ 
     &           &    3 &  2.07e-09 &  5.91e-17 &      \\ 
     &           &    4 &  2.06e-09 &  5.90e-17 &      \\ 
     &           &    5 &  2.06e-09 &  5.89e-17 &      \\ 
     &           &    6 &  2.06e-09 &  5.88e-17 &      \\ 
     &           &    7 &  2.06e-09 &  5.88e-17 &      \\ 
     &           &    8 &  2.05e-09 &  5.87e-17 &      \\ 
     &           &    9 &  2.05e-09 &  5.86e-17 &      \\ 
     &           &   10 &  2.05e-09 &  5.85e-17 &      \\ 
3202 &  3.20e+04 &   10 &           &           & iters  \\ 
 \hdashline 
     &           &    1 &  1.50e-08 &  1.46e-08 &      \\ 
     &           &    2 &  1.50e-08 &  4.28e-16 &      \\ 
     &           &    3 &  6.27e-08 &  4.28e-16 &      \\ 
     &           &    4 &  1.50e-08 &  1.79e-15 &      \\ 
     &           &    5 &  1.50e-08 &  4.30e-16 &      \\ 
     &           &    6 &  1.50e-08 &  4.29e-16 &      \\ 
     &           &    7 &  1.50e-08 &  4.28e-16 &      \\ 
     &           &    8 &  1.50e-08 &  4.28e-16 &      \\ 
     &           &    9 &  6.27e-08 &  4.27e-16 &      \\ 
     &           &   10 &  1.50e-08 &  1.79e-15 &      \\ 
3203 &  3.20e+04 &   10 &           &           & iters  \\ 
 \hdashline 
     &           &    1 &  4.71e-08 &  7.97e-09 &      \\ 
     &           &    2 &  3.06e-08 &  1.34e-15 &      \\ 
     &           &    3 &  4.71e-08 &  8.75e-16 &      \\ 
     &           &    4 &  3.07e-08 &  1.34e-15 &      \\ 
     &           &    5 &  3.06e-08 &  8.75e-16 &      \\ 
     &           &    6 &  4.71e-08 &  8.74e-16 &      \\ 
     &           &    7 &  3.06e-08 &  1.34e-15 &      \\ 
     &           &    8 &  3.06e-08 &  8.75e-16 &      \\ 
     &           &    9 &  4.71e-08 &  8.73e-16 &      \\ 
     &           &   10 &  3.06e-08 &  1.35e-15 &      \\ 
3204 &  3.20e+04 &   10 &           &           & iters  \\ 
 \hdashline 
     &           &    1 &  5.75e-08 &  4.57e-09 &      \\ 
     &           &    2 &  2.03e-08 &  1.64e-15 &      \\ 
     &           &    3 &  2.03e-08 &  5.80e-16 &      \\ 
     &           &    4 &  5.74e-08 &  5.79e-16 &      \\ 
     &           &    5 &  2.03e-08 &  1.64e-15 &      \\ 
     &           &    6 &  2.03e-08 &  5.80e-16 &      \\ 
     &           &    7 &  2.03e-08 &  5.80e-16 &      \\ 
     &           &    8 &  5.74e-08 &  5.79e-16 &      \\ 
     &           &    9 &  2.03e-08 &  1.64e-15 &      \\ 
     &           &   10 &  2.03e-08 &  5.80e-16 &      \\ 
3205 &  3.20e+04 &   10 &           &           & iters  \\ 
 \hdashline 
     &           &    1 &  3.36e-08 &  1.26e-09 &      \\ 
     &           &    2 &  4.41e-08 &  9.59e-16 &      \\ 
     &           &    3 &  3.36e-08 &  1.26e-15 &      \\ 
     &           &    4 &  3.36e-08 &  9.59e-16 &      \\ 
     &           &    5 &  4.42e-08 &  9.58e-16 &      \\ 
     &           &    6 &  3.36e-08 &  1.26e-15 &      \\ 
     &           &    7 &  4.42e-08 &  9.59e-16 &      \\ 
     &           &    8 &  3.36e-08 &  1.26e-15 &      \\ 
     &           &    9 &  4.41e-08 &  9.59e-16 &      \\ 
     &           &   10 &  3.36e-08 &  1.26e-15 &      \\ 
3206 &  3.20e+04 &   10 &           &           & iters  \\ 
 \hdashline 
     &           &    1 &  3.05e-08 &  2.27e-09 &      \\ 
     &           &    2 &  4.72e-08 &  8.71e-16 &      \\ 
     &           &    3 &  3.06e-08 &  1.35e-15 &      \\ 
     &           &    4 &  3.05e-08 &  8.72e-16 &      \\ 
     &           &    5 &  4.72e-08 &  8.71e-16 &      \\ 
     &           &    6 &  3.05e-08 &  1.35e-15 &      \\ 
     &           &    7 &  4.72e-08 &  8.72e-16 &      \\ 
     &           &    8 &  3.06e-08 &  1.35e-15 &      \\ 
     &           &    9 &  3.05e-08 &  8.72e-16 &      \\ 
     &           &   10 &  4.72e-08 &  8.71e-16 &      \\ 
3207 &  3.21e+04 &   10 &           &           & iters  \\ 
 \hdashline 
     &           &    1 &  1.63e-08 &  7.39e-09 &      \\ 
     &           &    2 &  1.62e-08 &  4.64e-16 &      \\ 
     &           &    3 &  1.62e-08 &  4.63e-16 &      \\ 
     &           &    4 &  6.15e-08 &  4.63e-16 &      \\ 
     &           &    5 &  1.63e-08 &  1.76e-15 &      \\ 
     &           &    6 &  1.63e-08 &  4.64e-16 &      \\ 
     &           &    7 &  1.62e-08 &  4.64e-16 &      \\ 
     &           &    8 &  1.62e-08 &  4.63e-16 &      \\ 
     &           &    9 &  6.15e-08 &  4.62e-16 &      \\ 
     &           &   10 &  1.63e-08 &  1.76e-15 &      \\ 
3208 &  3.21e+04 &   10 &           &           & iters  \\ 
 \hdashline 
     &           &    1 &  6.23e-09 &  5.04e-09 &      \\ 
     &           &    2 &  6.22e-09 &  1.78e-16 &      \\ 
     &           &    3 &  6.21e-09 &  1.78e-16 &      \\ 
     &           &    4 &  6.20e-09 &  1.77e-16 &      \\ 
     &           &    5 &  6.19e-09 &  1.77e-16 &      \\ 
     &           &    6 &  6.19e-09 &  1.77e-16 &      \\ 
     &           &    7 &  6.18e-09 &  1.77e-16 &      \\ 
     &           &    8 &  6.17e-09 &  1.76e-16 &      \\ 
     &           &    9 &  6.16e-09 &  1.76e-16 &      \\ 
     &           &   10 &  6.15e-09 &  1.76e-16 &      \\ 
3209 &  3.21e+04 &   10 &           &           & iters  \\ 
 \hdashline 
     &           &    1 &  6.50e-08 &  7.77e-09 &      \\ 
     &           &    2 &  1.28e-08 &  1.86e-15 &      \\ 
     &           &    3 &  1.28e-08 &  3.65e-16 &      \\ 
     &           &    4 &  1.27e-08 &  3.64e-16 &      \\ 
     &           &    5 &  1.27e-08 &  3.64e-16 &      \\ 
     &           &    6 &  1.27e-08 &  3.63e-16 &      \\ 
     &           &    7 &  6.50e-08 &  3.63e-16 &      \\ 
     &           &    8 &  1.28e-08 &  1.86e-15 &      \\ 
     &           &    9 &  1.28e-08 &  3.65e-16 &      \\ 
     &           &   10 &  1.27e-08 &  3.64e-16 &      \\ 
3210 &  3.21e+04 &   10 &           &           & iters  \\ 
 \hdashline 
     &           &    1 &  8.65e-08 &  6.13e-09 &      \\ 
     &           &    2 &  6.90e-08 &  2.47e-15 &      \\ 
     &           &    3 &  8.79e-09 &  1.97e-15 &      \\ 
     &           &    4 &  8.78e-09 &  2.51e-16 &      \\ 
     &           &    5 &  8.77e-09 &  2.51e-16 &      \\ 
     &           &    6 &  8.76e-09 &  2.50e-16 &      \\ 
     &           &    7 &  8.74e-09 &  2.50e-16 &      \\ 
     &           &    8 &  8.73e-09 &  2.50e-16 &      \\ 
     &           &    9 &  6.90e-08 &  2.49e-16 &      \\ 
     &           &   10 &  8.65e-08 &  1.97e-15 &      \\ 
3211 &  3.21e+04 &   10 &           &           & iters  \\ 
 \hdashline 
     &           &    1 &  6.41e-08 &  3.80e-09 &      \\ 
     &           &    2 &  1.37e-08 &  1.83e-15 &      \\ 
     &           &    3 &  1.37e-08 &  3.91e-16 &      \\ 
     &           &    4 &  6.40e-08 &  3.90e-16 &      \\ 
     &           &    5 &  1.37e-08 &  1.83e-15 &      \\ 
     &           &    6 &  1.37e-08 &  3.92e-16 &      \\ 
     &           &    7 &  1.37e-08 &  3.92e-16 &      \\ 
     &           &    8 &  1.37e-08 &  3.91e-16 &      \\ 
     &           &    9 &  1.37e-08 &  3.91e-16 &      \\ 
     &           &   10 &  6.40e-08 &  3.90e-16 &      \\ 
3212 &  3.21e+04 &   10 &           &           & iters  \\ 
 \hdashline 
     &           &    1 &  3.24e-08 &  1.59e-09 &      \\ 
     &           &    2 &  3.23e-08 &  9.24e-16 &      \\ 
     &           &    3 &  4.54e-08 &  9.23e-16 &      \\ 
     &           &    4 &  3.24e-08 &  1.30e-15 &      \\ 
     &           &    5 &  4.54e-08 &  9.24e-16 &      \\ 
     &           &    6 &  3.24e-08 &  1.29e-15 &      \\ 
     &           &    7 &  3.23e-08 &  9.24e-16 &      \\ 
     &           &    8 &  4.54e-08 &  9.23e-16 &      \\ 
     &           &    9 &  3.24e-08 &  1.30e-15 &      \\ 
     &           &   10 &  4.54e-08 &  9.23e-16 &      \\ 
3213 &  3.21e+04 &   10 &           &           & iters  \\ 
 \hdashline 
     &           &    1 &  7.25e-08 &  2.49e-09 &      \\ 
     &           &    2 &  5.33e-09 &  2.07e-15 &      \\ 
     &           &    3 &  5.32e-09 &  1.52e-16 &      \\ 
     &           &    4 &  5.31e-09 &  1.52e-16 &      \\ 
     &           &    5 &  5.30e-09 &  1.52e-16 &      \\ 
     &           &    6 &  5.30e-09 &  1.51e-16 &      \\ 
     &           &    7 &  5.29e-09 &  1.51e-16 &      \\ 
     &           &    8 &  5.28e-09 &  1.51e-16 &      \\ 
     &           &    9 &  5.27e-09 &  1.51e-16 &      \\ 
     &           &   10 &  5.27e-09 &  1.51e-16 &      \\ 
3214 &  3.21e+04 &   10 &           &           & iters  \\ 
 \hdashline 
     &           &    1 &  4.32e-08 &  5.79e-09 &      \\ 
     &           &    2 &  3.46e-08 &  1.23e-15 &      \\ 
     &           &    3 &  4.32e-08 &  9.86e-16 &      \\ 
     &           &    4 &  3.46e-08 &  1.23e-15 &      \\ 
     &           &    5 &  4.32e-08 &  9.87e-16 &      \\ 
     &           &    6 &  3.46e-08 &  1.23e-15 &      \\ 
     &           &    7 &  3.45e-08 &  9.87e-16 &      \\ 
     &           &    8 &  4.32e-08 &  9.86e-16 &      \\ 
     &           &    9 &  3.46e-08 &  1.23e-15 &      \\ 
     &           &   10 &  4.32e-08 &  9.86e-16 &      \\ 
3215 &  3.21e+04 &   10 &           &           & iters  \\ 
 \hdashline 
     &           &    1 &  8.27e-08 &  1.17e-08 &      \\ 
     &           &    2 &  7.28e-08 &  2.36e-15 &      \\ 
     &           &    3 &  5.03e-09 &  2.08e-15 &      \\ 
     &           &    4 &  5.02e-09 &  1.43e-16 &      \\ 
     &           &    5 &  5.01e-09 &  1.43e-16 &      \\ 
     &           &    6 &  5.01e-09 &  1.43e-16 &      \\ 
     &           &    7 &  5.00e-09 &  1.43e-16 &      \\ 
     &           &    8 &  4.99e-09 &  1.43e-16 &      \\ 
     &           &    9 &  4.98e-09 &  1.42e-16 &      \\ 
     &           &   10 &  4.98e-09 &  1.42e-16 &      \\ 
3216 &  3.22e+04 &   10 &           &           & iters  \\ 
 \hdashline 
     &           &    1 &  1.73e-08 &  3.53e-09 &      \\ 
     &           &    2 &  1.73e-08 &  4.94e-16 &      \\ 
     &           &    3 &  6.04e-08 &  4.93e-16 &      \\ 
     &           &    4 &  1.73e-08 &  1.72e-15 &      \\ 
     &           &    5 &  1.73e-08 &  4.95e-16 &      \\ 
     &           &    6 &  1.73e-08 &  4.94e-16 &      \\ 
     &           &    7 &  1.73e-08 &  4.93e-16 &      \\ 
     &           &    8 &  6.04e-08 &  4.93e-16 &      \\ 
     &           &    9 &  1.73e-08 &  1.73e-15 &      \\ 
     &           &   10 &  1.73e-08 &  4.94e-16 &      \\ 
3217 &  3.22e+04 &   10 &           &           & iters  \\ 
 \hdashline 
     &           &    1 &  2.25e-08 &  5.68e-09 &      \\ 
     &           &    2 &  5.52e-08 &  6.41e-16 &      \\ 
     &           &    3 &  2.25e-08 &  1.58e-15 &      \\ 
     &           &    4 &  2.25e-08 &  6.43e-16 &      \\ 
     &           &    5 &  5.52e-08 &  6.42e-16 &      \\ 
     &           &    6 &  2.25e-08 &  1.58e-15 &      \\ 
     &           &    7 &  2.25e-08 &  6.43e-16 &      \\ 
     &           &    8 &  2.25e-08 &  6.42e-16 &      \\ 
     &           &    9 &  5.52e-08 &  6.41e-16 &      \\ 
     &           &   10 &  2.25e-08 &  1.58e-15 &      \\ 
3218 &  3.22e+04 &   10 &           &           & iters  \\ 
 \hdashline 
     &           &    1 &  4.89e-08 &  2.04e-09 &      \\ 
     &           &    2 &  2.88e-08 &  1.40e-15 &      \\ 
     &           &    3 &  2.88e-08 &  8.22e-16 &      \\ 
     &           &    4 &  4.90e-08 &  8.21e-16 &      \\ 
     &           &    5 &  2.88e-08 &  1.40e-15 &      \\ 
     &           &    6 &  2.88e-08 &  8.22e-16 &      \\ 
     &           &    7 &  4.90e-08 &  8.21e-16 &      \\ 
     &           &    8 &  2.88e-08 &  1.40e-15 &      \\ 
     &           &    9 &  2.87e-08 &  8.22e-16 &      \\ 
     &           &   10 &  4.90e-08 &  8.20e-16 &      \\ 
3219 &  3.22e+04 &   10 &           &           & iters  \\ 
 \hdashline 
     &           &    1 &  1.04e-08 &  4.10e-09 &      \\ 
     &           &    2 &  1.04e-08 &  2.97e-16 &      \\ 
     &           &    3 &  1.04e-08 &  2.96e-16 &      \\ 
     &           &    4 &  1.03e-08 &  2.96e-16 &      \\ 
     &           &    5 &  1.03e-08 &  2.95e-16 &      \\ 
     &           &    6 &  1.03e-08 &  2.95e-16 &      \\ 
     &           &    7 &  6.74e-08 &  2.94e-16 &      \\ 
     &           &    8 &  1.04e-08 &  1.92e-15 &      \\ 
     &           &    9 &  1.04e-08 &  2.97e-16 &      \\ 
     &           &   10 &  1.04e-08 &  2.96e-16 &      \\ 
3220 &  3.22e+04 &   10 &           &           & iters  \\ 
 \hdashline 
     &           &    1 &  1.14e-08 &  8.42e-09 &      \\ 
     &           &    2 &  1.14e-08 &  3.25e-16 &      \\ 
     &           &    3 &  6.63e-08 &  3.24e-16 &      \\ 
     &           &    4 &  1.14e-08 &  1.89e-15 &      \\ 
     &           &    5 &  1.14e-08 &  3.26e-16 &      \\ 
     &           &    6 &  1.14e-08 &  3.26e-16 &      \\ 
     &           &    7 &  1.14e-08 &  3.25e-16 &      \\ 
     &           &    8 &  1.14e-08 &  3.25e-16 &      \\ 
     &           &    9 &  1.14e-08 &  3.24e-16 &      \\ 
     &           &   10 &  6.64e-08 &  3.24e-16 &      \\ 
3221 &  3.22e+04 &   10 &           &           & iters  \\ 
 \hdashline 
     &           &    1 &  3.72e-08 &  2.42e-09 &      \\ 
     &           &    2 &  4.05e-08 &  1.06e-15 &      \\ 
     &           &    3 &  3.72e-08 &  1.16e-15 &      \\ 
     &           &    4 &  4.05e-08 &  1.06e-15 &      \\ 
     &           &    5 &  3.72e-08 &  1.16e-15 &      \\ 
     &           &    6 &  4.05e-08 &  1.06e-15 &      \\ 
     &           &    7 &  3.72e-08 &  1.16e-15 &      \\ 
     &           &    8 &  4.05e-08 &  1.06e-15 &      \\ 
     &           &    9 &  3.72e-08 &  1.16e-15 &      \\ 
     &           &   10 &  4.05e-08 &  1.06e-15 &      \\ 
3222 &  3.22e+04 &   10 &           &           & iters  \\ 
 \hdashline 
     &           &    1 &  5.81e-08 &  4.03e-09 &      \\ 
     &           &    2 &  1.97e-08 &  1.66e-15 &      \\ 
     &           &    3 &  5.80e-08 &  5.63e-16 &      \\ 
     &           &    4 &  1.98e-08 &  1.66e-15 &      \\ 
     &           &    5 &  1.97e-08 &  5.64e-16 &      \\ 
     &           &    6 &  1.97e-08 &  5.64e-16 &      \\ 
     &           &    7 &  5.80e-08 &  5.63e-16 &      \\ 
     &           &    8 &  1.98e-08 &  1.66e-15 &      \\ 
     &           &    9 &  1.97e-08 &  5.64e-16 &      \\ 
     &           &   10 &  1.97e-08 &  5.63e-16 &      \\ 
3223 &  3.22e+04 &   10 &           &           & iters  \\ 
 \hdashline 
     &           &    1 &  1.02e-08 &  4.84e-09 &      \\ 
     &           &    2 &  1.02e-08 &  2.91e-16 &      \\ 
     &           &    3 &  6.75e-08 &  2.90e-16 &      \\ 
     &           &    4 &  8.79e-08 &  1.93e-15 &      \\ 
     &           &    5 &  6.76e-08 &  2.51e-15 &      \\ 
     &           &    6 &  1.02e-08 &  1.93e-15 &      \\ 
     &           &    7 &  1.02e-08 &  2.92e-16 &      \\ 
     &           &    8 &  1.02e-08 &  2.91e-16 &      \\ 
     &           &    9 &  1.02e-08 &  2.91e-16 &      \\ 
     &           &   10 &  1.02e-08 &  2.91e-16 &      \\ 
3224 &  3.22e+04 &   10 &           &           & iters  \\ 
 \hdashline 
     &           &    1 &  4.14e-08 &  1.90e-09 &      \\ 
     &           &    2 &  2.48e-09 &  1.18e-15 &      \\ 
     &           &    3 &  2.48e-09 &  7.08e-17 &      \\ 
     &           &    4 &  2.47e-09 &  7.07e-17 &      \\ 
     &           &    5 &  2.47e-09 &  7.06e-17 &      \\ 
     &           &    6 &  2.47e-09 &  7.05e-17 &      \\ 
     &           &    7 &  2.46e-09 &  7.04e-17 &      \\ 
     &           &    8 &  2.46e-09 &  7.03e-17 &      \\ 
     &           &    9 &  2.46e-09 &  7.02e-17 &      \\ 
     &           &   10 &  2.45e-09 &  7.01e-17 &      \\ 
3225 &  3.22e+04 &   10 &           &           & iters  \\ 
 \hdashline 
     &           &    1 &  1.67e-08 &  1.45e-09 &      \\ 
     &           &    2 &  6.10e-08 &  4.78e-16 &      \\ 
     &           &    3 &  1.68e-08 &  1.74e-15 &      \\ 
     &           &    4 &  1.68e-08 &  4.79e-16 &      \\ 
     &           &    5 &  1.67e-08 &  4.79e-16 &      \\ 
     &           &    6 &  1.67e-08 &  4.78e-16 &      \\ 
     &           &    7 &  6.10e-08 &  4.77e-16 &      \\ 
     &           &    8 &  1.68e-08 &  1.74e-15 &      \\ 
     &           &    9 &  1.68e-08 &  4.79e-16 &      \\ 
     &           &   10 &  1.67e-08 &  4.78e-16 &      \\ 
3226 &  3.22e+04 &   10 &           &           & iters  \\ 
 \hdashline 
     &           &    1 &  1.74e-08 &  2.11e-09 &      \\ 
     &           &    2 &  1.74e-08 &  4.98e-16 &      \\ 
     &           &    3 &  1.74e-08 &  4.97e-16 &      \\ 
     &           &    4 &  1.74e-08 &  4.96e-16 &      \\ 
     &           &    5 &  6.03e-08 &  4.96e-16 &      \\ 
     &           &    6 &  1.74e-08 &  1.72e-15 &      \\ 
     &           &    7 &  1.74e-08 &  4.98e-16 &      \\ 
     &           &    8 &  1.74e-08 &  4.97e-16 &      \\ 
     &           &    9 &  6.03e-08 &  4.96e-16 &      \\ 
     &           &   10 &  1.74e-08 &  1.72e-15 &      \\ 
3227 &  3.23e+04 &   10 &           &           & iters  \\ 
 \hdashline 
     &           &    1 &  1.25e-08 &  7.37e-09 &      \\ 
     &           &    2 &  1.25e-08 &  3.57e-16 &      \\ 
     &           &    3 &  1.25e-08 &  3.56e-16 &      \\ 
     &           &    4 &  2.64e-08 &  3.56e-16 &      \\ 
     &           &    5 &  1.25e-08 &  7.53e-16 &      \\ 
     &           &    6 &  1.25e-08 &  3.56e-16 &      \\ 
     &           &    7 &  2.64e-08 &  3.56e-16 &      \\ 
     &           &    8 &  1.25e-08 &  7.53e-16 &      \\ 
     &           &    9 &  1.25e-08 &  3.57e-16 &      \\ 
     &           &   10 &  2.64e-08 &  3.56e-16 &      \\ 
3228 &  3.23e+04 &   10 &           &           & iters  \\ 
 \hdashline 
     &           &    1 &  2.95e-09 &  2.96e-09 &      \\ 
     &           &    2 &  2.94e-09 &  8.41e-17 &      \\ 
     &           &    3 &  2.94e-09 &  8.40e-17 &      \\ 
     &           &    4 &  2.93e-09 &  8.39e-17 &      \\ 
     &           &    5 &  2.93e-09 &  8.37e-17 &      \\ 
     &           &    6 &  2.93e-09 &  8.36e-17 &      \\ 
     &           &    7 &  2.92e-09 &  8.35e-17 &      \\ 
     &           &    8 &  2.92e-09 &  8.34e-17 &      \\ 
     &           &    9 &  2.91e-09 &  8.33e-17 &      \\ 
     &           &   10 &  2.91e-09 &  8.31e-17 &      \\ 
3229 &  3.23e+04 &   10 &           &           & iters  \\ 
 \hdashline 
     &           &    1 &  3.39e-08 &  4.56e-09 &      \\ 
     &           &    2 &  4.99e-09 &  9.68e-16 &      \\ 
     &           &    3 &  4.98e-09 &  1.42e-16 &      \\ 
     &           &    4 &  4.97e-09 &  1.42e-16 &      \\ 
     &           &    5 &  4.97e-09 &  1.42e-16 &      \\ 
     &           &    6 &  4.96e-09 &  1.42e-16 &      \\ 
     &           &    7 &  4.95e-09 &  1.42e-16 &      \\ 
     &           &    8 &  4.95e-09 &  1.41e-16 &      \\ 
     &           &    9 &  3.39e-08 &  1.41e-16 &      \\ 
     &           &   10 &  4.99e-09 &  9.68e-16 &      \\ 
3230 &  3.23e+04 &   10 &           &           & iters  \\ 
 \hdashline 
     &           &    1 &  2.99e-08 &  4.40e-09 &      \\ 
     &           &    2 &  8.99e-09 &  8.53e-16 &      \\ 
     &           &    3 &  8.98e-09 &  2.57e-16 &      \\ 
     &           &    4 &  8.97e-09 &  2.56e-16 &      \\ 
     &           &    5 &  8.96e-09 &  2.56e-16 &      \\ 
     &           &    6 &  2.99e-08 &  2.56e-16 &      \\ 
     &           &    7 &  8.99e-09 &  8.53e-16 &      \\ 
     &           &    8 &  8.97e-09 &  2.56e-16 &      \\ 
     &           &    9 &  8.96e-09 &  2.56e-16 &      \\ 
     &           &   10 &  2.99e-08 &  2.56e-16 &      \\ 
3231 &  3.23e+04 &   10 &           &           & iters  \\ 
 \hdashline 
     &           &    1 &  2.35e-08 &  2.50e-09 &      \\ 
     &           &    2 &  2.34e-08 &  6.70e-16 &      \\ 
     &           &    3 &  5.43e-08 &  6.69e-16 &      \\ 
     &           &    4 &  2.35e-08 &  1.55e-15 &      \\ 
     &           &    5 &  2.34e-08 &  6.70e-16 &      \\ 
     &           &    6 &  5.43e-08 &  6.69e-16 &      \\ 
     &           &    7 &  2.35e-08 &  1.55e-15 &      \\ 
     &           &    8 &  2.35e-08 &  6.70e-16 &      \\ 
     &           &    9 &  5.43e-08 &  6.69e-16 &      \\ 
     &           &   10 &  2.35e-08 &  1.55e-15 &      \\ 
3232 &  3.23e+04 &   10 &           &           & iters  \\ 
 \hdashline 
     &           &    1 &  3.19e-08 &  2.97e-09 &      \\ 
     &           &    2 &  4.58e-08 &  9.12e-16 &      \\ 
     &           &    3 &  3.20e-08 &  1.31e-15 &      \\ 
     &           &    4 &  4.58e-08 &  9.12e-16 &      \\ 
     &           &    5 &  3.20e-08 &  1.31e-15 &      \\ 
     &           &    6 &  3.19e-08 &  9.13e-16 &      \\ 
     &           &    7 &  4.58e-08 &  9.12e-16 &      \\ 
     &           &    8 &  3.20e-08 &  1.31e-15 &      \\ 
     &           &    9 &  4.58e-08 &  9.12e-16 &      \\ 
     &           &   10 &  3.20e-08 &  1.31e-15 &      \\ 
3233 &  3.23e+04 &   10 &           &           & iters  \\ 
 \hdashline 
     &           &    1 &  1.32e-08 &  1.15e-09 &      \\ 
     &           &    2 &  6.45e-08 &  3.77e-16 &      \\ 
     &           &    3 &  9.10e-08 &  1.84e-15 &      \\ 
     &           &    4 &  6.45e-08 &  2.60e-15 &      \\ 
     &           &    5 &  1.33e-08 &  1.84e-15 &      \\ 
     &           &    6 &  1.32e-08 &  3.78e-16 &      \\ 
     &           &    7 &  1.32e-08 &  3.78e-16 &      \\ 
     &           &    8 &  6.45e-08 &  3.77e-16 &      \\ 
     &           &    9 &  1.33e-08 &  1.84e-15 &      \\ 
     &           &   10 &  1.33e-08 &  3.79e-16 &      \\ 
3234 &  3.23e+04 &   10 &           &           & iters  \\ 
 \hdashline 
     &           &    1 &  6.82e-10 &  5.57e-09 &      \\ 
     &           &    2 &  6.81e-10 &  1.95e-17 &      \\ 
     &           &    3 &  6.80e-10 &  1.94e-17 &      \\ 
     &           &    4 &  6.79e-10 &  1.94e-17 &      \\ 
     &           &    5 &  6.78e-10 &  1.94e-17 &      \\ 
     &           &    6 &  6.77e-10 &  1.94e-17 &      \\ 
     &           &    7 &  6.76e-10 &  1.93e-17 &      \\ 
     &           &    8 &  6.75e-10 &  1.93e-17 &      \\ 
     &           &    9 &  6.74e-10 &  1.93e-17 &      \\ 
     &           &   10 &  6.73e-10 &  1.92e-17 &      \\ 
3235 &  3.23e+04 &   10 &           &           & iters  \\ 
 \hdashline 
     &           &    1 &  1.89e-09 &  1.64e-09 &      \\ 
     &           &    2 &  1.89e-09 &  5.40e-17 &      \\ 
     &           &    3 &  1.89e-09 &  5.39e-17 &      \\ 
     &           &    4 &  1.88e-09 &  5.38e-17 &      \\ 
     &           &    5 &  1.88e-09 &  5.38e-17 &      \\ 
     &           &    6 &  1.88e-09 &  5.37e-17 &      \\ 
     &           &    7 &  1.88e-09 &  5.36e-17 &      \\ 
     &           &    8 &  1.87e-09 &  5.35e-17 &      \\ 
     &           &    9 &  1.87e-09 &  5.35e-17 &      \\ 
     &           &   10 &  1.87e-09 &  5.34e-17 &      \\ 
3236 &  3.24e+04 &   10 &           &           & iters  \\ 
 \hdashline 
     &           &    1 &  2.79e-08 &  4.01e-09 &      \\ 
     &           &    2 &  2.79e-08 &  7.96e-16 &      \\ 
     &           &    3 &  1.10e-08 &  7.95e-16 &      \\ 
     &           &    4 &  1.10e-08 &  3.15e-16 &      \\ 
     &           &    5 &  2.79e-08 &  3.14e-16 &      \\ 
     &           &    6 &  1.10e-08 &  7.95e-16 &      \\ 
     &           &    7 &  1.10e-08 &  3.15e-16 &      \\ 
     &           &    8 &  2.78e-08 &  3.14e-16 &      \\ 
     &           &    9 &  1.10e-08 &  7.95e-16 &      \\ 
     &           &   10 &  1.10e-08 &  3.15e-16 &      \\ 
3237 &  3.24e+04 &   10 &           &           & iters  \\ 
 \hdashline 
     &           &    1 &  1.04e-08 &  3.81e-09 &      \\ 
     &           &    2 &  6.73e-08 &  2.97e-16 &      \\ 
     &           &    3 &  1.05e-08 &  1.92e-15 &      \\ 
     &           &    4 &  1.05e-08 &  2.99e-16 &      \\ 
     &           &    5 &  1.04e-08 &  2.98e-16 &      \\ 
     &           &    6 &  1.04e-08 &  2.98e-16 &      \\ 
     &           &    7 &  1.04e-08 &  2.98e-16 &      \\ 
     &           &    8 &  1.04e-08 &  2.97e-16 &      \\ 
     &           &    9 &  6.73e-08 &  2.97e-16 &      \\ 
     &           &   10 &  8.82e-08 &  1.92e-15 &      \\ 
3238 &  3.24e+04 &   10 &           &           & iters  \\ 
 \hdashline 
     &           &    1 &  5.84e-08 &  1.76e-09 &      \\ 
     &           &    2 &  1.94e-08 &  1.67e-15 &      \\ 
     &           &    3 &  1.94e-08 &  5.53e-16 &      \\ 
     &           &    4 &  5.84e-08 &  5.53e-16 &      \\ 
     &           &    5 &  1.94e-08 &  1.67e-15 &      \\ 
     &           &    6 &  1.94e-08 &  5.54e-16 &      \\ 
     &           &    7 &  1.94e-08 &  5.53e-16 &      \\ 
     &           &    8 &  5.84e-08 &  5.53e-16 &      \\ 
     &           &    9 &  1.94e-08 &  1.67e-15 &      \\ 
     &           &   10 &  1.94e-08 &  5.54e-16 &      \\ 
3239 &  3.24e+04 &   10 &           &           & iters  \\ 
 \hdashline 
     &           &    1 &  1.60e-08 &  1.11e-09 &      \\ 
     &           &    2 &  1.60e-08 &  4.58e-16 &      \\ 
     &           &    3 &  6.17e-08 &  4.57e-16 &      \\ 
     &           &    4 &  1.61e-08 &  1.76e-15 &      \\ 
     &           &    5 &  1.61e-08 &  4.59e-16 &      \\ 
     &           &    6 &  1.60e-08 &  4.58e-16 &      \\ 
     &           &    7 &  1.60e-08 &  4.58e-16 &      \\ 
     &           &    8 &  6.17e-08 &  4.57e-16 &      \\ 
     &           &    9 &  1.61e-08 &  1.76e-15 &      \\ 
     &           &   10 &  1.61e-08 &  4.59e-16 &      \\ 
3240 &  3.24e+04 &   10 &           &           & iters  \\ 
 \hdashline 
     &           &    1 &  1.62e-08 &  8.73e-10 &      \\ 
     &           &    2 &  1.62e-08 &  4.64e-16 &      \\ 
     &           &    3 &  1.62e-08 &  4.63e-16 &      \\ 
     &           &    4 &  6.15e-08 &  4.62e-16 &      \\ 
     &           &    5 &  1.63e-08 &  1.76e-15 &      \\ 
     &           &    6 &  1.62e-08 &  4.64e-16 &      \\ 
     &           &    7 &  1.62e-08 &  4.63e-16 &      \\ 
     &           &    8 &  1.62e-08 &  4.63e-16 &      \\ 
     &           &    9 &  6.15e-08 &  4.62e-16 &      \\ 
     &           &   10 &  1.63e-08 &  1.76e-15 &      \\ 
3241 &  3.24e+04 &   10 &           &           & iters  \\ 
 \hdashline 
     &           &    1 &  3.20e-08 &  5.44e-10 &      \\ 
     &           &    2 &  4.57e-08 &  9.15e-16 &      \\ 
     &           &    3 &  3.21e-08 &  1.30e-15 &      \\ 
     &           &    4 &  4.57e-08 &  9.15e-16 &      \\ 
     &           &    5 &  3.21e-08 &  1.30e-15 &      \\ 
     &           &    6 &  3.20e-08 &  9.16e-16 &      \\ 
     &           &    7 &  4.57e-08 &  9.15e-16 &      \\ 
     &           &    8 &  3.21e-08 &  1.30e-15 &      \\ 
     &           &    9 &  4.57e-08 &  9.15e-16 &      \\ 
     &           &   10 &  3.21e-08 &  1.30e-15 &      \\ 
3242 &  3.24e+04 &   10 &           &           & iters  \\ 
 \hdashline 
     &           &    1 &  4.17e-08 &  7.01e-09 &      \\ 
     &           &    2 &  3.61e-08 &  1.19e-15 &      \\ 
     &           &    3 &  4.17e-08 &  1.03e-15 &      \\ 
     &           &    4 &  3.61e-08 &  1.19e-15 &      \\ 
     &           &    5 &  3.60e-08 &  1.03e-15 &      \\ 
     &           &    6 &  4.17e-08 &  1.03e-15 &      \\ 
     &           &    7 &  3.61e-08 &  1.19e-15 &      \\ 
     &           &    8 &  4.17e-08 &  1.03e-15 &      \\ 
     &           &    9 &  3.61e-08 &  1.19e-15 &      \\ 
     &           &   10 &  4.17e-08 &  1.03e-15 &      \\ 
3243 &  3.24e+04 &   10 &           &           & iters  \\ 
 \hdashline 
     &           &    1 &  2.75e-08 &  5.86e-09 &      \\ 
     &           &    2 &  1.14e-08 &  7.86e-16 &      \\ 
     &           &    3 &  1.13e-08 &  3.24e-16 &      \\ 
     &           &    4 &  2.75e-08 &  3.23e-16 &      \\ 
     &           &    5 &  1.14e-08 &  7.86e-16 &      \\ 
     &           &    6 &  1.13e-08 &  3.24e-16 &      \\ 
     &           &    7 &  1.13e-08 &  3.24e-16 &      \\ 
     &           &    8 &  2.75e-08 &  3.23e-16 &      \\ 
     &           &    9 &  1.13e-08 &  7.86e-16 &      \\ 
     &           &   10 &  1.13e-08 &  3.24e-16 &      \\ 
3244 &  3.24e+04 &   10 &           &           & iters  \\ 
 \hdashline 
     &           &    1 &  2.08e-08 &  6.34e-09 &      \\ 
     &           &    2 &  2.08e-08 &  5.93e-16 &      \\ 
     &           &    3 &  5.70e-08 &  5.93e-16 &      \\ 
     &           &    4 &  2.08e-08 &  1.63e-15 &      \\ 
     &           &    5 &  2.08e-08 &  5.94e-16 &      \\ 
     &           &    6 &  2.08e-08 &  5.93e-16 &      \\ 
     &           &    7 &  5.70e-08 &  5.92e-16 &      \\ 
     &           &    8 &  2.08e-08 &  1.63e-15 &      \\ 
     &           &    9 &  2.08e-08 &  5.94e-16 &      \\ 
     &           &   10 &  2.07e-08 &  5.93e-16 &      \\ 
3245 &  3.24e+04 &   10 &           &           & iters  \\ 
 \hdashline 
     &           &    1 &  1.75e-08 &  3.91e-09 &      \\ 
     &           &    2 &  1.75e-08 &  5.00e-16 &      \\ 
     &           &    3 &  1.75e-08 &  4.99e-16 &      \\ 
     &           &    4 &  6.02e-08 &  4.98e-16 &      \\ 
     &           &    5 &  1.75e-08 &  1.72e-15 &      \\ 
     &           &    6 &  1.75e-08 &  5.00e-16 &      \\ 
     &           &    7 &  1.75e-08 &  4.99e-16 &      \\ 
     &           &    8 &  1.75e-08 &  4.99e-16 &      \\ 
     &           &    9 &  6.03e-08 &  4.98e-16 &      \\ 
     &           &   10 &  1.75e-08 &  1.72e-15 &      \\ 
3246 &  3.24e+04 &   10 &           &           & iters  \\ 
 \hdashline 
     &           &    1 &  3.67e-08 &  7.30e-09 &      \\ 
     &           &    2 &  2.19e-09 &  1.05e-15 &      \\ 
     &           &    3 &  2.18e-09 &  6.24e-17 &      \\ 
     &           &    4 &  2.18e-09 &  6.23e-17 &      \\ 
     &           &    5 &  2.18e-09 &  6.22e-17 &      \\ 
     &           &    6 &  2.17e-09 &  6.21e-17 &      \\ 
     &           &    7 &  2.17e-09 &  6.20e-17 &      \\ 
     &           &    8 &  2.17e-09 &  6.20e-17 &      \\ 
     &           &    9 &  2.16e-09 &  6.19e-17 &      \\ 
     &           &   10 &  2.16e-09 &  6.18e-17 &      \\ 
3247 &  3.25e+04 &   10 &           &           & iters  \\ 
 \hdashline 
     &           &    1 &  7.10e-09 &  5.11e-09 &      \\ 
     &           &    2 &  7.09e-09 &  2.03e-16 &      \\ 
     &           &    3 &  7.08e-09 &  2.02e-16 &      \\ 
     &           &    4 &  3.18e-08 &  2.02e-16 &      \\ 
     &           &    5 &  7.11e-09 &  9.07e-16 &      \\ 
     &           &    6 &  7.10e-09 &  2.03e-16 &      \\ 
     &           &    7 &  7.09e-09 &  2.03e-16 &      \\ 
     &           &    8 &  7.08e-09 &  2.02e-16 &      \\ 
     &           &    9 &  3.18e-08 &  2.02e-16 &      \\ 
     &           &   10 &  7.12e-09 &  9.07e-16 &      \\ 
3248 &  3.25e+04 &   10 &           &           & iters  \\ 
 \hdashline 
     &           &    1 &  1.11e-08 &  4.47e-09 &      \\ 
     &           &    2 &  1.11e-08 &  3.17e-16 &      \\ 
     &           &    3 &  1.11e-08 &  3.16e-16 &      \\ 
     &           &    4 &  1.11e-08 &  3.16e-16 &      \\ 
     &           &    5 &  6.66e-08 &  3.16e-16 &      \\ 
     &           &    6 &  5.00e-08 &  1.90e-15 &      \\ 
     &           &    7 &  1.11e-08 &  1.43e-15 &      \\ 
     &           &    8 &  1.10e-08 &  3.16e-16 &      \\ 
     &           &    9 &  6.67e-08 &  3.15e-16 &      \\ 
     &           &   10 &  1.11e-08 &  1.90e-15 &      \\ 
3249 &  3.25e+04 &   10 &           &           & iters  \\ 
 \hdashline 
     &           &    1 &  4.17e-08 &  8.42e-09 &      \\ 
     &           &    2 &  3.60e-08 &  1.19e-15 &      \\ 
     &           &    3 &  4.17e-08 &  1.03e-15 &      \\ 
     &           &    4 &  3.60e-08 &  1.19e-15 &      \\ 
     &           &    5 &  4.17e-08 &  1.03e-15 &      \\ 
     &           &    6 &  3.60e-08 &  1.19e-15 &      \\ 
     &           &    7 &  3.60e-08 &  1.03e-15 &      \\ 
     &           &    8 &  4.18e-08 &  1.03e-15 &      \\ 
     &           &    9 &  3.60e-08 &  1.19e-15 &      \\ 
     &           &   10 &  4.18e-08 &  1.03e-15 &      \\ 
3250 &  3.25e+04 &   10 &           &           & iters  \\ 
 \hdashline 
     &           &    1 &  1.76e-08 &  4.99e-09 &      \\ 
     &           &    2 &  1.75e-08 &  5.01e-16 &      \\ 
     &           &    3 &  1.75e-08 &  5.00e-16 &      \\ 
     &           &    4 &  1.75e-08 &  5.00e-16 &      \\ 
     &           &    5 &  6.02e-08 &  4.99e-16 &      \\ 
     &           &    6 &  1.75e-08 &  1.72e-15 &      \\ 
     &           &    7 &  1.75e-08 &  5.01e-16 &      \\ 
     &           &    8 &  1.75e-08 &  5.00e-16 &      \\ 
     &           &    9 &  6.02e-08 &  4.99e-16 &      \\ 
     &           &   10 &  1.76e-08 &  1.72e-15 &      \\ 
3251 &  3.25e+04 &   10 &           &           & iters  \\ 
 \hdashline 
     &           &    1 &  3.87e-08 &  8.16e-10 &      \\ 
     &           &    2 &  3.91e-08 &  1.10e-15 &      \\ 
     &           &    3 &  3.87e-08 &  1.12e-15 &      \\ 
     &           &    4 &  3.91e-08 &  1.10e-15 &      \\ 
     &           &    5 &  3.87e-08 &  1.12e-15 &      \\ 
     &           &    6 &  3.91e-08 &  1.10e-15 &      \\ 
     &           &    7 &  3.87e-08 &  1.12e-15 &      \\ 
     &           &    8 &  3.91e-08 &  1.10e-15 &      \\ 
     &           &    9 &  3.87e-08 &  1.12e-15 &      \\ 
     &           &   10 &  3.91e-08 &  1.10e-15 &      \\ 
3252 &  3.25e+04 &   10 &           &           & iters  \\ 
 \hdashline 
     &           &    1 &  1.13e-08 &  6.18e-09 &      \\ 
     &           &    2 &  1.12e-08 &  3.21e-16 &      \\ 
     &           &    3 &  1.12e-08 &  3.21e-16 &      \\ 
     &           &    4 &  1.12e-08 &  3.20e-16 &      \\ 
     &           &    5 &  6.65e-08 &  3.20e-16 &      \\ 
     &           &    6 &  1.13e-08 &  1.90e-15 &      \\ 
     &           &    7 &  1.13e-08 &  3.22e-16 &      \\ 
     &           &    8 &  1.13e-08 &  3.22e-16 &      \\ 
     &           &    9 &  1.12e-08 &  3.21e-16 &      \\ 
     &           &   10 &  1.12e-08 &  3.21e-16 &      \\ 
3253 &  3.25e+04 &   10 &           &           & iters  \\ 
 \hdashline 
     &           &    1 &  5.31e-08 &  3.40e-09 &      \\ 
     &           &    2 &  1.41e-08 &  1.51e-15 &      \\ 
     &           &    3 &  6.36e-08 &  4.04e-16 &      \\ 
     &           &    4 &  1.42e-08 &  1.81e-15 &      \\ 
     &           &    5 &  1.42e-08 &  4.06e-16 &      \\ 
     &           &    6 &  1.42e-08 &  4.05e-16 &      \\ 
     &           &    7 &  1.42e-08 &  4.05e-16 &      \\ 
     &           &    8 &  1.41e-08 &  4.04e-16 &      \\ 
     &           &    9 &  6.36e-08 &  4.03e-16 &      \\ 
     &           &   10 &  1.42e-08 &  1.81e-15 &      \\ 
3254 &  3.25e+04 &   10 &           &           & iters  \\ 
 \hdashline 
     &           &    1 &  2.16e-10 &  3.85e-09 &      \\ 
     &           &    2 &  2.15e-10 &  6.15e-18 &      \\ 
     &           &    3 &  2.15e-10 &  6.15e-18 &      \\ 
     &           &    4 &  2.15e-10 &  6.14e-18 &      \\ 
     &           &    5 &  2.14e-10 &  6.13e-18 &      \\ 
     &           &    6 &  2.14e-10 &  6.12e-18 &      \\ 
     &           &    7 &  2.14e-10 &  6.11e-18 &      \\ 
     &           &    8 &  2.13e-10 &  6.10e-18 &      \\ 
     &           &    9 &  2.13e-10 &  6.09e-18 &      \\ 
     &           &   10 &  2.13e-10 &  6.08e-18 &      \\ 
3255 &  3.25e+04 &   10 &           &           & iters  \\ 
 \hdashline 
     &           &    1 &  2.59e-08 &  9.45e-10 &      \\ 
     &           &    2 &  2.58e-08 &  7.38e-16 &      \\ 
     &           &    3 &  5.19e-08 &  7.37e-16 &      \\ 
     &           &    4 &  2.59e-08 &  1.48e-15 &      \\ 
     &           &    5 &  2.58e-08 &  7.38e-16 &      \\ 
     &           &    6 &  5.19e-08 &  7.37e-16 &      \\ 
     &           &    7 &  2.59e-08 &  1.48e-15 &      \\ 
     &           &    8 &  2.58e-08 &  7.38e-16 &      \\ 
     &           &    9 &  5.19e-08 &  7.37e-16 &      \\ 
     &           &   10 &  2.59e-08 &  1.48e-15 &      \\ 
3256 &  3.26e+04 &   10 &           &           & iters  \\ 
 \hdashline 
     &           &    1 &  3.06e-08 &  3.60e-09 &      \\ 
     &           &    2 &  3.05e-08 &  8.73e-16 &      \\ 
     &           &    3 &  4.72e-08 &  8.72e-16 &      \\ 
     &           &    4 &  3.06e-08 &  1.35e-15 &      \\ 
     &           &    5 &  3.05e-08 &  8.72e-16 &      \\ 
     &           &    6 &  4.72e-08 &  8.71e-16 &      \\ 
     &           &    7 &  3.05e-08 &  1.35e-15 &      \\ 
     &           &    8 &  4.72e-08 &  8.72e-16 &      \\ 
     &           &    9 &  3.06e-08 &  1.35e-15 &      \\ 
     &           &   10 &  3.05e-08 &  8.72e-16 &      \\ 
3257 &  3.26e+04 &   10 &           &           & iters  \\ 
 \hdashline 
     &           &    1 &  5.55e-08 &  1.84e-09 &      \\ 
     &           &    2 &  2.23e-08 &  1.58e-15 &      \\ 
     &           &    3 &  2.23e-08 &  6.36e-16 &      \\ 
     &           &    4 &  5.55e-08 &  6.35e-16 &      \\ 
     &           &    5 &  2.23e-08 &  1.58e-15 &      \\ 
     &           &    6 &  2.23e-08 &  6.36e-16 &      \\ 
     &           &    7 &  5.55e-08 &  6.36e-16 &      \\ 
     &           &    8 &  2.23e-08 &  1.58e-15 &      \\ 
     &           &    9 &  2.23e-08 &  6.37e-16 &      \\ 
     &           &   10 &  2.23e-08 &  6.36e-16 &      \\ 
3258 &  3.26e+04 &   10 &           &           & iters  \\ 
 \hdashline 
     &           &    1 &  7.11e-08 &  2.16e-09 &      \\ 
     &           &    2 &  8.44e-08 &  2.03e-15 &      \\ 
     &           &    3 &  7.11e-08 &  2.41e-15 &      \\ 
     &           &    4 &  6.71e-09 &  2.03e-15 &      \\ 
     &           &    5 &  6.70e-09 &  1.91e-16 &      \\ 
     &           &    6 &  6.69e-09 &  1.91e-16 &      \\ 
     &           &    7 &  6.68e-09 &  1.91e-16 &      \\ 
     &           &    8 &  6.67e-09 &  1.91e-16 &      \\ 
     &           &    9 &  6.66e-09 &  1.90e-16 &      \\ 
     &           &   10 &  6.65e-09 &  1.90e-16 &      \\ 
3259 &  3.26e+04 &   10 &           &           & iters  \\ 
 \hdashline 
     &           &    1 &  1.28e-08 &  3.44e-09 &      \\ 
     &           &    2 &  1.28e-08 &  3.65e-16 &      \\ 
     &           &    3 &  1.27e-08 &  3.64e-16 &      \\ 
     &           &    4 &  6.50e-08 &  3.64e-16 &      \\ 
     &           &    5 &  9.05e-08 &  1.85e-15 &      \\ 
     &           &    6 &  6.50e-08 &  2.58e-15 &      \\ 
     &           &    7 &  1.28e-08 &  1.86e-15 &      \\ 
     &           &    8 &  1.28e-08 &  3.65e-16 &      \\ 
     &           &    9 &  1.27e-08 &  3.64e-16 &      \\ 
     &           &   10 &  6.50e-08 &  3.64e-16 &      \\ 
3260 &  3.26e+04 &   10 &           &           & iters  \\ 
 \hdashline 
     &           &    1 &  6.23e-08 &  8.97e-10 &      \\ 
     &           &    2 &  1.55e-08 &  1.78e-15 &      \\ 
     &           &    3 &  1.55e-08 &  4.42e-16 &      \\ 
     &           &    4 &  1.54e-08 &  4.41e-16 &      \\ 
     &           &    5 &  6.23e-08 &  4.41e-16 &      \\ 
     &           &    6 &  1.55e-08 &  1.78e-15 &      \\ 
     &           &    7 &  1.55e-08 &  4.43e-16 &      \\ 
     &           &    8 &  1.55e-08 &  4.42e-16 &      \\ 
     &           &    9 &  1.54e-08 &  4.41e-16 &      \\ 
     &           &   10 &  6.23e-08 &  4.41e-16 &      \\ 
3261 &  3.26e+04 &   10 &           &           & iters  \\ 
 \hdashline 
     &           &    1 &  4.33e-08 &  2.30e-09 &      \\ 
     &           &    2 &  3.45e-08 &  1.23e-15 &      \\ 
     &           &    3 &  4.32e-08 &  9.84e-16 &      \\ 
     &           &    4 &  3.45e-08 &  1.23e-15 &      \\ 
     &           &    5 &  4.32e-08 &  9.85e-16 &      \\ 
     &           &    6 &  3.45e-08 &  1.23e-15 &      \\ 
     &           &    7 &  4.32e-08 &  9.85e-16 &      \\ 
     &           &    8 &  3.45e-08 &  1.23e-15 &      \\ 
     &           &    9 &  4.32e-08 &  9.86e-16 &      \\ 
     &           &   10 &  3.45e-08 &  1.23e-15 &      \\ 
3262 &  3.26e+04 &   10 &           &           & iters  \\ 
 \hdashline 
     &           &    1 &  2.39e-08 &  2.25e-09 &      \\ 
     &           &    2 &  2.38e-08 &  6.81e-16 &      \\ 
     &           &    3 &  5.39e-08 &  6.80e-16 &      \\ 
     &           &    4 &  2.39e-08 &  1.54e-15 &      \\ 
     &           &    5 &  2.38e-08 &  6.81e-16 &      \\ 
     &           &    6 &  5.39e-08 &  6.80e-16 &      \\ 
     &           &    7 &  2.39e-08 &  1.54e-15 &      \\ 
     &           &    8 &  2.38e-08 &  6.81e-16 &      \\ 
     &           &    9 &  5.39e-08 &  6.80e-16 &      \\ 
     &           &   10 &  2.39e-08 &  1.54e-15 &      \\ 
3263 &  3.26e+04 &   10 &           &           & iters  \\ 
 \hdashline 
     &           &    1 &  7.15e-09 &  2.91e-10 &      \\ 
     &           &    2 &  7.14e-09 &  2.04e-16 &      \\ 
     &           &    3 &  7.13e-09 &  2.04e-16 &      \\ 
     &           &    4 &  7.12e-09 &  2.03e-16 &      \\ 
     &           &    5 &  3.17e-08 &  2.03e-16 &      \\ 
     &           &    6 &  7.15e-09 &  9.06e-16 &      \\ 
     &           &    7 &  7.14e-09 &  2.04e-16 &      \\ 
     &           &    8 &  7.13e-09 &  2.04e-16 &      \\ 
     &           &    9 &  7.12e-09 &  2.04e-16 &      \\ 
     &           &   10 &  3.17e-08 &  2.03e-16 &      \\ 
3264 &  3.26e+04 &   10 &           &           & iters  \\ 
 \hdashline 
     &           &    1 &  3.71e-08 &  2.31e-09 &      \\ 
     &           &    2 &  4.07e-08 &  1.06e-15 &      \\ 
     &           &    3 &  3.71e-08 &  1.16e-15 &      \\ 
     &           &    4 &  4.07e-08 &  1.06e-15 &      \\ 
     &           &    5 &  3.71e-08 &  1.16e-15 &      \\ 
     &           &    6 &  4.07e-08 &  1.06e-15 &      \\ 
     &           &    7 &  3.71e-08 &  1.16e-15 &      \\ 
     &           &    8 &  4.07e-08 &  1.06e-15 &      \\ 
     &           &    9 &  3.71e-08 &  1.16e-15 &      \\ 
     &           &   10 &  4.07e-08 &  1.06e-15 &      \\ 
3265 &  3.26e+04 &   10 &           &           & iters  \\ 
 \hdashline 
     &           &    1 &  1.80e-08 &  1.13e-09 &      \\ 
     &           &    2 &  1.80e-08 &  5.15e-16 &      \\ 
     &           &    3 &  1.80e-08 &  5.14e-16 &      \\ 
     &           &    4 &  1.80e-08 &  5.13e-16 &      \\ 
     &           &    5 &  5.98e-08 &  5.12e-16 &      \\ 
     &           &    6 &  1.80e-08 &  1.71e-15 &      \\ 
     &           &    7 &  1.80e-08 &  5.14e-16 &      \\ 
     &           &    8 &  1.80e-08 &  5.13e-16 &      \\ 
     &           &    9 &  5.98e-08 &  5.13e-16 &      \\ 
     &           &   10 &  1.80e-08 &  1.71e-15 &      \\ 
3266 &  3.26e+04 &   10 &           &           & iters  \\ 
 \hdashline 
     &           &    1 &  2.17e-08 &  1.08e-09 &      \\ 
     &           &    2 &  5.60e-08 &  6.20e-16 &      \\ 
     &           &    3 &  2.18e-08 &  1.60e-15 &      \\ 
     &           &    4 &  2.18e-08 &  6.22e-16 &      \\ 
     &           &    5 &  2.17e-08 &  6.21e-16 &      \\ 
     &           &    6 &  5.60e-08 &  6.20e-16 &      \\ 
     &           &    7 &  2.18e-08 &  1.60e-15 &      \\ 
     &           &    8 &  2.17e-08 &  6.21e-16 &      \\ 
     &           &    9 &  5.60e-08 &  6.21e-16 &      \\ 
     &           &   10 &  2.18e-08 &  1.60e-15 &      \\ 
3267 &  3.27e+04 &   10 &           &           & iters  \\ 
 \hdashline 
     &           &    1 &  2.92e-08 &  6.33e-10 &      \\ 
     &           &    2 &  9.67e-09 &  8.34e-16 &      \\ 
     &           &    3 &  9.66e-09 &  2.76e-16 &      \\ 
     &           &    4 &  9.64e-09 &  2.76e-16 &      \\ 
     &           &    5 &  2.92e-08 &  2.75e-16 &      \\ 
     &           &    6 &  9.67e-09 &  8.34e-16 &      \\ 
     &           &    7 &  9.66e-09 &  2.76e-16 &      \\ 
     &           &    8 &  9.64e-09 &  2.76e-16 &      \\ 
     &           &    9 &  2.92e-08 &  2.75e-16 &      \\ 
     &           &   10 &  9.67e-09 &  8.34e-16 &      \\ 
3268 &  3.27e+04 &   10 &           &           & iters  \\ 
 \hdashline 
     &           &    1 &  2.29e-08 &  2.56e-09 &      \\ 
     &           &    2 &  5.48e-08 &  6.54e-16 &      \\ 
     &           &    3 &  2.30e-08 &  1.56e-15 &      \\ 
     &           &    4 &  2.29e-08 &  6.55e-16 &      \\ 
     &           &    5 &  5.48e-08 &  6.54e-16 &      \\ 
     &           &    6 &  2.30e-08 &  1.56e-15 &      \\ 
     &           &    7 &  2.29e-08 &  6.56e-16 &      \\ 
     &           &    8 &  2.29e-08 &  6.55e-16 &      \\ 
     &           &    9 &  5.48e-08 &  6.54e-16 &      \\ 
     &           &   10 &  2.30e-08 &  1.56e-15 &      \\ 
3269 &  3.27e+04 &   10 &           &           & iters  \\ 
 \hdashline 
     &           &    1 &  2.60e-08 &  1.43e-09 &      \\ 
     &           &    2 &  5.17e-08 &  7.43e-16 &      \\ 
     &           &    3 &  2.61e-08 &  1.47e-15 &      \\ 
     &           &    4 &  2.60e-08 &  7.44e-16 &      \\ 
     &           &    5 &  5.17e-08 &  7.43e-16 &      \\ 
     &           &    6 &  2.61e-08 &  1.47e-15 &      \\ 
     &           &    7 &  2.60e-08 &  7.44e-16 &      \\ 
     &           &    8 &  5.17e-08 &  7.43e-16 &      \\ 
     &           &    9 &  2.61e-08 &  1.47e-15 &      \\ 
     &           &   10 &  2.60e-08 &  7.44e-16 &      \\ 
3270 &  3.27e+04 &   10 &           &           & iters  \\ 
 \hdashline 
     &           &    1 &  4.31e-08 &  5.21e-10 &      \\ 
     &           &    2 &  3.46e-08 &  1.23e-15 &      \\ 
     &           &    3 &  4.31e-08 &  9.89e-16 &      \\ 
     &           &    4 &  3.47e-08 &  1.23e-15 &      \\ 
     &           &    5 &  4.31e-08 &  9.89e-16 &      \\ 
     &           &    6 &  3.47e-08 &  1.23e-15 &      \\ 
     &           &    7 &  3.46e-08 &  9.89e-16 &      \\ 
     &           &    8 &  4.31e-08 &  9.88e-16 &      \\ 
     &           &    9 &  3.46e-08 &  1.23e-15 &      \\ 
     &           &   10 &  4.31e-08 &  9.88e-16 &      \\ 
3271 &  3.27e+04 &   10 &           &           & iters  \\ 
 \hdashline 
     &           &    1 &  5.63e-08 &  4.04e-09 &      \\ 
     &           &    2 &  2.14e-08 &  1.61e-15 &      \\ 
     &           &    3 &  2.14e-08 &  6.12e-16 &      \\ 
     &           &    4 &  2.14e-08 &  6.11e-16 &      \\ 
     &           &    5 &  5.63e-08 &  6.10e-16 &      \\ 
     &           &    6 &  2.14e-08 &  1.61e-15 &      \\ 
     &           &    7 &  2.14e-08 &  6.12e-16 &      \\ 
     &           &    8 &  5.63e-08 &  6.11e-16 &      \\ 
     &           &    9 &  2.14e-08 &  1.61e-15 &      \\ 
     &           &   10 &  2.14e-08 &  6.12e-16 &      \\ 
3272 &  3.27e+04 &   10 &           &           & iters  \\ 
 \hdashline 
     &           &    1 &  2.60e-09 &  5.87e-09 &      \\ 
     &           &    2 &  2.60e-09 &  7.42e-17 &      \\ 
     &           &    3 &  2.59e-09 &  7.41e-17 &      \\ 
     &           &    4 &  2.59e-09 &  7.40e-17 &      \\ 
     &           &    5 &  2.59e-09 &  7.39e-17 &      \\ 
     &           &    6 &  2.58e-09 &  7.38e-17 &      \\ 
     &           &    7 &  2.58e-09 &  7.37e-17 &      \\ 
     &           &    8 &  2.57e-09 &  7.36e-17 &      \\ 
     &           &    9 &  2.57e-09 &  7.35e-17 &      \\ 
     &           &   10 &  2.57e-09 &  7.34e-17 &      \\ 
3273 &  3.27e+04 &   10 &           &           & iters  \\ 
 \hdashline 
     &           &    1 &  6.68e-09 &  3.18e-09 &      \\ 
     &           &    2 &  6.67e-09 &  1.91e-16 &      \\ 
     &           &    3 &  6.66e-09 &  1.90e-16 &      \\ 
     &           &    4 &  7.10e-08 &  1.90e-16 &      \\ 
     &           &    5 &  8.44e-08 &  2.03e-15 &      \\ 
     &           &    6 &  7.11e-08 &  2.41e-15 &      \\ 
     &           &    7 &  6.73e-09 &  2.03e-15 &      \\ 
     &           &    8 &  6.72e-09 &  1.92e-16 &      \\ 
     &           &    9 &  6.71e-09 &  1.92e-16 &      \\ 
     &           &   10 &  6.70e-09 &  1.92e-16 &      \\ 
3274 &  3.27e+04 &   10 &           &           & iters  \\ 
 \hdashline 
     &           &    1 &  5.24e-08 &  1.24e-09 &      \\ 
     &           &    2 &  2.54e-08 &  1.50e-15 &      \\ 
     &           &    3 &  2.53e-08 &  7.24e-16 &      \\ 
     &           &    4 &  5.24e-08 &  7.23e-16 &      \\ 
     &           &    5 &  2.54e-08 &  1.50e-15 &      \\ 
     &           &    6 &  2.53e-08 &  7.24e-16 &      \\ 
     &           &    7 &  5.24e-08 &  7.23e-16 &      \\ 
     &           &    8 &  2.54e-08 &  1.50e-15 &      \\ 
     &           &    9 &  2.53e-08 &  7.24e-16 &      \\ 
     &           &   10 &  5.24e-08 &  7.23e-16 &      \\ 
3275 &  3.27e+04 &   10 &           &           & iters  \\ 
 \hdashline 
     &           &    1 &  2.45e-08 &  3.42e-09 &      \\ 
     &           &    2 &  2.45e-08 &  7.00e-16 &      \\ 
     &           &    3 &  2.45e-08 &  6.99e-16 &      \\ 
     &           &    4 &  5.33e-08 &  6.98e-16 &      \\ 
     &           &    5 &  2.45e-08 &  1.52e-15 &      \\ 
     &           &    6 &  2.45e-08 &  6.99e-16 &      \\ 
     &           &    7 &  5.33e-08 &  6.98e-16 &      \\ 
     &           &    8 &  2.45e-08 &  1.52e-15 &      \\ 
     &           &    9 &  2.45e-08 &  7.00e-16 &      \\ 
     &           &   10 &  5.32e-08 &  6.99e-16 &      \\ 
3276 &  3.28e+04 &   10 &           &           & iters  \\ 
 \hdashline 
     &           &    1 &  1.43e-09 &  8.48e-09 &      \\ 
     &           &    2 &  1.43e-09 &  4.08e-17 &      \\ 
     &           &    3 &  1.43e-09 &  4.07e-17 &      \\ 
     &           &    4 &  1.42e-09 &  4.07e-17 &      \\ 
     &           &    5 &  1.42e-09 &  4.06e-17 &      \\ 
     &           &    6 &  1.42e-09 &  4.06e-17 &      \\ 
     &           &    7 &  1.42e-09 &  4.05e-17 &      \\ 
     &           &    8 &  1.41e-09 &  4.04e-17 &      \\ 
     &           &    9 &  1.41e-09 &  4.04e-17 &      \\ 
     &           &   10 &  1.41e-09 &  4.03e-17 &      \\ 
3277 &  3.28e+04 &   10 &           &           & iters  \\ 
 \hdashline 
     &           &    1 &  1.29e-08 &  1.03e-08 &      \\ 
     &           &    2 &  1.29e-08 &  3.69e-16 &      \\ 
     &           &    3 &  6.48e-08 &  3.69e-16 &      \\ 
     &           &    4 &  1.30e-08 &  1.85e-15 &      \\ 
     &           &    5 &  1.30e-08 &  3.71e-16 &      \\ 
     &           &    6 &  1.30e-08 &  3.70e-16 &      \\ 
     &           &    7 &  1.29e-08 &  3.70e-16 &      \\ 
     &           &    8 &  1.29e-08 &  3.69e-16 &      \\ 
     &           &    9 &  6.48e-08 &  3.69e-16 &      \\ 
     &           &   10 &  1.30e-08 &  1.85e-15 &      \\ 
3278 &  3.28e+04 &   10 &           &           & iters  \\ 
 \hdashline 
     &           &    1 &  5.50e-08 &  8.28e-10 &      \\ 
     &           &    2 &  2.28e-08 &  1.57e-15 &      \\ 
     &           &    3 &  2.28e-08 &  6.51e-16 &      \\ 
     &           &    4 &  2.27e-08 &  6.50e-16 &      \\ 
     &           &    5 &  5.50e-08 &  6.49e-16 &      \\ 
     &           &    6 &  2.28e-08 &  1.57e-15 &      \\ 
     &           &    7 &  2.28e-08 &  6.50e-16 &      \\ 
     &           &    8 &  5.50e-08 &  6.49e-16 &      \\ 
     &           &    9 &  2.28e-08 &  1.57e-15 &      \\ 
     &           &   10 &  2.28e-08 &  6.51e-16 &      \\ 
3279 &  3.28e+04 &   10 &           &           & iters  \\ 
 \hdashline 
     &           &    1 &  5.16e-08 &  1.53e-08 &      \\ 
     &           &    2 &  2.62e-08 &  1.47e-15 &      \\ 
     &           &    3 &  2.62e-08 &  7.48e-16 &      \\ 
     &           &    4 &  5.16e-08 &  7.47e-16 &      \\ 
     &           &    5 &  2.62e-08 &  1.47e-15 &      \\ 
     &           &    6 &  2.62e-08 &  7.48e-16 &      \\ 
     &           &    7 &  5.16e-08 &  7.47e-16 &      \\ 
     &           &    8 &  2.62e-08 &  1.47e-15 &      \\ 
     &           &    9 &  2.62e-08 &  7.48e-16 &      \\ 
     &           &   10 &  5.16e-08 &  7.47e-16 &      \\ 
3280 &  3.28e+04 &   10 &           &           & iters  \\ 
 \hdashline 
     &           &    1 &  1.63e-08 &  1.28e-08 &      \\ 
     &           &    2 &  2.25e-08 &  4.66e-16 &      \\ 
     &           &    3 &  1.64e-08 &  6.43e-16 &      \\ 
     &           &    4 &  2.25e-08 &  4.67e-16 &      \\ 
     &           &    5 &  1.64e-08 &  6.43e-16 &      \\ 
     &           &    6 &  1.63e-08 &  4.67e-16 &      \\ 
     &           &    7 &  2.25e-08 &  4.66e-16 &      \\ 
     &           &    8 &  1.63e-08 &  6.43e-16 &      \\ 
     &           &    9 &  2.25e-08 &  4.67e-16 &      \\ 
     &           &   10 &  1.64e-08 &  6.43e-16 &      \\ 
3281 &  3.28e+04 &   10 &           &           & iters  \\ 
 \hdashline 
     &           &    1 &  7.36e-09 &  1.10e-09 &      \\ 
     &           &    2 &  7.35e-09 &  2.10e-16 &      \\ 
     &           &    3 &  7.34e-09 &  2.10e-16 &      \\ 
     &           &    4 &  7.33e-09 &  2.10e-16 &      \\ 
     &           &    5 &  7.04e-08 &  2.09e-16 &      \\ 
     &           &    6 &  8.51e-08 &  2.01e-15 &      \\ 
     &           &    7 &  7.04e-08 &  2.43e-15 &      \\ 
     &           &    8 &  7.40e-09 &  2.01e-15 &      \\ 
     &           &    9 &  7.39e-09 &  2.11e-16 &      \\ 
     &           &   10 &  7.38e-09 &  2.11e-16 &      \\ 
3282 &  3.28e+04 &   10 &           &           & iters  \\ 
 \hdashline 
     &           &    1 &  1.37e-08 &  2.24e-09 &      \\ 
     &           &    2 &  6.40e-08 &  3.90e-16 &      \\ 
     &           &    3 &  1.37e-08 &  1.83e-15 &      \\ 
     &           &    4 &  1.37e-08 &  3.92e-16 &      \\ 
     &           &    5 &  1.37e-08 &  3.92e-16 &      \\ 
     &           &    6 &  1.37e-08 &  3.91e-16 &      \\ 
     &           &    7 &  1.37e-08 &  3.91e-16 &      \\ 
     &           &    8 &  6.40e-08 &  3.90e-16 &      \\ 
     &           &    9 &  1.37e-08 &  1.83e-15 &      \\ 
     &           &   10 &  1.37e-08 &  3.92e-16 &      \\ 
3283 &  3.28e+04 &   10 &           &           & iters  \\ 
 \hdashline 
     &           &    1 &  2.68e-09 &  2.21e-09 &      \\ 
     &           &    2 &  3.62e-08 &  7.64e-17 &      \\ 
     &           &    3 &  2.73e-09 &  1.03e-15 &      \\ 
     &           &    4 &  2.72e-09 &  7.78e-17 &      \\ 
     &           &    5 &  2.72e-09 &  7.77e-17 &      \\ 
     &           &    6 &  2.71e-09 &  7.76e-17 &      \\ 
     &           &    7 &  2.71e-09 &  7.75e-17 &      \\ 
     &           &    8 &  2.71e-09 &  7.73e-17 &      \\ 
     &           &    9 &  2.70e-09 &  7.72e-17 &      \\ 
     &           &   10 &  2.70e-09 &  7.71e-17 &      \\ 
3284 &  3.28e+04 &   10 &           &           & iters  \\ 
 \hdashline 
     &           &    1 &  4.31e-08 &  6.13e-09 &      \\ 
     &           &    2 &  3.47e-08 &  1.23e-15 &      \\ 
     &           &    3 &  3.46e-08 &  9.90e-16 &      \\ 
     &           &    4 &  4.31e-08 &  9.88e-16 &      \\ 
     &           &    5 &  3.46e-08 &  1.23e-15 &      \\ 
     &           &    6 &  4.31e-08 &  9.89e-16 &      \\ 
     &           &    7 &  3.47e-08 &  1.23e-15 &      \\ 
     &           &    8 &  4.31e-08 &  9.89e-16 &      \\ 
     &           &    9 &  3.47e-08 &  1.23e-15 &      \\ 
     &           &   10 &  4.31e-08 &  9.89e-16 &      \\ 
3285 &  3.28e+04 &   10 &           &           & iters  \\ 
 \hdashline 
     &           &    1 &  2.35e-08 &  4.51e-09 &      \\ 
     &           &    2 &  2.34e-08 &  6.70e-16 &      \\ 
     &           &    3 &  2.34e-08 &  6.69e-16 &      \\ 
     &           &    4 &  5.43e-08 &  6.68e-16 &      \\ 
     &           &    5 &  2.35e-08 &  1.55e-15 &      \\ 
     &           &    6 &  2.34e-08 &  6.69e-16 &      \\ 
     &           &    7 &  5.43e-08 &  6.68e-16 &      \\ 
     &           &    8 &  2.35e-08 &  1.55e-15 &      \\ 
     &           &    9 &  2.34e-08 &  6.70e-16 &      \\ 
     &           &   10 &  2.34e-08 &  6.69e-16 &      \\ 
3286 &  3.28e+04 &   10 &           &           & iters  \\ 
 \hdashline 
     &           &    1 &  6.69e-08 &  3.39e-09 &      \\ 
     &           &    2 &  1.09e-08 &  1.91e-15 &      \\ 
     &           &    3 &  1.09e-08 &  3.12e-16 &      \\ 
     &           &    4 &  1.09e-08 &  3.12e-16 &      \\ 
     &           &    5 &  1.09e-08 &  3.11e-16 &      \\ 
     &           &    6 &  1.09e-08 &  3.11e-16 &      \\ 
     &           &    7 &  6.68e-08 &  3.10e-16 &      \\ 
     &           &    8 &  1.09e-08 &  1.91e-15 &      \\ 
     &           &    9 &  1.09e-08 &  3.12e-16 &      \\ 
     &           &   10 &  1.09e-08 &  3.12e-16 &      \\ 
3287 &  3.29e+04 &   10 &           &           & iters  \\ 
 \hdashline 
     &           &    1 &  1.54e-08 &  2.61e-09 &      \\ 
     &           &    2 &  1.54e-08 &  4.39e-16 &      \\ 
     &           &    3 &  6.24e-08 &  4.38e-16 &      \\ 
     &           &    4 &  1.54e-08 &  1.78e-15 &      \\ 
     &           &    5 &  1.54e-08 &  4.40e-16 &      \\ 
     &           &    6 &  1.54e-08 &  4.40e-16 &      \\ 
     &           &    7 &  1.54e-08 &  4.39e-16 &      \\ 
     &           &    8 &  6.23e-08 &  4.38e-16 &      \\ 
     &           &    9 &  1.54e-08 &  1.78e-15 &      \\ 
     &           &   10 &  1.54e-08 &  4.40e-16 &      \\ 
3288 &  3.29e+04 &   10 &           &           & iters  \\ 
 \hdashline 
     &           &    1 &  1.83e-08 &  5.97e-09 &      \\ 
     &           &    2 &  1.82e-08 &  5.21e-16 &      \\ 
     &           &    3 &  5.95e-08 &  5.20e-16 &      \\ 
     &           &    4 &  1.83e-08 &  1.70e-15 &      \\ 
     &           &    5 &  1.83e-08 &  5.22e-16 &      \\ 
     &           &    6 &  1.82e-08 &  5.21e-16 &      \\ 
     &           &    7 &  5.95e-08 &  5.20e-16 &      \\ 
     &           &    8 &  1.83e-08 &  1.70e-15 &      \\ 
     &           &    9 &  1.83e-08 &  5.22e-16 &      \\ 
     &           &   10 &  1.82e-08 &  5.21e-16 &      \\ 
3289 &  3.29e+04 &   10 &           &           & iters  \\ 
 \hdashline 
     &           &    1 &  4.34e-08 &  1.34e-08 &      \\ 
     &           &    2 &  3.43e-08 &  1.24e-15 &      \\ 
     &           &    3 &  4.34e-08 &  9.80e-16 &      \\ 
     &           &    4 &  3.43e-08 &  1.24e-15 &      \\ 
     &           &    5 &  3.43e-08 &  9.80e-16 &      \\ 
     &           &    6 &  4.35e-08 &  9.79e-16 &      \\ 
     &           &    7 &  3.43e-08 &  1.24e-15 &      \\ 
     &           &    8 &  4.34e-08 &  9.79e-16 &      \\ 
     &           &    9 &  3.43e-08 &  1.24e-15 &      \\ 
     &           &   10 &  4.34e-08 &  9.79e-16 &      \\ 
3290 &  3.29e+04 &   10 &           &           & iters  \\ 
 \hdashline 
     &           &    1 &  4.65e-08 &  1.20e-08 &      \\ 
     &           &    2 &  3.12e-08 &  1.33e-15 &      \\ 
     &           &    3 &  3.12e-08 &  8.91e-16 &      \\ 
     &           &    4 &  4.65e-08 &  8.90e-16 &      \\ 
     &           &    5 &  3.12e-08 &  1.33e-15 &      \\ 
     &           &    6 &  4.65e-08 &  8.91e-16 &      \\ 
     &           &    7 &  3.12e-08 &  1.33e-15 &      \\ 
     &           &    8 &  3.12e-08 &  8.91e-16 &      \\ 
     &           &    9 &  4.65e-08 &  8.90e-16 &      \\ 
     &           &   10 &  3.12e-08 &  1.33e-15 &      \\ 
3291 &  3.29e+04 &   10 &           &           & iters  \\ 
 \hdashline 
     &           &    1 &  4.38e-08 &  7.91e-09 &      \\ 
     &           &    2 &  4.37e-08 &  1.25e-15 &      \\ 
     &           &    3 &  3.40e-08 &  1.25e-15 &      \\ 
     &           &    4 &  3.40e-08 &  9.71e-16 &      \\ 
     &           &    5 &  4.38e-08 &  9.69e-16 &      \\ 
     &           &    6 &  3.40e-08 &  1.25e-15 &      \\ 
     &           &    7 &  4.38e-08 &  9.70e-16 &      \\ 
     &           &    8 &  3.40e-08 &  1.25e-15 &      \\ 
     &           &    9 &  4.38e-08 &  9.70e-16 &      \\ 
     &           &   10 &  3.40e-08 &  1.25e-15 &      \\ 
3292 &  3.29e+04 &   10 &           &           & iters  \\ 
 \hdashline 
     &           &    1 &  4.62e-09 &  3.48e-09 &      \\ 
     &           &    2 &  4.61e-09 &  1.32e-16 &      \\ 
     &           &    3 &  4.61e-09 &  1.32e-16 &      \\ 
     &           &    4 &  4.60e-09 &  1.31e-16 &      \\ 
     &           &    5 &  4.59e-09 &  1.31e-16 &      \\ 
     &           &    6 &  4.59e-09 &  1.31e-16 &      \\ 
     &           &    7 &  4.58e-09 &  1.31e-16 &      \\ 
     &           &    8 &  7.31e-08 &  1.31e-16 &      \\ 
     &           &    9 &  8.24e-08 &  2.09e-15 &      \\ 
     &           &   10 &  7.31e-08 &  2.35e-15 &      \\ 
3293 &  3.29e+04 &   10 &           &           & iters  \\ 
 \hdashline 
     &           &    1 &  7.75e-08 &  3.50e-11 &      \\ 
     &           &    2 &  7.80e-08 &  2.21e-15 &      \\ 
     &           &    3 &  7.75e-08 &  2.23e-15 &      \\ 
     &           &    4 &  7.80e-08 &  2.21e-15 &      \\ 
     &           &    5 &  7.75e-08 &  2.23e-15 &      \\ 
     &           &    6 &  7.80e-08 &  2.21e-15 &      \\ 
     &           &    7 &  7.75e-08 &  2.23e-15 &      \\ 
     &           &    8 &  7.80e-08 &  2.21e-15 &      \\ 
     &           &    9 &  7.75e-08 &  2.23e-15 &      \\ 
     &           &   10 &  7.80e-08 &  2.21e-15 &      \\ 
3294 &  3.29e+04 &   10 &           &           & iters  \\ 
 \hdashline 
     &           &    1 &  6.57e-08 &  4.69e-10 &      \\ 
     &           &    2 &  1.21e-08 &  1.88e-15 &      \\ 
     &           &    3 &  1.20e-08 &  3.44e-16 &      \\ 
     &           &    4 &  1.20e-08 &  3.44e-16 &      \\ 
     &           &    5 &  1.20e-08 &  3.43e-16 &      \\ 
     &           &    6 &  1.20e-08 &  3.43e-16 &      \\ 
     &           &    7 &  1.20e-08 &  3.42e-16 &      \\ 
     &           &    8 &  6.57e-08 &  3.42e-16 &      \\ 
     &           &    9 &  1.20e-08 &  1.88e-15 &      \\ 
     &           &   10 &  1.20e-08 &  3.44e-16 &      \\ 
3295 &  3.29e+04 &   10 &           &           & iters  \\ 
 \hdashline 
     &           &    1 &  6.13e-08 &  1.44e-09 &      \\ 
     &           &    2 &  1.65e-08 &  1.75e-15 &      \\ 
     &           &    3 &  1.65e-08 &  4.71e-16 &      \\ 
     &           &    4 &  1.65e-08 &  4.71e-16 &      \\ 
     &           &    5 &  6.12e-08 &  4.70e-16 &      \\ 
     &           &    6 &  1.65e-08 &  1.75e-15 &      \\ 
     &           &    7 &  1.65e-08 &  4.72e-16 &      \\ 
     &           &    8 &  1.65e-08 &  4.71e-16 &      \\ 
     &           &    9 &  1.65e-08 &  4.70e-16 &      \\ 
     &           &   10 &  6.13e-08 &  4.70e-16 &      \\ 
3296 &  3.30e+04 &   10 &           &           & iters  \\ 
 \hdashline 
     &           &    1 &  6.57e-08 &  5.32e-09 &      \\ 
     &           &    2 &  1.21e-08 &  1.87e-15 &      \\ 
     &           &    3 &  1.21e-08 &  3.46e-16 &      \\ 
     &           &    4 &  1.21e-08 &  3.46e-16 &      \\ 
     &           &    5 &  6.56e-08 &  3.45e-16 &      \\ 
     &           &    6 &  8.99e-08 &  1.87e-15 &      \\ 
     &           &    7 &  6.56e-08 &  2.56e-15 &      \\ 
     &           &    8 &  1.21e-08 &  1.87e-15 &      \\ 
     &           &    9 &  1.21e-08 &  3.46e-16 &      \\ 
     &           &   10 &  1.21e-08 &  3.46e-16 &      \\ 
3297 &  3.30e+04 &   10 &           &           & iters  \\ 
 \hdashline 
     &           &    1 &  5.87e-08 &  8.99e-09 &      \\ 
     &           &    2 &  1.90e-08 &  1.68e-15 &      \\ 
     &           &    3 &  1.90e-08 &  5.43e-16 &      \\ 
     &           &    4 &  1.90e-08 &  5.42e-16 &      \\ 
     &           &    5 &  5.87e-08 &  5.42e-16 &      \\ 
     &           &    6 &  1.90e-08 &  1.68e-15 &      \\ 
     &           &    7 &  1.90e-08 &  5.43e-16 &      \\ 
     &           &    8 &  1.90e-08 &  5.43e-16 &      \\ 
     &           &    9 &  5.87e-08 &  5.42e-16 &      \\ 
     &           &   10 &  1.90e-08 &  1.68e-15 &      \\ 
3298 &  3.30e+04 &   10 &           &           & iters  \\ 
 \hdashline 
     &           &    1 &  1.79e-08 &  9.10e-09 &      \\ 
     &           &    2 &  1.79e-08 &  5.10e-16 &      \\ 
     &           &    3 &  1.78e-08 &  5.10e-16 &      \\ 
     &           &    4 &  5.99e-08 &  5.09e-16 &      \\ 
     &           &    5 &  1.79e-08 &  1.71e-15 &      \\ 
     &           &    6 &  1.79e-08 &  5.11e-16 &      \\ 
     &           &    7 &  1.78e-08 &  5.10e-16 &      \\ 
     &           &    8 &  5.99e-08 &  5.09e-16 &      \\ 
     &           &    9 &  1.79e-08 &  1.71e-15 &      \\ 
     &           &   10 &  1.79e-08 &  5.11e-16 &      \\ 
3299 &  3.30e+04 &   10 &           &           & iters  \\ 
 \hdashline 
     &           &    1 &  3.26e-08 &  6.38e-09 &      \\ 
     &           &    2 &  4.51e-08 &  9.31e-16 &      \\ 
     &           &    3 &  3.26e-08 &  1.29e-15 &      \\ 
     &           &    4 &  4.51e-08 &  9.31e-16 &      \\ 
     &           &    5 &  3.26e-08 &  1.29e-15 &      \\ 
     &           &    6 &  3.26e-08 &  9.32e-16 &      \\ 
     &           &    7 &  4.51e-08 &  9.30e-16 &      \\ 
     &           &    8 &  3.26e-08 &  1.29e-15 &      \\ 
     &           &    9 &  4.51e-08 &  9.31e-16 &      \\ 
     &           &   10 &  3.26e-08 &  1.29e-15 &      \\ 
3300 &  3.30e+04 &   10 &           &           & iters  \\ 
 \hdashline 
     &           &    1 &  5.06e-09 &  5.60e-09 &      \\ 
     &           &    2 &  5.05e-09 &  1.44e-16 &      \\ 
     &           &    3 &  5.04e-09 &  1.44e-16 &      \\ 
     &           &    4 &  5.04e-09 &  1.44e-16 &      \\ 
     &           &    5 &  5.03e-09 &  1.44e-16 &      \\ 
     &           &    6 &  5.02e-09 &  1.44e-16 &      \\ 
     &           &    7 &  5.01e-09 &  1.43e-16 &      \\ 
     &           &    8 &  5.01e-09 &  1.43e-16 &      \\ 
     &           &    9 &  5.00e-09 &  1.43e-16 &      \\ 
     &           &   10 &  4.99e-09 &  1.43e-16 &      \\ 
3301 &  3.30e+04 &   10 &           &           & iters  \\ 
 \hdashline 
     &           &    1 &  3.03e-08 &  3.76e-09 &      \\ 
     &           &    2 &  3.03e-08 &  8.66e-16 &      \\ 
     &           &    3 &  4.74e-08 &  8.65e-16 &      \\ 
     &           &    4 &  3.03e-08 &  1.35e-15 &      \\ 
     &           &    5 &  4.74e-08 &  8.65e-16 &      \\ 
     &           &    6 &  3.03e-08 &  1.35e-15 &      \\ 
     &           &    7 &  3.03e-08 &  8.66e-16 &      \\ 
     &           &    8 &  4.74e-08 &  8.65e-16 &      \\ 
     &           &    9 &  3.03e-08 &  1.35e-15 &      \\ 
     &           &   10 &  4.74e-08 &  8.66e-16 &      \\ 
3302 &  3.30e+04 &   10 &           &           & iters  \\ 
 \hdashline 
     &           &    1 &  3.73e-08 &  3.77e-09 &      \\ 
     &           &    2 &  3.73e-08 &  1.07e-15 &      \\ 
     &           &    3 &  4.05e-08 &  1.06e-15 &      \\ 
     &           &    4 &  3.73e-08 &  1.15e-15 &      \\ 
     &           &    5 &  4.05e-08 &  1.06e-15 &      \\ 
     &           &    6 &  3.73e-08 &  1.15e-15 &      \\ 
     &           &    7 &  4.05e-08 &  1.06e-15 &      \\ 
     &           &    8 &  3.73e-08 &  1.15e-15 &      \\ 
     &           &    9 &  4.05e-08 &  1.06e-15 &      \\ 
     &           &   10 &  3.73e-08 &  1.15e-15 &      \\ 
3303 &  3.30e+04 &   10 &           &           & iters  \\ 
 \hdashline 
     &           &    1 &  4.85e-10 &  6.11e-09 &      \\ 
     &           &    2 &  4.84e-10 &  1.38e-17 &      \\ 
     &           &    3 &  4.84e-10 &  1.38e-17 &      \\ 
     &           &    4 &  4.83e-10 &  1.38e-17 &      \\ 
     &           &    5 &  4.82e-10 &  1.38e-17 &      \\ 
     &           &    6 &  4.82e-10 &  1.38e-17 &      \\ 
     &           &    7 &  4.81e-10 &  1.37e-17 &      \\ 
     &           &    8 &  4.80e-10 &  1.37e-17 &      \\ 
     &           &    9 &  4.80e-10 &  1.37e-17 &      \\ 
     &           &   10 &  4.79e-10 &  1.37e-17 &      \\ 
3304 &  3.30e+04 &   10 &           &           & iters  \\ 
 \hdashline 
     &           &    1 &  4.31e-08 &  1.50e-09 &      \\ 
     &           &    2 &  3.46e-08 &  1.23e-15 &      \\ 
     &           &    3 &  3.46e-08 &  9.88e-16 &      \\ 
     &           &    4 &  4.32e-08 &  9.87e-16 &      \\ 
     &           &    5 &  3.46e-08 &  1.23e-15 &      \\ 
     &           &    6 &  4.32e-08 &  9.87e-16 &      \\ 
     &           &    7 &  3.46e-08 &  1.23e-15 &      \\ 
     &           &    8 &  4.31e-08 &  9.87e-16 &      \\ 
     &           &    9 &  3.46e-08 &  1.23e-15 &      \\ 
     &           &   10 &  4.31e-08 &  9.88e-16 &      \\ 
3305 &  3.30e+04 &   10 &           &           & iters  \\ 
 \hdashline 
     &           &    1 &  1.58e-08 &  1.59e-09 &      \\ 
     &           &    2 &  1.57e-08 &  4.50e-16 &      \\ 
     &           &    3 &  2.31e-08 &  4.49e-16 &      \\ 
     &           &    4 &  1.57e-08 &  6.60e-16 &      \\ 
     &           &    5 &  1.57e-08 &  4.49e-16 &      \\ 
     &           &    6 &  2.31e-08 &  4.49e-16 &      \\ 
     &           &    7 &  1.57e-08 &  6.61e-16 &      \\ 
     &           &    8 &  2.31e-08 &  4.49e-16 &      \\ 
     &           &    9 &  1.57e-08 &  6.60e-16 &      \\ 
     &           &   10 &  1.57e-08 &  4.49e-16 &      \\ 
3306 &  3.30e+04 &   10 &           &           & iters  \\ 
 \hdashline 
     &           &    1 &  2.01e-08 &  4.21e-09 &      \\ 
     &           &    2 &  1.88e-08 &  5.73e-16 &      \\ 
     &           &    3 &  2.01e-08 &  5.36e-16 &      \\ 
     &           &    4 &  1.88e-08 &  5.73e-16 &      \\ 
     &           &    5 &  2.01e-08 &  5.36e-16 &      \\ 
     &           &    6 &  1.88e-08 &  5.73e-16 &      \\ 
     &           &    7 &  2.01e-08 &  5.36e-16 &      \\ 
     &           &    8 &  1.88e-08 &  5.73e-16 &      \\ 
     &           &    9 &  2.01e-08 &  5.36e-16 &      \\ 
     &           &   10 &  1.88e-08 &  5.73e-16 &      \\ 
3307 &  3.31e+04 &   10 &           &           & iters  \\ 
 \hdashline 
     &           &    1 &  4.57e-08 &  5.10e-09 &      \\ 
     &           &    2 &  3.21e-08 &  1.30e-15 &      \\ 
     &           &    3 &  3.21e-08 &  9.16e-16 &      \\ 
     &           &    4 &  4.57e-08 &  9.15e-16 &      \\ 
     &           &    5 &  3.21e-08 &  1.30e-15 &      \\ 
     &           &    6 &  4.57e-08 &  9.15e-16 &      \\ 
     &           &    7 &  3.21e-08 &  1.30e-15 &      \\ 
     &           &    8 &  3.20e-08 &  9.16e-16 &      \\ 
     &           &    9 &  4.57e-08 &  9.15e-16 &      \\ 
     &           &   10 &  3.21e-08 &  1.30e-15 &      \\ 
3308 &  3.31e+04 &   10 &           &           & iters  \\ 
 \hdashline 
     &           &    1 &  1.51e-08 &  4.15e-09 &      \\ 
     &           &    2 &  1.51e-08 &  4.31e-16 &      \\ 
     &           &    3 &  1.51e-08 &  4.31e-16 &      \\ 
     &           &    4 &  1.50e-08 &  4.30e-16 &      \\ 
     &           &    5 &  6.27e-08 &  4.29e-16 &      \\ 
     &           &    6 &  1.51e-08 &  1.79e-15 &      \\ 
     &           &    7 &  1.51e-08 &  4.31e-16 &      \\ 
     &           &    8 &  1.51e-08 &  4.31e-16 &      \\ 
     &           &    9 &  1.50e-08 &  4.30e-16 &      \\ 
     &           &   10 &  6.27e-08 &  4.29e-16 &      \\ 
3309 &  3.31e+04 &   10 &           &           & iters  \\ 
 \hdashline 
     &           &    1 &  4.52e-08 &  5.38e-10 &      \\ 
     &           &    2 &  3.26e-08 &  1.29e-15 &      \\ 
     &           &    3 &  4.52e-08 &  9.29e-16 &      \\ 
     &           &    4 &  3.26e-08 &  1.29e-15 &      \\ 
     &           &    5 &  3.25e-08 &  9.30e-16 &      \\ 
     &           &    6 &  4.52e-08 &  9.28e-16 &      \\ 
     &           &    7 &  3.25e-08 &  1.29e-15 &      \\ 
     &           &    8 &  4.52e-08 &  9.29e-16 &      \\ 
     &           &    9 &  3.26e-08 &  1.29e-15 &      \\ 
     &           &   10 &  4.52e-08 &  9.29e-16 &      \\ 
3310 &  3.31e+04 &   10 &           &           & iters  \\ 
 \hdashline 
     &           &    1 &  8.78e-08 &  3.43e-09 &      \\ 
     &           &    2 &  6.77e-08 &  2.51e-15 &      \\ 
     &           &    3 &  1.01e-08 &  1.93e-15 &      \\ 
     &           &    4 &  1.01e-08 &  2.88e-16 &      \\ 
     &           &    5 &  1.01e-08 &  2.88e-16 &      \\ 
     &           &    6 &  1.01e-08 &  2.88e-16 &      \\ 
     &           &    7 &  1.00e-08 &  2.87e-16 &      \\ 
     &           &    8 &  6.77e-08 &  2.87e-16 &      \\ 
     &           &    9 &  1.01e-08 &  1.93e-15 &      \\ 
     &           &   10 &  1.01e-08 &  2.89e-16 &      \\ 
3311 &  3.31e+04 &   10 &           &           & iters  \\ 
 \hdashline 
     &           &    1 &  3.69e-09 &  1.88e-09 &      \\ 
     &           &    2 &  3.68e-09 &  1.05e-16 &      \\ 
     &           &    3 &  3.68e-09 &  1.05e-16 &      \\ 
     &           &    4 &  3.67e-09 &  1.05e-16 &      \\ 
     &           &    5 &  3.67e-09 &  1.05e-16 &      \\ 
     &           &    6 &  3.66e-09 &  1.05e-16 &      \\ 
     &           &    7 &  3.66e-09 &  1.04e-16 &      \\ 
     &           &    8 &  3.65e-09 &  1.04e-16 &      \\ 
     &           &    9 &  3.64e-09 &  1.04e-16 &      \\ 
     &           &   10 &  3.64e-09 &  1.04e-16 &      \\ 
3312 &  3.31e+04 &   10 &           &           & iters  \\ 
 \hdashline 
     &           &    1 &  1.99e-08 &  2.35e-10 &      \\ 
     &           &    2 &  1.99e-08 &  5.68e-16 &      \\ 
     &           &    3 &  1.98e-08 &  5.67e-16 &      \\ 
     &           &    4 &  5.79e-08 &  5.66e-16 &      \\ 
     &           &    5 &  1.99e-08 &  1.65e-15 &      \\ 
     &           &    6 &  1.99e-08 &  5.68e-16 &      \\ 
     &           &    7 &  5.78e-08 &  5.67e-16 &      \\ 
     &           &    8 &  1.99e-08 &  1.65e-15 &      \\ 
     &           &    9 &  1.99e-08 &  5.69e-16 &      \\ 
     &           &   10 &  1.99e-08 &  5.68e-16 &      \\ 
3313 &  3.31e+04 &   10 &           &           & iters  \\ 
 \hdashline 
     &           &    1 &  9.90e-08 &  1.06e-09 &      \\ 
     &           &    2 &  5.65e-08 &  2.83e-15 &      \\ 
     &           &    3 &  2.13e-08 &  1.61e-15 &      \\ 
     &           &    4 &  2.12e-08 &  6.07e-16 &      \\ 
     &           &    5 &  5.65e-08 &  6.06e-16 &      \\ 
     &           &    6 &  2.13e-08 &  1.61e-15 &      \\ 
     &           &    7 &  2.13e-08 &  6.08e-16 &      \\ 
     &           &    8 &  5.65e-08 &  6.07e-16 &      \\ 
     &           &    9 &  2.13e-08 &  1.61e-15 &      \\ 
     &           &   10 &  2.13e-08 &  6.08e-16 &      \\ 
3314 &  3.31e+04 &   10 &           &           & iters  \\ 
 \hdashline 
     &           &    1 &  6.24e-09 &  1.13e-09 &      \\ 
     &           &    2 &  6.23e-09 &  1.78e-16 &      \\ 
     &           &    3 &  6.22e-09 &  1.78e-16 &      \\ 
     &           &    4 &  7.15e-08 &  1.77e-16 &      \\ 
     &           &    5 &  8.40e-08 &  2.04e-15 &      \\ 
     &           &    6 &  7.15e-08 &  2.40e-15 &      \\ 
     &           &    7 &  6.29e-09 &  2.04e-15 &      \\ 
     &           &    8 &  6.29e-09 &  1.80e-16 &      \\ 
     &           &    9 &  6.28e-09 &  1.79e-16 &      \\ 
     &           &   10 &  6.27e-09 &  1.79e-16 &      \\ 
3315 &  3.31e+04 &   10 &           &           & iters  \\ 
 \hdashline 
     &           &    1 &  7.37e-09 &  2.17e-09 &      \\ 
     &           &    2 &  7.36e-09 &  2.10e-16 &      \\ 
     &           &    3 &  7.35e-09 &  2.10e-16 &      \\ 
     &           &    4 &  7.34e-09 &  2.10e-16 &      \\ 
     &           &    5 &  7.33e-09 &  2.10e-16 &      \\ 
     &           &    6 &  7.32e-09 &  2.09e-16 &      \\ 
     &           &    7 &  7.31e-09 &  2.09e-16 &      \\ 
     &           &    8 &  7.30e-09 &  2.09e-16 &      \\ 
     &           &    9 &  7.04e-08 &  2.08e-16 &      \\ 
     &           &   10 &  7.39e-09 &  2.01e-15 &      \\ 
3316 &  3.32e+04 &   10 &           &           & iters  \\ 
 \hdashline 
     &           &    1 &  2.59e-08 &  3.81e-09 &      \\ 
     &           &    2 &  2.59e-08 &  7.40e-16 &      \\ 
     &           &    3 &  5.18e-08 &  7.39e-16 &      \\ 
     &           &    4 &  2.59e-08 &  1.48e-15 &      \\ 
     &           &    5 &  2.59e-08 &  7.40e-16 &      \\ 
     &           &    6 &  5.18e-08 &  7.39e-16 &      \\ 
     &           &    7 &  2.59e-08 &  1.48e-15 &      \\ 
     &           &    8 &  2.59e-08 &  7.40e-16 &      \\ 
     &           &    9 &  5.18e-08 &  7.39e-16 &      \\ 
     &           &   10 &  2.59e-08 &  1.48e-15 &      \\ 
3317 &  3.32e+04 &   10 &           &           & iters  \\ 
 \hdashline 
     &           &    1 &  2.84e-08 &  1.13e-09 &      \\ 
     &           &    2 &  2.84e-08 &  8.12e-16 &      \\ 
     &           &    3 &  4.93e-08 &  8.11e-16 &      \\ 
     &           &    4 &  2.84e-08 &  1.41e-15 &      \\ 
     &           &    5 &  2.84e-08 &  8.12e-16 &      \\ 
     &           &    6 &  4.93e-08 &  8.10e-16 &      \\ 
     &           &    7 &  2.84e-08 &  1.41e-15 &      \\ 
     &           &    8 &  4.93e-08 &  8.11e-16 &      \\ 
     &           &    9 &  2.85e-08 &  1.41e-15 &      \\ 
     &           &   10 &  2.84e-08 &  8.12e-16 &      \\ 
3318 &  3.32e+04 &   10 &           &           & iters  \\ 
 \hdashline 
     &           &    1 &  3.91e-08 &  4.07e-09 &      \\ 
     &           &    2 &  3.87e-08 &  1.12e-15 &      \\ 
     &           &    3 &  3.91e-08 &  1.10e-15 &      \\ 
     &           &    4 &  3.87e-08 &  1.12e-15 &      \\ 
     &           &    5 &  3.91e-08 &  1.10e-15 &      \\ 
     &           &    6 &  3.87e-08 &  1.12e-15 &      \\ 
     &           &    7 &  3.91e-08 &  1.10e-15 &      \\ 
     &           &    8 &  3.87e-08 &  1.12e-15 &      \\ 
     &           &    9 &  3.91e-08 &  1.10e-15 &      \\ 
     &           &   10 &  3.87e-08 &  1.12e-15 &      \\ 
3319 &  3.32e+04 &   10 &           &           & iters  \\ 
 \hdashline 
     &           &    1 &  1.13e-08 &  1.53e-09 &      \\ 
     &           &    2 &  1.13e-08 &  3.22e-16 &      \\ 
     &           &    3 &  1.13e-08 &  3.22e-16 &      \\ 
     &           &    4 &  1.13e-08 &  3.22e-16 &      \\ 
     &           &    5 &  6.65e-08 &  3.21e-16 &      \\ 
     &           &    6 &  1.13e-08 &  1.90e-15 &      \\ 
     &           &    7 &  1.13e-08 &  3.23e-16 &      \\ 
     &           &    8 &  1.13e-08 &  3.23e-16 &      \\ 
     &           &    9 &  1.13e-08 &  3.22e-16 &      \\ 
     &           &   10 &  1.13e-08 &  3.22e-16 &      \\ 
3320 &  3.32e+04 &   10 &           &           & iters  \\ 
 \hdashline 
     &           &    1 &  2.28e-08 &  2.43e-09 &      \\ 
     &           &    2 &  5.49e-08 &  6.51e-16 &      \\ 
     &           &    3 &  2.28e-08 &  1.57e-15 &      \\ 
     &           &    4 &  2.28e-08 &  6.52e-16 &      \\ 
     &           &    5 &  5.49e-08 &  6.51e-16 &      \\ 
     &           &    6 &  2.29e-08 &  1.57e-15 &      \\ 
     &           &    7 &  2.28e-08 &  6.52e-16 &      \\ 
     &           &    8 &  2.28e-08 &  6.51e-16 &      \\ 
     &           &    9 &  5.49e-08 &  6.50e-16 &      \\ 
     &           &   10 &  2.28e-08 &  1.57e-15 &      \\ 
3321 &  3.32e+04 &   10 &           &           & iters  \\ 
 \hdashline 
     &           &    1 &  2.59e-08 &  3.83e-09 &      \\ 
     &           &    2 &  5.18e-08 &  7.40e-16 &      \\ 
     &           &    3 &  2.60e-08 &  1.48e-15 &      \\ 
     &           &    4 &  2.59e-08 &  7.41e-16 &      \\ 
     &           &    5 &  5.18e-08 &  7.40e-16 &      \\ 
     &           &    6 &  2.60e-08 &  1.48e-15 &      \\ 
     &           &    7 &  2.59e-08 &  7.41e-16 &      \\ 
     &           &    8 &  5.18e-08 &  7.40e-16 &      \\ 
     &           &    9 &  2.60e-08 &  1.48e-15 &      \\ 
     &           &   10 &  2.59e-08 &  7.41e-16 &      \\ 
3322 &  3.32e+04 &   10 &           &           & iters  \\ 
 \hdashline 
     &           &    1 &  4.38e-08 &  1.97e-09 &      \\ 
     &           &    2 &  3.40e-08 &  1.25e-15 &      \\ 
     &           &    3 &  4.38e-08 &  9.69e-16 &      \\ 
     &           &    4 &  3.40e-08 &  1.25e-15 &      \\ 
     &           &    5 &  4.38e-08 &  9.70e-16 &      \\ 
     &           &    6 &  3.40e-08 &  1.25e-15 &      \\ 
     &           &    7 &  3.39e-08 &  9.70e-16 &      \\ 
     &           &    8 &  4.38e-08 &  9.69e-16 &      \\ 
     &           &    9 &  3.40e-08 &  1.25e-15 &      \\ 
     &           &   10 &  4.38e-08 &  9.69e-16 &      \\ 
3323 &  3.32e+04 &   10 &           &           & iters  \\ 
 \hdashline 
     &           &    1 &  1.13e-08 &  1.50e-10 &      \\ 
     &           &    2 &  1.13e-08 &  3.23e-16 &      \\ 
     &           &    3 &  6.64e-08 &  3.23e-16 &      \\ 
     &           &    4 &  8.91e-08 &  1.90e-15 &      \\ 
     &           &    5 &  6.64e-08 &  2.54e-15 &      \\ 
     &           &    6 &  1.13e-08 &  1.90e-15 &      \\ 
     &           &    7 &  1.13e-08 &  3.24e-16 &      \\ 
     &           &    8 &  1.13e-08 &  3.23e-16 &      \\ 
     &           &    9 &  1.13e-08 &  3.23e-16 &      \\ 
     &           &   10 &  6.64e-08 &  3.22e-16 &      \\ 
3324 &  3.32e+04 &   10 &           &           & iters  \\ 
 \hdashline 
     &           &    1 &  7.04e-08 &  1.30e-09 &      \\ 
     &           &    2 &  7.42e-09 &  2.01e-15 &      \\ 
     &           &    3 &  7.40e-09 &  2.12e-16 &      \\ 
     &           &    4 &  7.39e-09 &  2.11e-16 &      \\ 
     &           &    5 &  7.38e-09 &  2.11e-16 &      \\ 
     &           &    6 &  7.37e-09 &  2.11e-16 &      \\ 
     &           &    7 &  7.36e-09 &  2.10e-16 &      \\ 
     &           &    8 &  7.35e-09 &  2.10e-16 &      \\ 
     &           &    9 &  7.03e-08 &  2.10e-16 &      \\ 
     &           &   10 &  8.51e-08 &  2.01e-15 &      \\ 
3325 &  3.32e+04 &   10 &           &           & iters  \\ 
 \hdashline 
     &           &    1 &  1.14e-08 &  1.61e-10 &      \\ 
     &           &    2 &  1.14e-08 &  3.24e-16 &      \\ 
     &           &    3 &  1.13e-08 &  3.24e-16 &      \\ 
     &           &    4 &  1.13e-08 &  3.24e-16 &      \\ 
     &           &    5 &  1.13e-08 &  3.23e-16 &      \\ 
     &           &    6 &  1.13e-08 &  3.23e-16 &      \\ 
     &           &    7 &  6.64e-08 &  3.22e-16 &      \\ 
     &           &    8 &  1.14e-08 &  1.90e-15 &      \\ 
     &           &    9 &  1.13e-08 &  3.24e-16 &      \\ 
     &           &   10 &  1.13e-08 &  3.24e-16 &      \\ 
3326 &  3.32e+04 &   10 &           &           & iters  \\ 
 \hdashline 
     &           &    1 &  4.18e-08 &  1.29e-09 &      \\ 
     &           &    2 &  3.60e-08 &  1.19e-15 &      \\ 
     &           &    3 &  4.18e-08 &  1.03e-15 &      \\ 
     &           &    4 &  3.60e-08 &  1.19e-15 &      \\ 
     &           &    5 &  3.59e-08 &  1.03e-15 &      \\ 
     &           &    6 &  4.18e-08 &  1.03e-15 &      \\ 
     &           &    7 &  3.59e-08 &  1.19e-15 &      \\ 
     &           &    8 &  4.18e-08 &  1.03e-15 &      \\ 
     &           &    9 &  3.60e-08 &  1.19e-15 &      \\ 
     &           &   10 &  4.18e-08 &  1.03e-15 &      \\ 
3327 &  3.33e+04 &   10 &           &           & iters  \\ 
 \hdashline 
     &           &    1 &  9.43e-09 &  9.92e-09 &      \\ 
     &           &    2 &  9.41e-09 &  2.69e-16 &      \\ 
     &           &    3 &  9.40e-09 &  2.69e-16 &      \\ 
     &           &    4 &  9.38e-09 &  2.68e-16 &      \\ 
     &           &    5 &  9.37e-09 &  2.68e-16 &      \\ 
     &           &    6 &  9.36e-09 &  2.67e-16 &      \\ 
     &           &    7 &  2.95e-08 &  2.67e-16 &      \\ 
     &           &    8 &  9.39e-09 &  8.42e-16 &      \\ 
     &           &    9 &  9.37e-09 &  2.68e-16 &      \\ 
     &           &   10 &  9.36e-09 &  2.68e-16 &      \\ 
3328 &  3.33e+04 &   10 &           &           & iters  \\ 
 \hdashline 
     &           &    1 &  3.88e-08 &  1.10e-08 &      \\ 
     &           &    2 &  3.90e-08 &  1.11e-15 &      \\ 
     &           &    3 &  3.88e-08 &  1.11e-15 &      \\ 
     &           &    4 &  3.90e-08 &  1.11e-15 &      \\ 
     &           &    5 &  3.88e-08 &  1.11e-15 &      \\ 
     &           &    6 &  3.90e-08 &  1.11e-15 &      \\ 
     &           &    7 &  3.88e-08 &  1.11e-15 &      \\ 
     &           &    8 &  3.90e-08 &  1.11e-15 &      \\ 
     &           &    9 &  3.88e-08 &  1.11e-15 &      \\ 
     &           &   10 &  3.90e-08 &  1.11e-15 &      \\ 
3329 &  3.33e+04 &   10 &           &           & iters  \\ 
 \hdashline 
     &           &    1 &  4.99e-08 &  9.98e-11 &      \\ 
     &           &    2 &  2.79e-08 &  1.42e-15 &      \\ 
     &           &    3 &  2.78e-08 &  7.95e-16 &      \\ 
     &           &    4 &  4.99e-08 &  7.94e-16 &      \\ 
     &           &    5 &  2.78e-08 &  1.42e-15 &      \\ 
     &           &    6 &  2.78e-08 &  7.95e-16 &      \\ 
     &           &    7 &  4.99e-08 &  7.94e-16 &      \\ 
     &           &    8 &  2.78e-08 &  1.42e-15 &      \\ 
     &           &    9 &  2.78e-08 &  7.95e-16 &      \\ 
     &           &   10 &  4.99e-08 &  7.93e-16 &      \\ 
3330 &  3.33e+04 &   10 &           &           & iters  \\ 
 \hdashline 
     &           &    1 &  3.83e-08 &  2.58e-09 &      \\ 
     &           &    2 &  3.94e-08 &  1.09e-15 &      \\ 
     &           &    3 &  3.83e-08 &  1.13e-15 &      \\ 
     &           &    4 &  3.94e-08 &  1.09e-15 &      \\ 
     &           &    5 &  3.83e-08 &  1.13e-15 &      \\ 
     &           &    6 &  3.94e-08 &  1.09e-15 &      \\ 
     &           &    7 &  3.83e-08 &  1.13e-15 &      \\ 
     &           &    8 &  3.94e-08 &  1.09e-15 &      \\ 
     &           &    9 &  3.83e-08 &  1.13e-15 &      \\ 
     &           &   10 &  3.94e-08 &  1.09e-15 &      \\ 
3331 &  3.33e+04 &   10 &           &           & iters  \\ 
 \hdashline 
     &           &    1 &  4.95e-08 &  5.90e-09 &      \\ 
     &           &    2 &  2.82e-08 &  1.41e-15 &      \\ 
     &           &    3 &  2.82e-08 &  8.06e-16 &      \\ 
     &           &    4 &  4.95e-08 &  8.04e-16 &      \\ 
     &           &    5 &  2.82e-08 &  1.41e-15 &      \\ 
     &           &    6 &  2.82e-08 &  8.05e-16 &      \\ 
     &           &    7 &  4.96e-08 &  8.04e-16 &      \\ 
     &           &    8 &  2.82e-08 &  1.41e-15 &      \\ 
     &           &    9 &  4.95e-08 &  8.05e-16 &      \\ 
     &           &   10 &  2.82e-08 &  1.41e-15 &      \\ 
3332 &  3.33e+04 &   10 &           &           & iters  \\ 
 \hdashline 
     &           &    1 &  3.88e-08 &  9.42e-09 &      \\ 
     &           &    2 &  3.90e-08 &  1.11e-15 &      \\ 
     &           &    3 &  3.88e-08 &  1.11e-15 &      \\ 
     &           &    4 &  3.90e-08 &  1.11e-15 &      \\ 
     &           &    5 &  3.88e-08 &  1.11e-15 &      \\ 
     &           &    6 &  3.90e-08 &  1.11e-15 &      \\ 
     &           &    7 &  3.88e-08 &  1.11e-15 &      \\ 
     &           &    8 &  3.90e-08 &  1.11e-15 &      \\ 
     &           &    9 &  3.88e-08 &  1.11e-15 &      \\ 
     &           &   10 &  3.90e-08 &  1.11e-15 &      \\ 
3333 &  3.33e+04 &   10 &           &           & iters  \\ 
 \hdashline 
     &           &    1 &  6.36e-09 &  9.94e-10 &      \\ 
     &           &    2 &  6.35e-09 &  1.82e-16 &      \\ 
     &           &    3 &  6.34e-09 &  1.81e-16 &      \\ 
     &           &    4 &  6.33e-09 &  1.81e-16 &      \\ 
     &           &    5 &  6.33e-09 &  1.81e-16 &      \\ 
     &           &    6 &  7.14e-08 &  1.81e-16 &      \\ 
     &           &    7 &  4.53e-08 &  2.04e-15 &      \\ 
     &           &    8 &  6.35e-09 &  1.29e-15 &      \\ 
     &           &    9 &  6.34e-09 &  1.81e-16 &      \\ 
     &           &   10 &  6.34e-09 &  1.81e-16 &      \\ 
3334 &  3.33e+04 &   10 &           &           & iters  \\ 
 \hdashline 
     &           &    1 &  9.58e-09 &  1.24e-08 &      \\ 
     &           &    2 &  9.56e-09 &  2.73e-16 &      \\ 
     &           &    3 &  9.55e-09 &  2.73e-16 &      \\ 
     &           &    4 &  9.54e-09 &  2.73e-16 &      \\ 
     &           &    5 &  9.52e-09 &  2.72e-16 &      \\ 
     &           &    6 &  9.51e-09 &  2.72e-16 &      \\ 
     &           &    7 &  9.49e-09 &  2.71e-16 &      \\ 
     &           &    8 &  6.82e-08 &  2.71e-16 &      \\ 
     &           &    9 &  9.58e-09 &  1.95e-15 &      \\ 
     &           &   10 &  9.56e-09 &  2.73e-16 &      \\ 
3335 &  3.33e+04 &   10 &           &           & iters  \\ 
 \hdashline 
     &           &    1 &  4.55e-08 &  1.03e-08 &      \\ 
     &           &    2 &  3.23e-08 &  1.30e-15 &      \\ 
     &           &    3 &  3.22e-08 &  9.21e-16 &      \\ 
     &           &    4 &  4.55e-08 &  9.20e-16 &      \\ 
     &           &    5 &  3.22e-08 &  1.30e-15 &      \\ 
     &           &    6 &  4.55e-08 &  9.20e-16 &      \\ 
     &           &    7 &  3.23e-08 &  1.30e-15 &      \\ 
     &           &    8 &  3.22e-08 &  9.21e-16 &      \\ 
     &           &    9 &  4.55e-08 &  9.19e-16 &      \\ 
     &           &   10 &  3.22e-08 &  1.30e-15 &      \\ 
3336 &  3.34e+04 &   10 &           &           & iters  \\ 
 \hdashline 
     &           &    1 &  4.46e-08 &  9.26e-10 &      \\ 
     &           &    2 &  3.32e-08 &  1.27e-15 &      \\ 
     &           &    3 &  4.46e-08 &  9.46e-16 &      \\ 
     &           &    4 &  3.32e-08 &  1.27e-15 &      \\ 
     &           &    5 &  3.31e-08 &  9.47e-16 &      \\ 
     &           &    6 &  4.46e-08 &  9.46e-16 &      \\ 
     &           &    7 &  3.31e-08 &  1.27e-15 &      \\ 
     &           &    8 &  4.46e-08 &  9.46e-16 &      \\ 
     &           &    9 &  3.32e-08 &  1.27e-15 &      \\ 
     &           &   10 &  4.46e-08 &  9.46e-16 &      \\ 
3337 &  3.34e+04 &   10 &           &           & iters  \\ 
 \hdashline 
     &           &    1 &  6.17e-08 &  1.38e-09 &      \\ 
     &           &    2 &  1.61e-08 &  1.76e-15 &      \\ 
     &           &    3 &  1.61e-08 &  4.60e-16 &      \\ 
     &           &    4 &  1.61e-08 &  4.59e-16 &      \\ 
     &           &    5 &  6.16e-08 &  4.58e-16 &      \\ 
     &           &    6 &  1.61e-08 &  1.76e-15 &      \\ 
     &           &    7 &  1.61e-08 &  4.60e-16 &      \\ 
     &           &    8 &  1.61e-08 &  4.60e-16 &      \\ 
     &           &    9 &  1.61e-08 &  4.59e-16 &      \\ 
     &           &   10 &  6.17e-08 &  4.58e-16 &      \\ 
3338 &  3.34e+04 &   10 &           &           & iters  \\ 
 \hdashline 
     &           &    1 &  7.27e-09 &  5.45e-09 &      \\ 
     &           &    2 &  7.26e-09 &  2.07e-16 &      \\ 
     &           &    3 &  7.25e-09 &  2.07e-16 &      \\ 
     &           &    4 &  7.24e-09 &  2.07e-16 &      \\ 
     &           &    5 &  7.23e-09 &  2.07e-16 &      \\ 
     &           &    6 &  7.21e-09 &  2.06e-16 &      \\ 
     &           &    7 &  7.20e-09 &  2.06e-16 &      \\ 
     &           &    8 &  7.05e-08 &  2.06e-16 &      \\ 
     &           &    9 &  8.50e-08 &  2.01e-15 &      \\ 
     &           &   10 &  7.05e-08 &  2.43e-15 &      \\ 
3339 &  3.34e+04 &   10 &           &           & iters  \\ 
 \hdashline 
     &           &    1 &  4.09e-08 &  1.74e-09 &      \\ 
     &           &    2 &  3.68e-08 &  1.17e-15 &      \\ 
     &           &    3 &  4.09e-08 &  1.05e-15 &      \\ 
     &           &    4 &  3.68e-08 &  1.17e-15 &      \\ 
     &           &    5 &  4.09e-08 &  1.05e-15 &      \\ 
     &           &    6 &  3.68e-08 &  1.17e-15 &      \\ 
     &           &    7 &  4.09e-08 &  1.05e-15 &      \\ 
     &           &    8 &  3.68e-08 &  1.17e-15 &      \\ 
     &           &    9 &  4.09e-08 &  1.05e-15 &      \\ 
     &           &   10 &  3.68e-08 &  1.17e-15 &      \\ 
3340 &  3.34e+04 &   10 &           &           & iters  \\ 
 \hdashline 
     &           &    1 &  1.91e-08 &  1.60e-09 &      \\ 
     &           &    2 &  5.86e-08 &  5.46e-16 &      \\ 
     &           &    3 &  1.92e-08 &  1.67e-15 &      \\ 
     &           &    4 &  1.92e-08 &  5.47e-16 &      \\ 
     &           &    5 &  1.91e-08 &  5.47e-16 &      \\ 
     &           &    6 &  5.86e-08 &  5.46e-16 &      \\ 
     &           &    7 &  1.92e-08 &  1.67e-15 &      \\ 
     &           &    8 &  1.92e-08 &  5.47e-16 &      \\ 
     &           &    9 &  1.91e-08 &  5.47e-16 &      \\ 
     &           &   10 &  5.86e-08 &  5.46e-16 &      \\ 
3341 &  3.34e+04 &   10 &           &           & iters  \\ 
 \hdashline 
     &           &    1 &  4.11e-08 &  2.94e-09 &      \\ 
     &           &    2 &  3.66e-08 &  1.17e-15 &      \\ 
     &           &    3 &  4.11e-08 &  1.05e-15 &      \\ 
     &           &    4 &  3.67e-08 &  1.17e-15 &      \\ 
     &           &    5 &  4.11e-08 &  1.05e-15 &      \\ 
     &           &    6 &  3.67e-08 &  1.17e-15 &      \\ 
     &           &    7 &  4.11e-08 &  1.05e-15 &      \\ 
     &           &    8 &  3.67e-08 &  1.17e-15 &      \\ 
     &           &    9 &  4.11e-08 &  1.05e-15 &      \\ 
     &           &   10 &  3.67e-08 &  1.17e-15 &      \\ 
3342 &  3.34e+04 &   10 &           &           & iters  \\ 
 \hdashline 
     &           &    1 &  1.12e-08 &  1.09e-10 &      \\ 
     &           &    2 &  6.66e-08 &  3.18e-16 &      \\ 
     &           &    3 &  1.12e-08 &  1.90e-15 &      \\ 
     &           &    4 &  1.12e-08 &  3.21e-16 &      \\ 
     &           &    5 &  1.12e-08 &  3.20e-16 &      \\ 
     &           &    6 &  1.12e-08 &  3.20e-16 &      \\ 
     &           &    7 &  1.12e-08 &  3.19e-16 &      \\ 
     &           &    8 &  6.65e-08 &  3.19e-16 &      \\ 
     &           &    9 &  8.89e-08 &  1.90e-15 &      \\ 
     &           &   10 &  6.66e-08 &  2.54e-15 &      \\ 
3343 &  3.34e+04 &   10 &           &           & iters  \\ 
 \hdashline 
     &           &    1 &  3.38e-08 &  4.91e-09 &      \\ 
     &           &    2 &  3.38e-08 &  9.66e-16 &      \\ 
     &           &    3 &  3.37e-08 &  9.64e-16 &      \\ 
     &           &    4 &  4.40e-08 &  9.63e-16 &      \\ 
     &           &    5 &  3.38e-08 &  1.26e-15 &      \\ 
     &           &    6 &  4.40e-08 &  9.63e-16 &      \\ 
     &           &    7 &  3.38e-08 &  1.26e-15 &      \\ 
     &           &    8 &  4.40e-08 &  9.64e-16 &      \\ 
     &           &    9 &  3.38e-08 &  1.25e-15 &      \\ 
     &           &   10 &  4.40e-08 &  9.64e-16 &      \\ 
3344 &  3.34e+04 &   10 &           &           & iters  \\ 
 \hdashline 
     &           &    1 &  1.33e-08 &  4.99e-09 &      \\ 
     &           &    2 &  1.33e-08 &  3.79e-16 &      \\ 
     &           &    3 &  1.32e-08 &  3.78e-16 &      \\ 
     &           &    4 &  1.32e-08 &  3.78e-16 &      \\ 
     &           &    5 &  1.32e-08 &  3.77e-16 &      \\ 
     &           &    6 &  6.45e-08 &  3.77e-16 &      \\ 
     &           &    7 &  1.33e-08 &  1.84e-15 &      \\ 
     &           &    8 &  1.33e-08 &  3.79e-16 &      \\ 
     &           &    9 &  1.32e-08 &  3.78e-16 &      \\ 
     &           &   10 &  1.32e-08 &  3.78e-16 &      \\ 
3345 &  3.34e+04 &   10 &           &           & iters  \\ 
 \hdashline 
     &           &    1 &  5.94e-08 &  3.63e-09 &      \\ 
     &           &    2 &  1.84e-08 &  1.69e-15 &      \\ 
     &           &    3 &  5.93e-08 &  5.25e-16 &      \\ 
     &           &    4 &  1.85e-08 &  1.69e-15 &      \\ 
     &           &    5 &  1.84e-08 &  5.27e-16 &      \\ 
     &           &    6 &  1.84e-08 &  5.26e-16 &      \\ 
     &           &    7 &  5.93e-08 &  5.25e-16 &      \\ 
     &           &    8 &  1.85e-08 &  1.69e-15 &      \\ 
     &           &    9 &  1.84e-08 &  5.27e-16 &      \\ 
     &           &   10 &  1.84e-08 &  5.26e-16 &      \\ 
3346 &  3.34e+04 &   10 &           &           & iters  \\ 
 \hdashline 
     &           &    1 &  2.28e-08 &  4.13e-09 &      \\ 
     &           &    2 &  2.28e-08 &  6.51e-16 &      \\ 
     &           &    3 &  5.49e-08 &  6.50e-16 &      \\ 
     &           &    4 &  2.28e-08 &  1.57e-15 &      \\ 
     &           &    5 &  2.28e-08 &  6.52e-16 &      \\ 
     &           &    6 &  5.49e-08 &  6.51e-16 &      \\ 
     &           &    7 &  2.28e-08 &  1.57e-15 &      \\ 
     &           &    8 &  2.28e-08 &  6.52e-16 &      \\ 
     &           &    9 &  2.28e-08 &  6.51e-16 &      \\ 
     &           &   10 &  5.49e-08 &  6.50e-16 &      \\ 
3347 &  3.35e+04 &   10 &           &           & iters  \\ 
 \hdashline 
     &           &    1 &  2.21e-08 &  8.30e-10 &      \\ 
     &           &    2 &  5.56e-08 &  6.32e-16 &      \\ 
     &           &    3 &  2.22e-08 &  1.59e-15 &      \\ 
     &           &    4 &  2.21e-08 &  6.33e-16 &      \\ 
     &           &    5 &  2.21e-08 &  6.32e-16 &      \\ 
     &           &    6 &  5.56e-08 &  6.31e-16 &      \\ 
     &           &    7 &  2.22e-08 &  1.59e-15 &      \\ 
     &           &    8 &  2.21e-08 &  6.33e-16 &      \\ 
     &           &    9 &  5.56e-08 &  6.32e-16 &      \\ 
     &           &   10 &  2.22e-08 &  1.59e-15 &      \\ 
3348 &  3.35e+04 &   10 &           &           & iters  \\ 
 \hdashline 
     &           &    1 &  5.65e-08 &  5.37e-09 &      \\ 
     &           &    2 &  2.12e-08 &  1.61e-15 &      \\ 
     &           &    3 &  2.12e-08 &  6.06e-16 &      \\ 
     &           &    4 &  2.12e-08 &  6.05e-16 &      \\ 
     &           &    5 &  5.65e-08 &  6.04e-16 &      \\ 
     &           &    6 &  2.12e-08 &  1.61e-15 &      \\ 
     &           &    7 &  2.12e-08 &  6.06e-16 &      \\ 
     &           &    8 &  2.12e-08 &  6.05e-16 &      \\ 
     &           &    9 &  5.66e-08 &  6.04e-16 &      \\ 
     &           &   10 &  2.12e-08 &  1.61e-15 &      \\ 
3349 &  3.35e+04 &   10 &           &           & iters  \\ 
 \hdashline 
     &           &    1 &  2.35e-08 &  4.53e-09 &      \\ 
     &           &    2 &  5.42e-08 &  6.70e-16 &      \\ 
     &           &    3 &  2.35e-08 &  1.55e-15 &      \\ 
     &           &    4 &  2.35e-08 &  6.72e-16 &      \\ 
     &           &    5 &  5.42e-08 &  6.71e-16 &      \\ 
     &           &    6 &  2.35e-08 &  1.55e-15 &      \\ 
     &           &    7 &  2.35e-08 &  6.72e-16 &      \\ 
     &           &    8 &  2.35e-08 &  6.71e-16 &      \\ 
     &           &    9 &  5.42e-08 &  6.70e-16 &      \\ 
     &           &   10 &  2.35e-08 &  1.55e-15 &      \\ 
3350 &  3.35e+04 &   10 &           &           & iters  \\ 
 \hdashline 
     &           &    1 &  1.74e-08 &  6.65e-09 &      \\ 
     &           &    2 &  6.03e-08 &  4.96e-16 &      \\ 
     &           &    3 &  1.74e-08 &  1.72e-15 &      \\ 
     &           &    4 &  1.74e-08 &  4.98e-16 &      \\ 
     &           &    5 &  1.74e-08 &  4.97e-16 &      \\ 
     &           &    6 &  1.74e-08 &  4.96e-16 &      \\ 
     &           &    7 &  6.04e-08 &  4.95e-16 &      \\ 
     &           &    8 &  1.74e-08 &  1.72e-15 &      \\ 
     &           &    9 &  1.74e-08 &  4.97e-16 &      \\ 
     &           &   10 &  1.74e-08 &  4.97e-16 &      \\ 
3351 &  3.35e+04 &   10 &           &           & iters  \\ 
 \hdashline 
     &           &    1 &  2.84e-09 &  1.51e-08 &      \\ 
     &           &    2 &  2.83e-09 &  8.10e-17 &      \\ 
     &           &    3 &  2.83e-09 &  8.08e-17 &      \\ 
     &           &    4 &  2.82e-09 &  8.07e-17 &      \\ 
     &           &    5 &  2.82e-09 &  8.06e-17 &      \\ 
     &           &    6 &  2.82e-09 &  8.05e-17 &      \\ 
     &           &    7 &  2.81e-09 &  8.04e-17 &      \\ 
     &           &    8 &  2.81e-09 &  8.03e-17 &      \\ 
     &           &    9 &  2.80e-09 &  8.02e-17 &      \\ 
     &           &   10 &  2.80e-09 &  8.00e-17 &      \\ 
3352 &  3.35e+04 &   10 &           &           & iters  \\ 
 \hdashline 
     &           &    1 &  5.32e-08 &  1.38e-08 &      \\ 
     &           &    2 &  2.45e-08 &  1.52e-15 &      \\ 
     &           &    3 &  2.45e-08 &  7.00e-16 &      \\ 
     &           &    4 &  5.32e-08 &  6.99e-16 &      \\ 
     &           &    5 &  2.45e-08 &  1.52e-15 &      \\ 
     &           &    6 &  2.45e-08 &  7.00e-16 &      \\ 
     &           &    7 &  2.45e-08 &  6.99e-16 &      \\ 
     &           &    8 &  5.33e-08 &  6.98e-16 &      \\ 
     &           &    9 &  2.45e-08 &  1.52e-15 &      \\ 
     &           &   10 &  2.45e-08 &  6.99e-16 &      \\ 
3353 &  3.35e+04 &   10 &           &           & iters  \\ 
 \hdashline 
     &           &    1 &  5.24e-08 &  5.53e-09 &      \\ 
     &           &    2 &  2.53e-08 &  1.50e-15 &      \\ 
     &           &    3 &  5.24e-08 &  7.23e-16 &      \\ 
     &           &    4 &  2.54e-08 &  1.50e-15 &      \\ 
     &           &    5 &  2.53e-08 &  7.24e-16 &      \\ 
     &           &    6 &  2.53e-08 &  7.23e-16 &      \\ 
     &           &    7 &  5.24e-08 &  7.22e-16 &      \\ 
     &           &    8 &  2.53e-08 &  1.50e-15 &      \\ 
     &           &    9 &  2.53e-08 &  7.23e-16 &      \\ 
     &           &   10 &  5.24e-08 &  7.22e-16 &      \\ 
3354 &  3.35e+04 &   10 &           &           & iters  \\ 
 \hdashline 
     &           &    1 &  2.12e-08 &  1.70e-09 &      \\ 
     &           &    2 &  2.12e-08 &  6.05e-16 &      \\ 
     &           &    3 &  5.65e-08 &  6.04e-16 &      \\ 
     &           &    4 &  2.12e-08 &  1.61e-15 &      \\ 
     &           &    5 &  2.12e-08 &  6.06e-16 &      \\ 
     &           &    6 &  5.65e-08 &  6.05e-16 &      \\ 
     &           &    7 &  2.12e-08 &  1.61e-15 &      \\ 
     &           &    8 &  2.12e-08 &  6.06e-16 &      \\ 
     &           &    9 &  2.12e-08 &  6.06e-16 &      \\ 
     &           &   10 &  5.65e-08 &  6.05e-16 &      \\ 
3355 &  3.35e+04 &   10 &           &           & iters  \\ 
 \hdashline 
     &           &    1 &  3.47e-08 &  5.63e-09 &      \\ 
     &           &    2 &  4.15e-09 &  9.91e-16 &      \\ 
     &           &    3 &  4.15e-09 &  1.19e-16 &      \\ 
     &           &    4 &  4.14e-09 &  1.18e-16 &      \\ 
     &           &    5 &  3.47e-08 &  1.18e-16 &      \\ 
     &           &    6 &  4.19e-09 &  9.91e-16 &      \\ 
     &           &    7 &  4.18e-09 &  1.19e-16 &      \\ 
     &           &    8 &  4.17e-09 &  1.19e-16 &      \\ 
     &           &    9 &  4.17e-09 &  1.19e-16 &      \\ 
     &           &   10 &  4.16e-09 &  1.19e-16 &      \\ 
3356 &  3.36e+04 &   10 &           &           & iters  \\ 
 \hdashline 
     &           &    1 &  6.09e-09 &  4.12e-09 &      \\ 
     &           &    2 &  6.08e-09 &  1.74e-16 &      \\ 
     &           &    3 &  6.07e-09 &  1.74e-16 &      \\ 
     &           &    4 &  6.06e-09 &  1.73e-16 &      \\ 
     &           &    5 &  6.06e-09 &  1.73e-16 &      \\ 
     &           &    6 &  6.05e-09 &  1.73e-16 &      \\ 
     &           &    7 &  7.17e-08 &  1.73e-16 &      \\ 
     &           &    8 &  4.50e-08 &  2.04e-15 &      \\ 
     &           &    9 &  6.08e-09 &  1.28e-15 &      \\ 
     &           &   10 &  6.07e-09 &  1.73e-16 &      \\ 
3357 &  3.36e+04 &   10 &           &           & iters  \\ 
 \hdashline 
     &           &    1 &  1.09e-08 &  1.99e-09 &      \\ 
     &           &    2 &  1.09e-08 &  3.11e-16 &      \\ 
     &           &    3 &  1.09e-08 &  3.10e-16 &      \\ 
     &           &    4 &  2.80e-08 &  3.10e-16 &      \\ 
     &           &    5 &  1.09e-08 &  7.99e-16 &      \\ 
     &           &    6 &  1.09e-08 &  3.10e-16 &      \\ 
     &           &    7 &  1.08e-08 &  3.10e-16 &      \\ 
     &           &    8 &  2.80e-08 &  3.09e-16 &      \\ 
     &           &    9 &  1.09e-08 &  8.00e-16 &      \\ 
     &           &   10 &  1.09e-08 &  3.10e-16 &      \\ 
3358 &  3.36e+04 &   10 &           &           & iters  \\ 
 \hdashline 
     &           &    1 &  5.89e-10 &  8.13e-09 &      \\ 
     &           &    2 &  5.88e-10 &  1.68e-17 &      \\ 
     &           &    3 &  5.87e-10 &  1.68e-17 &      \\ 
     &           &    4 &  5.86e-10 &  1.68e-17 &      \\ 
     &           &    5 &  5.85e-10 &  1.67e-17 &      \\ 
     &           &    6 &  5.85e-10 &  1.67e-17 &      \\ 
     &           &    7 &  5.84e-10 &  1.67e-17 &      \\ 
     &           &    8 &  5.83e-10 &  1.67e-17 &      \\ 
     &           &    9 &  5.82e-10 &  1.66e-17 &      \\ 
     &           &   10 &  5.81e-10 &  1.66e-17 &      \\ 
3359 &  3.36e+04 &   10 &           &           & iters  \\ 
 \hdashline 
     &           &    1 &  2.16e-08 &  1.33e-08 &      \\ 
     &           &    2 &  2.16e-08 &  6.17e-16 &      \\ 
     &           &    3 &  5.61e-08 &  6.16e-16 &      \\ 
     &           &    4 &  2.16e-08 &  1.60e-15 &      \\ 
     &           &    5 &  2.16e-08 &  6.18e-16 &      \\ 
     &           &    6 &  2.16e-08 &  6.17e-16 &      \\ 
     &           &    7 &  5.61e-08 &  6.16e-16 &      \\ 
     &           &    8 &  2.16e-08 &  1.60e-15 &      \\ 
     &           &    9 &  2.16e-08 &  6.17e-16 &      \\ 
     &           &   10 &  5.61e-08 &  6.16e-16 &      \\ 
3360 &  3.36e+04 &   10 &           &           & iters  \\ 
 \hdashline 
     &           &    1 &  3.98e-10 &  9.10e-09 &      \\ 
     &           &    2 &  3.98e-10 &  1.14e-17 &      \\ 
     &           &    3 &  3.97e-10 &  1.13e-17 &      \\ 
     &           &    4 &  3.97e-10 &  1.13e-17 &      \\ 
     &           &    5 &  3.96e-10 &  1.13e-17 &      \\ 
     &           &    6 &  3.95e-10 &  1.13e-17 &      \\ 
     &           &    7 &  3.95e-10 &  1.13e-17 &      \\ 
     &           &    8 &  3.94e-10 &  1.13e-17 &      \\ 
     &           &    9 &  3.94e-10 &  1.13e-17 &      \\ 
     &           &   10 &  3.93e-10 &  1.12e-17 &      \\ 
3361 &  3.36e+04 &   10 &           &           & iters  \\ 
 \hdashline 
     &           &    1 &  3.24e-08 &  7.95e-09 &      \\ 
     &           &    2 &  4.54e-08 &  9.24e-16 &      \\ 
     &           &    3 &  3.24e-08 &  1.29e-15 &      \\ 
     &           &    4 &  4.53e-08 &  9.25e-16 &      \\ 
     &           &    5 &  3.24e-08 &  1.29e-15 &      \\ 
     &           &    6 &  3.24e-08 &  9.25e-16 &      \\ 
     &           &    7 &  4.54e-08 &  9.24e-16 &      \\ 
     &           &    8 &  3.24e-08 &  1.29e-15 &      \\ 
     &           &    9 &  4.53e-08 &  9.24e-16 &      \\ 
     &           &   10 &  3.24e-08 &  1.29e-15 &      \\ 
3362 &  3.36e+04 &   10 &           &           & iters  \\ 
 \hdashline 
     &           &    1 &  1.62e-09 &  9.22e-09 &      \\ 
     &           &    2 &  1.62e-09 &  4.64e-17 &      \\ 
     &           &    3 &  1.62e-09 &  4.63e-17 &      \\ 
     &           &    4 &  1.62e-09 &  4.62e-17 &      \\ 
     &           &    5 &  1.61e-09 &  4.62e-17 &      \\ 
     &           &    6 &  1.61e-09 &  4.61e-17 &      \\ 
     &           &    7 &  1.61e-09 &  4.60e-17 &      \\ 
     &           &    8 &  1.61e-09 &  4.60e-17 &      \\ 
     &           &    9 &  1.61e-09 &  4.59e-17 &      \\ 
     &           &   10 &  1.60e-09 &  4.58e-17 &      \\ 
3363 &  3.36e+04 &   10 &           &           & iters  \\ 
 \hdashline 
     &           &    1 &  3.49e-08 &  2.50e-09 &      \\ 
     &           &    2 &  4.28e-08 &  9.96e-16 &      \\ 
     &           &    3 &  3.49e-08 &  1.22e-15 &      \\ 
     &           &    4 &  4.28e-08 &  9.96e-16 &      \\ 
     &           &    5 &  3.49e-08 &  1.22e-15 &      \\ 
     &           &    6 &  4.28e-08 &  9.97e-16 &      \\ 
     &           &    7 &  3.49e-08 &  1.22e-15 &      \\ 
     &           &    8 &  4.28e-08 &  9.97e-16 &      \\ 
     &           &    9 &  3.49e-08 &  1.22e-15 &      \\ 
     &           &   10 &  3.49e-08 &  9.97e-16 &      \\ 
3364 &  3.36e+04 &   10 &           &           & iters  \\ 
 \hdashline 
     &           &    1 &  4.54e-09 &  6.80e-09 &      \\ 
     &           &    2 &  4.54e-09 &  1.30e-16 &      \\ 
     &           &    3 &  4.53e-09 &  1.29e-16 &      \\ 
     &           &    4 &  4.52e-09 &  1.29e-16 &      \\ 
     &           &    5 &  4.52e-09 &  1.29e-16 &      \\ 
     &           &    6 &  4.51e-09 &  1.29e-16 &      \\ 
     &           &    7 &  4.50e-09 &  1.29e-16 &      \\ 
     &           &    8 &  4.50e-09 &  1.29e-16 &      \\ 
     &           &    9 &  4.49e-09 &  1.28e-16 &      \\ 
     &           &   10 &  7.32e-08 &  1.28e-16 &      \\ 
3365 &  3.36e+04 &   10 &           &           & iters  \\ 
 \hdashline 
     &           &    1 &  4.31e-08 &  9.75e-09 &      \\ 
     &           &    2 &  4.30e-08 &  1.23e-15 &      \\ 
     &           &    3 &  3.48e-08 &  1.23e-15 &      \\ 
     &           &    4 &  3.47e-08 &  9.92e-16 &      \\ 
     &           &    5 &  4.30e-08 &  9.90e-16 &      \\ 
     &           &    6 &  3.47e-08 &  1.23e-15 &      \\ 
     &           &    7 &  4.30e-08 &  9.91e-16 &      \\ 
     &           &    8 &  3.47e-08 &  1.23e-15 &      \\ 
     &           &    9 &  4.30e-08 &  9.91e-16 &      \\ 
     &           &   10 &  3.47e-08 &  1.23e-15 &      \\ 
3366 &  3.36e+04 &   10 &           &           & iters  \\ 
 \hdashline 
     &           &    1 &  6.37e-08 &  6.15e-09 &      \\ 
     &           &    2 &  1.41e-08 &  1.82e-15 &      \\ 
     &           &    3 &  1.41e-08 &  4.02e-16 &      \\ 
     &           &    4 &  1.40e-08 &  4.01e-16 &      \\ 
     &           &    5 &  1.40e-08 &  4.00e-16 &      \\ 
     &           &    6 &  1.40e-08 &  4.00e-16 &      \\ 
     &           &    7 &  6.37e-08 &  3.99e-16 &      \\ 
     &           &    8 &  1.41e-08 &  1.82e-15 &      \\ 
     &           &    9 &  1.40e-08 &  4.01e-16 &      \\ 
     &           &   10 &  1.40e-08 &  4.01e-16 &      \\ 
3367 &  3.37e+04 &   10 &           &           & iters  \\ 
 \hdashline 
     &           &    1 &  3.03e-08 &  8.19e-10 &      \\ 
     &           &    2 &  4.74e-08 &  8.64e-16 &      \\ 
     &           &    3 &  3.03e-08 &  1.35e-15 &      \\ 
     &           &    4 &  4.74e-08 &  8.65e-16 &      \\ 
     &           &    5 &  3.03e-08 &  1.35e-15 &      \\ 
     &           &    6 &  3.03e-08 &  8.66e-16 &      \\ 
     &           &    7 &  4.74e-08 &  8.65e-16 &      \\ 
     &           &    8 &  3.03e-08 &  1.35e-15 &      \\ 
     &           &    9 &  4.74e-08 &  8.65e-16 &      \\ 
     &           &   10 &  3.03e-08 &  1.35e-15 &      \\ 
3368 &  3.37e+04 &   10 &           &           & iters  \\ 
 \hdashline 
     &           &    1 &  5.99e-08 &  2.56e-09 &      \\ 
     &           &    2 &  1.79e-08 &  1.71e-15 &      \\ 
     &           &    3 &  1.78e-08 &  5.09e-16 &      \\ 
     &           &    4 &  1.78e-08 &  5.09e-16 &      \\ 
     &           &    5 &  5.99e-08 &  5.08e-16 &      \\ 
     &           &    6 &  1.79e-08 &  1.71e-15 &      \\ 
     &           &    7 &  1.78e-08 &  5.10e-16 &      \\ 
     &           &    8 &  1.78e-08 &  5.09e-16 &      \\ 
     &           &    9 &  5.99e-08 &  5.08e-16 &      \\ 
     &           &   10 &  1.79e-08 &  1.71e-15 &      \\ 
3369 &  3.37e+04 &   10 &           &           & iters  \\ 
 \hdashline 
     &           &    1 &  4.43e-08 &  4.48e-09 &      \\ 
     &           &    2 &  4.42e-08 &  1.26e-15 &      \\ 
     &           &    3 &  3.36e-08 &  1.26e-15 &      \\ 
     &           &    4 &  3.35e-08 &  9.58e-16 &      \\ 
     &           &    5 &  4.42e-08 &  9.56e-16 &      \\ 
     &           &    6 &  3.35e-08 &  1.26e-15 &      \\ 
     &           &    7 &  4.42e-08 &  9.57e-16 &      \\ 
     &           &    8 &  3.35e-08 &  1.26e-15 &      \\ 
     &           &    9 &  4.42e-08 &  9.57e-16 &      \\ 
     &           &   10 &  3.36e-08 &  1.26e-15 &      \\ 
3370 &  3.37e+04 &   10 &           &           & iters  \\ 
 \hdashline 
     &           &    1 &  7.11e-09 &  1.02e-08 &      \\ 
     &           &    2 &  7.10e-09 &  2.03e-16 &      \\ 
     &           &    3 &  7.09e-09 &  2.03e-16 &      \\ 
     &           &    4 &  7.08e-09 &  2.02e-16 &      \\ 
     &           &    5 &  7.06e-08 &  2.02e-16 &      \\ 
     &           &    6 &  4.60e-08 &  2.02e-15 &      \\ 
     &           &    7 &  7.11e-09 &  1.31e-15 &      \\ 
     &           &    8 &  7.10e-09 &  2.03e-16 &      \\ 
     &           &    9 &  7.09e-09 &  2.03e-16 &      \\ 
     &           &   10 &  7.06e-08 &  2.02e-16 &      \\ 
3371 &  3.37e+04 &   10 &           &           & iters  \\ 
 \hdashline 
     &           &    1 &  5.09e-08 &  5.95e-09 &      \\ 
     &           &    2 &  2.69e-08 &  1.45e-15 &      \\ 
     &           &    3 &  5.08e-08 &  7.67e-16 &      \\ 
     &           &    4 &  2.69e-08 &  1.45e-15 &      \\ 
     &           &    5 &  2.69e-08 &  7.68e-16 &      \\ 
     &           &    6 &  5.08e-08 &  7.67e-16 &      \\ 
     &           &    7 &  2.69e-08 &  1.45e-15 &      \\ 
     &           &    8 &  2.69e-08 &  7.68e-16 &      \\ 
     &           &    9 &  5.09e-08 &  7.67e-16 &      \\ 
     &           &   10 &  2.69e-08 &  1.45e-15 &      \\ 
3372 &  3.37e+04 &   10 &           &           & iters  \\ 
 \hdashline 
     &           &    1 &  1.34e-08 &  3.65e-10 &      \\ 
     &           &    2 &  1.34e-08 &  3.82e-16 &      \\ 
     &           &    3 &  6.43e-08 &  3.81e-16 &      \\ 
     &           &    4 &  1.34e-08 &  1.84e-15 &      \\ 
     &           &    5 &  1.34e-08 &  3.84e-16 &      \\ 
     &           &    6 &  1.34e-08 &  3.83e-16 &      \\ 
     &           &    7 &  1.34e-08 &  3.82e-16 &      \\ 
     &           &    8 &  1.34e-08 &  3.82e-16 &      \\ 
     &           &    9 &  6.43e-08 &  3.81e-16 &      \\ 
     &           &   10 &  1.34e-08 &  1.84e-15 &      \\ 
3373 &  3.37e+04 &   10 &           &           & iters  \\ 
 \hdashline 
     &           &    1 &  9.49e-09 &  2.38e-09 &      \\ 
     &           &    2 &  9.47e-09 &  2.71e-16 &      \\ 
     &           &    3 &  9.46e-09 &  2.70e-16 &      \\ 
     &           &    4 &  6.82e-08 &  2.70e-16 &      \\ 
     &           &    5 &  9.54e-09 &  1.95e-15 &      \\ 
     &           &    6 &  9.53e-09 &  2.72e-16 &      \\ 
     &           &    7 &  9.52e-09 &  2.72e-16 &      \\ 
     &           &    8 &  9.50e-09 &  2.72e-16 &      \\ 
     &           &    9 &  9.49e-09 &  2.71e-16 &      \\ 
     &           &   10 &  9.48e-09 &  2.71e-16 &      \\ 
3374 &  3.37e+04 &   10 &           &           & iters  \\ 
 \hdashline 
     &           &    1 &  3.09e-08 &  1.89e-10 &      \\ 
     &           &    2 &  3.08e-08 &  8.81e-16 &      \\ 
     &           &    3 &  4.69e-08 &  8.80e-16 &      \\ 
     &           &    4 &  3.08e-08 &  1.34e-15 &      \\ 
     &           &    5 &  3.08e-08 &  8.80e-16 &      \\ 
     &           &    6 &  4.69e-08 &  8.79e-16 &      \\ 
     &           &    7 &  3.08e-08 &  1.34e-15 &      \\ 
     &           &    8 &  4.69e-08 &  8.80e-16 &      \\ 
     &           &    9 &  3.09e-08 &  1.34e-15 &      \\ 
     &           &   10 &  3.08e-08 &  8.81e-16 &      \\ 
3375 &  3.37e+04 &   10 &           &           & iters  \\ 
 \hdashline 
     &           &    1 &  4.34e-08 &  7.97e-09 &      \\ 
     &           &    2 &  4.44e-09 &  1.24e-15 &      \\ 
     &           &    3 &  4.44e-09 &  1.27e-16 &      \\ 
     &           &    4 &  4.43e-09 &  1.27e-16 &      \\ 
     &           &    5 &  4.42e-09 &  1.26e-16 &      \\ 
     &           &    6 &  4.42e-09 &  1.26e-16 &      \\ 
     &           &    7 &  4.41e-09 &  1.26e-16 &      \\ 
     &           &    8 &  4.41e-09 &  1.26e-16 &      \\ 
     &           &    9 &  4.40e-09 &  1.26e-16 &      \\ 
     &           &   10 &  4.39e-09 &  1.26e-16 &      \\ 
3376 &  3.38e+04 &   10 &           &           & iters  \\ 
 \hdashline 
     &           &    1 &  8.19e-08 &  7.19e-09 &      \\ 
     &           &    2 &  7.35e-08 &  2.34e-15 &      \\ 
     &           &    3 &  4.25e-09 &  2.10e-15 &      \\ 
     &           &    4 &  4.24e-09 &  1.21e-16 &      \\ 
     &           &    5 &  4.24e-09 &  1.21e-16 &      \\ 
     &           &    6 &  4.23e-09 &  1.21e-16 &      \\ 
     &           &    7 &  4.22e-09 &  1.21e-16 &      \\ 
     &           &    8 &  4.22e-09 &  1.21e-16 &      \\ 
     &           &    9 &  4.21e-09 &  1.20e-16 &      \\ 
     &           &   10 &  4.21e-09 &  1.20e-16 &      \\ 
3377 &  3.38e+04 &   10 &           &           & iters  \\ 
 \hdashline 
     &           &    1 &  2.00e-08 &  1.70e-09 &      \\ 
     &           &    2 &  5.77e-08 &  5.71e-16 &      \\ 
     &           &    3 &  2.00e-08 &  1.65e-15 &      \\ 
     &           &    4 &  2.00e-08 &  5.72e-16 &      \\ 
     &           &    5 &  2.00e-08 &  5.71e-16 &      \\ 
     &           &    6 &  5.77e-08 &  5.70e-16 &      \\ 
     &           &    7 &  2.00e-08 &  1.65e-15 &      \\ 
     &           &    8 &  2.00e-08 &  5.72e-16 &      \\ 
     &           &    9 &  2.00e-08 &  5.71e-16 &      \\ 
     &           &   10 &  5.77e-08 &  5.70e-16 &      \\ 
3378 &  3.38e+04 &   10 &           &           & iters  \\ 
 \hdashline 
     &           &    1 &  1.38e-09 &  4.98e-10 &      \\ 
     &           &    2 &  1.38e-09 &  3.93e-17 &      \\ 
     &           &    3 &  1.37e-09 &  3.93e-17 &      \\ 
     &           &    4 &  1.37e-09 &  3.92e-17 &      \\ 
     &           &    5 &  1.37e-09 &  3.92e-17 &      \\ 
     &           &    6 &  1.37e-09 &  3.91e-17 &      \\ 
     &           &    7 &  1.37e-09 &  3.91e-17 &      \\ 
     &           &    8 &  1.36e-09 &  3.90e-17 &      \\ 
     &           &    9 &  1.36e-09 &  3.89e-17 &      \\ 
     &           &   10 &  1.36e-09 &  3.89e-17 &      \\ 
3379 &  3.38e+04 &   10 &           &           & iters  \\ 
 \hdashline 
     &           &    1 &  1.83e-08 &  3.38e-09 &      \\ 
     &           &    2 &  1.82e-08 &  5.21e-16 &      \\ 
     &           &    3 &  1.82e-08 &  5.20e-16 &      \\ 
     &           &    4 &  1.82e-08 &  5.20e-16 &      \\ 
     &           &    5 &  5.95e-08 &  5.19e-16 &      \\ 
     &           &    6 &  1.82e-08 &  1.70e-15 &      \\ 
     &           &    7 &  1.82e-08 &  5.21e-16 &      \\ 
     &           &    8 &  1.82e-08 &  5.20e-16 &      \\ 
     &           &    9 &  5.95e-08 &  5.19e-16 &      \\ 
     &           &   10 &  1.82e-08 &  1.70e-15 &      \\ 
3380 &  3.38e+04 &   10 &           &           & iters  \\ 
 \hdashline 
     &           &    1 &  5.32e-09 &  4.27e-09 &      \\ 
     &           &    2 &  7.24e-08 &  1.52e-16 &      \\ 
     &           &    3 &  8.31e-08 &  2.07e-15 &      \\ 
     &           &    4 &  7.24e-08 &  2.37e-15 &      \\ 
     &           &    5 &  5.40e-09 &  2.07e-15 &      \\ 
     &           &    6 &  5.39e-09 &  1.54e-16 &      \\ 
     &           &    7 &  5.38e-09 &  1.54e-16 &      \\ 
     &           &    8 &  5.38e-09 &  1.54e-16 &      \\ 
     &           &    9 &  5.37e-09 &  1.53e-16 &      \\ 
     &           &   10 &  5.36e-09 &  1.53e-16 &      \\ 
3381 &  3.38e+04 &   10 &           &           & iters  \\ 
 \hdashline 
     &           &    1 &  1.62e-08 &  3.48e-09 &      \\ 
     &           &    2 &  2.27e-08 &  4.61e-16 &      \\ 
     &           &    3 &  1.62e-08 &  6.48e-16 &      \\ 
     &           &    4 &  2.27e-08 &  4.62e-16 &      \\ 
     &           &    5 &  1.62e-08 &  6.48e-16 &      \\ 
     &           &    6 &  1.62e-08 &  4.62e-16 &      \\ 
     &           &    7 &  2.27e-08 &  4.61e-16 &      \\ 
     &           &    8 &  1.62e-08 &  6.48e-16 &      \\ 
     &           &    9 &  2.27e-08 &  4.61e-16 &      \\ 
     &           &   10 &  1.62e-08 &  6.48e-16 &      \\ 
3382 &  3.38e+04 &   10 &           &           & iters  \\ 
 \hdashline 
     &           &    1 &  1.30e-09 &  3.15e-09 &      \\ 
     &           &    2 &  1.30e-09 &  3.72e-17 &      \\ 
     &           &    3 &  1.30e-09 &  3.71e-17 &      \\ 
     &           &    4 &  1.30e-09 &  3.70e-17 &      \\ 
     &           &    5 &  1.29e-09 &  3.70e-17 &      \\ 
     &           &    6 &  1.29e-09 &  3.69e-17 &      \\ 
     &           &    7 &  1.29e-09 &  3.69e-17 &      \\ 
     &           &    8 &  1.29e-09 &  3.68e-17 &      \\ 
     &           &    9 &  1.29e-09 &  3.68e-17 &      \\ 
     &           &   10 &  1.29e-09 &  3.67e-17 &      \\ 
3383 &  3.38e+04 &   10 &           &           & iters  \\ 
 \hdashline 
     &           &    1 &  1.24e-09 &  7.06e-11 &      \\ 
     &           &    2 &  1.23e-09 &  3.53e-17 &      \\ 
     &           &    3 &  1.23e-09 &  3.52e-17 &      \\ 
     &           &    4 &  1.23e-09 &  3.52e-17 &      \\ 
     &           &    5 &  1.23e-09 &  3.51e-17 &      \\ 
     &           &    6 &  1.23e-09 &  3.51e-17 &      \\ 
     &           &    7 &  1.23e-09 &  3.50e-17 &      \\ 
     &           &    8 &  1.22e-09 &  3.50e-17 &      \\ 
     &           &    9 &  3.76e-08 &  3.49e-17 &      \\ 
     &           &   10 &  1.28e-09 &  1.07e-15 &      \\ 
3384 &  3.38e+04 &   10 &           &           & iters  \\ 
 \hdashline 
     &           &    1 &  3.17e-08 &  1.21e-09 &      \\ 
     &           &    2 &  7.18e-09 &  9.05e-16 &      \\ 
     &           &    3 &  7.17e-09 &  2.05e-16 &      \\ 
     &           &    4 &  7.16e-09 &  2.05e-16 &      \\ 
     &           &    5 &  7.15e-09 &  2.04e-16 &      \\ 
     &           &    6 &  3.17e-08 &  2.04e-16 &      \\ 
     &           &    7 &  7.18e-09 &  9.05e-16 &      \\ 
     &           &    8 &  7.17e-09 &  2.05e-16 &      \\ 
     &           &    9 &  7.16e-09 &  2.05e-16 &      \\ 
     &           &   10 &  7.15e-09 &  2.04e-16 &      \\ 
3385 &  3.38e+04 &   10 &           &           & iters  \\ 
 \hdashline 
     &           &    1 &  1.43e-08 &  1.48e-09 &      \\ 
     &           &    2 &  1.42e-08 &  4.07e-16 &      \\ 
     &           &    3 &  1.42e-08 &  4.06e-16 &      \\ 
     &           &    4 &  6.35e-08 &  4.06e-16 &      \\ 
     &           &    5 &  1.43e-08 &  1.81e-15 &      \\ 
     &           &    6 &  1.43e-08 &  4.08e-16 &      \\ 
     &           &    7 &  1.42e-08 &  4.07e-16 &      \\ 
     &           &    8 &  1.42e-08 &  4.07e-16 &      \\ 
     &           &    9 &  6.35e-08 &  4.06e-16 &      \\ 
     &           &   10 &  1.43e-08 &  1.81e-15 &      \\ 
3386 &  3.38e+04 &   10 &           &           & iters  \\ 
 \hdashline 
     &           &    1 &  2.26e-08 &  1.26e-09 &      \\ 
     &           &    2 &  2.25e-08 &  6.44e-16 &      \\ 
     &           &    3 &  2.25e-08 &  6.43e-16 &      \\ 
     &           &    4 &  5.52e-08 &  6.42e-16 &      \\ 
     &           &    5 &  2.25e-08 &  1.58e-15 &      \\ 
     &           &    6 &  2.25e-08 &  6.44e-16 &      \\ 
     &           &    7 &  5.52e-08 &  6.43e-16 &      \\ 
     &           &    8 &  2.26e-08 &  1.58e-15 &      \\ 
     &           &    9 &  2.25e-08 &  6.44e-16 &      \\ 
     &           &   10 &  2.25e-08 &  6.43e-16 &      \\ 
3387 &  3.39e+04 &   10 &           &           & iters  \\ 
 \hdashline 
     &           &    1 &  4.10e-08 &  1.06e-10 &      \\ 
     &           &    2 &  2.12e-09 &  1.17e-15 &      \\ 
     &           &    3 &  2.12e-09 &  6.06e-17 &      \\ 
     &           &    4 &  2.12e-09 &  6.05e-17 &      \\ 
     &           &    5 &  2.12e-09 &  6.05e-17 &      \\ 
     &           &    6 &  2.11e-09 &  6.04e-17 &      \\ 
     &           &    7 &  2.11e-09 &  6.03e-17 &      \\ 
     &           &    8 &  2.11e-09 &  6.02e-17 &      \\ 
     &           &    9 &  2.10e-09 &  6.01e-17 &      \\ 
     &           &   10 &  2.10e-09 &  6.00e-17 &      \\ 
3388 &  3.39e+04 &   10 &           &           & iters  \\ 
 \hdashline 
     &           &    1 &  1.46e-09 &  7.87e-10 &      \\ 
     &           &    2 &  1.46e-09 &  4.18e-17 &      \\ 
     &           &    3 &  1.46e-09 &  4.17e-17 &      \\ 
     &           &    4 &  1.46e-09 &  4.17e-17 &      \\ 
     &           &    5 &  1.46e-09 &  4.16e-17 &      \\ 
     &           &    6 &  1.45e-09 &  4.16e-17 &      \\ 
     &           &    7 &  1.45e-09 &  4.15e-17 &      \\ 
     &           &    8 &  1.45e-09 &  4.14e-17 &      \\ 
     &           &    9 &  1.45e-09 &  4.14e-17 &      \\ 
     &           &   10 &  1.45e-09 &  4.13e-17 &      \\ 
3389 &  3.39e+04 &   10 &           &           & iters  \\ 
 \hdashline 
     &           &    1 &  5.37e-08 &  3.83e-09 &      \\ 
     &           &    2 &  1.48e-08 &  1.53e-15 &      \\ 
     &           &    3 &  6.29e-08 &  4.22e-16 &      \\ 
     &           &    4 &  1.49e-08 &  1.80e-15 &      \\ 
     &           &    5 &  1.48e-08 &  4.24e-16 &      \\ 
     &           &    6 &  1.48e-08 &  4.24e-16 &      \\ 
     &           &    7 &  1.48e-08 &  4.23e-16 &      \\ 
     &           &    8 &  1.48e-08 &  4.23e-16 &      \\ 
     &           &    9 &  6.29e-08 &  4.22e-16 &      \\ 
     &           &   10 &  1.49e-08 &  1.80e-15 &      \\ 
3390 &  3.39e+04 &   10 &           &           & iters  \\ 
 \hdashline 
     &           &    1 &  1.26e-08 &  4.98e-09 &      \\ 
     &           &    2 &  1.26e-08 &  3.59e-16 &      \\ 
     &           &    3 &  1.25e-08 &  3.58e-16 &      \\ 
     &           &    4 &  6.52e-08 &  3.58e-16 &      \\ 
     &           &    5 &  1.26e-08 &  1.86e-15 &      \\ 
     &           &    6 &  1.26e-08 &  3.60e-16 &      \\ 
     &           &    7 &  1.26e-08 &  3.59e-16 &      \\ 
     &           &    8 &  1.26e-08 &  3.59e-16 &      \\ 
     &           &    9 &  1.25e-08 &  3.58e-16 &      \\ 
     &           &   10 &  6.52e-08 &  3.58e-16 &      \\ 
3391 &  3.39e+04 &   10 &           &           & iters  \\ 
 \hdashline 
     &           &    1 &  2.18e-09 &  2.30e-09 &      \\ 
     &           &    2 &  2.18e-09 &  6.23e-17 &      \\ 
     &           &    3 &  2.18e-09 &  6.22e-17 &      \\ 
     &           &    4 &  2.17e-09 &  6.21e-17 &      \\ 
     &           &    5 &  2.17e-09 &  6.20e-17 &      \\ 
     &           &    6 &  2.17e-09 &  6.19e-17 &      \\ 
     &           &    7 &  2.16e-09 &  6.18e-17 &      \\ 
     &           &    8 &  2.16e-09 &  6.17e-17 &      \\ 
     &           &    9 &  2.16e-09 &  6.17e-17 &      \\ 
     &           &   10 &  2.15e-09 &  6.16e-17 &      \\ 
3392 &  3.39e+04 &   10 &           &           & iters  \\ 
 \hdashline 
     &           &    1 &  4.14e-08 &  2.72e-09 &      \\ 
     &           &    2 &  4.13e-08 &  1.18e-15 &      \\ 
     &           &    3 &  3.64e-08 &  1.18e-15 &      \\ 
     &           &    4 &  4.13e-08 &  1.04e-15 &      \\ 
     &           &    5 &  3.64e-08 &  1.18e-15 &      \\ 
     &           &    6 &  4.13e-08 &  1.04e-15 &      \\ 
     &           &    7 &  3.64e-08 &  1.18e-15 &      \\ 
     &           &    8 &  4.13e-08 &  1.04e-15 &      \\ 
     &           &    9 &  3.64e-08 &  1.18e-15 &      \\ 
     &           &   10 &  4.13e-08 &  1.04e-15 &      \\ 
3393 &  3.39e+04 &   10 &           &           & iters  \\ 
 \hdashline 
     &           &    1 &  9.26e-08 &  4.98e-09 &      \\ 
     &           &    2 &  6.29e-08 &  2.64e-15 &      \\ 
     &           &    3 &  1.49e-08 &  1.79e-15 &      \\ 
     &           &    4 &  1.49e-08 &  4.25e-16 &      \\ 
     &           &    5 &  1.49e-08 &  4.25e-16 &      \\ 
     &           &    6 &  1.48e-08 &  4.24e-16 &      \\ 
     &           &    7 &  6.29e-08 &  4.23e-16 &      \\ 
     &           &    8 &  1.49e-08 &  1.79e-15 &      \\ 
     &           &    9 &  1.49e-08 &  4.25e-16 &      \\ 
     &           &   10 &  1.49e-08 &  4.25e-16 &      \\ 
3394 &  3.39e+04 &   10 &           &           & iters  \\ 
 \hdashline 
     &           &    1 &  2.13e-08 &  1.82e-09 &      \\ 
     &           &    2 &  5.64e-08 &  6.08e-16 &      \\ 
     &           &    3 &  2.13e-08 &  1.61e-15 &      \\ 
     &           &    4 &  2.13e-08 &  6.09e-16 &      \\ 
     &           &    5 &  5.64e-08 &  6.08e-16 &      \\ 
     &           &    6 &  2.14e-08 &  1.61e-15 &      \\ 
     &           &    7 &  2.13e-08 &  6.10e-16 &      \\ 
     &           &    8 &  2.13e-08 &  6.09e-16 &      \\ 
     &           &    9 &  5.64e-08 &  6.08e-16 &      \\ 
     &           &   10 &  2.14e-08 &  1.61e-15 &      \\ 
3395 &  3.39e+04 &   10 &           &           & iters  \\ 
 \hdashline 
     &           &    1 &  2.33e-09 &  2.67e-09 &      \\ 
     &           &    2 &  2.33e-09 &  6.66e-17 &      \\ 
     &           &    3 &  2.33e-09 &  6.65e-17 &      \\ 
     &           &    4 &  2.32e-09 &  6.64e-17 &      \\ 
     &           &    5 &  2.32e-09 &  6.63e-17 &      \\ 
     &           &    6 &  2.32e-09 &  6.62e-17 &      \\ 
     &           &    7 &  2.31e-09 &  6.61e-17 &      \\ 
     &           &    8 &  2.31e-09 &  6.60e-17 &      \\ 
     &           &    9 &  2.31e-09 &  6.59e-17 &      \\ 
     &           &   10 &  2.30e-09 &  6.58e-17 &      \\ 
3396 &  3.40e+04 &   10 &           &           & iters  \\ 
 \hdashline 
     &           &    1 &  7.00e-08 &  1.91e-09 &      \\ 
     &           &    2 &  7.76e-09 &  2.00e-15 &      \\ 
     &           &    3 &  7.75e-09 &  2.21e-16 &      \\ 
     &           &    4 &  7.73e-09 &  2.21e-16 &      \\ 
     &           &    5 &  7.72e-09 &  2.21e-16 &      \\ 
     &           &    6 &  7.71e-09 &  2.20e-16 &      \\ 
     &           &    7 &  7.70e-09 &  2.20e-16 &      \\ 
     &           &    8 &  7.69e-09 &  2.20e-16 &      \\ 
     &           &    9 &  7.00e-08 &  2.19e-16 &      \\ 
     &           &   10 &  8.55e-08 &  2.00e-15 &      \\ 
3397 &  3.40e+04 &   10 &           &           & iters  \\ 
 \hdashline 
     &           &    1 &  2.53e-08 &  8.42e-10 &      \\ 
     &           &    2 &  2.52e-08 &  7.21e-16 &      \\ 
     &           &    3 &  5.25e-08 &  7.20e-16 &      \\ 
     &           &    4 &  2.53e-08 &  1.50e-15 &      \\ 
     &           &    5 &  2.52e-08 &  7.21e-16 &      \\ 
     &           &    6 &  5.25e-08 &  7.20e-16 &      \\ 
     &           &    7 &  2.53e-08 &  1.50e-15 &      \\ 
     &           &    8 &  2.52e-08 &  7.21e-16 &      \\ 
     &           &    9 &  5.25e-08 &  7.20e-16 &      \\ 
     &           &   10 &  2.53e-08 &  1.50e-15 &      \\ 
3398 &  3.40e+04 &   10 &           &           & iters  \\ 
 \hdashline 
     &           &    1 &  3.20e-08 &  2.00e-09 &      \\ 
     &           &    2 &  4.57e-08 &  9.13e-16 &      \\ 
     &           &    3 &  3.20e-08 &  1.31e-15 &      \\ 
     &           &    4 &  4.57e-08 &  9.14e-16 &      \\ 
     &           &    5 &  3.20e-08 &  1.30e-15 &      \\ 
     &           &    6 &  3.20e-08 &  9.15e-16 &      \\ 
     &           &    7 &  4.57e-08 &  9.13e-16 &      \\ 
     &           &    8 &  3.20e-08 &  1.31e-15 &      \\ 
     &           &    9 &  4.57e-08 &  9.14e-16 &      \\ 
     &           &   10 &  3.20e-08 &  1.30e-15 &      \\ 
3399 &  3.40e+04 &   10 &           &           & iters  \\ 
 \hdashline 
     &           &    1 &  2.70e-08 &  2.74e-10 &      \\ 
     &           &    2 &  5.08e-08 &  7.69e-16 &      \\ 
     &           &    3 &  2.70e-08 &  1.45e-15 &      \\ 
     &           &    4 &  5.07e-08 &  7.70e-16 &      \\ 
     &           &    5 &  2.70e-08 &  1.45e-15 &      \\ 
     &           &    6 &  2.70e-08 &  7.71e-16 &      \\ 
     &           &    7 &  5.07e-08 &  7.70e-16 &      \\ 
     &           &    8 &  2.70e-08 &  1.45e-15 &      \\ 
     &           &    9 &  2.70e-08 &  7.71e-16 &      \\ 
     &           &   10 &  5.07e-08 &  7.70e-16 &      \\ 
3400 &  3.40e+04 &   10 &           &           & iters  \\ 
 \hdashline 
     &           &    1 &  5.27e-08 &  2.84e-09 &      \\ 
     &           &    2 &  2.50e-08 &  1.51e-15 &      \\ 
     &           &    3 &  2.50e-08 &  7.14e-16 &      \\ 
     &           &    4 &  5.27e-08 &  7.13e-16 &      \\ 
     &           &    5 &  2.50e-08 &  1.51e-15 &      \\ 
     &           &    6 &  2.50e-08 &  7.14e-16 &      \\ 
     &           &    7 &  5.27e-08 &  7.13e-16 &      \\ 
     &           &    8 &  2.50e-08 &  1.51e-15 &      \\ 
     &           &    9 &  2.50e-08 &  7.14e-16 &      \\ 
     &           &   10 &  5.27e-08 &  7.13e-16 &      \\ 
3401 &  3.40e+04 &   10 &           &           & iters  \\ 
 \hdashline 
     &           &    1 &  2.70e-08 &  2.11e-09 &      \\ 
     &           &    2 &  5.07e-08 &  7.71e-16 &      \\ 
     &           &    3 &  2.70e-08 &  1.45e-15 &      \\ 
     &           &    4 &  2.70e-08 &  7.71e-16 &      \\ 
     &           &    5 &  5.07e-08 &  7.70e-16 &      \\ 
     &           &    6 &  2.70e-08 &  1.45e-15 &      \\ 
     &           &    7 &  2.70e-08 &  7.71e-16 &      \\ 
     &           &    8 &  5.07e-08 &  7.70e-16 &      \\ 
     &           &    9 &  2.70e-08 &  1.45e-15 &      \\ 
     &           &   10 &  2.70e-08 &  7.71e-16 &      \\ 
3402 &  3.40e+04 &   10 &           &           & iters  \\ 
 \hdashline 
     &           &    1 &  3.95e-08 &  1.61e-10 &      \\ 
     &           &    2 &  3.82e-08 &  1.13e-15 &      \\ 
     &           &    3 &  3.95e-08 &  1.09e-15 &      \\ 
     &           &    4 &  3.82e-08 &  1.13e-15 &      \\ 
     &           &    5 &  3.95e-08 &  1.09e-15 &      \\ 
     &           &    6 &  3.82e-08 &  1.13e-15 &      \\ 
     &           &    7 &  3.95e-08 &  1.09e-15 &      \\ 
     &           &    8 &  3.82e-08 &  1.13e-15 &      \\ 
     &           &    9 &  3.95e-08 &  1.09e-15 &      \\ 
     &           &   10 &  3.82e-08 &  1.13e-15 &      \\ 
3403 &  3.40e+04 &   10 &           &           & iters  \\ 
 \hdashline 
     &           &    1 &  2.74e-08 &  5.35e-10 &      \\ 
     &           &    2 &  2.73e-08 &  7.81e-16 &      \\ 
     &           &    3 &  5.04e-08 &  7.80e-16 &      \\ 
     &           &    4 &  2.74e-08 &  1.44e-15 &      \\ 
     &           &    5 &  2.73e-08 &  7.81e-16 &      \\ 
     &           &    6 &  5.04e-08 &  7.80e-16 &      \\ 
     &           &    7 &  2.74e-08 &  1.44e-15 &      \\ 
     &           &    8 &  2.73e-08 &  7.81e-16 &      \\ 
     &           &    9 &  5.04e-08 &  7.80e-16 &      \\ 
     &           &   10 &  2.74e-08 &  1.44e-15 &      \\ 
3404 &  3.40e+04 &   10 &           &           & iters  \\ 
 \hdashline 
     &           &    1 &  4.61e-08 &  2.80e-09 &      \\ 
     &           &    2 &  3.17e-08 &  1.31e-15 &      \\ 
     &           &    3 &  4.60e-08 &  9.05e-16 &      \\ 
     &           &    4 &  3.17e-08 &  1.31e-15 &      \\ 
     &           &    5 &  3.17e-08 &  9.05e-16 &      \\ 
     &           &    6 &  4.61e-08 &  9.04e-16 &      \\ 
     &           &    7 &  3.17e-08 &  1.31e-15 &      \\ 
     &           &    8 &  4.60e-08 &  9.05e-16 &      \\ 
     &           &    9 &  3.17e-08 &  1.31e-15 &      \\ 
     &           &   10 &  3.17e-08 &  9.05e-16 &      \\ 
3405 &  3.40e+04 &   10 &           &           & iters  \\ 
 \hdashline 
     &           &    1 &  1.02e-08 &  4.15e-09 &      \\ 
     &           &    2 &  6.75e-08 &  2.92e-16 &      \\ 
     &           &    3 &  4.92e-08 &  1.93e-15 &      \\ 
     &           &    4 &  1.02e-08 &  1.40e-15 &      \\ 
     &           &    5 &  1.02e-08 &  2.92e-16 &      \\ 
     &           &    6 &  6.75e-08 &  2.92e-16 &      \\ 
     &           &    7 &  1.03e-08 &  1.93e-15 &      \\ 
     &           &    8 &  1.03e-08 &  2.94e-16 &      \\ 
     &           &    9 &  1.03e-08 &  2.94e-16 &      \\ 
     &           &   10 &  1.03e-08 &  2.93e-16 &      \\ 
3406 &  3.40e+04 &   10 &           &           & iters  \\ 
 \hdashline 
     &           &    1 &  1.46e-08 &  2.59e-09 &      \\ 
     &           &    2 &  6.31e-08 &  4.16e-16 &      \\ 
     &           &    3 &  1.46e-08 &  1.80e-15 &      \\ 
     &           &    4 &  1.46e-08 &  4.18e-16 &      \\ 
     &           &    5 &  1.46e-08 &  4.17e-16 &      \\ 
     &           &    6 &  1.46e-08 &  4.17e-16 &      \\ 
     &           &    7 &  6.31e-08 &  4.16e-16 &      \\ 
     &           &    8 &  1.47e-08 &  1.80e-15 &      \\ 
     &           &    9 &  1.46e-08 &  4.18e-16 &      \\ 
     &           &   10 &  1.46e-08 &  4.18e-16 &      \\ 
3407 &  3.41e+04 &   10 &           &           & iters  \\ 
 \hdashline 
     &           &    1 &  4.35e-10 &  1.29e-09 &      \\ 
     &           &    2 &  4.34e-10 &  1.24e-17 &      \\ 
     &           &    3 &  4.34e-10 &  1.24e-17 &      \\ 
     &           &    4 &  4.33e-10 &  1.24e-17 &      \\ 
     &           &    5 &  4.32e-10 &  1.24e-17 &      \\ 
     &           &    6 &  4.32e-10 &  1.23e-17 &      \\ 
     &           &    7 &  4.31e-10 &  1.23e-17 &      \\ 
     &           &    8 &  4.31e-10 &  1.23e-17 &      \\ 
     &           &    9 &  4.30e-10 &  1.23e-17 &      \\ 
     &           &   10 &  4.29e-10 &  1.23e-17 &      \\ 
3408 &  3.41e+04 &   10 &           &           & iters  \\ 
 \hdashline 
     &           &    1 &  1.51e-08 &  2.45e-09 &      \\ 
     &           &    2 &  1.51e-08 &  4.32e-16 &      \\ 
     &           &    3 &  1.51e-08 &  4.32e-16 &      \\ 
     &           &    4 &  6.26e-08 &  4.31e-16 &      \\ 
     &           &    5 &  1.52e-08 &  1.79e-15 &      \\ 
     &           &    6 &  1.51e-08 &  4.33e-16 &      \\ 
     &           &    7 &  1.51e-08 &  4.32e-16 &      \\ 
     &           &    8 &  1.51e-08 &  4.32e-16 &      \\ 
     &           &    9 &  6.26e-08 &  4.31e-16 &      \\ 
     &           &   10 &  1.52e-08 &  1.79e-15 &      \\ 
3409 &  3.41e+04 &   10 &           &           & iters  \\ 
 \hdashline 
     &           &    1 &  3.85e-08 &  2.67e-09 &      \\ 
     &           &    2 &  3.92e-08 &  1.10e-15 &      \\ 
     &           &    3 &  3.85e-08 &  1.12e-15 &      \\ 
     &           &    4 &  3.92e-08 &  1.10e-15 &      \\ 
     &           &    5 &  3.86e-08 &  1.12e-15 &      \\ 
     &           &    6 &  3.92e-08 &  1.10e-15 &      \\ 
     &           &    7 &  3.86e-08 &  1.12e-15 &      \\ 
     &           &    8 &  3.92e-08 &  1.10e-15 &      \\ 
     &           &    9 &  3.86e-08 &  1.12e-15 &      \\ 
     &           &   10 &  3.92e-08 &  1.10e-15 &      \\ 
3410 &  3.41e+04 &   10 &           &           & iters  \\ 
 \hdashline 
     &           &    1 &  5.01e-08 &  2.57e-09 &      \\ 
     &           &    2 &  2.76e-08 &  1.43e-15 &      \\ 
     &           &    3 &  2.76e-08 &  7.89e-16 &      \\ 
     &           &    4 &  5.01e-08 &  7.87e-16 &      \\ 
     &           &    5 &  2.76e-08 &  1.43e-15 &      \\ 
     &           &    6 &  2.76e-08 &  7.88e-16 &      \\ 
     &           &    7 &  5.01e-08 &  7.87e-16 &      \\ 
     &           &    8 &  2.76e-08 &  1.43e-15 &      \\ 
     &           &    9 &  2.76e-08 &  7.88e-16 &      \\ 
     &           &   10 &  5.02e-08 &  7.87e-16 &      \\ 
3411 &  3.41e+04 &   10 &           &           & iters  \\ 
 \hdashline 
     &           &    1 &  3.65e-08 &  2.59e-09 &      \\ 
     &           &    2 &  4.13e-08 &  1.04e-15 &      \\ 
     &           &    3 &  3.65e-08 &  1.18e-15 &      \\ 
     &           &    4 &  4.13e-08 &  1.04e-15 &      \\ 
     &           &    5 &  3.65e-08 &  1.18e-15 &      \\ 
     &           &    6 &  4.12e-08 &  1.04e-15 &      \\ 
     &           &    7 &  3.65e-08 &  1.18e-15 &      \\ 
     &           &    8 &  3.64e-08 &  1.04e-15 &      \\ 
     &           &    9 &  4.13e-08 &  1.04e-15 &      \\ 
     &           &   10 &  3.65e-08 &  1.18e-15 &      \\ 
3412 &  3.41e+04 &   10 &           &           & iters  \\ 
 \hdashline 
     &           &    1 &  5.87e-08 &  3.62e-10 &      \\ 
     &           &    2 &  1.90e-08 &  1.68e-15 &      \\ 
     &           &    3 &  1.90e-08 &  5.43e-16 &      \\ 
     &           &    4 &  1.90e-08 &  5.43e-16 &      \\ 
     &           &    5 &  5.87e-08 &  5.42e-16 &      \\ 
     &           &    6 &  1.90e-08 &  1.68e-15 &      \\ 
     &           &    7 &  1.90e-08 &  5.43e-16 &      \\ 
     &           &    8 &  1.90e-08 &  5.43e-16 &      \\ 
     &           &    9 &  5.87e-08 &  5.42e-16 &      \\ 
     &           &   10 &  1.90e-08 &  1.68e-15 &      \\ 
3413 &  3.41e+04 &   10 &           &           & iters  \\ 
 \hdashline 
     &           &    1 &  1.36e-08 &  3.11e-09 &      \\ 
     &           &    2 &  1.36e-08 &  3.89e-16 &      \\ 
     &           &    3 &  6.41e-08 &  3.88e-16 &      \\ 
     &           &    4 &  1.37e-08 &  1.83e-15 &      \\ 
     &           &    5 &  1.37e-08 &  3.90e-16 &      \\ 
     &           &    6 &  1.36e-08 &  3.90e-16 &      \\ 
     &           &    7 &  1.36e-08 &  3.89e-16 &      \\ 
     &           &    8 &  1.36e-08 &  3.89e-16 &      \\ 
     &           &    9 &  6.41e-08 &  3.88e-16 &      \\ 
     &           &   10 &  1.37e-08 &  1.83e-15 &      \\ 
3414 &  3.41e+04 &   10 &           &           & iters  \\ 
 \hdashline 
     &           &    1 &  1.27e-09 &  5.44e-09 &      \\ 
     &           &    2 &  1.27e-09 &  3.62e-17 &      \\ 
     &           &    3 &  1.27e-09 &  3.62e-17 &      \\ 
     &           &    4 &  1.26e-09 &  3.61e-17 &      \\ 
     &           &    5 &  1.26e-09 &  3.61e-17 &      \\ 
     &           &    6 &  1.26e-09 &  3.60e-17 &      \\ 
     &           &    7 &  1.26e-09 &  3.60e-17 &      \\ 
     &           &    8 &  1.26e-09 &  3.59e-17 &      \\ 
     &           &    9 &  1.26e-09 &  3.59e-17 &      \\ 
     &           &   10 &  1.25e-09 &  3.58e-17 &      \\ 
3415 &  3.41e+04 &   10 &           &           & iters  \\ 
 \hdashline 
     &           &    1 &  3.66e-08 &  4.28e-09 &      \\ 
     &           &    2 &  4.12e-08 &  1.04e-15 &      \\ 
     &           &    3 &  3.66e-08 &  1.18e-15 &      \\ 
     &           &    4 &  4.12e-08 &  1.04e-15 &      \\ 
     &           &    5 &  3.66e-08 &  1.17e-15 &      \\ 
     &           &    6 &  4.12e-08 &  1.04e-15 &      \\ 
     &           &    7 &  3.66e-08 &  1.17e-15 &      \\ 
     &           &    8 &  4.12e-08 &  1.04e-15 &      \\ 
     &           &    9 &  3.66e-08 &  1.17e-15 &      \\ 
     &           &   10 &  4.11e-08 &  1.04e-15 &      \\ 
3416 &  3.42e+04 &   10 &           &           & iters  \\ 
 \hdashline 
     &           &    1 &  2.84e-08 &  2.35e-09 &      \\ 
     &           &    2 &  1.05e-08 &  8.11e-16 &      \\ 
     &           &    3 &  1.05e-08 &  2.99e-16 &      \\ 
     &           &    4 &  2.84e-08 &  2.99e-16 &      \\ 
     &           &    5 &  1.05e-08 &  8.10e-16 &      \\ 
     &           &    6 &  1.05e-08 &  2.99e-16 &      \\ 
     &           &    7 &  1.05e-08 &  2.99e-16 &      \\ 
     &           &    8 &  2.84e-08 &  2.99e-16 &      \\ 
     &           &    9 &  1.05e-08 &  8.11e-16 &      \\ 
     &           &   10 &  1.05e-08 &  2.99e-16 &      \\ 
3417 &  3.42e+04 &   10 &           &           & iters  \\ 
 \hdashline 
     &           &    1 &  2.34e-08 &  3.10e-09 &      \\ 
     &           &    2 &  5.43e-08 &  6.68e-16 &      \\ 
     &           &    3 &  2.35e-08 &  1.55e-15 &      \\ 
     &           &    4 &  2.34e-08 &  6.69e-16 &      \\ 
     &           &    5 &  5.43e-08 &  6.69e-16 &      \\ 
     &           &    6 &  2.35e-08 &  1.55e-15 &      \\ 
     &           &    7 &  2.34e-08 &  6.70e-16 &      \\ 
     &           &    8 &  5.43e-08 &  6.69e-16 &      \\ 
     &           &    9 &  2.35e-08 &  1.55e-15 &      \\ 
     &           &   10 &  2.34e-08 &  6.70e-16 &      \\ 
3418 &  3.42e+04 &   10 &           &           & iters  \\ 
 \hdashline 
     &           &    1 &  1.11e-08 &  2.67e-09 &      \\ 
     &           &    2 &  1.11e-08 &  3.17e-16 &      \\ 
     &           &    3 &  6.66e-08 &  3.16e-16 &      \\ 
     &           &    4 &  8.89e-08 &  1.90e-15 &      \\ 
     &           &    5 &  6.67e-08 &  2.54e-15 &      \\ 
     &           &    6 &  1.11e-08 &  1.90e-15 &      \\ 
     &           &    7 &  1.11e-08 &  3.18e-16 &      \\ 
     &           &    8 &  1.11e-08 &  3.17e-16 &      \\ 
     &           &    9 &  1.11e-08 &  3.17e-16 &      \\ 
     &           &   10 &  6.66e-08 &  3.16e-16 &      \\ 
3419 &  3.42e+04 &   10 &           &           & iters  \\ 
 \hdashline 
     &           &    1 &  2.49e-08 &  3.19e-10 &      \\ 
     &           &    2 &  2.49e-08 &  7.12e-16 &      \\ 
     &           &    3 &  2.49e-08 &  7.11e-16 &      \\ 
     &           &    4 &  2.48e-08 &  7.10e-16 &      \\ 
     &           &    5 &  5.29e-08 &  7.09e-16 &      \\ 
     &           &    6 &  2.49e-08 &  1.51e-15 &      \\ 
     &           &    7 &  2.48e-08 &  7.10e-16 &      \\ 
     &           &    8 &  5.29e-08 &  7.09e-16 &      \\ 
     &           &    9 &  2.49e-08 &  1.51e-15 &      \\ 
     &           &   10 &  2.48e-08 &  7.10e-16 &      \\ 
3420 &  3.42e+04 &   10 &           &           & iters  \\ 
 \hdashline 
     &           &    1 &  4.63e-08 &  1.69e-10 &      \\ 
     &           &    2 &  3.15e-08 &  1.32e-15 &      \\ 
     &           &    3 &  4.62e-08 &  8.99e-16 &      \\ 
     &           &    4 &  3.15e-08 &  1.32e-15 &      \\ 
     &           &    5 &  4.62e-08 &  8.99e-16 &      \\ 
     &           &    6 &  3.15e-08 &  1.32e-15 &      \\ 
     &           &    7 &  3.15e-08 &  9.00e-16 &      \\ 
     &           &    8 &  4.63e-08 &  8.98e-16 &      \\ 
     &           &    9 &  3.15e-08 &  1.32e-15 &      \\ 
     &           &   10 &  4.62e-08 &  8.99e-16 &      \\ 
3421 &  3.42e+04 &   10 &           &           & iters  \\ 
 \hdashline 
     &           &    1 &  2.41e-10 &  1.56e-09 &      \\ 
     &           &    2 &  2.40e-10 &  6.87e-18 &      \\ 
     &           &    3 &  2.40e-10 &  6.86e-18 &      \\ 
     &           &    4 &  2.40e-10 &  6.85e-18 &      \\ 
     &           &    5 &  2.39e-10 &  6.84e-18 &      \\ 
     &           &    6 &  2.39e-10 &  6.83e-18 &      \\ 
     &           &    7 &  2.39e-10 &  6.82e-18 &      \\ 
     &           &    8 &  2.38e-10 &  6.81e-18 &      \\ 
     &           &    9 &  2.38e-10 &  6.80e-18 &      \\ 
     &           &   10 &  2.38e-10 &  6.79e-18 &      \\ 
3422 &  3.42e+04 &   10 &           &           & iters  \\ 
 \hdashline 
     &           &    1 &  7.31e-08 &  2.65e-09 &      \\ 
     &           &    2 &  4.72e-09 &  2.09e-15 &      \\ 
     &           &    3 &  4.71e-09 &  1.35e-16 &      \\ 
     &           &    4 &  4.70e-09 &  1.34e-16 &      \\ 
     &           &    5 &  4.70e-09 &  1.34e-16 &      \\ 
     &           &    6 &  4.69e-09 &  1.34e-16 &      \\ 
     &           &    7 &  4.68e-09 &  1.34e-16 &      \\ 
     &           &    8 &  4.68e-09 &  1.34e-16 &      \\ 
     &           &    9 &  4.67e-09 &  1.34e-16 &      \\ 
     &           &   10 &  4.66e-09 &  1.33e-16 &      \\ 
3423 &  3.42e+04 &   10 &           &           & iters  \\ 
 \hdashline 
     &           &    1 &  2.96e-08 &  1.77e-09 &      \\ 
     &           &    2 &  4.81e-08 &  8.46e-16 &      \\ 
     &           &    3 &  2.97e-08 &  1.37e-15 &      \\ 
     &           &    4 &  2.96e-08 &  8.46e-16 &      \\ 
     &           &    5 &  4.81e-08 &  8.45e-16 &      \\ 
     &           &    6 &  2.96e-08 &  1.37e-15 &      \\ 
     &           &    7 &  2.96e-08 &  8.46e-16 &      \\ 
     &           &    8 &  4.81e-08 &  8.45e-16 &      \\ 
     &           &    9 &  2.96e-08 &  1.37e-15 &      \\ 
     &           &   10 &  4.81e-08 &  8.46e-16 &      \\ 
3424 &  3.42e+04 &   10 &           &           & iters  \\ 
 \hdashline 
     &           &    1 &  2.22e-08 &  1.68e-09 &      \\ 
     &           &    2 &  5.55e-08 &  6.34e-16 &      \\ 
     &           &    3 &  2.23e-08 &  1.58e-15 &      \\ 
     &           &    4 &  2.22e-08 &  6.35e-16 &      \\ 
     &           &    5 &  2.22e-08 &  6.34e-16 &      \\ 
     &           &    6 &  5.55e-08 &  6.34e-16 &      \\ 
     &           &    7 &  2.22e-08 &  1.58e-15 &      \\ 
     &           &    8 &  2.22e-08 &  6.35e-16 &      \\ 
     &           &    9 &  5.55e-08 &  6.34e-16 &      \\ 
     &           &   10 &  2.23e-08 &  1.58e-15 &      \\ 
3425 &  3.42e+04 &   10 &           &           & iters  \\ 
 \hdashline 
     &           &    1 &  3.20e-08 &  4.20e-09 &      \\ 
     &           &    2 &  4.57e-08 &  9.15e-16 &      \\ 
     &           &    3 &  3.21e-08 &  1.30e-15 &      \\ 
     &           &    4 &  3.20e-08 &  9.15e-16 &      \\ 
     &           &    5 &  4.57e-08 &  9.14e-16 &      \\ 
     &           &    6 &  3.20e-08 &  1.30e-15 &      \\ 
     &           &    7 &  4.57e-08 &  9.14e-16 &      \\ 
     &           &    8 &  3.21e-08 &  1.30e-15 &      \\ 
     &           &    9 &  4.57e-08 &  9.15e-16 &      \\ 
     &           &   10 &  3.21e-08 &  1.30e-15 &      \\ 
3426 &  3.42e+04 &   10 &           &           & iters  \\ 
 \hdashline 
     &           &    1 &  5.59e-08 &  1.94e-09 &      \\ 
     &           &    2 &  2.18e-08 &  1.60e-15 &      \\ 
     &           &    3 &  2.18e-08 &  6.23e-16 &      \\ 
     &           &    4 &  5.59e-08 &  6.22e-16 &      \\ 
     &           &    5 &  2.18e-08 &  1.60e-15 &      \\ 
     &           &    6 &  2.18e-08 &  6.23e-16 &      \\ 
     &           &    7 &  5.59e-08 &  6.22e-16 &      \\ 
     &           &    8 &  2.19e-08 &  1.60e-15 &      \\ 
     &           &    9 &  2.18e-08 &  6.24e-16 &      \\ 
     &           &   10 &  2.18e-08 &  6.23e-16 &      \\ 
3427 &  3.43e+04 &   10 &           &           & iters  \\ 
 \hdashline 
     &           &    1 &  1.81e-09 &  1.45e-10 &      \\ 
     &           &    2 &  1.80e-09 &  5.15e-17 &      \\ 
     &           &    3 &  1.80e-09 &  5.15e-17 &      \\ 
     &           &    4 &  1.80e-09 &  5.14e-17 &      \\ 
     &           &    5 &  1.80e-09 &  5.13e-17 &      \\ 
     &           &    6 &  1.79e-09 &  5.12e-17 &      \\ 
     &           &    7 &  1.79e-09 &  5.12e-17 &      \\ 
     &           &    8 &  1.79e-09 &  5.11e-17 &      \\ 
     &           &    9 &  1.79e-09 &  5.10e-17 &      \\ 
     &           &   10 &  1.78e-09 &  5.09e-17 &      \\ 
3428 &  3.43e+04 &   10 &           &           & iters  \\ 
 \hdashline 
     &           &    1 &  4.45e-08 &  7.01e-10 &      \\ 
     &           &    2 &  3.32e-08 &  1.27e-15 &      \\ 
     &           &    3 &  3.32e-08 &  9.48e-16 &      \\ 
     &           &    4 &  4.46e-08 &  9.47e-16 &      \\ 
     &           &    5 &  3.32e-08 &  1.27e-15 &      \\ 
     &           &    6 &  4.45e-08 &  9.47e-16 &      \\ 
     &           &    7 &  3.32e-08 &  1.27e-15 &      \\ 
     &           &    8 &  4.45e-08 &  9.48e-16 &      \\ 
     &           &    9 &  3.32e-08 &  1.27e-15 &      \\ 
     &           &   10 &  3.32e-08 &  9.48e-16 &      \\ 
3429 &  3.43e+04 &   10 &           &           & iters  \\ 
 \hdashline 
     &           &    1 &  1.45e-08 &  1.75e-09 &      \\ 
     &           &    2 &  1.45e-08 &  4.14e-16 &      \\ 
     &           &    3 &  1.45e-08 &  4.14e-16 &      \\ 
     &           &    4 &  1.45e-08 &  4.13e-16 &      \\ 
     &           &    5 &  1.44e-08 &  4.13e-16 &      \\ 
     &           &    6 &  6.33e-08 &  4.12e-16 &      \\ 
     &           &    7 &  1.45e-08 &  1.81e-15 &      \\ 
     &           &    8 &  1.45e-08 &  4.14e-16 &      \\ 
     &           &    9 &  1.45e-08 &  4.13e-16 &      \\ 
     &           &   10 &  1.44e-08 &  4.13e-16 &      \\ 
3430 &  3.43e+04 &   10 &           &           & iters  \\ 
 \hdashline 
     &           &    1 &  6.32e-08 &  3.48e-09 &      \\ 
     &           &    2 &  1.46e-08 &  1.80e-15 &      \\ 
     &           &    3 &  1.46e-08 &  4.17e-16 &      \\ 
     &           &    4 &  1.46e-08 &  4.16e-16 &      \\ 
     &           &    5 &  1.45e-08 &  4.15e-16 &      \\ 
     &           &    6 &  6.32e-08 &  4.15e-16 &      \\ 
     &           &    7 &  1.46e-08 &  1.80e-15 &      \\ 
     &           &    8 &  1.46e-08 &  4.17e-16 &      \\ 
     &           &    9 &  1.46e-08 &  4.16e-16 &      \\ 
     &           &   10 &  1.45e-08 &  4.16e-16 &      \\ 
3431 &  3.43e+04 &   10 &           &           & iters  \\ 
 \hdashline 
     &           &    1 &  4.38e-08 &  1.54e-09 &      \\ 
     &           &    2 &  3.39e-08 &  1.25e-15 &      \\ 
     &           &    3 &  4.38e-08 &  9.68e-16 &      \\ 
     &           &    4 &  3.39e-08 &  1.25e-15 &      \\ 
     &           &    5 &  3.39e-08 &  9.68e-16 &      \\ 
     &           &    6 &  4.39e-08 &  9.67e-16 &      \\ 
     &           &    7 &  3.39e-08 &  1.25e-15 &      \\ 
     &           &    8 &  4.38e-08 &  9.68e-16 &      \\ 
     &           &    9 &  3.39e-08 &  1.25e-15 &      \\ 
     &           &   10 &  4.38e-08 &  9.68e-16 &      \\ 
3432 &  3.43e+04 &   10 &           &           & iters  \\ 
 \hdashline 
     &           &    1 &  1.97e-09 &  2.91e-09 &      \\ 
     &           &    2 &  3.69e-08 &  5.61e-17 &      \\ 
     &           &    3 &  2.02e-09 &  1.05e-15 &      \\ 
     &           &    4 &  2.01e-09 &  5.75e-17 &      \\ 
     &           &    5 &  2.01e-09 &  5.74e-17 &      \\ 
     &           &    6 &  2.01e-09 &  5.74e-17 &      \\ 
     &           &    7 &  2.00e-09 &  5.73e-17 &      \\ 
     &           &    8 &  2.00e-09 &  5.72e-17 &      \\ 
     &           &    9 &  2.00e-09 &  5.71e-17 &      \\ 
     &           &   10 &  2.00e-09 &  5.70e-17 &      \\ 
3433 &  3.43e+04 &   10 &           &           & iters  \\ 
 \hdashline 
     &           &    1 &  5.57e-10 &  2.69e-09 &      \\ 
     &           &    2 &  5.56e-10 &  1.59e-17 &      \\ 
     &           &    3 &  5.55e-10 &  1.59e-17 &      \\ 
     &           &    4 &  5.54e-10 &  1.58e-17 &      \\ 
     &           &    5 &  5.54e-10 &  1.58e-17 &      \\ 
     &           &    6 &  5.53e-10 &  1.58e-17 &      \\ 
     &           &    7 &  5.52e-10 &  1.58e-17 &      \\ 
     &           &    8 &  5.51e-10 &  1.58e-17 &      \\ 
     &           &    9 &  5.50e-10 &  1.57e-17 &      \\ 
     &           &   10 &  5.50e-10 &  1.57e-17 &      \\ 
3434 &  3.43e+04 &   10 &           &           & iters  \\ 
 \hdashline 
     &           &    1 &  8.52e-09 &  4.01e-10 &      \\ 
     &           &    2 &  8.51e-09 &  2.43e-16 &      \\ 
     &           &    3 &  8.50e-09 &  2.43e-16 &      \\ 
     &           &    4 &  8.49e-09 &  2.43e-16 &      \\ 
     &           &    5 &  8.47e-09 &  2.42e-16 &      \\ 
     &           &    6 &  8.46e-09 &  2.42e-16 &      \\ 
     &           &    7 &  6.92e-08 &  2.42e-16 &      \\ 
     &           &    8 &  8.62e-08 &  1.98e-15 &      \\ 
     &           &    9 &  6.93e-08 &  2.46e-15 &      \\ 
     &           &   10 &  8.53e-09 &  1.98e-15 &      \\ 
3435 &  3.43e+04 &   10 &           &           & iters  \\ 
 \hdashline 
     &           &    1 &  1.88e-08 &  2.49e-10 &      \\ 
     &           &    2 &  1.87e-08 &  5.36e-16 &      \\ 
     &           &    3 &  1.87e-08 &  5.35e-16 &      \\ 
     &           &    4 &  5.90e-08 &  5.34e-16 &      \\ 
     &           &    5 &  1.88e-08 &  1.68e-15 &      \\ 
     &           &    6 &  1.87e-08 &  5.36e-16 &      \\ 
     &           &    7 &  1.87e-08 &  5.35e-16 &      \\ 
     &           &    8 &  5.90e-08 &  5.34e-16 &      \\ 
     &           &    9 &  1.88e-08 &  1.68e-15 &      \\ 
     &           &   10 &  1.88e-08 &  5.36e-16 &      \\ 
3436 &  3.44e+04 &   10 &           &           & iters  \\ 
 \hdashline 
     &           &    1 &  7.35e-08 &  2.71e-10 &      \\ 
     &           &    2 &  4.32e-09 &  2.10e-15 &      \\ 
     &           &    3 &  4.31e-09 &  1.23e-16 &      \\ 
     &           &    4 &  4.31e-09 &  1.23e-16 &      \\ 
     &           &    5 &  4.30e-09 &  1.23e-16 &      \\ 
     &           &    6 &  4.30e-09 &  1.23e-16 &      \\ 
     &           &    7 &  4.29e-09 &  1.23e-16 &      \\ 
     &           &    8 &  4.28e-09 &  1.22e-16 &      \\ 
     &           &    9 &  4.28e-09 &  1.22e-16 &      \\ 
     &           &   10 &  4.27e-09 &  1.22e-16 &      \\ 
3437 &  3.44e+04 &   10 &           &           & iters  \\ 
 \hdashline 
     &           &    1 &  1.46e-08 &  3.76e-10 &      \\ 
     &           &    2 &  1.46e-08 &  4.17e-16 &      \\ 
     &           &    3 &  6.31e-08 &  4.17e-16 &      \\ 
     &           &    4 &  1.47e-08 &  1.80e-15 &      \\ 
     &           &    5 &  1.47e-08 &  4.19e-16 &      \\ 
     &           &    6 &  1.46e-08 &  4.18e-16 &      \\ 
     &           &    7 &  1.46e-08 &  4.18e-16 &      \\ 
     &           &    8 &  6.31e-08 &  4.17e-16 &      \\ 
     &           &    9 &  1.47e-08 &  1.80e-15 &      \\ 
     &           &   10 &  1.47e-08 &  4.19e-16 &      \\ 
3438 &  3.44e+04 &   10 &           &           & iters  \\ 
 \hdashline 
     &           &    1 &  2.70e-09 &  1.26e-09 &      \\ 
     &           &    2 &  7.50e-08 &  7.69e-17 &      \\ 
     &           &    3 &  8.05e-08 &  2.14e-15 &      \\ 
     &           &    4 &  7.50e-08 &  2.30e-15 &      \\ 
     &           &    5 &  8.05e-08 &  2.14e-15 &      \\ 
     &           &    6 &  7.50e-08 &  2.30e-15 &      \\ 
     &           &    7 &  8.05e-08 &  2.14e-15 &      \\ 
     &           &    8 &  7.50e-08 &  2.30e-15 &      \\ 
     &           &    9 &  2.78e-09 &  2.14e-15 &      \\ 
     &           &   10 &  2.77e-09 &  7.92e-17 &      \\ 
3439 &  3.44e+04 &   10 &           &           & iters  \\ 
 \hdashline 
     &           &    1 &  8.55e-08 &  1.35e-09 &      \\ 
     &           &    2 &  7.00e-08 &  2.44e-15 &      \\ 
     &           &    3 &  8.55e-08 &  2.00e-15 &      \\ 
     &           &    4 &  7.00e-08 &  2.44e-15 &      \\ 
     &           &    5 &  7.75e-09 &  2.00e-15 &      \\ 
     &           &    6 &  7.74e-09 &  2.21e-16 &      \\ 
     &           &    7 &  7.73e-09 &  2.21e-16 &      \\ 
     &           &    8 &  7.72e-09 &  2.21e-16 &      \\ 
     &           &    9 &  7.71e-09 &  2.20e-16 &      \\ 
     &           &   10 &  7.70e-09 &  2.20e-16 &      \\ 
3440 &  3.44e+04 &   10 &           &           & iters  \\ 
 \hdashline 
     &           &    1 &  7.56e-09 &  1.35e-11 &      \\ 
     &           &    2 &  7.55e-09 &  2.16e-16 &      \\ 
     &           &    3 &  7.54e-09 &  2.16e-16 &      \\ 
     &           &    4 &  7.53e-09 &  2.15e-16 &      \\ 
     &           &    5 &  7.52e-09 &  2.15e-16 &      \\ 
     &           &    6 &  7.51e-09 &  2.15e-16 &      \\ 
     &           &    7 &  7.50e-09 &  2.14e-16 &      \\ 
     &           &    8 &  7.02e-08 &  2.14e-16 &      \\ 
     &           &    9 &  8.53e-08 &  2.00e-15 &      \\ 
     &           &   10 &  7.02e-08 &  2.43e-15 &      \\ 
3441 &  3.44e+04 &   10 &           &           & iters  \\ 
 \hdashline 
     &           &    1 &  7.00e-08 &  9.73e-10 &      \\ 
     &           &    2 &  7.82e-09 &  2.00e-15 &      \\ 
     &           &    3 &  7.80e-09 &  2.23e-16 &      \\ 
     &           &    4 &  7.79e-09 &  2.23e-16 &      \\ 
     &           &    5 &  7.78e-09 &  2.22e-16 &      \\ 
     &           &    6 &  7.77e-09 &  2.22e-16 &      \\ 
     &           &    7 &  7.76e-09 &  2.22e-16 &      \\ 
     &           &    8 &  7.75e-09 &  2.21e-16 &      \\ 
     &           &    9 &  7.00e-08 &  2.21e-16 &      \\ 
     &           &   10 &  8.55e-08 &  2.00e-15 &      \\ 
3442 &  3.44e+04 &   10 &           &           & iters  \\ 
 \hdashline 
     &           &    1 &  4.83e-08 &  4.76e-10 &      \\ 
     &           &    2 &  2.94e-08 &  1.38e-15 &      \\ 
     &           &    3 &  4.83e-08 &  8.40e-16 &      \\ 
     &           &    4 &  2.94e-08 &  1.38e-15 &      \\ 
     &           &    5 &  4.83e-08 &  8.40e-16 &      \\ 
     &           &    6 &  2.95e-08 &  1.38e-15 &      \\ 
     &           &    7 &  2.94e-08 &  8.41e-16 &      \\ 
     &           &    8 &  4.83e-08 &  8.40e-16 &      \\ 
     &           &    9 &  2.95e-08 &  1.38e-15 &      \\ 
     &           &   10 &  2.94e-08 &  8.41e-16 &      \\ 
3443 &  3.44e+04 &   10 &           &           & iters  \\ 
 \hdashline 
     &           &    1 &  2.94e-08 &  4.05e-10 &      \\ 
     &           &    2 &  4.83e-08 &  8.39e-16 &      \\ 
     &           &    3 &  2.94e-08 &  1.38e-15 &      \\ 
     &           &    4 &  2.94e-08 &  8.40e-16 &      \\ 
     &           &    5 &  4.83e-08 &  8.39e-16 &      \\ 
     &           &    6 &  2.94e-08 &  1.38e-15 &      \\ 
     &           &    7 &  2.94e-08 &  8.40e-16 &      \\ 
     &           &    8 &  4.84e-08 &  8.38e-16 &      \\ 
     &           &    9 &  2.94e-08 &  1.38e-15 &      \\ 
     &           &   10 &  4.83e-08 &  8.39e-16 &      \\ 
3444 &  3.44e+04 &   10 &           &           & iters  \\ 
 \hdashline 
     &           &    1 &  5.45e-08 &  1.32e-09 &      \\ 
     &           &    2 &  2.32e-08 &  1.56e-15 &      \\ 
     &           &    3 &  5.45e-08 &  6.63e-16 &      \\ 
     &           &    4 &  2.33e-08 &  1.56e-15 &      \\ 
     &           &    5 &  2.32e-08 &  6.64e-16 &      \\ 
     &           &    6 &  2.32e-08 &  6.63e-16 &      \\ 
     &           &    7 &  5.45e-08 &  6.62e-16 &      \\ 
     &           &    8 &  2.33e-08 &  1.56e-15 &      \\ 
     &           &    9 &  2.32e-08 &  6.64e-16 &      \\ 
     &           &   10 &  5.45e-08 &  6.63e-16 &      \\ 
3445 &  3.44e+04 &   10 &           &           & iters  \\ 
 \hdashline 
     &           &    1 &  4.75e-08 &  1.60e-09 &      \\ 
     &           &    2 &  8.62e-09 &  1.36e-15 &      \\ 
     &           &    3 &  8.61e-09 &  2.46e-16 &      \\ 
     &           &    4 &  8.60e-09 &  2.46e-16 &      \\ 
     &           &    5 &  8.59e-09 &  2.45e-16 &      \\ 
     &           &    6 &  6.91e-08 &  2.45e-16 &      \\ 
     &           &    7 &  4.75e-08 &  1.97e-15 &      \\ 
     &           &    8 &  8.60e-09 &  1.36e-15 &      \\ 
     &           &    9 &  8.59e-09 &  2.46e-16 &      \\ 
     &           &   10 &  6.91e-08 &  2.45e-16 &      \\ 
3446 &  3.44e+04 &   10 &           &           & iters  \\ 
 \hdashline 
     &           &    1 &  5.37e-08 &  1.17e-09 &      \\ 
     &           &    2 &  2.41e-08 &  1.53e-15 &      \\ 
     &           &    3 &  2.40e-08 &  6.87e-16 &      \\ 
     &           &    4 &  5.37e-08 &  6.86e-16 &      \\ 
     &           &    5 &  2.41e-08 &  1.53e-15 &      \\ 
     &           &    6 &  2.40e-08 &  6.87e-16 &      \\ 
     &           &    7 &  5.37e-08 &  6.86e-16 &      \\ 
     &           &    8 &  2.41e-08 &  1.53e-15 &      \\ 
     &           &    9 &  2.40e-08 &  6.87e-16 &      \\ 
     &           &   10 &  2.40e-08 &  6.86e-16 &      \\ 
3447 &  3.45e+04 &   10 &           &           & iters  \\ 
 \hdashline 
     &           &    1 &  4.55e-08 &  1.11e-09 &      \\ 
     &           &    2 &  6.57e-09 &  1.30e-15 &      \\ 
     &           &    3 &  6.56e-09 &  1.88e-16 &      \\ 
     &           &    4 &  6.55e-09 &  1.87e-16 &      \\ 
     &           &    5 &  6.54e-09 &  1.87e-16 &      \\ 
     &           &    6 &  6.53e-09 &  1.87e-16 &      \\ 
     &           &    7 &  7.12e-08 &  1.86e-16 &      \\ 
     &           &    8 &  4.55e-08 &  2.03e-15 &      \\ 
     &           &    9 &  6.56e-09 &  1.30e-15 &      \\ 
     &           &   10 &  6.55e-09 &  1.87e-16 &      \\ 
3448 &  3.45e+04 &   10 &           &           & iters  \\ 
 \hdashline 
     &           &    1 &  5.03e-08 &  7.84e-10 &      \\ 
     &           &    2 &  2.75e-08 &  1.44e-15 &      \\ 
     &           &    3 &  2.74e-08 &  7.84e-16 &      \\ 
     &           &    4 &  5.03e-08 &  7.83e-16 &      \\ 
     &           &    5 &  2.75e-08 &  1.44e-15 &      \\ 
     &           &    6 &  2.74e-08 &  7.84e-16 &      \\ 
     &           &    7 &  5.03e-08 &  7.83e-16 &      \\ 
     &           &    8 &  2.75e-08 &  1.44e-15 &      \\ 
     &           &    9 &  5.03e-08 &  7.84e-16 &      \\ 
     &           &   10 &  2.75e-08 &  1.43e-15 &      \\ 
3449 &  3.45e+04 &   10 &           &           & iters  \\ 
 \hdashline 
     &           &    1 &  3.63e-08 &  4.16e-11 &      \\ 
     &           &    2 &  4.14e-08 &  1.04e-15 &      \\ 
     &           &    3 &  3.63e-08 &  1.18e-15 &      \\ 
     &           &    4 &  4.14e-08 &  1.04e-15 &      \\ 
     &           &    5 &  3.64e-08 &  1.18e-15 &      \\ 
     &           &    6 &  4.14e-08 &  1.04e-15 &      \\ 
     &           &    7 &  3.64e-08 &  1.18e-15 &      \\ 
     &           &    8 &  4.14e-08 &  1.04e-15 &      \\ 
     &           &    9 &  3.64e-08 &  1.18e-15 &      \\ 
     &           &   10 &  4.14e-08 &  1.04e-15 &      \\ 
3450 &  3.45e+04 &   10 &           &           & iters  \\ 
 \hdashline 
     &           &    1 &  2.44e-08 &  2.76e-10 &      \\ 
     &           &    2 &  2.43e-08 &  6.95e-16 &      \\ 
     &           &    3 &  1.46e-08 &  6.94e-16 &      \\ 
     &           &    4 &  1.45e-08 &  4.16e-16 &      \\ 
     &           &    5 &  2.43e-08 &  4.15e-16 &      \\ 
     &           &    6 &  1.46e-08 &  6.94e-16 &      \\ 
     &           &    7 &  2.43e-08 &  4.15e-16 &      \\ 
     &           &    8 &  1.46e-08 &  6.94e-16 &      \\ 
     &           &    9 &  1.45e-08 &  4.16e-16 &      \\ 
     &           &   10 &  2.43e-08 &  4.15e-16 &      \\ 
3451 &  3.45e+04 &   10 &           &           & iters  \\ 
 \hdashline 
     &           &    1 &  3.98e-08 &  4.56e-10 &      \\ 
     &           &    2 &  3.80e-08 &  1.14e-15 &      \\ 
     &           &    3 &  3.98e-08 &  1.08e-15 &      \\ 
     &           &    4 &  3.80e-08 &  1.14e-15 &      \\ 
     &           &    5 &  3.98e-08 &  1.08e-15 &      \\ 
     &           &    6 &  3.80e-08 &  1.14e-15 &      \\ 
     &           &    7 &  3.98e-08 &  1.08e-15 &      \\ 
     &           &    8 &  3.80e-08 &  1.14e-15 &      \\ 
     &           &    9 &  3.98e-08 &  1.08e-15 &      \\ 
     &           &   10 &  3.80e-08 &  1.14e-15 &      \\ 
3452 &  3.45e+04 &   10 &           &           & iters  \\ 
 \hdashline 
     &           &    1 &  5.33e-09 &  5.90e-10 &      \\ 
     &           &    2 &  5.32e-09 &  1.52e-16 &      \\ 
     &           &    3 &  5.32e-09 &  1.52e-16 &      \\ 
     &           &    4 &  5.31e-09 &  1.52e-16 &      \\ 
     &           &    5 &  3.35e-08 &  1.52e-16 &      \\ 
     &           &    6 &  5.35e-09 &  9.57e-16 &      \\ 
     &           &    7 &  5.34e-09 &  1.53e-16 &      \\ 
     &           &    8 &  5.33e-09 &  1.52e-16 &      \\ 
     &           &    9 &  5.33e-09 &  1.52e-16 &      \\ 
     &           &   10 &  5.32e-09 &  1.52e-16 &      \\ 
3453 &  3.45e+04 &   10 &           &           & iters  \\ 
 \hdashline 
     &           &    1 &  7.59e-09 &  1.63e-09 &      \\ 
     &           &    2 &  7.58e-09 &  2.17e-16 &      \\ 
     &           &    3 &  7.56e-09 &  2.16e-16 &      \\ 
     &           &    4 &  7.55e-09 &  2.16e-16 &      \\ 
     &           &    5 &  7.01e-08 &  2.16e-16 &      \\ 
     &           &    6 &  4.65e-08 &  2.00e-15 &      \\ 
     &           &    7 &  7.58e-09 &  1.33e-15 &      \\ 
     &           &    8 &  7.57e-09 &  2.16e-16 &      \\ 
     &           &    9 &  7.56e-09 &  2.16e-16 &      \\ 
     &           &   10 &  7.01e-08 &  2.16e-16 &      \\ 
3454 &  3.45e+04 &   10 &           &           & iters  \\ 
 \hdashline 
     &           &    1 &  1.69e-09 &  1.99e-09 &      \\ 
     &           &    2 &  1.69e-09 &  4.83e-17 &      \\ 
     &           &    3 &  1.69e-09 &  4.82e-17 &      \\ 
     &           &    4 &  1.68e-09 &  4.81e-17 &      \\ 
     &           &    5 &  1.68e-09 &  4.81e-17 &      \\ 
     &           &    6 &  1.68e-09 &  4.80e-17 &      \\ 
     &           &    7 &  1.68e-09 &  4.79e-17 &      \\ 
     &           &    8 &  1.67e-09 &  4.78e-17 &      \\ 
     &           &    9 &  1.67e-09 &  4.78e-17 &      \\ 
     &           &   10 &  1.67e-09 &  4.77e-17 &      \\ 
3455 &  3.45e+04 &   10 &           &           & iters  \\ 
 \hdashline 
     &           &    1 &  2.48e-08 &  1.35e-09 &      \\ 
     &           &    2 &  2.47e-08 &  7.07e-16 &      \\ 
     &           &    3 &  5.30e-08 &  7.06e-16 &      \\ 
     &           &    4 &  2.48e-08 &  1.51e-15 &      \\ 
     &           &    5 &  2.48e-08 &  7.07e-16 &      \\ 
     &           &    6 &  5.30e-08 &  7.06e-16 &      \\ 
     &           &    7 &  2.48e-08 &  1.51e-15 &      \\ 
     &           &    8 &  2.48e-08 &  7.08e-16 &      \\ 
     &           &    9 &  5.30e-08 &  7.07e-16 &      \\ 
     &           &   10 &  2.48e-08 &  1.51e-15 &      \\ 
3456 &  3.46e+04 &   10 &           &           & iters  \\ 
 \hdashline 
     &           &    1 &  9.41e-10 &  2.25e-10 &      \\ 
     &           &    2 &  9.40e-10 &  2.69e-17 &      \\ 
     &           &    3 &  9.38e-10 &  2.68e-17 &      \\ 
     &           &    4 &  9.37e-10 &  2.68e-17 &      \\ 
     &           &    5 &  9.36e-10 &  2.67e-17 &      \\ 
     &           &    6 &  9.34e-10 &  2.67e-17 &      \\ 
     &           &    7 &  9.33e-10 &  2.67e-17 &      \\ 
     &           &    8 &  9.32e-10 &  2.66e-17 &      \\ 
     &           &    9 &  9.30e-10 &  2.66e-17 &      \\ 
     &           &   10 &  9.29e-10 &  2.66e-17 &      \\ 
3457 &  3.46e+04 &   10 &           &           & iters  \\ 
 \hdashline 
     &           &    1 &  2.43e-08 &  5.54e-11 &      \\ 
     &           &    2 &  5.35e-08 &  6.93e-16 &      \\ 
     &           &    3 &  2.43e-08 &  1.53e-15 &      \\ 
     &           &    4 &  2.43e-08 &  6.94e-16 &      \\ 
     &           &    5 &  5.34e-08 &  6.93e-16 &      \\ 
     &           &    6 &  2.43e-08 &  1.53e-15 &      \\ 
     &           &    7 &  2.43e-08 &  6.94e-16 &      \\ 
     &           &    8 &  5.34e-08 &  6.93e-16 &      \\ 
     &           &    9 &  2.43e-08 &  1.53e-15 &      \\ 
     &           &   10 &  2.43e-08 &  6.94e-16 &      \\ 
3458 &  3.46e+04 &   10 &           &           & iters  \\ 
 \hdashline 
     &           &    1 &  6.45e-11 &  1.21e-10 &      \\ 
     &           &    2 &  6.44e-11 &  1.84e-18 &      \\ 
     &           &    3 &  6.43e-11 &  1.84e-18 &      \\ 
     &           &    4 &  6.42e-11 &  1.84e-18 &      \\ 
     &           &    5 &  6.41e-11 &  1.83e-18 &      \\ 
     &           &    6 &  6.40e-11 &  1.83e-18 &      \\ 
     &           &    7 &  6.39e-11 &  1.83e-18 &      \\ 
     &           &    8 &  6.38e-11 &  1.82e-18 &      \\ 
     &           &    9 &  6.37e-11 &  1.82e-18 &      \\ 
     &           &   10 &  6.37e-11 &  1.82e-18 &      \\ 
3459 &  3.46e+04 &   10 &           &           & iters  \\ 
 \hdashline 
     &           &    1 &  9.93e-09 &  1.30e-10 &      \\ 
     &           &    2 &  9.92e-09 &  2.83e-16 &      \\ 
     &           &    3 &  9.90e-09 &  2.83e-16 &      \\ 
     &           &    4 &  6.78e-08 &  2.83e-16 &      \\ 
     &           &    5 &  9.99e-09 &  1.94e-15 &      \\ 
     &           &    6 &  9.97e-09 &  2.85e-16 &      \\ 
     &           &    7 &  9.96e-09 &  2.85e-16 &      \\ 
     &           &    8 &  9.94e-09 &  2.84e-16 &      \\ 
     &           &    9 &  9.93e-09 &  2.84e-16 &      \\ 
     &           &   10 &  9.91e-09 &  2.83e-16 &      \\ 
3460 &  3.46e+04 &   10 &           &           & iters  \\ 
 \hdashline 
     &           &    1 &  2.41e-09 &  4.28e-10 &      \\ 
     &           &    2 &  2.41e-09 &  6.88e-17 &      \\ 
     &           &    3 &  2.40e-09 &  6.87e-17 &      \\ 
     &           &    4 &  2.40e-09 &  6.86e-17 &      \\ 
     &           &    5 &  2.40e-09 &  6.85e-17 &      \\ 
     &           &    6 &  2.39e-09 &  6.84e-17 &      \\ 
     &           &    7 &  2.39e-09 &  6.83e-17 &      \\ 
     &           &    8 &  2.39e-09 &  6.82e-17 &      \\ 
     &           &    9 &  2.38e-09 &  6.81e-17 &      \\ 
     &           &   10 &  2.38e-09 &  6.80e-17 &      \\ 
3461 &  3.46e+04 &   10 &           &           & iters  \\ 
 \hdashline 
     &           &    1 &  1.11e-08 &  3.84e-10 &      \\ 
     &           &    2 &  1.11e-08 &  3.17e-16 &      \\ 
     &           &    3 &  1.11e-08 &  3.17e-16 &      \\ 
     &           &    4 &  1.11e-08 &  3.16e-16 &      \\ 
     &           &    5 &  1.10e-08 &  3.16e-16 &      \\ 
     &           &    6 &  6.67e-08 &  3.15e-16 &      \\ 
     &           &    7 &  1.11e-08 &  1.90e-15 &      \\ 
     &           &    8 &  1.11e-08 &  3.17e-16 &      \\ 
     &           &    9 &  1.11e-08 &  3.17e-16 &      \\ 
     &           &   10 &  1.11e-08 &  3.17e-16 &      \\ 
3462 &  3.46e+04 &   10 &           &           & iters  \\ 
 \hdashline 
     &           &    1 &  6.25e-08 &  3.94e-10 &      \\ 
     &           &    2 &  1.52e-08 &  1.79e-15 &      \\ 
     &           &    3 &  1.52e-08 &  4.35e-16 &      \\ 
     &           &    4 &  1.52e-08 &  4.34e-16 &      \\ 
     &           &    5 &  1.52e-08 &  4.33e-16 &      \\ 
     &           &    6 &  6.25e-08 &  4.33e-16 &      \\ 
     &           &    7 &  1.52e-08 &  1.79e-15 &      \\ 
     &           &    8 &  1.52e-08 &  4.35e-16 &      \\ 
     &           &    9 &  1.52e-08 &  4.34e-16 &      \\ 
     &           &   10 &  1.52e-08 &  4.34e-16 &      \\ 
3463 &  3.46e+04 &   10 &           &           & iters  \\ 
 \hdashline 
     &           &    1 &  6.37e-09 &  6.53e-10 &      \\ 
     &           &    2 &  6.36e-09 &  1.82e-16 &      \\ 
     &           &    3 &  6.35e-09 &  1.82e-16 &      \\ 
     &           &    4 &  6.35e-09 &  1.81e-16 &      \\ 
     &           &    5 &  6.34e-09 &  1.81e-16 &      \\ 
     &           &    6 &  7.14e-08 &  1.81e-16 &      \\ 
     &           &    7 &  6.43e-09 &  2.04e-15 &      \\ 
     &           &    8 &  6.42e-09 &  1.84e-16 &      \\ 
     &           &    9 &  6.41e-09 &  1.83e-16 &      \\ 
     &           &   10 &  6.40e-09 &  1.83e-16 &      \\ 
3464 &  3.46e+04 &   10 &           &           & iters  \\ 
 \hdashline 
     &           &    1 &  9.58e-09 &  3.30e-10 &      \\ 
     &           &    2 &  6.81e-08 &  2.74e-16 &      \\ 
     &           &    3 &  8.74e-08 &  1.94e-15 &      \\ 
     &           &    4 &  6.81e-08 &  2.49e-15 &      \\ 
     &           &    5 &  9.64e-09 &  1.94e-15 &      \\ 
     &           &    6 &  9.63e-09 &  2.75e-16 &      \\ 
     &           &    7 &  9.61e-09 &  2.75e-16 &      \\ 
     &           &    8 &  9.60e-09 &  2.74e-16 &      \\ 
     &           &    9 &  9.59e-09 &  2.74e-16 &      \\ 
     &           &   10 &  6.81e-08 &  2.74e-16 &      \\ 
3465 &  3.46e+04 &   10 &           &           & iters  \\ 
 \hdashline 
     &           &    1 &  2.59e-08 &  5.36e-10 &      \\ 
     &           &    2 &  2.59e-08 &  7.39e-16 &      \\ 
     &           &    3 &  5.19e-08 &  7.38e-16 &      \\ 
     &           &    4 &  2.59e-08 &  1.48e-15 &      \\ 
     &           &    5 &  2.59e-08 &  7.39e-16 &      \\ 
     &           &    6 &  5.19e-08 &  7.38e-16 &      \\ 
     &           &    7 &  2.59e-08 &  1.48e-15 &      \\ 
     &           &    8 &  2.59e-08 &  7.39e-16 &      \\ 
     &           &    9 &  5.19e-08 &  7.38e-16 &      \\ 
     &           &   10 &  2.59e-08 &  1.48e-15 &      \\ 
3466 &  3.46e+04 &   10 &           &           & iters  \\ 
 \hdashline 
     &           &    1 &  4.58e-09 &  5.68e-10 &      \\ 
     &           &    2 &  4.57e-09 &  1.31e-16 &      \\ 
     &           &    3 &  4.56e-09 &  1.30e-16 &      \\ 
     &           &    4 &  4.56e-09 &  1.30e-16 &      \\ 
     &           &    5 &  4.55e-09 &  1.30e-16 &      \\ 
     &           &    6 &  7.31e-08 &  1.30e-16 &      \\ 
     &           &    7 &  8.23e-08 &  2.09e-15 &      \\ 
     &           &    8 &  7.32e-08 &  2.35e-15 &      \\ 
     &           &    9 &  8.23e-08 &  2.09e-15 &      \\ 
     &           &   10 &  7.32e-08 &  2.35e-15 &      \\ 
3467 &  3.47e+04 &   10 &           &           & iters  \\ 
 \hdashline 
     &           &    1 &  3.97e-08 &  1.95e-10 &      \\ 
     &           &    2 &  3.81e-08 &  1.13e-15 &      \\ 
     &           &    3 &  3.96e-08 &  1.09e-15 &      \\ 
     &           &    4 &  3.81e-08 &  1.13e-15 &      \\ 
     &           &    5 &  3.96e-08 &  1.09e-15 &      \\ 
     &           &    6 &  3.81e-08 &  1.13e-15 &      \\ 
     &           &    7 &  3.96e-08 &  1.09e-15 &      \\ 
     &           &    8 &  3.81e-08 &  1.13e-15 &      \\ 
     &           &    9 &  3.96e-08 &  1.09e-15 &      \\ 
     &           &   10 &  3.81e-08 &  1.13e-15 &      \\ 
3468 &  3.47e+04 &   10 &           &           & iters  \\ 
 \hdashline 
     &           &    1 &  3.50e-08 &  5.36e-11 &      \\ 
     &           &    2 &  4.27e-08 &  1.00e-15 &      \\ 
     &           &    3 &  3.50e-08 &  1.22e-15 &      \\ 
     &           &    4 &  4.27e-08 &  1.00e-15 &      \\ 
     &           &    5 &  3.51e-08 &  1.22e-15 &      \\ 
     &           &    6 &  4.27e-08 &  1.00e-15 &      \\ 
     &           &    7 &  3.51e-08 &  1.22e-15 &      \\ 
     &           &    8 &  3.50e-08 &  1.00e-15 &      \\ 
     &           &    9 &  4.27e-08 &  1.00e-15 &      \\ 
     &           &   10 &  3.50e-08 &  1.22e-15 &      \\ 
3469 &  3.47e+04 &   10 &           &           & iters  \\ 
 \hdashline 
     &           &    1 &  1.55e-09 &  1.41e-09 &      \\ 
     &           &    2 &  1.55e-09 &  4.44e-17 &      \\ 
     &           &    3 &  1.55e-09 &  4.43e-17 &      \\ 
     &           &    4 &  1.55e-09 &  4.42e-17 &      \\ 
     &           &    5 &  1.55e-09 &  4.42e-17 &      \\ 
     &           &    6 &  1.54e-09 &  4.41e-17 &      \\ 
     &           &    7 &  1.54e-09 &  4.41e-17 &      \\ 
     &           &    8 &  1.54e-09 &  4.40e-17 &      \\ 
     &           &    9 &  1.54e-09 &  4.39e-17 &      \\ 
     &           &   10 &  1.53e-09 &  4.39e-17 &      \\ 
3470 &  3.47e+04 &   10 &           &           & iters  \\ 
 \hdashline 
     &           &    1 &  1.48e-08 &  2.73e-09 &      \\ 
     &           &    2 &  1.48e-08 &  4.22e-16 &      \\ 
     &           &    3 &  6.30e-08 &  4.21e-16 &      \\ 
     &           &    4 &  1.48e-08 &  1.80e-15 &      \\ 
     &           &    5 &  1.48e-08 &  4.23e-16 &      \\ 
     &           &    6 &  1.48e-08 &  4.23e-16 &      \\ 
     &           &    7 &  1.48e-08 &  4.22e-16 &      \\ 
     &           &    8 &  6.29e-08 &  4.21e-16 &      \\ 
     &           &    9 &  1.48e-08 &  1.80e-15 &      \\ 
     &           &   10 &  1.48e-08 &  4.23e-16 &      \\ 
3471 &  3.47e+04 &   10 &           &           & iters  \\ 
 \hdashline 
     &           &    1 &  4.94e-08 &  2.54e-09 &      \\ 
     &           &    2 &  2.83e-08 &  1.41e-15 &      \\ 
     &           &    3 &  2.83e-08 &  8.09e-16 &      \\ 
     &           &    4 &  4.94e-08 &  8.08e-16 &      \\ 
     &           &    5 &  2.83e-08 &  1.41e-15 &      \\ 
     &           &    6 &  2.83e-08 &  8.09e-16 &      \\ 
     &           &    7 &  4.94e-08 &  8.08e-16 &      \\ 
     &           &    8 &  2.83e-08 &  1.41e-15 &      \\ 
     &           &    9 &  2.83e-08 &  8.09e-16 &      \\ 
     &           &   10 &  4.94e-08 &  8.07e-16 &      \\ 
3472 &  3.47e+04 &   10 &           &           & iters  \\ 
 \hdashline 
     &           &    1 &  5.09e-08 &  1.56e-09 &      \\ 
     &           &    2 &  2.69e-08 &  1.45e-15 &      \\ 
     &           &    3 &  2.68e-08 &  7.67e-16 &      \\ 
     &           &    4 &  5.09e-08 &  7.66e-16 &      \\ 
     &           &    5 &  2.69e-08 &  1.45e-15 &      \\ 
     &           &    6 &  2.68e-08 &  7.67e-16 &      \\ 
     &           &    7 &  5.09e-08 &  7.66e-16 &      \\ 
     &           &    8 &  2.69e-08 &  1.45e-15 &      \\ 
     &           &    9 &  2.68e-08 &  7.67e-16 &      \\ 
     &           &   10 &  5.09e-08 &  7.66e-16 &      \\ 
3473 &  3.47e+04 &   10 &           &           & iters  \\ 
 \hdashline 
     &           &    1 &  1.86e-08 &  1.25e-09 &      \\ 
     &           &    2 &  1.86e-08 &  5.31e-16 &      \\ 
     &           &    3 &  2.03e-08 &  5.30e-16 &      \\ 
     &           &    4 &  1.86e-08 &  5.79e-16 &      \\ 
     &           &    5 &  2.03e-08 &  5.30e-16 &      \\ 
     &           &    6 &  1.86e-08 &  5.79e-16 &      \\ 
     &           &    7 &  2.03e-08 &  5.31e-16 &      \\ 
     &           &    8 &  1.86e-08 &  5.79e-16 &      \\ 
     &           &    9 &  2.03e-08 &  5.31e-16 &      \\ 
     &           &   10 &  1.86e-08 &  5.79e-16 &      \\ 
3474 &  3.47e+04 &   10 &           &           & iters  \\ 
 \hdashline 
     &           &    1 &  6.10e-09 &  1.42e-09 &      \\ 
     &           &    2 &  6.09e-09 &  1.74e-16 &      \\ 
     &           &    3 &  6.08e-09 &  1.74e-16 &      \\ 
     &           &    4 &  6.08e-09 &  1.74e-16 &      \\ 
     &           &    5 &  7.16e-08 &  1.73e-16 &      \\ 
     &           &    6 &  8.39e-08 &  2.04e-15 &      \\ 
     &           &    7 &  7.16e-08 &  2.39e-15 &      \\ 
     &           &    8 &  6.15e-09 &  2.04e-15 &      \\ 
     &           &    9 &  6.14e-09 &  1.76e-16 &      \\ 
     &           &   10 &  6.13e-09 &  1.75e-16 &      \\ 
3475 &  3.47e+04 &   10 &           &           & iters  \\ 
 \hdashline 
     &           &    1 &  2.64e-08 &  1.28e-09 &      \\ 
     &           &    2 &  1.25e-08 &  7.52e-16 &      \\ 
     &           &    3 &  1.25e-08 &  3.58e-16 &      \\ 
     &           &    4 &  1.25e-08 &  3.57e-16 &      \\ 
     &           &    5 &  2.64e-08 &  3.57e-16 &      \\ 
     &           &    6 &  1.25e-08 &  7.53e-16 &      \\ 
     &           &    7 &  1.25e-08 &  3.57e-16 &      \\ 
     &           &    8 &  2.64e-08 &  3.57e-16 &      \\ 
     &           &    9 &  1.25e-08 &  7.52e-16 &      \\ 
     &           &   10 &  1.25e-08 &  3.57e-16 &      \\ 
3476 &  3.48e+04 &   10 &           &           & iters  \\ 
 \hdashline 
     &           &    1 &  4.08e-09 &  7.78e-10 &      \\ 
     &           &    2 &  4.08e-09 &  1.17e-16 &      \\ 
     &           &    3 &  4.07e-09 &  1.16e-16 &      \\ 
     &           &    4 &  4.07e-09 &  1.16e-16 &      \\ 
     &           &    5 &  4.06e-09 &  1.16e-16 &      \\ 
     &           &    6 &  4.06e-09 &  1.16e-16 &      \\ 
     &           &    7 &  4.05e-09 &  1.16e-16 &      \\ 
     &           &    8 &  4.04e-09 &  1.16e-16 &      \\ 
     &           &    9 &  4.04e-09 &  1.15e-16 &      \\ 
     &           &   10 &  4.03e-09 &  1.15e-16 &      \\ 
3477 &  3.48e+04 &   10 &           &           & iters  \\ 
 \hdashline 
     &           &    1 &  3.75e-09 &  9.67e-10 &      \\ 
     &           &    2 &  3.74e-09 &  1.07e-16 &      \\ 
     &           &    3 &  3.74e-09 &  1.07e-16 &      \\ 
     &           &    4 &  3.73e-09 &  1.07e-16 &      \\ 
     &           &    5 &  3.73e-09 &  1.07e-16 &      \\ 
     &           &    6 &  3.72e-09 &  1.06e-16 &      \\ 
     &           &    7 &  3.72e-09 &  1.06e-16 &      \\ 
     &           &    8 &  7.40e-08 &  1.06e-16 &      \\ 
     &           &    9 &  4.27e-08 &  2.11e-15 &      \\ 
     &           &   10 &  3.76e-09 &  1.22e-15 &      \\ 
3478 &  3.48e+04 &   10 &           &           & iters  \\ 
 \hdashline 
     &           &    1 &  8.20e-09 &  3.29e-09 &      \\ 
     &           &    2 &  8.19e-09 &  2.34e-16 &      \\ 
     &           &    3 &  8.18e-09 &  2.34e-16 &      \\ 
     &           &    4 &  8.17e-09 &  2.33e-16 &      \\ 
     &           &    5 &  8.16e-09 &  2.33e-16 &      \\ 
     &           &    6 &  8.14e-09 &  2.33e-16 &      \\ 
     &           &    7 &  8.13e-09 &  2.32e-16 &      \\ 
     &           &    8 &  6.96e-08 &  2.32e-16 &      \\ 
     &           &    9 &  4.71e-08 &  1.99e-15 &      \\ 
     &           &   10 &  8.15e-09 &  1.34e-15 &      \\ 
3479 &  3.48e+04 &   10 &           &           & iters  \\ 
 \hdashline 
     &           &    1 &  3.88e-08 &  2.37e-09 &      \\ 
     &           &    2 &  3.90e-08 &  1.11e-15 &      \\ 
     &           &    3 &  3.88e-08 &  1.11e-15 &      \\ 
     &           &    4 &  3.90e-08 &  1.11e-15 &      \\ 
     &           &    5 &  3.88e-08 &  1.11e-15 &      \\ 
     &           &    6 &  3.90e-08 &  1.11e-15 &      \\ 
     &           &    7 &  3.88e-08 &  1.11e-15 &      \\ 
     &           &    8 &  3.90e-08 &  1.11e-15 &      \\ 
     &           &    9 &  3.88e-08 &  1.11e-15 &      \\ 
     &           &   10 &  3.90e-08 &  1.11e-15 &      \\ 
3480 &  3.48e+04 &   10 &           &           & iters  \\ 
 \hdashline 
     &           &    1 &  1.50e-08 &  2.51e-10 &      \\ 
     &           &    2 &  1.50e-08 &  4.29e-16 &      \\ 
     &           &    3 &  1.50e-08 &  4.28e-16 &      \\ 
     &           &    4 &  6.27e-08 &  4.27e-16 &      \\ 
     &           &    5 &  1.50e-08 &  1.79e-15 &      \\ 
     &           &    6 &  1.50e-08 &  4.29e-16 &      \\ 
     &           &    7 &  1.50e-08 &  4.29e-16 &      \\ 
     &           &    8 &  1.50e-08 &  4.28e-16 &      \\ 
     &           &    9 &  6.27e-08 &  4.28e-16 &      \\ 
     &           &   10 &  1.50e-08 &  1.79e-15 &      \\ 
3481 &  3.48e+04 &   10 &           &           & iters  \\ 
 \hdashline 
     &           &    1 &  5.92e-08 &  2.80e-10 &      \\ 
     &           &    2 &  1.86e-08 &  1.69e-15 &      \\ 
     &           &    3 &  1.86e-08 &  5.30e-16 &      \\ 
     &           &    4 &  1.85e-08 &  5.30e-16 &      \\ 
     &           &    5 &  1.85e-08 &  5.29e-16 &      \\ 
     &           &    6 &  5.92e-08 &  5.28e-16 &      \\ 
     &           &    7 &  1.86e-08 &  1.69e-15 &      \\ 
     &           &    8 &  1.85e-08 &  5.30e-16 &      \\ 
     &           &    9 &  1.85e-08 &  5.29e-16 &      \\ 
     &           &   10 &  5.92e-08 &  5.28e-16 &      \\ 
3482 &  3.48e+04 &   10 &           &           & iters  \\ 
 \hdashline 
     &           &    1 &  2.31e-08 &  1.02e-09 &      \\ 
     &           &    2 &  1.58e-08 &  6.59e-16 &      \\ 
     &           &    3 &  2.31e-08 &  4.50e-16 &      \\ 
     &           &    4 &  1.58e-08 &  6.59e-16 &      \\ 
     &           &    5 &  1.58e-08 &  4.51e-16 &      \\ 
     &           &    6 &  2.31e-08 &  4.50e-16 &      \\ 
     &           &    7 &  1.58e-08 &  6.59e-16 &      \\ 
     &           &    8 &  2.31e-08 &  4.50e-16 &      \\ 
     &           &    9 &  1.58e-08 &  6.59e-16 &      \\ 
     &           &   10 &  1.58e-08 &  4.51e-16 &      \\ 
3483 &  3.48e+04 &   10 &           &           & iters  \\ 
 \hdashline 
     &           &    1 &  4.47e-08 &  2.09e-09 &      \\ 
     &           &    2 &  3.30e-08 &  1.28e-15 &      \\ 
     &           &    3 &  3.30e-08 &  9.43e-16 &      \\ 
     &           &    4 &  4.47e-08 &  9.42e-16 &      \\ 
     &           &    5 &  3.30e-08 &  1.28e-15 &      \\ 
     &           &    6 &  4.47e-08 &  9.42e-16 &      \\ 
     &           &    7 &  3.30e-08 &  1.28e-15 &      \\ 
     &           &    8 &  4.47e-08 &  9.43e-16 &      \\ 
     &           &    9 &  3.30e-08 &  1.28e-15 &      \\ 
     &           &   10 &  3.30e-08 &  9.43e-16 &      \\ 
3484 &  3.48e+04 &   10 &           &           & iters  \\ 
 \hdashline 
     &           &    1 &  3.26e-08 &  2.38e-09 &      \\ 
     &           &    2 &  3.26e-08 &  9.31e-16 &      \\ 
     &           &    3 &  4.52e-08 &  9.29e-16 &      \\ 
     &           &    4 &  3.26e-08 &  1.29e-15 &      \\ 
     &           &    5 &  3.25e-08 &  9.30e-16 &      \\ 
     &           &    6 &  4.52e-08 &  9.29e-16 &      \\ 
     &           &    7 &  3.26e-08 &  1.29e-15 &      \\ 
     &           &    8 &  4.52e-08 &  9.29e-16 &      \\ 
     &           &    9 &  3.26e-08 &  1.29e-15 &      \\ 
     &           &   10 &  4.52e-08 &  9.30e-16 &      \\ 
3485 &  3.48e+04 &   10 &           &           & iters  \\ 
 \hdashline 
     &           &    1 &  1.98e-08 &  9.90e-10 &      \\ 
     &           &    2 &  5.80e-08 &  5.64e-16 &      \\ 
     &           &    3 &  1.98e-08 &  1.65e-15 &      \\ 
     &           &    4 &  1.98e-08 &  5.65e-16 &      \\ 
     &           &    5 &  1.98e-08 &  5.65e-16 &      \\ 
     &           &    6 &  5.80e-08 &  5.64e-16 &      \\ 
     &           &    7 &  1.98e-08 &  1.65e-15 &      \\ 
     &           &    8 &  1.98e-08 &  5.65e-16 &      \\ 
     &           &    9 &  1.98e-08 &  5.65e-16 &      \\ 
     &           &   10 &  5.80e-08 &  5.64e-16 &      \\ 
3486 &  3.48e+04 &   10 &           &           & iters  \\ 
 \hdashline 
     &           &    1 &  9.39e-08 &  7.82e-10 &      \\ 
     &           &    2 &  6.16e-08 &  2.68e-15 &      \\ 
     &           &    3 &  1.61e-08 &  1.76e-15 &      \\ 
     &           &    4 &  1.61e-08 &  4.61e-16 &      \\ 
     &           &    5 &  1.61e-08 &  4.60e-16 &      \\ 
     &           &    6 &  1.61e-08 &  4.59e-16 &      \\ 
     &           &    7 &  6.16e-08 &  4.59e-16 &      \\ 
     &           &    8 &  1.61e-08 &  1.76e-15 &      \\ 
     &           &    9 &  1.61e-08 &  4.60e-16 &      \\ 
     &           &   10 &  1.61e-08 &  4.60e-16 &      \\ 
3487 &  3.49e+04 &   10 &           &           & iters  \\ 
 \hdashline 
     &           &    1 &  4.08e-08 &  5.71e-10 &      \\ 
     &           &    2 &  3.69e-08 &  1.16e-15 &      \\ 
     &           &    3 &  4.08e-08 &  1.05e-15 &      \\ 
     &           &    4 &  3.69e-08 &  1.16e-15 &      \\ 
     &           &    5 &  4.08e-08 &  1.05e-15 &      \\ 
     &           &    6 &  3.69e-08 &  1.16e-15 &      \\ 
     &           &    7 &  4.08e-08 &  1.05e-15 &      \\ 
     &           &    8 &  3.70e-08 &  1.16e-15 &      \\ 
     &           &    9 &  4.08e-08 &  1.05e-15 &      \\ 
     &           &   10 &  3.70e-08 &  1.16e-15 &      \\ 
3488 &  3.49e+04 &   10 &           &           & iters  \\ 
 \hdashline 
     &           &    1 &  1.30e-08 &  9.98e-10 &      \\ 
     &           &    2 &  1.30e-08 &  3.71e-16 &      \\ 
     &           &    3 &  1.30e-08 &  3.71e-16 &      \\ 
     &           &    4 &  6.47e-08 &  3.70e-16 &      \\ 
     &           &    5 &  1.30e-08 &  1.85e-15 &      \\ 
     &           &    6 &  1.30e-08 &  3.72e-16 &      \\ 
     &           &    7 &  1.30e-08 &  3.72e-16 &      \\ 
     &           &    8 &  1.30e-08 &  3.71e-16 &      \\ 
     &           &    9 &  1.30e-08 &  3.71e-16 &      \\ 
     &           &   10 &  6.47e-08 &  3.70e-16 &      \\ 
3489 &  3.49e+04 &   10 &           &           & iters  \\ 
 \hdashline 
     &           &    1 &  2.59e-08 &  1.39e-09 &      \\ 
     &           &    2 &  2.58e-08 &  7.39e-16 &      \\ 
     &           &    3 &  5.19e-08 &  7.37e-16 &      \\ 
     &           &    4 &  2.59e-08 &  1.48e-15 &      \\ 
     &           &    5 &  2.58e-08 &  7.39e-16 &      \\ 
     &           &    6 &  5.19e-08 &  7.37e-16 &      \\ 
     &           &    7 &  2.59e-08 &  1.48e-15 &      \\ 
     &           &    8 &  2.58e-08 &  7.39e-16 &      \\ 
     &           &    9 &  5.19e-08 &  7.37e-16 &      \\ 
     &           &   10 &  2.59e-08 &  1.48e-15 &      \\ 
3490 &  3.49e+04 &   10 &           &           & iters  \\ 
 \hdashline 
     &           &    1 &  1.97e-08 &  7.61e-11 &      \\ 
     &           &    2 &  1.97e-08 &  5.62e-16 &      \\ 
     &           &    3 &  5.81e-08 &  5.61e-16 &      \\ 
     &           &    4 &  1.97e-08 &  1.66e-15 &      \\ 
     &           &    5 &  1.97e-08 &  5.62e-16 &      \\ 
     &           &    6 &  1.97e-08 &  5.62e-16 &      \\ 
     &           &    7 &  5.81e-08 &  5.61e-16 &      \\ 
     &           &    8 &  1.97e-08 &  1.66e-15 &      \\ 
     &           &    9 &  1.97e-08 &  5.62e-16 &      \\ 
     &           &   10 &  1.96e-08 &  5.62e-16 &      \\ 
3491 &  3.49e+04 &   10 &           &           & iters  \\ 
 \hdashline 
     &           &    1 &  5.97e-08 &  2.28e-09 &      \\ 
     &           &    2 &  1.81e-08 &  1.70e-15 &      \\ 
     &           &    3 &  1.81e-08 &  5.16e-16 &      \\ 
     &           &    4 &  1.80e-08 &  5.15e-16 &      \\ 
     &           &    5 &  5.97e-08 &  5.15e-16 &      \\ 
     &           &    6 &  1.81e-08 &  1.70e-15 &      \\ 
     &           &    7 &  1.81e-08 &  5.16e-16 &      \\ 
     &           &    8 &  1.80e-08 &  5.15e-16 &      \\ 
     &           &    9 &  5.97e-08 &  5.15e-16 &      \\ 
     &           &   10 &  1.81e-08 &  1.70e-15 &      \\ 
3492 &  3.49e+04 &   10 &           &           & iters  \\ 
 \hdashline 
     &           &    1 &  3.46e-08 &  1.86e-09 &      \\ 
     &           &    2 &  4.31e-08 &  9.88e-16 &      \\ 
     &           &    3 &  3.46e-08 &  1.23e-15 &      \\ 
     &           &    4 &  4.31e-08 &  9.88e-16 &      \\ 
     &           &    5 &  3.46e-08 &  1.23e-15 &      \\ 
     &           &    6 &  4.31e-08 &  9.89e-16 &      \\ 
     &           &    7 &  3.47e-08 &  1.23e-15 &      \\ 
     &           &    8 &  3.46e-08 &  9.89e-16 &      \\ 
     &           &    9 &  4.31e-08 &  9.88e-16 &      \\ 
     &           &   10 &  3.46e-08 &  1.23e-15 &      \\ 
3493 &  3.49e+04 &   10 &           &           & iters  \\ 
 \hdashline 
     &           &    1 &  3.22e-08 &  6.00e-10 &      \\ 
     &           &    2 &  3.22e-08 &  9.20e-16 &      \\ 
     &           &    3 &  4.55e-08 &  9.19e-16 &      \\ 
     &           &    4 &  3.22e-08 &  1.30e-15 &      \\ 
     &           &    5 &  3.22e-08 &  9.20e-16 &      \\ 
     &           &    6 &  4.56e-08 &  9.18e-16 &      \\ 
     &           &    7 &  3.22e-08 &  1.30e-15 &      \\ 
     &           &    8 &  4.55e-08 &  9.19e-16 &      \\ 
     &           &    9 &  3.22e-08 &  1.30e-15 &      \\ 
     &           &   10 &  3.22e-08 &  9.19e-16 &      \\ 
3494 &  3.49e+04 &   10 &           &           & iters  \\ 
 \hdashline 
     &           &    1 &  4.86e-08 &  5.21e-10 &      \\ 
     &           &    2 &  2.91e-08 &  1.39e-15 &      \\ 
     &           &    3 &  4.86e-08 &  8.32e-16 &      \\ 
     &           &    4 &  2.92e-08 &  1.39e-15 &      \\ 
     &           &    5 &  4.86e-08 &  8.32e-16 &      \\ 
     &           &    6 &  2.92e-08 &  1.39e-15 &      \\ 
     &           &    7 &  2.91e-08 &  8.33e-16 &      \\ 
     &           &    8 &  4.86e-08 &  8.32e-16 &      \\ 
     &           &    9 &  2.92e-08 &  1.39e-15 &      \\ 
     &           &   10 &  2.91e-08 &  8.33e-16 &      \\ 
3495 &  3.49e+04 &   10 &           &           & iters  \\ 
 \hdashline 
     &           &    1 &  6.71e-08 &  6.34e-10 &      \\ 
     &           &    2 &  4.95e-08 &  1.92e-15 &      \\ 
     &           &    3 &  1.06e-08 &  1.41e-15 &      \\ 
     &           &    4 &  1.06e-08 &  3.02e-16 &      \\ 
     &           &    5 &  6.71e-08 &  3.02e-16 &      \\ 
     &           &    6 &  1.07e-08 &  1.92e-15 &      \\ 
     &           &    7 &  1.06e-08 &  3.04e-16 &      \\ 
     &           &    8 &  1.06e-08 &  3.04e-16 &      \\ 
     &           &    9 &  1.06e-08 &  3.03e-16 &      \\ 
     &           &   10 &  1.06e-08 &  3.03e-16 &      \\ 
3496 &  3.50e+04 &   10 &           &           & iters  \\ 
 \hdashline 
     &           &    1 &  4.14e-08 &  3.01e-10 &      \\ 
     &           &    2 &  2.48e-09 &  1.18e-15 &      \\ 
     &           &    3 &  2.48e-09 &  7.08e-17 &      \\ 
     &           &    4 &  2.47e-09 &  7.07e-17 &      \\ 
     &           &    5 &  2.47e-09 &  7.06e-17 &      \\ 
     &           &    6 &  2.47e-09 &  7.05e-17 &      \\ 
     &           &    7 &  2.46e-09 &  7.04e-17 &      \\ 
     &           &    8 &  2.46e-09 &  7.03e-17 &      \\ 
     &           &    9 &  2.45e-09 &  7.02e-17 &      \\ 
     &           &   10 &  2.45e-09 &  7.01e-17 &      \\ 
3497 &  3.50e+04 &   10 &           &           & iters  \\ 
 \hdashline 
     &           &    1 &  6.06e-08 &  5.76e-11 &      \\ 
     &           &    2 &  1.71e-08 &  1.73e-15 &      \\ 
     &           &    3 &  1.71e-08 &  4.89e-16 &      \\ 
     &           &    4 &  1.71e-08 &  4.88e-16 &      \\ 
     &           &    5 &  6.06e-08 &  4.87e-16 &      \\ 
     &           &    6 &  1.71e-08 &  1.73e-15 &      \\ 
     &           &    7 &  1.71e-08 &  4.89e-16 &      \\ 
     &           &    8 &  1.71e-08 &  4.89e-16 &      \\ 
     &           &    9 &  1.71e-08 &  4.88e-16 &      \\ 
     &           &   10 &  6.06e-08 &  4.87e-16 &      \\ 
3498 &  3.50e+04 &   10 &           &           & iters  \\ 
 \hdashline 
     &           &    1 &  2.40e-08 &  1.02e-10 &      \\ 
     &           &    2 &  2.40e-08 &  6.85e-16 &      \\ 
     &           &    3 &  5.38e-08 &  6.84e-16 &      \\ 
     &           &    4 &  2.40e-08 &  1.53e-15 &      \\ 
     &           &    5 &  2.40e-08 &  6.85e-16 &      \\ 
     &           &    6 &  2.39e-08 &  6.84e-16 &      \\ 
     &           &    7 &  5.38e-08 &  6.83e-16 &      \\ 
     &           &    8 &  2.40e-08 &  1.54e-15 &      \\ 
     &           &    9 &  2.39e-08 &  6.84e-16 &      \\ 
     &           &   10 &  5.38e-08 &  6.83e-16 &      \\ 
3499 &  3.50e+04 &   10 &           &           & iters  \\ 
 \hdashline 
     &           &    1 &  4.44e-08 &  2.56e-10 &      \\ 
     &           &    2 &  3.33e-08 &  1.27e-15 &      \\ 
     &           &    3 &  4.44e-08 &  9.51e-16 &      \\ 
     &           &    4 &  3.33e-08 &  1.27e-15 &      \\ 
     &           &    5 &  4.44e-08 &  9.52e-16 &      \\ 
     &           &    6 &  3.34e-08 &  1.27e-15 &      \\ 
     &           &    7 &  3.33e-08 &  9.52e-16 &      \\ 
     &           &    8 &  4.44e-08 &  9.51e-16 &      \\ 
     &           &    9 &  3.33e-08 &  1.27e-15 &      \\ 
     &           &   10 &  4.44e-08 &  9.51e-16 &      \\ 
3500 &  3.50e+04 &   10 &           &           & iters  \\ 
 \hdashline 
     &           &    1 &  2.24e-08 &  3.75e-10 &      \\ 
     &           &    2 &  2.24e-08 &  6.39e-16 &      \\ 
     &           &    3 &  2.23e-08 &  6.38e-16 &      \\ 
     &           &    4 &  2.23e-08 &  6.37e-16 &      \\ 
     &           &    5 &  5.54e-08 &  6.36e-16 &      \\ 
     &           &    6 &  2.23e-08 &  1.58e-15 &      \\ 
     &           &    7 &  2.23e-08 &  6.38e-16 &      \\ 
     &           &    8 &  5.54e-08 &  6.37e-16 &      \\ 
     &           &    9 &  2.24e-08 &  1.58e-15 &      \\ 
     &           &   10 &  2.23e-08 &  6.38e-16 &      \\ 
3501 &  3.50e+04 &   10 &           &           & iters  \\ 
 \hdashline 
     &           &    1 &  5.99e-08 &  2.03e-10 &      \\ 
     &           &    2 &  1.79e-08 &  1.71e-15 &      \\ 
     &           &    3 &  1.79e-08 &  5.10e-16 &      \\ 
     &           &    4 &  1.78e-08 &  5.09e-16 &      \\ 
     &           &    5 &  1.78e-08 &  5.09e-16 &      \\ 
     &           &    6 &  5.99e-08 &  5.08e-16 &      \\ 
     &           &    7 &  1.79e-08 &  1.71e-15 &      \\ 
     &           &    8 &  1.78e-08 &  5.10e-16 &      \\ 
     &           &    9 &  1.78e-08 &  5.09e-16 &      \\ 
     &           &   10 &  5.99e-08 &  5.08e-16 &      \\ 
3502 &  3.50e+04 &   10 &           &           & iters  \\ 
 \hdashline 
     &           &    1 &  3.80e-08 &  6.38e-12 &      \\ 
     &           &    2 &  3.79e-08 &  1.08e-15 &      \\ 
     &           &    3 &  3.98e-08 &  1.08e-15 &      \\ 
     &           &    4 &  3.79e-08 &  1.14e-15 &      \\ 
     &           &    5 &  3.98e-08 &  1.08e-15 &      \\ 
     &           &    6 &  3.79e-08 &  1.14e-15 &      \\ 
     &           &    7 &  3.98e-08 &  1.08e-15 &      \\ 
     &           &    8 &  3.79e-08 &  1.14e-15 &      \\ 
     &           &    9 &  3.98e-08 &  1.08e-15 &      \\ 
     &           &   10 &  3.79e-08 &  1.14e-15 &      \\ 
3503 &  3.50e+04 &   10 &           &           & iters  \\ 
 \hdashline 
     &           &    1 &  5.46e-08 &  3.33e-10 &      \\ 
     &           &    2 &  2.32e-08 &  1.56e-15 &      \\ 
     &           &    3 &  2.32e-08 &  6.62e-16 &      \\ 
     &           &    4 &  5.45e-08 &  6.61e-16 &      \\ 
     &           &    5 &  2.32e-08 &  1.56e-15 &      \\ 
     &           &    6 &  2.32e-08 &  6.63e-16 &      \\ 
     &           &    7 &  5.45e-08 &  6.62e-16 &      \\ 
     &           &    8 &  2.32e-08 &  1.56e-15 &      \\ 
     &           &    9 &  2.32e-08 &  6.63e-16 &      \\ 
     &           &   10 &  5.45e-08 &  6.62e-16 &      \\ 
3504 &  3.50e+04 &   10 &           &           & iters  \\ 
 \hdashline 
     &           &    1 &  5.95e-08 &  3.39e-10 &      \\ 
     &           &    2 &  1.83e-08 &  1.70e-15 &      \\ 
     &           &    3 &  1.82e-08 &  5.21e-16 &      \\ 
     &           &    4 &  1.82e-08 &  5.21e-16 &      \\ 
     &           &    5 &  5.95e-08 &  5.20e-16 &      \\ 
     &           &    6 &  1.83e-08 &  1.70e-15 &      \\ 
     &           &    7 &  1.82e-08 &  5.22e-16 &      \\ 
     &           &    8 &  1.82e-08 &  5.21e-16 &      \\ 
     &           &    9 &  5.95e-08 &  5.20e-16 &      \\ 
     &           &   10 &  1.83e-08 &  1.70e-15 &      \\ 
3505 &  3.50e+04 &   10 &           &           & iters  \\ 
 \hdashline 
     &           &    1 &  3.73e-08 &  1.13e-10 &      \\ 
     &           &    2 &  4.04e-08 &  1.06e-15 &      \\ 
     &           &    3 &  3.73e-08 &  1.15e-15 &      \\ 
     &           &    4 &  4.04e-08 &  1.06e-15 &      \\ 
     &           &    5 &  3.73e-08 &  1.15e-15 &      \\ 
     &           &    6 &  4.04e-08 &  1.06e-15 &      \\ 
     &           &    7 &  3.73e-08 &  1.15e-15 &      \\ 
     &           &    8 &  3.73e-08 &  1.06e-15 &      \\ 
     &           &    9 &  4.05e-08 &  1.06e-15 &      \\ 
     &           &   10 &  3.73e-08 &  1.16e-15 &      \\ 
3506 &  3.50e+04 &   10 &           &           & iters  \\ 
 \hdashline 
     &           &    1 &  4.03e-08 &  2.42e-10 &      \\ 
     &           &    2 &  4.02e-08 &  1.15e-15 &      \\ 
     &           &    3 &  3.75e-08 &  1.15e-15 &      \\ 
     &           &    4 &  4.02e-08 &  1.07e-15 &      \\ 
     &           &    5 &  3.75e-08 &  1.15e-15 &      \\ 
     &           &    6 &  4.02e-08 &  1.07e-15 &      \\ 
     &           &    7 &  3.75e-08 &  1.15e-15 &      \\ 
     &           &    8 &  3.75e-08 &  1.07e-15 &      \\ 
     &           &    9 &  4.02e-08 &  1.07e-15 &      \\ 
     &           &   10 &  3.75e-08 &  1.15e-15 &      \\ 
3507 &  3.51e+04 &   10 &           &           & iters  \\ 
 \hdashline 
     &           &    1 &  3.71e-08 &  6.41e-11 &      \\ 
     &           &    2 &  3.70e-08 &  1.06e-15 &      \\ 
     &           &    3 &  4.07e-08 &  1.06e-15 &      \\ 
     &           &    4 &  3.70e-08 &  1.16e-15 &      \\ 
     &           &    5 &  4.07e-08 &  1.06e-15 &      \\ 
     &           &    6 &  3.70e-08 &  1.16e-15 &      \\ 
     &           &    7 &  4.07e-08 &  1.06e-15 &      \\ 
     &           &    8 &  3.70e-08 &  1.16e-15 &      \\ 
     &           &    9 &  4.07e-08 &  1.06e-15 &      \\ 
     &           &   10 &  3.70e-08 &  1.16e-15 &      \\ 
3508 &  3.51e+04 &   10 &           &           & iters  \\ 
 \hdashline 
     &           &    1 &  1.87e-08 &  4.20e-11 &      \\ 
     &           &    2 &  5.90e-08 &  5.34e-16 &      \\ 
     &           &    3 &  1.88e-08 &  1.68e-15 &      \\ 
     &           &    4 &  1.87e-08 &  5.35e-16 &      \\ 
     &           &    5 &  1.87e-08 &  5.35e-16 &      \\ 
     &           &    6 &  5.90e-08 &  5.34e-16 &      \\ 
     &           &    7 &  1.88e-08 &  1.68e-15 &      \\ 
     &           &    8 &  1.87e-08 &  5.35e-16 &      \\ 
     &           &    9 &  1.87e-08 &  5.35e-16 &      \\ 
     &           &   10 &  5.90e-08 &  5.34e-16 &      \\ 
3509 &  3.51e+04 &   10 &           &           & iters  \\ 
 \hdashline 
     &           &    1 &  2.30e-09 &  4.00e-10 &      \\ 
     &           &    2 &  2.29e-09 &  6.55e-17 &      \\ 
     &           &    3 &  2.29e-09 &  6.55e-17 &      \\ 
     &           &    4 &  2.29e-09 &  6.54e-17 &      \\ 
     &           &    5 &  2.28e-09 &  6.53e-17 &      \\ 
     &           &    6 &  2.28e-09 &  6.52e-17 &      \\ 
     &           &    7 &  2.28e-09 &  6.51e-17 &      \\ 
     &           &    8 &  2.27e-09 &  6.50e-17 &      \\ 
     &           &    9 &  2.27e-09 &  6.49e-17 &      \\ 
     &           &   10 &  2.27e-09 &  6.48e-17 &      \\ 
3510 &  3.51e+04 &   10 &           &           & iters  \\ 
 \hdashline 
     &           &    1 &  5.74e-09 &  1.09e-09 &      \\ 
     &           &    2 &  5.73e-09 &  1.64e-16 &      \\ 
     &           &    3 &  5.72e-09 &  1.64e-16 &      \\ 
     &           &    4 &  5.71e-09 &  1.63e-16 &      \\ 
     &           &    5 &  5.70e-09 &  1.63e-16 &      \\ 
     &           &    6 &  5.70e-09 &  1.63e-16 &      \\ 
     &           &    7 &  5.69e-09 &  1.63e-16 &      \\ 
     &           &    8 &  5.68e-09 &  1.62e-16 &      \\ 
     &           &    9 &  7.20e-08 &  1.62e-16 &      \\ 
     &           &   10 &  4.46e-08 &  2.06e-15 &      \\ 
3511 &  3.51e+04 &   10 &           &           & iters  \\ 
 \hdashline 
     &           &    1 &  1.68e-08 &  1.09e-09 &      \\ 
     &           &    2 &  6.09e-08 &  4.79e-16 &      \\ 
     &           &    3 &  1.68e-08 &  1.74e-15 &      \\ 
     &           &    4 &  1.68e-08 &  4.81e-16 &      \\ 
     &           &    5 &  1.68e-08 &  4.80e-16 &      \\ 
     &           &    6 &  6.09e-08 &  4.79e-16 &      \\ 
     &           &    7 &  1.69e-08 &  1.74e-15 &      \\ 
     &           &    8 &  1.68e-08 &  4.81e-16 &      \\ 
     &           &    9 &  1.68e-08 &  4.80e-16 &      \\ 
     &           &   10 &  1.68e-08 &  4.80e-16 &      \\ 
3512 &  3.51e+04 &   10 &           &           & iters  \\ 
 \hdashline 
     &           &    1 &  3.88e-08 &  2.45e-10 &      \\ 
     &           &    2 &  9.43e-11 &  1.11e-15 &      \\ 
     &           &    3 &  9.41e-11 &  2.69e-18 &      \\ 
     &           &    4 &  9.40e-11 &  2.69e-18 &      \\ 
     &           &    5 &  9.39e-11 &  2.68e-18 &      \\ 
     &           &    6 &  9.37e-11 &  2.68e-18 &      \\ 
     &           &    7 &  9.36e-11 &  2.68e-18 &      \\ 
     &           &    8 &  9.35e-11 &  2.67e-18 &      \\ 
     &           &    9 &  9.33e-11 &  2.67e-18 &      \\ 
     &           &   10 &  9.32e-11 &  2.66e-18 &      \\ 
3513 &  3.51e+04 &   10 &           &           & iters  \\ 
 \hdashline 
     &           &    1 &  4.59e-08 &  5.02e-10 &      \\ 
     &           &    2 &  3.19e-08 &  1.31e-15 &      \\ 
     &           &    3 &  3.18e-08 &  9.10e-16 &      \\ 
     &           &    4 &  4.59e-08 &  9.09e-16 &      \\ 
     &           &    5 &  3.19e-08 &  1.31e-15 &      \\ 
     &           &    6 &  4.59e-08 &  9.09e-16 &      \\ 
     &           &    7 &  3.19e-08 &  1.31e-15 &      \\ 
     &           &    8 &  3.18e-08 &  9.10e-16 &      \\ 
     &           &    9 &  4.59e-08 &  9.08e-16 &      \\ 
     &           &   10 &  3.18e-08 &  1.31e-15 &      \\ 
3514 &  3.51e+04 &   10 &           &           & iters  \\ 
 \hdashline 
     &           &    1 &  2.00e-08 &  2.71e-10 &      \\ 
     &           &    2 &  2.00e-08 &  5.72e-16 &      \\ 
     &           &    3 &  5.77e-08 &  5.71e-16 &      \\ 
     &           &    4 &  2.01e-08 &  1.65e-15 &      \\ 
     &           &    5 &  2.00e-08 &  5.72e-16 &      \\ 
     &           &    6 &  2.00e-08 &  5.72e-16 &      \\ 
     &           &    7 &  5.77e-08 &  5.71e-16 &      \\ 
     &           &    8 &  2.01e-08 &  1.65e-15 &      \\ 
     &           &    9 &  2.00e-08 &  5.72e-16 &      \\ 
     &           &   10 &  2.00e-08 &  5.71e-16 &      \\ 
3515 &  3.51e+04 &   10 &           &           & iters  \\ 
 \hdashline 
     &           &    1 &  4.39e-09 &  3.84e-10 &      \\ 
     &           &    2 &  4.39e-09 &  1.25e-16 &      \\ 
     &           &    3 &  4.38e-09 &  1.25e-16 &      \\ 
     &           &    4 &  4.38e-09 &  1.25e-16 &      \\ 
     &           &    5 &  4.37e-09 &  1.25e-16 &      \\ 
     &           &    6 &  4.36e-09 &  1.25e-16 &      \\ 
     &           &    7 &  4.36e-09 &  1.25e-16 &      \\ 
     &           &    8 &  4.35e-09 &  1.24e-16 &      \\ 
     &           &    9 &  4.34e-09 &  1.24e-16 &      \\ 
     &           &   10 &  4.34e-09 &  1.24e-16 &      \\ 
3516 &  3.52e+04 &   10 &           &           & iters  \\ 
 \hdashline 
     &           &    1 &  4.81e-08 &  1.33e-09 &      \\ 
     &           &    2 &  2.97e-08 &  1.37e-15 &      \\ 
     &           &    3 &  2.96e-08 &  8.46e-16 &      \\ 
     &           &    4 &  4.81e-08 &  8.45e-16 &      \\ 
     &           &    5 &  2.96e-08 &  1.37e-15 &      \\ 
     &           &    6 &  4.81e-08 &  8.46e-16 &      \\ 
     &           &    7 &  2.97e-08 &  1.37e-15 &      \\ 
     &           &    8 &  2.96e-08 &  8.47e-16 &      \\ 
     &           &    9 &  4.81e-08 &  8.46e-16 &      \\ 
     &           &   10 &  2.97e-08 &  1.37e-15 &      \\ 
3517 &  3.52e+04 &   10 &           &           & iters  \\ 
 \hdashline 
     &           &    1 &  2.48e-08 &  1.59e-09 &      \\ 
     &           &    2 &  2.48e-08 &  7.09e-16 &      \\ 
     &           &    3 &  5.29e-08 &  7.08e-16 &      \\ 
     &           &    4 &  2.48e-08 &  1.51e-15 &      \\ 
     &           &    5 &  2.48e-08 &  7.09e-16 &      \\ 
     &           &    6 &  5.29e-08 &  7.08e-16 &      \\ 
     &           &    7 &  2.48e-08 &  1.51e-15 &      \\ 
     &           &    8 &  2.48e-08 &  7.09e-16 &      \\ 
     &           &    9 &  5.29e-08 &  7.08e-16 &      \\ 
     &           &   10 &  2.49e-08 &  1.51e-15 &      \\ 
3518 &  3.52e+04 &   10 &           &           & iters  \\ 
 \hdashline 
     &           &    1 &  2.33e-09 &  5.94e-10 &      \\ 
     &           &    2 &  2.32e-09 &  6.64e-17 &      \\ 
     &           &    3 &  2.32e-09 &  6.63e-17 &      \\ 
     &           &    4 &  2.32e-09 &  6.62e-17 &      \\ 
     &           &    5 &  2.31e-09 &  6.61e-17 &      \\ 
     &           &    6 &  2.31e-09 &  6.60e-17 &      \\ 
     &           &    7 &  2.31e-09 &  6.59e-17 &      \\ 
     &           &    8 &  2.30e-09 &  6.58e-17 &      \\ 
     &           &    9 &  2.30e-09 &  6.58e-17 &      \\ 
     &           &   10 &  2.30e-09 &  6.57e-17 &      \\ 
3519 &  3.52e+04 &   10 &           &           & iters  \\ 
 \hdashline 
     &           &    1 &  5.08e-08 &  2.46e-10 &      \\ 
     &           &    2 &  2.70e-08 &  1.45e-15 &      \\ 
     &           &    3 &  5.07e-08 &  7.71e-16 &      \\ 
     &           &    4 &  2.70e-08 &  1.45e-15 &      \\ 
     &           &    5 &  2.70e-08 &  7.72e-16 &      \\ 
     &           &    6 &  5.07e-08 &  7.71e-16 &      \\ 
     &           &    7 &  2.70e-08 &  1.45e-15 &      \\ 
     &           &    8 &  2.70e-08 &  7.72e-16 &      \\ 
     &           &    9 &  5.07e-08 &  7.70e-16 &      \\ 
     &           &   10 &  2.70e-08 &  1.45e-15 &      \\ 
3520 &  3.52e+04 &   10 &           &           & iters  \\ 
 \hdashline 
     &           &    1 &  2.45e-08 &  3.46e-10 &      \\ 
     &           &    2 &  2.44e-08 &  6.98e-16 &      \\ 
     &           &    3 &  5.33e-08 &  6.97e-16 &      \\ 
     &           &    4 &  2.45e-08 &  1.52e-15 &      \\ 
     &           &    5 &  2.44e-08 &  6.99e-16 &      \\ 
     &           &    6 &  2.44e-08 &  6.98e-16 &      \\ 
     &           &    7 &  5.33e-08 &  6.97e-16 &      \\ 
     &           &    8 &  2.44e-08 &  1.52e-15 &      \\ 
     &           &    9 &  2.44e-08 &  6.98e-16 &      \\ 
     &           &   10 &  5.33e-08 &  6.97e-16 &      \\ 
3521 &  3.52e+04 &   10 &           &           & iters  \\ 
 \hdashline 
     &           &    1 &  4.02e-09 &  7.90e-10 &      \\ 
     &           &    2 &  4.02e-09 &  1.15e-16 &      \\ 
     &           &    3 &  4.01e-09 &  1.15e-16 &      \\ 
     &           &    4 &  4.00e-09 &  1.14e-16 &      \\ 
     &           &    5 &  4.00e-09 &  1.14e-16 &      \\ 
     &           &    6 &  3.99e-09 &  1.14e-16 &      \\ 
     &           &    7 &  3.99e-09 &  1.14e-16 &      \\ 
     &           &    8 &  3.98e-09 &  1.14e-16 &      \\ 
     &           &    9 &  3.98e-09 &  1.14e-16 &      \\ 
     &           &   10 &  3.97e-09 &  1.13e-16 &      \\ 
3522 &  3.52e+04 &   10 &           &           & iters  \\ 
 \hdashline 
     &           &    1 &  2.42e-08 &  2.07e-10 &      \\ 
     &           &    2 &  2.42e-08 &  6.91e-16 &      \\ 
     &           &    3 &  2.41e-08 &  6.90e-16 &      \\ 
     &           &    4 &  2.41e-08 &  6.89e-16 &      \\ 
     &           &    5 &  5.36e-08 &  6.88e-16 &      \\ 
     &           &    6 &  2.41e-08 &  1.53e-15 &      \\ 
     &           &    7 &  2.41e-08 &  6.89e-16 &      \\ 
     &           &    8 &  5.36e-08 &  6.88e-16 &      \\ 
     &           &    9 &  2.41e-08 &  1.53e-15 &      \\ 
     &           &   10 &  2.41e-08 &  6.89e-16 &      \\ 
3523 &  3.52e+04 &   10 &           &           & iters  \\ 
 \hdashline 
     &           &    1 &  5.88e-08 &  2.56e-10 &      \\ 
     &           &    2 &  1.89e-08 &  1.68e-15 &      \\ 
     &           &    3 &  1.89e-08 &  5.40e-16 &      \\ 
     &           &    4 &  1.89e-08 &  5.40e-16 &      \\ 
     &           &    5 &  5.88e-08 &  5.39e-16 &      \\ 
     &           &    6 &  1.89e-08 &  1.68e-15 &      \\ 
     &           &    7 &  1.89e-08 &  5.40e-16 &      \\ 
     &           &    8 &  1.89e-08 &  5.40e-16 &      \\ 
     &           &    9 &  1.89e-08 &  5.39e-16 &      \\ 
     &           &   10 &  5.89e-08 &  5.38e-16 &      \\ 
3524 &  3.52e+04 &   10 &           &           & iters  \\ 
 \hdashline 
     &           &    1 &  8.57e-09 &  1.06e-09 &      \\ 
     &           &    2 &  8.56e-09 &  2.45e-16 &      \\ 
     &           &    3 &  8.55e-09 &  2.44e-16 &      \\ 
     &           &    4 &  8.54e-09 &  2.44e-16 &      \\ 
     &           &    5 &  8.52e-09 &  2.44e-16 &      \\ 
     &           &    6 &  6.92e-08 &  2.43e-16 &      \\ 
     &           &    7 &  8.63e-08 &  1.97e-15 &      \\ 
     &           &    8 &  6.92e-08 &  2.46e-15 &      \\ 
     &           &    9 &  8.59e-09 &  1.98e-15 &      \\ 
     &           &   10 &  8.57e-09 &  2.45e-16 &      \\ 
3525 &  3.52e+04 &   10 &           &           & iters  \\ 
 \hdashline 
     &           &    1 &  1.72e-08 &  4.07e-10 &      \\ 
     &           &    2 &  1.71e-08 &  4.90e-16 &      \\ 
     &           &    3 &  1.71e-08 &  4.89e-16 &      \\ 
     &           &    4 &  6.06e-08 &  4.88e-16 &      \\ 
     &           &    5 &  1.72e-08 &  1.73e-15 &      \\ 
     &           &    6 &  1.72e-08 &  4.90e-16 &      \\ 
     &           &    7 &  1.71e-08 &  4.90e-16 &      \\ 
     &           &    8 &  6.06e-08 &  4.89e-16 &      \\ 
     &           &    9 &  1.72e-08 &  1.73e-15 &      \\ 
     &           &   10 &  1.72e-08 &  4.91e-16 &      \\ 
3526 &  3.52e+04 &   10 &           &           & iters  \\ 
 \hdashline 
     &           &    1 &  4.27e-08 &  1.30e-09 &      \\ 
     &           &    2 &  3.50e-08 &  1.22e-15 &      \\ 
     &           &    3 &  4.27e-08 &  1.00e-15 &      \\ 
     &           &    4 &  3.50e-08 &  1.22e-15 &      \\ 
     &           &    5 &  3.50e-08 &  1.00e-15 &      \\ 
     &           &    6 &  4.28e-08 &  9.98e-16 &      \\ 
     &           &    7 &  3.50e-08 &  1.22e-15 &      \\ 
     &           &    8 &  4.27e-08 &  9.99e-16 &      \\ 
     &           &    9 &  3.50e-08 &  1.22e-15 &      \\ 
     &           &   10 &  4.27e-08 &  9.99e-16 &      \\ 
3527 &  3.53e+04 &   10 &           &           & iters  \\ 
 \hdashline 
     &           &    1 &  4.00e-08 &  1.60e-09 &      \\ 
     &           &    2 &  3.78e-08 &  1.14e-15 &      \\ 
     &           &    3 &  4.00e-08 &  1.08e-15 &      \\ 
     &           &    4 &  3.78e-08 &  1.14e-15 &      \\ 
     &           &    5 &  4.00e-08 &  1.08e-15 &      \\ 
     &           &    6 &  3.78e-08 &  1.14e-15 &      \\ 
     &           &    7 &  4.00e-08 &  1.08e-15 &      \\ 
     &           &    8 &  3.78e-08 &  1.14e-15 &      \\ 
     &           &    9 &  4.00e-08 &  1.08e-15 &      \\ 
     &           &   10 &  3.78e-08 &  1.14e-15 &      \\ 
3528 &  3.53e+04 &   10 &           &           & iters  \\ 
 \hdashline 
     &           &    1 &  2.93e-08 &  1.28e-09 &      \\ 
     &           &    2 &  2.92e-08 &  8.36e-16 &      \\ 
     &           &    3 &  4.85e-08 &  8.34e-16 &      \\ 
     &           &    4 &  2.93e-08 &  1.38e-15 &      \\ 
     &           &    5 &  2.92e-08 &  8.35e-16 &      \\ 
     &           &    6 &  4.85e-08 &  8.34e-16 &      \\ 
     &           &    7 &  2.92e-08 &  1.38e-15 &      \\ 
     &           &    8 &  4.85e-08 &  8.35e-16 &      \\ 
     &           &    9 &  2.93e-08 &  1.38e-15 &      \\ 
     &           &   10 &  2.92e-08 &  8.35e-16 &      \\ 
3529 &  3.53e+04 &   10 &           &           & iters  \\ 
 \hdashline 
     &           &    1 &  3.64e-08 &  1.32e-09 &      \\ 
     &           &    2 &  3.64e-08 &  1.04e-15 &      \\ 
     &           &    3 &  2.53e-09 &  1.04e-15 &      \\ 
     &           &    4 &  2.53e-09 &  7.23e-17 &      \\ 
     &           &    5 &  2.53e-09 &  7.22e-17 &      \\ 
     &           &    6 &  2.52e-09 &  7.21e-17 &      \\ 
     &           &    7 &  2.52e-09 &  7.20e-17 &      \\ 
     &           &    8 &  2.52e-09 &  7.19e-17 &      \\ 
     &           &    9 &  2.51e-09 &  7.18e-17 &      \\ 
     &           &   10 &  2.51e-09 &  7.17e-17 &      \\ 
3530 &  3.53e+04 &   10 &           &           & iters  \\ 
 \hdashline 
     &           &    1 &  3.39e-08 &  1.32e-09 &      \\ 
     &           &    2 &  4.38e-08 &  9.67e-16 &      \\ 
     &           &    3 &  3.39e-08 &  1.25e-15 &      \\ 
     &           &    4 &  4.38e-08 &  9.68e-16 &      \\ 
     &           &    5 &  3.39e-08 &  1.25e-15 &      \\ 
     &           &    6 &  4.38e-08 &  9.68e-16 &      \\ 
     &           &    7 &  3.39e-08 &  1.25e-15 &      \\ 
     &           &    8 &  3.39e-08 &  9.68e-16 &      \\ 
     &           &    9 &  4.39e-08 &  9.67e-16 &      \\ 
     &           &   10 &  3.39e-08 &  1.25e-15 &      \\ 
3531 &  3.53e+04 &   10 &           &           & iters  \\ 
 \hdashline 
     &           &    1 &  1.09e-08 &  7.05e-10 &      \\ 
     &           &    2 &  1.09e-08 &  3.12e-16 &      \\ 
     &           &    3 &  1.09e-08 &  3.12e-16 &      \\ 
     &           &    4 &  2.79e-08 &  3.12e-16 &      \\ 
     &           &    5 &  1.09e-08 &  7.98e-16 &      \\ 
     &           &    6 &  1.09e-08 &  3.12e-16 &      \\ 
     &           &    7 &  1.09e-08 &  3.12e-16 &      \\ 
     &           &    8 &  2.80e-08 &  3.11e-16 &      \\ 
     &           &    9 &  1.09e-08 &  7.98e-16 &      \\ 
     &           &   10 &  1.09e-08 &  3.12e-16 &      \\ 
3532 &  3.53e+04 &   10 &           &           & iters  \\ 
 \hdashline 
     &           &    1 &  2.43e-08 &  5.66e-10 &      \\ 
     &           &    2 &  5.34e-08 &  6.94e-16 &      \\ 
     &           &    3 &  2.43e-08 &  1.52e-15 &      \\ 
     &           &    4 &  2.43e-08 &  6.95e-16 &      \\ 
     &           &    5 &  5.34e-08 &  6.94e-16 &      \\ 
     &           &    6 &  2.44e-08 &  1.52e-15 &      \\ 
     &           &    7 &  2.43e-08 &  6.95e-16 &      \\ 
     &           &    8 &  5.34e-08 &  6.94e-16 &      \\ 
     &           &    9 &  2.44e-08 &  1.52e-15 &      \\ 
     &           &   10 &  2.43e-08 &  6.95e-16 &      \\ 
3533 &  3.53e+04 &   10 &           &           & iters  \\ 
 \hdashline 
     &           &    1 &  4.14e-08 &  5.31e-10 &      \\ 
     &           &    2 &  3.63e-08 &  1.18e-15 &      \\ 
     &           &    3 &  3.63e-08 &  1.04e-15 &      \\ 
     &           &    4 &  4.15e-08 &  1.04e-15 &      \\ 
     &           &    5 &  3.63e-08 &  1.18e-15 &      \\ 
     &           &    6 &  4.15e-08 &  1.04e-15 &      \\ 
     &           &    7 &  3.63e-08 &  1.18e-15 &      \\ 
     &           &    8 &  4.15e-08 &  1.04e-15 &      \\ 
     &           &    9 &  3.63e-08 &  1.18e-15 &      \\ 
     &           &   10 &  4.14e-08 &  1.04e-15 &      \\ 
3534 &  3.53e+04 &   10 &           &           & iters  \\ 
 \hdashline 
     &           &    1 &  3.57e-08 &  8.60e-10 &      \\ 
     &           &    2 &  3.17e-09 &  1.02e-15 &      \\ 
     &           &    3 &  3.17e-09 &  9.05e-17 &      \\ 
     &           &    4 &  3.16e-09 &  9.04e-17 &      \\ 
     &           &    5 &  3.16e-09 &  9.03e-17 &      \\ 
     &           &    6 &  3.15e-09 &  9.01e-17 &      \\ 
     &           &    7 &  3.15e-09 &  9.00e-17 &      \\ 
     &           &    8 &  3.14e-09 &  8.99e-17 &      \\ 
     &           &    9 &  3.14e-09 &  8.98e-17 &      \\ 
     &           &   10 &  3.14e-09 &  8.96e-17 &      \\ 
3535 &  3.53e+04 &   10 &           &           & iters  \\ 
 \hdashline 
     &           &    1 &  1.30e-08 &  1.46e-09 &      \\ 
     &           &    2 &  1.29e-08 &  3.70e-16 &      \\ 
     &           &    3 &  1.29e-08 &  3.69e-16 &      \\ 
     &           &    4 &  6.48e-08 &  3.69e-16 &      \\ 
     &           &    5 &  5.18e-08 &  1.85e-15 &      \\ 
     &           &    6 &  1.29e-08 &  1.48e-15 &      \\ 
     &           &    7 &  6.48e-08 &  3.69e-16 &      \\ 
     &           &    8 &  5.18e-08 &  1.85e-15 &      \\ 
     &           &    9 &  1.29e-08 &  1.48e-15 &      \\ 
     &           &   10 &  6.48e-08 &  3.69e-16 &      \\ 
3536 &  3.54e+04 &   10 &           &           & iters  \\ 
 \hdashline 
     &           &    1 &  5.84e-08 &  4.94e-10 &      \\ 
     &           &    2 &  1.94e-08 &  1.67e-15 &      \\ 
     &           &    3 &  1.93e-08 &  5.52e-16 &      \\ 
     &           &    4 &  1.93e-08 &  5.51e-16 &      \\ 
     &           &    5 &  5.84e-08 &  5.51e-16 &      \\ 
     &           &    6 &  1.94e-08 &  1.67e-15 &      \\ 
     &           &    7 &  1.93e-08 &  5.52e-16 &      \\ 
     &           &    8 &  1.93e-08 &  5.51e-16 &      \\ 
     &           &    9 &  5.84e-08 &  5.51e-16 &      \\ 
     &           &   10 &  1.94e-08 &  1.67e-15 &      \\ 
3537 &  3.54e+04 &   10 &           &           & iters  \\ 
 \hdashline 
     &           &    1 &  2.10e-08 &  1.06e-09 &      \\ 
     &           &    2 &  2.09e-08 &  5.99e-16 &      \\ 
     &           &    3 &  5.68e-08 &  5.98e-16 &      \\ 
     &           &    4 &  2.10e-08 &  1.62e-15 &      \\ 
     &           &    5 &  2.10e-08 &  5.99e-16 &      \\ 
     &           &    6 &  5.67e-08 &  5.99e-16 &      \\ 
     &           &    7 &  2.10e-08 &  1.62e-15 &      \\ 
     &           &    8 &  2.10e-08 &  6.00e-16 &      \\ 
     &           &    9 &  2.10e-08 &  5.99e-16 &      \\ 
     &           &   10 &  5.68e-08 &  5.98e-16 &      \\ 
3538 &  3.54e+04 &   10 &           &           & iters  \\ 
 \hdashline 
     &           &    1 &  2.46e-08 &  1.43e-09 &      \\ 
     &           &    2 &  2.46e-08 &  7.02e-16 &      \\ 
     &           &    3 &  5.32e-08 &  7.01e-16 &      \\ 
     &           &    4 &  2.46e-08 &  1.52e-15 &      \\ 
     &           &    5 &  2.46e-08 &  7.02e-16 &      \\ 
     &           &    6 &  5.32e-08 &  7.01e-16 &      \\ 
     &           &    7 &  2.46e-08 &  1.52e-15 &      \\ 
     &           &    8 &  2.46e-08 &  7.02e-16 &      \\ 
     &           &    9 &  5.32e-08 &  7.01e-16 &      \\ 
     &           &   10 &  2.46e-08 &  1.52e-15 &      \\ 
3539 &  3.54e+04 &   10 &           &           & iters  \\ 
 \hdashline 
     &           &    1 &  1.52e-08 &  1.07e-09 &      \\ 
     &           &    2 &  1.51e-08 &  4.33e-16 &      \\ 
     &           &    3 &  1.51e-08 &  4.32e-16 &      \\ 
     &           &    4 &  6.26e-08 &  4.32e-16 &      \\ 
     &           &    5 &  1.52e-08 &  1.79e-15 &      \\ 
     &           &    6 &  1.52e-08 &  4.34e-16 &      \\ 
     &           &    7 &  1.51e-08 &  4.33e-16 &      \\ 
     &           &    8 &  1.51e-08 &  4.32e-16 &      \\ 
     &           &    9 &  6.26e-08 &  4.32e-16 &      \\ 
     &           &   10 &  1.52e-08 &  1.79e-15 &      \\ 
3540 &  3.54e+04 &   10 &           &           & iters  \\ 
 \hdashline 
     &           &    1 &  3.41e-08 &  1.06e-09 &      \\ 
     &           &    2 &  4.37e-08 &  9.72e-16 &      \\ 
     &           &    3 &  3.41e-08 &  1.25e-15 &      \\ 
     &           &    4 &  3.40e-08 &  9.73e-16 &      \\ 
     &           &    5 &  4.37e-08 &  9.71e-16 &      \\ 
     &           &    6 &  3.40e-08 &  1.25e-15 &      \\ 
     &           &    7 &  4.37e-08 &  9.72e-16 &      \\ 
     &           &    8 &  3.41e-08 &  1.25e-15 &      \\ 
     &           &    9 &  4.37e-08 &  9.72e-16 &      \\ 
     &           &   10 &  3.41e-08 &  1.25e-15 &      \\ 
3541 &  3.54e+04 &   10 &           &           & iters  \\ 
 \hdashline 
     &           &    1 &  7.51e-08 &  4.12e-10 &      \\ 
     &           &    2 &  8.04e-08 &  2.14e-15 &      \\ 
     &           &    3 &  7.51e-08 &  2.29e-15 &      \\ 
     &           &    4 &  8.04e-08 &  2.14e-15 &      \\ 
     &           &    5 &  7.51e-08 &  2.29e-15 &      \\ 
     &           &    6 &  2.66e-09 &  2.14e-15 &      \\ 
     &           &    7 &  2.66e-09 &  7.59e-17 &      \\ 
     &           &    8 &  2.65e-09 &  7.58e-17 &      \\ 
     &           &    9 &  2.65e-09 &  7.57e-17 &      \\ 
     &           &   10 &  2.64e-09 &  7.56e-17 &      \\ 
3542 &  3.54e+04 &   10 &           &           & iters  \\ 
 \hdashline 
     &           &    1 &  2.27e-08 &  1.83e-10 &      \\ 
     &           &    2 &  2.27e-08 &  6.48e-16 &      \\ 
     &           &    3 &  5.50e-08 &  6.47e-16 &      \\ 
     &           &    4 &  2.27e-08 &  1.57e-15 &      \\ 
     &           &    5 &  2.27e-08 &  6.48e-16 &      \\ 
     &           &    6 &  5.50e-08 &  6.48e-16 &      \\ 
     &           &    7 &  2.27e-08 &  1.57e-15 &      \\ 
     &           &    8 &  2.27e-08 &  6.49e-16 &      \\ 
     &           &    9 &  2.27e-08 &  6.48e-16 &      \\ 
     &           &   10 &  5.51e-08 &  6.47e-16 &      \\ 
3543 &  3.54e+04 &   10 &           &           & iters  \\ 
 \hdashline 
     &           &    1 &  2.75e-08 &  1.16e-10 &      \\ 
     &           &    2 &  2.74e-08 &  7.84e-16 &      \\ 
     &           &    3 &  5.03e-08 &  7.83e-16 &      \\ 
     &           &    4 &  2.75e-08 &  1.44e-15 &      \\ 
     &           &    5 &  2.74e-08 &  7.84e-16 &      \\ 
     &           &    6 &  5.03e-08 &  7.83e-16 &      \\ 
     &           &    7 &  2.75e-08 &  1.44e-15 &      \\ 
     &           &    8 &  2.74e-08 &  7.84e-16 &      \\ 
     &           &    9 &  5.03e-08 &  7.83e-16 &      \\ 
     &           &   10 &  2.75e-08 &  1.44e-15 &      \\ 
3544 &  3.54e+04 &   10 &           &           & iters  \\ 
 \hdashline 
     &           &    1 &  3.14e-08 &  3.79e-10 &      \\ 
     &           &    2 &  4.63e-08 &  8.97e-16 &      \\ 
     &           &    3 &  3.14e-08 &  1.32e-15 &      \\ 
     &           &    4 &  4.63e-08 &  8.98e-16 &      \\ 
     &           &    5 &  3.15e-08 &  1.32e-15 &      \\ 
     &           &    6 &  3.14e-08 &  8.98e-16 &      \\ 
     &           &    7 &  4.63e-08 &  8.97e-16 &      \\ 
     &           &    8 &  3.14e-08 &  1.32e-15 &      \\ 
     &           &    9 &  4.63e-08 &  8.97e-16 &      \\ 
     &           &   10 &  3.15e-08 &  1.32e-15 &      \\ 
3545 &  3.54e+04 &   10 &           &           & iters  \\ 
 \hdashline 
     &           &    1 &  2.03e-08 &  9.93e-10 &      \\ 
     &           &    2 &  5.74e-08 &  5.80e-16 &      \\ 
     &           &    3 &  2.04e-08 &  1.64e-15 &      \\ 
     &           &    4 &  2.04e-08 &  5.82e-16 &      \\ 
     &           &    5 &  2.03e-08 &  5.81e-16 &      \\ 
     &           &    6 &  5.74e-08 &  5.80e-16 &      \\ 
     &           &    7 &  2.04e-08 &  1.64e-15 &      \\ 
     &           &    8 &  2.04e-08 &  5.82e-16 &      \\ 
     &           &    9 &  2.03e-08 &  5.81e-16 &      \\ 
     &           &   10 &  5.74e-08 &  5.80e-16 &      \\ 
3546 &  3.54e+04 &   10 &           &           & iters  \\ 
 \hdashline 
     &           &    1 &  2.92e-08 &  1.14e-09 &      \\ 
     &           &    2 &  2.92e-08 &  8.33e-16 &      \\ 
     &           &    3 &  4.86e-08 &  8.32e-16 &      \\ 
     &           &    4 &  2.92e-08 &  1.39e-15 &      \\ 
     &           &    5 &  2.91e-08 &  8.33e-16 &      \\ 
     &           &    6 &  4.86e-08 &  8.32e-16 &      \\ 
     &           &    7 &  2.92e-08 &  1.39e-15 &      \\ 
     &           &    8 &  4.86e-08 &  8.32e-16 &      \\ 
     &           &    9 &  2.92e-08 &  1.39e-15 &      \\ 
     &           &   10 &  2.92e-08 &  8.33e-16 &      \\ 
3547 &  3.55e+04 &   10 &           &           & iters  \\ 
 \hdashline 
     &           &    1 &  3.21e-09 &  4.13e-10 &      \\ 
     &           &    2 &  3.20e-09 &  9.15e-17 &      \\ 
     &           &    3 &  3.20e-09 &  9.14e-17 &      \\ 
     &           &    4 &  3.19e-09 &  9.13e-17 &      \\ 
     &           &    5 &  3.19e-09 &  9.12e-17 &      \\ 
     &           &    6 &  3.18e-09 &  9.10e-17 &      \\ 
     &           &    7 &  3.18e-09 &  9.09e-17 &      \\ 
     &           &    8 &  3.18e-09 &  9.08e-17 &      \\ 
     &           &    9 &  3.17e-09 &  9.06e-17 &      \\ 
     &           &   10 &  3.17e-09 &  9.05e-17 &      \\ 
3548 &  3.55e+04 &   10 &           &           & iters  \\ 
 \hdashline 
     &           &    1 &  3.07e-08 &  1.43e-09 &      \\ 
     &           &    2 &  3.07e-08 &  8.77e-16 &      \\ 
     &           &    3 &  4.70e-08 &  8.76e-16 &      \\ 
     &           &    4 &  3.07e-08 &  1.34e-15 &      \\ 
     &           &    5 &  4.70e-08 &  8.77e-16 &      \\ 
     &           &    6 &  3.07e-08 &  1.34e-15 &      \\ 
     &           &    7 &  3.07e-08 &  8.77e-16 &      \\ 
     &           &    8 &  4.70e-08 &  8.76e-16 &      \\ 
     &           &    9 &  3.07e-08 &  1.34e-15 &      \\ 
     &           &   10 &  4.70e-08 &  8.77e-16 &      \\ 
3549 &  3.55e+04 &   10 &           &           & iters  \\ 
 \hdashline 
     &           &    1 &  3.18e-08 &  7.45e-10 &      \\ 
     &           &    2 &  4.60e-08 &  9.07e-16 &      \\ 
     &           &    3 &  3.18e-08 &  1.31e-15 &      \\ 
     &           &    4 &  3.17e-08 &  9.07e-16 &      \\ 
     &           &    5 &  4.60e-08 &  9.06e-16 &      \\ 
     &           &    6 &  3.18e-08 &  1.31e-15 &      \\ 
     &           &    7 &  4.60e-08 &  9.06e-16 &      \\ 
     &           &    8 &  3.18e-08 &  1.31e-15 &      \\ 
     &           &    9 &  3.17e-08 &  9.07e-16 &      \\ 
     &           &   10 &  4.60e-08 &  9.06e-16 &      \\ 
3550 &  3.55e+04 &   10 &           &           & iters  \\ 
 \hdashline 
     &           &    1 &  4.48e-09 &  1.83e-10 &      \\ 
     &           &    2 &  4.47e-09 &  1.28e-16 &      \\ 
     &           &    3 &  4.47e-09 &  1.28e-16 &      \\ 
     &           &    4 &  4.46e-09 &  1.27e-16 &      \\ 
     &           &    5 &  4.45e-09 &  1.27e-16 &      \\ 
     &           &    6 &  4.45e-09 &  1.27e-16 &      \\ 
     &           &    7 &  4.44e-09 &  1.27e-16 &      \\ 
     &           &    8 &  4.43e-09 &  1.27e-16 &      \\ 
     &           &    9 &  7.33e-08 &  1.27e-16 &      \\ 
     &           &   10 &  8.22e-08 &  2.09e-15 &      \\ 
3551 &  3.55e+04 &   10 &           &           & iters  \\ 
 \hdashline 
     &           &    1 &  6.06e-08 &  3.87e-10 &      \\ 
     &           &    2 &  1.72e-08 &  1.73e-15 &      \\ 
     &           &    3 &  1.72e-08 &  4.91e-16 &      \\ 
     &           &    4 &  1.72e-08 &  4.90e-16 &      \\ 
     &           &    5 &  6.06e-08 &  4.89e-16 &      \\ 
     &           &    6 &  1.72e-08 &  1.73e-15 &      \\ 
     &           &    7 &  1.72e-08 &  4.91e-16 &      \\ 
     &           &    8 &  1.72e-08 &  4.91e-16 &      \\ 
     &           &    9 &  6.05e-08 &  4.90e-16 &      \\ 
     &           &   10 &  1.72e-08 &  1.73e-15 &      \\ 
3552 &  3.55e+04 &   10 &           &           & iters  \\ 
 \hdashline 
     &           &    1 &  3.97e-08 &  4.01e-10 &      \\ 
     &           &    2 &  3.80e-08 &  1.13e-15 &      \\ 
     &           &    3 &  3.97e-08 &  1.09e-15 &      \\ 
     &           &    4 &  3.80e-08 &  1.13e-15 &      \\ 
     &           &    5 &  3.97e-08 &  1.09e-15 &      \\ 
     &           &    6 &  3.80e-08 &  1.13e-15 &      \\ 
     &           &    7 &  3.97e-08 &  1.09e-15 &      \\ 
     &           &    8 &  3.80e-08 &  1.13e-15 &      \\ 
     &           &    9 &  3.97e-08 &  1.09e-15 &      \\ 
     &           &   10 &  3.80e-08 &  1.13e-15 &      \\ 
3553 &  3.55e+04 &   10 &           &           & iters  \\ 
 \hdashline 
     &           &    1 &  1.61e-08 &  8.71e-10 &      \\ 
     &           &    2 &  1.61e-08 &  4.59e-16 &      \\ 
     &           &    3 &  1.60e-08 &  4.58e-16 &      \\ 
     &           &    4 &  6.17e-08 &  4.58e-16 &      \\ 
     &           &    5 &  1.61e-08 &  1.76e-15 &      \\ 
     &           &    6 &  1.61e-08 &  4.60e-16 &      \\ 
     &           &    7 &  1.61e-08 &  4.59e-16 &      \\ 
     &           &    8 &  1.60e-08 &  4.58e-16 &      \\ 
     &           &    9 &  6.17e-08 &  4.58e-16 &      \\ 
     &           &   10 &  1.61e-08 &  1.76e-15 &      \\ 
3554 &  3.55e+04 &   10 &           &           & iters  \\ 
 \hdashline 
     &           &    1 &  1.61e-08 &  1.11e-11 &      \\ 
     &           &    2 &  1.61e-08 &  4.61e-16 &      \\ 
     &           &    3 &  1.61e-08 &  4.60e-16 &      \\ 
     &           &    4 &  6.16e-08 &  4.59e-16 &      \\ 
     &           &    5 &  1.62e-08 &  1.76e-15 &      \\ 
     &           &    6 &  1.61e-08 &  4.61e-16 &      \\ 
     &           &    7 &  1.61e-08 &  4.61e-16 &      \\ 
     &           &    8 &  6.16e-08 &  4.60e-16 &      \\ 
     &           &    9 &  1.62e-08 &  1.76e-15 &      \\ 
     &           &   10 &  1.62e-08 &  4.62e-16 &      \\ 
3555 &  3.55e+04 &   10 &           &           & iters  \\ 
 \hdashline 
     &           &    1 &  3.17e-08 &  5.20e-10 &      \\ 
     &           &    2 &  3.17e-08 &  9.05e-16 &      \\ 
     &           &    3 &  4.61e-08 &  9.04e-16 &      \\ 
     &           &    4 &  3.17e-08 &  1.31e-15 &      \\ 
     &           &    5 &  4.61e-08 &  9.04e-16 &      \\ 
     &           &    6 &  3.17e-08 &  1.31e-15 &      \\ 
     &           &    7 &  4.60e-08 &  9.05e-16 &      \\ 
     &           &    8 &  3.17e-08 &  1.31e-15 &      \\ 
     &           &    9 &  3.17e-08 &  9.05e-16 &      \\ 
     &           &   10 &  4.61e-08 &  9.04e-16 &      \\ 
3556 &  3.56e+04 &   10 &           &           & iters  \\ 
 \hdashline 
     &           &    1 &  1.44e-08 &  7.65e-11 &      \\ 
     &           &    2 &  1.44e-08 &  4.12e-16 &      \\ 
     &           &    3 &  1.44e-08 &  4.11e-16 &      \\ 
     &           &    4 &  1.44e-08 &  4.11e-16 &      \\ 
     &           &    5 &  6.33e-08 &  4.10e-16 &      \\ 
     &           &    6 &  1.44e-08 &  1.81e-15 &      \\ 
     &           &    7 &  1.44e-08 &  4.12e-16 &      \\ 
     &           &    8 &  1.44e-08 &  4.11e-16 &      \\ 
     &           &    9 &  1.44e-08 &  4.11e-16 &      \\ 
     &           &   10 &  6.33e-08 &  4.10e-16 &      \\ 
3557 &  3.56e+04 &   10 &           &           & iters  \\ 
 \hdashline 
     &           &    1 &  1.91e-08 &  2.20e-10 &      \\ 
     &           &    2 &  1.91e-08 &  5.45e-16 &      \\ 
     &           &    3 &  5.87e-08 &  5.44e-16 &      \\ 
     &           &    4 &  1.91e-08 &  1.67e-15 &      \\ 
     &           &    5 &  1.91e-08 &  5.45e-16 &      \\ 
     &           &    6 &  1.91e-08 &  5.45e-16 &      \\ 
     &           &    7 &  5.87e-08 &  5.44e-16 &      \\ 
     &           &    8 &  1.91e-08 &  1.67e-15 &      \\ 
     &           &    9 &  1.91e-08 &  5.45e-16 &      \\ 
     &           &   10 &  1.91e-08 &  5.45e-16 &      \\ 
3558 &  3.56e+04 &   10 &           &           & iters  \\ 
 \hdashline 
     &           &    1 &  1.85e-08 &  6.67e-10 &      \\ 
     &           &    2 &  1.84e-08 &  5.27e-16 &      \\ 
     &           &    3 &  1.84e-08 &  5.26e-16 &      \\ 
     &           &    4 &  1.84e-08 &  5.26e-16 &      \\ 
     &           &    5 &  1.84e-08 &  5.25e-16 &      \\ 
     &           &    6 &  5.94e-08 &  5.24e-16 &      \\ 
     &           &    7 &  1.84e-08 &  1.69e-15 &      \\ 
     &           &    8 &  1.84e-08 &  5.26e-16 &      \\ 
     &           &    9 &  1.84e-08 &  5.25e-16 &      \\ 
     &           &   10 &  1.83e-08 &  5.24e-16 &      \\ 
3559 &  3.56e+04 &   10 &           &           & iters  \\ 
 \hdashline 
     &           &    1 &  5.89e-08 &  6.55e-10 &      \\ 
     &           &    2 &  1.89e-08 &  1.68e-15 &      \\ 
     &           &    3 &  1.89e-08 &  5.39e-16 &      \\ 
     &           &    4 &  5.89e-08 &  5.38e-16 &      \\ 
     &           &    5 &  1.89e-08 &  1.68e-15 &      \\ 
     &           &    6 &  1.89e-08 &  5.40e-16 &      \\ 
     &           &    7 &  1.89e-08 &  5.39e-16 &      \\ 
     &           &    8 &  5.89e-08 &  5.38e-16 &      \\ 
     &           &    9 &  1.89e-08 &  1.68e-15 &      \\ 
     &           &   10 &  1.89e-08 &  5.40e-16 &      \\ 
3560 &  3.56e+04 &   10 &           &           & iters  \\ 
 \hdashline 
     &           &    1 &  5.68e-08 &  2.78e-10 &      \\ 
     &           &    2 &  2.10e-08 &  1.62e-15 &      \\ 
     &           &    3 &  2.09e-08 &  5.99e-16 &      \\ 
     &           &    4 &  2.09e-08 &  5.98e-16 &      \\ 
     &           &    5 &  5.68e-08 &  5.97e-16 &      \\ 
     &           &    6 &  2.10e-08 &  1.62e-15 &      \\ 
     &           &    7 &  2.09e-08 &  5.99e-16 &      \\ 
     &           &    8 &  2.09e-08 &  5.98e-16 &      \\ 
     &           &    9 &  5.68e-08 &  5.97e-16 &      \\ 
     &           &   10 &  2.10e-08 &  1.62e-15 &      \\ 
3561 &  3.56e+04 &   10 &           &           & iters  \\ 
 \hdashline 
     &           &    1 &  1.06e-09 &  4.05e-10 &      \\ 
     &           &    2 &  1.06e-09 &  3.03e-17 &      \\ 
     &           &    3 &  1.06e-09 &  3.02e-17 &      \\ 
     &           &    4 &  1.06e-09 &  3.02e-17 &      \\ 
     &           &    5 &  1.06e-09 &  3.02e-17 &      \\ 
     &           &    6 &  1.05e-09 &  3.01e-17 &      \\ 
     &           &    7 &  1.05e-09 &  3.01e-17 &      \\ 
     &           &    8 &  1.05e-09 &  3.00e-17 &      \\ 
     &           &    9 &  1.05e-09 &  3.00e-17 &      \\ 
     &           &   10 &  1.05e-09 &  2.99e-17 &      \\ 
3562 &  3.56e+04 &   10 &           &           & iters  \\ 
 \hdashline 
     &           &    1 &  3.18e-10 &  6.39e-10 &      \\ 
     &           &    2 &  3.17e-10 &  9.07e-18 &      \\ 
     &           &    3 &  3.17e-10 &  9.06e-18 &      \\ 
     &           &    4 &  3.17e-10 &  9.05e-18 &      \\ 
     &           &    5 &  3.16e-10 &  9.03e-18 &      \\ 
     &           &    6 &  3.16e-10 &  9.02e-18 &      \\ 
     &           &    7 &  3.15e-10 &  9.01e-18 &      \\ 
     &           &    8 &  3.15e-10 &  9.00e-18 &      \\ 
     &           &    9 &  3.14e-10 &  8.98e-18 &      \\ 
     &           &   10 &  3.14e-10 &  8.97e-18 &      \\ 
3563 &  3.56e+04 &   10 &           &           & iters  \\ 
 \hdashline 
     &           &    1 &  5.07e-08 &  3.10e-10 &      \\ 
     &           &    2 &  2.71e-08 &  1.45e-15 &      \\ 
     &           &    3 &  2.71e-08 &  7.73e-16 &      \\ 
     &           &    4 &  5.07e-08 &  7.72e-16 &      \\ 
     &           &    5 &  2.71e-08 &  1.45e-15 &      \\ 
     &           &    6 &  2.71e-08 &  7.73e-16 &      \\ 
     &           &    7 &  5.07e-08 &  7.72e-16 &      \\ 
     &           &    8 &  2.71e-08 &  1.45e-15 &      \\ 
     &           &    9 &  2.70e-08 &  7.73e-16 &      \\ 
     &           &   10 &  5.07e-08 &  7.72e-16 &      \\ 
3564 &  3.56e+04 &   10 &           &           & iters  \\ 
 \hdashline 
     &           &    1 &  7.36e-08 &  8.15e-11 &      \\ 
     &           &    2 &  8.19e-08 &  2.10e-15 &      \\ 
     &           &    3 &  7.36e-08 &  2.34e-15 &      \\ 
     &           &    4 &  4.19e-09 &  2.10e-15 &      \\ 
     &           &    5 &  4.19e-09 &  1.20e-16 &      \\ 
     &           &    6 &  4.18e-09 &  1.19e-16 &      \\ 
     &           &    7 &  4.17e-09 &  1.19e-16 &      \\ 
     &           &    8 &  4.17e-09 &  1.19e-16 &      \\ 
     &           &    9 &  4.16e-09 &  1.19e-16 &      \\ 
     &           &   10 &  4.16e-09 &  1.19e-16 &      \\ 
3565 &  3.56e+04 &   10 &           &           & iters  \\ 
 \hdashline 
     &           &    1 &  8.89e-09 &  1.14e-10 &      \\ 
     &           &    2 &  8.88e-09 &  2.54e-16 &      \\ 
     &           &    3 &  6.88e-08 &  2.53e-16 &      \\ 
     &           &    4 &  8.97e-09 &  1.96e-15 &      \\ 
     &           &    5 &  8.95e-09 &  2.56e-16 &      \\ 
     &           &    6 &  8.94e-09 &  2.56e-16 &      \\ 
     &           &    7 &  8.93e-09 &  2.55e-16 &      \\ 
     &           &    8 &  8.92e-09 &  2.55e-16 &      \\ 
     &           &    9 &  8.90e-09 &  2.54e-16 &      \\ 
     &           &   10 &  8.89e-09 &  2.54e-16 &      \\ 
3566 &  3.56e+04 &   10 &           &           & iters  \\ 
 \hdashline 
     &           &    1 &  1.43e-08 &  2.49e-10 &      \\ 
     &           &    2 &  6.34e-08 &  4.08e-16 &      \\ 
     &           &    3 &  1.44e-08 &  1.81e-15 &      \\ 
     &           &    4 &  1.43e-08 &  4.10e-16 &      \\ 
     &           &    5 &  1.43e-08 &  4.09e-16 &      \\ 
     &           &    6 &  1.43e-08 &  4.08e-16 &      \\ 
     &           &    7 &  1.43e-08 &  4.08e-16 &      \\ 
     &           &    8 &  6.34e-08 &  4.07e-16 &      \\ 
     &           &    9 &  1.43e-08 &  1.81e-15 &      \\ 
     &           &   10 &  1.43e-08 &  4.09e-16 &      \\ 
3567 &  3.57e+04 &   10 &           &           & iters  \\ 
 \hdashline 
     &           &    1 &  5.42e-08 &  1.39e-10 &      \\ 
     &           &    2 &  2.35e-08 &  1.55e-15 &      \\ 
     &           &    3 &  2.35e-08 &  6.71e-16 &      \\ 
     &           &    4 &  5.42e-08 &  6.70e-16 &      \\ 
     &           &    5 &  2.35e-08 &  1.55e-15 &      \\ 
     &           &    6 &  2.35e-08 &  6.71e-16 &      \\ 
     &           &    7 &  2.35e-08 &  6.71e-16 &      \\ 
     &           &    8 &  5.43e-08 &  6.70e-16 &      \\ 
     &           &    9 &  2.35e-08 &  1.55e-15 &      \\ 
     &           &   10 &  2.35e-08 &  6.71e-16 &      \\ 
3568 &  3.57e+04 &   10 &           &           & iters  \\ 
 \hdashline 
     &           &    1 &  4.93e-08 &  3.57e-10 &      \\ 
     &           &    2 &  2.84e-08 &  1.41e-15 &      \\ 
     &           &    3 &  2.84e-08 &  8.11e-16 &      \\ 
     &           &    4 &  4.94e-08 &  8.10e-16 &      \\ 
     &           &    5 &  2.84e-08 &  1.41e-15 &      \\ 
     &           &    6 &  4.93e-08 &  8.11e-16 &      \\ 
     &           &    7 &  2.84e-08 &  1.41e-15 &      \\ 
     &           &    8 &  2.84e-08 &  8.12e-16 &      \\ 
     &           &    9 &  4.93e-08 &  8.10e-16 &      \\ 
     &           &   10 &  2.84e-08 &  1.41e-15 &      \\ 
3569 &  3.57e+04 &   10 &           &           & iters  \\ 
 \hdashline 
     &           &    1 &  2.11e-08 &  6.64e-10 &      \\ 
     &           &    2 &  2.11e-08 &  6.03e-16 &      \\ 
     &           &    3 &  5.66e-08 &  6.02e-16 &      \\ 
     &           &    4 &  2.12e-08 &  1.62e-15 &      \\ 
     &           &    5 &  2.11e-08 &  6.04e-16 &      \\ 
     &           &    6 &  5.66e-08 &  6.03e-16 &      \\ 
     &           &    7 &  2.12e-08 &  1.62e-15 &      \\ 
     &           &    8 &  2.11e-08 &  6.04e-16 &      \\ 
     &           &    9 &  2.11e-08 &  6.04e-16 &      \\ 
     &           &   10 &  5.66e-08 &  6.03e-16 &      \\ 
3570 &  3.57e+04 &   10 &           &           & iters  \\ 
 \hdashline 
     &           &    1 &  2.28e-09 &  3.72e-10 &      \\ 
     &           &    2 &  2.28e-09 &  6.50e-17 &      \\ 
     &           &    3 &  2.27e-09 &  6.49e-17 &      \\ 
     &           &    4 &  2.27e-09 &  6.48e-17 &      \\ 
     &           &    5 &  2.27e-09 &  6.47e-17 &      \\ 
     &           &    6 &  2.26e-09 &  6.47e-17 &      \\ 
     &           &    7 &  2.26e-09 &  6.46e-17 &      \\ 
     &           &    8 &  2.26e-09 &  6.45e-17 &      \\ 
     &           &    9 &  2.25e-09 &  6.44e-17 &      \\ 
     &           &   10 &  2.25e-09 &  6.43e-17 &      \\ 
3571 &  3.57e+04 &   10 &           &           & iters  \\ 
 \hdashline 
     &           &    1 &  5.03e-08 &  1.78e-10 &      \\ 
     &           &    2 &  2.75e-08 &  1.43e-15 &      \\ 
     &           &    3 &  2.75e-08 &  7.85e-16 &      \\ 
     &           &    4 &  5.03e-08 &  7.83e-16 &      \\ 
     &           &    5 &  2.75e-08 &  1.43e-15 &      \\ 
     &           &    6 &  2.74e-08 &  7.84e-16 &      \\ 
     &           &    7 &  5.03e-08 &  7.83e-16 &      \\ 
     &           &    8 &  2.75e-08 &  1.44e-15 &      \\ 
     &           &    9 &  2.74e-08 &  7.84e-16 &      \\ 
     &           &   10 &  5.03e-08 &  7.83e-16 &      \\ 
3572 &  3.57e+04 &   10 &           &           & iters  \\ 
 \hdashline 
     &           &    1 &  4.47e-08 &  3.57e-10 &      \\ 
     &           &    2 &  4.47e-08 &  1.28e-15 &      \\ 
     &           &    3 &  3.31e-08 &  1.27e-15 &      \\ 
     &           &    4 &  4.46e-08 &  9.45e-16 &      \\ 
     &           &    5 &  3.31e-08 &  1.27e-15 &      \\ 
     &           &    6 &  3.31e-08 &  9.45e-16 &      \\ 
     &           &    7 &  4.47e-08 &  9.44e-16 &      \\ 
     &           &    8 &  3.31e-08 &  1.27e-15 &      \\ 
     &           &    9 &  4.47e-08 &  9.44e-16 &      \\ 
     &           &   10 &  3.31e-08 &  1.27e-15 &      \\ 
3573 &  3.57e+04 &   10 &           &           & iters  \\ 
 \hdashline 
     &           &    1 &  4.47e-09 &  2.85e-10 &      \\ 
     &           &    2 &  4.46e-09 &  1.28e-16 &      \\ 
     &           &    3 &  4.46e-09 &  1.27e-16 &      \\ 
     &           &    4 &  4.45e-09 &  1.27e-16 &      \\ 
     &           &    5 &  4.44e-09 &  1.27e-16 &      \\ 
     &           &    6 &  4.44e-09 &  1.27e-16 &      \\ 
     &           &    7 &  4.43e-09 &  1.27e-16 &      \\ 
     &           &    8 &  4.42e-09 &  1.26e-16 &      \\ 
     &           &    9 &  4.42e-09 &  1.26e-16 &      \\ 
     &           &   10 &  4.41e-09 &  1.26e-16 &      \\ 
3574 &  3.57e+04 &   10 &           &           & iters  \\ 
 \hdashline 
     &           &    1 &  3.03e-08 &  1.97e-10 &      \\ 
     &           &    2 &  3.02e-08 &  8.64e-16 &      \\ 
     &           &    3 &  4.75e-08 &  8.63e-16 &      \\ 
     &           &    4 &  3.02e-08 &  1.36e-15 &      \\ 
     &           &    5 &  3.02e-08 &  8.63e-16 &      \\ 
     &           &    6 &  4.75e-08 &  8.62e-16 &      \\ 
     &           &    7 &  3.02e-08 &  1.36e-15 &      \\ 
     &           &    8 &  4.75e-08 &  8.63e-16 &      \\ 
     &           &    9 &  3.03e-08 &  1.36e-15 &      \\ 
     &           &   10 &  3.02e-08 &  8.63e-16 &      \\ 
3575 &  3.57e+04 &   10 &           &           & iters  \\ 
 \hdashline 
     &           &    1 &  1.96e-08 &  4.65e-10 &      \\ 
     &           &    2 &  1.96e-08 &  5.61e-16 &      \\ 
     &           &    3 &  1.96e-08 &  5.60e-16 &      \\ 
     &           &    4 &  1.96e-08 &  5.59e-16 &      \\ 
     &           &    5 &  1.95e-08 &  5.58e-16 &      \\ 
     &           &    6 &  5.82e-08 &  5.57e-16 &      \\ 
     &           &    7 &  1.96e-08 &  1.66e-15 &      \\ 
     &           &    8 &  1.96e-08 &  5.59e-16 &      \\ 
     &           &    9 &  1.95e-08 &  5.58e-16 &      \\ 
     &           &   10 &  5.82e-08 &  5.57e-16 &      \\ 
3576 &  3.58e+04 &   10 &           &           & iters  \\ 
 \hdashline 
     &           &    1 &  1.59e-08 &  3.05e-10 &      \\ 
     &           &    2 &  6.18e-08 &  4.54e-16 &      \\ 
     &           &    3 &  1.60e-08 &  1.76e-15 &      \\ 
     &           &    4 &  1.59e-08 &  4.56e-16 &      \\ 
     &           &    5 &  1.59e-08 &  4.55e-16 &      \\ 
     &           &    6 &  6.18e-08 &  4.54e-16 &      \\ 
     &           &    7 &  1.60e-08 &  1.76e-15 &      \\ 
     &           &    8 &  1.60e-08 &  4.56e-16 &      \\ 
     &           &    9 &  1.59e-08 &  4.56e-16 &      \\ 
     &           &   10 &  1.59e-08 &  4.55e-16 &      \\ 
3577 &  3.58e+04 &   10 &           &           & iters  \\ 
 \hdashline 
     &           &    1 &  2.12e-08 &  8.44e-10 &      \\ 
     &           &    2 &  5.65e-08 &  6.04e-16 &      \\ 
     &           &    3 &  2.12e-08 &  1.61e-15 &      \\ 
     &           &    4 &  2.12e-08 &  6.06e-16 &      \\ 
     &           &    5 &  5.65e-08 &  6.05e-16 &      \\ 
     &           &    6 &  2.12e-08 &  1.61e-15 &      \\ 
     &           &    7 &  2.12e-08 &  6.06e-16 &      \\ 
     &           &    8 &  2.12e-08 &  6.06e-16 &      \\ 
     &           &    9 &  5.65e-08 &  6.05e-16 &      \\ 
     &           &   10 &  2.12e-08 &  1.61e-15 &      \\ 
3578 &  3.58e+04 &   10 &           &           & iters  \\ 
 \hdashline 
     &           &    1 &  2.30e-08 &  1.79e-09 &      \\ 
     &           &    2 &  2.29e-08 &  6.55e-16 &      \\ 
     &           &    3 &  5.48e-08 &  6.54e-16 &      \\ 
     &           &    4 &  2.30e-08 &  1.56e-15 &      \\ 
     &           &    5 &  2.29e-08 &  6.56e-16 &      \\ 
     &           &    6 &  5.48e-08 &  6.55e-16 &      \\ 
     &           &    7 &  2.30e-08 &  1.56e-15 &      \\ 
     &           &    8 &  2.30e-08 &  6.56e-16 &      \\ 
     &           &    9 &  5.48e-08 &  6.55e-16 &      \\ 
     &           &   10 &  2.30e-08 &  1.56e-15 &      \\ 
3579 &  3.58e+04 &   10 &           &           & iters  \\ 
 \hdashline 
     &           &    1 &  5.17e-09 &  1.57e-09 &      \\ 
     &           &    2 &  5.17e-09 &  1.48e-16 &      \\ 
     &           &    3 &  5.16e-09 &  1.47e-16 &      \\ 
     &           &    4 &  5.15e-09 &  1.47e-16 &      \\ 
     &           &    5 &  7.25e-08 &  1.47e-16 &      \\ 
     &           &    6 &  4.41e-08 &  2.07e-15 &      \\ 
     &           &    7 &  5.19e-09 &  1.26e-15 &      \\ 
     &           &    8 &  5.18e-09 &  1.48e-16 &      \\ 
     &           &    9 &  5.17e-09 &  1.48e-16 &      \\ 
     &           &   10 &  5.16e-09 &  1.48e-16 &      \\ 
3580 &  3.58e+04 &   10 &           &           & iters  \\ 
 \hdashline 
     &           &    1 &  2.08e-08 &  1.01e-09 &      \\ 
     &           &    2 &  2.07e-08 &  5.92e-16 &      \\ 
     &           &    3 &  2.07e-08 &  5.91e-16 &      \\ 
     &           &    4 &  5.70e-08 &  5.91e-16 &      \\ 
     &           &    5 &  2.07e-08 &  1.63e-15 &      \\ 
     &           &    6 &  2.07e-08 &  5.92e-16 &      \\ 
     &           &    7 &  5.70e-08 &  5.91e-16 &      \\ 
     &           &    8 &  2.08e-08 &  1.63e-15 &      \\ 
     &           &    9 &  2.07e-08 &  5.93e-16 &      \\ 
     &           &   10 &  2.07e-08 &  5.92e-16 &      \\ 
3581 &  3.58e+04 &   10 &           &           & iters  \\ 
 \hdashline 
     &           &    1 &  1.41e-08 &  7.57e-10 &      \\ 
     &           &    2 &  1.41e-08 &  4.03e-16 &      \\ 
     &           &    3 &  6.36e-08 &  4.02e-16 &      \\ 
     &           &    4 &  1.42e-08 &  1.82e-15 &      \\ 
     &           &    5 &  1.41e-08 &  4.04e-16 &      \\ 
     &           &    6 &  1.41e-08 &  4.04e-16 &      \\ 
     &           &    7 &  1.41e-08 &  4.03e-16 &      \\ 
     &           &    8 &  6.36e-08 &  4.03e-16 &      \\ 
     &           &    9 &  1.42e-08 &  1.82e-15 &      \\ 
     &           &   10 &  1.42e-08 &  4.05e-16 &      \\ 
3582 &  3.58e+04 &   10 &           &           & iters  \\ 
 \hdashline 
     &           &    1 &  1.65e-08 &  1.69e-10 &      \\ 
     &           &    2 &  2.23e-08 &  4.71e-16 &      \\ 
     &           &    3 &  1.65e-08 &  6.38e-16 &      \\ 
     &           &    4 &  2.23e-08 &  4.72e-16 &      \\ 
     &           &    5 &  1.65e-08 &  6.38e-16 &      \\ 
     &           &    6 &  1.65e-08 &  4.72e-16 &      \\ 
     &           &    7 &  2.24e-08 &  4.71e-16 &      \\ 
     &           &    8 &  1.65e-08 &  6.38e-16 &      \\ 
     &           &    9 &  2.23e-08 &  4.72e-16 &      \\ 
     &           &   10 &  1.65e-08 &  6.38e-16 &      \\ 
3583 &  3.58e+04 &   10 &           &           & iters  \\ 
 \hdashline 
     &           &    1 &  2.86e-08 &  1.91e-11 &      \\ 
     &           &    2 &  4.91e-08 &  8.18e-16 &      \\ 
     &           &    3 &  2.87e-08 &  1.40e-15 &      \\ 
     &           &    4 &  2.86e-08 &  8.18e-16 &      \\ 
     &           &    5 &  4.91e-08 &  8.17e-16 &      \\ 
     &           &    6 &  2.87e-08 &  1.40e-15 &      \\ 
     &           &    7 &  4.91e-08 &  8.18e-16 &      \\ 
     &           &    8 &  2.87e-08 &  1.40e-15 &      \\ 
     &           &    9 &  2.87e-08 &  8.19e-16 &      \\ 
     &           &   10 &  4.91e-08 &  8.18e-16 &      \\ 
3584 &  3.58e+04 &   10 &           &           & iters  \\ 
 \hdashline 
     &           &    1 &  1.19e-08 &  1.03e-10 &      \\ 
     &           &    2 &  1.19e-08 &  3.40e-16 &      \\ 
     &           &    3 &  1.19e-08 &  3.40e-16 &      \\ 
     &           &    4 &  2.70e-08 &  3.39e-16 &      \\ 
     &           &    5 &  1.19e-08 &  7.70e-16 &      \\ 
     &           &    6 &  1.19e-08 &  3.40e-16 &      \\ 
     &           &    7 &  2.70e-08 &  3.39e-16 &      \\ 
     &           &    8 &  1.19e-08 &  7.70e-16 &      \\ 
     &           &    9 &  1.19e-08 &  3.40e-16 &      \\ 
     &           &   10 &  2.70e-08 &  3.39e-16 &      \\ 
3585 &  3.58e+04 &   10 &           &           & iters  \\ 
 \hdashline 
     &           &    1 &  5.46e-09 &  3.04e-10 &      \\ 
     &           &    2 &  3.34e-08 &  1.56e-16 &      \\ 
     &           &    3 &  5.50e-09 &  9.53e-16 &      \\ 
     &           &    4 &  5.49e-09 &  1.57e-16 &      \\ 
     &           &    5 &  5.49e-09 &  1.57e-16 &      \\ 
     &           &    6 &  5.48e-09 &  1.57e-16 &      \\ 
     &           &    7 &  5.47e-09 &  1.56e-16 &      \\ 
     &           &    8 &  5.46e-09 &  1.56e-16 &      \\ 
     &           &    9 &  3.34e-08 &  1.56e-16 &      \\ 
     &           &   10 &  5.50e-09 &  9.53e-16 &      \\ 
3586 &  3.58e+04 &   10 &           &           & iters  \\ 
 \hdashline 
     &           &    1 &  3.33e-08 &  9.33e-10 &      \\ 
     &           &    2 &  4.45e-08 &  9.50e-16 &      \\ 
     &           &    3 &  3.33e-08 &  1.27e-15 &      \\ 
     &           &    4 &  4.44e-08 &  9.50e-16 &      \\ 
     &           &    5 &  3.33e-08 &  1.27e-15 &      \\ 
     &           &    6 &  3.33e-08 &  9.51e-16 &      \\ 
     &           &    7 &  4.45e-08 &  9.49e-16 &      \\ 
     &           &    8 &  3.33e-08 &  1.27e-15 &      \\ 
     &           &    9 &  4.45e-08 &  9.50e-16 &      \\ 
     &           &   10 &  3.33e-08 &  1.27e-15 &      \\ 
3587 &  3.59e+04 &   10 &           &           & iters  \\ 
 \hdashline 
     &           &    1 &  8.18e-09 &  1.45e-09 &      \\ 
     &           &    2 &  8.17e-09 &  2.33e-16 &      \\ 
     &           &    3 &  8.16e-09 &  2.33e-16 &      \\ 
     &           &    4 &  6.95e-08 &  2.33e-16 &      \\ 
     &           &    5 &  8.24e-09 &  1.98e-15 &      \\ 
     &           &    6 &  8.23e-09 &  2.35e-16 &      \\ 
     &           &    7 &  8.22e-09 &  2.35e-16 &      \\ 
     &           &    8 &  8.21e-09 &  2.35e-16 &      \\ 
     &           &    9 &  8.20e-09 &  2.34e-16 &      \\ 
     &           &   10 &  8.18e-09 &  2.34e-16 &      \\ 
3588 &  3.59e+04 &   10 &           &           & iters  \\ 
 \hdashline 
     &           &    1 &  9.01e-09 &  1.19e-09 &      \\ 
     &           &    2 &  9.00e-09 &  2.57e-16 &      \\ 
     &           &    3 &  8.99e-09 &  2.57e-16 &      \\ 
     &           &    4 &  6.87e-08 &  2.56e-16 &      \\ 
     &           &    5 &  9.07e-09 &  1.96e-15 &      \\ 
     &           &    6 &  9.06e-09 &  2.59e-16 &      \\ 
     &           &    7 &  9.05e-09 &  2.59e-16 &      \\ 
     &           &    8 &  9.03e-09 &  2.58e-16 &      \\ 
     &           &    9 &  9.02e-09 &  2.58e-16 &      \\ 
     &           &   10 &  9.01e-09 &  2.57e-16 &      \\ 
3589 &  3.59e+04 &   10 &           &           & iters  \\ 
 \hdashline 
     &           &    1 &  6.59e-08 &  6.51e-10 &      \\ 
     &           &    2 &  1.19e-08 &  1.88e-15 &      \\ 
     &           &    3 &  1.19e-08 &  3.41e-16 &      \\ 
     &           &    4 &  1.19e-08 &  3.40e-16 &      \\ 
     &           &    5 &  1.19e-08 &  3.40e-16 &      \\ 
     &           &    6 &  1.19e-08 &  3.39e-16 &      \\ 
     &           &    7 &  1.18e-08 &  3.39e-16 &      \\ 
     &           &    8 &  6.59e-08 &  3.38e-16 &      \\ 
     &           &    9 &  1.19e-08 &  1.88e-15 &      \\ 
     &           &   10 &  1.19e-08 &  3.40e-16 &      \\ 
3590 &  3.59e+04 &   10 &           &           & iters  \\ 
 \hdashline 
     &           &    1 &  1.09e-08 &  6.15e-10 &      \\ 
     &           &    2 &  1.08e-08 &  3.10e-16 &      \\ 
     &           &    3 &  1.08e-08 &  3.09e-16 &      \\ 
     &           &    4 &  1.08e-08 &  3.09e-16 &      \\ 
     &           &    5 &  1.08e-08 &  3.09e-16 &      \\ 
     &           &    6 &  6.69e-08 &  3.08e-16 &      \\ 
     &           &    7 &  1.09e-08 &  1.91e-15 &      \\ 
     &           &    8 &  1.09e-08 &  3.10e-16 &      \\ 
     &           &    9 &  1.08e-08 &  3.10e-16 &      \\ 
     &           &   10 &  1.08e-08 &  3.10e-16 &      \\ 
3591 &  3.59e+04 &   10 &           &           & iters  \\ 
 \hdashline 
     &           &    1 &  3.20e-08 &  1.25e-11 &      \\ 
     &           &    2 &  3.19e-08 &  9.12e-16 &      \\ 
     &           &    3 &  4.58e-08 &  9.11e-16 &      \\ 
     &           &    4 &  3.19e-08 &  1.31e-15 &      \\ 
     &           &    5 &  3.19e-08 &  9.12e-16 &      \\ 
     &           &    6 &  4.58e-08 &  9.10e-16 &      \\ 
     &           &    7 &  3.19e-08 &  1.31e-15 &      \\ 
     &           &    8 &  4.58e-08 &  9.11e-16 &      \\ 
     &           &    9 &  3.19e-08 &  1.31e-15 &      \\ 
     &           &   10 &  4.58e-08 &  9.11e-16 &      \\ 
3592 &  3.59e+04 &   10 &           &           & iters  \\ 
 \hdashline 
     &           &    1 &  2.22e-08 &  6.43e-10 &      \\ 
     &           &    2 &  2.21e-08 &  6.32e-16 &      \\ 
     &           &    3 &  2.21e-08 &  6.31e-16 &      \\ 
     &           &    4 &  5.56e-08 &  6.30e-16 &      \\ 
     &           &    5 &  2.21e-08 &  1.59e-15 &      \\ 
     &           &    6 &  2.21e-08 &  6.32e-16 &      \\ 
     &           &    7 &  5.56e-08 &  6.31e-16 &      \\ 
     &           &    8 &  2.22e-08 &  1.59e-15 &      \\ 
     &           &    9 &  2.21e-08 &  6.32e-16 &      \\ 
     &           &   10 &  2.21e-08 &  6.31e-16 &      \\ 
3593 &  3.59e+04 &   10 &           &           & iters  \\ 
 \hdashline 
     &           &    1 &  2.54e-08 &  3.37e-10 &      \\ 
     &           &    2 &  2.54e-08 &  7.25e-16 &      \\ 
     &           &    3 &  2.53e-08 &  7.24e-16 &      \\ 
     &           &    4 &  5.24e-08 &  7.23e-16 &      \\ 
     &           &    5 &  2.54e-08 &  1.50e-15 &      \\ 
     &           &    6 &  2.53e-08 &  7.24e-16 &      \\ 
     &           &    7 &  5.24e-08 &  7.23e-16 &      \\ 
     &           &    8 &  2.54e-08 &  1.49e-15 &      \\ 
     &           &    9 &  2.53e-08 &  7.24e-16 &      \\ 
     &           &   10 &  5.24e-08 &  7.23e-16 &      \\ 
3594 &  3.59e+04 &   10 &           &           & iters  \\ 
 \hdashline 
     &           &    1 &  1.76e-08 &  9.56e-11 &      \\ 
     &           &    2 &  1.76e-08 &  5.03e-16 &      \\ 
     &           &    3 &  6.01e-08 &  5.02e-16 &      \\ 
     &           &    4 &  1.77e-08 &  1.72e-15 &      \\ 
     &           &    5 &  1.76e-08 &  5.04e-16 &      \\ 
     &           &    6 &  1.76e-08 &  5.03e-16 &      \\ 
     &           &    7 &  1.76e-08 &  5.02e-16 &      \\ 
     &           &    8 &  6.01e-08 &  5.02e-16 &      \\ 
     &           &    9 &  1.76e-08 &  1.72e-15 &      \\ 
     &           &   10 &  1.76e-08 &  5.04e-16 &      \\ 
3595 &  3.59e+04 &   10 &           &           & iters  \\ 
 \hdashline 
     &           &    1 &  5.19e-08 &  6.60e-10 &      \\ 
     &           &    2 &  2.58e-08 &  1.48e-15 &      \\ 
     &           &    3 &  2.58e-08 &  7.37e-16 &      \\ 
     &           &    4 &  5.19e-08 &  7.36e-16 &      \\ 
     &           &    5 &  2.58e-08 &  1.48e-15 &      \\ 
     &           &    6 &  2.58e-08 &  7.37e-16 &      \\ 
     &           &    7 &  5.19e-08 &  7.36e-16 &      \\ 
     &           &    8 &  2.58e-08 &  1.48e-15 &      \\ 
     &           &    9 &  2.58e-08 &  7.37e-16 &      \\ 
     &           &   10 &  5.19e-08 &  7.36e-16 &      \\ 
3596 &  3.60e+04 &   10 &           &           & iters  \\ 
 \hdashline 
     &           &    1 &  4.37e-08 &  8.91e-10 &      \\ 
     &           &    2 &  4.36e-08 &  1.25e-15 &      \\ 
     &           &    3 &  3.42e-08 &  1.24e-15 &      \\ 
     &           &    4 &  4.36e-08 &  9.75e-16 &      \\ 
     &           &    5 &  3.42e-08 &  1.24e-15 &      \\ 
     &           &    6 &  4.36e-08 &  9.75e-16 &      \\ 
     &           &    7 &  3.42e-08 &  1.24e-15 &      \\ 
     &           &    8 &  4.36e-08 &  9.75e-16 &      \\ 
     &           &    9 &  3.42e-08 &  1.24e-15 &      \\ 
     &           &   10 &  3.41e-08 &  9.76e-16 &      \\ 
3597 &  3.60e+04 &   10 &           &           & iters  \\ 
 \hdashline 
     &           &    1 &  5.18e-08 &  9.76e-11 &      \\ 
     &           &    2 &  2.60e-08 &  1.48e-15 &      \\ 
     &           &    3 &  2.60e-08 &  7.42e-16 &      \\ 
     &           &    4 &  5.18e-08 &  7.41e-16 &      \\ 
     &           &    5 &  2.60e-08 &  1.48e-15 &      \\ 
     &           &    6 &  2.60e-08 &  7.42e-16 &      \\ 
     &           &    7 &  5.18e-08 &  7.41e-16 &      \\ 
     &           &    8 &  2.60e-08 &  1.48e-15 &      \\ 
     &           &    9 &  2.60e-08 &  7.42e-16 &      \\ 
     &           &   10 &  5.18e-08 &  7.41e-16 &      \\ 
3598 &  3.60e+04 &   10 &           &           & iters  \\ 
 \hdashline 
     &           &    1 &  3.83e-08 &  8.85e-10 &      \\ 
     &           &    2 &  3.95e-08 &  1.09e-15 &      \\ 
     &           &    3 &  3.83e-08 &  1.13e-15 &      \\ 
     &           &    4 &  3.95e-08 &  1.09e-15 &      \\ 
     &           &    5 &  3.83e-08 &  1.13e-15 &      \\ 
     &           &    6 &  3.95e-08 &  1.09e-15 &      \\ 
     &           &    7 &  3.83e-08 &  1.13e-15 &      \\ 
     &           &    8 &  3.82e-08 &  1.09e-15 &      \\ 
     &           &    9 &  3.95e-08 &  1.09e-15 &      \\ 
     &           &   10 &  3.82e-08 &  1.13e-15 &      \\ 
3599 &  3.60e+04 &   10 &           &           & iters  \\ 
 \hdashline 
     &           &    1 &  1.86e-08 &  4.08e-10 &      \\ 
     &           &    2 &  1.85e-08 &  5.29e-16 &      \\ 
     &           &    3 &  1.85e-08 &  5.29e-16 &      \\ 
     &           &    4 &  5.92e-08 &  5.28e-16 &      \\ 
     &           &    5 &  1.86e-08 &  1.69e-15 &      \\ 
     &           &    6 &  1.85e-08 &  5.30e-16 &      \\ 
     &           &    7 &  1.85e-08 &  5.29e-16 &      \\ 
     &           &    8 &  5.92e-08 &  5.28e-16 &      \\ 
     &           &    9 &  1.86e-08 &  1.69e-15 &      \\ 
     &           &   10 &  1.85e-08 &  5.30e-16 &      \\ 
3600 &  3.60e+04 &   10 &           &           & iters  \\ 
 \hdashline 
     &           &    1 &  4.52e-08 &  4.30e-10 &      \\ 
     &           &    2 &  3.25e-08 &  1.29e-15 &      \\ 
     &           &    3 &  4.52e-08 &  9.28e-16 &      \\ 
     &           &    4 &  3.25e-08 &  1.29e-15 &      \\ 
     &           &    5 &  3.25e-08 &  9.29e-16 &      \\ 
     &           &    6 &  4.52e-08 &  9.28e-16 &      \\ 
     &           &    7 &  3.25e-08 &  1.29e-15 &      \\ 
     &           &    8 &  4.52e-08 &  9.28e-16 &      \\ 
     &           &    9 &  3.25e-08 &  1.29e-15 &      \\ 
     &           &   10 &  3.25e-08 &  9.29e-16 &      \\ 
3601 &  3.60e+04 &   10 &           &           & iters  \\ 
 \hdashline 
     &           &    1 &  3.36e-08 &  1.08e-09 &      \\ 
     &           &    2 &  3.35e-08 &  9.59e-16 &      \\ 
     &           &    3 &  4.42e-08 &  9.57e-16 &      \\ 
     &           &    4 &  3.36e-08 &  1.26e-15 &      \\ 
     &           &    5 &  4.42e-08 &  9.58e-16 &      \\ 
     &           &    6 &  3.36e-08 &  1.26e-15 &      \\ 
     &           &    7 &  4.42e-08 &  9.58e-16 &      \\ 
     &           &    8 &  3.36e-08 &  1.26e-15 &      \\ 
     &           &    9 &  3.35e-08 &  9.58e-16 &      \\ 
     &           &   10 &  4.42e-08 &  9.57e-16 &      \\ 
3602 &  3.60e+04 &   10 &           &           & iters  \\ 
 \hdashline 
     &           &    1 &  4.36e-09 &  9.99e-10 &      \\ 
     &           &    2 &  4.35e-09 &  1.24e-16 &      \\ 
     &           &    3 &  4.35e-09 &  1.24e-16 &      \\ 
     &           &    4 &  4.34e-09 &  1.24e-16 &      \\ 
     &           &    5 &  4.34e-09 &  1.24e-16 &      \\ 
     &           &    6 &  4.33e-09 &  1.24e-16 &      \\ 
     &           &    7 &  4.32e-09 &  1.24e-16 &      \\ 
     &           &    8 &  4.32e-09 &  1.23e-16 &      \\ 
     &           &    9 &  4.31e-09 &  1.23e-16 &      \\ 
     &           &   10 &  4.30e-09 &  1.23e-16 &      \\ 
3603 &  3.60e+04 &   10 &           &           & iters  \\ 
 \hdashline 
     &           &    1 &  4.78e-09 &  2.99e-10 &      \\ 
     &           &    2 &  4.77e-09 &  1.36e-16 &      \\ 
     &           &    3 &  4.76e-09 &  1.36e-16 &      \\ 
     &           &    4 &  4.75e-09 &  1.36e-16 &      \\ 
     &           &    5 &  4.75e-09 &  1.36e-16 &      \\ 
     &           &    6 &  4.74e-09 &  1.36e-16 &      \\ 
     &           &    7 &  4.73e-09 &  1.35e-16 &      \\ 
     &           &    8 &  4.73e-09 &  1.35e-16 &      \\ 
     &           &    9 &  4.72e-09 &  1.35e-16 &      \\ 
     &           &   10 &  4.71e-09 &  1.35e-16 &      \\ 
3604 &  3.60e+04 &   10 &           &           & iters  \\ 
 \hdashline 
     &           &    1 &  8.10e-09 &  1.26e-10 &      \\ 
     &           &    2 &  8.09e-09 &  2.31e-16 &      \\ 
     &           &    3 &  8.08e-09 &  2.31e-16 &      \\ 
     &           &    4 &  8.07e-09 &  2.31e-16 &      \\ 
     &           &    5 &  8.05e-09 &  2.30e-16 &      \\ 
     &           &    6 &  8.04e-09 &  2.30e-16 &      \\ 
     &           &    7 &  8.03e-09 &  2.30e-16 &      \\ 
     &           &    8 &  6.97e-08 &  2.29e-16 &      \\ 
     &           &    9 &  8.58e-08 &  1.99e-15 &      \\ 
     &           &   10 &  6.97e-08 &  2.45e-15 &      \\ 
3605 &  3.60e+04 &   10 &           &           & iters  \\ 
 \hdashline 
     &           &    1 &  1.88e-10 &  3.22e-10 &      \\ 
     &           &    2 &  1.87e-10 &  5.36e-18 &      \\ 
     &           &    3 &  1.87e-10 &  5.35e-18 &      \\ 
     &           &    4 &  1.87e-10 &  5.34e-18 &      \\ 
     &           &    5 &  1.87e-10 &  5.33e-18 &      \\ 
     &           &    6 &  1.86e-10 &  5.32e-18 &      \\ 
     &           &    7 &  1.86e-10 &  5.32e-18 &      \\ 
     &           &    8 &  1.86e-10 &  5.31e-18 &      \\ 
     &           &    9 &  1.85e-10 &  5.30e-18 &      \\ 
     &           &   10 &  1.85e-10 &  5.29e-18 &      \\ 
3606 &  3.60e+04 &   10 &           &           & iters  \\ 
 \hdashline 
     &           &    1 &  2.96e-08 &  5.09e-10 &      \\ 
     &           &    2 &  4.81e-08 &  8.46e-16 &      \\ 
     &           &    3 &  2.97e-08 &  1.37e-15 &      \\ 
     &           &    4 &  2.96e-08 &  8.47e-16 &      \\ 
     &           &    5 &  4.81e-08 &  8.45e-16 &      \\ 
     &           &    6 &  2.97e-08 &  1.37e-15 &      \\ 
     &           &    7 &  4.81e-08 &  8.46e-16 &      \\ 
     &           &    8 &  2.97e-08 &  1.37e-15 &      \\ 
     &           &    9 &  2.96e-08 &  8.47e-16 &      \\ 
     &           &   10 &  4.81e-08 &  8.46e-16 &      \\ 
3607 &  3.61e+04 &   10 &           &           & iters  \\ 
 \hdashline 
     &           &    1 &  5.06e-08 &  9.60e-10 &      \\ 
     &           &    2 &  2.71e-08 &  1.45e-15 &      \\ 
     &           &    3 &  5.06e-08 &  7.74e-16 &      \\ 
     &           &    4 &  2.72e-08 &  1.44e-15 &      \\ 
     &           &    5 &  2.71e-08 &  7.75e-16 &      \\ 
     &           &    6 &  5.06e-08 &  7.74e-16 &      \\ 
     &           &    7 &  2.72e-08 &  1.44e-15 &      \\ 
     &           &    8 &  2.71e-08 &  7.75e-16 &      \\ 
     &           &    9 &  5.06e-08 &  7.74e-16 &      \\ 
     &           &   10 &  2.71e-08 &  1.44e-15 &      \\ 
3608 &  3.61e+04 &   10 &           &           & iters  \\ 
 \hdashline 
     &           &    1 &  7.53e-09 &  1.17e-09 &      \\ 
     &           &    2 &  7.52e-09 &  2.15e-16 &      \\ 
     &           &    3 &  7.51e-09 &  2.15e-16 &      \\ 
     &           &    4 &  7.50e-09 &  2.14e-16 &      \\ 
     &           &    5 &  7.49e-09 &  2.14e-16 &      \\ 
     &           &    6 &  7.48e-09 &  2.14e-16 &      \\ 
     &           &    7 &  7.02e-08 &  2.13e-16 &      \\ 
     &           &    8 &  7.57e-09 &  2.00e-15 &      \\ 
     &           &    9 &  7.56e-09 &  2.16e-16 &      \\ 
     &           &   10 &  7.55e-09 &  2.16e-16 &      \\ 
3609 &  3.61e+04 &   10 &           &           & iters  \\ 
 \hdashline 
     &           &    1 &  3.82e-08 &  4.50e-10 &      \\ 
     &           &    2 &  3.95e-08 &  1.09e-15 &      \\ 
     &           &    3 &  3.82e-08 &  1.13e-15 &      \\ 
     &           &    4 &  3.95e-08 &  1.09e-15 &      \\ 
     &           &    5 &  3.82e-08 &  1.13e-15 &      \\ 
     &           &    6 &  3.95e-08 &  1.09e-15 &      \\ 
     &           &    7 &  3.83e-08 &  1.13e-15 &      \\ 
     &           &    8 &  3.95e-08 &  1.09e-15 &      \\ 
     &           &    9 &  3.83e-08 &  1.13e-15 &      \\ 
     &           &   10 &  3.95e-08 &  1.09e-15 &      \\ 
3610 &  3.61e+04 &   10 &           &           & iters  \\ 
 \hdashline 
     &           &    1 &  1.82e-08 &  9.33e-10 &      \\ 
     &           &    2 &  5.95e-08 &  5.19e-16 &      \\ 
     &           &    3 &  1.82e-08 &  1.70e-15 &      \\ 
     &           &    4 &  1.82e-08 &  5.21e-16 &      \\ 
     &           &    5 &  1.82e-08 &  5.20e-16 &      \\ 
     &           &    6 &  5.95e-08 &  5.19e-16 &      \\ 
     &           &    7 &  1.82e-08 &  1.70e-15 &      \\ 
     &           &    8 &  1.82e-08 &  5.21e-16 &      \\ 
     &           &    9 &  1.82e-08 &  5.20e-16 &      \\ 
     &           &   10 &  1.82e-08 &  5.19e-16 &      \\ 
3611 &  3.61e+04 &   10 &           &           & iters  \\ 
 \hdashline 
     &           &    1 &  2.10e-08 &  1.38e-09 &      \\ 
     &           &    2 &  2.09e-08 &  5.99e-16 &      \\ 
     &           &    3 &  2.09e-08 &  5.98e-16 &      \\ 
     &           &    4 &  5.68e-08 &  5.97e-16 &      \\ 
     &           &    5 &  2.10e-08 &  1.62e-15 &      \\ 
     &           &    6 &  2.09e-08 &  5.98e-16 &      \\ 
     &           &    7 &  2.09e-08 &  5.97e-16 &      \\ 
     &           &    8 &  5.68e-08 &  5.97e-16 &      \\ 
     &           &    9 &  2.10e-08 &  1.62e-15 &      \\ 
     &           &   10 &  2.09e-08 &  5.98e-16 &      \\ 
3612 &  3.61e+04 &   10 &           &           & iters  \\ 
 \hdashline 
     &           &    1 &  1.69e-08 &  6.48e-10 &      \\ 
     &           &    2 &  1.68e-08 &  4.81e-16 &      \\ 
     &           &    3 &  1.68e-08 &  4.80e-16 &      \\ 
     &           &    4 &  1.68e-08 &  4.80e-16 &      \\ 
     &           &    5 &  1.68e-08 &  4.79e-16 &      \\ 
     &           &    6 &  1.67e-08 &  4.78e-16 &      \\ 
     &           &    7 &  6.10e-08 &  4.78e-16 &      \\ 
     &           &    8 &  1.68e-08 &  1.74e-15 &      \\ 
     &           &    9 &  1.68e-08 &  4.79e-16 &      \\ 
     &           &   10 &  1.67e-08 &  4.79e-16 &      \\ 
3613 &  3.61e+04 &   10 &           &           & iters  \\ 
 \hdashline 
     &           &    1 &  3.48e-08 &  2.14e-10 &      \\ 
     &           &    2 &  4.30e-08 &  9.92e-16 &      \\ 
     &           &    3 &  3.48e-08 &  1.23e-15 &      \\ 
     &           &    4 &  3.47e-08 &  9.92e-16 &      \\ 
     &           &    5 &  4.30e-08 &  9.91e-16 &      \\ 
     &           &    6 &  3.47e-08 &  1.23e-15 &      \\ 
     &           &    7 &  4.30e-08 &  9.91e-16 &      \\ 
     &           &    8 &  3.47e-08 &  1.23e-15 &      \\ 
     &           &    9 &  4.30e-08 &  9.91e-16 &      \\ 
     &           &   10 &  3.48e-08 &  1.23e-15 &      \\ 
3614 &  3.61e+04 &   10 &           &           & iters  \\ 
 \hdashline 
     &           &    1 &  5.26e-08 &  3.33e-10 &      \\ 
     &           &    2 &  2.52e-08 &  1.50e-15 &      \\ 
     &           &    3 &  2.52e-08 &  7.19e-16 &      \\ 
     &           &    4 &  5.26e-08 &  7.18e-16 &      \\ 
     &           &    5 &  2.52e-08 &  1.50e-15 &      \\ 
     &           &    6 &  2.52e-08 &  7.19e-16 &      \\ 
     &           &    7 &  5.26e-08 &  7.18e-16 &      \\ 
     &           &    8 &  2.52e-08 &  1.50e-15 &      \\ 
     &           &    9 &  2.52e-08 &  7.19e-16 &      \\ 
     &           &   10 &  2.51e-08 &  7.18e-16 &      \\ 
3615 &  3.61e+04 &   10 &           &           & iters  \\ 
 \hdashline 
     &           &    1 &  5.64e-09 &  4.02e-10 &      \\ 
     &           &    2 &  5.64e-09 &  1.61e-16 &      \\ 
     &           &    3 &  5.63e-09 &  1.61e-16 &      \\ 
     &           &    4 &  5.62e-09 &  1.61e-16 &      \\ 
     &           &    5 &  5.61e-09 &  1.60e-16 &      \\ 
     &           &    6 &  5.60e-09 &  1.60e-16 &      \\ 
     &           &    7 &  5.60e-09 &  1.60e-16 &      \\ 
     &           &    8 &  5.59e-09 &  1.60e-16 &      \\ 
     &           &    9 &  5.58e-09 &  1.59e-16 &      \\ 
     &           &   10 &  5.57e-09 &  1.59e-16 &      \\ 
3616 &  3.62e+04 &   10 &           &           & iters  \\ 
 \hdashline 
     &           &    1 &  2.83e-08 &  8.19e-10 &      \\ 
     &           &    2 &  2.83e-08 &  8.09e-16 &      \\ 
     &           &    3 &  2.82e-08 &  8.07e-16 &      \\ 
     &           &    4 &  4.95e-08 &  8.06e-16 &      \\ 
     &           &    5 &  2.83e-08 &  1.41e-15 &      \\ 
     &           &    6 &  4.95e-08 &  8.07e-16 &      \\ 
     &           &    7 &  2.83e-08 &  1.41e-15 &      \\ 
     &           &    8 &  2.83e-08 &  8.08e-16 &      \\ 
     &           &    9 &  4.95e-08 &  8.07e-16 &      \\ 
     &           &   10 &  2.83e-08 &  1.41e-15 &      \\ 
3617 &  3.62e+04 &   10 &           &           & iters  \\ 
 \hdashline 
     &           &    1 &  2.16e-08 &  1.35e-09 &      \\ 
     &           &    2 &  5.61e-08 &  6.16e-16 &      \\ 
     &           &    3 &  2.17e-08 &  1.60e-15 &      \\ 
     &           &    4 &  2.16e-08 &  6.18e-16 &      \\ 
     &           &    5 &  5.61e-08 &  6.17e-16 &      \\ 
     &           &    6 &  2.17e-08 &  1.60e-15 &      \\ 
     &           &    7 &  2.16e-08 &  6.18e-16 &      \\ 
     &           &    8 &  2.16e-08 &  6.18e-16 &      \\ 
     &           &    9 &  5.61e-08 &  6.17e-16 &      \\ 
     &           &   10 &  2.17e-08 &  1.60e-15 &      \\ 
3618 &  3.62e+04 &   10 &           &           & iters  \\ 
 \hdashline 
     &           &    1 &  7.72e-09 &  1.12e-09 &      \\ 
     &           &    2 &  7.71e-09 &  2.20e-16 &      \\ 
     &           &    3 &  7.70e-09 &  2.20e-16 &      \\ 
     &           &    4 &  7.69e-09 &  2.20e-16 &      \\ 
     &           &    5 &  7.00e-08 &  2.19e-16 &      \\ 
     &           &    6 &  4.66e-08 &  2.00e-15 &      \\ 
     &           &    7 &  7.71e-09 &  1.33e-15 &      \\ 
     &           &    8 &  7.70e-09 &  2.20e-16 &      \\ 
     &           &    9 &  7.69e-09 &  2.20e-16 &      \\ 
     &           &   10 &  7.00e-08 &  2.19e-16 &      \\ 
3619 &  3.62e+04 &   10 &           &           & iters  \\ 
 \hdashline 
     &           &    1 &  6.65e-08 &  8.62e-10 &      \\ 
     &           &    2 &  1.13e-08 &  1.90e-15 &      \\ 
     &           &    3 &  1.13e-08 &  3.22e-16 &      \\ 
     &           &    4 &  1.12e-08 &  3.22e-16 &      \\ 
     &           &    5 &  1.12e-08 &  3.21e-16 &      \\ 
     &           &    6 &  1.12e-08 &  3.21e-16 &      \\ 
     &           &    7 &  6.65e-08 &  3.20e-16 &      \\ 
     &           &    8 &  8.90e-08 &  1.90e-15 &      \\ 
     &           &    9 &  6.65e-08 &  2.54e-15 &      \\ 
     &           &   10 &  1.13e-08 &  1.90e-15 &      \\ 
3620 &  3.62e+04 &   10 &           &           & iters  \\ 
 \hdashline 
     &           &    1 &  1.67e-08 &  6.16e-10 &      \\ 
     &           &    2 &  1.67e-08 &  4.76e-16 &      \\ 
     &           &    3 &  6.11e-08 &  4.75e-16 &      \\ 
     &           &    4 &  1.67e-08 &  1.74e-15 &      \\ 
     &           &    5 &  1.67e-08 &  4.77e-16 &      \\ 
     &           &    6 &  1.67e-08 &  4.77e-16 &      \\ 
     &           &    7 &  6.10e-08 &  4.76e-16 &      \\ 
     &           &    8 &  1.67e-08 &  1.74e-15 &      \\ 
     &           &    9 &  1.67e-08 &  4.78e-16 &      \\ 
     &           &   10 &  1.67e-08 &  4.77e-16 &      \\ 
3621 &  3.62e+04 &   10 &           &           & iters  \\ 
 \hdashline 
     &           &    1 &  5.42e-08 &  2.11e-10 &      \\ 
     &           &    2 &  2.36e-08 &  1.55e-15 &      \\ 
     &           &    3 &  2.36e-08 &  6.73e-16 &      \\ 
     &           &    4 &  2.35e-08 &  6.73e-16 &      \\ 
     &           &    5 &  5.42e-08 &  6.72e-16 &      \\ 
     &           &    6 &  2.36e-08 &  1.55e-15 &      \\ 
     &           &    7 &  2.35e-08 &  6.73e-16 &      \\ 
     &           &    8 &  5.42e-08 &  6.72e-16 &      \\ 
     &           &    9 &  2.36e-08 &  1.55e-15 &      \\ 
     &           &   10 &  2.36e-08 &  6.73e-16 &      \\ 
3622 &  3.62e+04 &   10 &           &           & iters  \\ 
 \hdashline 
     &           &    1 &  3.06e-08 &  3.48e-10 &      \\ 
     &           &    2 &  3.06e-08 &  8.74e-16 &      \\ 
     &           &    3 &  4.72e-08 &  8.73e-16 &      \\ 
     &           &    4 &  3.06e-08 &  1.35e-15 &      \\ 
     &           &    5 &  4.71e-08 &  8.73e-16 &      \\ 
     &           &    6 &  3.06e-08 &  1.35e-15 &      \\ 
     &           &    7 &  3.06e-08 &  8.74e-16 &      \\ 
     &           &    8 &  4.71e-08 &  8.73e-16 &      \\ 
     &           &    9 &  3.06e-08 &  1.35e-15 &      \\ 
     &           &   10 &  4.71e-08 &  8.74e-16 &      \\ 
3623 &  3.62e+04 &   10 &           &           & iters  \\ 
 \hdashline 
     &           &    1 &  2.61e-08 &  2.12e-11 &      \\ 
     &           &    2 &  2.61e-08 &  7.45e-16 &      \\ 
     &           &    3 &  5.17e-08 &  7.44e-16 &      \\ 
     &           &    4 &  2.61e-08 &  1.47e-15 &      \\ 
     &           &    5 &  2.61e-08 &  7.45e-16 &      \\ 
     &           &    6 &  5.17e-08 &  7.44e-16 &      \\ 
     &           &    7 &  2.61e-08 &  1.47e-15 &      \\ 
     &           &    8 &  2.61e-08 &  7.45e-16 &      \\ 
     &           &    9 &  5.17e-08 &  7.44e-16 &      \\ 
     &           &   10 &  2.61e-08 &  1.47e-15 &      \\ 
3624 &  3.62e+04 &   10 &           &           & iters  \\ 
 \hdashline 
     &           &    1 &  3.77e-08 &  1.15e-09 &      \\ 
     &           &    2 &  4.00e-08 &  1.08e-15 &      \\ 
     &           &    3 &  3.77e-08 &  1.14e-15 &      \\ 
     &           &    4 &  4.00e-08 &  1.08e-15 &      \\ 
     &           &    5 &  3.77e-08 &  1.14e-15 &      \\ 
     &           &    6 &  4.00e-08 &  1.08e-15 &      \\ 
     &           &    7 &  3.77e-08 &  1.14e-15 &      \\ 
     &           &    8 &  4.00e-08 &  1.08e-15 &      \\ 
     &           &    9 &  3.77e-08 &  1.14e-15 &      \\ 
     &           &   10 &  4.00e-08 &  1.08e-15 &      \\ 
3625 &  3.62e+04 &   10 &           &           & iters  \\ 
 \hdashline 
     &           &    1 &  2.10e-08 &  2.15e-09 &      \\ 
     &           &    2 &  5.67e-08 &  6.00e-16 &      \\ 
     &           &    3 &  2.11e-08 &  1.62e-15 &      \\ 
     &           &    4 &  2.10e-08 &  6.01e-16 &      \\ 
     &           &    5 &  2.10e-08 &  6.00e-16 &      \\ 
     &           &    6 &  5.67e-08 &  5.99e-16 &      \\ 
     &           &    7 &  2.11e-08 &  1.62e-15 &      \\ 
     &           &    8 &  2.10e-08 &  6.01e-16 &      \\ 
     &           &    9 &  5.67e-08 &  6.00e-16 &      \\ 
     &           &   10 &  2.11e-08 &  1.62e-15 &      \\ 
3626 &  3.62e+04 &   10 &           &           & iters  \\ 
 \hdashline 
     &           &    1 &  4.54e-08 &  1.99e-09 &      \\ 
     &           &    2 &  3.24e-08 &  1.30e-15 &      \\ 
     &           &    3 &  4.54e-08 &  9.24e-16 &      \\ 
     &           &    4 &  3.24e-08 &  1.29e-15 &      \\ 
     &           &    5 &  4.53e-08 &  9.24e-16 &      \\ 
     &           &    6 &  3.24e-08 &  1.29e-15 &      \\ 
     &           &    7 &  3.24e-08 &  9.25e-16 &      \\ 
     &           &    8 &  4.54e-08 &  9.24e-16 &      \\ 
     &           &    9 &  3.24e-08 &  1.29e-15 &      \\ 
     &           &   10 &  4.54e-08 &  9.24e-16 &      \\ 
3627 &  3.63e+04 &   10 &           &           & iters  \\ 
 \hdashline 
     &           &    1 &  4.81e-08 &  1.42e-10 &      \\ 
     &           &    2 &  2.96e-08 &  1.37e-15 &      \\ 
     &           &    3 &  4.81e-08 &  8.46e-16 &      \\ 
     &           &    4 &  2.97e-08 &  1.37e-15 &      \\ 
     &           &    5 &  2.96e-08 &  8.47e-16 &      \\ 
     &           &    6 &  4.81e-08 &  8.45e-16 &      \\ 
     &           &    7 &  2.96e-08 &  1.37e-15 &      \\ 
     &           &    8 &  4.81e-08 &  8.46e-16 &      \\ 
     &           &    9 &  2.97e-08 &  1.37e-15 &      \\ 
     &           &   10 &  2.96e-08 &  8.47e-16 &      \\ 
3628 &  3.63e+04 &   10 &           &           & iters  \\ 
 \hdashline 
     &           &    1 &  6.36e-09 &  1.56e-09 &      \\ 
     &           &    2 &  6.35e-09 &  1.82e-16 &      \\ 
     &           &    3 &  6.34e-09 &  1.81e-16 &      \\ 
     &           &    4 &  6.33e-09 &  1.81e-16 &      \\ 
     &           &    5 &  6.32e-09 &  1.81e-16 &      \\ 
     &           &    6 &  7.14e-08 &  1.80e-16 &      \\ 
     &           &    7 &  8.41e-08 &  2.04e-15 &      \\ 
     &           &    8 &  7.14e-08 &  2.40e-15 &      \\ 
     &           &    9 &  6.40e-09 &  2.04e-15 &      \\ 
     &           &   10 &  6.39e-09 &  1.83e-16 &      \\ 
3629 &  3.63e+04 &   10 &           &           & iters  \\ 
 \hdashline 
     &           &    1 &  9.62e-09 &  3.26e-10 &      \\ 
     &           &    2 &  9.61e-09 &  2.75e-16 &      \\ 
     &           &    3 &  9.60e-09 &  2.74e-16 &      \\ 
     &           &    4 &  9.58e-09 &  2.74e-16 &      \\ 
     &           &    5 &  9.57e-09 &  2.74e-16 &      \\ 
     &           &    6 &  9.56e-09 &  2.73e-16 &      \\ 
     &           &    7 &  6.81e-08 &  2.73e-16 &      \\ 
     &           &    8 &  9.64e-09 &  1.94e-15 &      \\ 
     &           &    9 &  9.63e-09 &  2.75e-16 &      \\ 
     &           &   10 &  9.61e-09 &  2.75e-16 &      \\ 
3630 &  3.63e+04 &   10 &           &           & iters  \\ 
 \hdashline 
     &           &    1 &  3.32e-08 &  7.43e-10 &      \\ 
     &           &    2 &  3.32e-08 &  9.49e-16 &      \\ 
     &           &    3 &  4.45e-08 &  9.47e-16 &      \\ 
     &           &    4 &  3.32e-08 &  1.27e-15 &      \\ 
     &           &    5 &  4.45e-08 &  9.48e-16 &      \\ 
     &           &    6 &  3.32e-08 &  1.27e-15 &      \\ 
     &           &    7 &  3.32e-08 &  9.48e-16 &      \\ 
     &           &    8 &  4.46e-08 &  9.47e-16 &      \\ 
     &           &    9 &  3.32e-08 &  1.27e-15 &      \\ 
     &           &   10 &  4.45e-08 &  9.47e-16 &      \\ 
3631 &  3.63e+04 &   10 &           &           & iters  \\ 
 \hdashline 
     &           &    1 &  3.00e-08 &  6.21e-10 &      \\ 
     &           &    2 &  4.77e-08 &  8.57e-16 &      \\ 
     &           &    3 &  3.00e-08 &  1.36e-15 &      \\ 
     &           &    4 &  3.00e-08 &  8.58e-16 &      \\ 
     &           &    5 &  4.77e-08 &  8.56e-16 &      \\ 
     &           &    6 &  3.00e-08 &  1.36e-15 &      \\ 
     &           &    7 &  4.77e-08 &  8.57e-16 &      \\ 
     &           &    8 &  3.01e-08 &  1.36e-15 &      \\ 
     &           &    9 &  3.00e-08 &  8.58e-16 &      \\ 
     &           &   10 &  4.77e-08 &  8.57e-16 &      \\ 
3632 &  3.63e+04 &   10 &           &           & iters  \\ 
 \hdashline 
     &           &    1 &  1.41e-08 &  1.13e-09 &      \\ 
     &           &    2 &  1.40e-08 &  4.02e-16 &      \\ 
     &           &    3 &  1.40e-08 &  4.01e-16 &      \\ 
     &           &    4 &  6.37e-08 &  4.00e-16 &      \\ 
     &           &    5 &  1.41e-08 &  1.82e-15 &      \\ 
     &           &    6 &  1.41e-08 &  4.02e-16 &      \\ 
     &           &    7 &  1.41e-08 &  4.02e-16 &      \\ 
     &           &    8 &  1.40e-08 &  4.01e-16 &      \\ 
     &           &    9 &  6.37e-08 &  4.01e-16 &      \\ 
     &           &   10 &  1.41e-08 &  1.82e-15 &      \\ 
3633 &  3.63e+04 &   10 &           &           & iters  \\ 
 \hdashline 
     &           &    1 &  5.37e-08 &  1.29e-09 &      \\ 
     &           &    2 &  2.41e-08 &  1.53e-15 &      \\ 
     &           &    3 &  2.40e-08 &  6.86e-16 &      \\ 
     &           &    4 &  5.37e-08 &  6.85e-16 &      \\ 
     &           &    5 &  2.41e-08 &  1.53e-15 &      \\ 
     &           &    6 &  2.40e-08 &  6.87e-16 &      \\ 
     &           &    7 &  2.40e-08 &  6.86e-16 &      \\ 
     &           &    8 &  5.37e-08 &  6.85e-16 &      \\ 
     &           &    9 &  2.40e-08 &  1.53e-15 &      \\ 
     &           &   10 &  2.40e-08 &  6.86e-16 &      \\ 
3634 &  3.63e+04 &   10 &           &           & iters  \\ 
 \hdashline 
     &           &    1 &  6.91e-08 &  7.67e-10 &      \\ 
     &           &    2 &  8.65e-09 &  1.97e-15 &      \\ 
     &           &    3 &  8.64e-09 &  2.47e-16 &      \\ 
     &           &    4 &  8.63e-09 &  2.47e-16 &      \\ 
     &           &    5 &  8.61e-09 &  2.46e-16 &      \\ 
     &           &    6 &  8.60e-09 &  2.46e-16 &      \\ 
     &           &    7 &  8.59e-09 &  2.45e-16 &      \\ 
     &           &    8 &  8.58e-09 &  2.45e-16 &      \\ 
     &           &    9 &  6.91e-08 &  2.45e-16 &      \\ 
     &           &   10 &  8.64e-08 &  1.97e-15 &      \\ 
3635 &  3.63e+04 &   10 &           &           & iters  \\ 
 \hdashline 
     &           &    1 &  3.63e-09 &  1.94e-09 &      \\ 
     &           &    2 &  3.62e-09 &  1.03e-16 &      \\ 
     &           &    3 &  3.61e-09 &  1.03e-16 &      \\ 
     &           &    4 &  3.61e-09 &  1.03e-16 &      \\ 
     &           &    5 &  3.60e-09 &  1.03e-16 &      \\ 
     &           &    6 &  3.60e-09 &  1.03e-16 &      \\ 
     &           &    7 &  3.59e-09 &  1.03e-16 &      \\ 
     &           &    8 &  3.59e-09 &  1.03e-16 &      \\ 
     &           &    9 &  3.58e-09 &  1.02e-16 &      \\ 
     &           &   10 &  3.58e-09 &  1.02e-16 &      \\ 
3636 &  3.64e+04 &   10 &           &           & iters  \\ 
 \hdashline 
     &           &    1 &  5.83e-09 &  4.73e-10 &      \\ 
     &           &    2 &  7.19e-08 &  1.66e-16 &      \\ 
     &           &    3 &  8.36e-08 &  2.05e-15 &      \\ 
     &           &    4 &  7.19e-08 &  2.39e-15 &      \\ 
     &           &    5 &  5.91e-09 &  2.05e-15 &      \\ 
     &           &    6 &  5.90e-09 &  1.69e-16 &      \\ 
     &           &    7 &  5.89e-09 &  1.68e-16 &      \\ 
     &           &    8 &  5.88e-09 &  1.68e-16 &      \\ 
     &           &    9 &  5.87e-09 &  1.68e-16 &      \\ 
     &           &   10 &  5.87e-09 &  1.68e-16 &      \\ 
3637 &  3.64e+04 &   10 &           &           & iters  \\ 
 \hdashline 
     &           &    1 &  1.69e-09 &  3.20e-10 &      \\ 
     &           &    2 &  1.69e-09 &  4.83e-17 &      \\ 
     &           &    3 &  1.69e-09 &  4.83e-17 &      \\ 
     &           &    4 &  1.69e-09 &  4.82e-17 &      \\ 
     &           &    5 &  1.68e-09 &  4.81e-17 &      \\ 
     &           &    6 &  1.68e-09 &  4.81e-17 &      \\ 
     &           &    7 &  1.68e-09 &  4.80e-17 &      \\ 
     &           &    8 &  1.68e-09 &  4.79e-17 &      \\ 
     &           &    9 &  1.67e-09 &  4.79e-17 &      \\ 
     &           &   10 &  1.67e-09 &  4.78e-17 &      \\ 
3638 &  3.64e+04 &   10 &           &           & iters  \\ 
 \hdashline 
     &           &    1 &  2.13e-08 &  1.41e-10 &      \\ 
     &           &    2 &  2.13e-08 &  6.09e-16 &      \\ 
     &           &    3 &  2.13e-08 &  6.08e-16 &      \\ 
     &           &    4 &  5.65e-08 &  6.07e-16 &      \\ 
     &           &    5 &  2.13e-08 &  1.61e-15 &      \\ 
     &           &    6 &  2.13e-08 &  6.08e-16 &      \\ 
     &           &    7 &  5.64e-08 &  6.08e-16 &      \\ 
     &           &    8 &  2.13e-08 &  1.61e-15 &      \\ 
     &           &    9 &  2.13e-08 &  6.09e-16 &      \\ 
     &           &   10 &  2.13e-08 &  6.08e-16 &      \\ 
3639 &  3.64e+04 &   10 &           &           & iters  \\ 
 \hdashline 
     &           &    1 &  2.96e-08 &  2.86e-10 &      \\ 
     &           &    2 &  2.95e-08 &  8.44e-16 &      \\ 
     &           &    3 &  4.82e-08 &  8.42e-16 &      \\ 
     &           &    4 &  2.95e-08 &  1.38e-15 &      \\ 
     &           &    5 &  4.82e-08 &  8.43e-16 &      \\ 
     &           &    6 &  2.96e-08 &  1.38e-15 &      \\ 
     &           &    7 &  2.95e-08 &  8.44e-16 &      \\ 
     &           &    8 &  4.82e-08 &  8.43e-16 &      \\ 
     &           &    9 &  2.96e-08 &  1.38e-15 &      \\ 
     &           &   10 &  2.95e-08 &  8.43e-16 &      \\ 
3640 &  3.64e+04 &   10 &           &           & iters  \\ 
 \hdashline 
     &           &    1 &  9.52e-09 &  1.58e-10 &      \\ 
     &           &    2 &  9.50e-09 &  2.72e-16 &      \\ 
     &           &    3 &  9.49e-09 &  2.71e-16 &      \\ 
     &           &    4 &  9.48e-09 &  2.71e-16 &      \\ 
     &           &    5 &  9.46e-09 &  2.70e-16 &      \\ 
     &           &    6 &  9.45e-09 &  2.70e-16 &      \\ 
     &           &    7 &  6.83e-08 &  2.70e-16 &      \\ 
     &           &    8 &  9.53e-09 &  1.95e-15 &      \\ 
     &           &    9 &  9.52e-09 &  2.72e-16 &      \\ 
     &           &   10 &  9.51e-09 &  2.72e-16 &      \\ 
3641 &  3.64e+04 &   10 &           &           & iters  \\ 
 \hdashline 
     &           &    1 &  1.04e-08 &  1.02e-09 &      \\ 
     &           &    2 &  1.04e-08 &  2.96e-16 &      \\ 
     &           &    3 &  1.03e-08 &  2.96e-16 &      \\ 
     &           &    4 &  1.03e-08 &  2.95e-16 &      \\ 
     &           &    5 &  1.03e-08 &  2.95e-16 &      \\ 
     &           &    6 &  6.74e-08 &  2.94e-16 &      \\ 
     &           &    7 &  1.04e-08 &  1.92e-15 &      \\ 
     &           &    8 &  1.04e-08 &  2.97e-16 &      \\ 
     &           &    9 &  1.04e-08 &  2.96e-16 &      \\ 
     &           &   10 &  1.04e-08 &  2.96e-16 &      \\ 
3642 &  3.64e+04 &   10 &           &           & iters  \\ 
 \hdashline 
     &           &    1 &  1.39e-08 &  1.10e-09 &      \\ 
     &           &    2 &  1.38e-08 &  3.95e-16 &      \\ 
     &           &    3 &  1.38e-08 &  3.95e-16 &      \\ 
     &           &    4 &  2.50e-08 &  3.94e-16 &      \\ 
     &           &    5 &  1.38e-08 &  7.15e-16 &      \\ 
     &           &    6 &  2.50e-08 &  3.95e-16 &      \\ 
     &           &    7 &  1.38e-08 &  7.14e-16 &      \\ 
     &           &    8 &  1.38e-08 &  3.95e-16 &      \\ 
     &           &    9 &  2.50e-08 &  3.95e-16 &      \\ 
     &           &   10 &  1.38e-08 &  7.15e-16 &      \\ 
3643 &  3.64e+04 &   10 &           &           & iters  \\ 
 \hdashline 
     &           &    1 &  2.84e-08 &  3.17e-10 &      \\ 
     &           &    2 &  4.93e-08 &  8.11e-16 &      \\ 
     &           &    3 &  2.84e-08 &  1.41e-15 &      \\ 
     &           &    4 &  2.84e-08 &  8.11e-16 &      \\ 
     &           &    5 &  4.93e-08 &  8.10e-16 &      \\ 
     &           &    6 &  2.84e-08 &  1.41e-15 &      \\ 
     &           &    7 &  4.93e-08 &  8.11e-16 &      \\ 
     &           &    8 &  2.85e-08 &  1.41e-15 &      \\ 
     &           &    9 &  2.84e-08 &  8.12e-16 &      \\ 
     &           &   10 &  4.93e-08 &  8.11e-16 &      \\ 
3644 &  3.64e+04 &   10 &           &           & iters  \\ 
 \hdashline 
     &           &    1 &  1.92e-09 &  5.87e-10 &      \\ 
     &           &    2 &  1.92e-09 &  5.48e-17 &      \\ 
     &           &    3 &  1.92e-09 &  5.48e-17 &      \\ 
     &           &    4 &  1.91e-09 &  5.47e-17 &      \\ 
     &           &    5 &  1.91e-09 &  5.46e-17 &      \\ 
     &           &    6 &  1.91e-09 &  5.45e-17 &      \\ 
     &           &    7 &  1.90e-09 &  5.44e-17 &      \\ 
     &           &    8 &  1.90e-09 &  5.44e-17 &      \\ 
     &           &    9 &  1.90e-09 &  5.43e-17 &      \\ 
     &           &   10 &  1.90e-09 &  5.42e-17 &      \\ 
3645 &  3.64e+04 &   10 &           &           & iters  \\ 
 \hdashline 
     &           &    1 &  1.18e-08 &  5.21e-10 &      \\ 
     &           &    2 &  1.18e-08 &  3.38e-16 &      \\ 
     &           &    3 &  6.59e-08 &  3.37e-16 &      \\ 
     &           &    4 &  1.19e-08 &  1.88e-15 &      \\ 
     &           &    5 &  1.19e-08 &  3.39e-16 &      \\ 
     &           &    6 &  1.19e-08 &  3.39e-16 &      \\ 
     &           &    7 &  1.18e-08 &  3.39e-16 &      \\ 
     &           &    8 &  1.18e-08 &  3.38e-16 &      \\ 
     &           &    9 &  6.59e-08 &  3.38e-16 &      \\ 
     &           &   10 &  8.96e-08 &  1.88e-15 &      \\ 
3646 &  3.64e+04 &   10 &           &           & iters  \\ 
 \hdashline 
     &           &    1 &  2.73e-08 &  1.85e-10 &      \\ 
     &           &    2 &  2.73e-08 &  7.79e-16 &      \\ 
     &           &    3 &  5.05e-08 &  7.78e-16 &      \\ 
     &           &    4 &  2.73e-08 &  1.44e-15 &      \\ 
     &           &    5 &  2.72e-08 &  7.79e-16 &      \\ 
     &           &    6 &  5.05e-08 &  7.78e-16 &      \\ 
     &           &    7 &  2.73e-08 &  1.44e-15 &      \\ 
     &           &    8 &  2.72e-08 &  7.79e-16 &      \\ 
     &           &    9 &  5.05e-08 &  7.78e-16 &      \\ 
     &           &   10 &  2.73e-08 &  1.44e-15 &      \\ 
3647 &  3.65e+04 &   10 &           &           & iters  \\ 
 \hdashline 
     &           &    1 &  1.65e-08 &  2.64e-10 &      \\ 
     &           &    2 &  6.13e-08 &  4.70e-16 &      \\ 
     &           &    3 &  1.65e-08 &  1.75e-15 &      \\ 
     &           &    4 &  1.65e-08 &  4.72e-16 &      \\ 
     &           &    5 &  1.65e-08 &  4.71e-16 &      \\ 
     &           &    6 &  1.65e-08 &  4.70e-16 &      \\ 
     &           &    7 &  6.13e-08 &  4.70e-16 &      \\ 
     &           &    8 &  1.65e-08 &  1.75e-15 &      \\ 
     &           &    9 &  1.65e-08 &  4.71e-16 &      \\ 
     &           &   10 &  1.65e-08 &  4.71e-16 &      \\ 
3648 &  3.65e+04 &   10 &           &           & iters  \\ 
 \hdashline 
     &           &    1 &  5.71e-08 &  5.67e-11 &      \\ 
     &           &    2 &  2.07e-08 &  1.63e-15 &      \\ 
     &           &    3 &  2.07e-08 &  5.90e-16 &      \\ 
     &           &    4 &  2.06e-08 &  5.89e-16 &      \\ 
     &           &    5 &  5.71e-08 &  5.89e-16 &      \\ 
     &           &    6 &  2.07e-08 &  1.63e-15 &      \\ 
     &           &    7 &  2.06e-08 &  5.90e-16 &      \\ 
     &           &    8 &  2.06e-08 &  5.89e-16 &      \\ 
     &           &    9 &  5.71e-08 &  5.88e-16 &      \\ 
     &           &   10 &  2.07e-08 &  1.63e-15 &      \\ 
3649 &  3.65e+04 &   10 &           &           & iters  \\ 
 \hdashline 
     &           &    1 &  2.59e-08 &  1.70e-10 &      \\ 
     &           &    2 &  5.19e-08 &  7.38e-16 &      \\ 
     &           &    3 &  2.59e-08 &  1.48e-15 &      \\ 
     &           &    4 &  2.59e-08 &  7.39e-16 &      \\ 
     &           &    5 &  5.19e-08 &  7.38e-16 &      \\ 
     &           &    6 &  2.59e-08 &  1.48e-15 &      \\ 
     &           &    7 &  2.59e-08 &  7.39e-16 &      \\ 
     &           &    8 &  5.19e-08 &  7.38e-16 &      \\ 
     &           &    9 &  2.59e-08 &  1.48e-15 &      \\ 
     &           &   10 &  2.59e-08 &  7.39e-16 &      \\ 
3650 &  3.65e+04 &   10 &           &           & iters  \\ 
 \hdashline 
     &           &    1 &  3.57e-08 &  2.41e-10 &      \\ 
     &           &    2 &  3.16e-09 &  1.02e-15 &      \\ 
     &           &    3 &  3.15e-09 &  9.01e-17 &      \\ 
     &           &    4 &  3.15e-09 &  9.00e-17 &      \\ 
     &           &    5 &  3.14e-09 &  8.99e-17 &      \\ 
     &           &    6 &  3.57e-08 &  8.97e-17 &      \\ 
     &           &    7 &  3.19e-09 &  1.02e-15 &      \\ 
     &           &    8 &  3.19e-09 &  9.11e-17 &      \\ 
     &           &    9 &  3.18e-09 &  9.09e-17 &      \\ 
     &           &   10 &  3.18e-09 &  9.08e-17 &      \\ 
3651 &  3.65e+04 &   10 &           &           & iters  \\ 
 \hdashline 
     &           &    1 &  6.50e-09 &  1.55e-10 &      \\ 
     &           &    2 &  6.49e-09 &  1.86e-16 &      \\ 
     &           &    3 &  6.48e-09 &  1.85e-16 &      \\ 
     &           &    4 &  7.12e-08 &  1.85e-16 &      \\ 
     &           &    5 &  8.43e-08 &  2.03e-15 &      \\ 
     &           &    6 &  7.12e-08 &  2.40e-15 &      \\ 
     &           &    7 &  6.56e-09 &  2.03e-15 &      \\ 
     &           &    8 &  6.55e-09 &  1.87e-16 &      \\ 
     &           &    9 &  6.54e-09 &  1.87e-16 &      \\ 
     &           &   10 &  6.53e-09 &  1.87e-16 &      \\ 
3652 &  3.65e+04 &   10 &           &           & iters  \\ 
 \hdashline 
     &           &    1 &  1.72e-08 &  1.60e-10 &      \\ 
     &           &    2 &  1.71e-08 &  4.90e-16 &      \\ 
     &           &    3 &  6.06e-08 &  4.89e-16 &      \\ 
     &           &    4 &  1.72e-08 &  1.73e-15 &      \\ 
     &           &    5 &  1.72e-08 &  4.91e-16 &      \\ 
     &           &    6 &  1.72e-08 &  4.90e-16 &      \\ 
     &           &    7 &  1.71e-08 &  4.89e-16 &      \\ 
     &           &    8 &  6.06e-08 &  4.89e-16 &      \\ 
     &           &    9 &  1.72e-08 &  1.73e-15 &      \\ 
     &           &   10 &  1.72e-08 &  4.91e-16 &      \\ 
3653 &  3.65e+04 &   10 &           &           & iters  \\ 
 \hdashline 
     &           &    1 &  1.57e-08 &  2.37e-10 &      \\ 
     &           &    2 &  1.57e-08 &  4.48e-16 &      \\ 
     &           &    3 &  1.56e-08 &  4.47e-16 &      \\ 
     &           &    4 &  6.21e-08 &  4.47e-16 &      \\ 
     &           &    5 &  1.57e-08 &  1.77e-15 &      \\ 
     &           &    6 &  1.57e-08 &  4.48e-16 &      \\ 
     &           &    7 &  1.57e-08 &  4.48e-16 &      \\ 
     &           &    8 &  1.56e-08 &  4.47e-16 &      \\ 
     &           &    9 &  6.21e-08 &  4.47e-16 &      \\ 
     &           &   10 &  1.57e-08 &  1.77e-15 &      \\ 
3654 &  3.65e+04 &   10 &           &           & iters  \\ 
 \hdashline 
     &           &    1 &  1.62e-08 &  3.29e-10 &      \\ 
     &           &    2 &  1.62e-08 &  4.62e-16 &      \\ 
     &           &    3 &  1.61e-08 &  4.61e-16 &      \\ 
     &           &    4 &  6.16e-08 &  4.61e-16 &      \\ 
     &           &    5 &  1.62e-08 &  1.76e-15 &      \\ 
     &           &    6 &  1.62e-08 &  4.62e-16 &      \\ 
     &           &    7 &  1.62e-08 &  4.62e-16 &      \\ 
     &           &    8 &  1.61e-08 &  4.61e-16 &      \\ 
     &           &    9 &  6.16e-08 &  4.60e-16 &      \\ 
     &           &   10 &  1.62e-08 &  1.76e-15 &      \\ 
3655 &  3.65e+04 &   10 &           &           & iters  \\ 
 \hdashline 
     &           &    1 &  1.55e-08 &  2.51e-10 &      \\ 
     &           &    2 &  1.55e-08 &  4.43e-16 &      \\ 
     &           &    3 &  6.22e-08 &  4.42e-16 &      \\ 
     &           &    4 &  1.56e-08 &  1.78e-15 &      \\ 
     &           &    5 &  1.55e-08 &  4.44e-16 &      \\ 
     &           &    6 &  1.55e-08 &  4.44e-16 &      \\ 
     &           &    7 &  1.55e-08 &  4.43e-16 &      \\ 
     &           &    8 &  6.22e-08 &  4.42e-16 &      \\ 
     &           &    9 &  1.56e-08 &  1.78e-15 &      \\ 
     &           &   10 &  1.55e-08 &  4.44e-16 &      \\ 
3656 &  3.66e+04 &   10 &           &           & iters  \\ 
 \hdashline 
     &           &    1 &  1.59e-08 &  1.20e-09 &      \\ 
     &           &    2 &  6.19e-08 &  4.52e-16 &      \\ 
     &           &    3 &  1.59e-08 &  1.77e-15 &      \\ 
     &           &    4 &  1.59e-08 &  4.54e-16 &      \\ 
     &           &    5 &  1.59e-08 &  4.54e-16 &      \\ 
     &           &    6 &  1.59e-08 &  4.53e-16 &      \\ 
     &           &    7 &  6.19e-08 &  4.52e-16 &      \\ 
     &           &    8 &  1.59e-08 &  1.77e-15 &      \\ 
     &           &    9 &  1.59e-08 &  4.54e-16 &      \\ 
     &           &   10 &  1.59e-08 &  4.54e-16 &      \\ 
3657 &  3.66e+04 &   10 &           &           & iters  \\ 
 \hdashline 
     &           &    1 &  1.28e-09 &  1.18e-09 &      \\ 
     &           &    2 &  1.28e-09 &  3.67e-17 &      \\ 
     &           &    3 &  1.28e-09 &  3.66e-17 &      \\ 
     &           &    4 &  1.28e-09 &  3.66e-17 &      \\ 
     &           &    5 &  1.28e-09 &  3.65e-17 &      \\ 
     &           &    6 &  1.28e-09 &  3.64e-17 &      \\ 
     &           &    7 &  1.27e-09 &  3.64e-17 &      \\ 
     &           &    8 &  1.27e-09 &  3.63e-17 &      \\ 
     &           &    9 &  1.27e-09 &  3.63e-17 &      \\ 
     &           &   10 &  1.27e-09 &  3.62e-17 &      \\ 
3658 &  3.66e+04 &   10 &           &           & iters  \\ 
 \hdashline 
     &           &    1 &  1.17e-09 &  9.74e-10 &      \\ 
     &           &    2 &  1.17e-09 &  3.34e-17 &      \\ 
     &           &    3 &  1.17e-09 &  3.34e-17 &      \\ 
     &           &    4 &  1.17e-09 &  3.33e-17 &      \\ 
     &           &    5 &  1.16e-09 &  3.33e-17 &      \\ 
     &           &    6 &  1.16e-09 &  3.32e-17 &      \\ 
     &           &    7 &  1.16e-09 &  3.32e-17 &      \\ 
     &           &    8 &  1.16e-09 &  3.31e-17 &      \\ 
     &           &    9 &  1.16e-09 &  3.31e-17 &      \\ 
     &           &   10 &  1.16e-09 &  3.30e-17 &      \\ 
3659 &  3.66e+04 &   10 &           &           & iters  \\ 
 \hdashline 
     &           &    1 &  1.79e-08 &  9.24e-10 &      \\ 
     &           &    2 &  1.79e-08 &  5.12e-16 &      \\ 
     &           &    3 &  1.79e-08 &  5.11e-16 &      \\ 
     &           &    4 &  1.79e-08 &  5.11e-16 &      \\ 
     &           &    5 &  1.78e-08 &  5.10e-16 &      \\ 
     &           &    6 &  5.99e-08 &  5.09e-16 &      \\ 
     &           &    7 &  1.79e-08 &  1.71e-15 &      \\ 
     &           &    8 &  1.79e-08 &  5.11e-16 &      \\ 
     &           &    9 &  1.78e-08 &  5.10e-16 &      \\ 
     &           &   10 &  1.78e-08 &  5.09e-16 &      \\ 
3660 &  3.66e+04 &   10 &           &           & iters  \\ 
 \hdashline 
     &           &    1 &  8.84e-08 &  3.54e-10 &      \\ 
     &           &    2 &  6.71e-08 &  2.52e-15 &      \\ 
     &           &    3 &  1.06e-08 &  1.92e-15 &      \\ 
     &           &    4 &  1.06e-08 &  3.04e-16 &      \\ 
     &           &    5 &  1.06e-08 &  3.03e-16 &      \\ 
     &           &    6 &  1.06e-08 &  3.03e-16 &      \\ 
     &           &    7 &  1.06e-08 &  3.03e-16 &      \\ 
     &           &    8 &  6.71e-08 &  3.02e-16 &      \\ 
     &           &    9 &  1.07e-08 &  1.92e-15 &      \\ 
     &           &   10 &  1.07e-08 &  3.04e-16 &      \\ 
3661 &  3.66e+04 &   10 &           &           & iters  \\ 
 \hdashline 
     &           &    1 &  4.28e-08 &  1.14e-10 &      \\ 
     &           &    2 &  3.50e-08 &  1.22e-15 &      \\ 
     &           &    3 &  4.27e-08 &  9.99e-16 &      \\ 
     &           &    4 &  3.50e-08 &  1.22e-15 &      \\ 
     &           &    5 &  4.27e-08 &  9.99e-16 &      \\ 
     &           &    6 &  3.50e-08 &  1.22e-15 &      \\ 
     &           &    7 &  4.27e-08 &  9.99e-16 &      \\ 
     &           &    8 &  3.50e-08 &  1.22e-15 &      \\ 
     &           &    9 &  4.27e-08 &  1.00e-15 &      \\ 
     &           &   10 &  3.50e-08 &  1.22e-15 &      \\ 
3662 &  3.66e+04 &   10 &           &           & iters  \\ 
 \hdashline 
     &           &    1 &  1.90e-08 &  2.72e-10 &      \\ 
     &           &    2 &  5.87e-08 &  5.42e-16 &      \\ 
     &           &    3 &  1.90e-08 &  1.68e-15 &      \\ 
     &           &    4 &  1.90e-08 &  5.43e-16 &      \\ 
     &           &    5 &  1.90e-08 &  5.43e-16 &      \\ 
     &           &    6 &  5.87e-08 &  5.42e-16 &      \\ 
     &           &    7 &  1.90e-08 &  1.68e-15 &      \\ 
     &           &    8 &  1.90e-08 &  5.43e-16 &      \\ 
     &           &    9 &  1.90e-08 &  5.43e-16 &      \\ 
     &           &   10 &  5.87e-08 &  5.42e-16 &      \\ 
3663 &  3.66e+04 &   10 &           &           & iters  \\ 
 \hdashline 
     &           &    1 &  3.94e-08 &  6.08e-10 &      \\ 
     &           &    2 &  3.84e-08 &  1.12e-15 &      \\ 
     &           &    3 &  3.94e-08 &  1.10e-15 &      \\ 
     &           &    4 &  3.84e-08 &  1.12e-15 &      \\ 
     &           &    5 &  3.94e-08 &  1.10e-15 &      \\ 
     &           &    6 &  3.84e-08 &  1.12e-15 &      \\ 
     &           &    7 &  3.94e-08 &  1.10e-15 &      \\ 
     &           &    8 &  3.84e-08 &  1.12e-15 &      \\ 
     &           &    9 &  3.94e-08 &  1.10e-15 &      \\ 
     &           &   10 &  3.84e-08 &  1.12e-15 &      \\ 
3664 &  3.66e+04 &   10 &           &           & iters  \\ 
 \hdashline 
     &           &    1 &  4.76e-08 &  1.20e-09 &      \\ 
     &           &    2 &  4.75e-08 &  1.36e-15 &      \\ 
     &           &    3 &  3.02e-08 &  1.36e-15 &      \\ 
     &           &    4 &  3.02e-08 &  8.63e-16 &      \\ 
     &           &    5 &  4.75e-08 &  8.62e-16 &      \\ 
     &           &    6 &  3.02e-08 &  1.36e-15 &      \\ 
     &           &    7 &  3.02e-08 &  8.62e-16 &      \\ 
     &           &    8 &  4.76e-08 &  8.61e-16 &      \\ 
     &           &    9 &  3.02e-08 &  1.36e-15 &      \\ 
     &           &   10 &  4.75e-08 &  8.62e-16 &      \\ 
3665 &  3.66e+04 &   10 &           &           & iters  \\ 
 \hdashline 
     &           &    1 &  6.09e-08 &  1.43e-09 &      \\ 
     &           &    2 &  2.19e-08 &  1.74e-15 &      \\ 
     &           &    3 &  5.58e-08 &  6.26e-16 &      \\ 
     &           &    4 &  2.20e-08 &  1.59e-15 &      \\ 
     &           &    5 &  2.20e-08 &  6.28e-16 &      \\ 
     &           &    6 &  5.58e-08 &  6.27e-16 &      \\ 
     &           &    7 &  2.20e-08 &  1.59e-15 &      \\ 
     &           &    8 &  2.20e-08 &  6.28e-16 &      \\ 
     &           &    9 &  2.20e-08 &  6.27e-16 &      \\ 
     &           &   10 &  5.58e-08 &  6.27e-16 &      \\ 
3666 &  3.66e+04 &   10 &           &           & iters  \\ 
 \hdashline 
     &           &    1 &  5.65e-08 &  1.13e-09 &      \\ 
     &           &    2 &  2.12e-08 &  1.61e-15 &      \\ 
     &           &    3 &  2.12e-08 &  6.06e-16 &      \\ 
     &           &    4 &  2.12e-08 &  6.06e-16 &      \\ 
     &           &    5 &  5.65e-08 &  6.05e-16 &      \\ 
     &           &    6 &  2.12e-08 &  1.61e-15 &      \\ 
     &           &    7 &  2.12e-08 &  6.06e-16 &      \\ 
     &           &    8 &  5.65e-08 &  6.05e-16 &      \\ 
     &           &    9 &  2.13e-08 &  1.61e-15 &      \\ 
     &           &   10 &  2.12e-08 &  6.07e-16 &      \\ 
3667 &  3.67e+04 &   10 &           &           & iters  \\ 
 \hdashline 
     &           &    1 &  2.11e-08 &  8.95e-10 &      \\ 
     &           &    2 &  2.11e-08 &  6.02e-16 &      \\ 
     &           &    3 &  5.67e-08 &  6.01e-16 &      \\ 
     &           &    4 &  2.11e-08 &  1.62e-15 &      \\ 
     &           &    5 &  2.11e-08 &  6.03e-16 &      \\ 
     &           &    6 &  5.66e-08 &  6.02e-16 &      \\ 
     &           &    7 &  2.11e-08 &  1.62e-15 &      \\ 
     &           &    8 &  2.11e-08 &  6.03e-16 &      \\ 
     &           &    9 &  2.11e-08 &  6.02e-16 &      \\ 
     &           &   10 &  5.66e-08 &  6.01e-16 &      \\ 
3668 &  3.67e+04 &   10 &           &           & iters  \\ 
 \hdashline 
     &           &    1 &  2.90e-08 &  7.36e-10 &      \\ 
     &           &    2 &  2.89e-08 &  8.27e-16 &      \\ 
     &           &    3 &  4.88e-08 &  8.26e-16 &      \\ 
     &           &    4 &  2.90e-08 &  1.39e-15 &      \\ 
     &           &    5 &  2.89e-08 &  8.26e-16 &      \\ 
     &           &    6 &  4.88e-08 &  8.25e-16 &      \\ 
     &           &    7 &  2.89e-08 &  1.39e-15 &      \\ 
     &           &    8 &  2.89e-08 &  8.26e-16 &      \\ 
     &           &    9 &  4.88e-08 &  8.25e-16 &      \\ 
     &           &   10 &  2.89e-08 &  1.39e-15 &      \\ 
3669 &  3.67e+04 &   10 &           &           & iters  \\ 
 \hdashline 
     &           &    1 &  3.79e-08 &  1.17e-10 &      \\ 
     &           &    2 &  3.99e-08 &  1.08e-15 &      \\ 
     &           &    3 &  3.79e-08 &  1.14e-15 &      \\ 
     &           &    4 &  3.99e-08 &  1.08e-15 &      \\ 
     &           &    5 &  3.79e-08 &  1.14e-15 &      \\ 
     &           &    6 &  3.78e-08 &  1.08e-15 &      \\ 
     &           &    7 &  3.99e-08 &  1.08e-15 &      \\ 
     &           &    8 &  3.78e-08 &  1.14e-15 &      \\ 
     &           &    9 &  3.99e-08 &  1.08e-15 &      \\ 
     &           &   10 &  3.78e-08 &  1.14e-15 &      \\ 
3670 &  3.67e+04 &   10 &           &           & iters  \\ 
 \hdashline 
     &           &    1 &  6.12e-08 &  9.61e-10 &      \\ 
     &           &    2 &  1.66e-08 &  1.75e-15 &      \\ 
     &           &    3 &  1.65e-08 &  4.73e-16 &      \\ 
     &           &    4 &  1.65e-08 &  4.72e-16 &      \\ 
     &           &    5 &  6.12e-08 &  4.72e-16 &      \\ 
     &           &    6 &  1.66e-08 &  1.75e-15 &      \\ 
     &           &    7 &  1.66e-08 &  4.73e-16 &      \\ 
     &           &    8 &  1.65e-08 &  4.73e-16 &      \\ 
     &           &    9 &  1.65e-08 &  4.72e-16 &      \\ 
     &           &   10 &  6.12e-08 &  4.71e-16 &      \\ 
3671 &  3.67e+04 &   10 &           &           & iters  \\ 
 \hdashline 
     &           &    1 &  3.98e-08 &  7.05e-10 &      \\ 
     &           &    2 &  3.79e-08 &  1.14e-15 &      \\ 
     &           &    3 &  3.98e-08 &  1.08e-15 &      \\ 
     &           &    4 &  3.79e-08 &  1.14e-15 &      \\ 
     &           &    5 &  3.98e-08 &  1.08e-15 &      \\ 
     &           &    6 &  3.79e-08 &  1.14e-15 &      \\ 
     &           &    7 &  3.98e-08 &  1.08e-15 &      \\ 
     &           &    8 &  3.79e-08 &  1.14e-15 &      \\ 
     &           &    9 &  3.98e-08 &  1.08e-15 &      \\ 
     &           &   10 &  3.79e-08 &  1.14e-15 &      \\ 
3672 &  3.67e+04 &   10 &           &           & iters  \\ 
 \hdashline 
     &           &    1 &  1.19e-08 &  4.33e-10 &      \\ 
     &           &    2 &  1.18e-08 &  3.39e-16 &      \\ 
     &           &    3 &  1.18e-08 &  3.38e-16 &      \\ 
     &           &    4 &  1.18e-08 &  3.38e-16 &      \\ 
     &           &    5 &  1.18e-08 &  3.37e-16 &      \\ 
     &           &    6 &  6.59e-08 &  3.37e-16 &      \\ 
     &           &    7 &  1.19e-08 &  1.88e-15 &      \\ 
     &           &    8 &  1.19e-08 &  3.39e-16 &      \\ 
     &           &    9 &  1.18e-08 &  3.38e-16 &      \\ 
     &           &   10 &  1.18e-08 &  3.38e-16 &      \\ 
3673 &  3.67e+04 &   10 &           &           & iters  \\ 
 \hdashline 
     &           &    1 &  1.64e-08 &  3.00e-10 &      \\ 
     &           &    2 &  1.64e-08 &  4.68e-16 &      \\ 
     &           &    3 &  1.63e-08 &  4.67e-16 &      \\ 
     &           &    4 &  6.14e-08 &  4.66e-16 &      \\ 
     &           &    5 &  1.64e-08 &  1.75e-15 &      \\ 
     &           &    6 &  1.64e-08 &  4.68e-16 &      \\ 
     &           &    7 &  1.64e-08 &  4.68e-16 &      \\ 
     &           &    8 &  1.63e-08 &  4.67e-16 &      \\ 
     &           &    9 &  6.14e-08 &  4.66e-16 &      \\ 
     &           &   10 &  1.64e-08 &  1.75e-15 &      \\ 
3674 &  3.67e+04 &   10 &           &           & iters  \\ 
 \hdashline 
     &           &    1 &  2.34e-08 &  2.87e-11 &      \\ 
     &           &    2 &  5.43e-08 &  6.69e-16 &      \\ 
     &           &    3 &  2.35e-08 &  1.55e-15 &      \\ 
     &           &    4 &  2.34e-08 &  6.70e-16 &      \\ 
     &           &    5 &  2.34e-08 &  6.69e-16 &      \\ 
     &           &    6 &  5.43e-08 &  6.68e-16 &      \\ 
     &           &    7 &  2.35e-08 &  1.55e-15 &      \\ 
     &           &    8 &  2.34e-08 &  6.70e-16 &      \\ 
     &           &    9 &  5.43e-08 &  6.69e-16 &      \\ 
     &           &   10 &  2.35e-08 &  1.55e-15 &      \\ 
3675 &  3.67e+04 &   10 &           &           & iters  \\ 
 \hdashline 
     &           &    1 &  2.38e-08 &  1.75e-10 &      \\ 
     &           &    2 &  1.51e-08 &  6.78e-16 &      \\ 
     &           &    3 &  1.51e-08 &  4.31e-16 &      \\ 
     &           &    4 &  2.38e-08 &  4.31e-16 &      \\ 
     &           &    5 &  1.51e-08 &  6.79e-16 &      \\ 
     &           &    6 &  2.38e-08 &  4.31e-16 &      \\ 
     &           &    7 &  1.51e-08 &  6.78e-16 &      \\ 
     &           &    8 &  1.51e-08 &  4.31e-16 &      \\ 
     &           &    9 &  2.38e-08 &  4.31e-16 &      \\ 
     &           &   10 &  1.51e-08 &  6.79e-16 &      \\ 
3676 &  3.68e+04 &   10 &           &           & iters  \\ 
 \hdashline 
     &           &    1 &  8.98e-09 &  4.97e-10 &      \\ 
     &           &    2 &  8.97e-09 &  2.56e-16 &      \\ 
     &           &    3 &  8.96e-09 &  2.56e-16 &      \\ 
     &           &    4 &  8.95e-09 &  2.56e-16 &      \\ 
     &           &    5 &  6.88e-08 &  2.55e-16 &      \\ 
     &           &    6 &  9.03e-09 &  1.96e-15 &      \\ 
     &           &    7 &  9.02e-09 &  2.58e-16 &      \\ 
     &           &    8 &  9.01e-09 &  2.57e-16 &      \\ 
     &           &    9 &  8.99e-09 &  2.57e-16 &      \\ 
     &           &   10 &  8.98e-09 &  2.57e-16 &      \\ 
3677 &  3.68e+04 &   10 &           &           & iters  \\ 
 \hdashline 
     &           &    1 &  6.62e-08 &  1.18e-09 &      \\ 
     &           &    2 &  8.92e-08 &  1.89e-15 &      \\ 
     &           &    3 &  6.63e-08 &  2.55e-15 &      \\ 
     &           &    4 &  1.15e-08 &  1.89e-15 &      \\ 
     &           &    5 &  1.15e-08 &  3.29e-16 &      \\ 
     &           &    6 &  1.15e-08 &  3.29e-16 &      \\ 
     &           &    7 &  1.15e-08 &  3.28e-16 &      \\ 
     &           &    8 &  6.62e-08 &  3.28e-16 &      \\ 
     &           &    9 &  1.16e-08 &  1.89e-15 &      \\ 
     &           &   10 &  1.15e-08 &  3.30e-16 &      \\ 
3678 &  3.68e+04 &   10 &           &           & iters  \\ 
 \hdashline 
     &           &    1 &  5.25e-08 &  1.81e-09 &      \\ 
     &           &    2 &  2.53e-08 &  1.50e-15 &      \\ 
     &           &    3 &  5.25e-08 &  7.21e-16 &      \\ 
     &           &    4 &  2.53e-08 &  1.50e-15 &      \\ 
     &           &    5 &  2.53e-08 &  7.22e-16 &      \\ 
     &           &    6 &  2.52e-08 &  7.21e-16 &      \\ 
     &           &    7 &  5.25e-08 &  7.20e-16 &      \\ 
     &           &    8 &  2.53e-08 &  1.50e-15 &      \\ 
     &           &    9 &  2.52e-08 &  7.21e-16 &      \\ 
     &           &   10 &  5.25e-08 &  7.20e-16 &      \\ 
3679 &  3.68e+04 &   10 &           &           & iters  \\ 
 \hdashline 
     &           &    1 &  9.47e-08 &  4.94e-10 &      \\ 
     &           &    2 &  6.08e-08 &  2.70e-15 &      \\ 
     &           &    3 &  1.69e-08 &  1.74e-15 &      \\ 
     &           &    4 &  1.69e-08 &  4.84e-16 &      \\ 
     &           &    5 &  1.69e-08 &  4.83e-16 &      \\ 
     &           &    6 &  6.08e-08 &  4.82e-16 &      \\ 
     &           &    7 &  1.70e-08 &  1.74e-15 &      \\ 
     &           &    8 &  1.69e-08 &  4.84e-16 &      \\ 
     &           &    9 &  1.69e-08 &  4.83e-16 &      \\ 
     &           &   10 &  6.08e-08 &  4.83e-16 &      \\ 
3680 &  3.68e+04 &   10 &           &           & iters  \\ 
 \hdashline 
     &           &    1 &  7.06e-08 &  1.18e-09 &      \\ 
     &           &    2 &  8.48e-08 &  2.02e-15 &      \\ 
     &           &    3 &  7.07e-08 &  2.42e-15 &      \\ 
     &           &    4 &  7.13e-09 &  2.02e-15 &      \\ 
     &           &    5 &  7.12e-09 &  2.04e-16 &      \\ 
     &           &    6 &  7.11e-09 &  2.03e-16 &      \\ 
     &           &    7 &  7.10e-09 &  2.03e-16 &      \\ 
     &           &    8 &  7.09e-09 &  2.03e-16 &      \\ 
     &           &    9 &  7.08e-09 &  2.02e-16 &      \\ 
     &           &   10 &  7.07e-09 &  2.02e-16 &      \\ 
3681 &  3.68e+04 &   10 &           &           & iters  \\ 
 \hdashline 
     &           &    1 &  1.99e-08 &  4.52e-10 &      \\ 
     &           &    2 &  1.99e-08 &  5.67e-16 &      \\ 
     &           &    3 &  1.98e-08 &  5.67e-16 &      \\ 
     &           &    4 &  5.79e-08 &  5.66e-16 &      \\ 
     &           &    5 &  1.99e-08 &  1.65e-15 &      \\ 
     &           &    6 &  1.99e-08 &  5.67e-16 &      \\ 
     &           &    7 &  1.98e-08 &  5.67e-16 &      \\ 
     &           &    8 &  5.79e-08 &  5.66e-16 &      \\ 
     &           &    9 &  1.99e-08 &  1.65e-15 &      \\ 
     &           &   10 &  1.98e-08 &  5.67e-16 &      \\ 
3682 &  3.68e+04 &   10 &           &           & iters  \\ 
 \hdashline 
     &           &    1 &  2.93e-08 &  3.09e-10 &      \\ 
     &           &    2 &  2.93e-08 &  8.37e-16 &      \\ 
     &           &    3 &  4.84e-08 &  8.36e-16 &      \\ 
     &           &    4 &  2.93e-08 &  1.38e-15 &      \\ 
     &           &    5 &  2.93e-08 &  8.37e-16 &      \\ 
     &           &    6 &  4.85e-08 &  8.35e-16 &      \\ 
     &           &    7 &  2.93e-08 &  1.38e-15 &      \\ 
     &           &    8 &  4.84e-08 &  8.36e-16 &      \\ 
     &           &    9 &  2.93e-08 &  1.38e-15 &      \\ 
     &           &   10 &  2.93e-08 &  8.37e-16 &      \\ 
3683 &  3.68e+04 &   10 &           &           & iters  \\ 
 \hdashline 
     &           &    1 &  9.75e-09 &  2.24e-10 &      \\ 
     &           &    2 &  9.73e-09 &  2.78e-16 &      \\ 
     &           &    3 &  9.72e-09 &  2.78e-16 &      \\ 
     &           &    4 &  9.71e-09 &  2.77e-16 &      \\ 
     &           &    5 &  9.69e-09 &  2.77e-16 &      \\ 
     &           &    6 &  6.80e-08 &  2.77e-16 &      \\ 
     &           &    7 &  9.78e-09 &  1.94e-15 &      \\ 
     &           &    8 &  9.76e-09 &  2.79e-16 &      \\ 
     &           &    9 &  9.75e-09 &  2.79e-16 &      \\ 
     &           &   10 &  9.73e-09 &  2.78e-16 &      \\ 
3684 &  3.68e+04 &   10 &           &           & iters  \\ 
 \hdashline 
     &           &    1 &  3.46e-08 &  3.38e-10 &      \\ 
     &           &    2 &  4.31e-08 &  9.88e-16 &      \\ 
     &           &    3 &  3.46e-08 &  1.23e-15 &      \\ 
     &           &    4 &  4.31e-08 &  9.88e-16 &      \\ 
     &           &    5 &  3.46e-08 &  1.23e-15 &      \\ 
     &           &    6 &  4.31e-08 &  9.89e-16 &      \\ 
     &           &    7 &  3.46e-08 &  1.23e-15 &      \\ 
     &           &    8 &  3.46e-08 &  9.89e-16 &      \\ 
     &           &    9 &  4.31e-08 &  9.87e-16 &      \\ 
     &           &   10 &  3.46e-08 &  1.23e-15 &      \\ 
3685 &  3.68e+04 &   10 &           &           & iters  \\ 
 \hdashline 
     &           &    1 &  7.86e-09 &  3.50e-10 &      \\ 
     &           &    2 &  6.98e-08 &  2.24e-16 &      \\ 
     &           &    3 &  7.94e-09 &  1.99e-15 &      \\ 
     &           &    4 &  7.93e-09 &  2.27e-16 &      \\ 
     &           &    5 &  7.92e-09 &  2.26e-16 &      \\ 
     &           &    6 &  7.91e-09 &  2.26e-16 &      \\ 
     &           &    7 &  7.90e-09 &  2.26e-16 &      \\ 
     &           &    8 &  7.89e-09 &  2.25e-16 &      \\ 
     &           &    9 &  7.88e-09 &  2.25e-16 &      \\ 
     &           &   10 &  7.87e-09 &  2.25e-16 &      \\ 
3686 &  3.68e+04 &   10 &           &           & iters  \\ 
 \hdashline 
     &           &    1 &  2.99e-08 &  6.15e-10 &      \\ 
     &           &    2 &  2.99e-08 &  8.54e-16 &      \\ 
     &           &    3 &  2.99e-08 &  8.53e-16 &      \\ 
     &           &    4 &  4.79e-08 &  8.52e-16 &      \\ 
     &           &    5 &  2.99e-08 &  1.37e-15 &      \\ 
     &           &    6 &  2.98e-08 &  8.53e-16 &      \\ 
     &           &    7 &  4.79e-08 &  8.52e-16 &      \\ 
     &           &    8 &  2.99e-08 &  1.37e-15 &      \\ 
     &           &    9 &  2.98e-08 &  8.52e-16 &      \\ 
     &           &   10 &  4.79e-08 &  8.51e-16 &      \\ 
3687 &  3.69e+04 &   10 &           &           & iters  \\ 
 \hdashline 
     &           &    1 &  1.34e-08 &  5.51e-10 &      \\ 
     &           &    2 &  1.34e-08 &  3.82e-16 &      \\ 
     &           &    3 &  1.33e-08 &  3.81e-16 &      \\ 
     &           &    4 &  1.33e-08 &  3.81e-16 &      \\ 
     &           &    5 &  6.44e-08 &  3.80e-16 &      \\ 
     &           &    6 &  1.34e-08 &  1.84e-15 &      \\ 
     &           &    7 &  1.34e-08 &  3.82e-16 &      \\ 
     &           &    8 &  1.34e-08 &  3.82e-16 &      \\ 
     &           &    9 &  1.33e-08 &  3.81e-16 &      \\ 
     &           &   10 &  1.33e-08 &  3.81e-16 &      \\ 
3688 &  3.69e+04 &   10 &           &           & iters  \\ 
 \hdashline 
     &           &    1 &  5.84e-08 &  2.38e-11 &      \\ 
     &           &    2 &  1.94e-08 &  1.67e-15 &      \\ 
     &           &    3 &  1.93e-08 &  5.53e-16 &      \\ 
     &           &    4 &  5.84e-08 &  5.52e-16 &      \\ 
     &           &    5 &  1.94e-08 &  1.67e-15 &      \\ 
     &           &    6 &  1.94e-08 &  5.54e-16 &      \\ 
     &           &    7 &  1.93e-08 &  5.53e-16 &      \\ 
     &           &    8 &  5.84e-08 &  5.52e-16 &      \\ 
     &           &    9 &  1.94e-08 &  1.67e-15 &      \\ 
     &           &   10 &  1.94e-08 &  5.54e-16 &      \\ 
3689 &  3.69e+04 &   10 &           &           & iters  \\ 
 \hdashline 
     &           &    1 &  2.43e-08 &  6.95e-10 &      \\ 
     &           &    2 &  5.34e-08 &  6.95e-16 &      \\ 
     &           &    3 &  2.44e-08 &  1.52e-15 &      \\ 
     &           &    4 &  2.44e-08 &  6.96e-16 &      \\ 
     &           &    5 &  5.34e-08 &  6.95e-16 &      \\ 
     &           &    6 &  2.44e-08 &  1.52e-15 &      \\ 
     &           &    7 &  2.44e-08 &  6.96e-16 &      \\ 
     &           &    8 &  2.43e-08 &  6.95e-16 &      \\ 
     &           &    9 &  5.34e-08 &  6.94e-16 &      \\ 
     &           &   10 &  2.44e-08 &  1.52e-15 &      \\ 
3690 &  3.69e+04 &   10 &           &           & iters  \\ 
 \hdashline 
     &           &    1 &  2.22e-08 &  5.70e-10 &      \\ 
     &           &    2 &  2.21e-08 &  6.33e-16 &      \\ 
     &           &    3 &  2.21e-08 &  6.32e-16 &      \\ 
     &           &    4 &  5.56e-08 &  6.31e-16 &      \\ 
     &           &    5 &  2.22e-08 &  1.59e-15 &      \\ 
     &           &    6 &  2.21e-08 &  6.32e-16 &      \\ 
     &           &    7 &  5.56e-08 &  6.31e-16 &      \\ 
     &           &    8 &  2.22e-08 &  1.59e-15 &      \\ 
     &           &    9 &  2.21e-08 &  6.33e-16 &      \\ 
     &           &   10 &  2.21e-08 &  6.32e-16 &      \\ 
3691 &  3.69e+04 &   10 &           &           & iters  \\ 
 \hdashline 
     &           &    1 &  2.03e-08 &  1.30e-10 &      \\ 
     &           &    2 &  2.03e-08 &  5.79e-16 &      \\ 
     &           &    3 &  2.02e-08 &  5.78e-16 &      \\ 
     &           &    4 &  5.75e-08 &  5.77e-16 &      \\ 
     &           &    5 &  2.03e-08 &  1.64e-15 &      \\ 
     &           &    6 &  2.03e-08 &  5.79e-16 &      \\ 
     &           &    7 &  2.02e-08 &  5.78e-16 &      \\ 
     &           &    8 &  5.75e-08 &  5.77e-16 &      \\ 
     &           &    9 &  2.03e-08 &  1.64e-15 &      \\ 
     &           &   10 &  2.02e-08 &  5.79e-16 &      \\ 
3692 &  3.69e+04 &   10 &           &           & iters  \\ 
 \hdashline 
     &           &    1 &  1.24e-08 &  4.46e-10 &      \\ 
     &           &    2 &  1.24e-08 &  3.55e-16 &      \\ 
     &           &    3 &  1.24e-08 &  3.54e-16 &      \\ 
     &           &    4 &  1.24e-08 &  3.54e-16 &      \\ 
     &           &    5 &  1.24e-08 &  3.53e-16 &      \\ 
     &           &    6 &  6.53e-08 &  3.53e-16 &      \\ 
     &           &    7 &  1.24e-08 &  1.86e-15 &      \\ 
     &           &    8 &  1.24e-08 &  3.55e-16 &      \\ 
     &           &    9 &  1.24e-08 &  3.54e-16 &      \\ 
     &           &   10 &  1.24e-08 &  3.54e-16 &      \\ 
3693 &  3.69e+04 &   10 &           &           & iters  \\ 
 \hdashline 
     &           &    1 &  2.70e-08 &  2.17e-10 &      \\ 
     &           &    2 &  2.70e-08 &  7.71e-16 &      \\ 
     &           &    3 &  2.69e-08 &  7.70e-16 &      \\ 
     &           &    4 &  5.08e-08 &  7.69e-16 &      \\ 
     &           &    5 &  2.70e-08 &  1.45e-15 &      \\ 
     &           &    6 &  2.69e-08 &  7.70e-16 &      \\ 
     &           &    7 &  5.08e-08 &  7.69e-16 &      \\ 
     &           &    8 &  2.70e-08 &  1.45e-15 &      \\ 
     &           &    9 &  2.69e-08 &  7.70e-16 &      \\ 
     &           &   10 &  5.08e-08 &  7.69e-16 &      \\ 
3694 &  3.69e+04 &   10 &           &           & iters  \\ 
 \hdashline 
     &           &    1 &  1.89e-08 &  5.01e-11 &      \\ 
     &           &    2 &  1.89e-08 &  5.40e-16 &      \\ 
     &           &    3 &  1.89e-08 &  5.40e-16 &      \\ 
     &           &    4 &  5.88e-08 &  5.39e-16 &      \\ 
     &           &    5 &  1.89e-08 &  1.68e-15 &      \\ 
     &           &    6 &  1.89e-08 &  5.40e-16 &      \\ 
     &           &    7 &  1.89e-08 &  5.40e-16 &      \\ 
     &           &    8 &  5.88e-08 &  5.39e-16 &      \\ 
     &           &    9 &  1.89e-08 &  1.68e-15 &      \\ 
     &           &   10 &  1.89e-08 &  5.41e-16 &      \\ 
3695 &  3.69e+04 &   10 &           &           & iters  \\ 
 \hdashline 
     &           &    1 &  2.88e-08 &  2.38e-11 &      \\ 
     &           &    2 &  4.89e-08 &  8.22e-16 &      \\ 
     &           &    3 &  2.88e-08 &  1.40e-15 &      \\ 
     &           &    4 &  2.88e-08 &  8.23e-16 &      \\ 
     &           &    5 &  4.89e-08 &  8.22e-16 &      \\ 
     &           &    6 &  2.88e-08 &  1.40e-15 &      \\ 
     &           &    7 &  4.89e-08 &  8.23e-16 &      \\ 
     &           &    8 &  2.89e-08 &  1.40e-15 &      \\ 
     &           &    9 &  2.88e-08 &  8.23e-16 &      \\ 
     &           &   10 &  4.89e-08 &  8.22e-16 &      \\ 
3696 &  3.70e+04 &   10 &           &           & iters  \\ 
 \hdashline 
     &           &    1 &  6.36e-08 &  3.59e-10 &      \\ 
     &           &    2 &  1.42e-08 &  1.81e-15 &      \\ 
     &           &    3 &  1.42e-08 &  4.05e-16 &      \\ 
     &           &    4 &  1.42e-08 &  4.05e-16 &      \\ 
     &           &    5 &  6.35e-08 &  4.04e-16 &      \\ 
     &           &    6 &  1.42e-08 &  1.81e-15 &      \\ 
     &           &    7 &  1.42e-08 &  4.06e-16 &      \\ 
     &           &    8 &  1.42e-08 &  4.06e-16 &      \\ 
     &           &    9 &  1.42e-08 &  4.05e-16 &      \\ 
     &           &   10 &  6.35e-08 &  4.04e-16 &      \\ 
3697 &  3.70e+04 &   10 &           &           & iters  \\ 
 \hdashline 
     &           &    1 &  1.61e-10 &  6.11e-10 &      \\ 
     &           &    2 &  1.60e-10 &  4.58e-18 &      \\ 
     &           &    3 &  1.60e-10 &  4.58e-18 &      \\ 
     &           &    4 &  1.60e-10 &  4.57e-18 &      \\ 
     &           &    5 &  1.60e-10 &  4.56e-18 &      \\ 
     &           &    6 &  1.59e-10 &  4.56e-18 &      \\ 
     &           &    7 &  1.59e-10 &  4.55e-18 &      \\ 
     &           &    8 &  1.59e-10 &  4.54e-18 &      \\ 
     &           &    9 &  1.59e-10 &  4.54e-18 &      \\ 
     &           &   10 &  1.58e-10 &  4.53e-18 &      \\ 
3698 &  3.70e+04 &   10 &           &           & iters  \\ 
 \hdashline 
     &           &    1 &  9.89e-09 &  5.30e-10 &      \\ 
     &           &    2 &  9.87e-09 &  2.82e-16 &      \\ 
     &           &    3 &  9.86e-09 &  2.82e-16 &      \\ 
     &           &    4 &  9.85e-09 &  2.81e-16 &      \\ 
     &           &    5 &  9.83e-09 &  2.81e-16 &      \\ 
     &           &    6 &  9.82e-09 &  2.81e-16 &      \\ 
     &           &    7 &  2.90e-08 &  2.80e-16 &      \\ 
     &           &    8 &  9.84e-09 &  8.29e-16 &      \\ 
     &           &    9 &  9.83e-09 &  2.81e-16 &      \\ 
     &           &   10 &  9.82e-09 &  2.81e-16 &      \\ 
3699 &  3.70e+04 &   10 &           &           & iters  \\ 
 \hdashline 
     &           &    1 &  4.55e-08 &  3.92e-10 &      \\ 
     &           &    2 &  3.22e-08 &  1.30e-15 &      \\ 
     &           &    3 &  4.55e-08 &  9.20e-16 &      \\ 
     &           &    4 &  3.23e-08 &  1.30e-15 &      \\ 
     &           &    5 &  3.22e-08 &  9.21e-16 &      \\ 
     &           &    6 &  4.55e-08 &  9.19e-16 &      \\ 
     &           &    7 &  3.22e-08 &  1.30e-15 &      \\ 
     &           &    8 &  4.55e-08 &  9.20e-16 &      \\ 
     &           &    9 &  3.22e-08 &  1.30e-15 &      \\ 
     &           &   10 &  3.22e-08 &  9.20e-16 &      \\ 
3700 &  3.70e+04 &   10 &           &           & iters  \\ 
 \hdashline 
     &           &    1 &  4.38e-08 &  1.38e-10 &      \\ 
     &           &    2 &  3.40e-08 &  1.25e-15 &      \\ 
     &           &    3 &  4.38e-08 &  9.70e-16 &      \\ 
     &           &    4 &  3.40e-08 &  1.25e-15 &      \\ 
     &           &    5 &  4.37e-08 &  9.70e-16 &      \\ 
     &           &    6 &  3.40e-08 &  1.25e-15 &      \\ 
     &           &    7 &  3.40e-08 &  9.71e-16 &      \\ 
     &           &    8 &  4.38e-08 &  9.69e-16 &      \\ 
     &           &    9 &  3.40e-08 &  1.25e-15 &      \\ 
     &           &   10 &  4.38e-08 &  9.70e-16 &      \\ 
3701 &  3.70e+04 &   10 &           &           & iters  \\ 
 \hdashline 
     &           &    1 &  4.02e-08 &  5.22e-10 &      \\ 
     &           &    2 &  3.75e-08 &  1.15e-15 &      \\ 
     &           &    3 &  4.02e-08 &  1.07e-15 &      \\ 
     &           &    4 &  3.75e-08 &  1.15e-15 &      \\ 
     &           &    5 &  4.02e-08 &  1.07e-15 &      \\ 
     &           &    6 &  3.75e-08 &  1.15e-15 &      \\ 
     &           &    7 &  4.02e-08 &  1.07e-15 &      \\ 
     &           &    8 &  3.75e-08 &  1.15e-15 &      \\ 
     &           &    9 &  3.75e-08 &  1.07e-15 &      \\ 
     &           &   10 &  4.03e-08 &  1.07e-15 &      \\ 
3702 &  3.70e+04 &   10 &           &           & iters  \\ 
 \hdashline 
     &           &    1 &  2.78e-09 &  3.22e-10 &      \\ 
     &           &    2 &  2.78e-09 &  7.94e-17 &      \\ 
     &           &    3 &  2.77e-09 &  7.93e-17 &      \\ 
     &           &    4 &  2.77e-09 &  7.92e-17 &      \\ 
     &           &    5 &  2.77e-09 &  7.91e-17 &      \\ 
     &           &    6 &  2.76e-09 &  7.90e-17 &      \\ 
     &           &    7 &  2.76e-09 &  7.89e-17 &      \\ 
     &           &    8 &  2.76e-09 &  7.87e-17 &      \\ 
     &           &    9 &  2.75e-09 &  7.86e-17 &      \\ 
     &           &   10 &  2.75e-09 &  7.85e-17 &      \\ 
3703 &  3.70e+04 &   10 &           &           & iters  \\ 
 \hdashline 
     &           &    1 &  1.26e-08 &  3.89e-10 &      \\ 
     &           &    2 &  1.25e-08 &  3.58e-16 &      \\ 
     &           &    3 &  1.25e-08 &  3.58e-16 &      \\ 
     &           &    4 &  2.63e-08 &  3.57e-16 &      \\ 
     &           &    5 &  1.25e-08 &  7.52e-16 &      \\ 
     &           &    6 &  1.25e-08 &  3.58e-16 &      \\ 
     &           &    7 &  2.63e-08 &  3.57e-16 &      \\ 
     &           &    8 &  1.25e-08 &  7.52e-16 &      \\ 
     &           &    9 &  1.25e-08 &  3.58e-16 &      \\ 
     &           &   10 &  2.63e-08 &  3.57e-16 &      \\ 
3704 &  3.70e+04 &   10 &           &           & iters  \\ 
 \hdashline 
     &           &    1 &  1.34e-08 &  4.58e-10 &      \\ 
     &           &    2 &  1.34e-08 &  3.83e-16 &      \\ 
     &           &    3 &  1.34e-08 &  3.83e-16 &      \\ 
     &           &    4 &  6.43e-08 &  3.82e-16 &      \\ 
     &           &    5 &  1.35e-08 &  1.84e-15 &      \\ 
     &           &    6 &  1.34e-08 &  3.84e-16 &      \\ 
     &           &    7 &  1.34e-08 &  3.84e-16 &      \\ 
     &           &    8 &  1.34e-08 &  3.83e-16 &      \\ 
     &           &    9 &  1.34e-08 &  3.82e-16 &      \\ 
     &           &   10 &  6.43e-08 &  3.82e-16 &      \\ 
3705 &  3.70e+04 &   10 &           &           & iters  \\ 
 \hdashline 
     &           &    1 &  6.72e-08 &  1.75e-10 &      \\ 
     &           &    2 &  1.05e-08 &  1.92e-15 &      \\ 
     &           &    3 &  1.05e-08 &  3.01e-16 &      \\ 
     &           &    4 &  1.05e-08 &  3.00e-16 &      \\ 
     &           &    5 &  1.05e-08 &  3.00e-16 &      \\ 
     &           &    6 &  6.72e-08 &  3.00e-16 &      \\ 
     &           &    7 &  8.83e-08 &  1.92e-15 &      \\ 
     &           &    8 &  6.72e-08 &  2.52e-15 &      \\ 
     &           &    9 &  1.05e-08 &  1.92e-15 &      \\ 
     &           &   10 &  1.05e-08 &  3.01e-16 &      \\ 
3706 &  3.70e+04 &   10 &           &           & iters  \\ 
 \hdashline 
     &           &    1 &  1.17e-08 &  5.88e-10 &      \\ 
     &           &    2 &  1.17e-08 &  3.33e-16 &      \\ 
     &           &    3 &  6.60e-08 &  3.33e-16 &      \\ 
     &           &    4 &  1.17e-08 &  1.89e-15 &      \\ 
     &           &    5 &  1.17e-08 &  3.35e-16 &      \\ 
     &           &    6 &  1.17e-08 &  3.34e-16 &      \\ 
     &           &    7 &  1.17e-08 &  3.34e-16 &      \\ 
     &           &    8 &  1.17e-08 &  3.33e-16 &      \\ 
     &           &    9 &  1.17e-08 &  3.33e-16 &      \\ 
     &           &   10 &  6.61e-08 &  3.33e-16 &      \\ 
3707 &  3.71e+04 &   10 &           &           & iters  \\ 
 \hdashline 
     &           &    1 &  6.10e-08 &  7.52e-11 &      \\ 
     &           &    2 &  1.68e-08 &  1.74e-15 &      \\ 
     &           &    3 &  1.68e-08 &  4.79e-16 &      \\ 
     &           &    4 &  1.67e-08 &  4.78e-16 &      \\ 
     &           &    5 &  1.67e-08 &  4.78e-16 &      \\ 
     &           &    6 &  6.10e-08 &  4.77e-16 &      \\ 
     &           &    7 &  1.68e-08 &  1.74e-15 &      \\ 
     &           &    8 &  1.68e-08 &  4.79e-16 &      \\ 
     &           &    9 &  1.67e-08 &  4.78e-16 &      \\ 
     &           &   10 &  6.10e-08 &  4.77e-16 &      \\ 
3708 &  3.71e+04 &   10 &           &           & iters  \\ 
 \hdashline 
     &           &    1 &  4.35e-08 &  3.29e-12 &      \\ 
     &           &    2 &  3.43e-08 &  1.24e-15 &      \\ 
     &           &    3 &  4.34e-08 &  9.79e-16 &      \\ 
     &           &    4 &  3.43e-08 &  1.24e-15 &      \\ 
     &           &    5 &  4.34e-08 &  9.79e-16 &      \\ 
     &           &    6 &  3.43e-08 &  1.24e-15 &      \\ 
     &           &    7 &  4.34e-08 &  9.80e-16 &      \\ 
     &           &    8 &  3.43e-08 &  1.24e-15 &      \\ 
     &           &    9 &  3.43e-08 &  9.80e-16 &      \\ 
     &           &   10 &  4.34e-08 &  9.79e-16 &      \\ 
3709 &  3.71e+04 &   10 &           &           & iters  \\ 
 \hdashline 
     &           &    1 &  7.47e-09 &  8.14e-10 &      \\ 
     &           &    2 &  7.46e-09 &  2.13e-16 &      \\ 
     &           &    3 &  3.14e-08 &  2.13e-16 &      \\ 
     &           &    4 &  7.50e-09 &  8.96e-16 &      \\ 
     &           &    5 &  7.49e-09 &  2.14e-16 &      \\ 
     &           &    6 &  7.48e-09 &  2.14e-16 &      \\ 
     &           &    7 &  7.47e-09 &  2.13e-16 &      \\ 
     &           &    8 &  3.14e-08 &  2.13e-16 &      \\ 
     &           &    9 &  7.50e-09 &  8.96e-16 &      \\ 
     &           &   10 &  7.49e-09 &  2.14e-16 &      \\ 
3710 &  3.71e+04 &   10 &           &           & iters  \\ 
 \hdashline 
     &           &    1 &  5.64e-09 &  1.10e-09 &      \\ 
     &           &    2 &  5.64e-09 &  1.61e-16 &      \\ 
     &           &    3 &  5.63e-09 &  1.61e-16 &      \\ 
     &           &    4 &  5.62e-09 &  1.61e-16 &      \\ 
     &           &    5 &  5.61e-09 &  1.60e-16 &      \\ 
     &           &    6 &  5.60e-09 &  1.60e-16 &      \\ 
     &           &    7 &  7.21e-08 &  1.60e-16 &      \\ 
     &           &    8 &  8.34e-08 &  2.06e-15 &      \\ 
     &           &    9 &  7.21e-08 &  2.38e-15 &      \\ 
     &           &   10 &  5.68e-09 &  2.06e-15 &      \\ 
3711 &  3.71e+04 &   10 &           &           & iters  \\ 
 \hdashline 
     &           &    1 &  1.94e-08 &  4.21e-10 &      \\ 
     &           &    2 &  5.83e-08 &  5.53e-16 &      \\ 
     &           &    3 &  1.94e-08 &  1.67e-15 &      \\ 
     &           &    4 &  1.94e-08 &  5.55e-16 &      \\ 
     &           &    5 &  1.94e-08 &  5.54e-16 &      \\ 
     &           &    6 &  5.83e-08 &  5.53e-16 &      \\ 
     &           &    7 &  1.94e-08 &  1.67e-15 &      \\ 
     &           &    8 &  1.94e-08 &  5.55e-16 &      \\ 
     &           &    9 &  1.94e-08 &  5.54e-16 &      \\ 
     &           &   10 &  5.83e-08 &  5.53e-16 &      \\ 
3712 &  3.71e+04 &   10 &           &           & iters  \\ 
 \hdashline 
     &           &    1 &  6.88e-09 &  3.37e-10 &      \\ 
     &           &    2 &  3.20e-08 &  1.96e-16 &      \\ 
     &           &    3 &  6.91e-09 &  9.13e-16 &      \\ 
     &           &    4 &  6.90e-09 &  1.97e-16 &      \\ 
     &           &    5 &  6.89e-09 &  1.97e-16 &      \\ 
     &           &    6 &  6.88e-09 &  1.97e-16 &      \\ 
     &           &    7 &  6.87e-09 &  1.96e-16 &      \\ 
     &           &    8 &  3.20e-08 &  1.96e-16 &      \\ 
     &           &    9 &  6.91e-09 &  9.13e-16 &      \\ 
     &           &   10 &  6.90e-09 &  1.97e-16 &      \\ 
3713 &  3.71e+04 &   10 &           &           & iters  \\ 
 \hdashline 
     &           &    1 &  2.49e-08 &  2.71e-10 &      \\ 
     &           &    2 &  5.28e-08 &  7.11e-16 &      \\ 
     &           &    3 &  2.50e-08 &  1.51e-15 &      \\ 
     &           &    4 &  2.49e-08 &  7.12e-16 &      \\ 
     &           &    5 &  5.28e-08 &  7.11e-16 &      \\ 
     &           &    6 &  2.50e-08 &  1.51e-15 &      \\ 
     &           &    7 &  2.49e-08 &  7.13e-16 &      \\ 
     &           &    8 &  5.28e-08 &  7.12e-16 &      \\ 
     &           &    9 &  2.50e-08 &  1.51e-15 &      \\ 
     &           &   10 &  2.49e-08 &  7.13e-16 &      \\ 
3714 &  3.71e+04 &   10 &           &           & iters  \\ 
 \hdashline 
     &           &    1 &  1.16e-08 &  6.50e-11 &      \\ 
     &           &    2 &  1.16e-08 &  3.30e-16 &      \\ 
     &           &    3 &  1.15e-08 &  3.30e-16 &      \\ 
     &           &    4 &  1.15e-08 &  3.29e-16 &      \\ 
     &           &    5 &  1.15e-08 &  3.29e-16 &      \\ 
     &           &    6 &  6.62e-08 &  3.28e-16 &      \\ 
     &           &    7 &  1.16e-08 &  1.89e-15 &      \\ 
     &           &    8 &  1.16e-08 &  3.31e-16 &      \\ 
     &           &    9 &  1.16e-08 &  3.30e-16 &      \\ 
     &           &   10 &  1.15e-08 &  3.30e-16 &      \\ 
3715 &  3.71e+04 &   10 &           &           & iters  \\ 
 \hdashline 
     &           &    1 &  4.67e-08 &  3.82e-10 &      \\ 
     &           &    2 &  3.11e-08 &  1.33e-15 &      \\ 
     &           &    3 &  3.10e-08 &  8.87e-16 &      \\ 
     &           &    4 &  4.67e-08 &  8.86e-16 &      \\ 
     &           &    5 &  3.11e-08 &  1.33e-15 &      \\ 
     &           &    6 &  4.67e-08 &  8.86e-16 &      \\ 
     &           &    7 &  3.11e-08 &  1.33e-15 &      \\ 
     &           &    8 &  3.10e-08 &  8.87e-16 &      \\ 
     &           &    9 &  4.67e-08 &  8.86e-16 &      \\ 
     &           &   10 &  3.11e-08 &  1.33e-15 &      \\ 
3716 &  3.72e+04 &   10 &           &           & iters  \\ 
 \hdashline 
     &           &    1 &  1.62e-08 &  9.06e-10 &      \\ 
     &           &    2 &  1.61e-08 &  4.61e-16 &      \\ 
     &           &    3 &  1.61e-08 &  4.61e-16 &      \\ 
     &           &    4 &  6.16e-08 &  4.60e-16 &      \\ 
     &           &    5 &  1.62e-08 &  1.76e-15 &      \\ 
     &           &    6 &  1.62e-08 &  4.62e-16 &      \\ 
     &           &    7 &  1.61e-08 &  4.61e-16 &      \\ 
     &           &    8 &  6.16e-08 &  4.60e-16 &      \\ 
     &           &    9 &  1.62e-08 &  1.76e-15 &      \\ 
     &           &   10 &  1.62e-08 &  4.62e-16 &      \\ 
3717 &  3.72e+04 &   10 &           &           & iters  \\ 
 \hdashline 
     &           &    1 &  5.60e-08 &  7.82e-10 &      \\ 
     &           &    2 &  2.18e-08 &  1.60e-15 &      \\ 
     &           &    3 &  2.17e-08 &  6.21e-16 &      \\ 
     &           &    4 &  2.17e-08 &  6.20e-16 &      \\ 
     &           &    5 &  5.60e-08 &  6.19e-16 &      \\ 
     &           &    6 &  2.17e-08 &  1.60e-15 &      \\ 
     &           &    7 &  2.17e-08 &  6.21e-16 &      \\ 
     &           &    8 &  2.17e-08 &  6.20e-16 &      \\ 
     &           &    9 &  5.60e-08 &  6.19e-16 &      \\ 
     &           &   10 &  2.17e-08 &  1.60e-15 &      \\ 
3718 &  3.72e+04 &   10 &           &           & iters  \\ 
 \hdashline 
     &           &    1 &  2.00e-08 &  6.27e-10 &      \\ 
     &           &    2 &  5.77e-08 &  5.70e-16 &      \\ 
     &           &    3 &  2.00e-08 &  1.65e-15 &      \\ 
     &           &    4 &  2.00e-08 &  5.72e-16 &      \\ 
     &           &    5 &  2.00e-08 &  5.71e-16 &      \\ 
     &           &    6 &  5.77e-08 &  5.70e-16 &      \\ 
     &           &    7 &  2.00e-08 &  1.65e-15 &      \\ 
     &           &    8 &  2.00e-08 &  5.72e-16 &      \\ 
     &           &    9 &  2.00e-08 &  5.71e-16 &      \\ 
     &           &   10 &  5.77e-08 &  5.70e-16 &      \\ 
3719 &  3.72e+04 &   10 &           &           & iters  \\ 
 \hdashline 
     &           &    1 &  2.93e-08 &  7.48e-10 &      \\ 
     &           &    2 &  2.92e-08 &  8.36e-16 &      \\ 
     &           &    3 &  2.92e-08 &  8.35e-16 &      \\ 
     &           &    4 &  4.85e-08 &  8.33e-16 &      \\ 
     &           &    5 &  2.92e-08 &  1.38e-15 &      \\ 
     &           &    6 &  4.85e-08 &  8.34e-16 &      \\ 
     &           &    7 &  2.93e-08 &  1.38e-15 &      \\ 
     &           &    8 &  2.92e-08 &  8.35e-16 &      \\ 
     &           &    9 &  4.85e-08 &  8.34e-16 &      \\ 
     &           &   10 &  2.92e-08 &  1.38e-15 &      \\ 
3720 &  3.72e+04 &   10 &           &           & iters  \\ 
 \hdashline 
     &           &    1 &  1.94e-08 &  7.22e-10 &      \\ 
     &           &    2 &  1.94e-08 &  5.53e-16 &      \\ 
     &           &    3 &  5.84e-08 &  5.52e-16 &      \\ 
     &           &    4 &  1.94e-08 &  1.67e-15 &      \\ 
     &           &    5 &  1.94e-08 &  5.54e-16 &      \\ 
     &           &    6 &  1.94e-08 &  5.53e-16 &      \\ 
     &           &    7 &  5.84e-08 &  5.52e-16 &      \\ 
     &           &    8 &  1.94e-08 &  1.67e-15 &      \\ 
     &           &    9 &  1.94e-08 &  5.54e-16 &      \\ 
     &           &   10 &  1.94e-08 &  5.53e-16 &      \\ 
3721 &  3.72e+04 &   10 &           &           & iters  \\ 
 \hdashline 
     &           &    1 &  1.57e-08 &  2.98e-10 &      \\ 
     &           &    2 &  1.57e-08 &  4.49e-16 &      \\ 
     &           &    3 &  1.57e-08 &  4.48e-16 &      \\ 
     &           &    4 &  6.20e-08 &  4.47e-16 &      \\ 
     &           &    5 &  1.57e-08 &  1.77e-15 &      \\ 
     &           &    6 &  1.57e-08 &  4.49e-16 &      \\ 
     &           &    7 &  1.57e-08 &  4.49e-16 &      \\ 
     &           &    8 &  1.57e-08 &  4.48e-16 &      \\ 
     &           &    9 &  6.20e-08 &  4.47e-16 &      \\ 
     &           &   10 &  1.57e-08 &  1.77e-15 &      \\ 
3722 &  3.72e+04 &   10 &           &           & iters  \\ 
 \hdashline 
     &           &    1 &  2.45e-08 &  1.08e-11 &      \\ 
     &           &    2 &  5.32e-08 &  6.99e-16 &      \\ 
     &           &    3 &  2.45e-08 &  1.52e-15 &      \\ 
     &           &    4 &  2.45e-08 &  7.00e-16 &      \\ 
     &           &    5 &  5.32e-08 &  6.99e-16 &      \\ 
     &           &    6 &  2.45e-08 &  1.52e-15 &      \\ 
     &           &    7 &  2.45e-08 &  7.00e-16 &      \\ 
     &           &    8 &  5.32e-08 &  6.99e-16 &      \\ 
     &           &    9 &  2.45e-08 &  1.52e-15 &      \\ 
     &           &   10 &  2.45e-08 &  7.01e-16 &      \\ 
3723 &  3.72e+04 &   10 &           &           & iters  \\ 
 \hdashline 
     &           &    1 &  2.00e-08 &  1.14e-10 &      \\ 
     &           &    2 &  1.99e-08 &  5.70e-16 &      \\ 
     &           &    3 &  1.99e-08 &  5.69e-16 &      \\ 
     &           &    4 &  1.99e-08 &  5.68e-16 &      \\ 
     &           &    5 &  5.78e-08 &  5.67e-16 &      \\ 
     &           &    6 &  1.99e-08 &  1.65e-15 &      \\ 
     &           &    7 &  1.99e-08 &  5.69e-16 &      \\ 
     &           &    8 &  1.99e-08 &  5.68e-16 &      \\ 
     &           &    9 &  5.78e-08 &  5.67e-16 &      \\ 
     &           &   10 &  1.99e-08 &  1.65e-15 &      \\ 
3724 &  3.72e+04 &   10 &           &           & iters  \\ 
 \hdashline 
     &           &    1 &  2.19e-08 &  1.90e-10 &      \\ 
     &           &    2 &  5.58e-08 &  6.25e-16 &      \\ 
     &           &    3 &  2.20e-08 &  1.59e-15 &      \\ 
     &           &    4 &  2.19e-08 &  6.27e-16 &      \\ 
     &           &    5 &  2.19e-08 &  6.26e-16 &      \\ 
     &           &    6 &  5.58e-08 &  6.25e-16 &      \\ 
     &           &    7 &  2.19e-08 &  1.59e-15 &      \\ 
     &           &    8 &  2.19e-08 &  6.26e-16 &      \\ 
     &           &    9 &  5.58e-08 &  6.25e-16 &      \\ 
     &           &   10 &  2.20e-08 &  1.59e-15 &      \\ 
3725 &  3.72e+04 &   10 &           &           & iters  \\ 
 \hdashline 
     &           &    1 &  8.06e-08 &  1.69e-10 &      \\ 
     &           &    2 &  7.49e-08 &  2.30e-15 &      \\ 
     &           &    3 &  8.06e-08 &  2.14e-15 &      \\ 
     &           &    4 &  7.49e-08 &  2.30e-15 &      \\ 
     &           &    5 &  8.06e-08 &  2.14e-15 &      \\ 
     &           &    6 &  7.49e-08 &  2.30e-15 &      \\ 
     &           &    7 &  8.06e-08 &  2.14e-15 &      \\ 
     &           &    8 &  7.49e-08 &  2.30e-15 &      \\ 
     &           &    9 &  2.92e-09 &  2.14e-15 &      \\ 
     &           &   10 &  2.91e-09 &  8.33e-17 &      \\ 
3726 &  3.72e+04 &   10 &           &           & iters  \\ 
 \hdashline 
     &           &    1 &  9.63e-09 &  9.64e-10 &      \\ 
     &           &    2 &  9.62e-09 &  2.75e-16 &      \\ 
     &           &    3 &  9.61e-09 &  2.75e-16 &      \\ 
     &           &    4 &  9.59e-09 &  2.74e-16 &      \\ 
     &           &    5 &  6.81e-08 &  2.74e-16 &      \\ 
     &           &    6 &  9.68e-09 &  1.94e-15 &      \\ 
     &           &    7 &  9.66e-09 &  2.76e-16 &      \\ 
     &           &    8 &  9.65e-09 &  2.76e-16 &      \\ 
     &           &    9 &  9.63e-09 &  2.75e-16 &      \\ 
     &           &   10 &  9.62e-09 &  2.75e-16 &      \\ 
3727 &  3.73e+04 &   10 &           &           & iters  \\ 
 \hdashline 
     &           &    1 &  2.23e-08 &  1.40e-09 &      \\ 
     &           &    2 &  5.54e-08 &  6.37e-16 &      \\ 
     &           &    3 &  2.24e-08 &  1.58e-15 &      \\ 
     &           &    4 &  2.23e-08 &  6.38e-16 &      \\ 
     &           &    5 &  5.54e-08 &  6.38e-16 &      \\ 
     &           &    6 &  2.24e-08 &  1.58e-15 &      \\ 
     &           &    7 &  2.24e-08 &  6.39e-16 &      \\ 
     &           &    8 &  2.23e-08 &  6.38e-16 &      \\ 
     &           &    9 &  5.54e-08 &  6.37e-16 &      \\ 
     &           &   10 &  2.24e-08 &  1.58e-15 &      \\ 
3728 &  3.73e+04 &   10 &           &           & iters  \\ 
 \hdashline 
     &           &    1 &  5.32e-09 &  1.20e-09 &      \\ 
     &           &    2 &  5.32e-09 &  1.52e-16 &      \\ 
     &           &    3 &  5.31e-09 &  1.52e-16 &      \\ 
     &           &    4 &  5.30e-09 &  1.51e-16 &      \\ 
     &           &    5 &  5.29e-09 &  1.51e-16 &      \\ 
     &           &    6 &  5.29e-09 &  1.51e-16 &      \\ 
     &           &    7 &  7.24e-08 &  1.51e-16 &      \\ 
     &           &    8 &  8.31e-08 &  2.07e-15 &      \\ 
     &           &    9 &  7.24e-08 &  2.37e-15 &      \\ 
     &           &   10 &  5.37e-09 &  2.07e-15 &      \\ 
3729 &  3.73e+04 &   10 &           &           & iters  \\ 
 \hdashline 
     &           &    1 &  1.09e-08 &  9.32e-10 &      \\ 
     &           &    2 &  6.68e-08 &  3.12e-16 &      \\ 
     &           &    3 &  4.98e-08 &  1.91e-15 &      \\ 
     &           &    4 &  1.09e-08 &  1.42e-15 &      \\ 
     &           &    5 &  1.09e-08 &  3.12e-16 &      \\ 
     &           &    6 &  6.68e-08 &  3.11e-16 &      \\ 
     &           &    7 &  1.10e-08 &  1.91e-15 &      \\ 
     &           &    8 &  1.10e-08 &  3.14e-16 &      \\ 
     &           &    9 &  1.10e-08 &  3.13e-16 &      \\ 
     &           &   10 &  1.09e-08 &  3.13e-16 &      \\ 
3730 &  3.73e+04 &   10 &           &           & iters  \\ 
 \hdashline 
     &           &    1 &  1.56e-08 &  6.21e-10 &      \\ 
     &           &    2 &  1.55e-08 &  4.44e-16 &      \\ 
     &           &    3 &  2.33e-08 &  4.43e-16 &      \\ 
     &           &    4 &  1.55e-08 &  6.66e-16 &      \\ 
     &           &    5 &  1.55e-08 &  4.44e-16 &      \\ 
     &           &    6 &  2.33e-08 &  4.43e-16 &      \\ 
     &           &    7 &  1.55e-08 &  6.66e-16 &      \\ 
     &           &    8 &  2.33e-08 &  4.43e-16 &      \\ 
     &           &    9 &  1.55e-08 &  6.66e-16 &      \\ 
     &           &   10 &  1.55e-08 &  4.44e-16 &      \\ 
3731 &  3.73e+04 &   10 &           &           & iters  \\ 
 \hdashline 
     &           &    1 &  1.09e-08 &  1.45e-10 &      \\ 
     &           &    2 &  1.09e-08 &  3.12e-16 &      \\ 
     &           &    3 &  1.09e-08 &  3.11e-16 &      \\ 
     &           &    4 &  1.09e-08 &  3.11e-16 &      \\ 
     &           &    5 &  1.09e-08 &  3.10e-16 &      \\ 
     &           &    6 &  6.68e-08 &  3.10e-16 &      \\ 
     &           &    7 &  1.09e-08 &  1.91e-15 &      \\ 
     &           &    8 &  1.09e-08 &  3.12e-16 &      \\ 
     &           &    9 &  1.09e-08 &  3.12e-16 &      \\ 
     &           &   10 &  1.09e-08 &  3.11e-16 &      \\ 
3732 &  3.73e+04 &   10 &           &           & iters  \\ 
 \hdashline 
     &           &    1 &  2.46e-08 &  2.92e-10 &      \\ 
     &           &    2 &  2.46e-08 &  7.02e-16 &      \\ 
     &           &    3 &  5.32e-08 &  7.01e-16 &      \\ 
     &           &    4 &  2.46e-08 &  1.52e-15 &      \\ 
     &           &    5 &  2.46e-08 &  7.02e-16 &      \\ 
     &           &    6 &  5.32e-08 &  7.01e-16 &      \\ 
     &           &    7 &  2.46e-08 &  1.52e-15 &      \\ 
     &           &    8 &  2.46e-08 &  7.02e-16 &      \\ 
     &           &    9 &  5.32e-08 &  7.01e-16 &      \\ 
     &           &   10 &  2.46e-08 &  1.52e-15 &      \\ 
3733 &  3.73e+04 &   10 &           &           & iters  \\ 
 \hdashline 
     &           &    1 &  6.33e-09 &  4.74e-10 &      \\ 
     &           &    2 &  6.32e-09 &  1.81e-16 &      \\ 
     &           &    3 &  6.31e-09 &  1.80e-16 &      \\ 
     &           &    4 &  3.25e-08 &  1.80e-16 &      \\ 
     &           &    5 &  6.35e-09 &  9.29e-16 &      \\ 
     &           &    6 &  6.34e-09 &  1.81e-16 &      \\ 
     &           &    7 &  6.33e-09 &  1.81e-16 &      \\ 
     &           &    8 &  6.32e-09 &  1.81e-16 &      \\ 
     &           &    9 &  6.32e-09 &  1.81e-16 &      \\ 
     &           &   10 &  6.31e-09 &  1.80e-16 &      \\ 
3734 &  3.73e+04 &   10 &           &           & iters  \\ 
 \hdashline 
     &           &    1 &  1.04e-09 &  6.84e-10 &      \\ 
     &           &    2 &  1.04e-09 &  2.98e-17 &      \\ 
     &           &    3 &  1.04e-09 &  2.98e-17 &      \\ 
     &           &    4 &  1.04e-09 &  2.97e-17 &      \\ 
     &           &    5 &  1.04e-09 &  2.97e-17 &      \\ 
     &           &    6 &  1.04e-09 &  2.96e-17 &      \\ 
     &           &    7 &  1.04e-09 &  2.96e-17 &      \\ 
     &           &    8 &  1.03e-09 &  2.96e-17 &      \\ 
     &           &    9 &  1.03e-09 &  2.95e-17 &      \\ 
     &           &   10 &  1.03e-09 &  2.95e-17 &      \\ 
3735 &  3.73e+04 &   10 &           &           & iters  \\ 
 \hdashline 
     &           &    1 &  2.12e-08 &  1.11e-09 &      \\ 
     &           &    2 &  2.12e-08 &  6.05e-16 &      \\ 
     &           &    3 &  5.66e-08 &  6.04e-16 &      \\ 
     &           &    4 &  2.12e-08 &  1.61e-15 &      \\ 
     &           &    5 &  2.12e-08 &  6.05e-16 &      \\ 
     &           &    6 &  2.12e-08 &  6.05e-16 &      \\ 
     &           &    7 &  5.66e-08 &  6.04e-16 &      \\ 
     &           &    8 &  2.12e-08 &  1.61e-15 &      \\ 
     &           &    9 &  2.12e-08 &  6.05e-16 &      \\ 
     &           &   10 &  2.11e-08 &  6.04e-16 &      \\ 
3736 &  3.74e+04 &   10 &           &           & iters  \\ 
 \hdashline 
     &           &    1 &  7.80e-08 &  1.33e-09 &      \\ 
     &           &    2 &  2.24e-10 &  2.23e-15 &      \\ 
     &           &    3 &  2.23e-10 &  6.38e-18 &      \\ 
     &           &    4 &  2.23e-10 &  6.38e-18 &      \\ 
     &           &    5 &  2.23e-10 &  6.37e-18 &      \\ 
     &           &    6 &  2.22e-10 &  6.36e-18 &      \\ 
     &           &    7 &  2.22e-10 &  6.35e-18 &      \\ 
     &           &    8 &  2.22e-10 &  6.34e-18 &      \\ 
     &           &    9 &  2.21e-10 &  6.33e-18 &      \\ 
     &           &   10 &  2.21e-10 &  6.32e-18 &      \\ 
3737 &  3.74e+04 &   10 &           &           & iters  \\ 
 \hdashline 
     &           &    1 &  3.29e-08 &  2.29e-10 &      \\ 
     &           &    2 &  4.48e-08 &  9.39e-16 &      \\ 
     &           &    3 &  3.29e-08 &  1.28e-15 &      \\ 
     &           &    4 &  4.48e-08 &  9.39e-16 &      \\ 
     &           &    5 &  3.29e-08 &  1.28e-15 &      \\ 
     &           &    6 &  3.29e-08 &  9.40e-16 &      \\ 
     &           &    7 &  4.49e-08 &  9.38e-16 &      \\ 
     &           &    8 &  3.29e-08 &  1.28e-15 &      \\ 
     &           &    9 &  4.48e-08 &  9.39e-16 &      \\ 
     &           &   10 &  3.29e-08 &  1.28e-15 &      \\ 
3738 &  3.74e+04 &   10 &           &           & iters  \\ 
 \hdashline 
     &           &    1 &  6.76e-09 &  1.19e-09 &      \\ 
     &           &    2 &  6.75e-09 &  1.93e-16 &      \\ 
     &           &    3 &  6.74e-09 &  1.93e-16 &      \\ 
     &           &    4 &  6.73e-09 &  1.92e-16 &      \\ 
     &           &    5 &  6.72e-09 &  1.92e-16 &      \\ 
     &           &    6 &  6.71e-09 &  1.92e-16 &      \\ 
     &           &    7 &  6.70e-09 &  1.92e-16 &      \\ 
     &           &    8 &  7.10e-08 &  1.91e-16 &      \\ 
     &           &    9 &  8.45e-08 &  2.03e-15 &      \\ 
     &           &   10 &  7.10e-08 &  2.41e-15 &      \\ 
3739 &  3.74e+04 &   10 &           &           & iters  \\ 
 \hdashline 
     &           &    1 &  3.51e-08 &  5.35e-10 &      \\ 
     &           &    2 &  3.76e-09 &  1.00e-15 &      \\ 
     &           &    3 &  3.75e-09 &  1.07e-16 &      \\ 
     &           &    4 &  3.75e-09 &  1.07e-16 &      \\ 
     &           &    5 &  3.74e-09 &  1.07e-16 &      \\ 
     &           &    6 &  3.74e-09 &  1.07e-16 &      \\ 
     &           &    7 &  3.73e-09 &  1.07e-16 &      \\ 
     &           &    8 &  3.72e-09 &  1.06e-16 &      \\ 
     &           &    9 &  3.51e-08 &  1.06e-16 &      \\ 
     &           &   10 &  3.77e-09 &  1.00e-15 &      \\ 
3740 &  3.74e+04 &   10 &           &           & iters  \\ 
 \hdashline 
     &           &    1 &  9.25e-11 &  3.91e-10 &      \\ 
     &           &    2 &  9.24e-11 &  2.64e-18 &      \\ 
     &           &    3 &  9.23e-11 &  2.64e-18 &      \\ 
     &           &    4 &  9.21e-11 &  2.63e-18 &      \\ 
     &           &    5 &  9.20e-11 &  2.63e-18 &      \\ 
     &           &    6 &  9.19e-11 &  2.63e-18 &      \\ 
     &           &    7 &  9.17e-11 &  2.62e-18 &      \\ 
     &           &    8 &  9.16e-11 &  2.62e-18 &      \\ 
     &           &    9 &  9.15e-11 &  2.61e-18 &      \\ 
     &           &   10 &  9.14e-11 &  2.61e-18 &      \\ 
3741 &  3.74e+04 &   10 &           &           & iters  \\ 
 \hdashline 
     &           &    1 &  1.54e-08 &  2.52e-10 &      \\ 
     &           &    2 &  1.53e-08 &  4.38e-16 &      \\ 
     &           &    3 &  6.24e-08 &  4.38e-16 &      \\ 
     &           &    4 &  1.54e-08 &  1.78e-15 &      \\ 
     &           &    5 &  1.54e-08 &  4.39e-16 &      \\ 
     &           &    6 &  1.54e-08 &  4.39e-16 &      \\ 
     &           &    7 &  1.53e-08 &  4.38e-16 &      \\ 
     &           &    8 &  6.24e-08 &  4.38e-16 &      \\ 
     &           &    9 &  1.54e-08 &  1.78e-15 &      \\ 
     &           &   10 &  1.54e-08 &  4.39e-16 &      \\ 
3742 &  3.74e+04 &   10 &           &           & iters  \\ 
 \hdashline 
     &           &    1 &  6.67e-09 &  4.31e-10 &      \\ 
     &           &    2 &  6.66e-09 &  1.90e-16 &      \\ 
     &           &    3 &  7.10e-08 &  1.90e-16 &      \\ 
     &           &    4 &  8.44e-08 &  2.03e-15 &      \\ 
     &           &    5 &  7.11e-08 &  2.41e-15 &      \\ 
     &           &    6 &  6.73e-09 &  2.03e-15 &      \\ 
     &           &    7 &  6.72e-09 &  1.92e-16 &      \\ 
     &           &    8 &  6.71e-09 &  1.92e-16 &      \\ 
     &           &    9 &  6.70e-09 &  1.91e-16 &      \\ 
     &           &   10 &  6.69e-09 &  1.91e-16 &      \\ 
3743 &  3.74e+04 &   10 &           &           & iters  \\ 
 \hdashline 
     &           &    1 &  1.89e-09 &  1.03e-09 &      \\ 
     &           &    2 &  1.89e-09 &  5.40e-17 &      \\ 
     &           &    3 &  1.89e-09 &  5.39e-17 &      \\ 
     &           &    4 &  1.88e-09 &  5.38e-17 &      \\ 
     &           &    5 &  1.88e-09 &  5.38e-17 &      \\ 
     &           &    6 &  1.88e-09 &  5.37e-17 &      \\ 
     &           &    7 &  1.88e-09 &  5.36e-17 &      \\ 
     &           &    8 &  1.87e-09 &  5.35e-17 &      \\ 
     &           &    9 &  1.87e-09 &  5.35e-17 &      \\ 
     &           &   10 &  1.87e-09 &  5.34e-17 &      \\ 
3744 &  3.74e+04 &   10 &           &           & iters  \\ 
 \hdashline 
     &           &    1 &  1.03e-08 &  8.67e-10 &      \\ 
     &           &    2 &  1.03e-08 &  2.95e-16 &      \\ 
     &           &    3 &  1.03e-08 &  2.95e-16 &      \\ 
     &           &    4 &  1.03e-08 &  2.94e-16 &      \\ 
     &           &    5 &  1.03e-08 &  2.94e-16 &      \\ 
     &           &    6 &  1.03e-08 &  2.93e-16 &      \\ 
     &           &    7 &  6.74e-08 &  2.93e-16 &      \\ 
     &           &    8 &  8.80e-08 &  1.92e-15 &      \\ 
     &           &    9 &  6.75e-08 &  2.51e-15 &      \\ 
     &           &   10 &  1.03e-08 &  1.93e-15 &      \\ 
3745 &  3.74e+04 &   10 &           &           & iters  \\ 
 \hdashline 
     &           &    1 &  1.72e-08 &  1.35e-10 &      \\ 
     &           &    2 &  1.71e-08 &  4.90e-16 &      \\ 
     &           &    3 &  1.71e-08 &  4.89e-16 &      \\ 
     &           &    4 &  6.06e-08 &  4.88e-16 &      \\ 
     &           &    5 &  1.72e-08 &  1.73e-15 &      \\ 
     &           &    6 &  1.71e-08 &  4.90e-16 &      \\ 
     &           &    7 &  1.71e-08 &  4.89e-16 &      \\ 
     &           &    8 &  6.06e-08 &  4.89e-16 &      \\ 
     &           &    9 &  1.72e-08 &  1.73e-15 &      \\ 
     &           &   10 &  1.72e-08 &  4.91e-16 &      \\ 
3746 &  3.74e+04 &   10 &           &           & iters  \\ 
 \hdashline 
     &           &    1 &  2.40e-08 &  1.91e-10 &      \\ 
     &           &    2 &  5.37e-08 &  6.86e-16 &      \\ 
     &           &    3 &  2.41e-08 &  1.53e-15 &      \\ 
     &           &    4 &  2.41e-08 &  6.88e-16 &      \\ 
     &           &    5 &  5.37e-08 &  6.87e-16 &      \\ 
     &           &    6 &  2.41e-08 &  1.53e-15 &      \\ 
     &           &    7 &  2.41e-08 &  6.88e-16 &      \\ 
     &           &    8 &  5.37e-08 &  6.87e-16 &      \\ 
     &           &    9 &  2.41e-08 &  1.53e-15 &      \\ 
     &           &   10 &  2.41e-08 &  6.88e-16 &      \\ 
3747 &  3.75e+04 &   10 &           &           & iters  \\ 
 \hdashline 
     &           &    1 &  1.17e-08 &  1.12e-09 &      \\ 
     &           &    2 &  1.17e-08 &  3.34e-16 &      \\ 
     &           &    3 &  1.17e-08 &  3.34e-16 &      \\ 
     &           &    4 &  1.17e-08 &  3.33e-16 &      \\ 
     &           &    5 &  1.16e-08 &  3.33e-16 &      \\ 
     &           &    6 &  6.61e-08 &  3.32e-16 &      \\ 
     &           &    7 &  1.17e-08 &  1.89e-15 &      \\ 
     &           &    8 &  1.17e-08 &  3.35e-16 &      \\ 
     &           &    9 &  1.17e-08 &  3.34e-16 &      \\ 
     &           &   10 &  1.17e-08 &  3.34e-16 &      \\ 
3748 &  3.75e+04 &   10 &           &           & iters  \\ 
 \hdashline 
     &           &    1 &  3.45e-08 &  5.06e-10 &      \\ 
     &           &    2 &  3.45e-08 &  9.85e-16 &      \\ 
     &           &    3 &  4.33e-08 &  9.83e-16 &      \\ 
     &           &    4 &  3.45e-08 &  1.24e-15 &      \\ 
     &           &    5 &  3.44e-08 &  9.84e-16 &      \\ 
     &           &    6 &  4.33e-08 &  9.82e-16 &      \\ 
     &           &    7 &  3.44e-08 &  1.24e-15 &      \\ 
     &           &    8 &  4.33e-08 &  9.83e-16 &      \\ 
     &           &    9 &  3.44e-08 &  1.24e-15 &      \\ 
     &           &   10 &  4.33e-08 &  9.83e-16 &      \\ 
3749 &  3.75e+04 &   10 &           &           & iters  \\ 
 \hdashline 
     &           &    1 &  2.04e-08 &  1.29e-09 &      \\ 
     &           &    2 &  2.03e-08 &  5.81e-16 &      \\ 
     &           &    3 &  5.74e-08 &  5.80e-16 &      \\ 
     &           &    4 &  2.04e-08 &  1.64e-15 &      \\ 
     &           &    5 &  2.04e-08 &  5.82e-16 &      \\ 
     &           &    6 &  2.03e-08 &  5.81e-16 &      \\ 
     &           &    7 &  5.74e-08 &  5.80e-16 &      \\ 
     &           &    8 &  2.04e-08 &  1.64e-15 &      \\ 
     &           &    9 &  2.03e-08 &  5.82e-16 &      \\ 
     &           &   10 &  2.03e-08 &  5.81e-16 &      \\ 
3750 &  3.75e+04 &   10 &           &           & iters  \\ 
 \hdashline 
     &           &    1 &  8.55e-08 &  3.05e-09 &      \\ 
     &           &    2 &  3.11e-08 &  2.44e-15 &      \\ 
     &           &    3 &  7.76e-09 &  8.88e-16 &      \\ 
     &           &    4 &  7.75e-09 &  2.22e-16 &      \\ 
     &           &    5 &  3.11e-08 &  2.21e-16 &      \\ 
     &           &    6 &  7.78e-09 &  8.88e-16 &      \\ 
     &           &    7 &  7.77e-09 &  2.22e-16 &      \\ 
     &           &    8 &  7.76e-09 &  2.22e-16 &      \\ 
     &           &    9 &  7.75e-09 &  2.22e-16 &      \\ 
     &           &   10 &  3.11e-08 &  2.21e-16 &      \\ 
3751 &  3.75e+04 &   10 &           &           & iters  \\ 
 \hdashline 
     &           &    1 &  2.33e-08 &  2.15e-09 &      \\ 
     &           &    2 &  1.56e-08 &  6.64e-16 &      \\ 
     &           &    3 &  2.32e-08 &  4.46e-16 &      \\ 
     &           &    4 &  1.56e-08 &  6.63e-16 &      \\ 
     &           &    5 &  1.56e-08 &  4.46e-16 &      \\ 
     &           &    6 &  2.33e-08 &  4.45e-16 &      \\ 
     &           &    7 &  1.56e-08 &  6.64e-16 &      \\ 
     &           &    8 &  2.32e-08 &  4.46e-16 &      \\ 
     &           &    9 &  1.56e-08 &  6.64e-16 &      \\ 
     &           &   10 &  1.56e-08 &  4.46e-16 &      \\ 
3752 &  3.75e+04 &   10 &           &           & iters  \\ 
 \hdashline 
     &           &    1 &  6.35e-09 &  4.80e-10 &      \\ 
     &           &    2 &  6.34e-09 &  1.81e-16 &      \\ 
     &           &    3 &  6.33e-09 &  1.81e-16 &      \\ 
     &           &    4 &  6.32e-09 &  1.81e-16 &      \\ 
     &           &    5 &  6.31e-09 &  1.80e-16 &      \\ 
     &           &    6 &  6.30e-09 &  1.80e-16 &      \\ 
     &           &    7 &  6.29e-09 &  1.80e-16 &      \\ 
     &           &    8 &  6.29e-09 &  1.80e-16 &      \\ 
     &           &    9 &  7.14e-08 &  1.79e-16 &      \\ 
     &           &   10 &  4.52e-08 &  2.04e-15 &      \\ 
3753 &  3.75e+04 &   10 &           &           & iters  \\ 
 \hdashline 
     &           &    1 &  1.10e-08 &  3.75e-10 &      \\ 
     &           &    2 &  2.79e-08 &  3.14e-16 &      \\ 
     &           &    3 &  1.10e-08 &  7.95e-16 &      \\ 
     &           &    4 &  1.10e-08 &  3.14e-16 &      \\ 
     &           &    5 &  2.79e-08 &  3.14e-16 &      \\ 
     &           &    6 &  1.10e-08 &  7.95e-16 &      \\ 
     &           &    7 &  1.10e-08 &  3.15e-16 &      \\ 
     &           &    8 &  1.10e-08 &  3.14e-16 &      \\ 
     &           &    9 &  2.79e-08 &  3.14e-16 &      \\ 
     &           &   10 &  1.10e-08 &  7.95e-16 &      \\ 
3754 &  3.75e+04 &   10 &           &           & iters  \\ 
 \hdashline 
     &           &    1 &  1.25e-08 &  2.73e-10 &      \\ 
     &           &    2 &  1.25e-08 &  3.58e-16 &      \\ 
     &           &    3 &  6.52e-08 &  3.57e-16 &      \\ 
     &           &    4 &  1.26e-08 &  1.86e-15 &      \\ 
     &           &    5 &  1.26e-08 &  3.59e-16 &      \\ 
     &           &    6 &  1.26e-08 &  3.59e-16 &      \\ 
     &           &    7 &  1.25e-08 &  3.58e-16 &      \\ 
     &           &    8 &  1.25e-08 &  3.58e-16 &      \\ 
     &           &    9 &  6.52e-08 &  3.57e-16 &      \\ 
     &           &   10 &  1.26e-08 &  1.86e-15 &      \\ 
3755 &  3.75e+04 &   10 &           &           & iters  \\ 
 \hdashline 
     &           &    1 &  4.19e-08 &  5.41e-10 &      \\ 
     &           &    2 &  3.04e-09 &  1.20e-15 &      \\ 
     &           &    3 &  3.03e-09 &  8.67e-17 &      \\ 
     &           &    4 &  3.03e-09 &  8.65e-17 &      \\ 
     &           &    5 &  3.02e-09 &  8.64e-17 &      \\ 
     &           &    6 &  3.02e-09 &  8.63e-17 &      \\ 
     &           &    7 &  3.01e-09 &  8.62e-17 &      \\ 
     &           &    8 &  3.01e-09 &  8.60e-17 &      \\ 
     &           &    9 &  3.01e-09 &  8.59e-17 &      \\ 
     &           &   10 &  3.00e-09 &  8.58e-17 &      \\ 
3756 &  3.76e+04 &   10 &           &           & iters  \\ 
 \hdashline 
     &           &    1 &  5.70e-08 &  9.48e-10 &      \\ 
     &           &    2 &  1.80e-08 &  1.63e-15 &      \\ 
     &           &    3 &  5.97e-08 &  5.15e-16 &      \\ 
     &           &    4 &  1.81e-08 &  1.70e-15 &      \\ 
     &           &    5 &  1.81e-08 &  5.17e-16 &      \\ 
     &           &    6 &  1.81e-08 &  5.16e-16 &      \\ 
     &           &    7 &  1.80e-08 &  5.15e-16 &      \\ 
     &           &    8 &  5.97e-08 &  5.15e-16 &      \\ 
     &           &    9 &  1.81e-08 &  1.70e-15 &      \\ 
     &           &   10 &  1.81e-08 &  5.16e-16 &      \\ 
3757 &  3.76e+04 &   10 &           &           & iters  \\ 
 \hdashline 
     &           &    1 &  2.80e-08 &  1.07e-09 &      \\ 
     &           &    2 &  2.80e-08 &  8.00e-16 &      \\ 
     &           &    3 &  4.97e-08 &  7.99e-16 &      \\ 
     &           &    4 &  2.80e-08 &  1.42e-15 &      \\ 
     &           &    5 &  2.80e-08 &  8.00e-16 &      \\ 
     &           &    6 &  4.97e-08 &  7.99e-16 &      \\ 
     &           &    7 &  2.80e-08 &  1.42e-15 &      \\ 
     &           &    8 &  2.80e-08 &  8.00e-16 &      \\ 
     &           &    9 &  4.97e-08 &  7.99e-16 &      \\ 
     &           &   10 &  2.80e-08 &  1.42e-15 &      \\ 
3758 &  3.76e+04 &   10 &           &           & iters  \\ 
 \hdashline 
     &           &    1 &  7.52e-09 &  1.28e-09 &      \\ 
     &           &    2 &  7.51e-09 &  2.15e-16 &      \\ 
     &           &    3 &  7.50e-09 &  2.14e-16 &      \\ 
     &           &    4 &  7.49e-09 &  2.14e-16 &      \\ 
     &           &    5 &  7.02e-08 &  2.14e-16 &      \\ 
     &           &    6 &  7.58e-09 &  2.00e-15 &      \\ 
     &           &    7 &  7.57e-09 &  2.16e-16 &      \\ 
     &           &    8 &  7.56e-09 &  2.16e-16 &      \\ 
     &           &    9 &  7.55e-09 &  2.16e-16 &      \\ 
     &           &   10 &  7.54e-09 &  2.15e-16 &      \\ 
3759 &  3.76e+04 &   10 &           &           & iters  \\ 
 \hdashline 
     &           &    1 &  1.51e-08 &  9.32e-10 &      \\ 
     &           &    2 &  1.51e-08 &  4.31e-16 &      \\ 
     &           &    3 &  6.26e-08 &  4.30e-16 &      \\ 
     &           &    4 &  1.51e-08 &  1.79e-15 &      \\ 
     &           &    5 &  1.51e-08 &  4.32e-16 &      \\ 
     &           &    6 &  1.51e-08 &  4.31e-16 &      \\ 
     &           &    7 &  1.51e-08 &  4.31e-16 &      \\ 
     &           &    8 &  6.26e-08 &  4.30e-16 &      \\ 
     &           &    9 &  1.51e-08 &  1.79e-15 &      \\ 
     &           &   10 &  1.51e-08 &  4.32e-16 &      \\ 
3760 &  3.76e+04 &   10 &           &           & iters  \\ 
 \hdashline 
     &           &    1 &  1.64e-08 &  5.63e-10 &      \\ 
     &           &    2 &  6.13e-08 &  4.67e-16 &      \\ 
     &           &    3 &  1.64e-08 &  1.75e-15 &      \\ 
     &           &    4 &  1.64e-08 &  4.69e-16 &      \\ 
     &           &    5 &  1.64e-08 &  4.68e-16 &      \\ 
     &           &    6 &  6.13e-08 &  4.68e-16 &      \\ 
     &           &    7 &  1.65e-08 &  1.75e-15 &      \\ 
     &           &    8 &  1.64e-08 &  4.70e-16 &      \\ 
     &           &    9 &  1.64e-08 &  4.69e-16 &      \\ 
     &           &   10 &  1.64e-08 &  4.68e-16 &      \\ 
3761 &  3.76e+04 &   10 &           &           & iters  \\ 
 \hdashline 
     &           &    1 &  1.13e-08 &  2.37e-09 &      \\ 
     &           &    2 &  1.13e-08 &  3.22e-16 &      \\ 
     &           &    3 &  1.13e-08 &  3.22e-16 &      \\ 
     &           &    4 &  1.12e-08 &  3.21e-16 &      \\ 
     &           &    5 &  6.65e-08 &  3.21e-16 &      \\ 
     &           &    6 &  8.90e-08 &  1.90e-15 &      \\ 
     &           &    7 &  6.65e-08 &  2.54e-15 &      \\ 
     &           &    8 &  1.13e-08 &  1.90e-15 &      \\ 
     &           &    9 &  1.13e-08 &  3.22e-16 &      \\ 
     &           &   10 &  1.13e-08 &  3.22e-16 &      \\ 
3762 &  3.76e+04 &   10 &           &           & iters  \\ 
 \hdashline 
     &           &    1 &  1.62e-09 &  2.24e-09 &      \\ 
     &           &    2 &  1.62e-09 &  4.62e-17 &      \\ 
     &           &    3 &  1.61e-09 &  4.62e-17 &      \\ 
     &           &    4 &  1.61e-09 &  4.61e-17 &      \\ 
     &           &    5 &  1.61e-09 &  4.60e-17 &      \\ 
     &           &    6 &  1.61e-09 &  4.60e-17 &      \\ 
     &           &    7 &  1.61e-09 &  4.59e-17 &      \\ 
     &           &    8 &  1.60e-09 &  4.58e-17 &      \\ 
     &           &    9 &  1.60e-09 &  4.58e-17 &      \\ 
     &           &   10 &  1.60e-09 &  4.57e-17 &      \\ 
3763 &  3.76e+04 &   10 &           &           & iters  \\ 
 \hdashline 
     &           &    1 &  1.83e-08 &  2.56e-10 &      \\ 
     &           &    2 &  2.06e-08 &  5.22e-16 &      \\ 
     &           &    3 &  1.83e-08 &  5.87e-16 &      \\ 
     &           &    4 &  1.83e-08 &  5.22e-16 &      \\ 
     &           &    5 &  2.06e-08 &  5.22e-16 &      \\ 
     &           &    6 &  1.83e-08 &  5.88e-16 &      \\ 
     &           &    7 &  2.06e-08 &  5.22e-16 &      \\ 
     &           &    8 &  1.83e-08 &  5.88e-16 &      \\ 
     &           &    9 &  2.06e-08 &  5.22e-16 &      \\ 
     &           &   10 &  1.83e-08 &  5.88e-16 &      \\ 
3764 &  3.76e+04 &   10 &           &           & iters  \\ 
 \hdashline 
     &           &    1 &  9.20e-09 &  1.26e-09 &      \\ 
     &           &    2 &  9.19e-09 &  2.63e-16 &      \\ 
     &           &    3 &  9.18e-09 &  2.62e-16 &      \\ 
     &           &    4 &  9.16e-09 &  2.62e-16 &      \\ 
     &           &    5 &  9.15e-09 &  2.62e-16 &      \\ 
     &           &    6 &  6.86e-08 &  2.61e-16 &      \\ 
     &           &    7 &  4.81e-08 &  1.96e-15 &      \\ 
     &           &    8 &  9.17e-09 &  1.37e-15 &      \\ 
     &           &    9 &  9.15e-09 &  2.62e-16 &      \\ 
     &           &   10 &  9.14e-09 &  2.61e-16 &      \\ 
3765 &  3.76e+04 &   10 &           &           & iters  \\ 
 \hdashline 
     &           &    1 &  4.11e-09 &  4.86e-10 &      \\ 
     &           &    2 &  4.11e-09 &  1.17e-16 &      \\ 
     &           &    3 &  7.36e-08 &  1.17e-16 &      \\ 
     &           &    4 &  4.31e-08 &  2.10e-15 &      \\ 
     &           &    5 &  4.14e-09 &  1.23e-15 &      \\ 
     &           &    6 &  4.14e-09 &  1.18e-16 &      \\ 
     &           &    7 &  4.13e-09 &  1.18e-16 &      \\ 
     &           &    8 &  4.13e-09 &  1.18e-16 &      \\ 
     &           &    9 &  4.12e-09 &  1.18e-16 &      \\ 
     &           &   10 &  4.11e-09 &  1.18e-16 &      \\ 
3766 &  3.76e+04 &   10 &           &           & iters  \\ 
 \hdashline 
     &           &    1 &  3.21e-08 &  8.80e-10 &      \\ 
     &           &    2 &  3.21e-08 &  9.16e-16 &      \\ 
     &           &    3 &  4.57e-08 &  9.15e-16 &      \\ 
     &           &    4 &  3.21e-08 &  1.30e-15 &      \\ 
     &           &    5 &  4.57e-08 &  9.16e-16 &      \\ 
     &           &    6 &  3.21e-08 &  1.30e-15 &      \\ 
     &           &    7 &  3.21e-08 &  9.16e-16 &      \\ 
     &           &    8 &  4.57e-08 &  9.15e-16 &      \\ 
     &           &    9 &  3.21e-08 &  1.30e-15 &      \\ 
     &           &   10 &  4.57e-08 &  9.15e-16 &      \\ 
3767 &  3.77e+04 &   10 &           &           & iters  \\ 
 \hdashline 
     &           &    1 &  2.84e-08 &  6.18e-10 &      \\ 
     &           &    2 &  1.05e-08 &  8.09e-16 &      \\ 
     &           &    3 &  2.83e-08 &  3.00e-16 &      \\ 
     &           &    4 &  1.06e-08 &  8.09e-16 &      \\ 
     &           &    5 &  1.05e-08 &  3.01e-16 &      \\ 
     &           &    6 &  1.05e-08 &  3.01e-16 &      \\ 
     &           &    7 &  2.83e-08 &  3.00e-16 &      \\ 
     &           &    8 &  1.05e-08 &  8.09e-16 &      \\ 
     &           &    9 &  1.05e-08 &  3.01e-16 &      \\ 
     &           &   10 &  1.05e-08 &  3.01e-16 &      \\ 
3768 &  3.77e+04 &   10 &           &           & iters  \\ 
 \hdashline 
     &           &    1 &  4.30e-08 &  1.54e-09 &      \\ 
     &           &    2 &  3.47e-08 &  1.23e-15 &      \\ 
     &           &    3 &  4.30e-08 &  9.91e-16 &      \\ 
     &           &    4 &  3.47e-08 &  1.23e-15 &      \\ 
     &           &    5 &  4.30e-08 &  9.91e-16 &      \\ 
     &           &    6 &  3.47e-08 &  1.23e-15 &      \\ 
     &           &    7 &  3.47e-08 &  9.92e-16 &      \\ 
     &           &    8 &  4.30e-08 &  9.90e-16 &      \\ 
     &           &    9 &  3.47e-08 &  1.23e-15 &      \\ 
     &           &   10 &  4.30e-08 &  9.91e-16 &      \\ 
3769 &  3.77e+04 &   10 &           &           & iters  \\ 
 \hdashline 
     &           &    1 &  3.39e-08 &  8.86e-10 &      \\ 
     &           &    2 &  3.38e-08 &  9.67e-16 &      \\ 
     &           &    3 &  4.39e-08 &  9.66e-16 &      \\ 
     &           &    4 &  3.38e-08 &  1.25e-15 &      \\ 
     &           &    5 &  4.39e-08 &  9.66e-16 &      \\ 
     &           &    6 &  3.39e-08 &  1.25e-15 &      \\ 
     &           &    7 &  4.39e-08 &  9.66e-16 &      \\ 
     &           &    8 &  3.39e-08 &  1.25e-15 &      \\ 
     &           &    9 &  4.39e-08 &  9.67e-16 &      \\ 
     &           &   10 &  3.39e-08 &  1.25e-15 &      \\ 
3770 &  3.77e+04 &   10 &           &           & iters  \\ 
 \hdashline 
     &           &    1 &  1.44e-09 &  1.42e-09 &      \\ 
     &           &    2 &  1.44e-09 &  4.12e-17 &      \\ 
     &           &    3 &  1.44e-09 &  4.12e-17 &      \\ 
     &           &    4 &  1.44e-09 &  4.11e-17 &      \\ 
     &           &    5 &  1.44e-09 &  4.10e-17 &      \\ 
     &           &    6 &  1.43e-09 &  4.10e-17 &      \\ 
     &           &    7 &  1.43e-09 &  4.09e-17 &      \\ 
     &           &    8 &  1.43e-09 &  4.09e-17 &      \\ 
     &           &    9 &  1.43e-09 &  4.08e-17 &      \\ 
     &           &   10 &  1.43e-09 &  4.07e-17 &      \\ 
3771 &  3.77e+04 &   10 &           &           & iters  \\ 
 \hdashline 
     &           &    1 &  2.56e-08 &  2.13e-09 &      \\ 
     &           &    2 &  5.21e-08 &  7.31e-16 &      \\ 
     &           &    3 &  2.57e-08 &  1.49e-15 &      \\ 
     &           &    4 &  2.56e-08 &  7.33e-16 &      \\ 
     &           &    5 &  5.21e-08 &  7.32e-16 &      \\ 
     &           &    6 &  2.57e-08 &  1.49e-15 &      \\ 
     &           &    7 &  2.56e-08 &  7.33e-16 &      \\ 
     &           &    8 &  5.21e-08 &  7.32e-16 &      \\ 
     &           &    9 &  2.57e-08 &  1.49e-15 &      \\ 
     &           &   10 &  2.56e-08 &  7.33e-16 &      \\ 
3772 &  3.77e+04 &   10 &           &           & iters  \\ 
 \hdashline 
     &           &    1 &  2.96e-09 &  4.96e-11 &      \\ 
     &           &    2 &  2.95e-09 &  8.44e-17 &      \\ 
     &           &    3 &  2.95e-09 &  8.43e-17 &      \\ 
     &           &    4 &  2.94e-09 &  8.41e-17 &      \\ 
     &           &    5 &  2.94e-09 &  8.40e-17 &      \\ 
     &           &    6 &  2.94e-09 &  8.39e-17 &      \\ 
     &           &    7 &  2.93e-09 &  8.38e-17 &      \\ 
     &           &    8 &  2.93e-09 &  8.37e-17 &      \\ 
     &           &    9 &  2.92e-09 &  8.35e-17 &      \\ 
     &           &   10 &  2.92e-09 &  8.34e-17 &      \\ 
3773 &  3.77e+04 &   10 &           &           & iters  \\ 
 \hdashline 
     &           &    1 &  3.55e-08 &  1.19e-09 &      \\ 
     &           &    2 &  4.22e-08 &  1.01e-15 &      \\ 
     &           &    3 &  3.55e-08 &  1.20e-15 &      \\ 
     &           &    4 &  4.22e-08 &  1.01e-15 &      \\ 
     &           &    5 &  3.56e-08 &  1.20e-15 &      \\ 
     &           &    6 &  4.22e-08 &  1.01e-15 &      \\ 
     &           &    7 &  3.56e-08 &  1.20e-15 &      \\ 
     &           &    8 &  4.22e-08 &  1.02e-15 &      \\ 
     &           &    9 &  3.56e-08 &  1.20e-15 &      \\ 
     &           &   10 &  4.22e-08 &  1.02e-15 &      \\ 
3774 &  3.77e+04 &   10 &           &           & iters  \\ 
 \hdashline 
     &           &    1 &  2.00e-08 &  1.03e-09 &      \\ 
     &           &    2 &  2.00e-08 &  5.72e-16 &      \\ 
     &           &    3 &  2.00e-08 &  5.71e-16 &      \\ 
     &           &    4 &  5.77e-08 &  5.70e-16 &      \\ 
     &           &    5 &  2.00e-08 &  1.65e-15 &      \\ 
     &           &    6 &  2.00e-08 &  5.72e-16 &      \\ 
     &           &    7 &  2.00e-08 &  5.71e-16 &      \\ 
     &           &    8 &  5.77e-08 &  5.70e-16 &      \\ 
     &           &    9 &  2.00e-08 &  1.65e-15 &      \\ 
     &           &   10 &  2.00e-08 &  5.71e-16 &      \\ 
3775 &  3.77e+04 &   10 &           &           & iters  \\ 
 \hdashline 
     &           &    1 &  1.55e-08 &  1.00e-09 &      \\ 
     &           &    2 &  1.55e-08 &  4.44e-16 &      \\ 
     &           &    3 &  6.22e-08 &  4.43e-16 &      \\ 
     &           &    4 &  1.56e-08 &  1.77e-15 &      \\ 
     &           &    5 &  1.56e-08 &  4.45e-16 &      \\ 
     &           &    6 &  1.55e-08 &  4.44e-16 &      \\ 
     &           &    7 &  1.55e-08 &  4.44e-16 &      \\ 
     &           &    8 &  6.22e-08 &  4.43e-16 &      \\ 
     &           &    9 &  1.56e-08 &  1.77e-15 &      \\ 
     &           &   10 &  1.56e-08 &  4.45e-16 &      \\ 
3776 &  3.78e+04 &   10 &           &           & iters  \\ 
 \hdashline 
     &           &    1 &  3.93e-08 &  3.78e-11 &      \\ 
     &           &    2 &  3.85e-08 &  1.12e-15 &      \\ 
     &           &    3 &  3.93e-08 &  1.10e-15 &      \\ 
     &           &    4 &  3.85e-08 &  1.12e-15 &      \\ 
     &           &    5 &  3.93e-08 &  1.10e-15 &      \\ 
     &           &    6 &  3.85e-08 &  1.12e-15 &      \\ 
     &           &    7 &  3.93e-08 &  1.10e-15 &      \\ 
     &           &    8 &  3.85e-08 &  1.12e-15 &      \\ 
     &           &    9 &  3.93e-08 &  1.10e-15 &      \\ 
     &           &   10 &  3.85e-08 &  1.12e-15 &      \\ 
3777 &  3.78e+04 &   10 &           &           & iters  \\ 
 \hdashline 
     &           &    1 &  5.81e-08 &  6.29e-10 &      \\ 
     &           &    2 &  1.97e-08 &  1.66e-15 &      \\ 
     &           &    3 &  1.97e-08 &  5.62e-16 &      \\ 
     &           &    4 &  1.96e-08 &  5.62e-16 &      \\ 
     &           &    5 &  5.81e-08 &  5.61e-16 &      \\ 
     &           &    6 &  1.97e-08 &  1.66e-15 &      \\ 
     &           &    7 &  1.97e-08 &  5.62e-16 &      \\ 
     &           &    8 &  1.96e-08 &  5.61e-16 &      \\ 
     &           &    9 &  5.81e-08 &  5.61e-16 &      \\ 
     &           &   10 &  1.97e-08 &  1.66e-15 &      \\ 
3778 &  3.78e+04 &   10 &           &           & iters  \\ 
 \hdashline 
     &           &    1 &  4.09e-08 &  2.41e-10 &      \\ 
     &           &    2 &  3.69e-08 &  1.17e-15 &      \\ 
     &           &    3 &  4.09e-08 &  1.05e-15 &      \\ 
     &           &    4 &  3.69e-08 &  1.17e-15 &      \\ 
     &           &    5 &  4.09e-08 &  1.05e-15 &      \\ 
     &           &    6 &  3.69e-08 &  1.17e-15 &      \\ 
     &           &    7 &  4.09e-08 &  1.05e-15 &      \\ 
     &           &    8 &  3.69e-08 &  1.17e-15 &      \\ 
     &           &    9 &  4.09e-08 &  1.05e-15 &      \\ 
     &           &   10 &  3.69e-08 &  1.17e-15 &      \\ 
3779 &  3.78e+04 &   10 &           &           & iters  \\ 
 \hdashline 
     &           &    1 &  4.90e-08 &  6.95e-10 &      \\ 
     &           &    2 &  2.88e-08 &  1.40e-15 &      \\ 
     &           &    3 &  2.88e-08 &  8.22e-16 &      \\ 
     &           &    4 &  4.90e-08 &  8.21e-16 &      \\ 
     &           &    5 &  2.88e-08 &  1.40e-15 &      \\ 
     &           &    6 &  2.87e-08 &  8.22e-16 &      \\ 
     &           &    7 &  4.90e-08 &  8.20e-16 &      \\ 
     &           &    8 &  2.88e-08 &  1.40e-15 &      \\ 
     &           &    9 &  4.90e-08 &  8.21e-16 &      \\ 
     &           &   10 &  2.88e-08 &  1.40e-15 &      \\ 
3780 &  3.78e+04 &   10 &           &           & iters  \\ 
 \hdashline 
     &           &    1 &  3.23e-08 &  1.28e-10 &      \\ 
     &           &    2 &  4.54e-08 &  9.23e-16 &      \\ 
     &           &    3 &  3.24e-08 &  1.30e-15 &      \\ 
     &           &    4 &  4.54e-08 &  9.24e-16 &      \\ 
     &           &    5 &  3.24e-08 &  1.29e-15 &      \\ 
     &           &    6 &  3.23e-08 &  9.24e-16 &      \\ 
     &           &    7 &  4.54e-08 &  9.23e-16 &      \\ 
     &           &    8 &  3.24e-08 &  1.30e-15 &      \\ 
     &           &    9 &  4.54e-08 &  9.24e-16 &      \\ 
     &           &   10 &  3.24e-08 &  1.30e-15 &      \\ 
3781 &  3.78e+04 &   10 &           &           & iters  \\ 
 \hdashline 
     &           &    1 &  8.53e-09 &  4.49e-10 &      \\ 
     &           &    2 &  8.51e-09 &  2.43e-16 &      \\ 
     &           &    3 &  8.50e-09 &  2.43e-16 &      \\ 
     &           &    4 &  8.49e-09 &  2.43e-16 &      \\ 
     &           &    5 &  8.48e-09 &  2.42e-16 &      \\ 
     &           &    6 &  6.92e-08 &  2.42e-16 &      \\ 
     &           &    7 &  8.63e-08 &  1.98e-15 &      \\ 
     &           &    8 &  6.92e-08 &  2.46e-15 &      \\ 
     &           &    9 &  8.54e-09 &  1.98e-15 &      \\ 
     &           &   10 &  8.53e-09 &  2.44e-16 &      \\ 
3782 &  3.78e+04 &   10 &           &           & iters  \\ 
 \hdashline 
     &           &    1 &  3.39e-08 &  7.85e-10 &      \\ 
     &           &    2 &  4.38e-08 &  9.68e-16 &      \\ 
     &           &    3 &  3.39e-08 &  1.25e-15 &      \\ 
     &           &    4 &  4.38e-08 &  9.68e-16 &      \\ 
     &           &    5 &  3.39e-08 &  1.25e-15 &      \\ 
     &           &    6 &  4.38e-08 &  9.69e-16 &      \\ 
     &           &    7 &  3.40e-08 &  1.25e-15 &      \\ 
     &           &    8 &  4.38e-08 &  9.69e-16 &      \\ 
     &           &    9 &  3.40e-08 &  1.25e-15 &      \\ 
     &           &   10 &  3.39e-08 &  9.70e-16 &      \\ 
3783 &  3.78e+04 &   10 &           &           & iters  \\ 
 \hdashline 
     &           &    1 &  2.45e-08 &  6.06e-10 &      \\ 
     &           &    2 &  2.45e-08 &  6.99e-16 &      \\ 
     &           &    3 &  5.33e-08 &  6.98e-16 &      \\ 
     &           &    4 &  2.45e-08 &  1.52e-15 &      \\ 
     &           &    5 &  2.45e-08 &  6.99e-16 &      \\ 
     &           &    6 &  5.33e-08 &  6.98e-16 &      \\ 
     &           &    7 &  2.45e-08 &  1.52e-15 &      \\ 
     &           &    8 &  2.45e-08 &  6.99e-16 &      \\ 
     &           &    9 &  5.33e-08 &  6.98e-16 &      \\ 
     &           &   10 &  2.45e-08 &  1.52e-15 &      \\ 
3784 &  3.78e+04 &   10 &           &           & iters  \\ 
 \hdashline 
     &           &    1 &  5.51e-08 &  7.97e-11 &      \\ 
     &           &    2 &  2.27e-08 &  1.57e-15 &      \\ 
     &           &    3 &  2.27e-08 &  6.48e-16 &      \\ 
     &           &    4 &  2.26e-08 &  6.47e-16 &      \\ 
     &           &    5 &  5.51e-08 &  6.46e-16 &      \\ 
     &           &    6 &  2.27e-08 &  1.57e-15 &      \\ 
     &           &    7 &  2.26e-08 &  6.47e-16 &      \\ 
     &           &    8 &  5.51e-08 &  6.46e-16 &      \\ 
     &           &    9 &  2.27e-08 &  1.57e-15 &      \\ 
     &           &   10 &  2.27e-08 &  6.48e-16 &      \\ 
3785 &  3.78e+04 &   10 &           &           & iters  \\ 
 \hdashline 
     &           &    1 &  2.83e-08 &  2.27e-10 &      \\ 
     &           &    2 &  1.06e-08 &  8.06e-16 &      \\ 
     &           &    3 &  1.06e-08 &  3.03e-16 &      \\ 
     &           &    4 &  1.06e-08 &  3.03e-16 &      \\ 
     &           &    5 &  2.83e-08 &  3.02e-16 &      \\ 
     &           &    6 &  1.06e-08 &  8.07e-16 &      \\ 
     &           &    7 &  1.06e-08 &  3.03e-16 &      \\ 
     &           &    8 &  1.06e-08 &  3.03e-16 &      \\ 
     &           &    9 &  2.83e-08 &  3.02e-16 &      \\ 
     &           &   10 &  1.06e-08 &  8.07e-16 &      \\ 
3786 &  3.78e+04 &   10 &           &           & iters  \\ 
 \hdashline 
     &           &    1 &  1.52e-08 &  1.38e-10 &      \\ 
     &           &    2 &  1.52e-08 &  4.34e-16 &      \\ 
     &           &    3 &  6.25e-08 &  4.33e-16 &      \\ 
     &           &    4 &  1.52e-08 &  1.78e-15 &      \\ 
     &           &    5 &  1.52e-08 &  4.35e-16 &      \\ 
     &           &    6 &  1.52e-08 &  4.34e-16 &      \\ 
     &           &    7 &  1.52e-08 &  4.34e-16 &      \\ 
     &           &    8 &  6.25e-08 &  4.33e-16 &      \\ 
     &           &    9 &  1.52e-08 &  1.78e-15 &      \\ 
     &           &   10 &  1.52e-08 &  4.35e-16 &      \\ 
3787 &  3.79e+04 &   10 &           &           & iters  \\ 
 \hdashline 
     &           &    1 &  6.78e-09 &  5.17e-10 &      \\ 
     &           &    2 &  6.77e-09 &  1.94e-16 &      \\ 
     &           &    3 &  3.21e-08 &  1.93e-16 &      \\ 
     &           &    4 &  6.81e-09 &  9.16e-16 &      \\ 
     &           &    5 &  6.80e-09 &  1.94e-16 &      \\ 
     &           &    6 &  6.79e-09 &  1.94e-16 &      \\ 
     &           &    7 &  6.78e-09 &  1.94e-16 &      \\ 
     &           &    8 &  3.21e-08 &  1.93e-16 &      \\ 
     &           &    9 &  6.81e-09 &  9.15e-16 &      \\ 
     &           &   10 &  6.80e-09 &  1.94e-16 &      \\ 
3788 &  3.79e+04 &   10 &           &           & iters  \\ 
 \hdashline 
     &           &    1 &  1.68e-08 &  1.18e-10 &      \\ 
     &           &    2 &  2.21e-08 &  4.80e-16 &      \\ 
     &           &    3 &  1.68e-08 &  6.29e-16 &      \\ 
     &           &    4 &  2.20e-08 &  4.80e-16 &      \\ 
     &           &    5 &  1.68e-08 &  6.29e-16 &      \\ 
     &           &    6 &  1.68e-08 &  4.80e-16 &      \\ 
     &           &    7 &  2.21e-08 &  4.80e-16 &      \\ 
     &           &    8 &  1.68e-08 &  6.30e-16 &      \\ 
     &           &    9 &  2.21e-08 &  4.80e-16 &      \\ 
     &           &   10 &  1.68e-08 &  6.29e-16 &      \\ 
3789 &  3.79e+04 &   10 &           &           & iters  \\ 
 \hdashline 
     &           &    1 &  4.37e-08 &  2.80e-10 &      \\ 
     &           &    2 &  3.41e-08 &  1.25e-15 &      \\ 
     &           &    3 &  3.40e-08 &  9.73e-16 &      \\ 
     &           &    4 &  4.37e-08 &  9.72e-16 &      \\ 
     &           &    5 &  3.41e-08 &  1.25e-15 &      \\ 
     &           &    6 &  4.37e-08 &  9.72e-16 &      \\ 
     &           &    7 &  3.41e-08 &  1.25e-15 &      \\ 
     &           &    8 &  4.37e-08 &  9.72e-16 &      \\ 
     &           &    9 &  3.41e-08 &  1.25e-15 &      \\ 
     &           &   10 &  4.37e-08 &  9.73e-16 &      \\ 
3790 &  3.79e+04 &   10 &           &           & iters  \\ 
 \hdashline 
     &           &    1 &  7.64e-08 &  2.27e-10 &      \\ 
     &           &    2 &  4.03e-08 &  2.18e-15 &      \\ 
     &           &    3 &  1.35e-09 &  1.15e-15 &      \\ 
     &           &    4 &  1.35e-09 &  3.86e-17 &      \\ 
     &           &    5 &  1.35e-09 &  3.86e-17 &      \\ 
     &           &    6 &  1.35e-09 &  3.85e-17 &      \\ 
     &           &    7 &  1.35e-09 &  3.85e-17 &      \\ 
     &           &    8 &  1.34e-09 &  3.84e-17 &      \\ 
     &           &    9 &  1.34e-09 &  3.83e-17 &      \\ 
     &           &   10 &  1.34e-09 &  3.83e-17 &      \\ 
3791 &  3.79e+04 &   10 &           &           & iters  \\ 
 \hdashline 
     &           &    1 &  5.72e-08 &  7.67e-10 &      \\ 
     &           &    2 &  2.05e-08 &  1.63e-15 &      \\ 
     &           &    3 &  2.05e-08 &  5.86e-16 &      \\ 
     &           &    4 &  2.05e-08 &  5.85e-16 &      \\ 
     &           &    5 &  5.72e-08 &  5.85e-16 &      \\ 
     &           &    6 &  2.05e-08 &  1.63e-15 &      \\ 
     &           &    7 &  2.05e-08 &  5.86e-16 &      \\ 
     &           &    8 &  2.05e-08 &  5.85e-16 &      \\ 
     &           &    9 &  5.72e-08 &  5.84e-16 &      \\ 
     &           &   10 &  2.05e-08 &  1.63e-15 &      \\ 
3792 &  3.79e+04 &   10 &           &           & iters  \\ 
 \hdashline 
     &           &    1 &  2.20e-08 &  6.73e-10 &      \\ 
     &           &    2 &  5.57e-08 &  6.27e-16 &      \\ 
     &           &    3 &  2.20e-08 &  1.59e-15 &      \\ 
     &           &    4 &  2.20e-08 &  6.29e-16 &      \\ 
     &           &    5 &  2.20e-08 &  6.28e-16 &      \\ 
     &           &    6 &  5.58e-08 &  6.27e-16 &      \\ 
     &           &    7 &  2.20e-08 &  1.59e-15 &      \\ 
     &           &    8 &  2.20e-08 &  6.28e-16 &      \\ 
     &           &    9 &  5.57e-08 &  6.27e-16 &      \\ 
     &           &   10 &  2.20e-08 &  1.59e-15 &      \\ 
3793 &  3.79e+04 &   10 &           &           & iters  \\ 
 \hdashline 
     &           &    1 &  2.34e-08 &  2.28e-10 &      \\ 
     &           &    2 &  2.33e-08 &  6.67e-16 &      \\ 
     &           &    3 &  5.44e-08 &  6.66e-16 &      \\ 
     &           &    4 &  2.34e-08 &  1.55e-15 &      \\ 
     &           &    5 &  2.33e-08 &  6.67e-16 &      \\ 
     &           &    6 &  5.44e-08 &  6.66e-16 &      \\ 
     &           &    7 &  2.34e-08 &  1.55e-15 &      \\ 
     &           &    8 &  2.34e-08 &  6.67e-16 &      \\ 
     &           &    9 &  2.33e-08 &  6.66e-16 &      \\ 
     &           &   10 &  5.44e-08 &  6.65e-16 &      \\ 
3794 &  3.79e+04 &   10 &           &           & iters  \\ 
 \hdashline 
     &           &    1 &  3.90e-08 &  3.08e-10 &      \\ 
     &           &    2 &  3.87e-08 &  1.11e-15 &      \\ 
     &           &    3 &  3.90e-08 &  1.11e-15 &      \\ 
     &           &    4 &  3.87e-08 &  1.11e-15 &      \\ 
     &           &    5 &  3.90e-08 &  1.11e-15 &      \\ 
     &           &    6 &  3.87e-08 &  1.11e-15 &      \\ 
     &           &    7 &  3.90e-08 &  1.11e-15 &      \\ 
     &           &    8 &  3.87e-08 &  1.11e-15 &      \\ 
     &           &    9 &  3.90e-08 &  1.11e-15 &      \\ 
     &           &   10 &  3.87e-08 &  1.11e-15 &      \\ 
3795 &  3.79e+04 &   10 &           &           & iters  \\ 
 \hdashline 
     &           &    1 &  1.20e-08 &  3.06e-10 &      \\ 
     &           &    2 &  1.20e-08 &  3.43e-16 &      \\ 
     &           &    3 &  1.20e-08 &  3.43e-16 &      \\ 
     &           &    4 &  2.69e-08 &  3.42e-16 &      \\ 
     &           &    5 &  1.20e-08 &  7.67e-16 &      \\ 
     &           &    6 &  1.20e-08 &  3.43e-16 &      \\ 
     &           &    7 &  2.69e-08 &  3.42e-16 &      \\ 
     &           &    8 &  1.20e-08 &  7.67e-16 &      \\ 
     &           &    9 &  1.20e-08 &  3.43e-16 &      \\ 
     &           &   10 &  1.20e-08 &  3.42e-16 &      \\ 
3796 &  3.80e+04 &   10 &           &           & iters  \\ 
 \hdashline 
     &           &    1 &  3.47e-09 &  1.55e-10 &      \\ 
     &           &    2 &  3.47e-09 &  9.91e-17 &      \\ 
     &           &    3 &  3.46e-09 &  9.89e-17 &      \\ 
     &           &    4 &  3.46e-09 &  9.88e-17 &      \\ 
     &           &    5 &  3.45e-09 &  9.86e-17 &      \\ 
     &           &    6 &  3.45e-09 &  9.85e-17 &      \\ 
     &           &    7 &  3.44e-09 &  9.84e-17 &      \\ 
     &           &    8 &  3.44e-09 &  9.82e-17 &      \\ 
     &           &    9 &  3.43e-09 &  9.81e-17 &      \\ 
     &           &   10 &  3.43e-09 &  9.79e-17 &      \\ 
3797 &  3.80e+04 &   10 &           &           & iters  \\ 
 \hdashline 
     &           &    1 &  1.18e-09 &  8.79e-10 &      \\ 
     &           &    2 &  1.17e-09 &  3.36e-17 &      \\ 
     &           &    3 &  1.17e-09 &  3.35e-17 &      \\ 
     &           &    4 &  1.17e-09 &  3.35e-17 &      \\ 
     &           &    5 &  1.17e-09 &  3.34e-17 &      \\ 
     &           &    6 &  1.17e-09 &  3.34e-17 &      \\ 
     &           &    7 &  1.17e-09 &  3.33e-17 &      \\ 
     &           &    8 &  1.16e-09 &  3.33e-17 &      \\ 
     &           &    9 &  1.16e-09 &  3.32e-17 &      \\ 
     &           &   10 &  3.77e-08 &  3.32e-17 &      \\ 
3798 &  3.80e+04 &   10 &           &           & iters  \\ 
 \hdashline 
     &           &    1 &  4.24e-08 &  1.52e-09 &      \\ 
     &           &    2 &  3.53e-08 &  1.21e-15 &      \\ 
     &           &    3 &  3.53e-08 &  1.01e-15 &      \\ 
     &           &    4 &  4.25e-08 &  1.01e-15 &      \\ 
     &           &    5 &  3.53e-08 &  1.21e-15 &      \\ 
     &           &    6 &  4.24e-08 &  1.01e-15 &      \\ 
     &           &    7 &  3.53e-08 &  1.21e-15 &      \\ 
     &           &    8 &  4.24e-08 &  1.01e-15 &      \\ 
     &           &    9 &  3.53e-08 &  1.21e-15 &      \\ 
     &           &   10 &  4.24e-08 &  1.01e-15 &      \\ 
3799 &  3.80e+04 &   10 &           &           & iters  \\ 
 \hdashline 
     &           &    1 &  3.03e-08 &  8.93e-10 &      \\ 
     &           &    2 &  4.75e-08 &  8.63e-16 &      \\ 
     &           &    3 &  3.03e-08 &  1.36e-15 &      \\ 
     &           &    4 &  3.02e-08 &  8.64e-16 &      \\ 
     &           &    5 &  4.75e-08 &  8.63e-16 &      \\ 
     &           &    6 &  3.03e-08 &  1.36e-15 &      \\ 
     &           &    7 &  4.75e-08 &  8.64e-16 &      \\ 
     &           &    8 &  3.03e-08 &  1.35e-15 &      \\ 
     &           &    9 &  3.02e-08 &  8.64e-16 &      \\ 
     &           &   10 &  4.75e-08 &  8.63e-16 &      \\ 
3800 &  3.80e+04 &   10 &           &           & iters  \\ 
 \hdashline 
     &           &    1 &  4.14e-08 &  2.99e-10 &      \\ 
     &           &    2 &  3.63e-08 &  1.18e-15 &      \\ 
     &           &    3 &  4.14e-08 &  1.04e-15 &      \\ 
     &           &    4 &  3.64e-08 &  1.18e-15 &      \\ 
     &           &    5 &  4.14e-08 &  1.04e-15 &      \\ 
     &           &    6 &  3.64e-08 &  1.18e-15 &      \\ 
     &           &    7 &  4.14e-08 &  1.04e-15 &      \\ 
     &           &    8 &  3.64e-08 &  1.18e-15 &      \\ 
     &           &    9 &  4.14e-08 &  1.04e-15 &      \\ 
     &           &   10 &  3.64e-08 &  1.18e-15 &      \\ 
3801 &  3.80e+04 &   10 &           &           & iters  \\ 
 \hdashline 
     &           &    1 &  2.52e-08 &  6.03e-10 &      \\ 
     &           &    2 &  5.25e-08 &  7.20e-16 &      \\ 
     &           &    3 &  2.53e-08 &  1.50e-15 &      \\ 
     &           &    4 &  2.52e-08 &  7.21e-16 &      \\ 
     &           &    5 &  5.25e-08 &  7.20e-16 &      \\ 
     &           &    6 &  2.53e-08 &  1.50e-15 &      \\ 
     &           &    7 &  2.52e-08 &  7.21e-16 &      \\ 
     &           &    8 &  5.25e-08 &  7.20e-16 &      \\ 
     &           &    9 &  2.53e-08 &  1.50e-15 &      \\ 
     &           &   10 &  2.52e-08 &  7.21e-16 &      \\ 
3802 &  3.80e+04 &   10 &           &           & iters  \\ 
 \hdashline 
     &           &    1 &  6.50e-08 &  7.03e-10 &      \\ 
     &           &    2 &  1.27e-08 &  1.86e-15 &      \\ 
     &           &    3 &  1.27e-08 &  3.63e-16 &      \\ 
     &           &    4 &  1.27e-08 &  3.63e-16 &      \\ 
     &           &    5 &  1.27e-08 &  3.62e-16 &      \\ 
     &           &    6 &  1.27e-08 &  3.62e-16 &      \\ 
     &           &    7 &  6.50e-08 &  3.61e-16 &      \\ 
     &           &    8 &  1.27e-08 &  1.86e-15 &      \\ 
     &           &    9 &  1.27e-08 &  3.64e-16 &      \\ 
     &           &   10 &  1.27e-08 &  3.63e-16 &      \\ 
3803 &  3.80e+04 &   10 &           &           & iters  \\ 
 \hdashline 
     &           &    1 &  1.73e-09 &  1.85e-10 &      \\ 
     &           &    2 &  1.73e-09 &  4.94e-17 &      \\ 
     &           &    3 &  1.73e-09 &  4.94e-17 &      \\ 
     &           &    4 &  1.72e-09 &  4.93e-17 &      \\ 
     &           &    5 &  1.72e-09 &  4.92e-17 &      \\ 
     &           &    6 &  1.72e-09 &  4.91e-17 &      \\ 
     &           &    7 &  7.60e-08 &  4.91e-17 &      \\ 
     &           &    8 &  7.95e-08 &  2.17e-15 &      \\ 
     &           &    9 &  7.60e-08 &  2.27e-15 &      \\ 
     &           &   10 &  7.95e-08 &  2.17e-15 &      \\ 
3804 &  3.80e+04 &   10 &           &           & iters  \\ 
 \hdashline 
     &           &    1 &  1.77e-08 &  5.80e-10 &      \\ 
     &           &    2 &  1.77e-08 &  5.06e-16 &      \\ 
     &           &    3 &  1.77e-08 &  5.06e-16 &      \\ 
     &           &    4 &  1.77e-08 &  5.05e-16 &      \\ 
     &           &    5 &  6.00e-08 &  5.04e-16 &      \\ 
     &           &    6 &  1.77e-08 &  1.71e-15 &      \\ 
     &           &    7 &  1.77e-08 &  5.06e-16 &      \\ 
     &           &    8 &  1.77e-08 &  5.05e-16 &      \\ 
     &           &    9 &  1.77e-08 &  5.05e-16 &      \\ 
     &           &   10 &  6.01e-08 &  5.04e-16 &      \\ 
3805 &  3.80e+04 &   10 &           &           & iters  \\ 
 \hdashline 
     &           &    1 &  2.20e-08 &  7.84e-10 &      \\ 
     &           &    2 &  2.19e-08 &  6.27e-16 &      \\ 
     &           &    3 &  5.58e-08 &  6.26e-16 &      \\ 
     &           &    4 &  2.20e-08 &  1.59e-15 &      \\ 
     &           &    5 &  2.20e-08 &  6.28e-16 &      \\ 
     &           &    6 &  2.19e-08 &  6.27e-16 &      \\ 
     &           &    7 &  5.58e-08 &  6.26e-16 &      \\ 
     &           &    8 &  2.20e-08 &  1.59e-15 &      \\ 
     &           &    9 &  2.19e-08 &  6.27e-16 &      \\ 
     &           &   10 &  5.58e-08 &  6.26e-16 &      \\ 
3806 &  3.80e+04 &   10 &           &           & iters  \\ 
 \hdashline 
     &           &    1 &  2.99e-08 &  4.73e-10 &      \\ 
     &           &    2 &  2.98e-08 &  8.53e-16 &      \\ 
     &           &    3 &  4.79e-08 &  8.52e-16 &      \\ 
     &           &    4 &  2.99e-08 &  1.37e-15 &      \\ 
     &           &    5 &  2.98e-08 &  8.53e-16 &      \\ 
     &           &    6 &  4.79e-08 &  8.51e-16 &      \\ 
     &           &    7 &  2.99e-08 &  1.37e-15 &      \\ 
     &           &    8 &  4.79e-08 &  8.52e-16 &      \\ 
     &           &    9 &  2.99e-08 &  1.37e-15 &      \\ 
     &           &   10 &  2.98e-08 &  8.53e-16 &      \\ 
3807 &  3.81e+04 &   10 &           &           & iters  \\ 
 \hdashline 
     &           &    1 &  9.70e-08 &  3.53e-10 &      \\ 
     &           &    2 &  5.86e-08 &  2.77e-15 &      \\ 
     &           &    3 &  1.92e-08 &  1.67e-15 &      \\ 
     &           &    4 &  1.92e-08 &  5.49e-16 &      \\ 
     &           &    5 &  5.85e-08 &  5.48e-16 &      \\ 
     &           &    6 &  1.92e-08 &  1.67e-15 &      \\ 
     &           &    7 &  1.92e-08 &  5.49e-16 &      \\ 
     &           &    8 &  1.92e-08 &  5.49e-16 &      \\ 
     &           &    9 &  5.85e-08 &  5.48e-16 &      \\ 
     &           &   10 &  1.93e-08 &  1.67e-15 &      \\ 
3808 &  3.81e+04 &   10 &           &           & iters  \\ 
 \hdashline 
     &           &    1 &  4.45e-08 &  3.88e-10 &      \\ 
     &           &    2 &  3.33e-08 &  1.27e-15 &      \\ 
     &           &    3 &  4.45e-08 &  9.49e-16 &      \\ 
     &           &    4 &  3.33e-08 &  1.27e-15 &      \\ 
     &           &    5 &  3.32e-08 &  9.50e-16 &      \\ 
     &           &    6 &  4.45e-08 &  9.48e-16 &      \\ 
     &           &    7 &  3.32e-08 &  1.27e-15 &      \\ 
     &           &    8 &  4.45e-08 &  9.49e-16 &      \\ 
     &           &    9 &  3.33e-08 &  1.27e-15 &      \\ 
     &           &   10 &  4.45e-08 &  9.49e-16 &      \\ 
3809 &  3.81e+04 &   10 &           &           & iters  \\ 
 \hdashline 
     &           &    1 &  4.95e-08 &  4.16e-10 &      \\ 
     &           &    2 &  1.05e-08 &  1.41e-15 &      \\ 
     &           &    3 &  1.05e-08 &  3.01e-16 &      \\ 
     &           &    4 &  1.05e-08 &  3.01e-16 &      \\ 
     &           &    5 &  6.72e-08 &  3.00e-16 &      \\ 
     &           &    6 &  4.94e-08 &  1.92e-15 &      \\ 
     &           &    7 &  1.05e-08 &  1.41e-15 &      \\ 
     &           &    8 &  1.05e-08 &  3.01e-16 &      \\ 
     &           &    9 &  6.72e-08 &  3.00e-16 &      \\ 
     &           &   10 &  1.06e-08 &  1.92e-15 &      \\ 
3810 &  3.81e+04 &   10 &           &           & iters  \\ 
 \hdashline 
     &           &    1 &  2.49e-08 &  3.42e-10 &      \\ 
     &           &    2 &  2.49e-08 &  7.12e-16 &      \\ 
     &           &    3 &  5.28e-08 &  7.11e-16 &      \\ 
     &           &    4 &  2.49e-08 &  1.51e-15 &      \\ 
     &           &    5 &  2.49e-08 &  7.12e-16 &      \\ 
     &           &    6 &  5.28e-08 &  7.11e-16 &      \\ 
     &           &    7 &  2.49e-08 &  1.51e-15 &      \\ 
     &           &    8 &  2.49e-08 &  7.12e-16 &      \\ 
     &           &    9 &  5.28e-08 &  7.11e-16 &      \\ 
     &           &   10 &  2.50e-08 &  1.51e-15 &      \\ 
3811 &  3.81e+04 &   10 &           &           & iters  \\ 
 \hdashline 
     &           &    1 &  2.75e-08 &  1.55e-10 &      \\ 
     &           &    2 &  5.03e-08 &  7.84e-16 &      \\ 
     &           &    3 &  2.75e-08 &  1.43e-15 &      \\ 
     &           &    4 &  2.74e-08 &  7.84e-16 &      \\ 
     &           &    5 &  5.03e-08 &  7.83e-16 &      \\ 
     &           &    6 &  2.75e-08 &  1.44e-15 &      \\ 
     &           &    7 &  5.02e-08 &  7.84e-16 &      \\ 
     &           &    8 &  2.75e-08 &  1.43e-15 &      \\ 
     &           &    9 &  2.75e-08 &  7.85e-16 &      \\ 
     &           &   10 &  5.03e-08 &  7.84e-16 &      \\ 
3812 &  3.81e+04 &   10 &           &           & iters  \\ 
 \hdashline 
     &           &    1 &  3.58e-08 &  6.08e-10 &      \\ 
     &           &    2 &  3.58e-08 &  1.02e-15 &      \\ 
     &           &    3 &  4.19e-08 &  1.02e-15 &      \\ 
     &           &    4 &  3.58e-08 &  1.20e-15 &      \\ 
     &           &    5 &  4.19e-08 &  1.02e-15 &      \\ 
     &           &    6 &  3.58e-08 &  1.20e-15 &      \\ 
     &           &    7 &  4.19e-08 &  1.02e-15 &      \\ 
     &           &    8 &  3.58e-08 &  1.20e-15 &      \\ 
     &           &    9 &  4.19e-08 &  1.02e-15 &      \\ 
     &           &   10 &  3.58e-08 &  1.20e-15 &      \\ 
3813 &  3.81e+04 &   10 &           &           & iters  \\ 
 \hdashline 
     &           &    1 &  3.45e-08 &  5.16e-10 &      \\ 
     &           &    2 &  4.32e-08 &  9.84e-16 &      \\ 
     &           &    3 &  3.45e-08 &  1.23e-15 &      \\ 
     &           &    4 &  4.32e-08 &  9.85e-16 &      \\ 
     &           &    5 &  3.45e-08 &  1.23e-15 &      \\ 
     &           &    6 &  3.45e-08 &  9.85e-16 &      \\ 
     &           &    7 &  4.33e-08 &  9.84e-16 &      \\ 
     &           &    8 &  3.45e-08 &  1.24e-15 &      \\ 
     &           &    9 &  4.33e-08 &  9.84e-16 &      \\ 
     &           &   10 &  3.45e-08 &  1.23e-15 &      \\ 
3814 &  3.81e+04 &   10 &           &           & iters  \\ 
 \hdashline 
     &           &    1 &  5.43e-11 &  2.56e-10 &      \\ 
     &           &    2 &  5.42e-11 &  1.55e-18 &      \\ 
     &           &    3 &  5.42e-11 &  1.55e-18 &      \\ 
     &           &    4 &  5.41e-11 &  1.55e-18 &      \\ 
     &           &    5 &  5.40e-11 &  1.54e-18 &      \\ 
     &           &    6 &  5.39e-11 &  1.54e-18 &      \\ 
     &           &    7 &  5.38e-11 &  1.54e-18 &      \\ 
     &           &    8 &  5.38e-11 &  1.54e-18 &      \\ 
     &           &    9 &  5.37e-11 &  1.53e-18 &      \\ 
     &           &   10 &  5.36e-11 &  1.53e-18 &      \\ 
3815 &  3.81e+04 &   10 &           &           & iters  \\ 
 \hdashline 
     &           &    1 &  2.49e-08 &  2.60e-11 &      \\ 
     &           &    2 &  2.48e-08 &  7.10e-16 &      \\ 
     &           &    3 &  2.48e-08 &  7.08e-16 &      \\ 
     &           &    4 &  2.48e-08 &  7.07e-16 &      \\ 
     &           &    5 &  5.30e-08 &  7.06e-16 &      \\ 
     &           &    6 &  2.48e-08 &  1.51e-15 &      \\ 
     &           &    7 &  2.48e-08 &  7.08e-16 &      \\ 
     &           &    8 &  5.30e-08 &  7.07e-16 &      \\ 
     &           &    9 &  2.48e-08 &  1.51e-15 &      \\ 
     &           &   10 &  2.48e-08 &  7.08e-16 &      \\ 
3816 &  3.82e+04 &   10 &           &           & iters  \\ 
 \hdashline 
     &           &    1 &  9.30e-09 &  1.98e-11 &      \\ 
     &           &    2 &  9.29e-09 &  2.65e-16 &      \\ 
     &           &    3 &  9.28e-09 &  2.65e-16 &      \\ 
     &           &    4 &  9.26e-09 &  2.65e-16 &      \\ 
     &           &    5 &  9.25e-09 &  2.64e-16 &      \\ 
     &           &    6 &  9.24e-09 &  2.64e-16 &      \\ 
     &           &    7 &  9.22e-09 &  2.64e-16 &      \\ 
     &           &    8 &  9.21e-09 &  2.63e-16 &      \\ 
     &           &    9 &  9.20e-09 &  2.63e-16 &      \\ 
     &           &   10 &  9.18e-09 &  2.62e-16 &      \\ 
3817 &  3.82e+04 &   10 &           &           & iters  \\ 
 \hdashline 
     &           &    1 &  7.94e-08 &  1.95e-10 &      \\ 
     &           &    2 &  7.61e-08 &  2.27e-15 &      \\ 
     &           &    3 &  7.94e-08 &  2.17e-15 &      \\ 
     &           &    4 &  7.61e-08 &  2.27e-15 &      \\ 
     &           &    5 &  7.94e-08 &  2.17e-15 &      \\ 
     &           &    6 &  7.61e-08 &  2.27e-15 &      \\ 
     &           &    7 &  1.72e-09 &  2.17e-15 &      \\ 
     &           &    8 &  1.71e-09 &  4.90e-17 &      \\ 
     &           &    9 &  1.71e-09 &  4.89e-17 &      \\ 
     &           &   10 &  1.71e-09 &  4.89e-17 &      \\ 
3818 &  3.82e+04 &   10 &           &           & iters  \\ 
 \hdashline 
     &           &    1 &  6.87e-08 &  3.78e-10 &      \\ 
     &           &    2 &  8.68e-08 &  1.96e-15 &      \\ 
     &           &    3 &  6.87e-08 &  2.48e-15 &      \\ 
     &           &    4 &  9.10e-09 &  1.96e-15 &      \\ 
     &           &    5 &  9.08e-09 &  2.60e-16 &      \\ 
     &           &    6 &  9.07e-09 &  2.59e-16 &      \\ 
     &           &    7 &  9.06e-09 &  2.59e-16 &      \\ 
     &           &    8 &  9.04e-09 &  2.59e-16 &      \\ 
     &           &    9 &  9.03e-09 &  2.58e-16 &      \\ 
     &           &   10 &  6.87e-08 &  2.58e-16 &      \\ 
3819 &  3.82e+04 &   10 &           &           & iters  \\ 
 \hdashline 
     &           &    1 &  8.38e-08 &  2.45e-10 &      \\ 
     &           &    2 &  7.17e-08 &  2.39e-15 &      \\ 
     &           &    3 &  6.08e-09 &  2.05e-15 &      \\ 
     &           &    4 &  6.07e-09 &  1.74e-16 &      \\ 
     &           &    5 &  6.06e-09 &  1.73e-16 &      \\ 
     &           &    6 &  6.05e-09 &  1.73e-16 &      \\ 
     &           &    7 &  6.04e-09 &  1.73e-16 &      \\ 
     &           &    8 &  6.04e-09 &  1.73e-16 &      \\ 
     &           &    9 &  6.03e-09 &  1.72e-16 &      \\ 
     &           &   10 &  6.02e-09 &  1.72e-16 &      \\ 
3820 &  3.82e+04 &   10 &           &           & iters  \\ 
 \hdashline 
     &           &    1 &  2.10e-08 &  3.66e-10 &      \\ 
     &           &    2 &  5.67e-08 &  6.00e-16 &      \\ 
     &           &    3 &  2.11e-08 &  1.62e-15 &      \\ 
     &           &    4 &  2.10e-08 &  6.01e-16 &      \\ 
     &           &    5 &  2.10e-08 &  6.00e-16 &      \\ 
     &           &    6 &  5.67e-08 &  5.99e-16 &      \\ 
     &           &    7 &  2.10e-08 &  1.62e-15 &      \\ 
     &           &    8 &  2.10e-08 &  6.01e-16 &      \\ 
     &           &    9 &  2.10e-08 &  6.00e-16 &      \\ 
     &           &   10 &  5.67e-08 &  5.99e-16 &      \\ 
3821 &  3.82e+04 &   10 &           &           & iters  \\ 
 \hdashline 
     &           &    1 &  6.85e-08 &  6.92e-10 &      \\ 
     &           &    2 &  9.32e-09 &  1.95e-15 &      \\ 
     &           &    3 &  9.31e-09 &  2.66e-16 &      \\ 
     &           &    4 &  9.29e-09 &  2.66e-16 &      \\ 
     &           &    5 &  9.28e-09 &  2.65e-16 &      \\ 
     &           &    6 &  9.27e-09 &  2.65e-16 &      \\ 
     &           &    7 &  9.25e-09 &  2.64e-16 &      \\ 
     &           &    8 &  9.24e-09 &  2.64e-16 &      \\ 
     &           &    9 &  6.85e-08 &  2.64e-16 &      \\ 
     &           &   10 &  8.70e-08 &  1.95e-15 &      \\ 
3822 &  3.82e+04 &   10 &           &           & iters  \\ 
 \hdashline 
     &           &    1 &  8.55e-08 &  4.46e-10 &      \\ 
     &           &    2 &  7.72e-09 &  2.44e-15 &      \\ 
     &           &    3 &  7.70e-09 &  2.20e-16 &      \\ 
     &           &    4 &  7.00e-08 &  2.20e-16 &      \\ 
     &           &    5 &  8.55e-08 &  2.00e-15 &      \\ 
     &           &    6 &  7.00e-08 &  2.44e-15 &      \\ 
     &           &    7 &  7.77e-09 &  2.00e-15 &      \\ 
     &           &    8 &  7.76e-09 &  2.22e-16 &      \\ 
     &           &    9 &  7.75e-09 &  2.21e-16 &      \\ 
     &           &   10 &  7.74e-09 &  2.21e-16 &      \\ 
3823 &  3.82e+04 &   10 &           &           & iters  \\ 
 \hdashline 
     &           &    1 &  5.20e-08 &  3.88e-10 &      \\ 
     &           &    2 &  2.58e-08 &  1.48e-15 &      \\ 
     &           &    3 &  5.19e-08 &  7.36e-16 &      \\ 
     &           &    4 &  2.58e-08 &  1.48e-15 &      \\ 
     &           &    5 &  2.58e-08 &  7.37e-16 &      \\ 
     &           &    6 &  5.19e-08 &  7.36e-16 &      \\ 
     &           &    7 &  2.58e-08 &  1.48e-15 &      \\ 
     &           &    8 &  2.58e-08 &  7.37e-16 &      \\ 
     &           &    9 &  5.19e-08 &  7.36e-16 &      \\ 
     &           &   10 &  2.58e-08 &  1.48e-15 &      \\ 
3824 &  3.82e+04 &   10 &           &           & iters  \\ 
 \hdashline 
     &           &    1 &  6.59e-08 &  3.74e-10 &      \\ 
     &           &    2 &  1.19e-08 &  1.88e-15 &      \\ 
     &           &    3 &  1.18e-08 &  3.38e-16 &      \\ 
     &           &    4 &  1.18e-08 &  3.38e-16 &      \\ 
     &           &    5 &  1.18e-08 &  3.37e-16 &      \\ 
     &           &    6 &  1.18e-08 &  3.37e-16 &      \\ 
     &           &    7 &  6.59e-08 &  3.36e-16 &      \\ 
     &           &    8 &  1.19e-08 &  1.88e-15 &      \\ 
     &           &    9 &  1.18e-08 &  3.39e-16 &      \\ 
     &           &   10 &  1.18e-08 &  3.38e-16 &      \\ 
3825 &  3.82e+04 &   10 &           &           & iters  \\ 
 \hdashline 
     &           &    1 &  1.15e-08 &  1.97e-10 &      \\ 
     &           &    2 &  1.15e-08 &  3.29e-16 &      \\ 
     &           &    3 &  6.62e-08 &  3.29e-16 &      \\ 
     &           &    4 &  1.16e-08 &  1.89e-15 &      \\ 
     &           &    5 &  1.16e-08 &  3.31e-16 &      \\ 
     &           &    6 &  1.16e-08 &  3.30e-16 &      \\ 
     &           &    7 &  1.15e-08 &  3.30e-16 &      \\ 
     &           &    8 &  1.15e-08 &  3.29e-16 &      \\ 
     &           &    9 &  1.15e-08 &  3.29e-16 &      \\ 
     &           &   10 &  6.62e-08 &  3.29e-16 &      \\ 
3826 &  3.82e+04 &   10 &           &           & iters  \\ 
 \hdashline 
     &           &    1 &  4.96e-08 &  2.30e-10 &      \\ 
     &           &    2 &  2.82e-08 &  1.41e-15 &      \\ 
     &           &    3 &  4.95e-08 &  8.05e-16 &      \\ 
     &           &    4 &  2.82e-08 &  1.41e-15 &      \\ 
     &           &    5 &  2.82e-08 &  8.06e-16 &      \\ 
     &           &    6 &  4.95e-08 &  8.05e-16 &      \\ 
     &           &    7 &  2.82e-08 &  1.41e-15 &      \\ 
     &           &    8 &  2.82e-08 &  8.06e-16 &      \\ 
     &           &    9 &  4.95e-08 &  8.04e-16 &      \\ 
     &           &   10 &  2.82e-08 &  1.41e-15 &      \\ 
3827 &  3.83e+04 &   10 &           &           & iters  \\ 
 \hdashline 
     &           &    1 &  8.19e-09 &  2.96e-10 &      \\ 
     &           &    2 &  8.18e-09 &  2.34e-16 &      \\ 
     &           &    3 &  6.95e-08 &  2.34e-16 &      \\ 
     &           &    4 &  8.60e-08 &  1.98e-15 &      \\ 
     &           &    5 &  6.95e-08 &  2.45e-15 &      \\ 
     &           &    6 &  8.25e-09 &  1.98e-15 &      \\ 
     &           &    7 &  8.23e-09 &  2.35e-16 &      \\ 
     &           &    8 &  8.22e-09 &  2.35e-16 &      \\ 
     &           &    9 &  8.21e-09 &  2.35e-16 &      \\ 
     &           &   10 &  8.20e-09 &  2.34e-16 &      \\ 
3828 &  3.83e+04 &   10 &           &           & iters  \\ 
 \hdashline 
     &           &    1 &  5.71e-08 &  9.33e-11 &      \\ 
     &           &    2 &  2.07e-08 &  1.63e-15 &      \\ 
     &           &    3 &  2.07e-08 &  5.91e-16 &      \\ 
     &           &    4 &  2.07e-08 &  5.90e-16 &      \\ 
     &           &    5 &  5.71e-08 &  5.89e-16 &      \\ 
     &           &    6 &  2.07e-08 &  1.63e-15 &      \\ 
     &           &    7 &  2.07e-08 &  5.91e-16 &      \\ 
     &           &    8 &  2.06e-08 &  5.90e-16 &      \\ 
     &           &    9 &  5.71e-08 &  5.89e-16 &      \\ 
     &           &   10 &  2.07e-08 &  1.63e-15 &      \\ 
3829 &  3.83e+04 &   10 &           &           & iters  \\ 
 \hdashline 
     &           &    1 &  2.63e-08 &  7.17e-10 &      \\ 
     &           &    2 &  1.26e-08 &  7.51e-16 &      \\ 
     &           &    3 &  1.26e-08 &  3.59e-16 &      \\ 
     &           &    4 &  2.63e-08 &  3.58e-16 &      \\ 
     &           &    5 &  1.26e-08 &  7.51e-16 &      \\ 
     &           &    6 &  1.26e-08 &  3.59e-16 &      \\ 
     &           &    7 &  2.63e-08 &  3.58e-16 &      \\ 
     &           &    8 &  1.26e-08 &  7.51e-16 &      \\ 
     &           &    9 &  1.26e-08 &  3.59e-16 &      \\ 
     &           &   10 &  1.25e-08 &  3.59e-16 &      \\ 
3830 &  3.83e+04 &   10 &           &           & iters  \\ 
 \hdashline 
     &           &    1 &  5.18e-08 &  7.35e-10 &      \\ 
     &           &    2 &  2.60e-08 &  1.48e-15 &      \\ 
     &           &    3 &  5.17e-08 &  7.42e-16 &      \\ 
     &           &    4 &  2.60e-08 &  1.48e-15 &      \\ 
     &           &    5 &  2.60e-08 &  7.43e-16 &      \\ 
     &           &    6 &  5.17e-08 &  7.42e-16 &      \\ 
     &           &    7 &  2.60e-08 &  1.48e-15 &      \\ 
     &           &    8 &  2.60e-08 &  7.43e-16 &      \\ 
     &           &    9 &  5.17e-08 &  7.42e-16 &      \\ 
     &           &   10 &  2.60e-08 &  1.48e-15 &      \\ 
3831 &  3.83e+04 &   10 &           &           & iters  \\ 
 \hdashline 
     &           &    1 &  5.41e-08 &  5.88e-10 &      \\ 
     &           &    2 &  2.37e-08 &  1.54e-15 &      \\ 
     &           &    3 &  2.37e-08 &  6.77e-16 &      \\ 
     &           &    4 &  5.40e-08 &  6.76e-16 &      \\ 
     &           &    5 &  2.37e-08 &  1.54e-15 &      \\ 
     &           &    6 &  2.37e-08 &  6.77e-16 &      \\ 
     &           &    7 &  5.40e-08 &  6.76e-16 &      \\ 
     &           &    8 &  2.37e-08 &  1.54e-15 &      \\ 
     &           &    9 &  2.37e-08 &  6.77e-16 &      \\ 
     &           &   10 &  5.40e-08 &  6.76e-16 &      \\ 
3832 &  3.83e+04 &   10 &           &           & iters  \\ 
 \hdashline 
     &           &    1 &  5.33e-08 &  2.83e-10 &      \\ 
     &           &    2 &  2.45e-08 &  1.52e-15 &      \\ 
     &           &    3 &  5.32e-08 &  6.99e-16 &      \\ 
     &           &    4 &  2.45e-08 &  1.52e-15 &      \\ 
     &           &    5 &  2.45e-08 &  7.00e-16 &      \\ 
     &           &    6 &  5.32e-08 &  6.99e-16 &      \\ 
     &           &    7 &  2.45e-08 &  1.52e-15 &      \\ 
     &           &    8 &  2.45e-08 &  7.01e-16 &      \\ 
     &           &    9 &  5.32e-08 &  7.00e-16 &      \\ 
     &           &   10 &  2.46e-08 &  1.52e-15 &      \\ 
3833 &  3.83e+04 &   10 &           &           & iters  \\ 
 \hdashline 
     &           &    1 &  1.09e-08 &  3.87e-10 &      \\ 
     &           &    2 &  1.09e-08 &  3.12e-16 &      \\ 
     &           &    3 &  1.09e-08 &  3.11e-16 &      \\ 
     &           &    4 &  1.09e-08 &  3.11e-16 &      \\ 
     &           &    5 &  1.09e-08 &  3.10e-16 &      \\ 
     &           &    6 &  1.08e-08 &  3.10e-16 &      \\ 
     &           &    7 &  1.08e-08 &  3.09e-16 &      \\ 
     &           &    8 &  6.69e-08 &  3.09e-16 &      \\ 
     &           &    9 &  1.09e-08 &  1.91e-15 &      \\ 
     &           &   10 &  1.09e-08 &  3.11e-16 &      \\ 
3834 &  3.83e+04 &   10 &           &           & iters  \\ 
 \hdashline 
     &           &    1 &  2.98e-08 &  4.40e-10 &      \\ 
     &           &    2 &  4.79e-08 &  8.50e-16 &      \\ 
     &           &    3 &  2.98e-08 &  1.37e-15 &      \\ 
     &           &    4 &  2.98e-08 &  8.51e-16 &      \\ 
     &           &    5 &  4.80e-08 &  8.50e-16 &      \\ 
     &           &    6 &  2.98e-08 &  1.37e-15 &      \\ 
     &           &    7 &  4.79e-08 &  8.50e-16 &      \\ 
     &           &    8 &  2.98e-08 &  1.37e-15 &      \\ 
     &           &    9 &  2.98e-08 &  8.51e-16 &      \\ 
     &           &   10 &  4.80e-08 &  8.50e-16 &      \\ 
3835 &  3.83e+04 &   10 &           &           & iters  \\ 
 \hdashline 
     &           &    1 &  1.22e-09 &  1.44e-10 &      \\ 
     &           &    2 &  1.22e-09 &  3.49e-17 &      \\ 
     &           &    3 &  1.22e-09 &  3.48e-17 &      \\ 
     &           &    4 &  1.22e-09 &  3.48e-17 &      \\ 
     &           &    5 &  1.22e-09 &  3.47e-17 &      \\ 
     &           &    6 &  1.21e-09 &  3.47e-17 &      \\ 
     &           &    7 &  1.21e-09 &  3.46e-17 &      \\ 
     &           &    8 &  1.21e-09 &  3.46e-17 &      \\ 
     &           &    9 &  1.21e-09 &  3.45e-17 &      \\ 
     &           &   10 &  1.21e-09 &  3.45e-17 &      \\ 
3836 &  3.84e+04 &   10 &           &           & iters  \\ 
 \hdashline 
     &           &    1 &  2.24e-09 &  5.94e-11 &      \\ 
     &           &    2 &  2.24e-09 &  6.40e-17 &      \\ 
     &           &    3 &  2.23e-09 &  6.39e-17 &      \\ 
     &           &    4 &  2.23e-09 &  6.38e-17 &      \\ 
     &           &    5 &  2.23e-09 &  6.37e-17 &      \\ 
     &           &    6 &  2.23e-09 &  6.36e-17 &      \\ 
     &           &    7 &  2.22e-09 &  6.35e-17 &      \\ 
     &           &    8 &  2.22e-09 &  6.34e-17 &      \\ 
     &           &    9 &  2.22e-09 &  6.33e-17 &      \\ 
     &           &   10 &  2.21e-09 &  6.32e-17 &      \\ 
3837 &  3.84e+04 &   10 &           &           & iters  \\ 
 \hdashline 
     &           &    1 &  7.18e-08 &  4.06e-12 &      \\ 
     &           &    2 &  6.04e-09 &  2.05e-15 &      \\ 
     &           &    3 &  6.03e-09 &  1.72e-16 &      \\ 
     &           &    4 &  6.02e-09 &  1.72e-16 &      \\ 
     &           &    5 &  6.01e-09 &  1.72e-16 &      \\ 
     &           &    6 &  6.00e-09 &  1.72e-16 &      \\ 
     &           &    7 &  5.99e-09 &  1.71e-16 &      \\ 
     &           &    8 &  5.98e-09 &  1.71e-16 &      \\ 
     &           &    9 &  5.97e-09 &  1.71e-16 &      \\ 
     &           &   10 &  7.17e-08 &  1.71e-16 &      \\ 
3838 &  3.84e+04 &   10 &           &           & iters  \\ 
 \hdashline 
     &           &    1 &  6.15e-09 &  3.57e-12 &      \\ 
     &           &    2 &  6.14e-09 &  1.76e-16 &      \\ 
     &           &    3 &  6.13e-09 &  1.75e-16 &      \\ 
     &           &    4 &  6.12e-09 &  1.75e-16 &      \\ 
     &           &    5 &  6.11e-09 &  1.75e-16 &      \\ 
     &           &    6 &  6.11e-09 &  1.75e-16 &      \\ 
     &           &    7 &  6.10e-09 &  1.74e-16 &      \\ 
     &           &    8 &  6.09e-09 &  1.74e-16 &      \\ 
     &           &    9 &  6.08e-09 &  1.74e-16 &      \\ 
     &           &   10 &  6.07e-09 &  1.74e-16 &      \\ 
3839 &  3.84e+04 &   10 &           &           & iters  \\ 
 \hdashline 
     &           &    1 &  2.21e-08 &  6.25e-11 &      \\ 
     &           &    2 &  2.20e-08 &  6.30e-16 &      \\ 
     &           &    3 &  5.57e-08 &  6.29e-16 &      \\ 
     &           &    4 &  2.21e-08 &  1.59e-15 &      \\ 
     &           &    5 &  2.20e-08 &  6.30e-16 &      \\ 
     &           &    6 &  5.57e-08 &  6.29e-16 &      \\ 
     &           &    7 &  2.21e-08 &  1.59e-15 &      \\ 
     &           &    8 &  2.21e-08 &  6.31e-16 &      \\ 
     &           &    9 &  2.20e-08 &  6.30e-16 &      \\ 
     &           &   10 &  5.57e-08 &  6.29e-16 &      \\ 
3840 &  3.84e+04 &   10 &           &           & iters  \\ 
 \hdashline 
     &           &    1 &  1.41e-08 &  9.41e-11 &      \\ 
     &           &    2 &  1.41e-08 &  4.03e-16 &      \\ 
     &           &    3 &  1.41e-08 &  4.02e-16 &      \\ 
     &           &    4 &  1.41e-08 &  4.02e-16 &      \\ 
     &           &    5 &  1.40e-08 &  4.01e-16 &      \\ 
     &           &    6 &  6.37e-08 &  4.01e-16 &      \\ 
     &           &    7 &  1.41e-08 &  1.82e-15 &      \\ 
     &           &    8 &  1.41e-08 &  4.03e-16 &      \\ 
     &           &    9 &  1.41e-08 &  4.02e-16 &      \\ 
     &           &   10 &  1.41e-08 &  4.02e-16 &      \\ 
3841 &  3.84e+04 &   10 &           &           & iters  \\ 
 \hdashline 
     &           &    1 &  5.66e-08 &  5.63e-11 &      \\ 
     &           &    2 &  2.11e-08 &  1.62e-15 &      \\ 
     &           &    3 &  5.66e-08 &  6.03e-16 &      \\ 
     &           &    4 &  2.12e-08 &  1.62e-15 &      \\ 
     &           &    5 &  2.11e-08 &  6.04e-16 &      \\ 
     &           &    6 &  5.66e-08 &  6.03e-16 &      \\ 
     &           &    7 &  2.12e-08 &  1.61e-15 &      \\ 
     &           &    8 &  2.12e-08 &  6.05e-16 &      \\ 
     &           &    9 &  2.11e-08 &  6.04e-16 &      \\ 
     &           &   10 &  5.66e-08 &  6.03e-16 &      \\ 
3842 &  3.84e+04 &   10 &           &           & iters  \\ 
 \hdashline 
     &           &    1 &  1.82e-08 &  8.83e-11 &      \\ 
     &           &    2 &  1.82e-08 &  5.21e-16 &      \\ 
     &           &    3 &  1.82e-08 &  5.20e-16 &      \\ 
     &           &    4 &  5.95e-08 &  5.19e-16 &      \\ 
     &           &    5 &  1.83e-08 &  1.70e-15 &      \\ 
     &           &    6 &  1.82e-08 &  5.21e-16 &      \\ 
     &           &    7 &  1.82e-08 &  5.20e-16 &      \\ 
     &           &    8 &  5.95e-08 &  5.19e-16 &      \\ 
     &           &    9 &  1.83e-08 &  1.70e-15 &      \\ 
     &           &   10 &  1.82e-08 &  5.21e-16 &      \\ 
3843 &  3.84e+04 &   10 &           &           & iters  \\ 
 \hdashline 
     &           &    1 &  3.40e-08 &  8.05e-11 &      \\ 
     &           &    2 &  3.39e-08 &  9.69e-16 &      \\ 
     &           &    3 &  4.38e-08 &  9.68e-16 &      \\ 
     &           &    4 &  3.39e-08 &  1.25e-15 &      \\ 
     &           &    5 &  4.38e-08 &  9.68e-16 &      \\ 
     &           &    6 &  3.39e-08 &  1.25e-15 &      \\ 
     &           &    7 &  4.38e-08 &  9.68e-16 &      \\ 
     &           &    8 &  3.39e-08 &  1.25e-15 &      \\ 
     &           &    9 &  4.38e-08 &  9.69e-16 &      \\ 
     &           &   10 &  3.40e-08 &  1.25e-15 &      \\ 
3844 &  3.84e+04 &   10 &           &           & iters  \\ 
 \hdashline 
     &           &    1 &  4.90e-09 &  3.11e-10 &      \\ 
     &           &    2 &  4.90e-09 &  1.40e-16 &      \\ 
     &           &    3 &  4.89e-09 &  1.40e-16 &      \\ 
     &           &    4 &  4.88e-09 &  1.40e-16 &      \\ 
     &           &    5 &  4.88e-09 &  1.39e-16 &      \\ 
     &           &    6 &  4.87e-09 &  1.39e-16 &      \\ 
     &           &    7 &  4.86e-09 &  1.39e-16 &      \\ 
     &           &    8 &  4.86e-09 &  1.39e-16 &      \\ 
     &           &    9 &  4.85e-09 &  1.39e-16 &      \\ 
     &           &   10 &  4.84e-09 &  1.38e-16 &      \\ 
3845 &  3.84e+04 &   10 &           &           & iters  \\ 
 \hdashline 
     &           &    1 &  2.86e-08 &  9.78e-10 &      \\ 
     &           &    2 &  2.85e-08 &  8.15e-16 &      \\ 
     &           &    3 &  1.04e-08 &  8.14e-16 &      \\ 
     &           &    4 &  1.03e-08 &  2.96e-16 &      \\ 
     &           &    5 &  1.03e-08 &  2.95e-16 &      \\ 
     &           &    6 &  2.85e-08 &  2.95e-16 &      \\ 
     &           &    7 &  1.04e-08 &  8.14e-16 &      \\ 
     &           &    8 &  1.03e-08 &  2.96e-16 &      \\ 
     &           &    9 &  1.03e-08 &  2.95e-16 &      \\ 
     &           &   10 &  2.85e-08 &  2.95e-16 &      \\ 
3846 &  3.84e+04 &   10 &           &           & iters  \\ 
 \hdashline 
     &           &    1 &  3.01e-08 &  1.45e-09 &      \\ 
     &           &    2 &  8.82e-09 &  8.58e-16 &      \\ 
     &           &    3 &  8.80e-09 &  2.52e-16 &      \\ 
     &           &    4 &  8.79e-09 &  2.51e-16 &      \\ 
     &           &    5 &  8.78e-09 &  2.51e-16 &      \\ 
     &           &    6 &  3.01e-08 &  2.51e-16 &      \\ 
     &           &    7 &  8.81e-09 &  8.58e-16 &      \\ 
     &           &    8 &  8.80e-09 &  2.51e-16 &      \\ 
     &           &    9 &  8.79e-09 &  2.51e-16 &      \\ 
     &           &   10 &  3.01e-08 &  2.51e-16 &      \\ 
3847 &  3.85e+04 &   10 &           &           & iters  \\ 
 \hdashline 
     &           &    1 &  5.70e-08 &  9.23e-10 &      \\ 
     &           &    2 &  2.08e-08 &  1.63e-15 &      \\ 
     &           &    3 &  2.08e-08 &  5.94e-16 &      \\ 
     &           &    4 &  2.08e-08 &  5.93e-16 &      \\ 
     &           &    5 &  5.70e-08 &  5.92e-16 &      \\ 
     &           &    6 &  2.08e-08 &  1.63e-15 &      \\ 
     &           &    7 &  2.08e-08 &  5.94e-16 &      \\ 
     &           &    8 &  2.07e-08 &  5.93e-16 &      \\ 
     &           &    9 &  5.70e-08 &  5.92e-16 &      \\ 
     &           &   10 &  2.08e-08 &  1.63e-15 &      \\ 
3848 &  3.85e+04 &   10 &           &           & iters  \\ 
 \hdashline 
     &           &    1 &  2.66e-08 &  6.07e-10 &      \\ 
     &           &    2 &  5.11e-08 &  7.59e-16 &      \\ 
     &           &    3 &  2.66e-08 &  1.46e-15 &      \\ 
     &           &    4 &  2.66e-08 &  7.60e-16 &      \\ 
     &           &    5 &  5.11e-08 &  7.59e-16 &      \\ 
     &           &    6 &  2.66e-08 &  1.46e-15 &      \\ 
     &           &    7 &  2.66e-08 &  7.60e-16 &      \\ 
     &           &    8 &  5.11e-08 &  7.59e-16 &      \\ 
     &           &    9 &  2.66e-08 &  1.46e-15 &      \\ 
     &           &   10 &  2.66e-08 &  7.60e-16 &      \\ 
3849 &  3.85e+04 &   10 &           &           & iters  \\ 
 \hdashline 
     &           &    1 &  1.64e-09 &  9.44e-10 &      \\ 
     &           &    2 &  1.64e-09 &  4.69e-17 &      \\ 
     &           &    3 &  1.64e-09 &  4.69e-17 &      \\ 
     &           &    4 &  1.64e-09 &  4.68e-17 &      \\ 
     &           &    5 &  1.64e-09 &  4.67e-17 &      \\ 
     &           &    6 &  1.63e-09 &  4.67e-17 &      \\ 
     &           &    7 &  1.63e-09 &  4.66e-17 &      \\ 
     &           &    8 &  1.63e-09 &  4.65e-17 &      \\ 
     &           &    9 &  1.63e-09 &  4.65e-17 &      \\ 
     &           &   10 &  1.62e-09 &  4.64e-17 &      \\ 
3850 &  3.85e+04 &   10 &           &           & iters  \\ 
 \hdashline 
     &           &    1 &  8.50e-09 &  7.93e-11 &      \\ 
     &           &    2 &  8.48e-09 &  2.42e-16 &      \\ 
     &           &    3 &  6.92e-08 &  2.42e-16 &      \\ 
     &           &    4 &  8.63e-08 &  1.98e-15 &      \\ 
     &           &    5 &  6.92e-08 &  2.46e-15 &      \\ 
     &           &    6 &  8.55e-09 &  1.98e-15 &      \\ 
     &           &    7 &  8.53e-09 &  2.44e-16 &      \\ 
     &           &    8 &  8.52e-09 &  2.44e-16 &      \\ 
     &           &    9 &  8.51e-09 &  2.43e-16 &      \\ 
     &           &   10 &  8.50e-09 &  2.43e-16 &      \\ 
3851 &  3.85e+04 &   10 &           &           & iters  \\ 
 \hdashline 
     &           &    1 &  5.67e-08 &  1.47e-10 &      \\ 
     &           &    2 &  2.11e-08 &  1.62e-15 &      \\ 
     &           &    3 &  2.10e-08 &  6.01e-16 &      \\ 
     &           &    4 &  2.10e-08 &  6.00e-16 &      \\ 
     &           &    5 &  5.67e-08 &  5.99e-16 &      \\ 
     &           &    6 &  2.10e-08 &  1.62e-15 &      \\ 
     &           &    7 &  2.10e-08 &  6.01e-16 &      \\ 
     &           &    8 &  2.10e-08 &  6.00e-16 &      \\ 
     &           &    9 &  5.67e-08 &  5.99e-16 &      \\ 
     &           &   10 &  2.10e-08 &  1.62e-15 &      \\ 
3852 &  3.85e+04 &   10 &           &           & iters  \\ 
 \hdashline 
     &           &    1 &  2.26e-08 &  2.65e-10 &      \\ 
     &           &    2 &  5.51e-08 &  6.44e-16 &      \\ 
     &           &    3 &  2.26e-08 &  1.57e-15 &      \\ 
     &           &    4 &  2.26e-08 &  6.46e-16 &      \\ 
     &           &    5 &  5.51e-08 &  6.45e-16 &      \\ 
     &           &    6 &  2.26e-08 &  1.57e-15 &      \\ 
     &           &    7 &  2.26e-08 &  6.46e-16 &      \\ 
     &           &    8 &  2.26e-08 &  6.45e-16 &      \\ 
     &           &    9 &  5.52e-08 &  6.44e-16 &      \\ 
     &           &   10 &  2.26e-08 &  1.57e-15 &      \\ 
3853 &  3.85e+04 &   10 &           &           & iters  \\ 
 \hdashline 
     &           &    1 &  1.82e-09 &  1.40e-11 &      \\ 
     &           &    2 &  1.82e-09 &  5.21e-17 &      \\ 
     &           &    3 &  1.82e-09 &  5.20e-17 &      \\ 
     &           &    4 &  1.82e-09 &  5.19e-17 &      \\ 
     &           &    5 &  1.81e-09 &  5.18e-17 &      \\ 
     &           &    6 &  1.81e-09 &  5.18e-17 &      \\ 
     &           &    7 &  1.81e-09 &  5.17e-17 &      \\ 
     &           &    8 &  1.81e-09 &  5.16e-17 &      \\ 
     &           &    9 &  1.80e-09 &  5.15e-17 &      \\ 
     &           &   10 &  1.80e-09 &  5.15e-17 &      \\ 
3854 &  3.85e+04 &   10 &           &           & iters  \\ 
 \hdashline 
     &           &    1 &  1.94e-09 &  1.03e-11 &      \\ 
     &           &    2 &  1.94e-09 &  5.53e-17 &      \\ 
     &           &    3 &  1.93e-09 &  5.52e-17 &      \\ 
     &           &    4 &  1.93e-09 &  5.52e-17 &      \\ 
     &           &    5 &  1.93e-09 &  5.51e-17 &      \\ 
     &           &    6 &  1.92e-09 &  5.50e-17 &      \\ 
     &           &    7 &  1.92e-09 &  5.49e-17 &      \\ 
     &           &    8 &  1.92e-09 &  5.48e-17 &      \\ 
     &           &    9 &  1.92e-09 &  5.48e-17 &      \\ 
     &           &   10 &  1.91e-09 &  5.47e-17 &      \\ 
3855 &  3.85e+04 &   10 &           &           & iters  \\ 
 \hdashline 
     &           &    1 &  2.44e-09 &  5.82e-10 &      \\ 
     &           &    2 &  2.44e-09 &  6.97e-17 &      \\ 
     &           &    3 &  2.43e-09 &  6.96e-17 &      \\ 
     &           &    4 &  2.43e-09 &  6.95e-17 &      \\ 
     &           &    5 &  2.43e-09 &  6.94e-17 &      \\ 
     &           &    6 &  2.42e-09 &  6.93e-17 &      \\ 
     &           &    7 &  2.42e-09 &  6.92e-17 &      \\ 
     &           &    8 &  2.42e-09 &  6.91e-17 &      \\ 
     &           &    9 &  2.41e-09 &  6.90e-17 &      \\ 
     &           &   10 &  2.41e-09 &  6.89e-17 &      \\ 
3856 &  3.86e+04 &   10 &           &           & iters  \\ 
 \hdashline 
     &           &    1 &  1.78e-08 &  8.25e-10 &      \\ 
     &           &    2 &  5.99e-08 &  5.09e-16 &      \\ 
     &           &    3 &  1.79e-08 &  1.71e-15 &      \\ 
     &           &    4 &  1.79e-08 &  5.11e-16 &      \\ 
     &           &    5 &  1.78e-08 &  5.10e-16 &      \\ 
     &           &    6 &  1.78e-08 &  5.09e-16 &      \\ 
     &           &    7 &  5.99e-08 &  5.09e-16 &      \\ 
     &           &    8 &  1.79e-08 &  1.71e-15 &      \\ 
     &           &    9 &  1.79e-08 &  5.10e-16 &      \\ 
     &           &   10 &  1.78e-08 &  5.10e-16 &      \\ 
3857 &  3.86e+04 &   10 &           &           & iters  \\ 
 \hdashline 
     &           &    1 &  2.66e-09 &  4.08e-10 &      \\ 
     &           &    2 &  2.65e-09 &  7.58e-17 &      \\ 
     &           &    3 &  2.65e-09 &  7.57e-17 &      \\ 
     &           &    4 &  2.65e-09 &  7.56e-17 &      \\ 
     &           &    5 &  2.64e-09 &  7.55e-17 &      \\ 
     &           &    6 &  2.64e-09 &  7.54e-17 &      \\ 
     &           &    7 &  2.63e-09 &  7.53e-17 &      \\ 
     &           &    8 &  2.63e-09 &  7.52e-17 &      \\ 
     &           &    9 &  2.63e-09 &  7.51e-17 &      \\ 
     &           &   10 &  2.62e-09 &  7.50e-17 &      \\ 
3858 &  3.86e+04 &   10 &           &           & iters  \\ 
 \hdashline 
     &           &    1 &  3.47e-08 &  2.73e-10 &      \\ 
     &           &    2 &  4.31e-08 &  9.89e-16 &      \\ 
     &           &    3 &  3.47e-08 &  1.23e-15 &      \\ 
     &           &    4 &  4.31e-08 &  9.89e-16 &      \\ 
     &           &    5 &  3.47e-08 &  1.23e-15 &      \\ 
     &           &    6 &  3.46e-08 &  9.90e-16 &      \\ 
     &           &    7 &  4.31e-08 &  9.88e-16 &      \\ 
     &           &    8 &  3.46e-08 &  1.23e-15 &      \\ 
     &           &    9 &  4.31e-08 &  9.89e-16 &      \\ 
     &           &   10 &  3.47e-08 &  1.23e-15 &      \\ 
3859 &  3.86e+04 &   10 &           &           & iters  \\ 
 \hdashline 
     &           &    1 &  3.69e-08 &  3.26e-10 &      \\ 
     &           &    2 &  4.08e-08 &  1.05e-15 &      \\ 
     &           &    3 &  3.69e-08 &  1.17e-15 &      \\ 
     &           &    4 &  4.08e-08 &  1.05e-15 &      \\ 
     &           &    5 &  3.69e-08 &  1.16e-15 &      \\ 
     &           &    6 &  4.08e-08 &  1.05e-15 &      \\ 
     &           &    7 &  3.69e-08 &  1.16e-15 &      \\ 
     &           &    8 &  4.08e-08 &  1.05e-15 &      \\ 
     &           &    9 &  3.69e-08 &  1.16e-15 &      \\ 
     &           &   10 &  3.69e-08 &  1.05e-15 &      \\ 
3860 &  3.86e+04 &   10 &           &           & iters  \\ 
 \hdashline 
     &           &    1 &  7.12e-09 &  4.16e-11 &      \\ 
     &           &    2 &  7.11e-09 &  2.03e-16 &      \\ 
     &           &    3 &  7.10e-09 &  2.03e-16 &      \\ 
     &           &    4 &  7.09e-09 &  2.03e-16 &      \\ 
     &           &    5 &  7.08e-09 &  2.02e-16 &      \\ 
     &           &    6 &  7.07e-09 &  2.02e-16 &      \\ 
     &           &    7 &  7.06e-09 &  2.02e-16 &      \\ 
     &           &    8 &  7.05e-09 &  2.01e-16 &      \\ 
     &           &    9 &  7.04e-09 &  2.01e-16 &      \\ 
     &           &   10 &  7.03e-09 &  2.01e-16 &      \\ 
3861 &  3.86e+04 &   10 &           &           & iters  \\ 
 \hdashline 
     &           &    1 &  2.16e-08 &  1.39e-10 &      \\ 
     &           &    2 &  2.16e-08 &  6.17e-16 &      \\ 
     &           &    3 &  2.16e-08 &  6.16e-16 &      \\ 
     &           &    4 &  5.62e-08 &  6.15e-16 &      \\ 
     &           &    5 &  2.16e-08 &  1.60e-15 &      \\ 
     &           &    6 &  2.16e-08 &  6.17e-16 &      \\ 
     &           &    7 &  5.61e-08 &  6.16e-16 &      \\ 
     &           &    8 &  2.16e-08 &  1.60e-15 &      \\ 
     &           &    9 &  2.16e-08 &  6.17e-16 &      \\ 
     &           &   10 &  2.16e-08 &  6.17e-16 &      \\ 
3862 &  3.86e+04 &   10 &           &           & iters  \\ 
 \hdashline 
     &           &    1 &  4.64e-08 &  4.77e-10 &      \\ 
     &           &    2 &  3.13e-08 &  1.32e-15 &      \\ 
     &           &    3 &  3.13e-08 &  8.94e-16 &      \\ 
     &           &    4 &  4.64e-08 &  8.93e-16 &      \\ 
     &           &    5 &  3.13e-08 &  1.33e-15 &      \\ 
     &           &    6 &  4.64e-08 &  8.94e-16 &      \\ 
     &           &    7 &  3.13e-08 &  1.32e-15 &      \\ 
     &           &    8 &  3.13e-08 &  8.94e-16 &      \\ 
     &           &    9 &  4.64e-08 &  8.93e-16 &      \\ 
     &           &   10 &  3.13e-08 &  1.33e-15 &      \\ 
3863 &  3.86e+04 &   10 &           &           & iters  \\ 
 \hdashline 
     &           &    1 &  1.15e-08 &  7.97e-10 &      \\ 
     &           &    2 &  1.15e-08 &  3.29e-16 &      \\ 
     &           &    3 &  1.15e-08 &  3.28e-16 &      \\ 
     &           &    4 &  1.15e-08 &  3.28e-16 &      \\ 
     &           &    5 &  1.14e-08 &  3.27e-16 &      \\ 
     &           &    6 &  6.63e-08 &  3.27e-16 &      \\ 
     &           &    7 &  1.15e-08 &  1.89e-15 &      \\ 
     &           &    8 &  1.15e-08 &  3.29e-16 &      \\ 
     &           &    9 &  1.15e-08 &  3.28e-16 &      \\ 
     &           &   10 &  1.15e-08 &  3.28e-16 &      \\ 
3864 &  3.86e+04 &   10 &           &           & iters  \\ 
 \hdashline 
     &           &    1 &  3.04e-09 &  6.43e-10 &      \\ 
     &           &    2 &  3.04e-09 &  8.68e-17 &      \\ 
     &           &    3 &  3.03e-09 &  8.67e-17 &      \\ 
     &           &    4 &  3.03e-09 &  8.66e-17 &      \\ 
     &           &    5 &  3.02e-09 &  8.64e-17 &      \\ 
     &           &    6 &  3.02e-09 &  8.63e-17 &      \\ 
     &           &    7 &  3.02e-09 &  8.62e-17 &      \\ 
     &           &    8 &  3.01e-09 &  8.61e-17 &      \\ 
     &           &    9 &  3.01e-09 &  8.59e-17 &      \\ 
     &           &   10 &  3.58e-08 &  8.58e-17 &      \\ 
3865 &  3.86e+04 &   10 &           &           & iters  \\ 
 \hdashline 
     &           &    1 &  3.48e-08 &  6.05e-10 &      \\ 
     &           &    2 &  3.47e-08 &  9.92e-16 &      \\ 
     &           &    3 &  4.30e-08 &  9.91e-16 &      \\ 
     &           &    4 &  3.47e-08 &  1.23e-15 &      \\ 
     &           &    5 &  4.30e-08 &  9.91e-16 &      \\ 
     &           &    6 &  3.47e-08 &  1.23e-15 &      \\ 
     &           &    7 &  4.30e-08 &  9.91e-16 &      \\ 
     &           &    8 &  3.48e-08 &  1.23e-15 &      \\ 
     &           &    9 &  4.30e-08 &  9.92e-16 &      \\ 
     &           &   10 &  3.48e-08 &  1.23e-15 &      \\ 
3866 &  3.86e+04 &   10 &           &           & iters  \\ 
 \hdashline 
     &           &    1 &  5.88e-08 &  7.21e-11 &      \\ 
     &           &    2 &  1.89e-08 &  1.68e-15 &      \\ 
     &           &    3 &  1.89e-08 &  5.41e-16 &      \\ 
     &           &    4 &  1.89e-08 &  5.40e-16 &      \\ 
     &           &    5 &  1.89e-08 &  5.39e-16 &      \\ 
     &           &    6 &  5.89e-08 &  5.38e-16 &      \\ 
     &           &    7 &  1.89e-08 &  1.68e-15 &      \\ 
     &           &    8 &  1.89e-08 &  5.40e-16 &      \\ 
     &           &    9 &  1.89e-08 &  5.39e-16 &      \\ 
     &           &   10 &  5.89e-08 &  5.38e-16 &      \\ 
3867 &  3.87e+04 &   10 &           &           & iters  \\ 
 \hdashline 
     &           &    1 &  1.98e-08 &  8.18e-10 &      \\ 
     &           &    2 &  1.98e-08 &  5.65e-16 &      \\ 
     &           &    3 &  1.91e-08 &  5.65e-16 &      \\ 
     &           &    4 &  1.98e-08 &  5.45e-16 &      \\ 
     &           &    5 &  1.91e-08 &  5.65e-16 &      \\ 
     &           &    6 &  1.98e-08 &  5.45e-16 &      \\ 
     &           &    7 &  1.91e-08 &  5.65e-16 &      \\ 
     &           &    8 &  1.98e-08 &  5.45e-16 &      \\ 
     &           &    9 &  1.91e-08 &  5.65e-16 &      \\ 
     &           &   10 &  1.98e-08 &  5.45e-16 &      \\ 
3868 &  3.87e+04 &   10 &           &           & iters  \\ 
 \hdashline 
     &           &    1 &  3.35e-08 &  7.78e-10 &      \\ 
     &           &    2 &  4.42e-08 &  9.56e-16 &      \\ 
     &           &    3 &  3.35e-08 &  1.26e-15 &      \\ 
     &           &    4 &  3.35e-08 &  9.57e-16 &      \\ 
     &           &    5 &  4.43e-08 &  9.55e-16 &      \\ 
     &           &    6 &  3.35e-08 &  1.26e-15 &      \\ 
     &           &    7 &  4.42e-08 &  9.56e-16 &      \\ 
     &           &    8 &  3.35e-08 &  1.26e-15 &      \\ 
     &           &    9 &  4.42e-08 &  9.56e-16 &      \\ 
     &           &   10 &  3.35e-08 &  1.26e-15 &      \\ 
3869 &  3.87e+04 &   10 &           &           & iters  \\ 
 \hdashline 
     &           &    1 &  1.11e-08 &  3.27e-10 &      \\ 
     &           &    2 &  1.11e-08 &  3.17e-16 &      \\ 
     &           &    3 &  1.11e-08 &  3.17e-16 &      \\ 
     &           &    4 &  1.11e-08 &  3.17e-16 &      \\ 
     &           &    5 &  1.11e-08 &  3.16e-16 &      \\ 
     &           &    6 &  6.66e-08 &  3.16e-16 &      \\ 
     &           &    7 &  1.11e-08 &  1.90e-15 &      \\ 
     &           &    8 &  1.11e-08 &  3.18e-16 &      \\ 
     &           &    9 &  1.11e-08 &  3.17e-16 &      \\ 
     &           &   10 &  1.11e-08 &  3.17e-16 &      \\ 
3870 &  3.87e+04 &   10 &           &           & iters  \\ 
 \hdashline 
     &           &    1 &  2.43e-08 &  1.18e-10 &      \\ 
     &           &    2 &  5.34e-08 &  6.93e-16 &      \\ 
     &           &    3 &  2.43e-08 &  1.53e-15 &      \\ 
     &           &    4 &  2.43e-08 &  6.94e-16 &      \\ 
     &           &    5 &  5.34e-08 &  6.93e-16 &      \\ 
     &           &    6 &  2.43e-08 &  1.52e-15 &      \\ 
     &           &    7 &  2.43e-08 &  6.94e-16 &      \\ 
     &           &    8 &  5.34e-08 &  6.93e-16 &      \\ 
     &           &    9 &  2.43e-08 &  1.52e-15 &      \\ 
     &           &   10 &  2.43e-08 &  6.95e-16 &      \\ 
3871 &  3.87e+04 &   10 &           &           & iters  \\ 
 \hdashline 
     &           &    1 &  3.89e-08 &  2.20e-10 &      \\ 
     &           &    2 &  1.43e-11 &  1.11e-15 &      \\ 
     &           &    3 &  1.43e-11 &  4.08e-19 &      \\ 
     &           &    4 &  1.42e-11 &  4.07e-19 &      \\ 
     &           &    5 &  1.42e-11 &  4.06e-19 &      \\ 
     &           &    6 &  1.42e-11 &  4.06e-19 &      \\ 
     &           &    7 &  1.42e-11 &  4.05e-19 &      \\ 
     &           &    8 &  1.42e-11 &  4.05e-19 &      \\ 
     &           &    9 &  1.41e-11 &  4.04e-19 &      \\ 
     &           &   10 &  1.41e-11 &  4.04e-19 &      \\ 
3872 &  3.87e+04 &   10 &           &           & iters  \\ 
 \hdashline 
     &           &    1 &  2.19e-08 &  1.84e-10 &      \\ 
     &           &    2 &  5.58e-08 &  6.26e-16 &      \\ 
     &           &    3 &  2.20e-08 &  1.59e-15 &      \\ 
     &           &    4 &  2.19e-08 &  6.27e-16 &      \\ 
     &           &    5 &  2.19e-08 &  6.26e-16 &      \\ 
     &           &    6 &  5.58e-08 &  6.25e-16 &      \\ 
     &           &    7 &  2.20e-08 &  1.59e-15 &      \\ 
     &           &    8 &  2.19e-08 &  6.27e-16 &      \\ 
     &           &    9 &  5.58e-08 &  6.26e-16 &      \\ 
     &           &   10 &  2.20e-08 &  1.59e-15 &      \\ 
3873 &  3.87e+04 &   10 &           &           & iters  \\ 
 \hdashline 
     &           &    1 &  4.84e-08 &  2.11e-10 &      \\ 
     &           &    2 &  2.94e-08 &  1.38e-15 &      \\ 
     &           &    3 &  4.84e-08 &  8.38e-16 &      \\ 
     &           &    4 &  2.94e-08 &  1.38e-15 &      \\ 
     &           &    5 &  2.93e-08 &  8.39e-16 &      \\ 
     &           &    6 &  4.84e-08 &  8.38e-16 &      \\ 
     &           &    7 &  2.94e-08 &  1.38e-15 &      \\ 
     &           &    8 &  2.93e-08 &  8.38e-16 &      \\ 
     &           &    9 &  4.84e-08 &  8.37e-16 &      \\ 
     &           &   10 &  2.94e-08 &  1.38e-15 &      \\ 
3874 &  3.87e+04 &   10 &           &           & iters  \\ 
 \hdashline 
     &           &    1 &  3.88e-08 &  4.69e-11 &      \\ 
     &           &    2 &  3.89e-08 &  1.11e-15 &      \\ 
     &           &    3 &  3.88e-08 &  1.11e-15 &      \\ 
     &           &    4 &  3.89e-08 &  1.11e-15 &      \\ 
     &           &    5 &  3.88e-08 &  1.11e-15 &      \\ 
     &           &    6 &  3.89e-08 &  1.11e-15 &      \\ 
     &           &    7 &  3.88e-08 &  1.11e-15 &      \\ 
     &           &    8 &  3.89e-08 &  1.11e-15 &      \\ 
     &           &    9 &  3.88e-08 &  1.11e-15 &      \\ 
     &           &   10 &  3.89e-08 &  1.11e-15 &      \\ 
3875 &  3.87e+04 &   10 &           &           & iters  \\ 
 \hdashline 
     &           &    1 &  1.79e-08 &  6.18e-11 &      \\ 
     &           &    2 &  1.79e-08 &  5.10e-16 &      \\ 
     &           &    3 &  5.99e-08 &  5.10e-16 &      \\ 
     &           &    4 &  1.79e-08 &  1.71e-15 &      \\ 
     &           &    5 &  1.79e-08 &  5.11e-16 &      \\ 
     &           &    6 &  1.79e-08 &  5.11e-16 &      \\ 
     &           &    7 &  5.98e-08 &  5.10e-16 &      \\ 
     &           &    8 &  1.79e-08 &  1.71e-15 &      \\ 
     &           &    9 &  1.79e-08 &  5.12e-16 &      \\ 
     &           &   10 &  1.79e-08 &  5.11e-16 &      \\ 
3876 &  3.88e+04 &   10 &           &           & iters  \\ 
 \hdashline 
     &           &    1 &  6.60e-08 &  1.01e-10 &      \\ 
     &           &    2 &  1.17e-08 &  1.88e-15 &      \\ 
     &           &    3 &  1.17e-08 &  3.35e-16 &      \\ 
     &           &    4 &  1.17e-08 &  3.35e-16 &      \\ 
     &           &    5 &  1.17e-08 &  3.34e-16 &      \\ 
     &           &    6 &  6.60e-08 &  3.34e-16 &      \\ 
     &           &    7 &  1.18e-08 &  1.88e-15 &      \\ 
     &           &    8 &  1.18e-08 &  3.36e-16 &      \\ 
     &           &    9 &  1.17e-08 &  3.35e-16 &      \\ 
     &           &   10 &  1.17e-08 &  3.35e-16 &      \\ 
3877 &  3.88e+04 &   10 &           &           & iters  \\ 
 \hdashline 
     &           &    1 &  3.52e-08 &  1.40e-10 &      \\ 
     &           &    2 &  3.52e-08 &  1.01e-15 &      \\ 
     &           &    3 &  4.26e-08 &  1.00e-15 &      \\ 
     &           &    4 &  3.52e-08 &  1.21e-15 &      \\ 
     &           &    5 &  4.25e-08 &  1.00e-15 &      \\ 
     &           &    6 &  3.52e-08 &  1.21e-15 &      \\ 
     &           &    7 &  4.25e-08 &  1.00e-15 &      \\ 
     &           &    8 &  3.52e-08 &  1.21e-15 &      \\ 
     &           &    9 &  4.25e-08 &  1.00e-15 &      \\ 
     &           &   10 &  3.52e-08 &  1.21e-15 &      \\ 
3878 &  3.88e+04 &   10 &           &           & iters  \\ 
 \hdashline 
     &           &    1 &  5.90e-08 &  1.42e-10 &      \\ 
     &           &    2 &  1.88e-08 &  1.68e-15 &      \\ 
     &           &    3 &  1.88e-08 &  5.37e-16 &      \\ 
     &           &    4 &  5.89e-08 &  5.36e-16 &      \\ 
     &           &    5 &  1.88e-08 &  1.68e-15 &      \\ 
     &           &    6 &  1.88e-08 &  5.38e-16 &      \\ 
     &           &    7 &  1.88e-08 &  5.37e-16 &      \\ 
     &           &    8 &  5.89e-08 &  5.36e-16 &      \\ 
     &           &    9 &  1.88e-08 &  1.68e-15 &      \\ 
     &           &   10 &  1.88e-08 &  5.38e-16 &      \\ 
3879 &  3.88e+04 &   10 &           &           & iters  \\ 
 \hdashline 
     &           &    1 &  4.34e-08 &  2.21e-10 &      \\ 
     &           &    2 &  3.43e-08 &  1.24e-15 &      \\ 
     &           &    3 &  3.43e-08 &  9.79e-16 &      \\ 
     &           &    4 &  4.35e-08 &  9.78e-16 &      \\ 
     &           &    5 &  3.43e-08 &  1.24e-15 &      \\ 
     &           &    6 &  4.35e-08 &  9.78e-16 &      \\ 
     &           &    7 &  3.43e-08 &  1.24e-15 &      \\ 
     &           &    8 &  4.34e-08 &  9.79e-16 &      \\ 
     &           &    9 &  3.43e-08 &  1.24e-15 &      \\ 
     &           &   10 &  4.34e-08 &  9.79e-16 &      \\ 
3880 &  3.88e+04 &   10 &           &           & iters  \\ 
 \hdashline 
     &           &    1 &  8.59e-08 &  4.05e-10 &      \\ 
     &           &    2 &  6.96e-08 &  2.45e-15 &      \\ 
     &           &    3 &  8.17e-09 &  1.99e-15 &      \\ 
     &           &    4 &  8.15e-09 &  2.33e-16 &      \\ 
     &           &    5 &  8.14e-09 &  2.33e-16 &      \\ 
     &           &    6 &  8.13e-09 &  2.32e-16 &      \\ 
     &           &    7 &  8.12e-09 &  2.32e-16 &      \\ 
     &           &    8 &  8.11e-09 &  2.32e-16 &      \\ 
     &           &    9 &  8.10e-09 &  2.31e-16 &      \\ 
     &           &   10 &  6.96e-08 &  2.31e-16 &      \\ 
3881 &  3.88e+04 &   10 &           &           & iters  \\ 
 \hdashline 
     &           &    1 &  9.28e-08 &  1.24e-09 &      \\ 
     &           &    2 &  6.27e-08 &  2.65e-15 &      \\ 
     &           &    3 &  1.51e-08 &  1.79e-15 &      \\ 
     &           &    4 &  1.51e-08 &  4.31e-16 &      \\ 
     &           &    5 &  1.50e-08 &  4.30e-16 &      \\ 
     &           &    6 &  6.27e-08 &  4.29e-16 &      \\ 
     &           &    7 &  1.51e-08 &  1.79e-15 &      \\ 
     &           &    8 &  1.51e-08 &  4.31e-16 &      \\ 
     &           &    9 &  1.51e-08 &  4.31e-16 &      \\ 
     &           &   10 &  1.51e-08 &  4.30e-16 &      \\ 
3882 &  3.88e+04 &   10 &           &           & iters  \\ 
 \hdashline 
     &           &    1 &  2.63e-09 &  7.42e-10 &      \\ 
     &           &    2 &  2.62e-09 &  7.50e-17 &      \\ 
     &           &    3 &  2.62e-09 &  7.48e-17 &      \\ 
     &           &    4 &  2.61e-09 &  7.47e-17 &      \\ 
     &           &    5 &  2.61e-09 &  7.46e-17 &      \\ 
     &           &    6 &  2.61e-09 &  7.45e-17 &      \\ 
     &           &    7 &  2.60e-09 &  7.44e-17 &      \\ 
     &           &    8 &  2.60e-09 &  7.43e-17 &      \\ 
     &           &    9 &  2.60e-09 &  7.42e-17 &      \\ 
     &           &   10 &  2.59e-09 &  7.41e-17 &      \\ 
3883 &  3.88e+04 &   10 &           &           & iters  \\ 
 \hdashline 
     &           &    1 &  5.11e-08 &  4.66e-10 &      \\ 
     &           &    2 &  2.67e-08 &  1.46e-15 &      \\ 
     &           &    3 &  2.67e-08 &  7.62e-16 &      \\ 
     &           &    4 &  5.11e-08 &  7.61e-16 &      \\ 
     &           &    5 &  2.67e-08 &  1.46e-15 &      \\ 
     &           &    6 &  2.67e-08 &  7.62e-16 &      \\ 
     &           &    7 &  5.11e-08 &  7.61e-16 &      \\ 
     &           &    8 &  2.67e-08 &  1.46e-15 &      \\ 
     &           &    9 &  2.67e-08 &  7.62e-16 &      \\ 
     &           &   10 &  5.11e-08 &  7.61e-16 &      \\ 
3884 &  3.88e+04 &   10 &           &           & iters  \\ 
 \hdashline 
     &           &    1 &  7.18e-08 &  4.68e-10 &      \\ 
     &           &    2 &  8.36e-08 &  2.05e-15 &      \\ 
     &           &    3 &  7.19e-08 &  2.39e-15 &      \\ 
     &           &    4 &  5.93e-09 &  2.05e-15 &      \\ 
     &           &    5 &  5.92e-09 &  1.69e-16 &      \\ 
     &           &    6 &  5.91e-09 &  1.69e-16 &      \\ 
     &           &    7 &  5.90e-09 &  1.69e-16 &      \\ 
     &           &    8 &  5.89e-09 &  1.68e-16 &      \\ 
     &           &    9 &  5.89e-09 &  1.68e-16 &      \\ 
     &           &   10 &  5.88e-09 &  1.68e-16 &      \\ 
3885 &  3.88e+04 &   10 &           &           & iters  \\ 
 \hdashline 
     &           &    1 &  7.90e-09 &  1.26e-10 &      \\ 
     &           &    2 &  7.89e-09 &  2.26e-16 &      \\ 
     &           &    3 &  7.88e-09 &  2.25e-16 &      \\ 
     &           &    4 &  7.87e-09 &  2.25e-16 &      \\ 
     &           &    5 &  7.86e-09 &  2.25e-16 &      \\ 
     &           &    6 &  7.85e-09 &  2.24e-16 &      \\ 
     &           &    7 &  7.84e-09 &  2.24e-16 &      \\ 
     &           &    8 &  6.99e-08 &  2.24e-16 &      \\ 
     &           &    9 &  4.68e-08 &  1.99e-15 &      \\ 
     &           &   10 &  7.86e-09 &  1.33e-15 &      \\ 
3886 &  3.88e+04 &   10 &           &           & iters  \\ 
 \hdashline 
     &           &    1 &  1.58e-09 &  5.34e-10 &      \\ 
     &           &    2 &  1.58e-09 &  4.52e-17 &      \\ 
     &           &    3 &  1.58e-09 &  4.52e-17 &      \\ 
     &           &    4 &  1.58e-09 &  4.51e-17 &      \\ 
     &           &    5 &  1.58e-09 &  4.50e-17 &      \\ 
     &           &    6 &  1.57e-09 &  4.50e-17 &      \\ 
     &           &    7 &  1.57e-09 &  4.49e-17 &      \\ 
     &           &    8 &  1.57e-09 &  4.48e-17 &      \\ 
     &           &    9 &  1.57e-09 &  4.48e-17 &      \\ 
     &           &   10 &  1.56e-09 &  4.47e-17 &      \\ 
3887 &  3.89e+04 &   10 &           &           & iters  \\ 
 \hdashline 
     &           &    1 &  5.21e-08 &  7.45e-10 &      \\ 
     &           &    2 &  2.56e-08 &  1.49e-15 &      \\ 
     &           &    3 &  2.56e-08 &  7.32e-16 &      \\ 
     &           &    4 &  5.21e-08 &  7.31e-16 &      \\ 
     &           &    5 &  2.56e-08 &  1.49e-15 &      \\ 
     &           &    6 &  2.56e-08 &  7.32e-16 &      \\ 
     &           &    7 &  5.21e-08 &  7.31e-16 &      \\ 
     &           &    8 &  2.56e-08 &  1.49e-15 &      \\ 
     &           &    9 &  2.56e-08 &  7.32e-16 &      \\ 
     &           &   10 &  5.21e-08 &  7.31e-16 &      \\ 
3888 &  3.89e+04 &   10 &           &           & iters  \\ 
 \hdashline 
     &           &    1 &  4.05e-08 &  6.49e-10 &      \\ 
     &           &    2 &  3.72e-08 &  1.16e-15 &      \\ 
     &           &    3 &  4.05e-08 &  1.06e-15 &      \\ 
     &           &    4 &  3.72e-08 &  1.16e-15 &      \\ 
     &           &    5 &  4.05e-08 &  1.06e-15 &      \\ 
     &           &    6 &  3.72e-08 &  1.16e-15 &      \\ 
     &           &    7 &  4.05e-08 &  1.06e-15 &      \\ 
     &           &    8 &  3.72e-08 &  1.16e-15 &      \\ 
     &           &    9 &  4.05e-08 &  1.06e-15 &      \\ 
     &           &   10 &  3.72e-08 &  1.16e-15 &      \\ 
3889 &  3.89e+04 &   10 &           &           & iters  \\ 
 \hdashline 
     &           &    1 &  5.19e-08 &  4.21e-10 &      \\ 
     &           &    2 &  2.59e-08 &  1.48e-15 &      \\ 
     &           &    3 &  2.58e-08 &  7.38e-16 &      \\ 
     &           &    4 &  5.19e-08 &  7.37e-16 &      \\ 
     &           &    5 &  2.59e-08 &  1.48e-15 &      \\ 
     &           &    6 &  2.58e-08 &  7.38e-16 &      \\ 
     &           &    7 &  5.19e-08 &  7.37e-16 &      \\ 
     &           &    8 &  2.59e-08 &  1.48e-15 &      \\ 
     &           &    9 &  2.58e-08 &  7.38e-16 &      \\ 
     &           &   10 &  5.19e-08 &  7.37e-16 &      \\ 
3890 &  3.89e+04 &   10 &           &           & iters  \\ 
 \hdashline 
     &           &    1 &  2.29e-08 &  5.38e-10 &      \\ 
     &           &    2 &  2.28e-08 &  6.53e-16 &      \\ 
     &           &    3 &  2.28e-08 &  6.52e-16 &      \\ 
     &           &    4 &  5.49e-08 &  6.51e-16 &      \\ 
     &           &    5 &  2.28e-08 &  1.57e-15 &      \\ 
     &           &    6 &  2.28e-08 &  6.52e-16 &      \\ 
     &           &    7 &  5.49e-08 &  6.51e-16 &      \\ 
     &           &    8 &  2.29e-08 &  1.57e-15 &      \\ 
     &           &    9 &  2.28e-08 &  6.52e-16 &      \\ 
     &           &   10 &  5.49e-08 &  6.51e-16 &      \\ 
3891 &  3.89e+04 &   10 &           &           & iters  \\ 
 \hdashline 
     &           &    1 &  6.02e-10 &  8.83e-10 &      \\ 
     &           &    2 &  6.01e-10 &  1.72e-17 &      \\ 
     &           &    3 &  6.00e-10 &  1.71e-17 &      \\ 
     &           &    4 &  5.99e-10 &  1.71e-17 &      \\ 
     &           &    5 &  5.98e-10 &  1.71e-17 &      \\ 
     &           &    6 &  5.97e-10 &  1.71e-17 &      \\ 
     &           &    7 &  5.97e-10 &  1.70e-17 &      \\ 
     &           &    8 &  5.96e-10 &  1.70e-17 &      \\ 
     &           &    9 &  5.95e-10 &  1.70e-17 &      \\ 
     &           &   10 &  5.94e-10 &  1.70e-17 &      \\ 
3892 &  3.89e+04 &   10 &           &           & iters  \\ 
 \hdashline 
     &           &    1 &  1.19e-08 &  2.28e-11 &      \\ 
     &           &    2 &  1.19e-08 &  3.40e-16 &      \\ 
     &           &    3 &  1.19e-08 &  3.40e-16 &      \\ 
     &           &    4 &  1.19e-08 &  3.39e-16 &      \\ 
     &           &    5 &  1.19e-08 &  3.39e-16 &      \\ 
     &           &    6 &  6.58e-08 &  3.38e-16 &      \\ 
     &           &    7 &  8.96e-08 &  1.88e-15 &      \\ 
     &           &    8 &  6.59e-08 &  2.56e-15 &      \\ 
     &           &    9 &  1.19e-08 &  1.88e-15 &      \\ 
     &           &   10 &  1.19e-08 &  3.40e-16 &      \\ 
3893 &  3.89e+04 &   10 &           &           & iters  \\ 
 \hdashline 
     &           &    1 &  2.83e-08 &  2.20e-10 &      \\ 
     &           &    2 &  4.94e-08 &  8.07e-16 &      \\ 
     &           &    3 &  2.83e-08 &  1.41e-15 &      \\ 
     &           &    4 &  2.83e-08 &  8.08e-16 &      \\ 
     &           &    5 &  4.94e-08 &  8.07e-16 &      \\ 
     &           &    6 &  2.83e-08 &  1.41e-15 &      \\ 
     &           &    7 &  4.94e-08 &  8.08e-16 &      \\ 
     &           &    8 &  2.83e-08 &  1.41e-15 &      \\ 
     &           &    9 &  2.83e-08 &  8.09e-16 &      \\ 
     &           &   10 &  4.94e-08 &  8.08e-16 &      \\ 
3894 &  3.89e+04 &   10 &           &           & iters  \\ 
 \hdashline 
     &           &    1 &  5.83e-09 &  4.90e-10 &      \\ 
     &           &    2 &  5.83e-09 &  1.66e-16 &      \\ 
     &           &    3 &  5.82e-09 &  1.66e-16 &      \\ 
     &           &    4 &  5.81e-09 &  1.66e-16 &      \\ 
     &           &    5 &  5.80e-09 &  1.66e-16 &      \\ 
     &           &    6 &  5.79e-09 &  1.66e-16 &      \\ 
     &           &    7 &  7.19e-08 &  1.65e-16 &      \\ 
     &           &    8 &  8.36e-08 &  2.05e-15 &      \\ 
     &           &    9 &  7.19e-08 &  2.39e-15 &      \\ 
     &           &   10 &  5.87e-09 &  2.05e-15 &      \\ 
3895 &  3.89e+04 &   10 &           &           & iters  \\ 
 \hdashline 
     &           &    1 &  5.31e-08 &  9.98e-10 &      \\ 
     &           &    2 &  2.47e-08 &  1.52e-15 &      \\ 
     &           &    3 &  2.46e-08 &  7.04e-16 &      \\ 
     &           &    4 &  5.31e-08 &  7.03e-16 &      \\ 
     &           &    5 &  2.47e-08 &  1.52e-15 &      \\ 
     &           &    6 &  2.46e-08 &  7.04e-16 &      \\ 
     &           &    7 &  5.31e-08 &  7.03e-16 &      \\ 
     &           &    8 &  2.47e-08 &  1.51e-15 &      \\ 
     &           &    9 &  2.46e-08 &  7.04e-16 &      \\ 
     &           &   10 &  5.31e-08 &  7.03e-16 &      \\ 
3896 &  3.90e+04 &   10 &           &           & iters  \\ 
 \hdashline 
     &           &    1 &  5.74e-08 &  1.14e-09 &      \\ 
     &           &    2 &  2.04e-08 &  1.64e-15 &      \\ 
     &           &    3 &  2.04e-08 &  5.82e-16 &      \\ 
     &           &    4 &  2.03e-08 &  5.81e-16 &      \\ 
     &           &    5 &  5.74e-08 &  5.80e-16 &      \\ 
     &           &    6 &  2.04e-08 &  1.64e-15 &      \\ 
     &           &    7 &  2.03e-08 &  5.82e-16 &      \\ 
     &           &    8 &  2.03e-08 &  5.81e-16 &      \\ 
     &           &    9 &  5.74e-08 &  5.80e-16 &      \\ 
     &           &   10 &  2.04e-08 &  1.64e-15 &      \\ 
3897 &  3.90e+04 &   10 &           &           & iters  \\ 
 \hdashline 
     &           &    1 &  8.70e-09 &  7.14e-10 &      \\ 
     &           &    2 &  8.69e-09 &  2.48e-16 &      \\ 
     &           &    3 &  8.67e-09 &  2.48e-16 &      \\ 
     &           &    4 &  8.66e-09 &  2.48e-16 &      \\ 
     &           &    5 &  8.65e-09 &  2.47e-16 &      \\ 
     &           &    6 &  8.64e-09 &  2.47e-16 &      \\ 
     &           &    7 &  6.91e-08 &  2.47e-16 &      \\ 
     &           &    8 &  8.72e-09 &  1.97e-15 &      \\ 
     &           &    9 &  8.71e-09 &  2.49e-16 &      \\ 
     &           &   10 &  8.70e-09 &  2.49e-16 &      \\ 
3898 &  3.90e+04 &   10 &           &           & iters  \\ 
 \hdashline 
     &           &    1 &  3.29e-08 &  4.25e-10 &      \\ 
     &           &    2 &  4.48e-08 &  9.39e-16 &      \\ 
     &           &    3 &  3.29e-08 &  1.28e-15 &      \\ 
     &           &    4 &  4.48e-08 &  9.40e-16 &      \\ 
     &           &    5 &  3.29e-08 &  1.28e-15 &      \\ 
     &           &    6 &  4.48e-08 &  9.40e-16 &      \\ 
     &           &    7 &  3.30e-08 &  1.28e-15 &      \\ 
     &           &    8 &  3.29e-08 &  9.41e-16 &      \\ 
     &           &    9 &  4.48e-08 &  9.39e-16 &      \\ 
     &           &   10 &  3.29e-08 &  1.28e-15 &      \\ 
3899 &  3.90e+04 &   10 &           &           & iters  \\ 
 \hdashline 
     &           &    1 &  3.84e-08 &  9.55e-10 &      \\ 
     &           &    2 &  5.45e-10 &  1.09e-15 &      \\ 
     &           &    3 &  5.44e-10 &  1.56e-17 &      \\ 
     &           &    4 &  5.44e-10 &  1.55e-17 &      \\ 
     &           &    5 &  5.43e-10 &  1.55e-17 &      \\ 
     &           &    6 &  5.42e-10 &  1.55e-17 &      \\ 
     &           &    7 &  5.41e-10 &  1.55e-17 &      \\ 
     &           &    8 &  5.40e-10 &  1.54e-17 &      \\ 
     &           &    9 &  5.40e-10 &  1.54e-17 &      \\ 
     &           &   10 &  5.39e-10 &  1.54e-17 &      \\ 
3900 &  3.90e+04 &   10 &           &           & iters  \\ 
 \hdashline 
     &           &    1 &  6.04e-08 &  7.56e-10 &      \\ 
     &           &    2 &  1.74e-08 &  1.72e-15 &      \\ 
     &           &    3 &  1.73e-08 &  4.95e-16 &      \\ 
     &           &    4 &  1.73e-08 &  4.95e-16 &      \\ 
     &           &    5 &  1.73e-08 &  4.94e-16 &      \\ 
     &           &    6 &  6.04e-08 &  4.93e-16 &      \\ 
     &           &    7 &  1.73e-08 &  1.72e-15 &      \\ 
     &           &    8 &  1.73e-08 &  4.95e-16 &      \\ 
     &           &    9 &  1.73e-08 &  4.94e-16 &      \\ 
     &           &   10 &  6.04e-08 &  4.93e-16 &      \\ 
3901 &  3.90e+04 &   10 &           &           & iters  \\ 
 \hdashline 
     &           &    1 &  1.83e-09 &  7.24e-10 &      \\ 
     &           &    2 &  1.83e-09 &  5.22e-17 &      \\ 
     &           &    3 &  1.82e-09 &  5.21e-17 &      \\ 
     &           &    4 &  1.82e-09 &  5.21e-17 &      \\ 
     &           &    5 &  1.82e-09 &  5.20e-17 &      \\ 
     &           &    6 &  1.82e-09 &  5.19e-17 &      \\ 
     &           &    7 &  1.81e-09 &  5.19e-17 &      \\ 
     &           &    8 &  1.81e-09 &  5.18e-17 &      \\ 
     &           &    9 &  1.81e-09 &  5.17e-17 &      \\ 
     &           &   10 &  1.81e-09 &  5.16e-17 &      \\ 
3902 &  3.90e+04 &   10 &           &           & iters  \\ 
 \hdashline 
     &           &    1 &  2.33e-08 &  4.79e-11 &      \\ 
     &           &    2 &  2.32e-08 &  6.64e-16 &      \\ 
     &           &    3 &  5.45e-08 &  6.63e-16 &      \\ 
     &           &    4 &  2.33e-08 &  1.55e-15 &      \\ 
     &           &    5 &  2.33e-08 &  6.65e-16 &      \\ 
     &           &    6 &  2.32e-08 &  6.64e-16 &      \\ 
     &           &    7 &  5.45e-08 &  6.63e-16 &      \\ 
     &           &    8 &  2.33e-08 &  1.56e-15 &      \\ 
     &           &    9 &  2.32e-08 &  6.64e-16 &      \\ 
     &           &   10 &  5.45e-08 &  6.63e-16 &      \\ 
3903 &  3.90e+04 &   10 &           &           & iters  \\ 
 \hdashline 
     &           &    1 &  3.34e-09 &  7.42e-10 &      \\ 
     &           &    2 &  3.33e-09 &  9.53e-17 &      \\ 
     &           &    3 &  3.33e-09 &  9.51e-17 &      \\ 
     &           &    4 &  3.32e-09 &  9.50e-17 &      \\ 
     &           &    5 &  7.44e-08 &  9.49e-17 &      \\ 
     &           &    6 &  4.23e-08 &  2.12e-15 &      \\ 
     &           &    7 &  3.37e-09 &  1.21e-15 &      \\ 
     &           &    8 &  3.36e-09 &  9.60e-17 &      \\ 
     &           &    9 &  3.36e-09 &  9.59e-17 &      \\ 
     &           &   10 &  3.35e-09 &  9.58e-17 &      \\ 
3904 &  3.90e+04 &   10 &           &           & iters  \\ 
 \hdashline 
     &           &    1 &  1.24e-08 &  8.74e-12 &      \\ 
     &           &    2 &  1.24e-08 &  3.54e-16 &      \\ 
     &           &    3 &  6.53e-08 &  3.53e-16 &      \\ 
     &           &    4 &  1.25e-08 &  1.86e-15 &      \\ 
     &           &    5 &  1.24e-08 &  3.55e-16 &      \\ 
     &           &    6 &  1.24e-08 &  3.55e-16 &      \\ 
     &           &    7 &  1.24e-08 &  3.54e-16 &      \\ 
     &           &    8 &  1.24e-08 &  3.54e-16 &      \\ 
     &           &    9 &  6.53e-08 &  3.53e-16 &      \\ 
     &           &   10 &  1.25e-08 &  1.86e-15 &      \\ 
3905 &  3.90e+04 &   10 &           &           & iters  \\ 
 \hdashline 
     &           &    1 &  3.28e-08 &  5.07e-10 &      \\ 
     &           &    2 &  4.49e-08 &  9.36e-16 &      \\ 
     &           &    3 &  3.28e-08 &  1.28e-15 &      \\ 
     &           &    4 &  4.49e-08 &  9.37e-16 &      \\ 
     &           &    5 &  3.28e-08 &  1.28e-15 &      \\ 
     &           &    6 &  3.28e-08 &  9.37e-16 &      \\ 
     &           &    7 &  4.49e-08 &  9.36e-16 &      \\ 
     &           &    8 &  3.28e-08 &  1.28e-15 &      \\ 
     &           &    9 &  4.49e-08 &  9.36e-16 &      \\ 
     &           &   10 &  3.28e-08 &  1.28e-15 &      \\ 
3906 &  3.90e+04 &   10 &           &           & iters  \\ 
 \hdashline 
     &           &    1 &  1.15e-08 &  2.35e-10 &      \\ 
     &           &    2 &  1.15e-08 &  3.28e-16 &      \\ 
     &           &    3 &  1.15e-08 &  3.27e-16 &      \\ 
     &           &    4 &  6.63e-08 &  3.27e-16 &      \\ 
     &           &    5 &  1.15e-08 &  1.89e-15 &      \\ 
     &           &    6 &  1.15e-08 &  3.29e-16 &      \\ 
     &           &    7 &  1.15e-08 &  3.29e-16 &      \\ 
     &           &    8 &  1.15e-08 &  3.28e-16 &      \\ 
     &           &    9 &  1.15e-08 &  3.28e-16 &      \\ 
     &           &   10 &  1.14e-08 &  3.27e-16 &      \\ 
3907 &  3.91e+04 &   10 &           &           & iters  \\ 
 \hdashline 
     &           &    1 &  7.09e-08 &  5.53e-12 &      \\ 
     &           &    2 &  6.89e-09 &  2.02e-15 &      \\ 
     &           &    3 &  6.88e-09 &  1.97e-16 &      \\ 
     &           &    4 &  6.87e-09 &  1.96e-16 &      \\ 
     &           &    5 &  6.86e-09 &  1.96e-16 &      \\ 
     &           &    6 &  6.85e-09 &  1.96e-16 &      \\ 
     &           &    7 &  6.84e-09 &  1.95e-16 &      \\ 
     &           &    8 &  6.83e-09 &  1.95e-16 &      \\ 
     &           &    9 &  6.82e-09 &  1.95e-16 &      \\ 
     &           &   10 &  7.09e-08 &  1.95e-16 &      \\ 
3908 &  3.91e+04 &   10 &           &           & iters  \\ 
 \hdashline 
     &           &    1 &  8.74e-08 &  6.67e-10 &      \\ 
     &           &    2 &  6.81e-08 &  2.49e-15 &      \\ 
     &           &    3 &  9.69e-09 &  1.94e-15 &      \\ 
     &           &    4 &  9.67e-09 &  2.76e-16 &      \\ 
     &           &    5 &  9.66e-09 &  2.76e-16 &      \\ 
     &           &    6 &  9.65e-09 &  2.76e-16 &      \\ 
     &           &    7 &  9.63e-09 &  2.75e-16 &      \\ 
     &           &    8 &  6.81e-08 &  2.75e-16 &      \\ 
     &           &    9 &  8.74e-08 &  1.94e-15 &      \\ 
     &           &   10 &  6.81e-08 &  2.49e-15 &      \\ 
3909 &  3.91e+04 &   10 &           &           & iters  \\ 
 \hdashline 
     &           &    1 &  7.16e-09 &  8.01e-10 &      \\ 
     &           &    2 &  7.15e-09 &  2.04e-16 &      \\ 
     &           &    3 &  7.06e-08 &  2.04e-16 &      \\ 
     &           &    4 &  4.61e-08 &  2.01e-15 &      \\ 
     &           &    5 &  7.17e-09 &  1.32e-15 &      \\ 
     &           &    6 &  7.16e-09 &  2.05e-16 &      \\ 
     &           &    7 &  7.15e-09 &  2.04e-16 &      \\ 
     &           &    8 &  7.14e-09 &  2.04e-16 &      \\ 
     &           &    9 &  7.06e-08 &  2.04e-16 &      \\ 
     &           &   10 &  4.61e-08 &  2.01e-15 &      \\ 
3910 &  3.91e+04 &   10 &           &           & iters  \\ 
 \hdashline 
     &           &    1 &  2.34e-08 &  3.54e-10 &      \\ 
     &           &    2 &  5.44e-08 &  6.67e-16 &      \\ 
     &           &    3 &  2.34e-08 &  1.55e-15 &      \\ 
     &           &    4 &  2.34e-08 &  6.68e-16 &      \\ 
     &           &    5 &  5.43e-08 &  6.67e-16 &      \\ 
     &           &    6 &  2.34e-08 &  1.55e-15 &      \\ 
     &           &    7 &  2.34e-08 &  6.68e-16 &      \\ 
     &           &    8 &  5.43e-08 &  6.67e-16 &      \\ 
     &           &    9 &  2.34e-08 &  1.55e-15 &      \\ 
     &           &   10 &  2.34e-08 &  6.69e-16 &      \\ 
3911 &  3.91e+04 &   10 &           &           & iters  \\ 
 \hdashline 
     &           &    1 &  1.77e-08 &  9.02e-10 &      \\ 
     &           &    2 &  1.77e-08 &  5.06e-16 &      \\ 
     &           &    3 &  6.00e-08 &  5.06e-16 &      \\ 
     &           &    4 &  1.78e-08 &  1.71e-15 &      \\ 
     &           &    5 &  1.78e-08 &  5.07e-16 &      \\ 
     &           &    6 &  1.77e-08 &  5.07e-16 &      \\ 
     &           &    7 &  1.77e-08 &  5.06e-16 &      \\ 
     &           &    8 &  6.00e-08 &  5.05e-16 &      \\ 
     &           &    9 &  1.78e-08 &  1.71e-15 &      \\ 
     &           &   10 &  1.77e-08 &  5.07e-16 &      \\ 
3912 &  3.91e+04 &   10 &           &           & iters  \\ 
 \hdashline 
     &           &    1 &  4.10e-08 &  6.20e-10 &      \\ 
     &           &    2 &  3.67e-08 &  1.17e-15 &      \\ 
     &           &    3 &  4.10e-08 &  1.05e-15 &      \\ 
     &           &    4 &  3.67e-08 &  1.17e-15 &      \\ 
     &           &    5 &  4.10e-08 &  1.05e-15 &      \\ 
     &           &    6 &  3.67e-08 &  1.17e-15 &      \\ 
     &           &    7 &  4.10e-08 &  1.05e-15 &      \\ 
     &           &    8 &  3.67e-08 &  1.17e-15 &      \\ 
     &           &    9 &  4.10e-08 &  1.05e-15 &      \\ 
     &           &   10 &  3.67e-08 &  1.17e-15 &      \\ 
3913 &  3.91e+04 &   10 &           &           & iters  \\ 
 \hdashline 
     &           &    1 &  1.30e-08 &  5.54e-10 &      \\ 
     &           &    2 &  1.30e-08 &  3.71e-16 &      \\ 
     &           &    3 &  1.30e-08 &  3.70e-16 &      \\ 
     &           &    4 &  6.47e-08 &  3.70e-16 &      \\ 
     &           &    5 &  9.07e-08 &  1.85e-15 &      \\ 
     &           &    6 &  6.48e-08 &  2.59e-15 &      \\ 
     &           &    7 &  1.30e-08 &  1.85e-15 &      \\ 
     &           &    8 &  1.30e-08 &  3.71e-16 &      \\ 
     &           &    9 &  1.30e-08 &  3.70e-16 &      \\ 
     &           &   10 &  6.47e-08 &  3.70e-16 &      \\ 
3914 &  3.91e+04 &   10 &           &           & iters  \\ 
 \hdashline 
     &           &    1 &  2.33e-08 &  2.11e-10 &      \\ 
     &           &    2 &  2.33e-08 &  6.65e-16 &      \\ 
     &           &    3 &  2.32e-08 &  6.64e-16 &      \\ 
     &           &    4 &  5.45e-08 &  6.63e-16 &      \\ 
     &           &    5 &  2.33e-08 &  1.56e-15 &      \\ 
     &           &    6 &  2.32e-08 &  6.64e-16 &      \\ 
     &           &    7 &  5.45e-08 &  6.63e-16 &      \\ 
     &           &    8 &  2.33e-08 &  1.55e-15 &      \\ 
     &           &    9 &  2.32e-08 &  6.65e-16 &      \\ 
     &           &   10 &  5.45e-08 &  6.64e-16 &      \\ 
3915 &  3.91e+04 &   10 &           &           & iters  \\ 
 \hdashline 
     &           &    1 &  4.67e-08 &  9.83e-12 &      \\ 
     &           &    2 &  3.10e-08 &  1.33e-15 &      \\ 
     &           &    3 &  4.67e-08 &  8.86e-16 &      \\ 
     &           &    4 &  3.11e-08 &  1.33e-15 &      \\ 
     &           &    5 &  3.10e-08 &  8.87e-16 &      \\ 
     &           &    6 &  4.67e-08 &  8.86e-16 &      \\ 
     &           &    7 &  3.10e-08 &  1.33e-15 &      \\ 
     &           &    8 &  4.67e-08 &  8.86e-16 &      \\ 
     &           &    9 &  3.11e-08 &  1.33e-15 &      \\ 
     &           &   10 &  3.10e-08 &  8.87e-16 &      \\ 
3916 &  3.92e+04 &   10 &           &           & iters  \\ 
 \hdashline 
     &           &    1 &  2.90e-08 &  3.13e-10 &      \\ 
     &           &    2 &  2.90e-08 &  8.28e-16 &      \\ 
     &           &    3 &  2.89e-08 &  8.27e-16 &      \\ 
     &           &    4 &  4.88e-08 &  8.26e-16 &      \\ 
     &           &    5 &  2.90e-08 &  1.39e-15 &      \\ 
     &           &    6 &  2.89e-08 &  8.27e-16 &      \\ 
     &           &    7 &  4.88e-08 &  8.25e-16 &      \\ 
     &           &    8 &  2.90e-08 &  1.39e-15 &      \\ 
     &           &    9 &  2.89e-08 &  8.26e-16 &      \\ 
     &           &   10 &  4.88e-08 &  8.25e-16 &      \\ 
3917 &  3.92e+04 &   10 &           &           & iters  \\ 
 \hdashline 
     &           &    1 &  5.46e-08 &  2.78e-10 &      \\ 
     &           &    2 &  2.31e-08 &  1.56e-15 &      \\ 
     &           &    3 &  2.31e-08 &  6.61e-16 &      \\ 
     &           &    4 &  2.31e-08 &  6.60e-16 &      \\ 
     &           &    5 &  5.46e-08 &  6.59e-16 &      \\ 
     &           &    6 &  2.31e-08 &  1.56e-15 &      \\ 
     &           &    7 &  2.31e-08 &  6.60e-16 &      \\ 
     &           &    8 &  5.46e-08 &  6.59e-16 &      \\ 
     &           &    9 &  2.31e-08 &  1.56e-15 &      \\ 
     &           &   10 &  2.31e-08 &  6.60e-16 &      \\ 
3918 &  3.92e+04 &   10 &           &           & iters  \\ 
 \hdashline 
     &           &    1 &  1.97e-08 &  1.33e-10 &      \\ 
     &           &    2 &  5.80e-08 &  5.62e-16 &      \\ 
     &           &    3 &  1.97e-08 &  1.66e-15 &      \\ 
     &           &    4 &  1.97e-08 &  5.63e-16 &      \\ 
     &           &    5 &  1.97e-08 &  5.63e-16 &      \\ 
     &           &    6 &  5.80e-08 &  5.62e-16 &      \\ 
     &           &    7 &  1.97e-08 &  1.66e-15 &      \\ 
     &           &    8 &  1.97e-08 &  5.63e-16 &      \\ 
     &           &    9 &  1.97e-08 &  5.63e-16 &      \\ 
     &           &   10 &  5.80e-08 &  5.62e-16 &      \\ 
3919 &  3.92e+04 &   10 &           &           & iters  \\ 
 \hdashline 
     &           &    1 &  5.60e-08 &  5.79e-10 &      \\ 
     &           &    2 &  2.17e-08 &  1.60e-15 &      \\ 
     &           &    3 &  2.17e-08 &  6.21e-16 &      \\ 
     &           &    4 &  5.60e-08 &  6.20e-16 &      \\ 
     &           &    5 &  2.18e-08 &  1.60e-15 &      \\ 
     &           &    6 &  2.17e-08 &  6.21e-16 &      \\ 
     &           &    7 &  2.17e-08 &  6.20e-16 &      \\ 
     &           &    8 &  5.60e-08 &  6.19e-16 &      \\ 
     &           &    9 &  2.18e-08 &  1.60e-15 &      \\ 
     &           &   10 &  2.17e-08 &  6.21e-16 &      \\ 
3920 &  3.92e+04 &   10 &           &           & iters  \\ 
 \hdashline 
     &           &    1 &  4.88e-08 &  6.01e-10 &      \\ 
     &           &    2 &  2.90e-08 &  1.39e-15 &      \\ 
     &           &    3 &  2.89e-08 &  8.27e-16 &      \\ 
     &           &    4 &  4.88e-08 &  8.26e-16 &      \\ 
     &           &    5 &  2.90e-08 &  1.39e-15 &      \\ 
     &           &    6 &  2.89e-08 &  8.27e-16 &      \\ 
     &           &    7 &  4.88e-08 &  8.26e-16 &      \\ 
     &           &    8 &  2.90e-08 &  1.39e-15 &      \\ 
     &           &    9 &  4.88e-08 &  8.26e-16 &      \\ 
     &           &   10 &  2.90e-08 &  1.39e-15 &      \\ 
3921 &  3.92e+04 &   10 &           &           & iters  \\ 
 \hdashline 
     &           &    1 &  2.43e-08 &  9.02e-11 &      \\ 
     &           &    2 &  2.43e-08 &  6.95e-16 &      \\ 
     &           &    3 &  5.34e-08 &  6.94e-16 &      \\ 
     &           &    4 &  2.43e-08 &  1.52e-15 &      \\ 
     &           &    5 &  2.43e-08 &  6.95e-16 &      \\ 
     &           &    6 &  5.34e-08 &  6.94e-16 &      \\ 
     &           &    7 &  2.44e-08 &  1.52e-15 &      \\ 
     &           &    8 &  2.43e-08 &  6.95e-16 &      \\ 
     &           &    9 &  5.34e-08 &  6.94e-16 &      \\ 
     &           &   10 &  2.44e-08 &  1.52e-15 &      \\ 
3922 &  3.92e+04 &   10 &           &           & iters  \\ 
 \hdashline 
     &           &    1 &  5.65e-08 &  3.90e-10 &      \\ 
     &           &    2 &  2.13e-08 &  1.61e-15 &      \\ 
     &           &    3 &  2.13e-08 &  6.08e-16 &      \\ 
     &           &    4 &  5.65e-08 &  6.07e-16 &      \\ 
     &           &    5 &  2.13e-08 &  1.61e-15 &      \\ 
     &           &    6 &  2.13e-08 &  6.08e-16 &      \\ 
     &           &    7 &  2.13e-08 &  6.08e-16 &      \\ 
     &           &    8 &  5.65e-08 &  6.07e-16 &      \\ 
     &           &    9 &  2.13e-08 &  1.61e-15 &      \\ 
     &           &   10 &  2.13e-08 &  6.08e-16 &      \\ 
3923 &  3.92e+04 &   10 &           &           & iters  \\ 
 \hdashline 
     &           &    1 &  3.84e-08 &  1.34e-10 &      \\ 
     &           &    2 &  3.94e-08 &  1.10e-15 &      \\ 
     &           &    3 &  3.84e-08 &  1.12e-15 &      \\ 
     &           &    4 &  3.94e-08 &  1.10e-15 &      \\ 
     &           &    5 &  3.84e-08 &  1.12e-15 &      \\ 
     &           &    6 &  3.94e-08 &  1.10e-15 &      \\ 
     &           &    7 &  3.84e-08 &  1.12e-15 &      \\ 
     &           &    8 &  3.94e-08 &  1.10e-15 &      \\ 
     &           &    9 &  3.84e-08 &  1.12e-15 &      \\ 
     &           &   10 &  3.94e-08 &  1.10e-15 &      \\ 
3924 &  3.92e+04 &   10 &           &           & iters  \\ 
 \hdashline 
     &           &    1 &  1.65e-08 &  5.13e-10 &      \\ 
     &           &    2 &  1.65e-08 &  4.72e-16 &      \\ 
     &           &    3 &  1.65e-08 &  4.72e-16 &      \\ 
     &           &    4 &  1.65e-08 &  4.71e-16 &      \\ 
     &           &    5 &  6.12e-08 &  4.70e-16 &      \\ 
     &           &    6 &  1.65e-08 &  1.75e-15 &      \\ 
     &           &    7 &  1.65e-08 &  4.72e-16 &      \\ 
     &           &    8 &  1.65e-08 &  4.71e-16 &      \\ 
     &           &    9 &  6.12e-08 &  4.71e-16 &      \\ 
     &           &   10 &  1.66e-08 &  1.75e-15 &      \\ 
3925 &  3.92e+04 &   10 &           &           & iters  \\ 
 \hdashline 
     &           &    1 &  9.18e-09 &  8.88e-10 &      \\ 
     &           &    2 &  9.17e-09 &  2.62e-16 &      \\ 
     &           &    3 &  9.15e-09 &  2.62e-16 &      \\ 
     &           &    4 &  9.14e-09 &  2.61e-16 &      \\ 
     &           &    5 &  9.13e-09 &  2.61e-16 &      \\ 
     &           &    6 &  6.86e-08 &  2.60e-16 &      \\ 
     &           &    7 &  4.81e-08 &  1.96e-15 &      \\ 
     &           &    8 &  9.14e-09 &  1.37e-15 &      \\ 
     &           &    9 &  9.13e-09 &  2.61e-16 &      \\ 
     &           &   10 &  6.86e-08 &  2.61e-16 &      \\ 
3926 &  3.92e+04 &   10 &           &           & iters  \\ 
 \hdashline 
     &           &    1 &  4.94e-08 &  6.46e-10 &      \\ 
     &           &    2 &  1.05e-08 &  1.41e-15 &      \\ 
     &           &    3 &  1.05e-08 &  3.00e-16 &      \\ 
     &           &    4 &  6.72e-08 &  3.00e-16 &      \\ 
     &           &    5 &  4.94e-08 &  1.92e-15 &      \\ 
     &           &    6 &  1.05e-08 &  1.41e-15 &      \\ 
     &           &    7 &  1.05e-08 &  3.00e-16 &      \\ 
     &           &    8 &  6.72e-08 &  3.00e-16 &      \\ 
     &           &    9 &  1.06e-08 &  1.92e-15 &      \\ 
     &           &   10 &  1.06e-08 &  3.02e-16 &      \\ 
3927 &  3.93e+04 &   10 &           &           & iters  \\ 
 \hdashline 
     &           &    1 &  1.54e-08 &  3.29e-10 &      \\ 
     &           &    2 &  2.35e-08 &  4.39e-16 &      \\ 
     &           &    3 &  1.54e-08 &  6.70e-16 &      \\ 
     &           &    4 &  1.54e-08 &  4.40e-16 &      \\ 
     &           &    5 &  2.35e-08 &  4.39e-16 &      \\ 
     &           &    6 &  1.54e-08 &  6.70e-16 &      \\ 
     &           &    7 &  2.35e-08 &  4.39e-16 &      \\ 
     &           &    8 &  1.54e-08 &  6.70e-16 &      \\ 
     &           &    9 &  1.54e-08 &  4.40e-16 &      \\ 
     &           &   10 &  2.35e-08 &  4.39e-16 &      \\ 
3928 &  3.93e+04 &   10 &           &           & iters  \\ 
 \hdashline 
     &           &    1 &  4.65e-08 &  5.77e-10 &      \\ 
     &           &    2 &  3.13e-08 &  1.33e-15 &      \\ 
     &           &    3 &  3.12e-08 &  8.93e-16 &      \\ 
     &           &    4 &  4.65e-08 &  8.91e-16 &      \\ 
     &           &    5 &  3.13e-08 &  1.33e-15 &      \\ 
     &           &    6 &  4.65e-08 &  8.92e-16 &      \\ 
     &           &    7 &  3.13e-08 &  1.33e-15 &      \\ 
     &           &    8 &  3.12e-08 &  8.93e-16 &      \\ 
     &           &    9 &  4.65e-08 &  8.91e-16 &      \\ 
     &           &   10 &  3.13e-08 &  1.33e-15 &      \\ 
3929 &  3.93e+04 &   10 &           &           & iters  \\ 
 \hdashline 
     &           &    1 &  5.04e-08 &  2.30e-10 &      \\ 
     &           &    2 &  2.73e-08 &  1.44e-15 &      \\ 
     &           &    3 &  2.73e-08 &  7.80e-16 &      \\ 
     &           &    4 &  5.04e-08 &  7.79e-16 &      \\ 
     &           &    5 &  2.73e-08 &  1.44e-15 &      \\ 
     &           &    6 &  2.73e-08 &  7.80e-16 &      \\ 
     &           &    7 &  5.04e-08 &  7.79e-16 &      \\ 
     &           &    8 &  2.73e-08 &  1.44e-15 &      \\ 
     &           &    9 &  2.73e-08 &  7.80e-16 &      \\ 
     &           &   10 &  5.05e-08 &  7.78e-16 &      \\ 
3930 &  3.93e+04 &   10 &           &           & iters  \\ 
 \hdashline 
     &           &    1 &  3.20e-08 &  4.84e-10 &      \\ 
     &           &    2 &  3.20e-08 &  9.14e-16 &      \\ 
     &           &    3 &  4.58e-08 &  9.13e-16 &      \\ 
     &           &    4 &  3.20e-08 &  1.31e-15 &      \\ 
     &           &    5 &  4.57e-08 &  9.13e-16 &      \\ 
     &           &    6 &  3.20e-08 &  1.31e-15 &      \\ 
     &           &    7 &  3.20e-08 &  9.14e-16 &      \\ 
     &           &    8 &  4.58e-08 &  9.12e-16 &      \\ 
     &           &    9 &  3.20e-08 &  1.31e-15 &      \\ 
     &           &   10 &  4.57e-08 &  9.13e-16 &      \\ 
3931 &  3.93e+04 &   10 &           &           & iters  \\ 
 \hdashline 
     &           &    1 &  4.22e-08 &  1.49e-10 &      \\ 
     &           &    2 &  3.56e-08 &  1.20e-15 &      \\ 
     &           &    3 &  4.21e-08 &  1.02e-15 &      \\ 
     &           &    4 &  3.56e-08 &  1.20e-15 &      \\ 
     &           &    5 &  4.21e-08 &  1.02e-15 &      \\ 
     &           &    6 &  3.56e-08 &  1.20e-15 &      \\ 
     &           &    7 &  4.21e-08 &  1.02e-15 &      \\ 
     &           &    8 &  3.56e-08 &  1.20e-15 &      \\ 
     &           &    9 &  4.21e-08 &  1.02e-15 &      \\ 
     &           &   10 &  3.56e-08 &  1.20e-15 &      \\ 
3932 &  3.93e+04 &   10 &           &           & iters  \\ 
 \hdashline 
     &           &    1 &  1.49e-08 &  3.37e-10 &      \\ 
     &           &    2 &  1.49e-08 &  4.25e-16 &      \\ 
     &           &    3 &  1.49e-08 &  4.25e-16 &      \\ 
     &           &    4 &  6.28e-08 &  4.24e-16 &      \\ 
     &           &    5 &  1.49e-08 &  1.79e-15 &      \\ 
     &           &    6 &  1.49e-08 &  4.26e-16 &      \\ 
     &           &    7 &  1.49e-08 &  4.26e-16 &      \\ 
     &           &    8 &  1.49e-08 &  4.25e-16 &      \\ 
     &           &    9 &  6.28e-08 &  4.24e-16 &      \\ 
     &           &   10 &  1.49e-08 &  1.79e-15 &      \\ 
3933 &  3.93e+04 &   10 &           &           & iters  \\ 
 \hdashline 
     &           &    1 &  3.50e-09 &  7.79e-10 &      \\ 
     &           &    2 &  3.49e-09 &  9.98e-17 &      \\ 
     &           &    3 &  3.49e-09 &  9.96e-17 &      \\ 
     &           &    4 &  3.48e-09 &  9.95e-17 &      \\ 
     &           &    5 &  3.48e-09 &  9.93e-17 &      \\ 
     &           &    6 &  3.47e-09 &  9.92e-17 &      \\ 
     &           &    7 &  3.47e-09 &  9.91e-17 &      \\ 
     &           &    8 &  3.46e-09 &  9.89e-17 &      \\ 
     &           &    9 &  3.54e-08 &  9.88e-17 &      \\ 
     &           &   10 &  3.51e-09 &  1.01e-15 &      \\ 
3934 &  3.93e+04 &   10 &           &           & iters  \\ 
 \hdashline 
     &           &    1 &  2.35e-08 &  6.91e-10 &      \\ 
     &           &    2 &  2.35e-08 &  6.71e-16 &      \\ 
     &           &    3 &  5.43e-08 &  6.70e-16 &      \\ 
     &           &    4 &  2.35e-08 &  1.55e-15 &      \\ 
     &           &    5 &  2.35e-08 &  6.71e-16 &      \\ 
     &           &    6 &  2.34e-08 &  6.70e-16 &      \\ 
     &           &    7 &  5.43e-08 &  6.69e-16 &      \\ 
     &           &    8 &  2.35e-08 &  1.55e-15 &      \\ 
     &           &    9 &  2.35e-08 &  6.70e-16 &      \\ 
     &           &   10 &  5.43e-08 &  6.69e-16 &      \\ 
3935 &  3.93e+04 &   10 &           &           & iters  \\ 
 \hdashline 
     &           &    1 &  1.47e-09 &  1.44e-10 &      \\ 
     &           &    2 &  1.47e-09 &  4.20e-17 &      \\ 
     &           &    3 &  1.47e-09 &  4.20e-17 &      \\ 
     &           &    4 &  1.47e-09 &  4.19e-17 &      \\ 
     &           &    5 &  1.46e-09 &  4.18e-17 &      \\ 
     &           &    6 &  1.46e-09 &  4.18e-17 &      \\ 
     &           &    7 &  1.46e-09 &  4.17e-17 &      \\ 
     &           &    8 &  1.46e-09 &  4.17e-17 &      \\ 
     &           &    9 &  1.46e-09 &  4.16e-17 &      \\ 
     &           &   10 &  1.45e-09 &  4.15e-17 &      \\ 
3936 &  3.94e+04 &   10 &           &           & iters  \\ 
 \hdashline 
     &           &    1 &  1.90e-08 &  1.60e-10 &      \\ 
     &           &    2 &  1.89e-08 &  5.41e-16 &      \\ 
     &           &    3 &  1.89e-08 &  5.40e-16 &      \\ 
     &           &    4 &  1.89e-08 &  5.39e-16 &      \\ 
     &           &    5 &  5.88e-08 &  5.39e-16 &      \\ 
     &           &    6 &  1.89e-08 &  1.68e-15 &      \\ 
     &           &    7 &  1.89e-08 &  5.40e-16 &      \\ 
     &           &    8 &  1.89e-08 &  5.39e-16 &      \\ 
     &           &    9 &  5.88e-08 &  5.39e-16 &      \\ 
     &           &   10 &  1.89e-08 &  1.68e-15 &      \\ 
3937 &  3.94e+04 &   10 &           &           & iters  \\ 
 \hdashline 
     &           &    1 &  3.80e-09 &  4.07e-11 &      \\ 
     &           &    2 &  3.80e-09 &  1.08e-16 &      \\ 
     &           &    3 &  3.79e-09 &  1.08e-16 &      \\ 
     &           &    4 &  3.78e-09 &  1.08e-16 &      \\ 
     &           &    5 &  3.78e-09 &  1.08e-16 &      \\ 
     &           &    6 &  3.77e-09 &  1.08e-16 &      \\ 
     &           &    7 &  3.77e-09 &  1.08e-16 &      \\ 
     &           &    8 &  3.76e-09 &  1.08e-16 &      \\ 
     &           &    9 &  3.76e-09 &  1.07e-16 &      \\ 
     &           &   10 &  3.75e-09 &  1.07e-16 &      \\ 
3938 &  3.94e+04 &   10 &           &           & iters  \\ 
 \hdashline 
     &           &    1 &  8.21e-08 &  1.71e-10 &      \\ 
     &           &    2 &  7.34e-08 &  2.34e-15 &      \\ 
     &           &    3 &  4.36e-09 &  2.10e-15 &      \\ 
     &           &    4 &  4.35e-09 &  1.24e-16 &      \\ 
     &           &    5 &  4.34e-09 &  1.24e-16 &      \\ 
     &           &    6 &  4.34e-09 &  1.24e-16 &      \\ 
     &           &    7 &  4.33e-09 &  1.24e-16 &      \\ 
     &           &    8 &  4.32e-09 &  1.24e-16 &      \\ 
     &           &    9 &  4.32e-09 &  1.23e-16 &      \\ 
     &           &   10 &  4.31e-09 &  1.23e-16 &      \\ 
3939 &  3.94e+04 &   10 &           &           & iters  \\ 
 \hdashline 
     &           &    1 &  2.14e-08 &  1.10e-10 &      \\ 
     &           &    2 &  5.63e-08 &  6.11e-16 &      \\ 
     &           &    3 &  2.14e-08 &  1.61e-15 &      \\ 
     &           &    4 &  2.14e-08 &  6.12e-16 &      \\ 
     &           &    5 &  5.63e-08 &  6.11e-16 &      \\ 
     &           &    6 &  2.15e-08 &  1.61e-15 &      \\ 
     &           &    7 &  2.14e-08 &  6.13e-16 &      \\ 
     &           &    8 &  2.14e-08 &  6.12e-16 &      \\ 
     &           &    9 &  5.63e-08 &  6.11e-16 &      \\ 
     &           &   10 &  2.15e-08 &  1.61e-15 &      \\ 
3940 &  3.94e+04 &   10 &           &           & iters  \\ 
 \hdashline 
     &           &    1 &  1.34e-08 &  2.36e-11 &      \\ 
     &           &    2 &  1.34e-08 &  3.83e-16 &      \\ 
     &           &    3 &  1.34e-08 &  3.82e-16 &      \\ 
     &           &    4 &  6.43e-08 &  3.82e-16 &      \\ 
     &           &    5 &  1.34e-08 &  1.84e-15 &      \\ 
     &           &    6 &  1.34e-08 &  3.84e-16 &      \\ 
     &           &    7 &  1.34e-08 &  3.83e-16 &      \\ 
     &           &    8 &  1.34e-08 &  3.83e-16 &      \\ 
     &           &    9 &  1.34e-08 &  3.82e-16 &      \\ 
     &           &   10 &  6.43e-08 &  3.82e-16 &      \\ 
3941 &  3.94e+04 &   10 &           &           & iters  \\ 
 \hdashline 
     &           &    1 &  2.06e-08 &  3.97e-10 &      \\ 
     &           &    2 &  2.06e-08 &  5.88e-16 &      \\ 
     &           &    3 &  5.72e-08 &  5.87e-16 &      \\ 
     &           &    4 &  2.06e-08 &  1.63e-15 &      \\ 
     &           &    5 &  2.06e-08 &  5.88e-16 &      \\ 
     &           &    6 &  2.06e-08 &  5.88e-16 &      \\ 
     &           &    7 &  5.72e-08 &  5.87e-16 &      \\ 
     &           &    8 &  2.06e-08 &  1.63e-15 &      \\ 
     &           &    9 &  2.06e-08 &  5.88e-16 &      \\ 
     &           &   10 &  5.71e-08 &  5.87e-16 &      \\ 
3942 &  3.94e+04 &   10 &           &           & iters  \\ 
 \hdashline 
     &           &    1 &  8.12e-08 &  5.49e-10 &      \\ 
     &           &    2 &  3.55e-08 &  2.32e-15 &      \\ 
     &           &    3 &  3.43e-09 &  1.01e-15 &      \\ 
     &           &    4 &  3.43e-09 &  9.79e-17 &      \\ 
     &           &    5 &  3.42e-09 &  9.78e-17 &      \\ 
     &           &    6 &  3.42e-09 &  9.76e-17 &      \\ 
     &           &    7 &  3.41e-09 &  9.75e-17 &      \\ 
     &           &    8 &  3.41e-09 &  9.73e-17 &      \\ 
     &           &    9 &  3.40e-09 &  9.72e-17 &      \\ 
     &           &   10 &  3.40e-09 &  9.71e-17 &      \\ 
3943 &  3.94e+04 &   10 &           &           & iters  \\ 
 \hdashline 
     &           &    1 &  1.90e-08 &  2.24e-10 &      \\ 
     &           &    2 &  1.90e-08 &  5.42e-16 &      \\ 
     &           &    3 &  5.87e-08 &  5.41e-16 &      \\ 
     &           &    4 &  1.90e-08 &  1.68e-15 &      \\ 
     &           &    5 &  1.90e-08 &  5.43e-16 &      \\ 
     &           &    6 &  1.90e-08 &  5.42e-16 &      \\ 
     &           &    7 &  5.87e-08 &  5.41e-16 &      \\ 
     &           &    8 &  1.90e-08 &  1.68e-15 &      \\ 
     &           &    9 &  1.90e-08 &  5.43e-16 &      \\ 
     &           &   10 &  1.90e-08 &  5.42e-16 &      \\ 
3944 &  3.94e+04 &   10 &           &           & iters  \\ 
 \hdashline 
     &           &    1 &  8.01e-09 &  2.51e-10 &      \\ 
     &           &    2 &  8.00e-09 &  2.29e-16 &      \\ 
     &           &    3 &  7.99e-09 &  2.28e-16 &      \\ 
     &           &    4 &  7.98e-09 &  2.28e-16 &      \\ 
     &           &    5 &  7.96e-09 &  2.28e-16 &      \\ 
     &           &    6 &  7.95e-09 &  2.27e-16 &      \\ 
     &           &    7 &  7.94e-09 &  2.27e-16 &      \\ 
     &           &    8 &  7.93e-09 &  2.27e-16 &      \\ 
     &           &    9 &  6.98e-08 &  2.26e-16 &      \\ 
     &           &   10 &  4.69e-08 &  1.99e-15 &      \\ 
3945 &  3.94e+04 &   10 &           &           & iters  \\ 
 \hdashline 
     &           &    1 &  5.38e-08 &  4.42e-10 &      \\ 
     &           &    2 &  2.40e-08 &  1.53e-15 &      \\ 
     &           &    3 &  2.40e-08 &  6.85e-16 &      \\ 
     &           &    4 &  5.38e-08 &  6.84e-16 &      \\ 
     &           &    5 &  2.40e-08 &  1.53e-15 &      \\ 
     &           &    6 &  2.40e-08 &  6.85e-16 &      \\ 
     &           &    7 &  2.39e-08 &  6.84e-16 &      \\ 
     &           &    8 &  5.38e-08 &  6.83e-16 &      \\ 
     &           &    9 &  2.40e-08 &  1.54e-15 &      \\ 
     &           &   10 &  2.39e-08 &  6.84e-16 &      \\ 
3946 &  3.94e+04 &   10 &           &           & iters  \\ 
 \hdashline 
     &           &    1 &  2.14e-08 &  3.14e-10 &      \\ 
     &           &    2 &  5.63e-08 &  6.11e-16 &      \\ 
     &           &    3 &  2.14e-08 &  1.61e-15 &      \\ 
     &           &    4 &  2.14e-08 &  6.12e-16 &      \\ 
     &           &    5 &  2.14e-08 &  6.11e-16 &      \\ 
     &           &    6 &  5.63e-08 &  6.10e-16 &      \\ 
     &           &    7 &  2.14e-08 &  1.61e-15 &      \\ 
     &           &    8 &  2.14e-08 &  6.12e-16 &      \\ 
     &           &    9 &  2.14e-08 &  6.11e-16 &      \\ 
     &           &   10 &  5.63e-08 &  6.10e-16 &      \\ 
3947 &  3.95e+04 &   10 &           &           & iters  \\ 
 \hdashline 
     &           &    1 &  3.13e-08 &  9.76e-11 &      \\ 
     &           &    2 &  4.64e-08 &  8.95e-16 &      \\ 
     &           &    3 &  3.14e-08 &  1.32e-15 &      \\ 
     &           &    4 &  3.13e-08 &  8.95e-16 &      \\ 
     &           &    5 &  4.64e-08 &  8.94e-16 &      \\ 
     &           &    6 &  3.13e-08 &  1.32e-15 &      \\ 
     &           &    7 &  4.64e-08 &  8.95e-16 &      \\ 
     &           &    8 &  3.14e-08 &  1.32e-15 &      \\ 
     &           &    9 &  3.13e-08 &  8.95e-16 &      \\ 
     &           &   10 &  4.64e-08 &  8.94e-16 &      \\ 
3948 &  3.95e+04 &   10 &           &           & iters  \\ 
 \hdashline 
     &           &    1 &  1.68e-08 &  2.80e-11 &      \\ 
     &           &    2 &  1.68e-08 &  4.80e-16 &      \\ 
     &           &    3 &  1.68e-08 &  4.79e-16 &      \\ 
     &           &    4 &  6.10e-08 &  4.78e-16 &      \\ 
     &           &    5 &  1.68e-08 &  1.74e-15 &      \\ 
     &           &    6 &  1.68e-08 &  4.80e-16 &      \\ 
     &           &    7 &  1.68e-08 &  4.79e-16 &      \\ 
     &           &    8 &  1.67e-08 &  4.79e-16 &      \\ 
     &           &    9 &  6.10e-08 &  4.78e-16 &      \\ 
     &           &   10 &  1.68e-08 &  1.74e-15 &      \\ 
3949 &  3.95e+04 &   10 &           &           & iters  \\ 
 \hdashline 
     &           &    1 &  3.33e-08 &  1.28e-10 &      \\ 
     &           &    2 &  4.44e-08 &  9.52e-16 &      \\ 
     &           &    3 &  3.34e-08 &  1.27e-15 &      \\ 
     &           &    4 &  3.33e-08 &  9.52e-16 &      \\ 
     &           &    5 &  4.44e-08 &  9.51e-16 &      \\ 
     &           &    6 &  3.33e-08 &  1.27e-15 &      \\ 
     &           &    7 &  4.44e-08 &  9.51e-16 &      \\ 
     &           &    8 &  3.33e-08 &  1.27e-15 &      \\ 
     &           &    9 &  4.44e-08 &  9.52e-16 &      \\ 
     &           &   10 &  3.34e-08 &  1.27e-15 &      \\ 
3950 &  3.95e+04 &   10 &           &           & iters  \\ 
 \hdashline 
     &           &    1 &  3.92e-08 &  2.15e-10 &      \\ 
     &           &    2 &  3.85e-08 &  1.12e-15 &      \\ 
     &           &    3 &  3.92e-08 &  1.10e-15 &      \\ 
     &           &    4 &  3.85e-08 &  1.12e-15 &      \\ 
     &           &    5 &  3.92e-08 &  1.10e-15 &      \\ 
     &           &    6 &  3.85e-08 &  1.12e-15 &      \\ 
     &           &    7 &  3.92e-08 &  1.10e-15 &      \\ 
     &           &    8 &  3.85e-08 &  1.12e-15 &      \\ 
     &           &    9 &  3.92e-08 &  1.10e-15 &      \\ 
     &           &   10 &  3.85e-08 &  1.12e-15 &      \\ 
3951 &  3.95e+04 &   10 &           &           & iters  \\ 
 \hdashline 
     &           &    1 &  2.24e-08 &  2.32e-11 &      \\ 
     &           &    2 &  2.24e-08 &  6.40e-16 &      \\ 
     &           &    3 &  5.53e-08 &  6.39e-16 &      \\ 
     &           &    4 &  2.24e-08 &  1.58e-15 &      \\ 
     &           &    5 &  2.24e-08 &  6.41e-16 &      \\ 
     &           &    6 &  2.24e-08 &  6.40e-16 &      \\ 
     &           &    7 &  5.53e-08 &  6.39e-16 &      \\ 
     &           &    8 &  2.24e-08 &  1.58e-15 &      \\ 
     &           &    9 &  2.24e-08 &  6.40e-16 &      \\ 
     &           &   10 &  5.53e-08 &  6.39e-16 &      \\ 
3952 &  3.95e+04 &   10 &           &           & iters  \\ 
 \hdashline 
     &           &    1 &  7.49e-08 &  2.01e-10 &      \\ 
     &           &    2 &  8.06e-08 &  2.14e-15 &      \\ 
     &           &    3 &  7.49e-08 &  2.30e-15 &      \\ 
     &           &    4 &  8.06e-08 &  2.14e-15 &      \\ 
     &           &    5 &  7.49e-08 &  2.30e-15 &      \\ 
     &           &    6 &  2.92e-09 &  2.14e-15 &      \\ 
     &           &    7 &  2.91e-09 &  8.33e-17 &      \\ 
     &           &    8 &  2.91e-09 &  8.31e-17 &      \\ 
     &           &    9 &  2.90e-09 &  8.30e-17 &      \\ 
     &           &   10 &  2.90e-09 &  8.29e-17 &      \\ 
3953 &  3.95e+04 &   10 &           &           & iters  \\ 
 \hdashline 
     &           &    1 &  6.38e-08 &  1.69e-10 &      \\ 
     &           &    2 &  1.40e-08 &  1.82e-15 &      \\ 
     &           &    3 &  1.40e-08 &  4.00e-16 &      \\ 
     &           &    4 &  1.40e-08 &  3.99e-16 &      \\ 
     &           &    5 &  6.37e-08 &  3.99e-16 &      \\ 
     &           &    6 &  1.40e-08 &  1.82e-15 &      \\ 
     &           &    7 &  1.40e-08 &  4.01e-16 &      \\ 
     &           &    8 &  1.40e-08 &  4.00e-16 &      \\ 
     &           &    9 &  1.40e-08 &  4.00e-16 &      \\ 
     &           &   10 &  1.40e-08 &  3.99e-16 &      \\ 
3954 &  3.95e+04 &   10 &           &           & iters  \\ 
 \hdashline 
     &           &    1 &  2.50e-08 &  8.03e-11 &      \\ 
     &           &    2 &  2.50e-08 &  7.15e-16 &      \\ 
     &           &    3 &  5.27e-08 &  7.14e-16 &      \\ 
     &           &    4 &  2.50e-08 &  1.50e-15 &      \\ 
     &           &    5 &  2.50e-08 &  7.15e-16 &      \\ 
     &           &    6 &  5.27e-08 &  7.14e-16 &      \\ 
     &           &    7 &  2.50e-08 &  1.50e-15 &      \\ 
     &           &    8 &  2.50e-08 &  7.15e-16 &      \\ 
     &           &    9 &  5.27e-08 &  7.14e-16 &      \\ 
     &           &   10 &  2.50e-08 &  1.50e-15 &      \\ 
3955 &  3.95e+04 &   10 &           &           & iters  \\ 
 \hdashline 
     &           &    1 &  6.41e-09 &  1.05e-10 &      \\ 
     &           &    2 &  6.40e-09 &  1.83e-16 &      \\ 
     &           &    3 &  6.39e-09 &  1.83e-16 &      \\ 
     &           &    4 &  6.38e-09 &  1.82e-16 &      \\ 
     &           &    5 &  6.37e-09 &  1.82e-16 &      \\ 
     &           &    6 &  6.37e-09 &  1.82e-16 &      \\ 
     &           &    7 &  7.13e-08 &  1.82e-16 &      \\ 
     &           &    8 &  8.41e-08 &  2.04e-15 &      \\ 
     &           &    9 &  7.14e-08 &  2.40e-15 &      \\ 
     &           &   10 &  6.44e-09 &  2.04e-15 &      \\ 
3956 &  3.96e+04 &   10 &           &           & iters  \\ 
 \hdashline 
     &           &    1 &  1.07e-08 &  2.06e-10 &      \\ 
     &           &    2 &  1.06e-08 &  3.04e-16 &      \\ 
     &           &    3 &  1.06e-08 &  3.04e-16 &      \\ 
     &           &    4 &  6.71e-08 &  3.03e-16 &      \\ 
     &           &    5 &  8.84e-08 &  1.91e-15 &      \\ 
     &           &    6 &  6.71e-08 &  2.52e-15 &      \\ 
     &           &    7 &  1.07e-08 &  1.92e-15 &      \\ 
     &           &    8 &  1.07e-08 &  3.05e-16 &      \\ 
     &           &    9 &  1.06e-08 &  3.04e-16 &      \\ 
     &           &   10 &  1.06e-08 &  3.04e-16 &      \\ 
3957 &  3.96e+04 &   10 &           &           & iters  \\ 
 \hdashline 
     &           &    1 &  1.73e-08 &  2.28e-10 &      \\ 
     &           &    2 &  1.72e-08 &  4.93e-16 &      \\ 
     &           &    3 &  6.05e-08 &  4.92e-16 &      \\ 
     &           &    4 &  1.73e-08 &  1.73e-15 &      \\ 
     &           &    5 &  1.73e-08 &  4.94e-16 &      \\ 
     &           &    6 &  1.73e-08 &  4.93e-16 &      \\ 
     &           &    7 &  1.72e-08 &  4.93e-16 &      \\ 
     &           &    8 &  6.05e-08 &  4.92e-16 &      \\ 
     &           &    9 &  1.73e-08 &  1.73e-15 &      \\ 
     &           &   10 &  1.73e-08 &  4.94e-16 &      \\ 
3958 &  3.96e+04 &   10 &           &           & iters  \\ 
 \hdashline 
     &           &    1 &  7.59e-09 &  8.67e-11 &      \\ 
     &           &    2 &  7.57e-09 &  2.16e-16 &      \\ 
     &           &    3 &  7.56e-09 &  2.16e-16 &      \\ 
     &           &    4 &  7.55e-09 &  2.16e-16 &      \\ 
     &           &    5 &  7.54e-09 &  2.16e-16 &      \\ 
     &           &    6 &  7.53e-09 &  2.15e-16 &      \\ 
     &           &    7 &  7.52e-09 &  2.15e-16 &      \\ 
     &           &    8 &  7.51e-09 &  2.15e-16 &      \\ 
     &           &    9 &  7.50e-09 &  2.14e-16 &      \\ 
     &           &   10 &  7.49e-09 &  2.14e-16 &      \\ 
3959 &  3.96e+04 &   10 &           &           & iters  \\ 
 \hdashline 
     &           &    1 &  7.17e-08 &  1.11e-10 &      \\ 
     &           &    2 &  6.07e-09 &  2.05e-15 &      \\ 
     &           &    3 &  6.06e-09 &  1.73e-16 &      \\ 
     &           &    4 &  6.06e-09 &  1.73e-16 &      \\ 
     &           &    5 &  6.05e-09 &  1.73e-16 &      \\ 
     &           &    6 &  6.04e-09 &  1.73e-16 &      \\ 
     &           &    7 &  6.03e-09 &  1.72e-16 &      \\ 
     &           &    8 &  6.02e-09 &  1.72e-16 &      \\ 
     &           &    9 &  6.01e-09 &  1.72e-16 &      \\ 
     &           &   10 &  6.00e-09 &  1.72e-16 &      \\ 
3960 &  3.96e+04 &   10 &           &           & iters  \\ 
 \hdashline 
     &           &    1 &  1.61e-08 &  4.74e-10 &      \\ 
     &           &    2 &  1.61e-08 &  4.59e-16 &      \\ 
     &           &    3 &  1.60e-08 &  4.58e-16 &      \\ 
     &           &    4 &  1.60e-08 &  4.57e-16 &      \\ 
     &           &    5 &  6.17e-08 &  4.57e-16 &      \\ 
     &           &    6 &  1.61e-08 &  1.76e-15 &      \\ 
     &           &    7 &  1.60e-08 &  4.59e-16 &      \\ 
     &           &    8 &  1.60e-08 &  4.58e-16 &      \\ 
     &           &    9 &  1.60e-08 &  4.57e-16 &      \\ 
     &           &   10 &  6.17e-08 &  4.57e-16 &      \\ 
3961 &  3.96e+04 &   10 &           &           & iters  \\ 
 \hdashline 
     &           &    1 &  2.04e-08 &  6.37e-10 &      \\ 
     &           &    2 &  2.04e-08 &  5.83e-16 &      \\ 
     &           &    3 &  2.04e-08 &  5.83e-16 &      \\ 
     &           &    4 &  5.73e-08 &  5.82e-16 &      \\ 
     &           &    5 &  2.04e-08 &  1.64e-15 &      \\ 
     &           &    6 &  2.04e-08 &  5.83e-16 &      \\ 
     &           &    7 &  5.73e-08 &  5.82e-16 &      \\ 
     &           &    8 &  2.05e-08 &  1.64e-15 &      \\ 
     &           &    9 &  2.04e-08 &  5.84e-16 &      \\ 
     &           &   10 &  2.04e-08 &  5.83e-16 &      \\ 
3962 &  3.96e+04 &   10 &           &           & iters  \\ 
 \hdashline 
     &           &    1 &  5.45e-08 &  2.03e-10 &      \\ 
     &           &    2 &  2.32e-08 &  1.56e-15 &      \\ 
     &           &    3 &  2.32e-08 &  6.63e-16 &      \\ 
     &           &    4 &  5.45e-08 &  6.62e-16 &      \\ 
     &           &    5 &  2.33e-08 &  1.56e-15 &      \\ 
     &           &    6 &  2.32e-08 &  6.64e-16 &      \\ 
     &           &    7 &  5.45e-08 &  6.63e-16 &      \\ 
     &           &    8 &  2.33e-08 &  1.56e-15 &      \\ 
     &           &    9 &  2.32e-08 &  6.64e-16 &      \\ 
     &           &   10 &  2.32e-08 &  6.63e-16 &      \\ 
3963 &  3.96e+04 &   10 &           &           & iters  \\ 
 \hdashline 
     &           &    1 &  3.25e-09 &  4.13e-10 &      \\ 
     &           &    2 &  3.25e-09 &  9.28e-17 &      \\ 
     &           &    3 &  3.24e-09 &  9.26e-17 &      \\ 
     &           &    4 &  3.24e-09 &  9.25e-17 &      \\ 
     &           &    5 &  3.23e-09 &  9.24e-17 &      \\ 
     &           &    6 &  3.23e-09 &  9.22e-17 &      \\ 
     &           &    7 &  3.22e-09 &  9.21e-17 &      \\ 
     &           &    8 &  3.22e-09 &  9.20e-17 &      \\ 
     &           &    9 &  3.21e-09 &  9.18e-17 &      \\ 
     &           &   10 &  3.21e-09 &  9.17e-17 &      \\ 
3964 &  3.96e+04 &   10 &           &           & iters  \\ 
 \hdashline 
     &           &    1 &  3.97e-08 &  5.46e-10 &      \\ 
     &           &    2 &  3.80e-08 &  1.13e-15 &      \\ 
     &           &    3 &  3.97e-08 &  1.09e-15 &      \\ 
     &           &    4 &  3.80e-08 &  1.13e-15 &      \\ 
     &           &    5 &  3.97e-08 &  1.09e-15 &      \\ 
     &           &    6 &  3.80e-08 &  1.13e-15 &      \\ 
     &           &    7 &  3.80e-08 &  1.09e-15 &      \\ 
     &           &    8 &  3.98e-08 &  1.08e-15 &      \\ 
     &           &    9 &  3.80e-08 &  1.14e-15 &      \\ 
     &           &   10 &  3.98e-08 &  1.08e-15 &      \\ 
3965 &  3.96e+04 &   10 &           &           & iters  \\ 
 \hdashline 
     &           &    1 &  1.34e-08 &  3.31e-10 &      \\ 
     &           &    2 &  1.34e-08 &  3.82e-16 &      \\ 
     &           &    3 &  1.34e-08 &  3.82e-16 &      \\ 
     &           &    4 &  6.43e-08 &  3.81e-16 &      \\ 
     &           &    5 &  1.34e-08 &  1.84e-15 &      \\ 
     &           &    6 &  1.34e-08 &  3.83e-16 &      \\ 
     &           &    7 &  1.34e-08 &  3.83e-16 &      \\ 
     &           &    8 &  1.34e-08 &  3.82e-16 &      \\ 
     &           &    9 &  6.43e-08 &  3.82e-16 &      \\ 
     &           &   10 &  9.11e-08 &  1.84e-15 &      \\ 
3966 &  3.96e+04 &   10 &           &           & iters  \\ 
 \hdashline 
     &           &    1 &  2.76e-10 &  8.88e-11 &      \\ 
     &           &    2 &  2.75e-10 &  7.87e-18 &      \\ 
     &           &    3 &  2.75e-10 &  7.85e-18 &      \\ 
     &           &    4 &  2.74e-10 &  7.84e-18 &      \\ 
     &           &    5 &  2.74e-10 &  7.83e-18 &      \\ 
     &           &    6 &  2.74e-10 &  7.82e-18 &      \\ 
     &           &    7 &  2.73e-10 &  7.81e-18 &      \\ 
     &           &    8 &  2.73e-10 &  7.80e-18 &      \\ 
     &           &    9 &  2.72e-10 &  7.79e-18 &      \\ 
     &           &   10 &  2.72e-10 &  7.78e-18 &      \\ 
3967 &  3.97e+04 &   10 &           &           & iters  \\ 
 \hdashline 
     &           &    1 &  3.56e-08 &  1.30e-10 &      \\ 
     &           &    2 &  3.30e-09 &  1.02e-15 &      \\ 
     &           &    3 &  3.30e-09 &  9.43e-17 &      \\ 
     &           &    4 &  3.29e-09 &  9.41e-17 &      \\ 
     &           &    5 &  3.29e-09 &  9.40e-17 &      \\ 
     &           &    6 &  3.28e-09 &  9.38e-17 &      \\ 
     &           &    7 &  3.56e-08 &  9.37e-17 &      \\ 
     &           &    8 &  3.33e-09 &  1.02e-15 &      \\ 
     &           &    9 &  3.32e-09 &  9.50e-17 &      \\ 
     &           &   10 &  3.32e-09 &  9.49e-17 &      \\ 
3968 &  3.97e+04 &   10 &           &           & iters  \\ 
 \hdashline 
     &           &    1 &  5.68e-09 &  1.07e-10 &      \\ 
     &           &    2 &  5.67e-09 &  1.62e-16 &      \\ 
     &           &    3 &  5.66e-09 &  1.62e-16 &      \\ 
     &           &    4 &  5.65e-09 &  1.62e-16 &      \\ 
     &           &    5 &  5.64e-09 &  1.61e-16 &      \\ 
     &           &    6 &  5.64e-09 &  1.61e-16 &      \\ 
     &           &    7 &  7.21e-08 &  1.61e-16 &      \\ 
     &           &    8 &  4.46e-08 &  2.06e-15 &      \\ 
     &           &    9 &  5.67e-09 &  1.27e-15 &      \\ 
     &           &   10 &  5.66e-09 &  1.62e-16 &      \\ 
3969 &  3.97e+04 &   10 &           &           & iters  \\ 
 \hdashline 
     &           &    1 &  4.74e-09 &  5.75e-11 &      \\ 
     &           &    2 &  4.74e-09 &  1.35e-16 &      \\ 
     &           &    3 &  4.73e-09 &  1.35e-16 &      \\ 
     &           &    4 &  4.72e-09 &  1.35e-16 &      \\ 
     &           &    5 &  4.71e-09 &  1.35e-16 &      \\ 
     &           &    6 &  4.71e-09 &  1.35e-16 &      \\ 
     &           &    7 &  4.70e-09 &  1.34e-16 &      \\ 
     &           &    8 &  4.69e-09 &  1.34e-16 &      \\ 
     &           &    9 &  4.69e-09 &  1.34e-16 &      \\ 
     &           &   10 &  3.42e-08 &  1.34e-16 &      \\ 
3970 &  3.97e+04 &   10 &           &           & iters  \\ 
 \hdashline 
     &           &    1 &  2.42e-08 &  2.63e-10 &      \\ 
     &           &    2 &  5.35e-08 &  6.91e-16 &      \\ 
     &           &    3 &  2.42e-08 &  1.53e-15 &      \\ 
     &           &    4 &  2.42e-08 &  6.92e-16 &      \\ 
     &           &    5 &  5.35e-08 &  6.91e-16 &      \\ 
     &           &    6 &  2.43e-08 &  1.53e-15 &      \\ 
     &           &    7 &  2.42e-08 &  6.92e-16 &      \\ 
     &           &    8 &  2.42e-08 &  6.91e-16 &      \\ 
     &           &    9 &  5.35e-08 &  6.90e-16 &      \\ 
     &           &   10 &  2.42e-08 &  1.53e-15 &      \\ 
3971 &  3.97e+04 &   10 &           &           & iters  \\ 
 \hdashline 
     &           &    1 &  8.72e-09 &  2.01e-10 &      \\ 
     &           &    2 &  8.71e-09 &  2.49e-16 &      \\ 
     &           &    3 &  8.70e-09 &  2.49e-16 &      \\ 
     &           &    4 &  6.90e-08 &  2.48e-16 &      \\ 
     &           &    5 &  8.65e-08 &  1.97e-15 &      \\ 
     &           &    6 &  6.90e-08 &  2.47e-15 &      \\ 
     &           &    7 &  8.76e-09 &  1.97e-15 &      \\ 
     &           &    8 &  8.75e-09 &  2.50e-16 &      \\ 
     &           &    9 &  8.73e-09 &  2.50e-16 &      \\ 
     &           &   10 &  8.72e-09 &  2.49e-16 &      \\ 
3972 &  3.97e+04 &   10 &           &           & iters  \\ 
 \hdashline 
     &           &    1 &  2.91e-08 &  1.74e-10 &      \\ 
     &           &    2 &  4.86e-08 &  8.31e-16 &      \\ 
     &           &    3 &  2.91e-08 &  1.39e-15 &      \\ 
     &           &    4 &  2.91e-08 &  8.32e-16 &      \\ 
     &           &    5 &  4.86e-08 &  8.31e-16 &      \\ 
     &           &    6 &  2.91e-08 &  1.39e-15 &      \\ 
     &           &    7 &  2.91e-08 &  8.31e-16 &      \\ 
     &           &    8 &  4.86e-08 &  8.30e-16 &      \\ 
     &           &    9 &  2.91e-08 &  1.39e-15 &      \\ 
     &           &   10 &  4.86e-08 &  8.31e-16 &      \\ 
3973 &  3.97e+04 &   10 &           &           & iters  \\ 
 \hdashline 
     &           &    1 &  2.18e-08 &  3.17e-10 &      \\ 
     &           &    2 &  2.17e-08 &  6.21e-16 &      \\ 
     &           &    3 &  5.60e-08 &  6.20e-16 &      \\ 
     &           &    4 &  2.18e-08 &  1.60e-15 &      \\ 
     &           &    5 &  2.17e-08 &  6.21e-16 &      \\ 
     &           &    6 &  5.60e-08 &  6.21e-16 &      \\ 
     &           &    7 &  2.18e-08 &  1.60e-15 &      \\ 
     &           &    8 &  2.18e-08 &  6.22e-16 &      \\ 
     &           &    9 &  2.17e-08 &  6.21e-16 &      \\ 
     &           &   10 &  5.60e-08 &  6.20e-16 &      \\ 
3974 &  3.97e+04 &   10 &           &           & iters  \\ 
 \hdashline 
     &           &    1 &  2.53e-08 &  3.50e-11 &      \\ 
     &           &    2 &  1.36e-08 &  7.21e-16 &      \\ 
     &           &    3 &  2.53e-08 &  3.88e-16 &      \\ 
     &           &    4 &  1.36e-08 &  7.21e-16 &      \\ 
     &           &    5 &  1.36e-08 &  3.89e-16 &      \\ 
     &           &    6 &  2.53e-08 &  3.88e-16 &      \\ 
     &           &    7 &  1.36e-08 &  7.21e-16 &      \\ 
     &           &    8 &  1.36e-08 &  3.89e-16 &      \\ 
     &           &    9 &  2.53e-08 &  3.88e-16 &      \\ 
     &           &   10 &  1.36e-08 &  7.21e-16 &      \\ 
3975 &  3.97e+04 &   10 &           &           & iters  \\ 
 \hdashline 
     &           &    1 &  1.84e-08 &  4.12e-10 &      \\ 
     &           &    2 &  1.84e-08 &  5.27e-16 &      \\ 
     &           &    3 &  1.84e-08 &  5.26e-16 &      \\ 
     &           &    4 &  5.93e-08 &  5.25e-16 &      \\ 
     &           &    5 &  1.85e-08 &  1.69e-15 &      \\ 
     &           &    6 &  1.84e-08 &  5.27e-16 &      \\ 
     &           &    7 &  1.84e-08 &  5.26e-16 &      \\ 
     &           &    8 &  5.93e-08 &  5.25e-16 &      \\ 
     &           &    9 &  1.85e-08 &  1.69e-15 &      \\ 
     &           &   10 &  1.84e-08 &  5.27e-16 &      \\ 
3976 &  3.98e+04 &   10 &           &           & iters  \\ 
 \hdashline 
     &           &    1 &  2.69e-09 &  5.49e-10 &      \\ 
     &           &    2 &  2.68e-09 &  7.67e-17 &      \\ 
     &           &    3 &  2.68e-09 &  7.66e-17 &      \\ 
     &           &    4 &  2.67e-09 &  7.64e-17 &      \\ 
     &           &    5 &  2.67e-09 &  7.63e-17 &      \\ 
     &           &    6 &  2.67e-09 &  7.62e-17 &      \\ 
     &           &    7 &  2.66e-09 &  7.61e-17 &      \\ 
     &           &    8 &  2.66e-09 &  7.60e-17 &      \\ 
     &           &    9 &  2.66e-09 &  7.59e-17 &      \\ 
     &           &   10 &  2.65e-09 &  7.58e-17 &      \\ 
3977 &  3.98e+04 &   10 &           &           & iters  \\ 
 \hdashline 
     &           &    1 &  3.86e-08 &  7.61e-11 &      \\ 
     &           &    2 &  3.92e-08 &  1.10e-15 &      \\ 
     &           &    3 &  3.86e-08 &  1.12e-15 &      \\ 
     &           &    4 &  3.91e-08 &  1.10e-15 &      \\ 
     &           &    5 &  3.86e-08 &  1.12e-15 &      \\ 
     &           &    6 &  3.91e-08 &  1.10e-15 &      \\ 
     &           &    7 &  3.86e-08 &  1.12e-15 &      \\ 
     &           &    8 &  3.91e-08 &  1.10e-15 &      \\ 
     &           &    9 &  3.86e-08 &  1.12e-15 &      \\ 
     &           &   10 &  3.91e-08 &  1.10e-15 &      \\ 
3978 &  3.98e+04 &   10 &           &           & iters  \\ 
 \hdashline 
     &           &    1 &  4.72e-08 &  3.84e-10 &      \\ 
     &           &    2 &  4.71e-08 &  1.35e-15 &      \\ 
     &           &    3 &  3.06e-08 &  1.34e-15 &      \\ 
     &           &    4 &  4.71e-08 &  8.75e-16 &      \\ 
     &           &    5 &  3.07e-08 &  1.34e-15 &      \\ 
     &           &    6 &  3.06e-08 &  8.75e-16 &      \\ 
     &           &    7 &  4.71e-08 &  8.74e-16 &      \\ 
     &           &    8 &  3.07e-08 &  1.34e-15 &      \\ 
     &           &    9 &  3.06e-08 &  8.75e-16 &      \\ 
     &           &   10 &  4.71e-08 &  8.74e-16 &      \\ 
3979 &  3.98e+04 &   10 &           &           & iters  \\ 
 \hdashline 
     &           &    1 &  4.23e-08 &  2.61e-10 &      \\ 
     &           &    2 &  3.55e-08 &  1.21e-15 &      \\ 
     &           &    3 &  4.23e-08 &  1.01e-15 &      \\ 
     &           &    4 &  3.55e-08 &  1.21e-15 &      \\ 
     &           &    5 &  4.23e-08 &  1.01e-15 &      \\ 
     &           &    6 &  3.55e-08 &  1.21e-15 &      \\ 
     &           &    7 &  4.23e-08 &  1.01e-15 &      \\ 
     &           &    8 &  3.55e-08 &  1.21e-15 &      \\ 
     &           &    9 &  3.54e-08 &  1.01e-15 &      \\ 
     &           &   10 &  4.23e-08 &  1.01e-15 &      \\ 
3980 &  3.98e+04 &   10 &           &           & iters  \\ 
 \hdashline 
     &           &    1 &  1.95e-08 &  1.24e-10 &      \\ 
     &           &    2 &  1.95e-08 &  5.57e-16 &      \\ 
     &           &    3 &  5.82e-08 &  5.57e-16 &      \\ 
     &           &    4 &  1.96e-08 &  1.66e-15 &      \\ 
     &           &    5 &  1.95e-08 &  5.58e-16 &      \\ 
     &           &    6 &  1.95e-08 &  5.57e-16 &      \\ 
     &           &    7 &  5.82e-08 &  5.57e-16 &      \\ 
     &           &    8 &  1.96e-08 &  1.66e-15 &      \\ 
     &           &    9 &  1.95e-08 &  5.58e-16 &      \\ 
     &           &   10 &  1.95e-08 &  5.57e-16 &      \\ 
3981 &  3.98e+04 &   10 &           &           & iters  \\ 
 \hdashline 
     &           &    1 &  1.51e-08 &  2.38e-10 &      \\ 
     &           &    2 &  1.51e-08 &  4.32e-16 &      \\ 
     &           &    3 &  1.51e-08 &  4.31e-16 &      \\ 
     &           &    4 &  2.38e-08 &  4.31e-16 &      \\ 
     &           &    5 &  1.51e-08 &  6.79e-16 &      \\ 
     &           &    6 &  1.51e-08 &  4.31e-16 &      \\ 
     &           &    7 &  2.38e-08 &  4.30e-16 &      \\ 
     &           &    8 &  1.51e-08 &  6.79e-16 &      \\ 
     &           &    9 &  2.38e-08 &  4.31e-16 &      \\ 
     &           &   10 &  1.51e-08 &  6.78e-16 &      \\ 
3982 &  3.98e+04 &   10 &           &           & iters  \\ 
 \hdashline 
     &           &    1 &  3.05e-08 &  2.36e-10 &      \\ 
     &           &    2 &  4.73e-08 &  8.69e-16 &      \\ 
     &           &    3 &  3.05e-08 &  1.35e-15 &      \\ 
     &           &    4 &  4.73e-08 &  8.70e-16 &      \\ 
     &           &    5 &  3.05e-08 &  1.35e-15 &      \\ 
     &           &    6 &  3.05e-08 &  8.70e-16 &      \\ 
     &           &    7 &  4.73e-08 &  8.69e-16 &      \\ 
     &           &    8 &  3.05e-08 &  1.35e-15 &      \\ 
     &           &    9 &  4.73e-08 &  8.70e-16 &      \\ 
     &           &   10 &  3.05e-08 &  1.35e-15 &      \\ 
3983 &  3.98e+04 &   10 &           &           & iters  \\ 
 \hdashline 
     &           &    1 &  3.63e-08 &  1.74e-10 &      \\ 
     &           &    2 &  4.14e-08 &  1.04e-15 &      \\ 
     &           &    3 &  3.63e-08 &  1.18e-15 &      \\ 
     &           &    4 &  4.14e-08 &  1.04e-15 &      \\ 
     &           &    5 &  3.64e-08 &  1.18e-15 &      \\ 
     &           &    6 &  4.14e-08 &  1.04e-15 &      \\ 
     &           &    7 &  3.64e-08 &  1.18e-15 &      \\ 
     &           &    8 &  4.14e-08 &  1.04e-15 &      \\ 
     &           &    9 &  3.64e-08 &  1.18e-15 &      \\ 
     &           &   10 &  4.14e-08 &  1.04e-15 &      \\ 
3984 &  3.98e+04 &   10 &           &           & iters  \\ 
 \hdashline 
     &           &    1 &  3.73e-08 &  1.90e-10 &      \\ 
     &           &    2 &  1.56e-09 &  1.07e-15 &      \\ 
     &           &    3 &  1.56e-09 &  4.45e-17 &      \\ 
     &           &    4 &  1.55e-09 &  4.44e-17 &      \\ 
     &           &    5 &  1.55e-09 &  4.43e-17 &      \\ 
     &           &    6 &  1.55e-09 &  4.43e-17 &      \\ 
     &           &    7 &  1.55e-09 &  4.42e-17 &      \\ 
     &           &    8 &  1.54e-09 &  4.41e-17 &      \\ 
     &           &    9 &  1.54e-09 &  4.41e-17 &      \\ 
     &           &   10 &  1.54e-09 &  4.40e-17 &      \\ 
3985 &  3.98e+04 &   10 &           &           & iters  \\ 
 \hdashline 
     &           &    1 &  2.23e-08 &  1.72e-10 &      \\ 
     &           &    2 &  1.66e-08 &  6.36e-16 &      \\ 
     &           &    3 &  2.23e-08 &  4.74e-16 &      \\ 
     &           &    4 &  1.66e-08 &  6.36e-16 &      \\ 
     &           &    5 &  2.23e-08 &  4.74e-16 &      \\ 
     &           &    6 &  1.66e-08 &  6.36e-16 &      \\ 
     &           &    7 &  1.66e-08 &  4.74e-16 &      \\ 
     &           &    8 &  2.23e-08 &  4.73e-16 &      \\ 
     &           &    9 &  1.66e-08 &  6.36e-16 &      \\ 
     &           &   10 &  2.23e-08 &  4.74e-16 &      \\ 
3986 &  3.98e+04 &   10 &           &           & iters  \\ 
 \hdashline 
     &           &    1 &  4.94e-08 &  5.00e-11 &      \\ 
     &           &    2 &  2.83e-08 &  1.41e-15 &      \\ 
     &           &    3 &  2.83e-08 &  8.08e-16 &      \\ 
     &           &    4 &  4.94e-08 &  8.07e-16 &      \\ 
     &           &    5 &  2.83e-08 &  1.41e-15 &      \\ 
     &           &    6 &  2.83e-08 &  8.08e-16 &      \\ 
     &           &    7 &  4.95e-08 &  8.07e-16 &      \\ 
     &           &    8 &  2.83e-08 &  1.41e-15 &      \\ 
     &           &    9 &  4.94e-08 &  8.08e-16 &      \\ 
     &           &   10 &  2.83e-08 &  1.41e-15 &      \\ 
3987 &  3.99e+04 &   10 &           &           & iters  \\ 
 \hdashline 
     &           &    1 &  2.78e-08 &  2.52e-10 &      \\ 
     &           &    2 &  1.11e-08 &  7.93e-16 &      \\ 
     &           &    3 &  1.11e-08 &  3.17e-16 &      \\ 
     &           &    4 &  2.78e-08 &  3.17e-16 &      \\ 
     &           &    5 &  1.11e-08 &  7.92e-16 &      \\ 
     &           &    6 &  1.11e-08 &  3.17e-16 &      \\ 
     &           &    7 &  1.11e-08 &  3.17e-16 &      \\ 
     &           &    8 &  2.78e-08 &  3.16e-16 &      \\ 
     &           &    9 &  1.11e-08 &  7.93e-16 &      \\ 
     &           &   10 &  1.11e-08 &  3.17e-16 &      \\ 
3988 &  3.99e+04 &   10 &           &           & iters  \\ 
 \hdashline 
     &           &    1 &  1.20e-08 &  1.89e-10 &      \\ 
     &           &    2 &  2.68e-08 &  3.44e-16 &      \\ 
     &           &    3 &  1.21e-08 &  7.65e-16 &      \\ 
     &           &    4 &  1.21e-08 &  3.45e-16 &      \\ 
     &           &    5 &  1.20e-08 &  3.44e-16 &      \\ 
     &           &    6 &  2.68e-08 &  3.44e-16 &      \\ 
     &           &    7 &  1.21e-08 &  7.66e-16 &      \\ 
     &           &    8 &  1.20e-08 &  3.44e-16 &      \\ 
     &           &    9 &  2.68e-08 &  3.44e-16 &      \\ 
     &           &   10 &  1.21e-08 &  7.65e-16 &      \\ 
3989 &  3.99e+04 &   10 &           &           & iters  \\ 
 \hdashline 
     &           &    1 &  1.47e-10 &  1.67e-10 &      \\ 
     &           &    2 &  1.47e-10 &  4.20e-18 &      \\ 
     &           &    3 &  1.47e-10 &  4.19e-18 &      \\ 
     &           &    4 &  1.46e-10 &  4.18e-18 &      \\ 
     &           &    5 &  1.46e-10 &  4.18e-18 &      \\ 
     &           &    6 &  1.46e-10 &  4.17e-18 &      \\ 
     &           &    7 &  1.46e-10 &  4.17e-18 &      \\ 
     &           &    8 &  1.46e-10 &  4.16e-18 &      \\ 
     &           &    9 &  1.45e-10 &  4.15e-18 &      \\ 
     &           &   10 &  1.45e-10 &  4.15e-18 &      \\ 
3990 &  3.99e+04 &   10 &           &           & iters  \\ 
 \hdashline 
     &           &    1 &  1.69e-08 &  2.61e-10 &      \\ 
     &           &    2 &  1.69e-08 &  4.83e-16 &      \\ 
     &           &    3 &  6.08e-08 &  4.83e-16 &      \\ 
     &           &    4 &  1.70e-08 &  1.74e-15 &      \\ 
     &           &    5 &  1.70e-08 &  4.85e-16 &      \\ 
     &           &    6 &  1.69e-08 &  4.84e-16 &      \\ 
     &           &    7 &  1.69e-08 &  4.83e-16 &      \\ 
     &           &    8 &  6.08e-08 &  4.82e-16 &      \\ 
     &           &    9 &  1.70e-08 &  1.74e-15 &      \\ 
     &           &   10 &  1.69e-08 &  4.84e-16 &      \\ 
3991 &  3.99e+04 &   10 &           &           & iters  \\ 
 \hdashline 
     &           &    1 &  5.64e-08 &  1.54e-10 &      \\ 
     &           &    2 &  2.14e-08 &  1.61e-15 &      \\ 
     &           &    3 &  2.14e-08 &  6.10e-16 &      \\ 
     &           &    4 &  2.13e-08 &  6.09e-16 &      \\ 
     &           &    5 &  5.64e-08 &  6.09e-16 &      \\ 
     &           &    6 &  2.14e-08 &  1.61e-15 &      \\ 
     &           &    7 &  2.13e-08 &  6.10e-16 &      \\ 
     &           &    8 &  5.64e-08 &  6.09e-16 &      \\ 
     &           &    9 &  2.14e-08 &  1.61e-15 &      \\ 
     &           &   10 &  2.14e-08 &  6.11e-16 &      \\ 
3992 &  3.99e+04 &   10 &           &           & iters  \\ 
 \hdashline 
     &           &    1 &  1.57e-08 &  8.63e-11 &      \\ 
     &           &    2 &  1.56e-08 &  4.47e-16 &      \\ 
     &           &    3 &  6.21e-08 &  4.47e-16 &      \\ 
     &           &    4 &  1.57e-08 &  1.77e-15 &      \\ 
     &           &    5 &  1.57e-08 &  4.49e-16 &      \\ 
     &           &    6 &  1.57e-08 &  4.48e-16 &      \\ 
     &           &    7 &  1.56e-08 &  4.47e-16 &      \\ 
     &           &    8 &  6.21e-08 &  4.47e-16 &      \\ 
     &           &    9 &  1.57e-08 &  1.77e-15 &      \\ 
     &           &   10 &  1.57e-08 &  4.48e-16 &      \\ 
3993 &  3.99e+04 &   10 &           &           & iters  \\ 
 \hdashline 
     &           &    1 &  2.37e-09 &  4.85e-11 &      \\ 
     &           &    2 &  2.37e-09 &  6.78e-17 &      \\ 
     &           &    3 &  2.37e-09 &  6.77e-17 &      \\ 
     &           &    4 &  2.36e-09 &  6.76e-17 &      \\ 
     &           &    5 &  2.36e-09 &  6.75e-17 &      \\ 
     &           &    6 &  2.36e-09 &  6.74e-17 &      \\ 
     &           &    7 &  2.35e-09 &  6.73e-17 &      \\ 
     &           &    8 &  2.35e-09 &  6.72e-17 &      \\ 
     &           &    9 &  2.35e-09 &  6.71e-17 &      \\ 
     &           &   10 &  3.65e-08 &  6.70e-17 &      \\ 
3994 &  3.99e+04 &   10 &           &           & iters  \\ 
 \hdashline 
     &           &    1 &  3.20e-08 &  1.37e-10 &      \\ 
     &           &    2 &  3.20e-08 &  9.14e-16 &      \\ 
     &           &    3 &  4.57e-08 &  9.13e-16 &      \\ 
     &           &    4 &  3.20e-08 &  1.31e-15 &      \\ 
     &           &    5 &  4.57e-08 &  9.14e-16 &      \\ 
     &           &    6 &  3.20e-08 &  1.30e-15 &      \\ 
     &           &    7 &  3.20e-08 &  9.14e-16 &      \\ 
     &           &    8 &  4.57e-08 &  9.13e-16 &      \\ 
     &           &    9 &  3.20e-08 &  1.31e-15 &      \\ 
     &           &   10 &  4.57e-08 &  9.13e-16 &      \\ 
3995 &  3.99e+04 &   10 &           &           & iters  \\ 
 \hdashline 
     &           &    1 &  2.04e-08 &  5.55e-11 &      \\ 
     &           &    2 &  2.04e-08 &  5.82e-16 &      \\ 
     &           &    3 &  5.74e-08 &  5.81e-16 &      \\ 
     &           &    4 &  2.04e-08 &  1.64e-15 &      \\ 
     &           &    5 &  2.04e-08 &  5.82e-16 &      \\ 
     &           &    6 &  2.03e-08 &  5.82e-16 &      \\ 
     &           &    7 &  5.74e-08 &  5.81e-16 &      \\ 
     &           &    8 &  2.04e-08 &  1.64e-15 &      \\ 
     &           &    9 &  2.04e-08 &  5.82e-16 &      \\ 
     &           &   10 &  2.03e-08 &  5.81e-16 &      \\ 
3996 &  4.00e+04 &   10 &           &           & iters  \\ 
 \hdashline 
     &           &    1 &  7.62e-08 &  1.27e-10 &      \\ 
     &           &    2 &  7.93e-08 &  2.18e-15 &      \\ 
     &           &    3 &  7.62e-08 &  2.26e-15 &      \\ 
     &           &    4 &  1.56e-09 &  2.18e-15 &      \\ 
     &           &    5 &  1.56e-09 &  4.46e-17 &      \\ 
     &           &    6 &  1.56e-09 &  4.45e-17 &      \\ 
     &           &    7 &  1.55e-09 &  4.44e-17 &      \\ 
     &           &    8 &  1.55e-09 &  4.44e-17 &      \\ 
     &           &    9 &  1.55e-09 &  4.43e-17 &      \\ 
     &           &   10 &  1.55e-09 &  4.42e-17 &      \\ 
3997 &  4.00e+04 &   10 &           &           & iters  \\ 
 \hdashline 
     &           &    1 &  4.31e-08 &  1.10e-10 &      \\ 
     &           &    2 &  3.47e-08 &  1.23e-15 &      \\ 
     &           &    3 &  4.31e-08 &  9.89e-16 &      \\ 
     &           &    4 &  3.47e-08 &  1.23e-15 &      \\ 
     &           &    5 &  4.31e-08 &  9.89e-16 &      \\ 
     &           &    6 &  3.47e-08 &  1.23e-15 &      \\ 
     &           &    7 &  3.46e-08 &  9.90e-16 &      \\ 
     &           &    8 &  4.31e-08 &  9.88e-16 &      \\ 
     &           &    9 &  3.46e-08 &  1.23e-15 &      \\ 
     &           &   10 &  4.31e-08 &  9.89e-16 &      \\ 
3998 &  4.00e+04 &   10 &           &           & iters  \\ 
 \hdashline 
     &           &    1 &  5.05e-08 &  5.26e-11 &      \\ 
     &           &    2 &  2.73e-08 &  1.44e-15 &      \\ 
     &           &    3 &  2.72e-08 &  7.78e-16 &      \\ 
     &           &    4 &  5.05e-08 &  7.77e-16 &      \\ 
     &           &    5 &  2.73e-08 &  1.44e-15 &      \\ 
     &           &    6 &  2.72e-08 &  7.78e-16 &      \\ 
     &           &    7 &  5.05e-08 &  7.77e-16 &      \\ 
     &           &    8 &  2.73e-08 &  1.44e-15 &      \\ 
     &           &    9 &  2.72e-08 &  7.78e-16 &      \\ 
     &           &   10 &  5.05e-08 &  7.77e-16 &      \\ 
3999 &  4.00e+04 &   10 &           &           & iters  \\ 
 \hdashline 
     &           &    1 &  2.24e-08 &  1.18e-10 &      \\ 
     &           &    2 &  2.24e-08 &  6.39e-16 &      \\ 
     &           &    3 &  2.23e-08 &  6.38e-16 &      \\ 
     &           &    4 &  5.54e-08 &  6.37e-16 &      \\ 
     &           &    5 &  2.24e-08 &  1.58e-15 &      \\ 
     &           &    6 &  2.23e-08 &  6.39e-16 &      \\ 
     &           &    7 &  5.54e-08 &  6.38e-16 &      \\ 
     &           &    8 &  2.24e-08 &  1.58e-15 &      \\ 
     &           &    9 &  2.24e-08 &  6.39e-16 &      \\ 
     &           &   10 &  2.23e-08 &  6.38e-16 &      \\ 
4000 &  4.00e+04 &   10 &           &           & iters  \\ 
 \hdashline 
     &           &    1 &  1.15e-07 &  2.10e-11 &      \\ 
     &           &    2 &  4.10e-08 &  3.27e-15 &      \\ 
     &           &    3 &  4.10e-08 &  1.17e-15 &      \\ 
     &           &    4 &  3.68e-08 &  1.17e-15 &      \\ 
     &           &    5 &  4.10e-08 &  1.05e-15 &      \\ 
     &           &    6 &  3.68e-08 &  1.17e-15 &      \\ 
     &           &    7 &  4.09e-08 &  1.05e-15 &      \\ 
     &           &    8 &  3.68e-08 &  1.17e-15 &      \\ 
     &           &    9 &  3.67e-08 &  1.05e-15 &      \\ 
     &           &   10 &  4.10e-08 &  1.05e-15 &      \\ 
4001 &  4.00e+04 &   10 &           &           & iters  \\ 
 \hdashline 
     &           &    1 &  1.64e-08 &  1.75e-10 &      \\ 
     &           &    2 &  2.25e-08 &  4.67e-16 &      \\ 
     &           &    3 &  1.64e-08 &  6.42e-16 &      \\ 
     &           &    4 &  2.25e-08 &  4.68e-16 &      \\ 
     &           &    5 &  1.64e-08 &  6.42e-16 &      \\ 
     &           &    6 &  2.25e-08 &  4.68e-16 &      \\ 
     &           &    7 &  1.64e-08 &  6.41e-16 &      \\ 
     &           &    8 &  1.64e-08 &  4.68e-16 &      \\ 
     &           &    9 &  2.25e-08 &  4.67e-16 &      \\ 
     &           &   10 &  1.64e-08 &  6.42e-16 &      \\ 
4002 &  4.00e+04 &   10 &           &           & iters  \\ 
 \hdashline 
     &           &    1 &  1.54e-08 &  1.48e-10 &      \\ 
     &           &    2 &  1.54e-08 &  4.40e-16 &      \\ 
     &           &    3 &  6.23e-08 &  4.39e-16 &      \\ 
     &           &    4 &  1.55e-08 &  1.78e-15 &      \\ 
     &           &    5 &  1.54e-08 &  4.41e-16 &      \\ 
     &           &    6 &  1.54e-08 &  4.40e-16 &      \\ 
     &           &    7 &  1.54e-08 &  4.40e-16 &      \\ 
     &           &    8 &  6.23e-08 &  4.39e-16 &      \\ 
     &           &    9 &  1.55e-08 &  1.78e-15 &      \\ 
     &           &   10 &  1.54e-08 &  4.41e-16 &      \\ 
4003 &  4.00e+04 &   10 &           &           & iters  \\ 
 \hdashline 
     &           &    1 &  1.29e-08 &  3.05e-12 &      \\ 
     &           &    2 &  1.28e-08 &  3.67e-16 &      \\ 
     &           &    3 &  2.60e-08 &  3.67e-16 &      \\ 
     &           &    4 &  1.29e-08 &  7.43e-16 &      \\ 
     &           &    5 &  1.28e-08 &  3.67e-16 &      \\ 
     &           &    6 &  2.60e-08 &  3.67e-16 &      \\ 
     &           &    7 &  1.29e-08 &  7.43e-16 &      \\ 
     &           &    8 &  1.28e-08 &  3.67e-16 &      \\ 
     &           &    9 &  2.60e-08 &  3.67e-16 &      \\ 
     &           &   10 &  1.29e-08 &  7.43e-16 &      \\ 
4004 &  4.00e+04 &   10 &           &           & iters  \\ 
 \hdashline 
     &           &    1 &  2.77e-08 &  5.17e-11 &      \\ 
     &           &    2 &  5.01e-08 &  7.90e-16 &      \\ 
     &           &    3 &  2.77e-08 &  1.43e-15 &      \\ 
     &           &    4 &  5.00e-08 &  7.91e-16 &      \\ 
     &           &    5 &  2.77e-08 &  1.43e-15 &      \\ 
     &           &    6 &  2.77e-08 &  7.92e-16 &      \\ 
     &           &    7 &  5.00e-08 &  7.91e-16 &      \\ 
     &           &    8 &  2.77e-08 &  1.43e-15 &      \\ 
     &           &    9 &  2.77e-08 &  7.91e-16 &      \\ 
     &           &   10 &  5.00e-08 &  7.90e-16 &      \\ 
4005 &  4.00e+04 &   10 &           &           & iters  \\ 
 \hdashline 
     &           &    1 &  3.42e-08 &  6.35e-11 &      \\ 
     &           &    2 &  4.70e-09 &  9.76e-16 &      \\ 
     &           &    3 &  4.69e-09 &  1.34e-16 &      \\ 
     &           &    4 &  4.68e-09 &  1.34e-16 &      \\ 
     &           &    5 &  4.68e-09 &  1.34e-16 &      \\ 
     &           &    6 &  3.42e-08 &  1.34e-16 &      \\ 
     &           &    7 &  4.72e-09 &  9.75e-16 &      \\ 
     &           &    8 &  4.71e-09 &  1.35e-16 &      \\ 
     &           &    9 &  4.71e-09 &  1.35e-16 &      \\ 
     &           &   10 &  4.70e-09 &  1.34e-16 &      \\ 
4006 &  4.00e+04 &   10 &           &           & iters  \\ 
 \hdashline 
     &           &    1 &  6.44e-09 &  1.42e-10 &      \\ 
     &           &    2 &  6.43e-09 &  1.84e-16 &      \\ 
     &           &    3 &  6.42e-09 &  1.84e-16 &      \\ 
     &           &    4 &  6.41e-09 &  1.83e-16 &      \\ 
     &           &    5 &  6.40e-09 &  1.83e-16 &      \\ 
     &           &    6 &  6.39e-09 &  1.83e-16 &      \\ 
     &           &    7 &  7.13e-08 &  1.83e-16 &      \\ 
     &           &    8 &  4.53e-08 &  2.04e-15 &      \\ 
     &           &    9 &  6.42e-09 &  1.29e-15 &      \\ 
     &           &   10 &  6.41e-09 &  1.83e-16 &      \\ 
4007 &  4.01e+04 &   10 &           &           & iters  \\ 
 \hdashline 
     &           &    1 &  5.33e-10 &  1.07e-10 &      \\ 
     &           &    2 &  5.32e-10 &  1.52e-17 &      \\ 
     &           &    3 &  5.31e-10 &  1.52e-17 &      \\ 
     &           &    4 &  5.30e-10 &  1.52e-17 &      \\ 
     &           &    5 &  5.29e-10 &  1.51e-17 &      \\ 
     &           &    6 &  5.29e-10 &  1.51e-17 &      \\ 
     &           &    7 &  5.28e-10 &  1.51e-17 &      \\ 
     &           &    8 &  5.27e-10 &  1.51e-17 &      \\ 
     &           &    9 &  5.26e-10 &  1.50e-17 &      \\ 
     &           &   10 &  5.26e-10 &  1.50e-17 &      \\ 
4008 &  4.01e+04 &   10 &           &           & iters  \\ 
 \hdashline 
     &           &    1 &  3.46e-08 &  6.03e-11 &      \\ 
     &           &    2 &  4.31e-08 &  9.88e-16 &      \\ 
     &           &    3 &  3.46e-08 &  1.23e-15 &      \\ 
     &           &    4 &  3.46e-08 &  9.89e-16 &      \\ 
     &           &    5 &  4.32e-08 &  9.87e-16 &      \\ 
     &           &    6 &  3.46e-08 &  1.23e-15 &      \\ 
     &           &    7 &  4.31e-08 &  9.87e-16 &      \\ 
     &           &    8 &  3.46e-08 &  1.23e-15 &      \\ 
     &           &    9 &  4.31e-08 &  9.88e-16 &      \\ 
     &           &   10 &  3.46e-08 &  1.23e-15 &      \\ 
4009 &  4.01e+04 &   10 &           &           & iters  \\ 
 \hdashline 
     &           &    1 &  1.40e-08 &  6.94e-11 &      \\ 
     &           &    2 &  2.49e-08 &  4.00e-16 &      \\ 
     &           &    3 &  1.40e-08 &  7.09e-16 &      \\ 
     &           &    4 &  2.48e-08 &  4.00e-16 &      \\ 
     &           &    5 &  1.40e-08 &  7.09e-16 &      \\ 
     &           &    6 &  1.40e-08 &  4.01e-16 &      \\ 
     &           &    7 &  2.48e-08 &  4.00e-16 &      \\ 
     &           &    8 &  1.40e-08 &  7.09e-16 &      \\ 
     &           &    9 &  1.40e-08 &  4.01e-16 &      \\ 
     &           &   10 &  2.48e-08 &  4.00e-16 &      \\ 
4010 &  4.01e+04 &   10 &           &           & iters  \\ 
 \hdashline 
     &           &    1 &  2.60e-09 &  2.27e-10 &      \\ 
     &           &    2 &  2.59e-09 &  7.41e-17 &      \\ 
     &           &    3 &  2.59e-09 &  7.40e-17 &      \\ 
     &           &    4 &  2.59e-09 &  7.39e-17 &      \\ 
     &           &    5 &  2.58e-09 &  7.38e-17 &      \\ 
     &           &    6 &  2.58e-09 &  7.37e-17 &      \\ 
     &           &    7 &  2.57e-09 &  7.36e-17 &      \\ 
     &           &    8 &  2.57e-09 &  7.35e-17 &      \\ 
     &           &    9 &  3.63e-08 &  7.34e-17 &      \\ 
     &           &   10 &  2.62e-09 &  1.04e-15 &      \\ 
4011 &  4.01e+04 &   10 &           &           & iters  \\ 
 \hdashline 
     &           &    1 &  7.98e-08 &  2.69e-10 &      \\ 
     &           &    2 &  7.57e-08 &  2.28e-15 &      \\ 
     &           &    3 &  2.13e-09 &  2.16e-15 &      \\ 
     &           &    4 &  2.13e-09 &  6.08e-17 &      \\ 
     &           &    5 &  2.13e-09 &  6.08e-17 &      \\ 
     &           &    6 &  2.12e-09 &  6.07e-17 &      \\ 
     &           &    7 &  2.12e-09 &  6.06e-17 &      \\ 
     &           &    8 &  2.12e-09 &  6.05e-17 &      \\ 
     &           &    9 &  2.11e-09 &  6.04e-17 &      \\ 
     &           &   10 &  2.11e-09 &  6.03e-17 &      \\ 
4012 &  4.01e+04 &   10 &           &           & iters  \\ 
 \hdashline 
     &           &    1 &  2.16e-09 &  2.35e-10 &      \\ 
     &           &    2 &  2.16e-09 &  6.16e-17 &      \\ 
     &           &    3 &  2.15e-09 &  6.15e-17 &      \\ 
     &           &    4 &  2.15e-09 &  6.14e-17 &      \\ 
     &           &    5 &  2.15e-09 &  6.13e-17 &      \\ 
     &           &    6 &  2.14e-09 &  6.12e-17 &      \\ 
     &           &    7 &  2.14e-09 &  6.12e-17 &      \\ 
     &           &    8 &  2.14e-09 &  6.11e-17 &      \\ 
     &           &    9 &  2.13e-09 &  6.10e-17 &      \\ 
     &           &   10 &  2.13e-09 &  6.09e-17 &      \\ 
4013 &  4.01e+04 &   10 &           &           & iters  \\ 
 \hdashline 
     &           &    1 &  5.29e-09 &  6.91e-11 &      \\ 
     &           &    2 &  7.24e-08 &  1.51e-16 &      \\ 
     &           &    3 &  8.31e-08 &  2.07e-15 &      \\ 
     &           &    4 &  7.24e-08 &  2.37e-15 &      \\ 
     &           &    5 &  5.37e-09 &  2.07e-15 &      \\ 
     &           &    6 &  5.36e-09 &  1.53e-16 &      \\ 
     &           &    7 &  5.35e-09 &  1.53e-16 &      \\ 
     &           &    8 &  5.34e-09 &  1.53e-16 &      \\ 
     &           &    9 &  5.34e-09 &  1.53e-16 &      \\ 
     &           &   10 &  5.33e-09 &  1.52e-16 &      \\ 
4014 &  4.01e+04 &   10 &           &           & iters  \\ 
 \hdashline 
     &           &    1 &  8.46e-09 &  1.65e-10 &      \\ 
     &           &    2 &  8.44e-09 &  2.41e-16 &      \\ 
     &           &    3 &  8.43e-09 &  2.41e-16 &      \\ 
     &           &    4 &  8.42e-09 &  2.41e-16 &      \\ 
     &           &    5 &  8.41e-09 &  2.40e-16 &      \\ 
     &           &    6 &  8.40e-09 &  2.40e-16 &      \\ 
     &           &    7 &  8.38e-09 &  2.40e-16 &      \\ 
     &           &    8 &  6.93e-08 &  2.39e-16 &      \\ 
     &           &    9 &  8.47e-09 &  1.98e-15 &      \\ 
     &           &   10 &  8.46e-09 &  2.42e-16 &      \\ 
4015 &  4.01e+04 &   10 &           &           & iters  \\ 
 \hdashline 
     &           &    1 &  2.61e-08 &  1.95e-10 &      \\ 
     &           &    2 &  5.16e-08 &  7.44e-16 &      \\ 
     &           &    3 &  2.61e-08 &  1.47e-15 &      \\ 
     &           &    4 &  2.61e-08 &  7.45e-16 &      \\ 
     &           &    5 &  5.16e-08 &  7.44e-16 &      \\ 
     &           &    6 &  2.61e-08 &  1.47e-15 &      \\ 
     &           &    7 &  2.61e-08 &  7.45e-16 &      \\ 
     &           &    8 &  5.16e-08 &  7.44e-16 &      \\ 
     &           &    9 &  2.61e-08 &  1.47e-15 &      \\ 
     &           &   10 &  2.61e-08 &  7.45e-16 &      \\ 
4016 &  4.02e+04 &   10 &           &           & iters  \\ 
 \hdashline 
     &           &    1 &  3.20e-08 &  1.00e-10 &      \\ 
     &           &    2 &  3.20e-08 &  9.13e-16 &      \\ 
     &           &    3 &  4.58e-08 &  9.12e-16 &      \\ 
     &           &    4 &  3.20e-08 &  1.31e-15 &      \\ 
     &           &    5 &  3.19e-08 &  9.13e-16 &      \\ 
     &           &    6 &  4.58e-08 &  9.11e-16 &      \\ 
     &           &    7 &  3.19e-08 &  1.31e-15 &      \\ 
     &           &    8 &  4.58e-08 &  9.12e-16 &      \\ 
     &           &    9 &  3.20e-08 &  1.31e-15 &      \\ 
     &           &   10 &  4.58e-08 &  9.12e-16 &      \\ 
4017 &  4.02e+04 &   10 &           &           & iters  \\ 
 \hdashline 
     &           &    1 &  4.17e-08 &  1.72e-10 &      \\ 
     &           &    2 &  2.84e-09 &  1.19e-15 &      \\ 
     &           &    3 &  2.84e-09 &  8.11e-17 &      \\ 
     &           &    4 &  2.84e-09 &  8.10e-17 &      \\ 
     &           &    5 &  2.83e-09 &  8.09e-17 &      \\ 
     &           &    6 &  2.83e-09 &  8.08e-17 &      \\ 
     &           &    7 &  2.82e-09 &  8.07e-17 &      \\ 
     &           &    8 &  2.82e-09 &  8.06e-17 &      \\ 
     &           &    9 &  2.81e-09 &  8.05e-17 &      \\ 
     &           &   10 &  7.49e-08 &  8.03e-17 &      \\ 
4018 &  4.02e+04 &   10 &           &           & iters  \\ 
 \hdashline 
     &           &    1 &  3.48e-08 &  3.83e-10 &      \\ 
     &           &    2 &  3.47e-08 &  9.92e-16 &      \\ 
     &           &    3 &  4.30e-08 &  9.91e-16 &      \\ 
     &           &    4 &  3.47e-08 &  1.23e-15 &      \\ 
     &           &    5 &  4.30e-08 &  9.91e-16 &      \\ 
     &           &    6 &  3.47e-08 &  1.23e-15 &      \\ 
     &           &    7 &  4.30e-08 &  9.91e-16 &      \\ 
     &           &    8 &  3.47e-08 &  1.23e-15 &      \\ 
     &           &    9 &  3.47e-08 &  9.92e-16 &      \\ 
     &           &   10 &  4.30e-08 &  9.90e-16 &      \\ 
4019 &  4.02e+04 &   10 &           &           & iters  \\ 
 \hdashline 
     &           &    1 &  6.55e-10 &  2.95e-10 &      \\ 
     &           &    2 &  6.54e-10 &  1.87e-17 &      \\ 
     &           &    3 &  6.54e-10 &  1.87e-17 &      \\ 
     &           &    4 &  6.53e-10 &  1.87e-17 &      \\ 
     &           &    5 &  6.52e-10 &  1.86e-17 &      \\ 
     &           &    6 &  6.51e-10 &  1.86e-17 &      \\ 
     &           &    7 &  6.50e-10 &  1.86e-17 &      \\ 
     &           &    8 &  6.49e-10 &  1.85e-17 &      \\ 
     &           &    9 &  6.48e-10 &  1.85e-17 &      \\ 
     &           &   10 &  6.47e-10 &  1.85e-17 &      \\ 
4020 &  4.02e+04 &   10 &           &           & iters  \\ 
 \hdashline 
     &           &    1 &  5.55e-09 &  9.27e-11 &      \\ 
     &           &    2 &  5.54e-09 &  1.58e-16 &      \\ 
     &           &    3 &  5.53e-09 &  1.58e-16 &      \\ 
     &           &    4 &  5.53e-09 &  1.58e-16 &      \\ 
     &           &    5 &  5.52e-09 &  1.58e-16 &      \\ 
     &           &    6 &  5.51e-09 &  1.58e-16 &      \\ 
     &           &    7 &  5.50e-09 &  1.57e-16 &      \\ 
     &           &    8 &  5.50e-09 &  1.57e-16 &      \\ 
     &           &    9 &  5.49e-09 &  1.57e-16 &      \\ 
     &           &   10 &  7.22e-08 &  1.57e-16 &      \\ 
4021 &  4.02e+04 &   10 &           &           & iters  \\ 
 \hdashline 
     &           &    1 &  3.64e-08 &  3.30e-10 &      \\ 
     &           &    2 &  2.45e-09 &  1.04e-15 &      \\ 
     &           &    3 &  2.45e-09 &  6.99e-17 &      \\ 
     &           &    4 &  2.44e-09 &  6.98e-17 &      \\ 
     &           &    5 &  2.44e-09 &  6.97e-17 &      \\ 
     &           &    6 &  2.44e-09 &  6.96e-17 &      \\ 
     &           &    7 &  2.43e-09 &  6.95e-17 &      \\ 
     &           &    8 &  2.43e-09 &  6.94e-17 &      \\ 
     &           &    9 &  2.43e-09 &  6.93e-17 &      \\ 
     &           &   10 &  2.42e-09 &  6.92e-17 &      \\ 
4022 &  4.02e+04 &   10 &           &           & iters  \\ 
 \hdashline 
     &           &    1 &  3.64e-08 &  3.49e-10 &      \\ 
     &           &    2 &  4.13e-08 &  1.04e-15 &      \\ 
     &           &    3 &  3.64e-08 &  1.18e-15 &      \\ 
     &           &    4 &  4.13e-08 &  1.04e-15 &      \\ 
     &           &    5 &  3.64e-08 &  1.18e-15 &      \\ 
     &           &    6 &  4.13e-08 &  1.04e-15 &      \\ 
     &           &    7 &  3.64e-08 &  1.18e-15 &      \\ 
     &           &    8 &  3.64e-08 &  1.04e-15 &      \\ 
     &           &    9 &  4.14e-08 &  1.04e-15 &      \\ 
     &           &   10 &  3.64e-08 &  1.18e-15 &      \\ 
4023 &  4.02e+04 &   10 &           &           & iters  \\ 
 \hdashline 
     &           &    1 &  4.19e-08 &  2.97e-10 &      \\ 
     &           &    2 &  3.59e-08 &  1.19e-15 &      \\ 
     &           &    3 &  4.19e-08 &  1.02e-15 &      \\ 
     &           &    4 &  3.59e-08 &  1.19e-15 &      \\ 
     &           &    5 &  4.19e-08 &  1.02e-15 &      \\ 
     &           &    6 &  3.59e-08 &  1.19e-15 &      \\ 
     &           &    7 &  3.58e-08 &  1.02e-15 &      \\ 
     &           &    8 &  4.19e-08 &  1.02e-15 &      \\ 
     &           &    9 &  3.59e-08 &  1.20e-15 &      \\ 
     &           &   10 &  4.19e-08 &  1.02e-15 &      \\ 
4024 &  4.02e+04 &   10 &           &           & iters  \\ 
 \hdashline 
     &           &    1 &  4.86e-08 &  2.46e-11 &      \\ 
     &           &    2 &  2.92e-08 &  1.39e-15 &      \\ 
     &           &    3 &  4.85e-08 &  8.33e-16 &      \\ 
     &           &    4 &  2.92e-08 &  1.39e-15 &      \\ 
     &           &    5 &  2.92e-08 &  8.34e-16 &      \\ 
     &           &    6 &  4.86e-08 &  8.33e-16 &      \\ 
     &           &    7 &  2.92e-08 &  1.39e-15 &      \\ 
     &           &    8 &  2.92e-08 &  8.34e-16 &      \\ 
     &           &    9 &  4.86e-08 &  8.32e-16 &      \\ 
     &           &   10 &  2.92e-08 &  1.39e-15 &      \\ 
4025 &  4.02e+04 &   10 &           &           & iters  \\ 
 \hdashline 
     &           &    1 &  2.54e-10 &  4.19e-10 &      \\ 
     &           &    2 &  2.53e-10 &  7.24e-18 &      \\ 
     &           &    3 &  2.53e-10 &  7.23e-18 &      \\ 
     &           &    4 &  2.53e-10 &  7.22e-18 &      \\ 
     &           &    5 &  2.52e-10 &  7.21e-18 &      \\ 
     &           &    6 &  2.52e-10 &  7.20e-18 &      \\ 
     &           &    7 &  2.51e-10 &  7.19e-18 &      \\ 
     &           &    8 &  2.51e-10 &  7.18e-18 &      \\ 
     &           &    9 &  2.51e-10 &  7.17e-18 &      \\ 
     &           &   10 &  2.50e-10 &  7.16e-18 &      \\ 
4026 &  4.02e+04 &   10 &           &           & iters  \\ 
 \hdashline 
     &           &    1 &  1.54e-08 &  4.21e-10 &      \\ 
     &           &    2 &  1.54e-08 &  4.41e-16 &      \\ 
     &           &    3 &  1.54e-08 &  4.40e-16 &      \\ 
     &           &    4 &  6.23e-08 &  4.40e-16 &      \\ 
     &           &    5 &  1.55e-08 &  1.78e-15 &      \\ 
     &           &    6 &  1.54e-08 &  4.42e-16 &      \\ 
     &           &    7 &  1.54e-08 &  4.41e-16 &      \\ 
     &           &    8 &  1.54e-08 &  4.40e-16 &      \\ 
     &           &    9 &  6.23e-08 &  4.40e-16 &      \\ 
     &           &   10 &  1.55e-08 &  1.78e-15 &      \\ 
4027 &  4.03e+04 &   10 &           &           & iters  \\ 
 \hdashline 
     &           &    1 &  2.96e-08 &  9.21e-11 &      \\ 
     &           &    2 &  4.81e-08 &  8.44e-16 &      \\ 
     &           &    3 &  2.96e-08 &  1.37e-15 &      \\ 
     &           &    4 &  4.81e-08 &  8.45e-16 &      \\ 
     &           &    5 &  2.96e-08 &  1.37e-15 &      \\ 
     &           &    6 &  2.96e-08 &  8.46e-16 &      \\ 
     &           &    7 &  4.81e-08 &  8.45e-16 &      \\ 
     &           &    8 &  2.96e-08 &  1.37e-15 &      \\ 
     &           &    9 &  2.96e-08 &  8.45e-16 &      \\ 
     &           &   10 &  4.82e-08 &  8.44e-16 &      \\ 
4028 &  4.03e+04 &   10 &           &           & iters  \\ 
 \hdashline 
     &           &    1 &  3.07e-08 &  2.03e-10 &      \\ 
     &           &    2 &  4.71e-08 &  8.75e-16 &      \\ 
     &           &    3 &  3.07e-08 &  1.34e-15 &      \\ 
     &           &    4 &  4.70e-08 &  8.76e-16 &      \\ 
     &           &    5 &  3.07e-08 &  1.34e-15 &      \\ 
     &           &    6 &  3.07e-08 &  8.76e-16 &      \\ 
     &           &    7 &  4.71e-08 &  8.75e-16 &      \\ 
     &           &    8 &  3.07e-08 &  1.34e-15 &      \\ 
     &           &    9 &  4.70e-08 &  8.76e-16 &      \\ 
     &           &   10 &  3.07e-08 &  1.34e-15 &      \\ 
4029 &  4.03e+04 &   10 &           &           & iters  \\ 
 \hdashline 
     &           &    1 &  7.49e-09 &  3.23e-10 &      \\ 
     &           &    2 &  7.48e-09 &  2.14e-16 &      \\ 
     &           &    3 &  3.14e-08 &  2.13e-16 &      \\ 
     &           &    4 &  7.51e-09 &  8.96e-16 &      \\ 
     &           &    5 &  7.50e-09 &  2.14e-16 &      \\ 
     &           &    6 &  7.49e-09 &  2.14e-16 &      \\ 
     &           &    7 &  7.48e-09 &  2.14e-16 &      \\ 
     &           &    8 &  3.14e-08 &  2.13e-16 &      \\ 
     &           &    9 &  7.51e-09 &  8.96e-16 &      \\ 
     &           &   10 &  7.50e-09 &  2.14e-16 &      \\ 
4030 &  4.03e+04 &   10 &           &           & iters  \\ 
 \hdashline 
     &           &    1 &  4.79e-09 &  4.25e-10 &      \\ 
     &           &    2 &  4.78e-09 &  1.37e-16 &      \\ 
     &           &    3 &  4.77e-09 &  1.36e-16 &      \\ 
     &           &    4 &  4.77e-09 &  1.36e-16 &      \\ 
     &           &    5 &  4.76e-09 &  1.36e-16 &      \\ 
     &           &    6 &  7.29e-08 &  1.36e-16 &      \\ 
     &           &    7 &  8.25e-08 &  2.08e-15 &      \\ 
     &           &    8 &  7.29e-08 &  2.36e-15 &      \\ 
     &           &    9 &  4.84e-09 &  2.08e-15 &      \\ 
     &           &   10 &  4.84e-09 &  1.38e-16 &      \\ 
4031 &  4.03e+04 &   10 &           &           & iters  \\ 
 \hdashline 
     &           &    1 &  2.74e-08 &  3.32e-10 &      \\ 
     &           &    2 &  5.03e-08 &  7.82e-16 &      \\ 
     &           &    3 &  2.74e-08 &  1.44e-15 &      \\ 
     &           &    4 &  2.74e-08 &  7.82e-16 &      \\ 
     &           &    5 &  5.04e-08 &  7.81e-16 &      \\ 
     &           &    6 &  2.74e-08 &  1.44e-15 &      \\ 
     &           &    7 &  5.03e-08 &  7.82e-16 &      \\ 
     &           &    8 &  2.74e-08 &  1.44e-15 &      \\ 
     &           &    9 &  2.74e-08 &  7.83e-16 &      \\ 
     &           &   10 &  5.03e-08 &  7.82e-16 &      \\ 
4032 &  4.03e+04 &   10 &           &           & iters  \\ 
 \hdashline 
     &           &    1 &  2.35e-08 &  1.03e-10 &      \\ 
     &           &    2 &  2.34e-08 &  6.70e-16 &      \\ 
     &           &    3 &  5.43e-08 &  6.69e-16 &      \\ 
     &           &    4 &  2.35e-08 &  1.55e-15 &      \\ 
     &           &    5 &  2.34e-08 &  6.70e-16 &      \\ 
     &           &    6 &  5.43e-08 &  6.69e-16 &      \\ 
     &           &    7 &  2.35e-08 &  1.55e-15 &      \\ 
     &           &    8 &  2.34e-08 &  6.70e-16 &      \\ 
     &           &    9 &  2.34e-08 &  6.69e-16 &      \\ 
     &           &   10 &  5.43e-08 &  6.68e-16 &      \\ 
4033 &  4.03e+04 &   10 &           &           & iters  \\ 
 \hdashline 
     &           &    1 &  3.05e-08 &  2.26e-10 &      \\ 
     &           &    2 &  4.72e-08 &  8.71e-16 &      \\ 
     &           &    3 &  3.05e-08 &  1.35e-15 &      \\ 
     &           &    4 &  3.05e-08 &  8.72e-16 &      \\ 
     &           &    5 &  4.72e-08 &  8.70e-16 &      \\ 
     &           &    6 &  3.05e-08 &  1.35e-15 &      \\ 
     &           &    7 &  4.72e-08 &  8.71e-16 &      \\ 
     &           &    8 &  3.05e-08 &  1.35e-15 &      \\ 
     &           &    9 &  3.05e-08 &  8.72e-16 &      \\ 
     &           &   10 &  4.72e-08 &  8.71e-16 &      \\ 
4034 &  4.03e+04 &   10 &           &           & iters  \\ 
 \hdashline 
     &           &    1 &  3.99e-08 &  3.66e-10 &      \\ 
     &           &    2 &  3.79e-08 &  1.14e-15 &      \\ 
     &           &    3 &  3.99e-08 &  1.08e-15 &      \\ 
     &           &    4 &  3.79e-08 &  1.14e-15 &      \\ 
     &           &    5 &  3.99e-08 &  1.08e-15 &      \\ 
     &           &    6 &  3.79e-08 &  1.14e-15 &      \\ 
     &           &    7 &  3.99e-08 &  1.08e-15 &      \\ 
     &           &    8 &  3.79e-08 &  1.14e-15 &      \\ 
     &           &    9 &  3.98e-08 &  1.08e-15 &      \\ 
     &           &   10 &  3.79e-08 &  1.14e-15 &      \\ 
4035 &  4.03e+04 &   10 &           &           & iters  \\ 
 \hdashline 
     &           &    1 &  7.01e-09 &  1.23e-10 &      \\ 
     &           &    2 &  7.00e-09 &  2.00e-16 &      \\ 
     &           &    3 &  6.99e-09 &  2.00e-16 &      \\ 
     &           &    4 &  6.98e-09 &  1.99e-16 &      \\ 
     &           &    5 &  6.97e-09 &  1.99e-16 &      \\ 
     &           &    6 &  7.07e-08 &  1.99e-16 &      \\ 
     &           &    7 &  8.47e-08 &  2.02e-15 &      \\ 
     &           &    8 &  7.08e-08 &  2.42e-15 &      \\ 
     &           &    9 &  7.04e-09 &  2.02e-15 &      \\ 
     &           &   10 &  7.03e-09 &  2.01e-16 &      \\ 
4036 &  4.04e+04 &   10 &           &           & iters  \\ 
 \hdashline 
     &           &    1 &  1.03e-08 &  2.44e-10 &      \\ 
     &           &    2 &  1.03e-08 &  2.95e-16 &      \\ 
     &           &    3 &  1.03e-08 &  2.95e-16 &      \\ 
     &           &    4 &  1.03e-08 &  2.94e-16 &      \\ 
     &           &    5 &  6.74e-08 &  2.94e-16 &      \\ 
     &           &    6 &  1.04e-08 &  1.92e-15 &      \\ 
     &           &    7 &  1.04e-08 &  2.96e-16 &      \\ 
     &           &    8 &  1.03e-08 &  2.96e-16 &      \\ 
     &           &    9 &  1.03e-08 &  2.95e-16 &      \\ 
     &           &   10 &  1.03e-08 &  2.95e-16 &      \\ 
4037 &  4.04e+04 &   10 &           &           & iters  \\ 
 \hdashline 
     &           &    1 &  8.20e-08 &  4.45e-10 &      \\ 
     &           &    2 &  7.35e-08 &  2.34e-15 &      \\ 
     &           &    3 &  4.34e-09 &  2.10e-15 &      \\ 
     &           &    4 &  4.34e-09 &  1.24e-16 &      \\ 
     &           &    5 &  4.33e-09 &  1.24e-16 &      \\ 
     &           &    6 &  4.32e-09 &  1.24e-16 &      \\ 
     &           &    7 &  4.32e-09 &  1.23e-16 &      \\ 
     &           &    8 &  4.31e-09 &  1.23e-16 &      \\ 
     &           &    9 &  4.30e-09 &  1.23e-16 &      \\ 
     &           &   10 &  4.30e-09 &  1.23e-16 &      \\ 
4038 &  4.04e+04 &   10 &           &           & iters  \\ 
 \hdashline 
     &           &    1 &  2.41e-08 &  6.22e-10 &      \\ 
     &           &    2 &  1.48e-08 &  6.88e-16 &      \\ 
     &           &    3 &  1.48e-08 &  4.22e-16 &      \\ 
     &           &    4 &  2.41e-08 &  4.21e-16 &      \\ 
     &           &    5 &  1.48e-08 &  6.88e-16 &      \\ 
     &           &    6 &  1.47e-08 &  4.21e-16 &      \\ 
     &           &    7 &  2.41e-08 &  4.21e-16 &      \\ 
     &           &    8 &  1.48e-08 &  6.88e-16 &      \\ 
     &           &    9 &  2.41e-08 &  4.21e-16 &      \\ 
     &           &   10 &  1.48e-08 &  6.88e-16 &      \\ 
4039 &  4.04e+04 &   10 &           &           & iters  \\ 
 \hdashline 
     &           &    1 &  1.93e-08 &  5.73e-10 &      \\ 
     &           &    2 &  1.93e-08 &  5.52e-16 &      \\ 
     &           &    3 &  5.84e-08 &  5.51e-16 &      \\ 
     &           &    4 &  1.94e-08 &  1.67e-15 &      \\ 
     &           &    5 &  1.93e-08 &  5.53e-16 &      \\ 
     &           &    6 &  1.93e-08 &  5.52e-16 &      \\ 
     &           &    7 &  5.84e-08 &  5.51e-16 &      \\ 
     &           &    8 &  1.94e-08 &  1.67e-15 &      \\ 
     &           &    9 &  1.93e-08 &  5.53e-16 &      \\ 
     &           &   10 &  1.93e-08 &  5.52e-16 &      \\ 
4040 &  4.04e+04 &   10 &           &           & iters  \\ 
 \hdashline 
     &           &    1 &  5.14e-08 &  9.61e-11 &      \\ 
     &           &    2 &  2.63e-08 &  1.47e-15 &      \\ 
     &           &    3 &  2.63e-08 &  7.52e-16 &      \\ 
     &           &    4 &  5.14e-08 &  7.51e-16 &      \\ 
     &           &    5 &  2.63e-08 &  1.47e-15 &      \\ 
     &           &    6 &  2.63e-08 &  7.52e-16 &      \\ 
     &           &    7 &  5.14e-08 &  7.51e-16 &      \\ 
     &           &    8 &  2.63e-08 &  1.47e-15 &      \\ 
     &           &    9 &  2.63e-08 &  7.52e-16 &      \\ 
     &           &   10 &  5.14e-08 &  7.50e-16 &      \\ 
4041 &  4.04e+04 &   10 &           &           & iters  \\ 
 \hdashline 
     &           &    1 &  1.37e-08 &  8.84e-11 &      \\ 
     &           &    2 &  2.52e-08 &  3.91e-16 &      \\ 
     &           &    3 &  1.37e-08 &  7.18e-16 &      \\ 
     &           &    4 &  1.37e-08 &  3.92e-16 &      \\ 
     &           &    5 &  2.52e-08 &  3.91e-16 &      \\ 
     &           &    6 &  1.37e-08 &  7.18e-16 &      \\ 
     &           &    7 &  2.51e-08 &  3.91e-16 &      \\ 
     &           &    8 &  1.37e-08 &  7.18e-16 &      \\ 
     &           &    9 &  1.37e-08 &  3.92e-16 &      \\ 
     &           &   10 &  2.51e-08 &  3.91e-16 &      \\ 
4042 &  4.04e+04 &   10 &           &           & iters  \\ 
 \hdashline 
     &           &    1 &  4.17e-08 &  9.57e-11 &      \\ 
     &           &    2 &  3.60e-08 &  1.19e-15 &      \\ 
     &           &    3 &  3.60e-08 &  1.03e-15 &      \\ 
     &           &    4 &  4.18e-08 &  1.03e-15 &      \\ 
     &           &    5 &  3.60e-08 &  1.19e-15 &      \\ 
     &           &    6 &  4.18e-08 &  1.03e-15 &      \\ 
     &           &    7 &  3.60e-08 &  1.19e-15 &      \\ 
     &           &    8 &  4.18e-08 &  1.03e-15 &      \\ 
     &           &    9 &  3.60e-08 &  1.19e-15 &      \\ 
     &           &   10 &  4.18e-08 &  1.03e-15 &      \\ 
4043 &  4.04e+04 &   10 &           &           & iters  \\ 
 \hdashline 
     &           &    1 &  3.69e-08 &  5.03e-11 &      \\ 
     &           &    2 &  4.08e-08 &  1.05e-15 &      \\ 
     &           &    3 &  3.69e-08 &  1.16e-15 &      \\ 
     &           &    4 &  4.08e-08 &  1.05e-15 &      \\ 
     &           &    5 &  3.69e-08 &  1.16e-15 &      \\ 
     &           &    6 &  4.08e-08 &  1.05e-15 &      \\ 
     &           &    7 &  3.69e-08 &  1.16e-15 &      \\ 
     &           &    8 &  4.08e-08 &  1.05e-15 &      \\ 
     &           &    9 &  3.70e-08 &  1.16e-15 &      \\ 
     &           &   10 &  4.08e-08 &  1.05e-15 &      \\ 
4044 &  4.04e+04 &   10 &           &           & iters  \\ 
 \hdashline 
     &           &    1 &  2.22e-08 &  5.01e-11 &      \\ 
     &           &    2 &  2.22e-08 &  6.34e-16 &      \\ 
     &           &    3 &  1.67e-08 &  6.33e-16 &      \\ 
     &           &    4 &  2.22e-08 &  4.77e-16 &      \\ 
     &           &    5 &  1.67e-08 &  6.32e-16 &      \\ 
     &           &    6 &  2.21e-08 &  4.77e-16 &      \\ 
     &           &    7 &  1.67e-08 &  6.32e-16 &      \\ 
     &           &    8 &  1.67e-08 &  4.77e-16 &      \\ 
     &           &    9 &  2.22e-08 &  4.77e-16 &      \\ 
     &           &   10 &  1.67e-08 &  6.33e-16 &      \\ 
4045 &  4.04e+04 &   10 &           &           & iters  \\ 
 \hdashline 
     &           &    1 &  3.58e-08 &  4.30e-11 &      \\ 
     &           &    2 &  4.20e-08 &  1.02e-15 &      \\ 
     &           &    3 &  3.58e-08 &  1.20e-15 &      \\ 
     &           &    4 &  4.20e-08 &  1.02e-15 &      \\ 
     &           &    5 &  3.58e-08 &  1.20e-15 &      \\ 
     &           &    6 &  4.20e-08 &  1.02e-15 &      \\ 
     &           &    7 &  3.58e-08 &  1.20e-15 &      \\ 
     &           &    8 &  4.19e-08 &  1.02e-15 &      \\ 
     &           &    9 &  3.58e-08 &  1.20e-15 &      \\ 
     &           &   10 &  3.58e-08 &  1.02e-15 &      \\ 
4046 &  4.04e+04 &   10 &           &           & iters  \\ 
 \hdashline 
     &           &    1 &  5.84e-08 &  3.80e-11 &      \\ 
     &           &    2 &  1.94e-08 &  1.67e-15 &      \\ 
     &           &    3 &  1.94e-08 &  5.53e-16 &      \\ 
     &           &    4 &  5.84e-08 &  5.52e-16 &      \\ 
     &           &    5 &  1.94e-08 &  1.67e-15 &      \\ 
     &           &    6 &  1.94e-08 &  5.54e-16 &      \\ 
     &           &    7 &  1.94e-08 &  5.53e-16 &      \\ 
     &           &    8 &  5.84e-08 &  5.52e-16 &      \\ 
     &           &    9 &  1.94e-08 &  1.67e-15 &      \\ 
     &           &   10 &  1.94e-08 &  5.54e-16 &      \\ 
4047 &  4.05e+04 &   10 &           &           & iters  \\ 
 \hdashline 
     &           &    1 &  1.08e-08 &  1.04e-10 &      \\ 
     &           &    2 &  1.08e-08 &  3.09e-16 &      \\ 
     &           &    3 &  1.08e-08 &  3.09e-16 &      \\ 
     &           &    4 &  6.69e-08 &  3.09e-16 &      \\ 
     &           &    5 &  1.09e-08 &  1.91e-15 &      \\ 
     &           &    6 &  1.09e-08 &  3.11e-16 &      \\ 
     &           &    7 &  1.09e-08 &  3.10e-16 &      \\ 
     &           &    8 &  1.08e-08 &  3.10e-16 &      \\ 
     &           &    9 &  1.08e-08 &  3.10e-16 &      \\ 
     &           &   10 &  1.08e-08 &  3.09e-16 &      \\ 
4048 &  4.05e+04 &   10 &           &           & iters  \\ 
 \hdashline 
     &           &    1 &  7.26e-08 &  1.98e-10 &      \\ 
     &           &    2 &  8.29e-08 &  2.07e-15 &      \\ 
     &           &    3 &  7.26e-08 &  2.37e-15 &      \\ 
     &           &    4 &  5.17e-09 &  2.07e-15 &      \\ 
     &           &    5 &  5.17e-09 &  1.48e-16 &      \\ 
     &           &    6 &  5.16e-09 &  1.47e-16 &      \\ 
     &           &    7 &  5.15e-09 &  1.47e-16 &      \\ 
     &           &    8 &  5.14e-09 &  1.47e-16 &      \\ 
     &           &    9 &  5.14e-09 &  1.47e-16 &      \\ 
     &           &   10 &  5.13e-09 &  1.47e-16 &      \\ 
4049 &  4.05e+04 &   10 &           &           & iters  \\ 
 \hdashline 
     &           &    1 &  1.65e-08 &  2.04e-10 &      \\ 
     &           &    2 &  2.24e-08 &  4.71e-16 &      \\ 
     &           &    3 &  1.65e-08 &  6.38e-16 &      \\ 
     &           &    4 &  2.24e-08 &  4.71e-16 &      \\ 
     &           &    5 &  1.65e-08 &  6.38e-16 &      \\ 
     &           &    6 &  1.65e-08 &  4.71e-16 &      \\ 
     &           &    7 &  2.24e-08 &  4.71e-16 &      \\ 
     &           &    8 &  1.65e-08 &  6.39e-16 &      \\ 
     &           &    9 &  2.24e-08 &  4.71e-16 &      \\ 
     &           &   10 &  1.65e-08 &  6.38e-16 &      \\ 
4050 &  4.05e+04 &   10 &           &           & iters  \\ 
 \hdashline 
     &           &    1 &  9.46e-09 &  1.01e-10 &      \\ 
     &           &    2 &  9.44e-09 &  2.70e-16 &      \\ 
     &           &    3 &  9.43e-09 &  2.70e-16 &      \\ 
     &           &    4 &  9.42e-09 &  2.69e-16 &      \\ 
     &           &    5 &  9.40e-09 &  2.69e-16 &      \\ 
     &           &    6 &  6.83e-08 &  2.68e-16 &      \\ 
     &           &    7 &  9.49e-09 &  1.95e-15 &      \\ 
     &           &    8 &  9.47e-09 &  2.71e-16 &      \\ 
     &           &    9 &  9.46e-09 &  2.70e-16 &      \\ 
     &           &   10 &  9.45e-09 &  2.70e-16 &      \\ 
4051 &  4.05e+04 &   10 &           &           & iters  \\ 
 \hdashline 
     &           &    1 &  4.15e-08 &  5.11e-11 &      \\ 
     &           &    2 &  3.62e-08 &  1.18e-15 &      \\ 
     &           &    3 &  4.15e-08 &  1.03e-15 &      \\ 
     &           &    4 &  3.63e-08 &  1.18e-15 &      \\ 
     &           &    5 &  4.15e-08 &  1.03e-15 &      \\ 
     &           &    6 &  3.63e-08 &  1.18e-15 &      \\ 
     &           &    7 &  3.62e-08 &  1.03e-15 &      \\ 
     &           &    8 &  4.15e-08 &  1.03e-15 &      \\ 
     &           &    9 &  3.62e-08 &  1.19e-15 &      \\ 
     &           &   10 &  4.15e-08 &  1.03e-15 &      \\ 
4052 &  4.05e+04 &   10 &           &           & iters  \\ 
 \hdashline 
     &           &    1 &  1.66e-08 &  5.87e-11 &      \\ 
     &           &    2 &  2.23e-08 &  4.72e-16 &      \\ 
     &           &    3 &  1.66e-08 &  6.37e-16 &      \\ 
     &           &    4 &  1.65e-08 &  4.73e-16 &      \\ 
     &           &    5 &  2.23e-08 &  4.72e-16 &      \\ 
     &           &    6 &  1.65e-08 &  6.37e-16 &      \\ 
     &           &    7 &  2.23e-08 &  4.72e-16 &      \\ 
     &           &    8 &  1.66e-08 &  6.37e-16 &      \\ 
     &           &    9 &  2.23e-08 &  4.73e-16 &      \\ 
     &           &   10 &  1.66e-08 &  6.37e-16 &      \\ 
4053 &  4.05e+04 &   10 &           &           & iters  \\ 
 \hdashline 
     &           &    1 &  1.84e-08 &  1.25e-10 &      \\ 
     &           &    2 &  5.93e-08 &  5.26e-16 &      \\ 
     &           &    3 &  1.85e-08 &  1.69e-15 &      \\ 
     &           &    4 &  1.84e-08 &  5.27e-16 &      \\ 
     &           &    5 &  1.84e-08 &  5.27e-16 &      \\ 
     &           &    6 &  1.84e-08 &  5.26e-16 &      \\ 
     &           &    7 &  5.93e-08 &  5.25e-16 &      \\ 
     &           &    8 &  1.85e-08 &  1.69e-15 &      \\ 
     &           &    9 &  1.84e-08 &  5.27e-16 &      \\ 
     &           &   10 &  1.84e-08 &  5.26e-16 &      \\ 
4054 &  4.05e+04 &   10 &           &           & iters  \\ 
 \hdashline 
     &           &    1 &  3.25e-08 &  1.62e-10 &      \\ 
     &           &    2 &  3.24e-08 &  9.27e-16 &      \\ 
     &           &    3 &  4.53e-08 &  9.26e-16 &      \\ 
     &           &    4 &  3.25e-08 &  1.29e-15 &      \\ 
     &           &    5 &  4.53e-08 &  9.26e-16 &      \\ 
     &           &    6 &  3.25e-08 &  1.29e-15 &      \\ 
     &           &    7 &  3.24e-08 &  9.27e-16 &      \\ 
     &           &    8 &  4.53e-08 &  9.26e-16 &      \\ 
     &           &    9 &  3.24e-08 &  1.29e-15 &      \\ 
     &           &   10 &  4.53e-08 &  9.26e-16 &      \\ 
4055 &  4.05e+04 &   10 &           &           & iters  \\ 
 \hdashline 
     &           &    1 &  3.54e-09 &  1.92e-11 &      \\ 
     &           &    2 &  3.53e-09 &  1.01e-16 &      \\ 
     &           &    3 &  3.53e-09 &  1.01e-16 &      \\ 
     &           &    4 &  3.52e-09 &  1.01e-16 &      \\ 
     &           &    5 &  3.52e-09 &  1.00e-16 &      \\ 
     &           &    6 &  3.51e-09 &  1.00e-16 &      \\ 
     &           &    7 &  3.50e-09 &  1.00e-16 &      \\ 
     &           &    8 &  7.42e-08 &  1.00e-16 &      \\ 
     &           &    9 &  8.13e-08 &  2.12e-15 &      \\ 
     &           &   10 &  7.42e-08 &  2.32e-15 &      \\ 
4056 &  4.06e+04 &   10 &           &           & iters  \\ 
 \hdashline 
     &           &    1 &  7.20e-08 &  7.93e-11 &      \\ 
     &           &    2 &  5.82e-09 &  2.05e-15 &      \\ 
     &           &    3 &  5.81e-09 &  1.66e-16 &      \\ 
     &           &    4 &  5.80e-09 &  1.66e-16 &      \\ 
     &           &    5 &  5.80e-09 &  1.66e-16 &      \\ 
     &           &    6 &  5.79e-09 &  1.65e-16 &      \\ 
     &           &    7 &  7.19e-08 &  1.65e-16 &      \\ 
     &           &    8 &  8.36e-08 &  2.05e-15 &      \\ 
     &           &    9 &  7.19e-08 &  2.39e-15 &      \\ 
     &           &   10 &  5.86e-09 &  2.05e-15 &      \\ 
4057 &  4.06e+04 &   10 &           &           & iters  \\ 
 \hdashline 
     &           &    1 &  8.88e-08 &  6.90e-12 &      \\ 
     &           &    2 &  6.67e-08 &  2.53e-15 &      \\ 
     &           &    3 &  1.11e-08 &  1.90e-15 &      \\ 
     &           &    4 &  1.11e-08 &  3.16e-16 &      \\ 
     &           &    5 &  1.11e-08 &  3.16e-16 &      \\ 
     &           &    6 &  1.10e-08 &  3.15e-16 &      \\ 
     &           &    7 &  1.10e-08 &  3.15e-16 &      \\ 
     &           &    8 &  1.10e-08 &  3.15e-16 &      \\ 
     &           &    9 &  6.67e-08 &  3.14e-16 &      \\ 
     &           &   10 &  1.11e-08 &  1.90e-15 &      \\ 
4058 &  4.06e+04 &   10 &           &           & iters  \\ 
 \hdashline 
     &           &    1 &  4.71e-09 &  7.59e-11 &      \\ 
     &           &    2 &  4.70e-09 &  1.34e-16 &      \\ 
     &           &    3 &  4.70e-09 &  1.34e-16 &      \\ 
     &           &    4 &  4.69e-09 &  1.34e-16 &      \\ 
     &           &    5 &  4.68e-09 &  1.34e-16 &      \\ 
     &           &    6 &  4.68e-09 &  1.34e-16 &      \\ 
     &           &    7 &  4.67e-09 &  1.34e-16 &      \\ 
     &           &    8 &  4.66e-09 &  1.33e-16 &      \\ 
     &           &    9 &  4.66e-09 &  1.33e-16 &      \\ 
     &           &   10 &  4.65e-09 &  1.33e-16 &      \\ 
4059 &  4.06e+04 &   10 &           &           & iters  \\ 
 \hdashline 
     &           &    1 &  2.75e-08 &  9.85e-14 &      \\ 
     &           &    2 &  5.02e-08 &  7.84e-16 &      \\ 
     &           &    3 &  2.75e-08 &  1.43e-15 &      \\ 
     &           &    4 &  2.75e-08 &  7.85e-16 &      \\ 
     &           &    5 &  5.02e-08 &  7.84e-16 &      \\ 
     &           &    6 &  2.75e-08 &  1.43e-15 &      \\ 
     &           &    7 &  2.75e-08 &  7.85e-16 &      \\ 
     &           &    8 &  5.03e-08 &  7.84e-16 &      \\ 
     &           &    9 &  2.75e-08 &  1.43e-15 &      \\ 
     &           &   10 &  2.75e-08 &  7.85e-16 &      \\ 
4060 &  4.06e+04 &   10 &           &           & iters  \\ 
 \hdashline 
     &           &    1 &  2.22e-08 &  9.54e-11 &      \\ 
     &           &    2 &  5.55e-08 &  6.33e-16 &      \\ 
     &           &    3 &  2.22e-08 &  1.59e-15 &      \\ 
     &           &    4 &  2.22e-08 &  6.34e-16 &      \\ 
     &           &    5 &  2.22e-08 &  6.33e-16 &      \\ 
     &           &    6 &  5.56e-08 &  6.32e-16 &      \\ 
     &           &    7 &  2.22e-08 &  1.59e-15 &      \\ 
     &           &    8 &  2.22e-08 &  6.34e-16 &      \\ 
     &           &    9 &  5.55e-08 &  6.33e-16 &      \\ 
     &           &   10 &  2.22e-08 &  1.59e-15 &      \\ 
4061 &  4.06e+04 &   10 &           &           & iters  \\ 
 \hdashline 
     &           &    1 &  4.71e-08 &  6.53e-11 &      \\ 
     &           &    2 &  3.07e-08 &  1.34e-15 &      \\ 
     &           &    3 &  3.06e-08 &  8.75e-16 &      \\ 
     &           &    4 &  4.71e-08 &  8.74e-16 &      \\ 
     &           &    5 &  3.06e-08 &  1.34e-15 &      \\ 
     &           &    6 &  4.71e-08 &  8.74e-16 &      \\ 
     &           &    7 &  3.07e-08 &  1.34e-15 &      \\ 
     &           &    8 &  3.06e-08 &  8.75e-16 &      \\ 
     &           &    9 &  4.71e-08 &  8.74e-16 &      \\ 
     &           &   10 &  3.06e-08 &  1.34e-15 &      \\ 
4062 &  4.06e+04 &   10 &           &           & iters  \\ 
 \hdashline 
     &           &    1 &  1.07e-08 &  1.94e-11 &      \\ 
     &           &    2 &  1.07e-08 &  3.07e-16 &      \\ 
     &           &    3 &  1.07e-08 &  3.06e-16 &      \\ 
     &           &    4 &  2.82e-08 &  3.06e-16 &      \\ 
     &           &    5 &  1.07e-08 &  8.03e-16 &      \\ 
     &           &    6 &  1.07e-08 &  3.06e-16 &      \\ 
     &           &    7 &  2.81e-08 &  3.06e-16 &      \\ 
     &           &    8 &  1.07e-08 &  8.03e-16 &      \\ 
     &           &    9 &  1.07e-08 &  3.07e-16 &      \\ 
     &           &   10 &  1.07e-08 &  3.06e-16 &      \\ 
4063 &  4.06e+04 &   10 &           &           & iters  \\ 
 \hdashline 
     &           &    1 &  6.12e-09 &  5.81e-12 &      \\ 
     &           &    2 &  6.11e-09 &  1.75e-16 &      \\ 
     &           &    3 &  6.10e-09 &  1.74e-16 &      \\ 
     &           &    4 &  6.09e-09 &  1.74e-16 &      \\ 
     &           &    5 &  6.08e-09 &  1.74e-16 &      \\ 
     &           &    6 &  6.08e-09 &  1.74e-16 &      \\ 
     &           &    7 &  6.07e-09 &  1.73e-16 &      \\ 
     &           &    8 &  6.06e-09 &  1.73e-16 &      \\ 
     &           &    9 &  6.05e-09 &  1.73e-16 &      \\ 
     &           &   10 &  6.04e-09 &  1.73e-16 &      \\ 
4064 &  4.06e+04 &   10 &           &           & iters  \\ 
 \hdashline 
     &           &    1 &  7.96e-08 &  3.21e-11 &      \\ 
     &           &    2 &  3.71e-08 &  2.27e-15 &      \\ 
     &           &    3 &  1.82e-09 &  1.06e-15 &      \\ 
     &           &    4 &  1.81e-09 &  5.19e-17 &      \\ 
     &           &    5 &  1.81e-09 &  5.18e-17 &      \\ 
     &           &    6 &  1.81e-09 &  5.17e-17 &      \\ 
     &           &    7 &  1.81e-09 &  5.16e-17 &      \\ 
     &           &    8 &  1.80e-09 &  5.16e-17 &      \\ 
     &           &    9 &  1.80e-09 &  5.15e-17 &      \\ 
     &           &   10 &  1.80e-09 &  5.14e-17 &      \\ 
4065 &  4.06e+04 &   10 &           &           & iters  \\ 
 \hdashline 
     &           &    1 &  1.59e-08 &  2.64e-11 &      \\ 
     &           &    2 &  1.59e-08 &  4.54e-16 &      \\ 
     &           &    3 &  6.18e-08 &  4.54e-16 &      \\ 
     &           &    4 &  1.60e-08 &  1.76e-15 &      \\ 
     &           &    5 &  1.59e-08 &  4.55e-16 &      \\ 
     &           &    6 &  1.59e-08 &  4.55e-16 &      \\ 
     &           &    7 &  6.18e-08 &  4.54e-16 &      \\ 
     &           &    8 &  1.60e-08 &  1.76e-15 &      \\ 
     &           &    9 &  1.60e-08 &  4.56e-16 &      \\ 
     &           &   10 &  1.59e-08 &  4.55e-16 &      \\ 
4066 &  4.06e+04 &   10 &           &           & iters  \\ 
 \hdashline 
     &           &    1 &  5.53e-08 &  1.88e-11 &      \\ 
     &           &    2 &  2.25e-08 &  1.58e-15 &      \\ 
     &           &    3 &  2.24e-08 &  6.41e-16 &      \\ 
     &           &    4 &  2.24e-08 &  6.41e-16 &      \\ 
     &           &    5 &  5.53e-08 &  6.40e-16 &      \\ 
     &           &    6 &  2.25e-08 &  1.58e-15 &      \\ 
     &           &    7 &  2.24e-08 &  6.41e-16 &      \\ 
     &           &    8 &  5.53e-08 &  6.40e-16 &      \\ 
     &           &    9 &  2.25e-08 &  1.58e-15 &      \\ 
     &           &   10 &  2.24e-08 &  6.41e-16 &      \\ 
4067 &  4.07e+04 &   10 &           &           & iters  \\ 
 \hdashline 
     &           &    1 &  6.01e-09 &  1.70e-11 &      \\ 
     &           &    2 &  6.00e-09 &  1.72e-16 &      \\ 
     &           &    3 &  3.28e-08 &  1.71e-16 &      \\ 
     &           &    4 &  6.04e-09 &  9.38e-16 &      \\ 
     &           &    5 &  6.03e-09 &  1.72e-16 &      \\ 
     &           &    6 &  6.03e-09 &  1.72e-16 &      \\ 
     &           &    7 &  6.02e-09 &  1.72e-16 &      \\ 
     &           &    8 &  6.01e-09 &  1.72e-16 &      \\ 
     &           &    9 &  3.28e-08 &  1.71e-16 &      \\ 
     &           &   10 &  6.05e-09 &  9.37e-16 &      \\ 
4068 &  4.07e+04 &   10 &           &           & iters  \\ 
 \hdashline 
     &           &    1 &  5.11e-08 &  1.37e-11 &      \\ 
     &           &    2 &  2.66e-08 &  1.46e-15 &      \\ 
     &           &    3 &  2.66e-08 &  7.61e-16 &      \\ 
     &           &    4 &  5.11e-08 &  7.60e-16 &      \\ 
     &           &    5 &  2.66e-08 &  1.46e-15 &      \\ 
     &           &    6 &  2.66e-08 &  7.61e-16 &      \\ 
     &           &    7 &  5.11e-08 &  7.59e-16 &      \\ 
     &           &    8 &  2.66e-08 &  1.46e-15 &      \\ 
     &           &    9 &  2.66e-08 &  7.60e-16 &      \\ 
     &           &   10 &  5.11e-08 &  7.59e-16 &      \\ 
4069 &  4.07e+04 &   10 &           &           & iters  \\ 
 \hdashline 
     &           &    1 &  4.71e-08 &  5.17e-12 &      \\ 
     &           &    2 &  8.20e-09 &  1.34e-15 &      \\ 
     &           &    3 &  8.19e-09 &  2.34e-16 &      \\ 
     &           &    4 &  8.17e-09 &  2.34e-16 &      \\ 
     &           &    5 &  6.95e-08 &  2.33e-16 &      \\ 
     &           &    6 &  4.71e-08 &  1.98e-15 &      \\ 
     &           &    7 &  8.19e-09 &  1.34e-15 &      \\ 
     &           &    8 &  8.18e-09 &  2.34e-16 &      \\ 
     &           &    9 &  8.17e-09 &  2.34e-16 &      \\ 
     &           &   10 &  6.95e-08 &  2.33e-16 &      \\ 
4070 &  4.07e+04 &   10 &           &           & iters  \\ 
 \hdashline 
     &           &    1 &  4.68e-08 &  3.84e-11 &      \\ 
     &           &    2 &  7.90e-09 &  1.34e-15 &      \\ 
     &           &    3 &  7.89e-09 &  2.25e-16 &      \\ 
     &           &    4 &  7.88e-09 &  2.25e-16 &      \\ 
     &           &    5 &  6.98e-08 &  2.25e-16 &      \\ 
     &           &    6 &  4.68e-08 &  1.99e-15 &      \\ 
     &           &    7 &  7.90e-09 &  1.34e-15 &      \\ 
     &           &    8 &  7.89e-09 &  2.25e-16 &      \\ 
     &           &    9 &  6.98e-08 &  2.25e-16 &      \\ 
     &           &   10 &  4.68e-08 &  1.99e-15 &      \\ 
4071 &  4.07e+04 &   10 &           &           & iters  \\ 
 \hdashline 
     &           &    1 &  4.45e-08 &  2.06e-11 &      \\ 
     &           &    2 &  3.33e-08 &  1.27e-15 &      \\ 
     &           &    3 &  3.32e-08 &  9.50e-16 &      \\ 
     &           &    4 &  4.45e-08 &  9.49e-16 &      \\ 
     &           &    5 &  3.33e-08 &  1.27e-15 &      \\ 
     &           &    6 &  4.45e-08 &  9.49e-16 &      \\ 
     &           &    7 &  3.33e-08 &  1.27e-15 &      \\ 
     &           &    8 &  4.45e-08 &  9.50e-16 &      \\ 
     &           &    9 &  3.33e-08 &  1.27e-15 &      \\ 
     &           &   10 &  3.33e-08 &  9.50e-16 &      \\ 
4072 &  4.07e+04 &   10 &           &           & iters  \\ 
 \hdashline 
     &           &    1 &  3.69e-08 &  2.52e-12 &      \\ 
     &           &    2 &  4.09e-08 &  1.05e-15 &      \\ 
     &           &    3 &  3.69e-08 &  1.17e-15 &      \\ 
     &           &    4 &  4.09e-08 &  1.05e-15 &      \\ 
     &           &    5 &  3.69e-08 &  1.17e-15 &      \\ 
     &           &    6 &  4.09e-08 &  1.05e-15 &      \\ 
     &           &    7 &  3.69e-08 &  1.17e-15 &      \\ 
     &           &    8 &  4.08e-08 &  1.05e-15 &      \\ 
     &           &    9 &  3.69e-08 &  1.17e-15 &      \\ 
     &           &   10 &  3.68e-08 &  1.05e-15 &      \\ 
4073 &  4.07e+04 &   10 &           &           & iters  \\ 
 \hdashline 
     &           &    1 &  4.08e-08 &  1.30e-11 &      \\ 
     &           &    2 &  1.93e-09 &  1.17e-15 &      \\ 
     &           &    3 &  1.93e-09 &  5.51e-17 &      \\ 
     &           &    4 &  1.93e-09 &  5.50e-17 &      \\ 
     &           &    5 &  1.92e-09 &  5.50e-17 &      \\ 
     &           &    6 &  1.92e-09 &  5.49e-17 &      \\ 
     &           &    7 &  1.92e-09 &  5.48e-17 &      \\ 
     &           &    8 &  1.91e-09 &  5.47e-17 &      \\ 
     &           &    9 &  1.91e-09 &  5.46e-17 &      \\ 
     &           &   10 &  1.91e-09 &  5.46e-17 &      \\ 
4074 &  4.07e+04 &   10 &           &           & iters  \\ 
 \hdashline 
     &           &    1 &  1.71e-08 &  9.51e-12 &      \\ 
     &           &    2 &  1.71e-08 &  4.89e-16 &      \\ 
     &           &    3 &  2.18e-08 &  4.88e-16 &      \\ 
     &           &    4 &  1.71e-08 &  6.21e-16 &      \\ 
     &           &    5 &  2.18e-08 &  4.88e-16 &      \\ 
     &           &    6 &  1.71e-08 &  6.21e-16 &      \\ 
     &           &    7 &  1.71e-08 &  4.89e-16 &      \\ 
     &           &    8 &  2.18e-08 &  4.88e-16 &      \\ 
     &           &    9 &  1.71e-08 &  6.21e-16 &      \\ 
     &           &   10 &  2.18e-08 &  4.88e-16 &      \\ 
4075 &  4.07e+04 &   10 &           &           & iters  \\ 
 \hdashline 
     &           &    1 &  1.92e-08 &  1.72e-11 &      \\ 
     &           &    2 &  1.92e-08 &  5.48e-16 &      \\ 
     &           &    3 &  5.85e-08 &  5.48e-16 &      \\ 
     &           &    4 &  1.92e-08 &  1.67e-15 &      \\ 
     &           &    5 &  1.92e-08 &  5.49e-16 &      \\ 
     &           &    6 &  1.92e-08 &  5.48e-16 &      \\ 
     &           &    7 &  5.85e-08 &  5.48e-16 &      \\ 
     &           &    8 &  1.92e-08 &  1.67e-15 &      \\ 
     &           &    9 &  1.92e-08 &  5.49e-16 &      \\ 
     &           &   10 &  1.92e-08 &  5.48e-16 &      \\ 
4076 &  4.08e+04 &   10 &           &           & iters  \\ 
 \hdashline 
     &           &    1 &  5.16e-08 &  2.55e-11 &      \\ 
     &           &    2 &  2.62e-08 &  1.47e-15 &      \\ 
     &           &    3 &  2.62e-08 &  7.48e-16 &      \\ 
     &           &    4 &  5.16e-08 &  7.47e-16 &      \\ 
     &           &    5 &  2.62e-08 &  1.47e-15 &      \\ 
     &           &    6 &  2.62e-08 &  7.48e-16 &      \\ 
     &           &    7 &  5.16e-08 &  7.46e-16 &      \\ 
     &           &    8 &  2.62e-08 &  1.47e-15 &      \\ 
     &           &    9 &  2.62e-08 &  7.48e-16 &      \\ 
     &           &   10 &  5.16e-08 &  7.46e-16 &      \\ 
4077 &  4.08e+04 &   10 &           &           & iters  \\ 
 \hdashline 
     &           &    1 &  4.60e-08 &  8.80e-12 &      \\ 
     &           &    2 &  3.18e-08 &  1.31e-15 &      \\ 
     &           &    3 &  4.59e-08 &  9.07e-16 &      \\ 
     &           &    4 &  3.18e-08 &  1.31e-15 &      \\ 
     &           &    5 &  3.18e-08 &  9.08e-16 &      \\ 
     &           &    6 &  4.60e-08 &  9.06e-16 &      \\ 
     &           &    7 &  3.18e-08 &  1.31e-15 &      \\ 
     &           &    8 &  4.60e-08 &  9.07e-16 &      \\ 
     &           &    9 &  3.18e-08 &  1.31e-15 &      \\ 
     &           &   10 &  3.18e-08 &  9.08e-16 &      \\ 
4078 &  4.08e+04 &   10 &           &           & iters  \\ 
 \hdashline 
     &           &    1 &  2.57e-08 &  2.45e-11 &      \\ 
     &           &    2 &  2.57e-08 &  7.35e-16 &      \\ 
     &           &    3 &  5.20e-08 &  7.34e-16 &      \\ 
     &           &    4 &  2.57e-08 &  1.48e-15 &      \\ 
     &           &    5 &  2.57e-08 &  7.35e-16 &      \\ 
     &           &    6 &  5.20e-08 &  7.34e-16 &      \\ 
     &           &    7 &  2.57e-08 &  1.48e-15 &      \\ 
     &           &    8 &  2.57e-08 &  7.35e-16 &      \\ 
     &           &    9 &  5.20e-08 &  7.34e-16 &      \\ 
     &           &   10 &  2.57e-08 &  1.48e-15 &      \\ 
4079 &  4.08e+04 &   10 &           &           & iters  \\ 
 \hdashline 
     &           &    1 &  3.99e-10 &  1.69e-11 &      \\ 
     &           &    2 &  3.99e-10 &  1.14e-17 &      \\ 
     &           &    3 &  3.98e-10 &  1.14e-17 &      \\ 
     &           &    4 &  3.98e-10 &  1.14e-17 &      \\ 
     &           &    5 &  3.97e-10 &  1.13e-17 &      \\ 
     &           &    6 &  3.96e-10 &  1.13e-17 &      \\ 
     &           &    7 &  3.96e-10 &  1.13e-17 &      \\ 
     &           &    8 &  3.95e-10 &  1.13e-17 &      \\ 
     &           &    9 &  3.95e-10 &  1.13e-17 &      \\ 
     &           &   10 &  3.94e-10 &  1.13e-17 &      \\ 
4080 &  4.08e+04 &   10 &           &           & iters  \\ 
 \hdashline 
     &           &    1 &  1.50e-08 &  1.55e-11 &      \\ 
     &           &    2 &  6.27e-08 &  4.28e-16 &      \\ 
     &           &    3 &  1.51e-08 &  1.79e-15 &      \\ 
     &           &    4 &  1.50e-08 &  4.30e-16 &      \\ 
     &           &    5 &  1.50e-08 &  4.29e-16 &      \\ 
     &           &    6 &  1.50e-08 &  4.28e-16 &      \\ 
     &           &    7 &  6.27e-08 &  4.28e-16 &      \\ 
     &           &    8 &  1.51e-08 &  1.79e-15 &      \\ 
     &           &    9 &  1.50e-08 &  4.30e-16 &      \\ 
     &           &   10 &  1.50e-08 &  4.29e-16 &      \\ 
4081 &  4.08e+04 &   10 &           &           & iters  \\ 
 \hdashline 
     &           &    1 &  1.99e-08 &  3.33e-11 &      \\ 
     &           &    2 &  1.89e-08 &  5.69e-16 &      \\ 
     &           &    3 &  1.99e-08 &  5.41e-16 &      \\ 
     &           &    4 &  1.89e-08 &  5.69e-16 &      \\ 
     &           &    5 &  1.99e-08 &  5.41e-16 &      \\ 
     &           &    6 &  1.89e-08 &  5.69e-16 &      \\ 
     &           &    7 &  1.99e-08 &  5.41e-16 &      \\ 
     &           &    8 &  1.90e-08 &  5.69e-16 &      \\ 
     &           &    9 &  1.99e-08 &  5.41e-16 &      \\ 
     &           &   10 &  1.90e-08 &  5.69e-16 &      \\ 
4082 &  4.08e+04 &   10 &           &           & iters  \\ 
 \hdashline 
     &           &    1 &  1.89e-08 &  1.22e-11 &      \\ 
     &           &    2 &  1.89e-08 &  5.40e-16 &      \\ 
     &           &    3 &  5.88e-08 &  5.39e-16 &      \\ 
     &           &    4 &  1.90e-08 &  1.68e-15 &      \\ 
     &           &    5 &  1.89e-08 &  5.41e-16 &      \\ 
     &           &    6 &  1.89e-08 &  5.40e-16 &      \\ 
     &           &    7 &  5.88e-08 &  5.39e-16 &      \\ 
     &           &    8 &  1.90e-08 &  1.68e-15 &      \\ 
     &           &    9 &  1.89e-08 &  5.41e-16 &      \\ 
     &           &   10 &  1.89e-08 &  5.40e-16 &      \\ 
4083 &  4.08e+04 &   10 &           &           & iters  \\ 
 \hdashline 
     &           &    1 &  2.24e-08 &  1.18e-11 &      \\ 
     &           &    2 &  2.24e-08 &  6.40e-16 &      \\ 
     &           &    3 &  2.24e-08 &  6.39e-16 &      \\ 
     &           &    4 &  5.53e-08 &  6.39e-16 &      \\ 
     &           &    5 &  2.24e-08 &  1.58e-15 &      \\ 
     &           &    6 &  2.24e-08 &  6.40e-16 &      \\ 
     &           &    7 &  5.53e-08 &  6.39e-16 &      \\ 
     &           &    8 &  2.24e-08 &  1.58e-15 &      \\ 
     &           &    9 &  2.24e-08 &  6.40e-16 &      \\ 
     &           &   10 &  2.24e-08 &  6.39e-16 &      \\ 
4084 &  4.08e+04 &   10 &           &           & iters  \\ 
 \hdashline 
     &           &    1 &  2.76e-08 &  1.18e-11 &      \\ 
     &           &    2 &  2.76e-08 &  7.88e-16 &      \\ 
     &           &    3 &  1.13e-08 &  7.87e-16 &      \\ 
     &           &    4 &  1.13e-08 &  3.22e-16 &      \\ 
     &           &    5 &  2.76e-08 &  3.22e-16 &      \\ 
     &           &    6 &  1.13e-08 &  7.87e-16 &      \\ 
     &           &    7 &  1.13e-08 &  3.23e-16 &      \\ 
     &           &    8 &  1.13e-08 &  3.22e-16 &      \\ 
     &           &    9 &  2.76e-08 &  3.22e-16 &      \\ 
     &           &   10 &  1.13e-08 &  7.87e-16 &      \\ 
4085 &  4.08e+04 &   10 &           &           & iters  \\ 
 \hdashline 
     &           &    1 &  2.48e-08 &  2.15e-12 &      \\ 
     &           &    2 &  5.29e-08 &  7.07e-16 &      \\ 
     &           &    3 &  2.48e-08 &  1.51e-15 &      \\ 
     &           &    4 &  2.48e-08 &  7.08e-16 &      \\ 
     &           &    5 &  5.29e-08 &  7.07e-16 &      \\ 
     &           &    6 &  2.48e-08 &  1.51e-15 &      \\ 
     &           &    7 &  2.48e-08 &  7.09e-16 &      \\ 
     &           &    8 &  5.29e-08 &  7.08e-16 &      \\ 
     &           &    9 &  2.48e-08 &  1.51e-15 &      \\ 
     &           &   10 &  2.48e-08 &  7.09e-16 &      \\ 
4086 &  4.08e+04 &   10 &           &           & iters  \\ 
 \hdashline 
     &           &    1 &  1.50e-08 &  8.24e-12 &      \\ 
     &           &    2 &  1.50e-08 &  4.30e-16 &      \\ 
     &           &    3 &  1.50e-08 &  4.29e-16 &      \\ 
     &           &    4 &  6.27e-08 &  4.28e-16 &      \\ 
     &           &    5 &  1.51e-08 &  1.79e-15 &      \\ 
     &           &    6 &  1.51e-08 &  4.30e-16 &      \\ 
     &           &    7 &  1.50e-08 &  4.30e-16 &      \\ 
     &           &    8 &  1.50e-08 &  4.29e-16 &      \\ 
     &           &    9 &  1.50e-08 &  4.28e-16 &      \\ 
     &           &   10 &  6.27e-08 &  4.28e-16 &      \\ 
4087 &  4.09e+04 &   10 &           &           & iters  \\ 
 \hdashline 
     &           &    1 &  6.60e-08 &  1.04e-11 &      \\ 
     &           &    2 &  1.18e-08 &  1.88e-15 &      \\ 
     &           &    3 &  1.18e-08 &  3.37e-16 &      \\ 
     &           &    4 &  1.18e-08 &  3.37e-16 &      \\ 
     &           &    5 &  1.18e-08 &  3.36e-16 &      \\ 
     &           &    6 &  1.18e-08 &  3.36e-16 &      \\ 
     &           &    7 &  1.17e-08 &  3.36e-16 &      \\ 
     &           &    8 &  6.60e-08 &  3.35e-16 &      \\ 
     &           &    9 &  1.18e-08 &  1.88e-15 &      \\ 
     &           &   10 &  1.18e-08 &  3.37e-16 &      \\ 
4088 &  4.09e+04 &   10 &           &           & iters  \\ 
 \hdashline 
     &           &    1 &  1.58e-08 &  2.65e-12 &      \\ 
     &           &    2 &  1.58e-08 &  4.52e-16 &      \\ 
     &           &    3 &  1.58e-08 &  4.51e-16 &      \\ 
     &           &    4 &  6.19e-08 &  4.50e-16 &      \\ 
     &           &    5 &  1.58e-08 &  1.77e-15 &      \\ 
     &           &    6 &  1.58e-08 &  4.52e-16 &      \\ 
     &           &    7 &  1.58e-08 &  4.52e-16 &      \\ 
     &           &    8 &  1.58e-08 &  4.51e-16 &      \\ 
     &           &    9 &  6.19e-08 &  4.50e-16 &      \\ 
     &           &   10 &  1.58e-08 &  1.77e-15 &      \\ 
4089 &  4.09e+04 &   10 &           &           & iters  \\ 
 \hdashline 
     &           &    1 &  9.84e-09 &  2.14e-12 &      \\ 
     &           &    2 &  9.82e-09 &  2.81e-16 &      \\ 
     &           &    3 &  6.79e-08 &  2.80e-16 &      \\ 
     &           &    4 &  9.91e-09 &  1.94e-15 &      \\ 
     &           &    5 &  9.89e-09 &  2.83e-16 &      \\ 
     &           &    6 &  9.88e-09 &  2.82e-16 &      \\ 
     &           &    7 &  9.86e-09 &  2.82e-16 &      \\ 
     &           &    8 &  9.85e-09 &  2.82e-16 &      \\ 
     &           &    9 &  9.84e-09 &  2.81e-16 &      \\ 
     &           &   10 &  6.79e-08 &  2.81e-16 &      \\ 
4090 &  4.09e+04 &   10 &           &           & iters  \\ 
 \hdashline 
     &           &    1 &  6.97e-08 &  6.72e-13 &      \\ 
     &           &    2 &  8.12e-09 &  1.99e-15 &      \\ 
     &           &    3 &  8.11e-09 &  2.32e-16 &      \\ 
     &           &    4 &  8.10e-09 &  2.31e-16 &      \\ 
     &           &    5 &  8.08e-09 &  2.31e-16 &      \\ 
     &           &    6 &  8.07e-09 &  2.31e-16 &      \\ 
     &           &    7 &  8.06e-09 &  2.30e-16 &      \\ 
     &           &    8 &  8.05e-09 &  2.30e-16 &      \\ 
     &           &    9 &  6.96e-08 &  2.30e-16 &      \\ 
     &           &   10 &  4.70e-08 &  1.99e-15 &      \\ 
4091 &  4.09e+04 &   10 &           &           & iters  \\ 
 \hdashline 
     &           &    1 &  4.97e-08 &  1.95e-11 &      \\ 
     &           &    2 &  2.81e-08 &  1.42e-15 &      \\ 
     &           &    3 &  4.97e-08 &  8.01e-16 &      \\ 
     &           &    4 &  2.81e-08 &  1.42e-15 &      \\ 
     &           &    5 &  2.81e-08 &  8.02e-16 &      \\ 
     &           &    6 &  4.97e-08 &  8.01e-16 &      \\ 
     &           &    7 &  2.81e-08 &  1.42e-15 &      \\ 
     &           &    8 &  2.81e-08 &  8.02e-16 &      \\ 
     &           &    9 &  4.97e-08 &  8.01e-16 &      \\ 
     &           &   10 &  2.81e-08 &  1.42e-15 &      \\ 
4092 &  4.09e+04 &   10 &           &           & iters  \\ 
 \hdashline 
     &           &    1 &  3.79e-08 &  3.92e-11 &      \\ 
     &           &    2 &  3.99e-08 &  1.08e-15 &      \\ 
     &           &    3 &  3.79e-08 &  1.14e-15 &      \\ 
     &           &    4 &  3.99e-08 &  1.08e-15 &      \\ 
     &           &    5 &  3.79e-08 &  1.14e-15 &      \\ 
     &           &    6 &  3.99e-08 &  1.08e-15 &      \\ 
     &           &    7 &  3.79e-08 &  1.14e-15 &      \\ 
     &           &    8 &  3.99e-08 &  1.08e-15 &      \\ 
     &           &    9 &  3.79e-08 &  1.14e-15 &      \\ 
     &           &   10 &  3.99e-08 &  1.08e-15 &      \\ 
4093 &  4.09e+04 &   10 &           &           & iters  \\ 
 \hdashline 
     &           &    1 &  3.73e-08 &  2.32e-11 &      \\ 
     &           &    2 &  4.05e-08 &  1.06e-15 &      \\ 
     &           &    3 &  3.73e-08 &  1.16e-15 &      \\ 
     &           &    4 &  4.05e-08 &  1.06e-15 &      \\ 
     &           &    5 &  3.73e-08 &  1.16e-15 &      \\ 
     &           &    6 &  4.05e-08 &  1.06e-15 &      \\ 
     &           &    7 &  3.73e-08 &  1.16e-15 &      \\ 
     &           &    8 &  4.05e-08 &  1.06e-15 &      \\ 
     &           &    9 &  3.73e-08 &  1.16e-15 &      \\ 
     &           &   10 &  4.05e-08 &  1.06e-15 &      \\ 
4094 &  4.09e+04 &   10 &           &           & iters  \\ 
 \hdashline 
     &           &    1 &  5.73e-09 &  1.62e-11 &      \\ 
     &           &    2 &  5.72e-09 &  1.64e-16 &      \\ 
     &           &    3 &  5.72e-09 &  1.63e-16 &      \\ 
     &           &    4 &  5.71e-09 &  1.63e-16 &      \\ 
     &           &    5 &  5.70e-09 &  1.63e-16 &      \\ 
     &           &    6 &  5.69e-09 &  1.63e-16 &      \\ 
     &           &    7 &  7.20e-08 &  1.62e-16 &      \\ 
     &           &    8 &  5.79e-09 &  2.06e-15 &      \\ 
     &           &    9 &  5.78e-09 &  1.65e-16 &      \\ 
     &           &   10 &  5.77e-09 &  1.65e-16 &      \\ 
4095 &  4.09e+04 &   10 &           &           & iters  \\ 
 \hdashline 
     &           &    1 &  4.72e-08 &  1.34e-11 &      \\ 
     &           &    2 &  3.06e-08 &  1.35e-15 &      \\ 
     &           &    3 &  3.05e-08 &  8.73e-16 &      \\ 
     &           &    4 &  4.72e-08 &  8.71e-16 &      \\ 
     &           &    5 &  3.06e-08 &  1.35e-15 &      \\ 
     &           &    6 &  4.72e-08 &  8.72e-16 &      \\ 
     &           &    7 &  3.06e-08 &  1.35e-15 &      \\ 
     &           &    8 &  3.05e-08 &  8.73e-16 &      \\ 
     &           &    9 &  4.72e-08 &  8.71e-16 &      \\ 
     &           &   10 &  3.06e-08 &  1.35e-15 &      \\ 
4096 &  4.10e+04 &   10 &           &           & iters  \\ 
 \hdashline 
     &           &    1 &  3.89e-08 &  1.56e-12 &      \\ 
     &           &    2 &  3.89e-08 &  1.11e-15 &      \\ 
     &           &    3 &  3.89e-08 &  1.11e-15 &      \\ 
     &           &    4 &  3.89e-08 &  1.11e-15 &      \\ 
     &           &    5 &  3.89e-08 &  1.11e-15 &      \\ 
     &           &    6 &  3.89e-08 &  1.11e-15 &      \\ 
     &           &    7 &  3.89e-08 &  1.11e-15 &      \\ 
     &           &    8 &  3.89e-08 &  1.11e-15 &      \\ 
     &           &    9 &  3.89e-08 &  1.11e-15 &      \\ 
     &           &   10 &  3.89e-08 &  1.11e-15 &      \\ 
4097 &  4.10e+04 &   10 &           &           & iters  \\ 
 \hdashline 
     &           &    1 &  3.32e-08 &  7.95e-12 &      \\ 
     &           &    2 &  4.45e-08 &  9.48e-16 &      \\ 
     &           &    3 &  3.32e-08 &  1.27e-15 &      \\ 
     &           &    4 &  4.45e-08 &  9.48e-16 &      \\ 
     &           &    5 &  3.32e-08 &  1.27e-15 &      \\ 
     &           &    6 &  3.32e-08 &  9.49e-16 &      \\ 
     &           &    7 &  4.45e-08 &  9.48e-16 &      \\ 
     &           &    8 &  3.32e-08 &  1.27e-15 &      \\ 
     &           &    9 &  4.45e-08 &  9.48e-16 &      \\ 
     &           &   10 &  3.32e-08 &  1.27e-15 &      \\ 
4098 &  4.10e+04 &   10 &           &           & iters  \\ 
 \hdashline 
     &           &    1 &  2.49e-08 &  1.00e-11 &      \\ 
     &           &    2 &  2.49e-08 &  7.11e-16 &      \\ 
     &           &    3 &  5.28e-08 &  7.10e-16 &      \\ 
     &           &    4 &  2.49e-08 &  1.51e-15 &      \\ 
     &           &    5 &  2.49e-08 &  7.11e-16 &      \\ 
     &           &    6 &  5.28e-08 &  7.10e-16 &      \\ 
     &           &    7 &  2.49e-08 &  1.51e-15 &      \\ 
     &           &    8 &  2.49e-08 &  7.11e-16 &      \\ 
     &           &    9 &  5.28e-08 &  7.10e-16 &      \\ 
     &           &   10 &  2.49e-08 &  1.51e-15 &      \\ 
4099 &  4.10e+04 &   10 &           &           & iters  \\ 
 \hdashline 
     &           &    1 &  1.51e-08 &  3.52e-12 &      \\ 
     &           &    2 &  1.51e-08 &  4.32e-16 &      \\ 
     &           &    3 &  1.51e-08 &  4.31e-16 &      \\ 
     &           &    4 &  1.51e-08 &  4.30e-16 &      \\ 
     &           &    5 &  6.26e-08 &  4.30e-16 &      \\ 
     &           &    6 &  1.51e-08 &  1.79e-15 &      \\ 
     &           &    7 &  1.51e-08 &  4.32e-16 &      \\ 
     &           &    8 &  1.51e-08 &  4.31e-16 &      \\ 
     &           &    9 &  1.51e-08 &  4.31e-16 &      \\ 
     &           &   10 &  6.26e-08 &  4.30e-16 &      \\ 
4100 &  4.10e+04 &   10 &           &           & iters  \\ 
 \hdashline 
     &           &    1 &  3.03e-08 &  2.61e-12 &      \\ 
     &           &    2 &  3.03e-08 &  8.65e-16 &      \\ 
     &           &    3 &  4.75e-08 &  8.64e-16 &      \\ 
     &           &    4 &  3.03e-08 &  1.35e-15 &      \\ 
     &           &    5 &  4.74e-08 &  8.65e-16 &      \\ 
     &           &    6 &  3.03e-08 &  1.35e-15 &      \\ 
     &           &    7 &  3.03e-08 &  8.65e-16 &      \\ 
     &           &    8 &  4.75e-08 &  8.64e-16 &      \\ 
     &           &    9 &  3.03e-08 &  1.35e-15 &      \\ 
     &           &   10 &  3.03e-08 &  8.65e-16 &      \\ 
4101 &  4.10e+04 &   10 &           &           & iters  \\ 
 \hdashline 
     &           &    1 &  9.71e-08 &  5.51e-12 &      \\ 
     &           &    2 &  5.84e-08 &  2.77e-15 &      \\ 
     &           &    3 &  1.94e-08 &  1.67e-15 &      \\ 
     &           &    4 &  1.94e-08 &  5.53e-16 &      \\ 
     &           &    5 &  5.84e-08 &  5.52e-16 &      \\ 
     &           &    6 &  1.94e-08 &  1.67e-15 &      \\ 
     &           &    7 &  1.94e-08 &  5.54e-16 &      \\ 
     &           &    8 &  1.94e-08 &  5.53e-16 &      \\ 
     &           &    9 &  5.84e-08 &  5.52e-16 &      \\ 
     &           &   10 &  1.94e-08 &  1.67e-15 &      \\ 
4102 &  4.10e+04 &   10 &           &           & iters  \\ 
 \hdashline 
     &           &    1 &  4.09e-09 &  7.86e-12 &      \\ 
     &           &    2 &  4.08e-09 &  1.17e-16 &      \\ 
     &           &    3 &  4.08e-09 &  1.17e-16 &      \\ 
     &           &    4 &  4.07e-09 &  1.16e-16 &      \\ 
     &           &    5 &  4.07e-09 &  1.16e-16 &      \\ 
     &           &    6 &  4.06e-09 &  1.16e-16 &      \\ 
     &           &    7 &  4.05e-09 &  1.16e-16 &      \\ 
     &           &    8 &  4.05e-09 &  1.16e-16 &      \\ 
     &           &    9 &  4.04e-09 &  1.16e-16 &      \\ 
     &           &   10 &  7.37e-08 &  1.15e-16 &      \\ 
4103 &  4.10e+04 &   10 &           &           & iters  \\ 
 \hdashline 
     &           &    1 &  8.42e-08 &  1.42e-12 &      \\ 
     &           &    2 &  6.42e-09 &  2.40e-15 &      \\ 
     &           &    3 &  6.42e-09 &  1.83e-16 &      \\ 
     &           &    4 &  6.41e-09 &  1.83e-16 &      \\ 
     &           &    5 &  7.13e-08 &  1.83e-16 &      \\ 
     &           &    6 &  8.42e-08 &  2.03e-15 &      \\ 
     &           &    7 &  7.13e-08 &  2.40e-15 &      \\ 
     &           &    8 &  6.48e-09 &  2.04e-15 &      \\ 
     &           &    9 &  6.47e-09 &  1.85e-16 &      \\ 
     &           &   10 &  6.46e-09 &  1.85e-16 &      \\ 
4104 &  4.10e+04 &   10 &           &           & iters  \\ 
 \hdashline 
     &           &    1 &  1.61e-08 &  6.28e-12 &      \\ 
     &           &    2 &  1.61e-08 &  4.59e-16 &      \\ 
     &           &    3 &  6.17e-08 &  4.58e-16 &      \\ 
     &           &    4 &  1.61e-08 &  1.76e-15 &      \\ 
     &           &    5 &  1.61e-08 &  4.60e-16 &      \\ 
     &           &    6 &  1.61e-08 &  4.59e-16 &      \\ 
     &           &    7 &  1.61e-08 &  4.59e-16 &      \\ 
     &           &    8 &  6.17e-08 &  4.58e-16 &      \\ 
     &           &    9 &  1.61e-08 &  1.76e-15 &      \\ 
     &           &   10 &  1.61e-08 &  4.60e-16 &      \\ 
4105 &  4.10e+04 &   10 &           &           & iters  \\ 
 \hdashline 
     &           &    1 &  3.40e-09 &  9.39e-12 &      \\ 
     &           &    2 &  3.40e-09 &  9.71e-17 &      \\ 
     &           &    3 &  3.39e-09 &  9.70e-17 &      \\ 
     &           &    4 &  3.39e-09 &  9.69e-17 &      \\ 
     &           &    5 &  3.38e-09 &  9.67e-17 &      \\ 
     &           &    6 &  3.38e-09 &  9.66e-17 &      \\ 
     &           &    7 &  3.37e-09 &  9.64e-17 &      \\ 
     &           &    8 &  3.37e-09 &  9.63e-17 &      \\ 
     &           &    9 &  3.36e-09 &  9.62e-17 &      \\ 
     &           &   10 &  3.36e-09 &  9.60e-17 &      \\ 
4106 &  4.10e+04 &   10 &           &           & iters  \\ 
 \hdashline 
     &           &    1 &  1.50e-08 &  7.76e-12 &      \\ 
     &           &    2 &  1.50e-08 &  4.29e-16 &      \\ 
     &           &    3 &  1.50e-08 &  4.28e-16 &      \\ 
     &           &    4 &  6.27e-08 &  4.27e-16 &      \\ 
     &           &    5 &  1.50e-08 &  1.79e-15 &      \\ 
     &           &    6 &  1.50e-08 &  4.29e-16 &      \\ 
     &           &    7 &  1.50e-08 &  4.29e-16 &      \\ 
     &           &    8 &  1.50e-08 &  4.28e-16 &      \\ 
     &           &    9 &  6.27e-08 &  4.28e-16 &      \\ 
     &           &   10 &  1.50e-08 &  1.79e-15 &      \\ 
4107 &  4.11e+04 &   10 &           &           & iters  \\ 
 \hdashline 
     &           &    1 &  6.35e-09 &  5.44e-12 &      \\ 
     &           &    2 &  6.34e-09 &  1.81e-16 &      \\ 
     &           &    3 &  6.33e-09 &  1.81e-16 &      \\ 
     &           &    4 &  6.32e-09 &  1.81e-16 &      \\ 
     &           &    5 &  6.31e-09 &  1.80e-16 &      \\ 
     &           &    6 &  6.31e-09 &  1.80e-16 &      \\ 
     &           &    7 &  6.30e-09 &  1.80e-16 &      \\ 
     &           &    8 &  6.29e-09 &  1.80e-16 &      \\ 
     &           &    9 &  6.28e-09 &  1.79e-16 &      \\ 
     &           &   10 &  6.27e-09 &  1.79e-16 &      \\ 
4108 &  4.11e+04 &   10 &           &           & iters  \\ 
 \hdashline 
     &           &    1 &  6.08e-08 &  5.91e-13 &      \\ 
     &           &    2 &  1.69e-08 &  1.74e-15 &      \\ 
     &           &    3 &  1.69e-08 &  4.83e-16 &      \\ 
     &           &    4 &  1.69e-08 &  4.83e-16 &      \\ 
     &           &    5 &  6.08e-08 &  4.82e-16 &      \\ 
     &           &    6 &  1.69e-08 &  1.74e-15 &      \\ 
     &           &    7 &  1.69e-08 &  4.84e-16 &      \\ 
     &           &    8 &  1.69e-08 &  4.83e-16 &      \\ 
     &           &    9 &  1.69e-08 &  4.82e-16 &      \\ 
     &           &   10 &  6.08e-08 &  4.82e-16 &      \\ 
4109 &  4.11e+04 &   10 &           &           & iters  \\ 
 \hdashline 
     &           &    1 &  4.57e-08 &  2.33e-12 &      \\ 
     &           &    2 &  3.21e-08 &  1.30e-15 &      \\ 
     &           &    3 &  4.57e-08 &  9.15e-16 &      \\ 
     &           &    4 &  3.21e-08 &  1.30e-15 &      \\ 
     &           &    5 &  3.20e-08 &  9.16e-16 &      \\ 
     &           &    6 &  4.57e-08 &  9.14e-16 &      \\ 
     &           &    7 &  3.21e-08 &  1.30e-15 &      \\ 
     &           &    8 &  4.57e-08 &  9.15e-16 &      \\ 
     &           &    9 &  3.21e-08 &  1.30e-15 &      \\ 
     &           &   10 &  4.57e-08 &  9.15e-16 &      \\ 
4110 &  4.11e+04 &   10 &           &           & iters  \\ 
 \hdashline 
     &           &    1 &  4.50e-08 &  3.63e-13 &      \\ 
     &           &    2 &  3.27e-08 &  1.29e-15 &      \\ 
     &           &    3 &  4.50e-08 &  9.34e-16 &      \\ 
     &           &    4 &  3.27e-08 &  1.28e-15 &      \\ 
     &           &    5 &  3.27e-08 &  9.34e-16 &      \\ 
     &           &    6 &  4.50e-08 &  9.33e-16 &      \\ 
     &           &    7 &  3.27e-08 &  1.29e-15 &      \\ 
     &           &    8 &  4.50e-08 &  9.34e-16 &      \\ 
     &           &    9 &  3.27e-08 &  1.29e-15 &      \\ 
     &           &   10 &  3.27e-08 &  9.34e-16 &      \\ 
4111 &  4.11e+04 &   10 &           &           & iters  \\ 
 \hdashline 
     &           &    1 &  9.97e-08 &  1.11e-12 &      \\ 
     &           &    2 &  5.58e-08 &  2.85e-15 &      \\ 
     &           &    3 &  2.19e-08 &  1.59e-15 &      \\ 
     &           &    4 &  2.19e-08 &  6.26e-16 &      \\ 
     &           &    5 &  5.58e-08 &  6.25e-16 &      \\ 
     &           &    6 &  2.20e-08 &  1.59e-15 &      \\ 
     &           &    7 &  2.19e-08 &  6.27e-16 &      \\ 
     &           &    8 &  5.58e-08 &  6.26e-16 &      \\ 
     &           &    9 &  2.20e-08 &  1.59e-15 &      \\ 
     &           &   10 &  2.19e-08 &  6.27e-16 &      \\ 
4112 &  4.11e+04 &   10 &           &           & iters  \\ 
 \hdashline 
     &           &    1 &  4.86e-08 &  6.67e-13 &      \\ 
     &           &    2 &  2.91e-08 &  1.39e-15 &      \\ 
     &           &    3 &  2.91e-08 &  8.31e-16 &      \\ 
     &           &    4 &  4.86e-08 &  8.30e-16 &      \\ 
     &           &    5 &  2.91e-08 &  1.39e-15 &      \\ 
     &           &    6 &  2.91e-08 &  8.31e-16 &      \\ 
     &           &    7 &  4.87e-08 &  8.30e-16 &      \\ 
     &           &    8 &  2.91e-08 &  1.39e-15 &      \\ 
     &           &    9 &  4.86e-08 &  8.30e-16 &      \\ 
     &           &   10 &  2.91e-08 &  1.39e-15 &      \\ 
4113 &  4.11e+04 &   10 &           &           & iters  \\ 
 \hdashline 
     &           &    1 &  1.49e-08 &  7.27e-12 &      \\ 
     &           &    2 &  6.28e-08 &  4.24e-16 &      \\ 
     &           &    3 &  1.49e-08 &  1.79e-15 &      \\ 
     &           &    4 &  1.49e-08 &  4.26e-16 &      \\ 
     &           &    5 &  1.49e-08 &  4.26e-16 &      \\ 
     &           &    6 &  1.49e-08 &  4.25e-16 &      \\ 
     &           &    7 &  6.28e-08 &  4.24e-16 &      \\ 
     &           &    8 &  1.49e-08 &  1.79e-15 &      \\ 
     &           &    9 &  1.49e-08 &  4.26e-16 &      \\ 
     &           &   10 &  1.49e-08 &  4.26e-16 &      \\ 
4114 &  4.11e+04 &   10 &           &           & iters  \\ 
 \hdashline 
     &           &    1 &  1.65e-08 &  9.27e-12 &      \\ 
     &           &    2 &  1.65e-08 &  4.71e-16 &      \\ 
     &           &    3 &  6.12e-08 &  4.70e-16 &      \\ 
     &           &    4 &  1.65e-08 &  1.75e-15 &      \\ 
     &           &    5 &  1.65e-08 &  4.72e-16 &      \\ 
     &           &    6 &  1.65e-08 &  4.71e-16 &      \\ 
     &           &    7 &  1.65e-08 &  4.70e-16 &      \\ 
     &           &    8 &  6.13e-08 &  4.70e-16 &      \\ 
     &           &    9 &  1.65e-08 &  1.75e-15 &      \\ 
     &           &   10 &  1.65e-08 &  4.72e-16 &      \\ 
4115 &  4.11e+04 &   10 &           &           & iters  \\ 
 \hdashline 
     &           &    1 &  2.00e-08 &  7.63e-12 &      \\ 
     &           &    2 &  1.99e-08 &  5.70e-16 &      \\ 
     &           &    3 &  5.78e-08 &  5.69e-16 &      \\ 
     &           &    4 &  2.00e-08 &  1.65e-15 &      \\ 
     &           &    5 &  2.00e-08 &  5.70e-16 &      \\ 
     &           &    6 &  1.99e-08 &  5.70e-16 &      \\ 
     &           &    7 &  5.78e-08 &  5.69e-16 &      \\ 
     &           &    8 &  2.00e-08 &  1.65e-15 &      \\ 
     &           &    9 &  2.00e-08 &  5.70e-16 &      \\ 
     &           &   10 &  1.99e-08 &  5.70e-16 &      \\ 
4116 &  4.12e+04 &   10 &           &           & iters  \\ 
 \hdashline 
     &           &    1 &  4.94e-08 &  4.79e-12 &      \\ 
     &           &    2 &  4.93e-08 &  1.41e-15 &      \\ 
     &           &    3 &  2.85e-08 &  1.41e-15 &      \\ 
     &           &    4 &  2.84e-08 &  8.13e-16 &      \\ 
     &           &    5 &  4.93e-08 &  8.12e-16 &      \\ 
     &           &    6 &  2.85e-08 &  1.41e-15 &      \\ 
     &           &    7 &  2.84e-08 &  8.12e-16 &      \\ 
     &           &    8 &  4.93e-08 &  8.11e-16 &      \\ 
     &           &    9 &  2.85e-08 &  1.41e-15 &      \\ 
     &           &   10 &  4.93e-08 &  8.12e-16 &      \\ 
4117 &  4.12e+04 &   10 &           &           & iters  \\ 
 \hdashline 
     &           &    1 &  7.43e-08 &  1.01e-12 &      \\ 
     &           &    2 &  3.54e-09 &  2.12e-15 &      \\ 
     &           &    3 &  3.53e-09 &  1.01e-16 &      \\ 
     &           &    4 &  3.53e-09 &  1.01e-16 &      \\ 
     &           &    5 &  3.52e-09 &  1.01e-16 &      \\ 
     &           &    6 &  3.52e-09 &  1.01e-16 &      \\ 
     &           &    7 &  3.51e-09 &  1.00e-16 &      \\ 
     &           &    8 &  3.51e-09 &  1.00e-16 &      \\ 
     &           &    9 &  3.50e-09 &  1.00e-16 &      \\ 
     &           &   10 &  3.50e-09 &  1.00e-16 &      \\ 
4118 &  4.12e+04 &   10 &           &           & iters  \\ 
 \hdashline 
     &           &    1 &  3.82e-09 &  1.70e-12 &      \\ 
     &           &    2 &  3.82e-09 &  1.09e-16 &      \\ 
     &           &    3 &  3.81e-09 &  1.09e-16 &      \\ 
     &           &    4 &  3.81e-09 &  1.09e-16 &      \\ 
     &           &    5 &  3.80e-09 &  1.09e-16 &      \\ 
     &           &    6 &  3.80e-09 &  1.08e-16 &      \\ 
     &           &    7 &  3.79e-09 &  1.08e-16 &      \\ 
     &           &    8 &  3.78e-09 &  1.08e-16 &      \\ 
     &           &    9 &  3.78e-09 &  1.08e-16 &      \\ 
     &           &   10 &  3.77e-09 &  1.08e-16 &      \\ 
4119 &  4.12e+04 &   10 &           &           & iters  \\ 
 \hdashline 
     &           &    1 &  5.93e-08 &  2.28e-12 &      \\ 
     &           &    2 &  1.85e-08 &  1.69e-15 &      \\ 
     &           &    3 &  5.92e-08 &  5.28e-16 &      \\ 
     &           &    4 &  1.85e-08 &  1.69e-15 &      \\ 
     &           &    5 &  1.85e-08 &  5.29e-16 &      \\ 
     &           &    6 &  1.85e-08 &  5.28e-16 &      \\ 
     &           &    7 &  1.85e-08 &  5.28e-16 &      \\ 
     &           &    8 &  5.93e-08 &  5.27e-16 &      \\ 
     &           &    9 &  1.85e-08 &  1.69e-15 &      \\ 
     &           &   10 &  1.85e-08 &  5.29e-16 &      \\ 
4120 &  4.12e+04 &   10 &           &           & iters  \\ 
 \hdashline 
     &           &    1 &  1.58e-08 &  3.49e-12 &      \\ 
     &           &    2 &  1.58e-08 &  4.51e-16 &      \\ 
     &           &    3 &  6.19e-08 &  4.50e-16 &      \\ 
     &           &    4 &  1.58e-08 &  1.77e-15 &      \\ 
     &           &    5 &  1.58e-08 &  4.52e-16 &      \\ 
     &           &    6 &  1.58e-08 &  4.51e-16 &      \\ 
     &           &    7 &  1.58e-08 &  4.50e-16 &      \\ 
     &           &    8 &  6.19e-08 &  4.50e-16 &      \\ 
     &           &    9 &  1.58e-08 &  1.77e-15 &      \\ 
     &           &   10 &  1.58e-08 &  4.52e-16 &      \\ 
4121 &  4.12e+04 &   10 &           &           & iters  \\ 
 \hdashline 
     &           &    1 &  1.14e-08 &  7.29e-13 &      \\ 
     &           &    2 &  1.14e-08 &  3.25e-16 &      \\ 
     &           &    3 &  6.63e-08 &  3.24e-16 &      \\ 
     &           &    4 &  8.91e-08 &  1.89e-15 &      \\ 
     &           &    5 &  6.64e-08 &  2.54e-15 &      \\ 
     &           &    6 &  1.14e-08 &  1.89e-15 &      \\ 
     &           &    7 &  1.14e-08 &  3.26e-16 &      \\ 
     &           &    8 &  1.14e-08 &  3.25e-16 &      \\ 
     &           &    9 &  1.14e-08 &  3.25e-16 &      \\ 
     &           &   10 &  6.63e-08 &  3.24e-16 &      \\ 
4122 &  4.12e+04 &   10 &           &           & iters  \\ 
 \hdashline 
     &           &    1 &  8.64e-08 &  6.41e-13 &      \\ 
     &           &    2 &  6.91e-08 &  2.47e-15 &      \\ 
     &           &    3 &  8.71e-09 &  1.97e-15 &      \\ 
     &           &    4 &  8.70e-09 &  2.49e-16 &      \\ 
     &           &    5 &  8.69e-09 &  2.48e-16 &      \\ 
     &           &    6 &  8.68e-09 &  2.48e-16 &      \\ 
     &           &    7 &  8.66e-09 &  2.48e-16 &      \\ 
     &           &    8 &  8.65e-09 &  2.47e-16 &      \\ 
     &           &    9 &  6.91e-08 &  2.47e-16 &      \\ 
     &           &   10 &  8.64e-08 &  1.97e-15 &      \\ 
4123 &  4.12e+04 &   10 &           &           & iters  \\ 
 \hdashline 
     &           &    1 &  7.22e-08 &  1.34e-12 &      \\ 
     &           &    2 &  5.61e-09 &  2.06e-15 &      \\ 
     &           &    3 &  5.60e-09 &  1.60e-16 &      \\ 
     &           &    4 &  5.60e-09 &  1.60e-16 &      \\ 
     &           &    5 &  5.59e-09 &  1.60e-16 &      \\ 
     &           &    6 &  5.58e-09 &  1.59e-16 &      \\ 
     &           &    7 &  5.57e-09 &  1.59e-16 &      \\ 
     &           &    8 &  5.56e-09 &  1.59e-16 &      \\ 
     &           &    9 &  5.56e-09 &  1.59e-16 &      \\ 
     &           &   10 &  7.21e-08 &  1.59e-16 &      \\ 
4124 &  4.12e+04 &   10 &           &           & iters  \\ 
 \hdashline 
     &           &    1 &  3.38e-08 &  4.57e-14 &      \\ 
     &           &    2 &  3.38e-08 &  9.66e-16 &      \\ 
     &           &    3 &  4.39e-08 &  9.65e-16 &      \\ 
     &           &    4 &  3.38e-08 &  1.25e-15 &      \\ 
     &           &    5 &  4.39e-08 &  9.65e-16 &      \\ 
     &           &    6 &  3.38e-08 &  1.25e-15 &      \\ 
     &           &    7 &  4.39e-08 &  9.65e-16 &      \\ 
     &           &    8 &  3.38e-08 &  1.25e-15 &      \\ 
     &           &    9 &  3.38e-08 &  9.66e-16 &      \\ 
     &           &   10 &  4.39e-08 &  9.64e-16 &      \\ 
4125 &  4.12e+04 &   10 &           &           & iters  \\ 
 \hdashline 
     &           &    1 &  8.30e-09 &  1.24e-12 &      \\ 
     &           &    2 &  8.29e-09 &  2.37e-16 &      \\ 
     &           &    3 &  8.28e-09 &  2.37e-16 &      \\ 
     &           &    4 &  6.94e-08 &  2.36e-16 &      \\ 
     &           &    5 &  8.37e-09 &  1.98e-15 &      \\ 
     &           &    6 &  8.35e-09 &  2.39e-16 &      \\ 
     &           &    7 &  8.34e-09 &  2.38e-16 &      \\ 
     &           &    8 &  8.33e-09 &  2.38e-16 &      \\ 
     &           &    9 &  8.32e-09 &  2.38e-16 &      \\ 
     &           &   10 &  8.31e-09 &  2.37e-16 &      \\ 
4126 &  4.12e+04 &   10 &           &           & iters  \\ 
 \hdashline 
     &           &    1 &  5.09e-08 &  2.92e-12 &      \\ 
     &           &    2 &  2.68e-08 &  1.45e-15 &      \\ 
     &           &    3 &  2.68e-08 &  7.65e-16 &      \\ 
     &           &    4 &  5.10e-08 &  7.64e-16 &      \\ 
     &           &    5 &  2.68e-08 &  1.45e-15 &      \\ 
     &           &    6 &  2.68e-08 &  7.65e-16 &      \\ 
     &           &    7 &  5.10e-08 &  7.64e-16 &      \\ 
     &           &    8 &  2.68e-08 &  1.45e-15 &      \\ 
     &           &    9 &  2.68e-08 &  7.65e-16 &      \\ 
     &           &   10 &  5.10e-08 &  7.64e-16 &      \\ 
4127 &  4.13e+04 &   10 &           &           & iters  \\ 
 \hdashline 
     &           &    1 &  3.46e-08 &  1.35e-12 &      \\ 
     &           &    2 &  4.31e-08 &  9.87e-16 &      \\ 
     &           &    3 &  3.46e-08 &  1.23e-15 &      \\ 
     &           &    4 &  4.31e-08 &  9.88e-16 &      \\ 
     &           &    5 &  3.46e-08 &  1.23e-15 &      \\ 
     &           &    6 &  3.46e-08 &  9.88e-16 &      \\ 
     &           &    7 &  4.32e-08 &  9.87e-16 &      \\ 
     &           &    8 &  3.46e-08 &  1.23e-15 &      \\ 
     &           &    9 &  4.32e-08 &  9.87e-16 &      \\ 
     &           &   10 &  3.46e-08 &  1.23e-15 &      \\ 
4128 &  4.13e+04 &   10 &           &           & iters  \\ 
 \hdashline 
     &           &    1 &  7.30e-08 &  3.09e-13 &      \\ 
     &           &    2 &  8.25e-08 &  2.08e-15 &      \\ 
     &           &    3 &  7.30e-08 &  2.35e-15 &      \\ 
     &           &    4 &  8.25e-08 &  2.08e-15 &      \\ 
     &           &    5 &  7.30e-08 &  2.35e-15 &      \\ 
     &           &    6 &  4.75e-09 &  2.08e-15 &      \\ 
     &           &    7 &  4.75e-09 &  1.36e-16 &      \\ 
     &           &    8 &  4.74e-09 &  1.35e-16 &      \\ 
     &           &    9 &  4.73e-09 &  1.35e-16 &      \\ 
     &           &   10 &  4.73e-09 &  1.35e-16 &      \\ 
4129 &  4.13e+04 &   10 &           &           & iters  \\ 
 \hdashline 
     &           &    1 &  5.85e-09 &  9.88e-13 &      \\ 
     &           &    2 &  5.84e-09 &  1.67e-16 &      \\ 
     &           &    3 &  5.83e-09 &  1.67e-16 &      \\ 
     &           &    4 &  7.19e-08 &  1.67e-16 &      \\ 
     &           &    5 &  8.36e-08 &  2.05e-15 &      \\ 
     &           &    6 &  7.19e-08 &  2.39e-15 &      \\ 
     &           &    7 &  5.91e-09 &  2.05e-15 &      \\ 
     &           &    8 &  5.90e-09 &  1.69e-16 &      \\ 
     &           &    9 &  5.90e-09 &  1.69e-16 &      \\ 
     &           &   10 &  5.89e-09 &  1.68e-16 &      \\ 
4130 &  4.13e+04 &   10 &           &           & iters  \\ 
 \hdashline 
     &           &    1 &  6.28e-08 &  8.31e-13 &      \\ 
     &           &    2 &  1.50e-08 &  1.79e-15 &      \\ 
     &           &    3 &  1.49e-08 &  4.27e-16 &      \\ 
     &           &    4 &  1.49e-08 &  4.26e-16 &      \\ 
     &           &    5 &  6.28e-08 &  4.26e-16 &      \\ 
     &           &    6 &  1.50e-08 &  1.79e-15 &      \\ 
     &           &    7 &  1.50e-08 &  4.28e-16 &      \\ 
     &           &    8 &  1.49e-08 &  4.27e-16 &      \\ 
     &           &    9 &  1.49e-08 &  4.26e-16 &      \\ 
     &           &   10 &  6.28e-08 &  4.26e-16 &      \\ 
4131 &  4.13e+04 &   10 &           &           & iters  \\ 
 \hdashline 
     &           &    1 &  9.24e-08 &  4.25e-13 &      \\ 
     &           &    2 &  6.31e-08 &  2.64e-15 &      \\ 
     &           &    3 &  1.46e-08 &  1.80e-15 &      \\ 
     &           &    4 &  1.46e-08 &  4.18e-16 &      \\ 
     &           &    5 &  1.46e-08 &  4.17e-16 &      \\ 
     &           &    6 &  6.31e-08 &  4.17e-16 &      \\ 
     &           &    7 &  1.47e-08 &  1.80e-15 &      \\ 
     &           &    8 &  1.47e-08 &  4.19e-16 &      \\ 
     &           &    9 &  1.46e-08 &  4.18e-16 &      \\ 
     &           &   10 &  1.46e-08 &  4.18e-16 &      \\ 
4132 &  4.13e+04 &   10 &           &           & iters  \\ 
 \hdashline 
     &           &    1 &  2.65e-08 &  1.03e-12 &      \\ 
     &           &    2 &  2.65e-08 &  7.57e-16 &      \\ 
     &           &    3 &  5.12e-08 &  7.56e-16 &      \\ 
     &           &    4 &  2.65e-08 &  1.46e-15 &      \\ 
     &           &    5 &  2.65e-08 &  7.57e-16 &      \\ 
     &           &    6 &  5.13e-08 &  7.56e-16 &      \\ 
     &           &    7 &  2.65e-08 &  1.46e-15 &      \\ 
     &           &    8 &  5.12e-08 &  7.57e-16 &      \\ 
     &           &    9 &  2.65e-08 &  1.46e-15 &      \\ 
     &           &   10 &  2.65e-08 &  7.58e-16 &      \\ 
4133 &  4.13e+04 &   10 &           &           & iters  \\ 
 \hdashline 
     &           &    1 &  2.17e-08 &  6.32e-13 &      \\ 
     &           &    2 &  2.16e-08 &  6.18e-16 &      \\ 
     &           &    3 &  5.61e-08 &  6.17e-16 &      \\ 
     &           &    4 &  2.17e-08 &  1.60e-15 &      \\ 
     &           &    5 &  2.17e-08 &  6.19e-16 &      \\ 
     &           &    6 &  2.16e-08 &  6.18e-16 &      \\ 
     &           &    7 &  5.61e-08 &  6.17e-16 &      \\ 
     &           &    8 &  2.17e-08 &  1.60e-15 &      \\ 
     &           &    9 &  2.16e-08 &  6.18e-16 &      \\ 
     &           &   10 &  5.61e-08 &  6.18e-16 &      \\ 
4134 &  4.13e+04 &   10 &           &           & iters  \\ 
 \hdashline 
     &           &    1 &  2.11e-08 &  1.62e-12 &      \\ 
     &           &    2 &  2.10e-08 &  6.01e-16 &      \\ 
     &           &    3 &  2.10e-08 &  6.00e-16 &      \\ 
     &           &    4 &  5.67e-08 &  5.99e-16 &      \\ 
     &           &    5 &  2.11e-08 &  1.62e-15 &      \\ 
     &           &    6 &  2.10e-08 &  6.01e-16 &      \\ 
     &           &    7 &  2.10e-08 &  6.00e-16 &      \\ 
     &           &    8 &  5.67e-08 &  5.99e-16 &      \\ 
     &           &    9 &  2.10e-08 &  1.62e-15 &      \\ 
     &           &   10 &  2.10e-08 &  6.01e-16 &      \\ 
4135 &  4.13e+04 &   10 &           &           & iters  \\ 
 \hdashline 
     &           &    1 &  4.31e-08 &  3.96e-13 &      \\ 
     &           &    2 &  3.46e-08 &  1.23e-15 &      \\ 
     &           &    3 &  4.31e-08 &  9.89e-16 &      \\ 
     &           &    4 &  3.46e-08 &  1.23e-15 &      \\ 
     &           &    5 &  4.31e-08 &  9.89e-16 &      \\ 
     &           &    6 &  3.47e-08 &  1.23e-15 &      \\ 
     &           &    7 &  3.46e-08 &  9.89e-16 &      \\ 
     &           &    8 &  4.31e-08 &  9.88e-16 &      \\ 
     &           &    9 &  3.46e-08 &  1.23e-15 &      \\ 
     &           &   10 &  4.31e-08 &  9.88e-16 &      \\ 
4136 &  4.14e+04 &   10 &           &           & iters  \\ 
 \hdashline 
     &           &    1 &  2.37e-08 &  1.40e-12 &      \\ 
     &           &    2 &  2.37e-08 &  6.78e-16 &      \\ 
     &           &    3 &  2.37e-08 &  6.77e-16 &      \\ 
     &           &    4 &  2.36e-08 &  6.76e-16 &      \\ 
     &           &    5 &  5.41e-08 &  6.75e-16 &      \\ 
     &           &    6 &  2.37e-08 &  1.54e-15 &      \\ 
     &           &    7 &  2.37e-08 &  6.76e-16 &      \\ 
     &           &    8 &  5.41e-08 &  6.75e-16 &      \\ 
     &           &    9 &  2.37e-08 &  1.54e-15 &      \\ 
     &           &   10 &  2.37e-08 &  6.76e-16 &      \\ 
4137 &  4.14e+04 &   10 &           &           & iters  \\ 
 \hdashline 
     &           &    1 &  2.93e-08 &  3.06e-13 &      \\ 
     &           &    2 &  2.93e-08 &  8.38e-16 &      \\ 
     &           &    3 &  4.84e-08 &  8.36e-16 &      \\ 
     &           &    4 &  2.93e-08 &  1.38e-15 &      \\ 
     &           &    5 &  2.93e-08 &  8.37e-16 &      \\ 
     &           &    6 &  4.84e-08 &  8.36e-16 &      \\ 
     &           &    7 &  2.93e-08 &  1.38e-15 &      \\ 
     &           &    8 &  4.84e-08 &  8.37e-16 &      \\ 
     &           &    9 &  2.93e-08 &  1.38e-15 &      \\ 
     &           &   10 &  2.93e-08 &  8.37e-16 &      \\ 
4138 &  4.14e+04 &   10 &           &           & iters  \\ 
 \hdashline 
     &           &    1 &  3.36e-08 &  1.39e-12 &      \\ 
     &           &    2 &  4.41e-08 &  9.60e-16 &      \\ 
     &           &    3 &  3.37e-08 &  1.26e-15 &      \\ 
     &           &    4 &  3.36e-08 &  9.61e-16 &      \\ 
     &           &    5 &  4.41e-08 &  9.59e-16 &      \\ 
     &           &    6 &  3.36e-08 &  1.26e-15 &      \\ 
     &           &    7 &  4.41e-08 &  9.60e-16 &      \\ 
     &           &    8 &  3.36e-08 &  1.26e-15 &      \\ 
     &           &    9 &  4.41e-08 &  9.60e-16 &      \\ 
     &           &   10 &  3.37e-08 &  1.26e-15 &      \\ 
4139 &  4.14e+04 &   10 &           &           & iters  \\ 
 \hdashline 
     &           &    1 &  2.88e-09 &  2.91e-13 &      \\ 
     &           &    2 &  2.88e-09 &  8.23e-17 &      \\ 
     &           &    3 &  2.88e-09 &  8.22e-17 &      \\ 
     &           &    4 &  2.87e-09 &  8.21e-17 &      \\ 
     &           &    5 &  2.87e-09 &  8.20e-17 &      \\ 
     &           &    6 &  2.86e-09 &  8.19e-17 &      \\ 
     &           &    7 &  2.86e-09 &  8.17e-17 &      \\ 
     &           &    8 &  2.86e-09 &  8.16e-17 &      \\ 
     &           &    9 &  2.85e-09 &  8.15e-17 &      \\ 
     &           &   10 &  2.85e-09 &  8.14e-17 &      \\ 
4140 &  4.14e+04 &   10 &           &           & iters  \\ 
 \hdashline 
     &           &    1 &  2.27e-08 &  1.43e-12 &      \\ 
     &           &    2 &  2.27e-08 &  6.48e-16 &      \\ 
     &           &    3 &  5.51e-08 &  6.47e-16 &      \\ 
     &           &    4 &  2.27e-08 &  1.57e-15 &      \\ 
     &           &    5 &  2.27e-08 &  6.48e-16 &      \\ 
     &           &    6 &  2.27e-08 &  6.47e-16 &      \\ 
     &           &    7 &  5.51e-08 &  6.46e-16 &      \\ 
     &           &    8 &  2.27e-08 &  1.57e-15 &      \\ 
     &           &    9 &  2.27e-08 &  6.48e-16 &      \\ 
     &           &   10 &  5.51e-08 &  6.47e-16 &      \\ 
4141 &  4.14e+04 &   10 &           &           & iters  \\ 
 \hdashline 
     &           &    1 &  2.77e-08 &  4.43e-13 &      \\ 
     &           &    2 &  2.77e-08 &  7.91e-16 &      \\ 
     &           &    3 &  5.00e-08 &  7.90e-16 &      \\ 
     &           &    4 &  2.77e-08 &  1.43e-15 &      \\ 
     &           &    5 &  5.00e-08 &  7.91e-16 &      \\ 
     &           &    6 &  2.77e-08 &  1.43e-15 &      \\ 
     &           &    7 &  2.77e-08 &  7.92e-16 &      \\ 
     &           &    8 &  5.00e-08 &  7.91e-16 &      \\ 
     &           &    9 &  2.77e-08 &  1.43e-15 &      \\ 
     &           &   10 &  2.77e-08 &  7.92e-16 &      \\ 
4142 &  4.14e+04 &   10 &           &           & iters  \\ 
 \hdashline 
     &           &    1 &  2.45e-08 &  1.79e-12 &      \\ 
     &           &    2 &  2.45e-08 &  7.01e-16 &      \\ 
     &           &    3 &  5.32e-08 &  7.00e-16 &      \\ 
     &           &    4 &  2.46e-08 &  1.52e-15 &      \\ 
     &           &    5 &  2.45e-08 &  7.01e-16 &      \\ 
     &           &    6 &  5.32e-08 &  7.00e-16 &      \\ 
     &           &    7 &  2.46e-08 &  1.52e-15 &      \\ 
     &           &    8 &  2.45e-08 &  7.01e-16 &      \\ 
     &           &    9 &  5.32e-08 &  7.00e-16 &      \\ 
     &           &   10 &  2.46e-08 &  1.52e-15 &      \\ 
4143 &  4.14e+04 &   10 &           &           & iters  \\ 
 \hdashline 
     &           &    1 &  1.80e-08 &  2.48e-13 &      \\ 
     &           &    2 &  5.97e-08 &  5.13e-16 &      \\ 
     &           &    3 &  1.80e-08 &  1.70e-15 &      \\ 
     &           &    4 &  1.80e-08 &  5.15e-16 &      \\ 
     &           &    5 &  1.80e-08 &  5.14e-16 &      \\ 
     &           &    6 &  5.97e-08 &  5.13e-16 &      \\ 
     &           &    7 &  1.80e-08 &  1.70e-15 &      \\ 
     &           &    8 &  1.80e-08 &  5.15e-16 &      \\ 
     &           &    9 &  1.80e-08 &  5.14e-16 &      \\ 
     &           &   10 &  5.97e-08 &  5.14e-16 &      \\ 
4144 &  4.14e+04 &   10 &           &           & iters  \\ 
 \hdashline 
     &           &    1 &  4.97e-08 &  1.20e-12 &      \\ 
     &           &    2 &  4.96e-08 &  1.42e-15 &      \\ 
     &           &    3 &  2.81e-08 &  1.42e-15 &      \\ 
     &           &    4 &  4.96e-08 &  8.02e-16 &      \\ 
     &           &    5 &  2.81e-08 &  1.42e-15 &      \\ 
     &           &    6 &  2.81e-08 &  8.03e-16 &      \\ 
     &           &    7 &  4.96e-08 &  8.02e-16 &      \\ 
     &           &    8 &  2.81e-08 &  1.42e-15 &      \\ 
     &           &    9 &  2.81e-08 &  8.03e-16 &      \\ 
     &           &   10 &  4.96e-08 &  8.02e-16 &      \\ 
4145 &  4.14e+04 &   10 &           &           & iters  \\ 
 \hdashline 
     &           &    1 &  8.48e-08 &  9.53e-14 &      \\ 
     &           &    2 &  6.94e-09 &  2.42e-15 &      \\ 
     &           &    3 &  7.08e-08 &  1.98e-16 &      \\ 
     &           &    4 &  8.47e-08 &  2.02e-15 &      \\ 
     &           &    5 &  7.08e-08 &  2.42e-15 &      \\ 
     &           &    6 &  7.01e-09 &  2.02e-15 &      \\ 
     &           &    7 &  7.00e-09 &  2.00e-16 &      \\ 
     &           &    8 &  6.99e-09 &  2.00e-16 &      \\ 
     &           &    9 &  6.98e-09 &  2.00e-16 &      \\ 
     &           &   10 &  6.97e-09 &  1.99e-16 &      \\ 
4146 &  4.14e+04 &   10 &           &           & iters  \\ 
 \hdashline 
     &           &    1 &  1.25e-09 &  1.44e-12 &      \\ 
     &           &    2 &  1.25e-09 &  3.56e-17 &      \\ 
     &           &    3 &  1.24e-09 &  3.56e-17 &      \\ 
     &           &    4 &  1.24e-09 &  3.55e-17 &      \\ 
     &           &    5 &  1.24e-09 &  3.55e-17 &      \\ 
     &           &    6 &  1.24e-09 &  3.54e-17 &      \\ 
     &           &    7 &  1.24e-09 &  3.54e-17 &      \\ 
     &           &    8 &  1.24e-09 &  3.53e-17 &      \\ 
     &           &    9 &  1.23e-09 &  3.53e-17 &      \\ 
     &           &   10 &  1.23e-09 &  3.52e-17 &      \\ 
4147 &  4.15e+04 &   10 &           &           & iters  \\ 
 \hdashline 
     &           &    1 &  2.07e-08 &  2.28e-14 &      \\ 
     &           &    2 &  2.07e-08 &  5.92e-16 &      \\ 
     &           &    3 &  2.07e-08 &  5.91e-16 &      \\ 
     &           &    4 &  5.70e-08 &  5.90e-16 &      \\ 
     &           &    5 &  2.07e-08 &  1.63e-15 &      \\ 
     &           &    6 &  2.07e-08 &  5.92e-16 &      \\ 
     &           &    7 &  2.07e-08 &  5.91e-16 &      \\ 
     &           &    8 &  5.71e-08 &  5.90e-16 &      \\ 
     &           &    9 &  2.07e-08 &  1.63e-15 &      \\ 
     &           &   10 &  2.07e-08 &  5.91e-16 &      \\ 
4148 &  4.15e+04 &   10 &           &           & iters  \\ 
 \hdashline 
     &           &    1 &  2.23e-08 &  1.26e-12 &      \\ 
     &           &    2 &  2.22e-08 &  6.36e-16 &      \\ 
     &           &    3 &  2.22e-08 &  6.35e-16 &      \\ 
     &           &    4 &  5.55e-08 &  6.34e-16 &      \\ 
     &           &    5 &  2.23e-08 &  1.58e-15 &      \\ 
     &           &    6 &  2.22e-08 &  6.35e-16 &      \\ 
     &           &    7 &  5.55e-08 &  6.34e-16 &      \\ 
     &           &    8 &  2.23e-08 &  1.58e-15 &      \\ 
     &           &    9 &  2.22e-08 &  6.36e-16 &      \\ 
     &           &   10 &  2.22e-08 &  6.35e-16 &      \\ 
4149 &  4.15e+04 &   10 &           &           & iters  \\ 
 \hdashline 
     &           &    1 &  5.58e-08 &  2.21e-13 &      \\ 
     &           &    2 &  2.19e-08 &  1.59e-15 &      \\ 
     &           &    3 &  2.19e-08 &  6.26e-16 &      \\ 
     &           &    4 &  5.58e-08 &  6.25e-16 &      \\ 
     &           &    5 &  2.19e-08 &  1.59e-15 &      \\ 
     &           &    6 &  2.19e-08 &  6.26e-16 &      \\ 
     &           &    7 &  2.19e-08 &  6.25e-16 &      \\ 
     &           &    8 &  5.58e-08 &  6.24e-16 &      \\ 
     &           &    9 &  2.19e-08 &  1.59e-15 &      \\ 
     &           &   10 &  2.19e-08 &  6.26e-16 &      \\ 
4150 &  4.15e+04 &   10 &           &           & iters  \\ 
 \hdashline 
     &           &    1 &  9.04e-09 &  1.21e-12 &      \\ 
     &           &    2 &  9.03e-09 &  2.58e-16 &      \\ 
     &           &    3 &  6.87e-08 &  2.58e-16 &      \\ 
     &           &    4 &  9.12e-09 &  1.96e-15 &      \\ 
     &           &    5 &  9.10e-09 &  2.60e-16 &      \\ 
     &           &    6 &  9.09e-09 &  2.60e-16 &      \\ 
     &           &    7 &  9.08e-09 &  2.59e-16 &      \\ 
     &           &    8 &  9.06e-09 &  2.59e-16 &      \\ 
     &           &    9 &  9.05e-09 &  2.59e-16 &      \\ 
     &           &   10 &  9.04e-09 &  2.58e-16 &      \\ 
4151 &  4.15e+04 &   10 &           &           & iters  \\ 
 \hdashline 
     &           &    1 &  4.05e-08 &  3.08e-13 &      \\ 
     &           &    2 &  4.05e-08 &  1.16e-15 &      \\ 
     &           &    3 &  3.73e-08 &  1.16e-15 &      \\ 
     &           &    4 &  4.05e-08 &  1.06e-15 &      \\ 
     &           &    5 &  3.73e-08 &  1.16e-15 &      \\ 
     &           &    6 &  4.05e-08 &  1.06e-15 &      \\ 
     &           &    7 &  3.73e-08 &  1.15e-15 &      \\ 
     &           &    8 &  3.72e-08 &  1.06e-15 &      \\ 
     &           &    9 &  4.05e-08 &  1.06e-15 &      \\ 
     &           &   10 &  3.72e-08 &  1.16e-15 &      \\ 
4152 &  4.15e+04 &   10 &           &           & iters  \\ 
 \hdashline 
     &           &    1 &  3.11e-08 &  1.20e-12 &      \\ 
     &           &    2 &  4.66e-08 &  8.88e-16 &      \\ 
     &           &    3 &  3.11e-08 &  1.33e-15 &      \\ 
     &           &    4 &  4.66e-08 &  8.89e-16 &      \\ 
     &           &    5 &  3.12e-08 &  1.33e-15 &      \\ 
     &           &    6 &  3.11e-08 &  8.89e-16 &      \\ 
     &           &    7 &  4.66e-08 &  8.88e-16 &      \\ 
     &           &    8 &  3.11e-08 &  1.33e-15 &      \\ 
     &           &    9 &  4.66e-08 &  8.89e-16 &      \\ 
     &           &   10 &  3.12e-08 &  1.33e-15 &      \\ 
4153 &  4.15e+04 &   10 &           &           & iters  \\ 
 \hdashline 
     &           &    1 &  2.87e-08 &  3.85e-13 &      \\ 
     &           &    2 &  2.87e-08 &  8.20e-16 &      \\ 
     &           &    3 &  4.91e-08 &  8.18e-16 &      \\ 
     &           &    4 &  2.87e-08 &  1.40e-15 &      \\ 
     &           &    5 &  2.87e-08 &  8.19e-16 &      \\ 
     &           &    6 &  4.91e-08 &  8.18e-16 &      \\ 
     &           &    7 &  2.87e-08 &  1.40e-15 &      \\ 
     &           &    8 &  4.90e-08 &  8.19e-16 &      \\ 
     &           &    9 &  2.87e-08 &  1.40e-15 &      \\ 
     &           &   10 &  2.87e-08 &  8.20e-16 &      \\ 
4154 &  4.15e+04 &   10 &           &           & iters  \\ 
 \hdashline 
     &           &    1 &  7.14e-08 &  1.14e-12 &      \\ 
     &           &    2 &  6.41e-09 &  2.04e-15 &      \\ 
     &           &    3 &  6.40e-09 &  1.83e-16 &      \\ 
     &           &    4 &  6.39e-09 &  1.83e-16 &      \\ 
     &           &    5 &  6.38e-09 &  1.82e-16 &      \\ 
     &           &    6 &  6.37e-09 &  1.82e-16 &      \\ 
     &           &    7 &  6.36e-09 &  1.82e-16 &      \\ 
     &           &    8 &  6.35e-09 &  1.82e-16 &      \\ 
     &           &    9 &  6.34e-09 &  1.81e-16 &      \\ 
     &           &   10 &  6.33e-09 &  1.81e-16 &      \\ 
4155 &  4.15e+04 &   10 &           &           & iters  \\ 
 \hdashline 
     &           &    1 &  2.18e-08 &  4.85e-13 &      \\ 
     &           &    2 &  5.60e-08 &  6.21e-16 &      \\ 
     &           &    3 &  2.18e-08 &  1.60e-15 &      \\ 
     &           &    4 &  2.18e-08 &  6.23e-16 &      \\ 
     &           &    5 &  2.17e-08 &  6.22e-16 &      \\ 
     &           &    6 &  5.60e-08 &  6.21e-16 &      \\ 
     &           &    7 &  2.18e-08 &  1.60e-15 &      \\ 
     &           &    8 &  2.18e-08 &  6.22e-16 &      \\ 
     &           &    9 &  5.60e-08 &  6.21e-16 &      \\ 
     &           &   10 &  2.18e-08 &  1.60e-15 &      \\ 
4156 &  4.16e+04 &   10 &           &           & iters  \\ 
 \hdashline 
     &           &    1 &  2.67e-08 &  1.09e-12 &      \\ 
     &           &    2 &  5.11e-08 &  7.61e-16 &      \\ 
     &           &    3 &  2.67e-08 &  1.46e-15 &      \\ 
     &           &    4 &  2.66e-08 &  7.62e-16 &      \\ 
     &           &    5 &  5.11e-08 &  7.61e-16 &      \\ 
     &           &    6 &  2.67e-08 &  1.46e-15 &      \\ 
     &           &    7 &  5.10e-08 &  7.62e-16 &      \\ 
     &           &    8 &  2.67e-08 &  1.46e-15 &      \\ 
     &           &    9 &  2.67e-08 &  7.63e-16 &      \\ 
     &           &   10 &  5.10e-08 &  7.61e-16 &      \\ 
4157 &  4.16e+04 &   10 &           &           & iters  \\ 
 \hdashline 
     &           &    1 &  5.93e-08 &  5.98e-13 &      \\ 
     &           &    2 &  1.85e-08 &  1.69e-15 &      \\ 
     &           &    3 &  1.84e-08 &  5.27e-16 &      \\ 
     &           &    4 &  5.93e-08 &  5.27e-16 &      \\ 
     &           &    5 &  1.85e-08 &  1.69e-15 &      \\ 
     &           &    6 &  1.85e-08 &  5.28e-16 &      \\ 
     &           &    7 &  1.85e-08 &  5.27e-16 &      \\ 
     &           &    8 &  5.93e-08 &  5.27e-16 &      \\ 
     &           &    9 &  1.85e-08 &  1.69e-15 &      \\ 
     &           &   10 &  1.85e-08 &  5.28e-16 &      \\ 
4158 &  4.16e+04 &   10 &           &           & iters  \\ 
 \hdashline 
     &           &    1 &  1.16e-08 &  9.97e-13 &      \\ 
     &           &    2 &  1.15e-08 &  3.30e-16 &      \\ 
     &           &    3 &  1.15e-08 &  3.29e-16 &      \\ 
     &           &    4 &  1.15e-08 &  3.29e-16 &      \\ 
     &           &    5 &  1.15e-08 &  3.28e-16 &      \\ 
     &           &    6 &  1.15e-08 &  3.28e-16 &      \\ 
     &           &    7 &  6.62e-08 &  3.27e-16 &      \\ 
     &           &    8 &  1.16e-08 &  1.89e-15 &      \\ 
     &           &    9 &  1.15e-08 &  3.30e-16 &      \\ 
     &           &   10 &  1.15e-08 &  3.29e-16 &      \\ 
4159 &  4.16e+04 &   10 &           &           & iters  \\ 
 \hdashline 
     &           &    1 &  7.01e-08 &  7.21e-13 &      \\ 
     &           &    2 &  7.66e-09 &  2.00e-15 &      \\ 
     &           &    3 &  7.65e-09 &  2.19e-16 &      \\ 
     &           &    4 &  7.64e-09 &  2.18e-16 &      \\ 
     &           &    5 &  7.62e-09 &  2.18e-16 &      \\ 
     &           &    6 &  7.61e-09 &  2.18e-16 &      \\ 
     &           &    7 &  7.60e-09 &  2.17e-16 &      \\ 
     &           &    8 &  7.59e-09 &  2.17e-16 &      \\ 
     &           &    9 &  7.01e-08 &  2.17e-16 &      \\ 
     &           &   10 &  8.54e-08 &  2.00e-15 &      \\ 
4160 &  4.16e+04 &   10 &           &           & iters  \\ 
 \hdashline 
     &           &    1 &  5.31e-08 &  9.22e-13 &      \\ 
     &           &    2 &  2.47e-08 &  1.52e-15 &      \\ 
     &           &    3 &  2.46e-08 &  7.04e-16 &      \\ 
     &           &    4 &  2.46e-08 &  7.03e-16 &      \\ 
     &           &    5 &  5.31e-08 &  7.02e-16 &      \\ 
     &           &    6 &  2.46e-08 &  1.52e-15 &      \\ 
     &           &    7 &  2.46e-08 &  7.03e-16 &      \\ 
     &           &    8 &  5.31e-08 &  7.02e-16 &      \\ 
     &           &    9 &  2.46e-08 &  1.52e-15 &      \\ 
     &           &   10 &  2.46e-08 &  7.03e-16 &      \\ 
4161 &  4.16e+04 &   10 &           &           & iters  \\ 
 \hdashline 
     &           &    1 &  3.54e-08 &  8.25e-13 &      \\ 
     &           &    2 &  4.23e-08 &  1.01e-15 &      \\ 
     &           &    3 &  3.54e-08 &  1.21e-15 &      \\ 
     &           &    4 &  3.54e-08 &  1.01e-15 &      \\ 
     &           &    5 &  4.24e-08 &  1.01e-15 &      \\ 
     &           &    6 &  3.54e-08 &  1.21e-15 &      \\ 
     &           &    7 &  4.24e-08 &  1.01e-15 &      \\ 
     &           &    8 &  3.54e-08 &  1.21e-15 &      \\ 
     &           &    9 &  4.23e-08 &  1.01e-15 &      \\ 
     &           &   10 &  3.54e-08 &  1.21e-15 &      \\ 
4162 &  4.16e+04 &   10 &           &           & iters  \\ 
 \hdashline 
     &           &    1 &  1.82e-09 &  8.44e-13 &      \\ 
     &           &    2 &  1.82e-09 &  5.21e-17 &      \\ 
     &           &    3 &  1.82e-09 &  5.20e-17 &      \\ 
     &           &    4 &  1.82e-09 &  5.19e-17 &      \\ 
     &           &    5 &  1.81e-09 &  5.18e-17 &      \\ 
     &           &    6 &  1.81e-09 &  5.18e-17 &      \\ 
     &           &    7 &  1.81e-09 &  5.17e-17 &      \\ 
     &           &    8 &  1.81e-09 &  5.16e-17 &      \\ 
     &           &    9 &  1.80e-09 &  5.15e-17 &      \\ 
     &           &   10 &  1.80e-09 &  5.15e-17 &      \\ 
4163 &  4.16e+04 &   10 &           &           & iters  \\ 
 \hdashline 
 
\end{longtable}

\end{document}