\documentclass[a4paper,10pt]{article} 
\usepackage[a4paper,margin=20mm]{geometry} 
\usepackage{longtable} 
\usepackage{float} 
\usepackage{adjustbox} 
\usepackage{graphicx} 
\usepackage{color} 
\usepackage{booktabs} 
\aboverulesep=0ex 
\belowrulesep=0ex 
\usepackage{array} 
\newcolumntype{?}{!{\vrule width 1pt}}
\usepackage{fancyhdr} 
\pagestyle{fancy} 
\fancyhf{} 
\rhead{Problem: Arc-LengthLogarithmicStrain } 
\lfoot{Date: \today} 
\rfoot{Page \thepage} 
\renewcommand{\footrulewidth}{1pt} 
\renewcommand{\headrulewidth}{1.5pt} 
\setlength{\parindent}{0pt} 
\usepackage[T1]{fontenc} 
\usepackage{libertine} 
\usepackage{arydshln} 
\definecolor{miblue}{rgb}{0,0.1,0.38} 
\usepackage{titlesec} 
\titleformat{\section}{\normalfont\Large\color{miblue}\bfseries}{\color{miblue}\sectionmark\thesection}{0.5em}{}[{\color{miblue}\titlerule[0.5pt]}] 

\begin{document} 
This is an ONSAS automatically-generated report with part of the results obtained after the analysis. The user can access other magnitudes and results through the GNU-Octave/MATLAB console. The code is provided AS IS \textbf{WITHOUT WARRANTY of any kind}, express or implied.

\section{Analysis results}

\begin{longtable}{cccccc} 
$\#t$ & $t$ & its & $\| RHS \|$ & $\| \Delta u \|$ & flagExit \\ \hline 
 \endhead 
   1 &  0.00e+00 &    0 &           &           &   \\ 
 \hdashline 
     &           &    1 &  1.60e-10 &  3.52e-08 &      \\ 
     &           &    2 &  2.26e+00 &  3.00e-01 &      \\ 
     &           &    3 &  0.00e+00 &  0.00e+00 &      \\ 
   2 &  1.00e+00 &    3 &           &           & displac  \\ 
 \hdashline 
     &           &    1 &  2.26e+00 &  3.00e-01 &      \\ 
     &           &    2 &  0.00e+00 &  0.00e+00 &      \\ 
   3 &  2.00e+00 &    2 &           &           & displac  \\ 
 \hdashline 
     &           &    1 &  7.32e+02 &  3.00e-01 &      \\ 
     &           &    2 &  0.00e+00 &  0.00e+00 &      \\ 
   4 &  3.00e+00 &    2 &           &           & displac  \\ 
 \hdashline 
     &           &    1 &  1.97e-02 &  3.00e-01 &      \\ 
     &           &    2 &  0.00e+00 &  0.00e+00 &      \\ 
   5 &  4.00e+00 &    2 &           &           & displac  \\ 
 \hdashline 
     &           &    1 &  1.97e-02 &  3.00e-01 &      \\ 
     &           &    2 &  0.00e+00 &  0.00e+00 &      \\ 
   6 &  5.00e+00 &    2 &           &           & displac  \\ 
 \hdashline 
     &           &    1 &  1.97e-02 &  3.00e-01 &      \\ 
     &           &    2 &  4.55e-13 &  4.61e-14 &      \\ 
   7 &  6.00e+00 &    2 &           &           & forces  \\ 
 \hdashline 
     &           &    1 &  1.97e-02 &  3.00e-01 &      \\ 
     &           &    2 &  4.55e-13 &  4.67e-14 &      \\ 
   8 &  7.00e+00 &    2 &           &           & forces  \\ 
 \hdashline 
     &           &    1 &  8.24e+02 &  3.00e-01 &      \\ 
     &           &    2 &  0.00e+00 &  0.00e+00 &      \\ 
   9 &  8.00e+00 &    2 &           &           & displac  \\ 
 \hdashline 
     &           &    1 &  1.97e-02 &  3.00e-01 &      \\ 
     &           &    2 &  0.00e+00 &  0.00e+00 &      \\ 
  10 &  9.00e+00 &    2 &           &           & displac  \\ 
 \hdashline 
     &           &    1 &  1.98e-02 &  3.00e-01 &      \\ 
     &           &    2 &  0.00e+00 &  0.00e+00 &      \\ 
  11 &  1.00e+01 &    2 &           &           & displac  \\ 
 \hdashline 
     &           &    1 &  1.98e-02 &  3.00e-01 &      \\ 
     &           &    2 &  6.82e-13 &  2.47e-14 &      \\ 
  12 &  1.10e+01 &    2 &           &           & forces  \\ 
 \hdashline 
     &           &    1 &  1.98e-02 &  3.00e-01 &      \\ 
     &           &    2 &  1.14e-12 &  7.51e-14 &      \\ 
  13 &  1.20e+01 &    2 &           &           & forces  \\ 
 \hdashline 
     &           &    1 &  1.98e-02 &  3.00e-01 &      \\ 
     &           &    2 &  4.55e-13 &  1.27e-13 &      \\ 
  14 &  1.30e+01 &    2 &           &           & forces  \\ 
 \hdashline 
     &           &    1 &  1.98e-02 &  3.00e-01 &      \\ 
     &           &    2 &  4.55e-13 &  5.16e-14 &      \\ 
  15 &  1.40e+01 &    2 &           &           & forces  \\ 
 \hdashline 
     &           &    1 &  1.98e-02 &  3.00e-01 &      \\ 
     &           &    2 &  0.00e+00 &  5.23e-14 &      \\ 
  16 &  1.50e+01 &    2 &           &           & displac  \\ 
 \hdashline 
     &           &    1 &  1.98e-02 &  3.00e-01 &      \\ 
     &           &    2 &  0.00e+00 &  0.00e+00 &      \\ 
  17 &  1.60e+01 &    2 &           &           & displac  \\ 
 \hdashline 
     &           &    1 &  1.98e-02 &  3.00e-01 &      \\ 
     &           &    2 &  0.00e+00 &  0.00e+00 &      \\ 
  18 &  1.70e+01 &    2 &           &           & displac  \\ 
 \hdashline 
     &           &    1 &  1.98e-02 &  3.00e-01 &      \\ 
     &           &    2 &  0.00e+00 &  0.00e+00 &      \\ 
  19 &  1.80e+01 &    2 &           &           & displac  \\ 
 \hdashline 
     &           &    1 &  1.98e-02 &  3.00e-01 &      \\ 
     &           &    2 &  0.00e+00 &  0.00e+00 &      \\ 
  20 &  1.90e+01 &    2 &           &           & displac  \\ 
 \hdashline 
     &           &    1 &  1.99e-02 &  3.00e-01 &      \\ 
     &           &    2 &  0.00e+00 &  0.00e+00 &      \\ 
  21 &  2.00e+01 &    2 &           &           & displac  \\ 
 \hdashline 
     &           &    1 &  1.99e-02 &  3.00e-01 &      \\ 
     &           &    2 &  0.00e+00 &  0.00e+00 &      \\ 
  22 &  2.10e+01 &    2 &           &           & displac  \\ 
 \hdashline 
     &           &    1 &  1.99e-02 &  3.00e-01 &      \\ 
     &           &    2 &  0.00e+00 &  0.00e+00 &      \\ 
  23 &  2.20e+01 &    2 &           &           & displac  \\ 
 \hdashline 
     &           &    1 &  1.99e-02 &  3.00e-01 &      \\ 
     &           &    2 &  0.00e+00 &  0.00e+00 &      \\ 
  24 &  2.30e+01 &    2 &           &           & displac  \\ 
 \hdashline 
     &           &    1 &  1.99e-02 &  3.00e-01 &      \\ 
     &           &    2 &  0.00e+00 &  0.00e+00 &      \\ 
  25 &  2.40e+01 &    2 &           &           & displac  \\ 
 \hdashline 
     &           &    1 &  1.99e-02 &  3.00e-01 &      \\ 
     &           &    2 &  4.55e-13 &  3.09e-14 &      \\ 
  26 &  2.50e+01 &    2 &           &           & forces  \\ 
 \hdashline 
     &           &    1 &  7.70e+02 &  3.00e-01 &      \\ 
     &           &    2 &  2.27e-13 &  0.00e+00 &      \\ 
  27 &  2.60e+01 &    2 &           &           & forces  \\ 
 \hdashline 
     &           &    1 &  1.99e-02 &  3.00e-01 &      \\ 
     &           &    2 &  9.09e-13 &  3.20e-14 &      \\ 
  28 &  2.70e+01 &    2 &           &           & forces  \\ 
 \hdashline 
     &           &    1 &  7.64e+02 &  3.00e-01 &      \\ 
     &           &    2 &  0.00e+00 &  0.00e+00 &      \\ 
  29 &  2.80e+01 &    2 &           &           & displac  \\ 
 \hdashline 
     &           &    1 &  1.99e-02 &  3.00e-01 &      \\ 
     &           &    2 &  0.00e+00 &  0.00e+00 &      \\ 
  30 &  2.90e+01 &    2 &           &           & displac  \\ 
 \hdashline 
     &           &    1 &  1.99e-02 &  3.00e-01 &      \\ 
     &           &    2 &  0.00e+00 &  0.00e+00 &      \\ 
  31 &  3.00e+01 &    2 &           &           & displac  \\ 
 \hdashline 
     &           &    1 &  1.99e-02 &  3.00e-01 &      \\ 
     &           &    2 &  0.00e+00 &  4.34e-19 &      \\ 
  32 &  3.10e+01 &    2 &           &           & displac  \\ 
 \hdashline 
     &           &    1 &  1.99e-02 &  3.00e-01 &      \\ 
     &           &    2 &  0.00e+00 &  4.34e-19 &      \\ 
  33 &  3.20e+01 &    2 &           &           & displac  \\ 
 \hdashline 
     &           &    1 &  1.99e-02 &  3.00e-01 &      \\ 
     &           &    2 &  2.27e-13 &  7.22e-14 &      \\ 
  34 &  3.30e+01 &    2 &           &           & forces  \\ 
 \hdashline 
     &           &    1 &  1.99e-02 &  3.00e-01 &      \\ 
     &           &    2 &  4.55e-13 &  3.69e-14 &      \\ 
  35 &  3.40e+01 &    2 &           &           & forces  \\ 
 \hdashline 
     &           &    1 &  1.99e-02 &  3.00e-01 &      \\ 
     &           &    2 &  2.27e-13 &  7.53e-14 &      \\ 
  36 &  3.50e+01 &    2 &           &           & forces  \\ 
 \hdashline 
     &           &    1 &  2.00e-02 &  3.00e-01 &      \\ 
     &           &    2 &  4.55e-13 &  3.85e-14 &      \\ 
  37 &  3.60e+01 &    2 &           &           & forces  \\ 
 \hdashline 
     &           &    1 &  2.00e-02 &  3.00e-01 &      \\ 
     &           &    2 &  4.55e-13 &  7.88e-14 &      \\ 
  38 &  3.70e+01 &    2 &           &           & forces  \\ 
 \hdashline 
     &           &    1 &  2.00e-02 &  3.00e-01 &      \\ 
     &           &    2 &  6.82e-13 &  8.07e-14 &      \\ 
  39 &  3.80e+01 &    2 &           &           & forces  \\ 
 \hdashline 
     &           &    1 &  2.00e-02 &  3.00e-01 &      \\ 
     &           &    2 &  0.00e+00 &  1.24e-13 &      \\ 
  40 &  3.90e+01 &    2 &           &           & displac  \\ 
 \hdashline 
     &           &    1 &  2.00e-02 &  3.00e-01 &      \\ 
     &           &    2 &  0.00e+00 &  4.34e-19 &      \\ 
  41 &  4.00e+01 &    2 &           &           & displac  \\ 
 \hdashline 
     &           &    1 &  2.00e-02 &  3.00e-01 &      \\ 
     &           &    2 &  0.00e+00 &  0.00e+00 &      \\ 
  42 &  4.10e+01 &    2 &           &           & displac  \\ 
 \hdashline 
     &           &    1 &  7.23e+02 &  3.00e-01 &      \\ 
     &           &    2 &  0.00e+00 &  0.00e+00 &      \\ 
  43 &  4.20e+01 &    2 &           &           & displac  \\ 
 \hdashline 
     &           &    1 &  2.00e-02 &  3.00e-01 &      \\ 
     &           &    2 &  0.00e+00 &  0.00e+00 &      \\ 
  44 &  4.30e+01 &    2 &           &           & displac  \\ 
 \hdashline 
     &           &    1 &  2.00e-02 &  3.00e-01 &      \\ 
     &           &    2 &  1.36e-12 &  9.40e-14 &      \\ 
  45 &  4.40e+01 &    2 &           &           & forces  \\ 
 \hdashline 
     &           &    1 &  2.00e-02 &  3.00e-01 &      \\ 
     &           &    2 &  2.27e-13 &  2.90e-13 &      \\ 
  46 &  4.50e+01 &    2 &           &           & forces  \\ 
 \hdashline 
     &           &    1 &  2.00e-02 &  3.00e-01 &      \\ 
     &           &    2 &  4.55e-13 &  4.97e-14 &      \\ 
  47 &  4.60e+01 &    2 &           &           & forces  \\ 
 \hdashline 
     &           &    1 &  2.00e-02 &  3.00e-01 &      \\ 
     &           &    2 &  4.55e-13 &  1.02e-13 &      \\ 
  48 &  4.70e+01 &    2 &           &           & forces  \\ 
 \hdashline 
     &           &    1 &  2.00e-02 &  3.00e-01 &      \\ 
     &           &    2 &  2.27e-13 &  1.06e-13 &      \\ 
  49 &  4.80e+01 &    2 &           &           & forces  \\ 
 \hdashline 
     &           &    1 &  2.00e-02 &  3.00e-01 &      \\ 
     &           &    2 &  9.09e-13 &  5.45e-14 &      \\ 
  50 &  4.90e+01 &    2 &           &           & forces  \\ 
 \hdashline 
     &           &    1 &  2.00e-02 &  3.00e-01 &      \\ 
     &           &    2 &  2.27e-13 &  2.25e-13 &      \\ 
  51 &  5.00e+01 &    2 &           &           & forces  \\ 
 \hdashline 
     &           &    1 &  2.00e-02 &  3.00e-01 &      \\ 
     &           &    2 &  4.55e-13 &  5.82e-14 &      \\ 
  52 &  5.10e+01 &    2 &           &           & forces  \\ 
 \hdashline 
     &           &    1 &  2.00e-02 &  3.00e-01 &      \\ 
     &           &    2 &  1.36e-12 &  1.21e-13 &      \\ 
  53 &  5.20e+01 &    2 &           &           & forces  \\ 
 \hdashline 
     &           &    1 &  2.00e-02 &  3.00e-01 &      \\ 
     &           &    2 &  4.55e-13 &  3.75e-13 &      \\ 
  54 &  5.30e+01 &    2 &           &           & forces  \\ 
 \hdashline 
     &           &    1 &  2.00e-02 &  3.00e-01 &      \\ 
     &           &    2 &  2.27e-13 &  1.30e-13 &      \\ 
  55 &  5.40e+01 &    2 &           &           & forces  \\ 
 \hdashline 
     &           &    1 &  2.00e-02 &  3.00e-01 &      \\ 
     &           &    2 &  2.27e-13 &  6.74e-14 &      \\ 
  56 &  5.50e+01 &    2 &           &           & forces  \\ 
 \hdashline 
     &           &    1 &  2.00e-02 &  3.00e-01 &      \\ 
     &           &    2 &  1.14e-12 &  7.02e-14 &      \\ 
  57 &  5.60e+01 &    2 &           &           & forces  \\ 
 \hdashline 
     &           &    1 &  6.78e+02 &  3.00e-01 &      \\ 
     &           &    2 &  0.00e+00 &  0.00e+00 &      \\ 
  58 &  5.70e+01 &    2 &           &           & displac  \\ 
 \hdashline 
     &           &    1 &  2.00e-02 &  3.00e-01 &      \\ 
     &           &    2 &  0.00e+00 &  8.67e-19 &      \\ 
  59 &  5.80e+01 &    2 &           &           & displac  \\ 
 \hdashline 
     &           &    1 &  2.00e-02 &  3.00e-01 &      \\ 
     &           &    2 &  0.00e+00 &  0.00e+00 &      \\ 
  60 &  5.90e+01 &    2 &           &           & displac  \\ 
 \hdashline 
     &           &    1 &  2.00e-02 &  3.00e-01 &      \\ 
     &           &    2 &  0.00e+00 &  0.00e+00 &      \\ 
  61 &  6.00e+01 &    2 &           &           & displac  \\ 
 \hdashline 
     &           &    1 &  2.00e-02 &  3.00e-01 &      \\ 
     &           &    2 &  0.00e+00 &  0.00e+00 &      \\ 
  62 &  6.10e+01 &    2 &           &           & displac  \\ 
 \hdashline 
     &           &    1 &  2.00e-02 &  3.00e-01 &      \\ 
     &           &    2 &  0.00e+00 &  0.00e+00 &      \\ 
  63 &  6.20e+01 &    2 &           &           & displac  \\ 
 \hdashline 
     &           &    1 &  6.60e+02 &  3.00e-01 &      \\ 
     &           &    2 &  0.00e+00 &  0.00e+00 &      \\ 
  64 &  6.30e+01 &    2 &           &           & displac  \\ 
 \hdashline 
     &           &    1 &  2.00e-02 &  3.00e-01 &      \\ 
     &           &    2 &  4.55e-13 &  2.10e-13 &      \\ 
  65 &  6.40e+01 &    2 &           &           & forces  \\ 
 \hdashline 
     &           &    1 &  6.55e+02 &  3.00e-01 &      \\ 
     &           &    2 &  2.27e-13 &  0.00e+00 &      \\ 
  66 &  6.50e+01 &    2 &           &           & forces  \\ 
 \hdashline 
     &           &    1 &  2.00e-02 &  3.00e-01 &      \\ 
     &           &    2 &  1.59e-12 &  1.20e-13 &      \\ 
  67 &  6.60e+01 &    2 &           &           & forces  \\ 
 \hdashline 
     &           &    1 &  2.00e-02 &  3.00e-01 &      \\ 
     &           &    2 &  0.00e+00 &  8.99e-13 &      \\ 
  68 &  6.70e+01 &    2 &           &           & displac  \\ 
 \hdashline 
     &           &    1 &  2.00e-02 &  3.00e-01 &      \\ 
     &           &    2 &  0.00e+00 &  1.73e-18 &      \\ 
  69 &  6.80e+01 &    2 &           &           & displac  \\ 
 \hdashline 
     &           &    1 &  2.00e-02 &  3.00e-01 &      \\ 
     &           &    2 &  0.00e+00 &  0.00e+00 &      \\ 
  70 &  6.90e+01 &    2 &           &           & displac  \\ 
 \hdashline 
     &           &    1 &  2.00e-02 &  3.00e-01 &      \\ 
     &           &    2 &  0.00e+00 &  0.00e+00 &      \\ 
  71 &  7.00e+01 &    2 &           &           & displac  \\ 
 \hdashline 
     &           &    1 &  2.00e-02 &  3.00e-01 &      \\ 
     &           &    2 &  0.00e+00 &  0.00e+00 &      \\ 
  72 &  7.10e+01 &    2 &           &           & displac  \\ 
 \hdashline 
     &           &    1 &  2.00e-02 &  3.00e-01 &      \\ 
     &           &    2 &  0.00e+00 &  0.00e+00 &      \\ 
  73 &  7.20e+01 &    2 &           &           & displac  \\ 
 \hdashline 
     &           &    1 &  2.00e-02 &  3.00e-01 &      \\ 
     &           &    2 &  0.00e+00 &  0.00e+00 &      \\ 
  74 &  7.30e+01 &    2 &           &           & displac  \\ 
 \hdashline 
     &           &    1 &  2.00e-02 &  3.00e-01 &      \\ 
     &           &    2 &  0.00e+00 &  0.00e+00 &      \\ 
  75 &  7.40e+01 &    2 &           &           & displac  \\ 
 \hdashline 
     &           &    1 &  2.00e-02 &  3.00e-01 &      \\ 
     &           &    2 &  0.00e+00 &  0.00e+00 &      \\ 
  76 &  7.50e+01 &    2 &           &           & displac  \\ 
 \hdashline 
     &           &    1 &  2.00e-02 &  3.00e-01 &      \\ 
     &           &    2 &  4.55e-13 &  3.98e-13 &      \\ 
  77 &  7.60e+01 &    2 &           &           & forces  \\ 
 \hdashline 
     &           &    1 &  6.19e+02 &  3.00e-01 &      \\ 
     &           &    2 &  0.00e+00 &  0.00e+00 &      \\ 
  78 &  7.70e+01 &    2 &           &           & displac  \\ 
 \hdashline 
     &           &    1 &  2.00e-02 &  3.00e-01 &      \\ 
     &           &    2 &  0.00e+00 &  0.00e+00 &      \\ 
  79 &  7.80e+01 &    2 &           &           & displac  \\ 
 \hdashline 
     &           &    1 &  1.99e-02 &  3.00e-01 &      \\ 
     &           &    2 &  0.00e+00 &  0.00e+00 &      \\ 
  80 &  7.90e+01 &    2 &           &           & displac  \\ 
 \hdashline 
     &           &    1 &  1.99e-02 &  3.00e-01 &      \\ 
     &           &    2 &  0.00e+00 &  0.00e+00 &      \\ 
  81 &  8.00e+01 &    2 &           &           & displac  \\ 
 \hdashline 
     &           &    1 &  1.99e-02 &  3.00e-01 &      \\ 
     &           &    2 &  0.00e+00 &  0.00e+00 &      \\ 
  82 &  8.10e+01 &    2 &           &           & displac  \\ 
 \hdashline 
     &           &    1 &  6.05e+02 &  3.00e-01 &      \\ 
     &           &    2 &  0.00e+00 &  0.00e+00 &      \\ 
  83 &  8.20e+01 &    2 &           &           & displac  \\ 
 \hdashline 
     &           &    1 &  1.99e-02 &  3.00e-01 &      \\ 
     &           &    2 &  0.00e+00 &  0.00e+00 &      \\ 
  84 &  8.30e+01 &    2 &           &           & displac  \\ 
 \hdashline 
     &           &    1 &  1.99e-02 &  3.00e-01 &      \\ 
     &           &    2 &  0.00e+00 &  0.00e+00 &      \\ 
  85 &  8.40e+01 &    2 &           &           & displac  \\ 
 \hdashline 
     &           &    1 &  5.96e+02 &  3.00e-01 &      \\ 
     &           &    2 &  1.36e-12 &  1.14e-13 &      \\ 
  86 &  8.50e+01 &    2 &           &           & forces  \\ 
 \hdashline 
     &           &    1 &  1.99e-02 &  3.00e-01 &      \\ 
     &           &    2 &  2.27e-13 &  2.10e-12 &      \\ 
  87 &  8.60e+01 &    2 &           &           & forces  \\ 
 \hdashline 
     &           &    1 &  5.90e+02 &  3.00e-01 &      \\ 
     &           &    2 &  0.00e+00 &  0.00e+00 &      \\ 
  88 &  8.70e+01 &    2 &           &           & displac  \\ 
 \hdashline 
     &           &    1 &  1.99e-02 &  3.00e-01 &      \\ 
     &           &    2 &  0.00e+00 &  0.00e+00 &      \\ 
  89 &  8.80e+01 &    2 &           &           & displac  \\ 
 \hdashline 
     &           &    1 &  1.99e-02 &  3.00e-01 &      \\ 
     &           &    2 &  0.00e+00 &  1.73e-18 &      \\ 
  90 &  8.90e+01 &    2 &           &           & displac  \\ 
 \hdashline 
     &           &    1 &  1.99e-02 &  3.00e-01 &      \\ 
     &           &    2 &  0.00e+00 &  0.00e+00 &      \\ 
  91 &  9.00e+01 &    2 &           &           & displac  \\ 
 \hdashline 
     &           &    1 &  1.99e-02 &  3.00e-01 &      \\ 
     &           &    2 &  6.82e-13 &  1.60e-13 &      \\ 
  92 &  9.10e+01 &    2 &           &           & forces  \\ 
 \hdashline 
     &           &    1 &  5.76e+02 &  3.00e-01 &      \\ 
     &           &    2 &  4.55e-13 &  5.68e-14 &      \\ 
  93 &  9.20e+01 &    2 &           &           & forces  \\ 
 \hdashline 
     &           &    1 &  1.98e-02 &  3.00e-01 &      \\ 
     &           &    2 &  2.27e-13 &  2.70e-13 &      \\ 
  94 &  9.30e+01 &    2 &           &           & forces  \\ 
 \hdashline 
     &           &    1 &  5.70e+02 &  3.00e-01 &      \\ 
     &           &    2 &  0.00e+00 &  0.00e+00 &      \\ 
  95 &  9.40e+01 &    2 &           &           & displac  \\ 
 \hdashline 
     &           &    1 &  1.98e-02 &  3.00e-01 &      \\ 
     &           &    2 &  4.55e-13 &  2.33e-13 &      \\ 
  96 &  9.50e+01 &    2 &           &           & forces  \\ 
 \hdashline 
     &           &    1 &  1.98e-02 &  3.00e-01 &      \\ 
     &           &    2 &  0.00e+00 &  2.18e-13 &      \\ 
  97 &  9.60e+01 &    2 &           &           & displac  \\ 
 \hdashline 
     &           &    1 &  1.98e-02 &  3.00e-01 &      \\ 
     &           &    2 &  0.00e+00 &  0.00e+00 &      \\ 
  98 &  9.70e+01 &    2 &           &           & displac  \\ 
 \hdashline 
     &           &    1 &  1.98e-02 &  3.00e-01 &      \\ 
     &           &    2 &  0.00e+00 &  0.00e+00 &      \\ 
  99 &  9.80e+01 &    2 &           &           & displac  \\ 
 \hdashline 
     &           &    1 &  1.98e-02 &  3.00e-01 &      \\ 
     &           &    2 &  0.00e+00 &  0.00e+00 &      \\ 
 100 &  9.90e+01 &    2 &           &           & displac  \\ 
 \hdashline 
     &           &    1 &  1.98e-02 &  3.00e-01 &      \\ 
     &           &    2 &  0.00e+00 &  8.67e-19 &      \\ 
 101 &  1.00e+02 &    2 &           &           & displac  \\ 
 \hdashline 
     &           &    1 &  1.98e-02 &  3.00e-01 &      \\ 
     &           &    2 &  0.00e+00 &  0.00e+00 &      \\ 
 102 &  1.01e+02 &    2 &           &           & displac  \\ 
 \hdashline 
     &           &    1 &  1.98e-02 &  3.00e-01 &      \\ 
     &           &    2 &  0.00e+00 &  0.00e+00 &      \\ 
 103 &  1.02e+02 &    2 &           &           & displac  \\ 
 \hdashline 
     &           &    1 &  1.97e-02 &  3.00e-01 &      \\ 
     &           &    2 &  0.00e+00 &  8.67e-19 &      \\ 
 104 &  1.03e+02 &    2 &           &           & displac  \\ 
 \hdashline 
     &           &    1 &  1.97e-02 &  3.00e-01 &      \\ 
     &           &    2 &  0.00e+00 &  8.67e-19 &      \\ 
 105 &  1.04e+02 &    2 &           &           & displac  \\ 
 \hdashline 
     &           &    1 &  1.97e-02 &  3.00e-01 &      \\ 
     &           &    2 &  0.00e+00 &  0.00e+00 &      \\ 
 106 &  1.05e+02 &    2 &           &           & displac  \\ 
 \hdashline 
     &           &    1 &  1.97e-02 &  3.00e-01 &      \\ 
     &           &    2 &  0.00e+00 &  0.00e+00 &      \\ 
 107 &  1.06e+02 &    2 &           &           & displac  \\ 
 \hdashline 
     &           &    1 &  1.97e-02 &  3.00e-01 &      \\ 
     &           &    2 &  0.00e+00 &  0.00e+00 &      \\ 
 108 &  1.07e+02 &    2 &           &           & displac  \\ 
 \hdashline 
     &           &    1 &  1.97e-02 &  3.00e-01 &      \\ 
     &           &    2 &  0.00e+00 &  0.00e+00 &      \\ 
 109 &  1.08e+02 &    2 &           &           & displac  \\ 
 \hdashline 
     &           &    1 &  1.97e-02 &  3.00e-01 &      \\ 
     &           &    2 &  0.00e+00 &  0.00e+00 &      \\ 
 110 &  1.09e+02 &    2 &           &           & displac  \\ 
 \hdashline 
     &           &    1 &  1.96e-02 &  3.00e-01 &      \\ 
     &           &    2 &  0.00e+00 &  0.00e+00 &      \\ 
 111 &  1.10e+02 &    2 &           &           & displac  \\ 
 \hdashline 
     &           &    1 &  1.96e-02 &  3.00e-01 &      \\ 
     &           &    2 &  0.00e+00 &  0.00e+00 &      \\ 
 112 &  1.11e+02 &    2 &           &           & displac  \\ 
 \hdashline 
     &           &    1 &  1.96e-02 &  3.00e-01 &      \\ 
     &           &    2 &  0.00e+00 &  0.00e+00 &      \\ 
 113 &  1.12e+02 &    2 &           &           & displac  \\ 
 \hdashline 
     &           &    1 &  1.96e-02 &  3.00e-01 &      \\ 
     &           &    2 &  0.00e+00 &  0.00e+00 &      \\ 
 114 &  1.13e+02 &    2 &           &           & displac  \\ 
 \hdashline 
     &           &    1 &  1.96e-02 &  3.00e-01 &      \\ 
     &           &    2 &  0.00e+00 &  1.02e-13 &      \\ 
 115 &  1.14e+02 &    2 &           &           & displac  \\ 
 \hdashline 
     &           &    1 &  5.11e+02 &  3.00e-01 &      \\ 
     &           &    2 &  0.00e+00 &  0.00e+00 &      \\ 
 116 &  1.15e+02 &    2 &           &           & displac  \\ 
 \hdashline 
     &           &    1 &  1.96e-02 &  3.00e-01 &      \\ 
     &           &    2 &  0.00e+00 &  0.00e+00 &      \\ 
 117 &  1.16e+02 &    2 &           &           & displac  \\ 
 \hdashline 
     &           &    1 &  1.95e-02 &  3.00e-01 &      \\ 
     &           &    2 &  1.14e-12 &  4.70e-14 &      \\ 
 118 &  1.17e+02 &    2 &           &           & forces  \\ 
 \hdashline 
     &           &    1 &  1.95e-02 &  3.00e-01 &      \\ 
     &           &    2 &  2.27e-13 &  2.29e-13 &      \\ 
 119 &  1.18e+02 &    2 &           &           & forces  \\ 
 \hdashline 
     &           &    1 &  1.95e-02 &  3.00e-01 &      \\ 
     &           &    2 &  9.09e-13 &  4.46e-14 &      \\ 
 120 &  1.19e+02 &    2 &           &           & forces  \\ 
 \hdashline 
     &           &    1 &  1.95e-02 &  3.00e-01 &      \\ 
     &           &    2 &  0.00e+00 &  1.74e-13 &      \\ 
 121 &  1.20e+02 &    2 &           &           & displac  \\ 
 \hdashline 
     &           &    1 &  4.94e+02 &  3.00e-01 &      \\ 
     &           &    2 &  0.00e+00 &  0.00e+00 &      \\ 
 122 &  1.21e+02 &    2 &           &           & displac  \\ 
 \hdashline 
     &           &    1 &  1.95e-02 &  3.00e-01 &      \\ 
     &           &    2 &  6.82e-13 &  4.14e-14 &      \\ 
 123 &  1.22e+02 &    2 &           &           & forces  \\ 
 \hdashline 
     &           &    1 &  1.94e-02 &  3.00e-01 &      \\ 
     &           &    2 &  6.82e-13 &  1.21e-13 &      \\ 
 124 &  1.23e+02 &    2 &           &           & forces  \\ 
 \hdashline 
     &           &    1 &  1.94e-02 &  3.00e-01 &      \\ 
     &           &    2 &  4.55e-13 &  1.19e-13 &      \\ 
 125 &  1.24e+02 &    2 &           &           & forces  \\ 
 \hdashline 
     &           &    1 &  1.94e-02 &  3.00e-01 &      \\ 
     &           &    2 &  2.27e-13 &  7.74e-14 &      \\ 
 126 &  1.25e+02 &    2 &           &           & forces  \\ 
 \hdashline 
     &           &    1 &  1.94e-02 &  3.00e-01 &      \\ 
     &           &    2 &  0.00e+00 &  0.00e+00 &      \\ 
 127 &  1.26e+02 &    2 &           &           & displac  \\ 
 \hdashline 
     &           &    1 &  1.94e-02 &  3.00e-01 &      \\ 
     &           &    2 &  0.00e+00 &  0.00e+00 &      \\ 
 128 &  1.27e+02 &    2 &           &           & displac  \\ 
 \hdashline 
     &           &    1 &  1.93e-02 &  3.00e-01 &      \\ 
     &           &    2 &  0.00e+00 &  0.00e+00 &      \\ 
 129 &  1.28e+02 &    2 &           &           & displac  \\ 
 \hdashline 
     &           &    1 &  1.93e-02 &  3.00e-01 &      \\ 
     &           &    2 &  0.00e+00 &  0.00e+00 &      \\ 
 130 &  1.29e+02 &    2 &           &           & displac  \\ 
 \hdashline 
     &           &    1 &  1.93e-02 &  3.00e-01 &      \\ 
     &           &    2 &  0.00e+00 &  0.00e+00 &      \\ 
 131 &  1.30e+02 &    2 &           &           & displac  \\ 
 \hdashline 
     &           &    1 &  1.93e-02 &  3.00e-01 &      \\ 
     &           &    2 &  0.00e+00 &  0.00e+00 &      \\ 
 132 &  1.31e+02 &    2 &           &           & displac  \\ 
 \hdashline 
     &           &    1 &  1.93e-02 &  3.00e-01 &      \\ 
     &           &    2 &  0.00e+00 &  0.00e+00 &      \\ 
 133 &  1.32e+02 &    2 &           &           & displac  \\ 
 \hdashline 
     &           &    1 &  1.92e-02 &  3.00e-01 &      \\ 
     &           &    2 &  0.00e+00 &  6.58e-14 &      \\ 
 134 &  1.33e+02 &    2 &           &           & displac  \\ 
 \hdashline 
     &           &    1 &  1.92e-02 &  3.00e-01 &      \\ 
     &           &    2 &  0.00e+00 &  0.00e+00 &      \\ 
 135 &  1.34e+02 &    2 &           &           & displac  \\ 
 \hdashline 
     &           &    1 &  1.92e-02 &  3.00e-01 &      \\ 
     &           &    2 &  0.00e+00 &  0.00e+00 &      \\ 
 136 &  1.35e+02 &    2 &           &           & displac  \\ 
 \hdashline 
     &           &    1 &  1.92e-02 &  3.00e-01 &      \\ 
     &           &    2 &  0.00e+00 &  0.00e+00 &      \\ 
 137 &  1.36e+02 &    2 &           &           & displac  \\ 
 \hdashline 
     &           &    1 &  1.92e-02 &  3.00e-01 &      \\ 
     &           &    2 &  0.00e+00 &  0.00e+00 &      \\ 
 138 &  1.37e+02 &    2 &           &           & displac  \\ 
 \hdashline 
     &           &    1 &  1.91e-02 &  3.00e-01 &      \\ 
     &           &    2 &  0.00e+00 &  0.00e+00 &      \\ 
 139 &  1.38e+02 &    2 &           &           & displac  \\ 
 \hdashline 
     &           &    1 &  1.91e-02 &  3.00e-01 &      \\ 
     &           &    2 &  0.00e+00 &  0.00e+00 &      \\ 
 140 &  1.39e+02 &    2 &           &           & displac  \\ 
 \hdashline 
     &           &    1 &  1.91e-02 &  3.00e-01 &      \\ 
     &           &    2 &  0.00e+00 &  0.00e+00 &      \\ 
 141 &  1.40e+02 &    2 &           &           & displac  \\ 
 \hdashline 
     &           &    1 &  1.91e-02 &  3.00e-01 &      \\ 
     &           &    2 &  0.00e+00 &  0.00e+00 &      \\ 
 142 &  1.41e+02 &    2 &           &           & displac  \\ 
 \hdashline 
     &           &    1 &  1.90e-02 &  3.00e-01 &      \\ 
     &           &    2 &  9.09e-13 &  2.82e-14 &      \\ 
 143 &  1.42e+02 &    2 &           &           & forces  \\ 
 \hdashline 
     &           &    1 &  1.90e-02 &  3.00e-01 &      \\ 
     &           &    2 &  4.55e-13 &  1.11e-13 &      \\ 
 144 &  1.43e+02 &    2 &           &           & forces  \\ 
 \hdashline 
     &           &    1 &  1.90e-02 &  3.00e-01 &      \\ 
     &           &    2 &  1.14e-12 &  5.47e-14 &      \\ 
 145 &  1.44e+02 &    2 &           &           & forces  \\ 
 \hdashline 
     &           &    1 &  1.90e-02 &  3.00e-01 &      \\ 
     &           &    2 &  2.27e-13 &  1.35e-13 &      \\ 
 146 &  1.45e+02 &    2 &           &           & forces  \\ 
 \hdashline 
     &           &    1 &  1.89e-02 &  3.00e-01 &      \\ 
     &           &    2 &  2.27e-13 &  2.65e-14 &      \\ 
 147 &  1.46e+02 &    2 &           &           & forces  \\ 
 \hdashline 
     &           &    1 &  1.89e-02 &  3.00e-01 &      \\ 
     &           &    2 &  4.55e-13 &  2.61e-14 &      \\ 
 148 &  1.47e+02 &    2 &           &           & forces  \\ 
 \hdashline 
     &           &    1 &  1.89e-02 &  3.00e-01 &      \\ 
     &           &    2 &  4.55e-13 &  5.16e-14 &      \\ 
 149 &  1.48e+02 &    2 &           &           & forces  \\ 
 \hdashline 
     &           &    1 &  1.89e-02 &  3.00e-01 &      \\ 
     &           &    2 &  6.82e-13 &  5.08e-14 &      \\ 
 150 &  1.49e+02 &    2 &           &           & forces  \\ 
 \hdashline 
     &           &    1 &  1.88e-02 &  3.00e-01 &      \\ 
     &           &    2 &  2.27e-13 &  7.52e-14 &      \\ 
 151 &  1.50e+02 &    2 &           &           & forces  \\ 
 \hdashline 
     &           &    1 &  1.88e-02 &  3.00e-01 &      \\ 
     &           &    2 &  0.00e+00 &  2.47e-14 &      \\ 
 152 &  1.51e+02 &    2 &           &           & displac  \\ 
 \hdashline 
     &           &    1 &  1.88e-02 &  3.00e-01 &      \\ 
     &           &    2 &  0.00e+00 &  0.00e+00 &      \\ 
 153 &  1.52e+02 &    2 &           &           & displac  \\ 
 \hdashline 
     &           &    1 &  1.88e-02 &  3.00e-01 &      \\ 
     &           &    2 &  0.00e+00 &  0.00e+00 &      \\ 
 154 &  1.53e+02 &    2 &           &           & displac  \\ 
 \hdashline 
     &           &    1 &  1.87e-02 &  3.00e-01 &      \\ 
     &           &    2 &  0.00e+00 &  0.00e+00 &      \\ 
 155 &  1.54e+02 &    2 &           &           & displac  \\ 
 \hdashline 
     &           &    1 &  1.87e-02 &  3.00e-01 &      \\ 
     &           &    2 &  0.00e+00 &  0.00e+00 &      \\ 
 156 &  1.55e+02 &    2 &           &           & displac  \\ 
 \hdashline 
     &           &    1 &  1.87e-02 &  3.00e-01 &      \\ 
     &           &    2 &  0.00e+00 &  0.00e+00 &      \\ 
 157 &  1.56e+02 &    2 &           &           & displac  \\ 
 \hdashline 
     &           &    1 &  3.97e+02 &  3.00e-01 &      \\ 
     &           &    2 &  0.00e+00 &  0.00e+00 &      \\ 
 158 &  1.57e+02 &    2 &           &           & displac  \\ 
 \hdashline 
     &           &    1 &  1.86e-02 &  3.00e-01 &      \\ 
     &           &    2 &  0.00e+00 &  2.17e-19 &      \\ 
 159 &  1.58e+02 &    2 &           &           & displac  \\ 
 \hdashline 
     &           &    1 &  1.86e-02 &  3.00e-01 &      \\ 
     &           &    2 &  2.27e-13 &  2.23e-14 &      \\ 
 160 &  1.59e+02 &    2 &           &           & forces  \\ 
 \hdashline 
     &           &    1 &  3.89e+02 &  3.00e-01 &      \\ 
     &           &    2 &  0.00e+00 &  0.00e+00 &      \\ 
 161 &  1.60e+02 &    2 &           &           & displac  \\ 
 \hdashline 
     &           &    1 &  1.85e-02 &  3.00e-01 &      \\ 
     &           &    2 &  0.00e+00 &  2.17e-19 &      \\ 
 162 &  1.61e+02 &    2 &           &           & displac  \\ 
 \hdashline 
     &           &    1 &  1.85e-02 &  3.00e-01 &      \\ 
     &           &    2 &  0.00e+00 &  0.00e+00 &      \\ 
 163 &  1.62e+02 &    2 &           &           & displac  \\ 
 \hdashline 
     &           &    1 &  1.85e-02 &  3.00e-01 &      \\ 
     &           &    2 &  0.00e+00 &  4.26e-14 &      \\ 
 164 &  1.63e+02 &    2 &           &           & displac  \\ 
 \hdashline 
     &           &    1 &  1.84e-02 &  3.00e-01 &      \\ 
     &           &    2 &  0.00e+00 &  2.17e-19 &      \\ 
 165 &  1.64e+02 &    2 &           &           & displac  \\ 
 \hdashline 
     &           &    1 &  1.84e-02 &  3.00e-01 &      \\ 
     &           &    2 &  6.82e-13 &  4.16e-14 &      \\ 
 166 &  1.65e+02 &    2 &           &           & forces  \\ 
 \hdashline 
     &           &    1 &  1.84e-02 &  3.00e-01 &      \\ 
     &           &    2 &  1.36e-12 &  6.17e-14 &      \\ 
 167 &  1.66e+02 &    2 &           &           & forces  \\ 
 \hdashline 
     &           &    1 &  1.83e-02 &  3.00e-01 &      \\ 
     &           &    2 &  2.27e-13 &  1.22e-13 &      \\ 
 168 &  1.67e+02 &    2 &           &           & forces  \\ 
 \hdashline 
     &           &    1 &  1.83e-02 &  3.00e-01 &      \\ 
     &           &    2 &  2.27e-13 &  2.01e-14 &      \\ 
 169 &  1.68e+02 &    2 &           &           & forces  \\ 
 \hdashline 
     &           &    1 &  1.83e-02 &  3.00e-01 &      \\ 
     &           &    2 &  4.55e-13 &  1.99e-14 &      \\ 
 170 &  1.69e+02 &    2 &           &           & forces  \\ 
 \hdashline 
     &           &    1 &  3.63e+02 &  3.00e-01 &      \\ 
     &           &    2 &  0.00e+00 &  0.00e+00 &      \\ 
 171 &  1.70e+02 &    2 &           &           & displac  \\ 
 \hdashline 
     &           &    1 &  1.82e-02 &  3.00e-01 &      \\ 
     &           &    2 &  0.00e+00 &  2.17e-19 &      \\ 
 172 &  1.71e+02 &    2 &           &           & displac  \\ 
 \hdashline 
     &           &    1 &  1.82e-02 &  3.00e-01 &      \\ 
     &           &    2 &  0.00e+00 &  2.17e-19 &      \\ 
 173 &  1.72e+02 &    2 &           &           & displac  \\ 
 \hdashline 
     &           &    1 &  1.81e-02 &  3.00e-01 &      \\ 
     &           &    2 &  0.00e+00 &  0.00e+00 &      \\ 
 174 &  1.73e+02 &    2 &           &           & displac  \\ 
 \hdashline 
     &           &    1 &  1.81e-02 &  3.00e-01 &      \\ 
     &           &    2 &  4.55e-13 &  3.78e-14 &      \\ 
 175 &  1.74e+02 &    2 &           &           & forces  \\ 
 \hdashline 
     &           &    1 &  3.50e+02 &  3.00e-01 &      \\ 
     &           &    2 &  0.00e+00 &  0.00e+00 &      \\ 
 176 &  1.75e+02 &    2 &           &           & displac  \\ 
 \hdashline 
     &           &    1 &  1.80e-02 &  3.00e-01 &      \\ 
     &           &    2 &  0.00e+00 &  0.00e+00 &      \\ 
 177 &  1.76e+02 &    2 &           &           & displac  \\ 
 \hdashline 
     &           &    1 &  3.45e+02 &  3.00e-01 &      \\ 
     &           &    2 &  0.00e+00 &  0.00e+00 &      \\ 
 178 &  1.77e+02 &    2 &           &           & displac  \\ 
 \hdashline 
     &           &    1 &  3.43e+02 &  3.00e-01 &      \\ 
     &           &    2 &  0.00e+00 &  3.55e-15 &      \\ 
 179 &  1.78e+02 &    2 &           &           & displac  \\ 
 \hdashline 
     &           &    1 &  1.79e-02 &  3.00e-01 &      \\ 
     &           &    2 &  6.82e-13 &  1.80e-14 &      \\ 
 180 &  1.79e+02 &    2 &           &           & forces  \\ 
 \hdashline 
     &           &    1 &  3.38e+02 &  3.00e-01 &      \\ 
     &           &    2 &  0.00e+00 &  0.00e+00 &      \\ 
 181 &  1.80e+02 &    2 &           &           & displac  \\ 
 \hdashline 
     &           &    1 &  1.79e-02 &  3.00e-01 &      \\ 
     &           &    2 &  0.00e+00 &  0.00e+00 &      \\ 
 182 &  1.81e+02 &    2 &           &           & displac  \\ 
 \hdashline 
     &           &    1 &  1.78e-02 &  3.00e-01 &      \\ 
     &           &    2 &  0.00e+00 &  0.00e+00 &      \\ 
 183 &  1.82e+02 &    2 &           &           & displac  \\ 
 \hdashline 
     &           &    1 &  3.30e+02 &  3.00e-01 &      \\ 
     &           &    2 &  0.00e+00 &  0.00e+00 &      \\ 
 184 &  1.83e+02 &    2 &           &           & displac  \\ 
 \hdashline 
     &           &    1 &  1.77e-02 &  3.00e-01 &      \\ 
     &           &    2 &  0.00e+00 &  0.00e+00 &      \\ 
 185 &  1.84e+02 &    2 &           &           & displac  \\ 
 \hdashline 
     &           &    1 &  1.77e-02 &  3.00e-01 &      \\ 
     &           &    2 &  0.00e+00 &  2.17e-19 &      \\ 
 186 &  1.85e+02 &    2 &           &           & displac  \\ 
 \hdashline 
     &           &    1 &  1.77e-02 &  3.00e-01 &      \\ 
     &           &    2 &  0.00e+00 &  5.55e-17 &      \\ 
 187 &  1.86e+02 &    2 &           &           & displac  \\ 
 \hdashline 
     &           &    1 &  3.21e+02 &  3.00e-01 &      \\ 
     &           &    2 &  0.00e+00 &  0.00e+00 &      \\ 
 188 &  1.87e+02 &    2 &           &           & displac  \\ 
 \hdashline 
     &           &    1 &  1.76e-02 &  3.00e-01 &      \\ 
     &           &    2 &  0.00e+00 &  0.00e+00 &      \\ 
 189 &  1.88e+02 &    2 &           &           & displac  \\ 
 \hdashline 
     &           &    1 &  1.75e-02 &  3.00e-01 &      \\ 
     &           &    2 &  0.00e+00 &  0.00e+00 &      \\ 
 190 &  1.89e+02 &    2 &           &           & displac  \\ 
 \hdashline 
     &           &    1 &  1.75e-02 &  3.00e-01 &      \\ 
     &           &    2 &  0.00e+00 &  0.00e+00 &      \\ 
 191 &  1.90e+02 &    2 &           &           & displac  \\ 
 \hdashline 
     &           &    1 &  1.75e-02 &  3.00e-01 &      \\ 
     &           &    2 &  0.00e+00 &  0.00e+00 &      \\ 
 192 &  1.91e+02 &    2 &           &           & displac  \\ 
 \hdashline 
     &           &    1 &  1.74e-02 &  3.00e-01 &      \\ 
     &           &    2 &  2.27e-13 &  1.60e-14 &      \\ 
 193 &  1.92e+02 &    2 &           &           & forces  \\ 
 \hdashline 
     &           &    1 &  1.74e-02 &  3.00e-01 &      \\ 
     &           &    2 &  2.27e-13 &  1.59e-14 &      \\ 
 194 &  1.93e+02 &    2 &           &           & forces  \\ 
 \hdashline 
     &           &    1 &  1.73e-02 &  3.00e-01 &      \\ 
     &           &    2 &  2.27e-13 &  1.58e-14 &      \\ 
 195 &  1.94e+02 &    2 &           &           & forces  \\ 
 \hdashline 
     &           &    1 &  1.73e-02 &  3.00e-01 &      \\ 
     &           &    2 &  2.27e-13 &  1.57e-14 &      \\ 
 196 &  1.95e+02 &    2 &           &           & forces  \\ 
 \hdashline 
     &           &    1 &  2.99e+02 &  3.00e-01 &      \\ 
     &           &    2 &  0.00e+00 &  0.00e+00 &      \\ 
 197 &  1.96e+02 &    2 &           &           & displac  \\ 
 \hdashline 
     &           &    1 &  1.72e-02 &  3.00e-01 &      \\ 
     &           &    2 &  4.55e-13 &  4.63e-14 &      \\ 
 198 &  1.97e+02 &    2 &           &           & forces  \\ 
 \hdashline 
     &           &    1 &  2.94e+02 &  3.00e-01 &      \\ 
     &           &    2 &  0.00e+00 &  0.00e+00 &      \\ 
 199 &  1.98e+02 &    2 &           &           & displac  \\ 
 \hdashline 
     &           &    1 &  1.71e-02 &  3.00e-01 &      \\ 
     &           &    2 &  0.00e+00 &  0.00e+00 &      \\ 
 200 &  1.99e+02 &    2 &           &           & displac  \\ 
 \hdashline 
     &           &    1 &  1.71e-02 &  3.00e-01 &      \\ 
     &           &    2 &  0.00e+00 &  0.00e+00 &      \\ 
 201 &  2.00e+02 &    2 &           &           & displac  \\ 
 \hdashline 
     &           &    1 &  1.70e-02 &  3.00e-01 &      \\ 
     &           &    2 &  0.00e+00 &  0.00e+00 &      \\ 
 202 &  2.01e+02 &    2 &           &           & displac  \\ 
 \hdashline 
     &           &    1 &  1.70e-02 &  3.00e-01 &      \\ 
     &           &    2 &  2.27e-13 &  1.48e-14 &      \\ 
 203 &  2.02e+02 &    2 &           &           & forces  \\ 
 \hdashline 
     &           &    1 &  2.82e+02 &  3.00e-01 &      \\ 
     &           &    2 &  0.00e+00 &  0.00e+00 &      \\ 
 204 &  2.03e+02 &    2 &           &           & displac  \\ 
 \hdashline 
     &           &    1 &  1.69e-02 &  3.00e-01 &      \\ 
     &           &    2 &  0.00e+00 &  0.00e+00 &      \\ 
 205 &  2.04e+02 &    2 &           &           & displac  \\ 
 \hdashline 
     &           &    1 &  1.69e-02 &  3.00e-01 &      \\ 
     &           &    2 &  0.00e+00 &  0.00e+00 &      \\ 
 206 &  2.05e+02 &    2 &           &           & displac  \\ 
 \hdashline 
     &           &    1 &  1.68e-02 &  3.00e-01 &      \\ 
     &           &    2 &  0.00e+00 &  0.00e+00 &      \\ 
 207 &  2.06e+02 &    2 &           &           & displac  \\ 
 \hdashline 
     &           &    1 &  1.68e-02 &  3.00e-01 &      \\ 
     &           &    2 &  0.00e+00 &  0.00e+00 &      \\ 
 208 &  2.07e+02 &    2 &           &           & displac  \\ 
 \hdashline 
     &           &    1 &  2.71e+02 &  3.00e-01 &      \\ 
     &           &    2 &  0.00e+00 &  0.00e+00 &      \\ 
 209 &  2.08e+02 &    2 &           &           & displac  \\ 
 \hdashline 
     &           &    1 &  1.67e-02 &  3.00e-01 &      \\ 
     &           &    2 &  4.55e-13 &  1.41e-14 &      \\ 
 210 &  2.09e+02 &    2 &           &           & forces  \\ 
 \hdashline 
     &           &    1 &  1.66e-02 &  3.00e-01 &      \\ 
     &           &    2 &  2.27e-13 &  2.80e-14 &      \\ 
 211 &  2.10e+02 &    2 &           &           & forces  \\ 
 \hdashline 
     &           &    1 &  1.66e-02 &  3.00e-01 &      \\ 
     &           &    2 &  2.27e-13 &  1.39e-14 &      \\ 
 212 &  2.11e+02 &    2 &           &           & forces  \\ 
 \hdashline 
     &           &    1 &  2.62e+02 &  3.00e-01 &      \\ 
     &           &    2 &  0.00e+00 &  0.00e+00 &      \\ 
 213 &  2.12e+02 &    2 &           &           & displac  \\ 
 \hdashline 
     &           &    1 &  1.65e-02 &  3.00e-01 &      \\ 
     &           &    2 &  0.00e+00 &  0.00e+00 &      \\ 
 214 &  2.13e+02 &    2 &           &           & displac  \\ 
 \hdashline 
     &           &    1 &  1.64e-02 &  3.00e-01 &      \\ 
     &           &    2 &  2.27e-13 &  1.37e-14 &      \\ 
 215 &  2.14e+02 &    2 &           &           & forces  \\ 
 \hdashline 
     &           &    1 &  2.55e+02 &  3.00e-01 &      \\ 
     &           &    2 &  0.00e+00 &  0.00e+00 &      \\ 
 216 &  2.15e+02 &    2 &           &           & displac  \\ 
 \hdashline 
     &           &    1 &  1.63e-02 &  3.00e-01 &      \\ 
     &           &    2 &  2.27e-13 &  1.35e-14 &      \\ 
 217 &  2.16e+02 &    2 &           &           & forces  \\ 
 \hdashline 
     &           &    1 &  1.63e-02 &  3.00e-01 &      \\ 
     &           &    2 &  0.00e+00 &  1.34e-14 &      \\ 
 218 &  2.17e+02 &    2 &           &           & displac  \\ 
 \hdashline 
     &           &    1 &  1.62e-02 &  3.00e-01 &      \\ 
     &           &    2 &  0.00e+00 &  1.08e-19 &      \\ 
 219 &  2.18e+02 &    2 &           &           & displac  \\ 
 \hdashline 
     &           &    1 &  1.62e-02 &  3.00e-01 &      \\ 
     &           &    2 &  0.00e+00 &  0.00e+00 &      \\ 
 220 &  2.19e+02 &    2 &           &           & displac  \\ 
 \hdashline 
     &           &    1 &  2.44e+02 &  3.00e-01 &      \\ 
     &           &    2 &  0.00e+00 &  0.00e+00 &      \\ 
 221 &  2.20e+02 &    2 &           &           & displac  \\ 
 \hdashline 
     &           &    1 &  1.61e-02 &  3.00e-01 &      \\ 
     &           &    2 &  0.00e+00 &  0.00e+00 &      \\ 
 222 &  2.21e+02 &    2 &           &           & displac  \\ 
 \hdashline 
     &           &    1 &  1.60e-02 &  3.00e-01 &      \\ 
     &           &    2 &  0.00e+00 &  0.00e+00 &      \\ 
 223 &  2.22e+02 &    2 &           &           & displac  \\ 
 \hdashline 
     &           &    1 &  1.60e-02 &  3.00e-01 &      \\ 
     &           &    2 &  4.55e-13 &  1.29e-14 &      \\ 
 224 &  2.23e+02 &    2 &           &           & forces  \\ 
 \hdashline 
     &           &    1 &  1.59e-02 &  3.00e-01 &      \\ 
     &           &    2 &  9.09e-13 &  2.56e-14 &      \\ 
 225 &  2.24e+02 &    2 &           &           & forces  \\ 
 \hdashline 
     &           &    1 &  1.59e-02 &  3.00e-01 &      \\ 
     &           &    2 &  4.55e-13 &  5.10e-14 &      \\ 
 226 &  2.25e+02 &    2 &           &           & forces  \\ 
 \hdashline 
     &           &    1 &  1.58e-02 &  3.00e-01 &      \\ 
     &           &    2 &  0.00e+00 &  2.54e-14 &      \\ 
 227 &  2.26e+02 &    2 &           &           & displac  \\ 
 \hdashline 
     &           &    1 &  1.58e-02 &  3.00e-01 &      \\ 
     &           &    2 &  0.00e+00 &  0.00e+00 &      \\ 
 228 &  2.27e+02 &    2 &           &           & displac  \\ 
 \hdashline 
     &           &    1 &  1.57e-02 &  3.00e-01 &      \\ 
     &           &    2 &  0.00e+00 &  1.08e-19 &      \\ 
 229 &  2.28e+02 &    2 &           &           & displac  \\ 
 \hdashline 
     &           &    1 &  1.57e-02 &  3.00e-01 &      \\ 
     &           &    2 &  0.00e+00 &  0.00e+00 &      \\ 
 230 &  2.29e+02 &    2 &           &           & displac  \\ 
 \hdashline 
     &           &    1 &  1.56e-02 &  3.00e-01 &      \\ 
     &           &    2 &  0.00e+00 &  0.00e+00 &      \\ 
 231 &  2.30e+02 &    2 &           &           & displac  \\ 
 \hdashline 
     &           &    1 &  1.56e-02 &  3.00e-01 &      \\ 
     &           &    2 &  0.00e+00 &  0.00e+00 &      \\ 
 232 &  2.31e+02 &    2 &           &           & displac  \\ 
 \hdashline 
     &           &    1 &  1.55e-02 &  3.00e-01 &      \\ 
     &           &    2 &  0.00e+00 &  0.00e+00 &      \\ 
 233 &  2.32e+02 &    2 &           &           & displac  \\ 
 \hdashline 
     &           &    1 &  1.55e-02 &  3.00e-01 &      \\ 
     &           &    2 &  0.00e+00 &  0.00e+00 &      \\ 
 234 &  2.33e+02 &    2 &           &           & displac  \\ 
 \hdashline 
     &           &    1 &  1.54e-02 &  3.00e-01 &      \\ 
     &           &    2 &  0.00e+00 &  0.00e+00 &      \\ 
 235 &  2.34e+02 &    2 &           &           & displac  \\ 
 \hdashline 
     &           &    1 &  1.53e-02 &  3.00e-01 &      \\ 
     &           &    2 &  2.27e-13 &  1.20e-14 &      \\ 
 236 &  2.35e+02 &    2 &           &           & forces  \\ 
 \hdashline 
     &           &    1 &  1.53e-02 &  3.00e-01 &      \\ 
     &           &    2 &  2.27e-13 &  1.20e-14 &      \\ 
 237 &  2.36e+02 &    2 &           &           & forces  \\ 
 \hdashline 
     &           &    1 &  2.07e+02 &  3.00e-01 &      \\ 
     &           &    2 &  0.00e+00 &  0.00e+00 &      \\ 
 238 &  2.37e+02 &    2 &           &           & displac  \\ 
 \hdashline 
     &           &    1 &  1.52e-02 &  3.00e-01 &      \\ 
     &           &    2 &  0.00e+00 &  1.19e-14 &      \\ 
 239 &  2.38e+02 &    2 &           &           & displac  \\ 
 \hdashline 
     &           &    1 &  1.51e-02 &  3.00e-01 &      \\ 
     &           &    2 &  0.00e+00 &  0.00e+00 &      \\ 
 240 &  2.39e+02 &    2 &           &           & displac  \\ 
 \hdashline 
     &           &    1 &  1.51e-02 &  3.00e-01 &      \\ 
     &           &    2 &  0.00e+00 &  0.00e+00 &      \\ 
 241 &  2.40e+02 &    2 &           &           & displac  \\ 
 \hdashline 
     &           &    1 &  1.50e-02 &  3.00e-01 &      \\ 
     &           &    2 &  0.00e+00 &  0.00e+00 &      \\ 
 242 &  2.41e+02 &    2 &           &           & displac  \\ 
 \hdashline 
     &           &    1 &  1.49e-02 &  3.00e-01 &      \\ 
     &           &    2 &  0.00e+00 &  0.00e+00 &      \\ 
 243 &  2.42e+02 &    2 &           &           & displac  \\ 
 \hdashline 
     &           &    1 &  1.49e-02 &  3.00e-01 &      \\ 
     &           &    2 &  0.00e+00 &  0.00e+00 &      \\ 
 244 &  2.43e+02 &    2 &           &           & displac  \\ 
 \hdashline 
     &           &    1 &  1.48e-02 &  3.00e-01 &      \\ 
     &           &    2 &  0.00e+00 &  0.00e+00 &      \\ 
 245 &  2.44e+02 &    2 &           &           & displac  \\ 
 \hdashline 
     &           &    1 &  1.48e-02 &  3.00e-01 &      \\ 
     &           &    2 &  0.00e+00 &  0.00e+00 &      \\ 
 246 &  2.45e+02 &    2 &           &           & displac  \\ 
 \hdashline 
     &           &    1 &  1.47e-02 &  3.00e-01 &      \\ 
     &           &    2 &  0.00e+00 &  0.00e+00 &      \\ 
 247 &  2.46e+02 &    2 &           &           & displac  \\ 
 \hdashline 
     &           &    1 &  1.46e-02 &  3.00e-01 &      \\ 
     &           &    2 &  0.00e+00 &  0.00e+00 &      \\ 
 248 &  2.47e+02 &    2 &           &           & displac  \\ 
 \hdashline 
     &           &    1 &  1.46e-02 &  3.00e-01 &      \\ 
     &           &    2 &  0.00e+00 &  0.00e+00 &      \\ 
 249 &  2.48e+02 &    2 &           &           & displac  \\ 
 \hdashline 
     &           &    1 &  1.83e+02 &  3.00e-01 &      \\ 
     &           &    2 &  0.00e+00 &  0.00e+00 &      \\ 
 250 &  2.49e+02 &    2 &           &           & displac  \\ 
 \hdashline 
     &           &    1 &  1.45e-02 &  3.00e-01 &      \\ 
     &           &    2 &  4.55e-13 &  1.12e-14 &      \\ 
 251 &  2.50e+02 &    2 &           &           & forces  \\ 
 \hdashline 
     &           &    1 &  1.79e+02 &  3.00e-01 &      \\ 
     &           &    2 &  0.00e+00 &  0.00e+00 &      \\ 
 252 &  2.51e+02 &    2 &           &           & displac  \\ 
 \hdashline 
     &           &    1 &  1.43e-02 &  3.00e-01 &      \\ 
     &           &    2 &  0.00e+00 &  0.00e+00 &      \\ 
 253 &  2.52e+02 &    2 &           &           & displac  \\ 
 \hdashline 
     &           &    1 &  1.43e-02 &  3.00e-01 &      \\ 
     &           &    2 &  0.00e+00 &  0.00e+00 &      \\ 
 254 &  2.53e+02 &    2 &           &           & displac  \\ 
 \hdashline 
     &           &    1 &  1.73e+02 &  3.00e-01 &      \\ 
     &           &    2 &  0.00e+00 &  0.00e+00 &      \\ 
 255 &  2.54e+02 &    2 &           &           & displac  \\ 
 \hdashline 
     &           &    1 &  1.42e-02 &  3.00e-01 &      \\ 
     &           &    2 &  0.00e+00 &  1.09e-14 &      \\ 
 256 &  2.55e+02 &    2 &           &           & displac  \\ 
 \hdashline 
     &           &    1 &  1.41e-02 &  3.00e-01 &      \\ 
     &           &    2 &  0.00e+00 &  0.00e+00 &      \\ 
 257 &  2.56e+02 &    2 &           &           & displac  \\ 
 \hdashline 
     &           &    1 &  1.40e-02 &  3.00e-01 &      \\ 
     &           &    2 &  0.00e+00 &  0.00e+00 &      \\ 
 258 &  2.57e+02 &    2 &           &           & displac  \\ 
 \hdashline 
     &           &    1 &  1.40e-02 &  3.00e-01 &      \\ 
     &           &    2 &  0.00e+00 &  0.00e+00 &      \\ 
 259 &  2.58e+02 &    2 &           &           & displac  \\ 
 \hdashline 
     &           &    1 &  1.39e-02 &  3.00e-01 &      \\ 
     &           &    2 &  0.00e+00 &  0.00e+00 &      \\ 
 260 &  2.59e+02 &    2 &           &           & displac  \\ 
 \hdashline 
     &           &    1 &  1.62e+02 &  3.00e-01 &      \\ 
     &           &    2 &  0.00e+00 &  0.00e+00 &      \\ 
 261 &  2.60e+02 &    2 &           &           & displac  \\ 
 \hdashline 
     &           &    1 &  1.38e-02 &  3.00e-01 &      \\ 
     &           &    2 &  0.00e+00 &  0.00e+00 &      \\ 
 262 &  2.61e+02 &    2 &           &           & displac  \\ 
 \hdashline 
     &           &    1 &  1.37e-02 &  3.00e-01 &      \\ 
     &           &    2 &  0.00e+00 &  0.00e+00 &      \\ 
 263 &  2.62e+02 &    2 &           &           & displac  \\ 
 \hdashline 
     &           &    1 &  1.36e-02 &  3.00e-01 &      \\ 
     &           &    2 &  0.00e+00 &  0.00e+00 &      \\ 
 264 &  2.63e+02 &    2 &           &           & displac  \\ 
 \hdashline 
     &           &    1 &  1.54e+02 &  3.00e-01 &      \\ 
     &           &    2 &  0.00e+00 &  0.00e+00 &      \\ 
 265 &  2.64e+02 &    2 &           &           & displac  \\ 
 \hdashline 
     &           &    1 &  1.35e-02 &  3.00e-01 &      \\ 
     &           &    2 &  0.00e+00 &  0.00e+00 &      \\ 
 266 &  2.65e+02 &    2 &           &           & displac  \\ 
 \hdashline 
     &           &    1 &  1.34e-02 &  3.00e-01 &      \\ 
     &           &    2 &  0.00e+00 &  0.00e+00 &      \\ 
 267 &  2.66e+02 &    2 &           &           & displac  \\ 
 \hdashline 
     &           &    1 &  1.49e+02 &  3.00e-01 &      \\ 
     &           &    2 &  0.00e+00 &  0.00e+00 &      \\ 
 268 &  2.67e+02 &    2 &           &           & displac  \\ 
 \hdashline 
     &           &    1 &  1.33e-02 &  3.00e-01 &      \\ 
     &           &    2 &  0.00e+00 &  0.00e+00 &      \\ 
 269 &  2.68e+02 &    2 &           &           & displac  \\ 
 \hdashline 
     &           &    1 &  1.32e-02 &  3.00e-01 &      \\ 
     &           &    2 &  0.00e+00 &  0.00e+00 &      \\ 
 270 &  2.69e+02 &    2 &           &           & displac  \\ 
 \hdashline 
     &           &    1 &  1.32e-02 &  3.00e-01 &      \\ 
     &           &    2 &  0.00e+00 &  0.00e+00 &      \\ 
 271 &  2.70e+02 &    2 &           &           & displac  \\ 
 \hdashline 
     &           &    1 &  1.31e-02 &  3.00e-01 &      \\ 
     &           &    2 &  2.27e-13 &  1.02e-14 &      \\ 
 272 &  2.71e+02 &    2 &           &           & forces  \\ 
 \hdashline 
     &           &    1 &  1.30e-02 &  3.00e-01 &      \\ 
     &           &    2 &  9.09e-13 &  2.03e-14 &      \\ 
 273 &  2.72e+02 &    2 &           &           & forces  \\ 
 \hdashline 
     &           &    1 &  1.30e-02 &  3.00e-01 &      \\ 
     &           &    2 &  6.82e-13 &  4.05e-14 &      \\ 
 274 &  2.73e+02 &    2 &           &           & forces  \\ 
 \hdashline 
     &           &    1 &  1.29e-02 &  3.00e-01 &      \\ 
     &           &    2 &  0.00e+00 &  3.03e-14 &      \\ 
 275 &  2.74e+02 &    2 &           &           & displac  \\ 
 \hdashline 
     &           &    1 &  1.28e-02 &  3.00e-01 &      \\ 
     &           &    2 &  0.00e+00 &  0.00e+00 &      \\ 
 276 &  2.75e+02 &    2 &           &           & displac  \\ 
 \hdashline 
     &           &    1 &  1.33e+02 &  3.00e-01 &      \\ 
     &           &    2 &  0.00e+00 &  0.00e+00 &      \\ 
 277 &  2.76e+02 &    2 &           &           & displac  \\ 
 \hdashline 
     &           &    1 &  1.27e-02 &  3.00e-01 &      \\ 
     &           &    2 &  0.00e+00 &  0.00e+00 &      \\ 
 278 &  2.77e+02 &    2 &           &           & displac  \\ 
 \hdashline 
     &           &    1 &  1.29e+02 &  3.00e-01 &      \\ 
     &           &    2 &  0.00e+00 &  0.00e+00 &      \\ 
 279 &  2.78e+02 &    2 &           &           & displac  \\ 
 \hdashline 
     &           &    1 &  1.26e-02 &  3.00e-01 &      \\ 
     &           &    2 &  0.00e+00 &  0.00e+00 &      \\ 
 280 &  2.79e+02 &    2 &           &           & displac  \\ 
 \hdashline 
     &           &    1 &  1.25e-02 &  3.00e-01 &      \\ 
     &           &    2 &  2.27e-13 &  9.88e-15 &      \\ 
 281 &  2.80e+02 &    2 &           &           & forces  \\ 
 \hdashline 
     &           &    1 &  1.24e-02 &  3.00e-01 &      \\ 
     &           &    2 &  0.00e+00 &  9.83e-15 &      \\ 
 282 &  2.81e+02 &    2 &           &           & displac  \\ 
 \hdashline 
     &           &    1 &  1.23e-02 &  3.00e-01 &      \\ 
     &           &    2 &  0.00e+00 &  0.00e+00 &      \\ 
 283 &  2.82e+02 &    2 &           &           & displac  \\ 
 \hdashline 
     &           &    1 &  1.23e-02 &  3.00e-01 &      \\ 
     &           &    2 &  0.00e+00 &  0.00e+00 &      \\ 
 284 &  2.83e+02 &    2 &           &           & displac  \\ 
 \hdashline 
     &           &    1 &  1.19e+02 &  3.00e-01 &      \\ 
     &           &    2 &  0.00e+00 &  0.00e+00 &      \\ 
 285 &  2.84e+02 &    2 &           &           & displac  \\ 
 \hdashline 
     &           &    1 &  1.21e-02 &  3.00e-01 &      \\ 
     &           &    2 &  0.00e+00 &  0.00e+00 &      \\ 
 286 &  2.85e+02 &    2 &           &           & displac  \\ 
 \hdashline 
     &           &    1 &  1.21e-02 &  3.00e-01 &      \\ 
     &           &    2 &  0.00e+00 &  0.00e+00 &      \\ 
 287 &  2.86e+02 &    2 &           &           & displac  \\ 
 \hdashline 
     &           &    1 &  1.20e-02 &  3.00e-01 &      \\ 
     &           &    2 &  0.00e+00 &  0.00e+00 &      \\ 
 288 &  2.87e+02 &    2 &           &           & displac  \\ 
 \hdashline 
     &           &    1 &  1.19e-02 &  3.00e-01 &      \\ 
     &           &    2 &  0.00e+00 &  0.00e+00 &      \\ 
 289 &  2.88e+02 &    2 &           &           & displac  \\ 
 \hdashline 
     &           &    1 &  1.18e-02 &  3.00e-01 &      \\ 
     &           &    2 &  0.00e+00 &  0.00e+00 &      \\ 
 290 &  2.89e+02 &    2 &           &           & displac  \\ 
 \hdashline 
     &           &    1 &  1.18e-02 &  3.00e-01 &      \\ 
     &           &    2 &  0.00e+00 &  0.00e+00 &      \\ 
 291 &  2.90e+02 &    2 &           &           & displac  \\ 
 \hdashline 
     &           &    1 &  1.17e-02 &  3.00e-01 &      \\ 
     &           &    2 &  0.00e+00 &  0.00e+00 &      \\ 
 292 &  2.91e+02 &    2 &           &           & displac  \\ 
 \hdashline 
     &           &    1 &  1.07e+02 &  3.00e-01 &      \\ 
     &           &    2 &  0.00e+00 &  0.00e+00 &      \\ 
 293 &  2.92e+02 &    2 &           &           & displac  \\ 
 \hdashline 
     &           &    1 &  1.15e-02 &  3.00e-01 &      \\ 
     &           &    2 &  0.00e+00 &  0.00e+00 &      \\ 
 294 &  2.93e+02 &    2 &           &           & displac  \\ 
 \hdashline 
     &           &    1 &  1.04e+02 &  3.00e-01 &      \\ 
     &           &    2 &  0.00e+00 &  0.00e+00 &      \\ 
 295 &  2.94e+02 &    2 &           &           & displac  \\ 
 \hdashline 
     &           &    1 &  1.14e-02 &  3.00e-01 &      \\ 
     &           &    2 &  0.00e+00 &  5.42e-20 &      \\ 
 296 &  2.95e+02 &    2 &           &           & displac  \\ 
 \hdashline 
     &           &    1 &  1.13e-02 &  3.00e-01 &      \\ 
     &           &    2 &  0.00e+00 &  9.33e-15 &      \\ 
 297 &  2.96e+02 &    2 &           &           & displac  \\ 
 \hdashline 
     &           &    1 &  9.90e+01 &  3.00e-01 &      \\ 
     &           &    2 &  0.00e+00 &  0.00e+00 &      \\ 
 298 &  2.97e+02 &    2 &           &           & displac  \\ 
 \hdashline 
     &           &    1 &  1.12e-02 &  3.00e-01 &      \\ 
     &           &    2 &  0.00e+00 &  0.00e+00 &      \\ 
 299 &  2.98e+02 &    2 &           &           & displac  \\ 
 \hdashline 
     &           &    1 &  1.11e-02 &  3.00e-01 &      \\ 
     &           &    2 &  0.00e+00 &  0.00e+00 &      \\ 
 300 &  2.99e+02 &    2 &           &           & displac  \\ 
 \hdashline 
     &           &    1 &  1.10e-02 &  3.00e-01 &      \\ 
     &           &    2 &  2.27e-13 &  9.21e-15 &      \\ 
 301 &  3.00e+02 &    2 &           &           & forces  \\ 
 \hdashline 
     &           &    1 &  9.30e+01 &  3.00e-01 &      \\ 
     &           &    2 &  0.00e+00 &  8.88e-16 &      \\ 
 302 &  3.01e+02 &    2 &           &           & displac  \\ 
 \hdashline 
     &           &    1 &  1.09e-02 &  3.00e-01 &      \\ 
     &           &    2 &  0.00e+00 &  0.00e+00 &      \\ 
 303 &  3.02e+02 &    2 &           &           & displac  \\ 
 \hdashline 
     &           &    1 &  1.08e-02 &  3.00e-01 &      \\ 
     &           &    2 &  0.00e+00 &  0.00e+00 &      \\ 
 304 &  3.03e+02 &    2 &           &           & displac  \\ 
 \hdashline 
     &           &    1 &  8.86e+01 &  3.00e-01 &      \\ 
     &           &    2 &  0.00e+00 &  0.00e+00 &      \\ 
 305 &  3.04e+02 &    2 &           &           & displac  \\ 
 \hdashline 
     &           &    1 &  1.06e-02 &  3.00e-01 &      \\ 
     &           &    2 &  0.00e+00 &  0.00e+00 &      \\ 
 306 &  3.05e+02 &    2 &           &           & displac  \\ 
 \hdashline 
     &           &    1 &  1.06e-02 &  3.00e-01 &      \\ 
     &           &    2 &  1.14e-13 &  4.55e-15 &      \\ 
 307 &  3.06e+02 &    2 &           &           & forces  \\ 
 \hdashline 
     &           &    1 &  1.05e-02 &  3.00e-01 &      \\ 
     &           &    2 &  1.14e-13 &  4.50e-15 &      \\ 
 308 &  3.07e+02 &    2 &           &           & forces  \\ 
 \hdashline 
     &           &    1 &  1.04e-02 &  3.00e-01 &      \\ 
     &           &    2 &  1.14e-13 &  4.50e-15 &      \\ 
 309 &  3.08e+02 &    2 &           &           & forces  \\ 
 \hdashline 
     &           &    1 &  1.03e-02 &  3.00e-01 &      \\ 
     &           &    2 &  1.14e-13 &  4.50e-15 &      \\ 
 310 &  3.09e+02 &    2 &           &           & forces  \\ 
 \hdashline 
     &           &    1 &  8.02e+01 &  3.00e-01 &      \\ 
     &           &    2 &  0.00e+00 &  4.44e-16 &      \\ 
 311 &  3.10e+02 &    2 &           &           & displac  \\ 
 \hdashline 
     &           &    1 &  1.02e-02 &  3.00e-01 &      \\ 
     &           &    2 &  0.00e+00 &  0.00e+00 &      \\ 
 312 &  3.11e+02 &    2 &           &           & displac  \\ 
 \hdashline 
     &           &    1 &  1.01e-02 &  3.00e-01 &      \\ 
     &           &    2 &  0.00e+00 &  5.42e-20 &      \\ 
 313 &  3.12e+02 &    2 &           &           & displac  \\ 
 \hdashline 
     &           &    1 &  1.00e-02 &  3.00e-01 &      \\ 
     &           &    2 &  0.00e+00 &  0.00e+00 &      \\ 
 314 &  3.13e+02 &    2 &           &           & displac  \\ 
 \hdashline 
     &           &    1 &  9.92e-03 &  3.00e-01 &      \\ 
     &           &    2 &  0.00e+00 &  0.00e+00 &      \\ 
 315 &  3.14e+02 &    2 &           &           & displac  \\ 
 \hdashline 
     &           &    1 &  7.35e+01 &  3.00e-01 &      \\ 
     &           &    2 &  0.00e+00 &  4.44e-16 &      \\ 
 316 &  3.15e+02 &    2 &           &           & displac  \\ 
 \hdashline 
     &           &    1 &  9.76e-03 &  3.00e-01 &      \\ 
     &           &    2 &  0.00e+00 &  0.00e+00 &      \\ 
 317 &  3.16e+02 &    2 &           &           & displac  \\ 
 \hdashline 
     &           &    1 &  9.68e-03 &  3.00e-01 &      \\ 
     &           &    2 &  1.14e-13 &  4.39e-15 &      \\ 
 318 &  3.17e+02 &    2 &           &           & forces  \\ 
 \hdashline 
     &           &    1 &  9.59e-03 &  3.00e-01 &      \\ 
     &           &    2 &  1.14e-13 &  4.39e-15 &      \\ 
 319 &  3.18e+02 &    2 &           &           & forces  \\ 
 \hdashline 
     &           &    1 &  9.51e-03 &  3.00e-01 &      \\ 
     &           &    2 &  1.14e-13 &  4.39e-15 &      \\ 
 320 &  3.19e+02 &    2 &           &           & forces  \\ 
 \hdashline 
     &           &    1 &  6.71e+01 &  3.00e-01 &      \\ 
     &           &    2 &  0.00e+00 &  0.00e+00 &      \\ 
 321 &  3.20e+02 &    2 &           &           & displac  \\ 
 \hdashline 
     &           &    1 &  9.35e-03 &  3.00e-01 &      \\ 
     &           &    2 &  1.14e-13 &  4.39e-15 &      \\ 
 322 &  3.21e+02 &    2 &           &           & forces  \\ 
 \hdashline 
     &           &    1 &  9.27e-03 &  3.00e-01 &      \\ 
     &           &    2 &  1.14e-13 &  4.33e-15 &      \\ 
 323 &  3.22e+02 &    2 &           &           & forces  \\ 
 \hdashline 
     &           &    1 &  9.19e-03 &  3.00e-01 &      \\ 
     &           &    2 &  1.14e-13 &  4.33e-15 &      \\ 
 324 &  3.23e+02 &    2 &           &           & forces  \\ 
 \hdashline 
     &           &    1 &  9.10e-03 &  3.00e-01 &      \\ 
     &           &    2 &  1.14e-13 &  4.33e-15 &      \\ 
 325 &  3.24e+02 &    2 &           &           & forces  \\ 
 \hdashline 
     &           &    1 &  9.02e-03 &  3.00e-01 &      \\ 
     &           &    2 &  1.14e-13 &  4.33e-15 &      \\ 
 326 &  3.25e+02 &    2 &           &           & forces  \\ 
 \hdashline 
     &           &    1 &  8.94e-03 &  3.00e-01 &      \\ 
     &           &    2 &  1.14e-13 &  4.33e-15 &      \\ 
 327 &  3.26e+02 &    2 &           &           & forces  \\ 
 \hdashline 
     &           &    1 &  5.85e+01 &  3.00e-01 &      \\ 
     &           &    2 &  0.00e+00 &  8.88e-16 &      \\ 
 328 &  3.27e+02 &    2 &           &           & displac  \\ 
 \hdashline 
     &           &    1 &  8.77e-03 &  3.00e-01 &      \\ 
     &           &    2 &  0.00e+00 &  5.42e-20 &      \\ 
 329 &  3.28e+02 &    2 &           &           & displac  \\ 
 \hdashline 
     &           &    1 &  8.69e-03 &  3.00e-01 &      \\ 
     &           &    2 &  0.00e+00 &  0.00e+00 &      \\ 
 330 &  3.29e+02 &    2 &           &           & displac  \\ 
 \hdashline 
     &           &    1 &  8.60e-03 &  3.00e-01 &      \\ 
     &           &    2 &  1.14e-13 &  4.27e-15 &      \\ 
 331 &  3.30e+02 &    2 &           &           & forces  \\ 
 \hdashline 
     &           &    1 &  8.52e-03 &  3.00e-01 &      \\ 
     &           &    2 &  1.14e-13 &  4.27e-15 &      \\ 
 332 &  3.31e+02 &    2 &           &           & forces  \\ 
 \hdashline 
     &           &    1 &  8.43e-03 &  3.00e-01 &      \\ 
     &           &    2 &  1.14e-13 &  4.33e-15 &      \\ 
 333 &  3.32e+02 &    2 &           &           & forces  \\ 
 \hdashline 
     &           &    1 &  8.35e-03 &  3.00e-01 &      \\ 
     &           &    2 &  1.14e-13 &  4.22e-15 &      \\ 
 334 &  3.33e+02 &    2 &           &           & forces  \\ 
 \hdashline 
     &           &    1 &  8.26e-03 &  3.00e-01 &      \\ 
     &           &    2 &  1.14e-13 &  8.49e-15 &      \\ 
 335 &  3.34e+02 &    2 &           &           & forces  \\ 
 \hdashline 
     &           &    1 &  8.18e-03 &  3.00e-01 &      \\ 
     &           &    2 &  1.14e-13 &  4.22e-15 &      \\ 
 336 &  3.35e+02 &    2 &           &           & forces  \\ 
 \hdashline 
     &           &    1 &  8.09e-03 &  3.00e-01 &      \\ 
     &           &    2 &  1.14e-13 &  4.22e-15 &      \\ 
 337 &  3.36e+02 &    2 &           &           & forces  \\ 
 \hdashline 
     &           &    1 &  8.01e-03 &  3.00e-01 &      \\ 
     &           &    2 &  1.14e-13 &  4.22e-15 &      \\ 
 338 &  3.37e+02 &    2 &           &           & forces  \\ 
 \hdashline 
     &           &    1 &  4.62e+01 &  3.00e-01 &      \\ 
     &           &    2 &  0.00e+00 &  6.66e-16 &      \\ 
 339 &  3.38e+02 &    2 &           &           & displac  \\ 
 \hdashline 
     &           &    1 &  7.84e-03 &  3.00e-01 &      \\ 
     &           &    2 &  0.00e+00 &  0.00e+00 &      \\ 
 340 &  3.39e+02 &    2 &           &           & displac  \\ 
 \hdashline 
     &           &    1 &  7.75e-03 &  3.00e-01 &      \\ 
     &           &    2 &  0.00e+00 &  5.55e-17 &      \\ 
 341 &  3.40e+02 &    2 &           &           & displac  \\ 
 \hdashline 
     &           &    1 &  7.66e-03 &  3.00e-01 &      \\ 
     &           &    2 &  0.00e+00 &  0.00e+00 &      \\ 
 342 &  3.41e+02 &    2 &           &           & displac  \\ 
 \hdashline 
     &           &    1 &  7.58e-03 &  3.00e-01 &      \\ 
     &           &    2 &  0.00e+00 &  0.00e+00 &      \\ 
 343 &  3.42e+02 &    2 &           &           & displac  \\ 
 \hdashline 
     &           &    1 &  7.49e-03 &  3.00e-01 &      \\ 
     &           &    2 &  0.00e+00 &  0.00e+00 &      \\ 
 344 &  3.43e+02 &    2 &           &           & displac  \\ 
 \hdashline 
     &           &    1 &  7.40e-03 &  3.00e-01 &      \\ 
     &           &    2 &  0.00e+00 &  0.00e+00 &      \\ 
 345 &  3.44e+02 &    2 &           &           & displac  \\ 
 \hdashline 
     &           &    1 &  7.32e-03 &  3.00e-01 &      \\ 
     &           &    2 &  0.00e+00 &  0.00e+00 &      \\ 
 346 &  3.45e+02 &    2 &           &           & displac  \\ 
 \hdashline 
     &           &    1 &  7.23e-03 &  3.00e-01 &      \\ 
     &           &    2 &  1.14e-13 &  4.16e-15 &      \\ 
 347 &  3.46e+02 &    2 &           &           & forces  \\ 
 \hdashline 
     &           &    1 &  7.14e-03 &  3.00e-01 &      \\ 
     &           &    2 &  1.14e-13 &  4.16e-15 &      \\ 
 348 &  3.47e+02 &    2 &           &           & forces  \\ 
 \hdashline 
     &           &    1 &  3.62e+01 &  3.00e-01 &      \\ 
     &           &    2 &  0.00e+00 &  1.33e-15 &      \\ 
 349 &  3.48e+02 &    2 &           &           & displac  \\ 
 \hdashline 
     &           &    1 &  6.97e-03 &  3.00e-01 &      \\ 
     &           &    2 &  1.14e-13 &  4.11e-15 &      \\ 
 350 &  3.49e+02 &    2 &           &           & forces  \\ 
 \hdashline 
     &           &    1 &  6.88e-03 &  3.00e-01 &      \\ 
     &           &    2 &  1.14e-13 &  4.11e-15 &      \\ 
 351 &  3.50e+02 &    2 &           &           & forces  \\ 
 \hdashline 
     &           &    1 &  6.79e-03 &  3.00e-01 &      \\ 
     &           &    2 &  1.14e-13 &  4.16e-15 &      \\ 
 352 &  3.51e+02 &    2 &           &           & forces  \\ 
 \hdashline 
     &           &    1 &  3.26e+01 &  3.00e-01 &      \\ 
     &           &    2 &  0.00e+00 &  1.33e-15 &      \\ 
 353 &  3.52e+02 &    2 &           &           & displac  \\ 
 \hdashline 
     &           &    1 &  6.61e-03 &  3.00e-01 &      \\ 
     &           &    2 &  1.14e-13 &  4.11e-15 &      \\ 
 354 &  3.53e+02 &    2 &           &           & forces  \\ 
 \hdashline 
     &           &    1 &  3.08e+01 &  3.00e-01 &      \\ 
     &           &    2 &  0.00e+00 &  1.55e-15 &      \\ 
 355 &  3.54e+02 &    2 &           &           & displac  \\ 
 \hdashline 
     &           &    1 &  6.43e-03 &  3.00e-01 &      \\ 
     &           &    2 &  1.14e-13 &  4.00e-15 &      \\ 
 356 &  3.55e+02 &    2 &           &           & forces  \\ 
 \hdashline 
     &           &    1 &  6.34e-03 &  3.00e-01 &      \\ 
     &           &    2 &  1.14e-13 &  4.05e-15 &      \\ 
 357 &  3.56e+02 &    2 &           &           & forces  \\ 
 \hdashline 
     &           &    1 &  6.25e-03 &  3.00e-01 &      \\ 
     &           &    2 &  1.14e-13 &  8.10e-15 &      \\ 
 358 &  3.57e+02 &    2 &           &           & forces  \\ 
 \hdashline 
     &           &    1 &  2.74e+01 &  3.00e-01 &      \\ 
     &           &    2 &  0.00e+00 &  1.78e-15 &      \\ 
 359 &  3.58e+02 &    2 &           &           & displac  \\ 
 \hdashline 
     &           &    1 &  6.08e-03 &  3.00e-01 &      \\ 
     &           &    2 &  1.14e-13 &  4.05e-15 &      \\ 
 360 &  3.59e+02 &    2 &           &           & forces  \\ 
 \hdashline 
     &           &    1 &  5.99e-03 &  3.00e-01 &      \\ 
     &           &    2 &  1.14e-13 &  4.05e-15 &      \\ 
 361 &  3.60e+02 &    2 &           &           & forces  \\ 
 \hdashline 
     &           &    1 &  5.90e-03 &  3.00e-01 &      \\ 
     &           &    2 &  1.14e-13 &  4.05e-15 &      \\ 
 362 &  3.61e+02 &    2 &           &           & forces  \\ 
 \hdashline 
     &           &    1 &  5.81e-03 &  3.00e-01 &      \\ 
     &           &    2 &  1.14e-13 &  4.05e-15 &      \\ 
 363 &  3.62e+02 &    2 &           &           & forces  \\ 
 \hdashline 
     &           &    1 &  2.35e+01 &  3.00e-01 &      \\ 
     &           &    2 &  1.14e-13 &  2.33e-15 &      \\ 
 364 &  3.63e+02 &    2 &           &           & forces  \\ 
 \hdashline 
     &           &    1 &  5.62e-03 &  3.00e-01 &      \\ 
     &           &    2 &  1.14e-13 &  8.05e-15 &      \\ 
 365 &  3.64e+02 &    2 &           &           & forces  \\ 
 \hdashline 
     &           &    1 &  2.20e+01 &  3.00e-01 &      \\ 
     &           &    2 &  5.68e-14 &  2.55e-15 &      \\ 
 366 &  3.65e+02 &    2 &           &           & forces  \\ 
 \hdashline 
     &           &    1 &  5.44e-03 &  3.00e-01 &      \\ 
     &           &    2 &  1.14e-13 &  4.00e-15 &      \\ 
 367 &  3.66e+02 &    2 &           &           & forces  \\ 
 \hdashline 
     &           &    1 &  5.35e-03 &  3.00e-01 &      \\ 
     &           &    2 &  1.14e-13 &  4.00e-15 &      \\ 
 368 &  3.67e+02 &    2 &           &           & forces  \\ 
 \hdashline 
     &           &    1 &  5.26e-03 &  3.00e-01 &      \\ 
     &           &    2 &  1.71e-13 &  5.94e-15 &      \\ 
 369 &  3.68e+02 &    2 &           &           & forces  \\ 
 \hdashline 
     &           &    1 &  5.17e-03 &  3.00e-01 &      \\ 
     &           &    2 &  1.71e-13 &  6.00e-15 &      \\ 
 370 &  3.69e+02 &    2 &           &           & forces  \\ 
 \hdashline 
     &           &    1 &  1.85e+01 &  3.00e-01 &      \\ 
     &           &    2 &  1.71e-13 &  5.11e-15 &      \\ 
 371 &  3.70e+02 &    2 &           &           & forces  \\ 
 \hdashline 
     &           &    1 &  4.99e-03 &  3.00e-01 &      \\ 
     &           &    2 &  1.71e-13 &  6.00e-15 &      \\ 
 372 &  3.71e+02 &    2 &           &           & forces  \\ 
 \hdashline 
     &           &    1 &  4.89e-03 &  3.00e-01 &      \\ 
     &           &    2 &  1.71e-13 &  6.00e-15 &      \\ 
 373 &  3.72e+02 &    2 &           &           & forces  \\ 
 \hdashline 
     &           &    1 &  4.80e-03 &  3.00e-01 &      \\ 
     &           &    2 &  1.71e-13 &  6.00e-15 &      \\ 
 374 &  3.73e+02 &    2 &           &           & forces  \\ 
 \hdashline 
     &           &    1 &  1.59e+01 &  3.00e-01 &      \\ 
     &           &    2 &  2.27e-13 &  7.11e-15 &      \\ 
 375 &  3.74e+02 &    2 &           &           & forces  \\ 
 \hdashline 
     &           &    1 &  4.62e-03 &  3.00e-01 &      \\ 
     &           &    2 &  5.68e-14 &  9.94e-15 &      \\ 
 376 &  3.75e+02 &    2 &           &           & forces  \\ 
 \hdashline 
     &           &    1 &  4.53e-03 &  3.00e-01 &      \\ 
     &           &    2 &  5.68e-14 &  2.00e-15 &      \\ 
 377 &  3.76e+02 &    2 &           &           & forces  \\ 
 \hdashline 
     &           &    1 &  4.43e-03 &  3.00e-01 &      \\ 
     &           &    2 &  5.68e-14 &  2.00e-15 &      \\ 
 378 &  3.77e+02 &    2 &           &           & forces  \\ 
 \hdashline 
     &           &    1 &  1.34e+01 &  3.00e-01 &      \\ 
     &           &    2 &  1.14e-13 &  3.55e-15 &      \\ 
 379 &  3.78e+02 &    2 &           &           & forces  \\ 
 \hdashline 
     &           &    1 &  4.25e-03 &  3.00e-01 &      \\ 
     &           &    2 &  1.71e-13 &  5.94e-15 &      \\ 
 380 &  3.79e+02 &    2 &           &           & forces  \\ 
 \hdashline 
     &           &    1 &  4.16e-03 &  3.00e-01 &      \\ 
     &           &    2 &  1.71e-13 &  5.94e-15 &      \\ 
 381 &  3.80e+02 &    2 &           &           & forces  \\ 
 \hdashline 
     &           &    1 &  4.06e-03 &  3.00e-01 &      \\ 
     &           &    2 &  1.71e-13 &  7.88e-15 &      \\ 
 382 &  3.81e+02 &    2 &           &           & forces  \\ 
 \hdashline 
     &           &    1 &  1.12e+01 &  3.00e-01 &      \\ 
     &           &    2 &  1.71e-13 &  1.98e-14 &      \\ 
 383 &  3.82e+02 &    2 &           &           & forces  \\ 
 \hdashline 
     &           &    1 &  3.88e-03 &  3.00e-01 &      \\ 
     &           &    2 &  2.27e-13 &  7.88e-15 &      \\ 
 384 &  3.83e+02 &    2 &           &           & forces  \\ 
 \hdashline 
     &           &    1 &  1.02e+01 &  3.00e-01 &      \\ 
     &           &    2 &  5.68e-14 &  4.45e-14 &      \\ 
 385 &  3.84e+02 &    2 &           &           & forces  \\ 
 \hdashline 
     &           &    1 &  3.69e-03 &  3.00e-01 &      \\ 
     &           &    2 &  5.68e-14 &  1.94e-15 &      \\ 
 386 &  3.85e+02 &    2 &           &           & forces  \\ 
 \hdashline 
     &           &    1 &  9.22e+00 &  3.00e-01 &      \\ 
     &           &    2 &  2.84e-13 &  3.19e-14 &      \\ 
 387 &  3.86e+02 &    2 &           &           & forces  \\ 
 \hdashline 
     &           &    1 &  3.50e-03 &  3.00e-01 &      \\ 
     &           &    2 &  1.14e-13 &  9.83e-15 &      \\ 
 388 &  3.87e+02 &    2 &           &           & forces  \\ 
 \hdashline 
     &           &    1 &  3.41e-03 &  3.00e-01 &      \\ 
     &           &    2 &  1.14e-13 &  3.94e-15 &      \\ 
 389 &  3.88e+02 &    2 &           &           & forces  \\ 
 \hdashline 
     &           &    1 &  3.32e-03 &  3.00e-01 &      \\ 
     &           &    2 &  1.71e-13 &  5.94e-15 &      \\ 
 390 &  3.89e+02 &    2 &           &           & forces  \\ 
 \hdashline 
     &           &    1 &  3.22e-03 &  3.00e-01 &      \\ 
     &           &    2 &  1.71e-13 &  5.88e-15 &      \\ 
 391 &  3.90e+02 &    2 &           &           & forces  \\ 
 \hdashline 
     &           &    1 &  3.13e-03 &  3.00e-01 &      \\ 
     &           &    2 &  1.71e-13 &  5.88e-15 &      \\ 
 392 &  3.91e+02 &    2 &           &           & forces  \\ 
 \hdashline 
     &           &    1 &  3.04e-03 &  3.00e-01 &      \\ 
     &           &    2 &  1.71e-13 &  5.88e-15 &      \\ 
 393 &  3.92e+02 &    2 &           &           & forces  \\ 
 \hdashline 
     &           &    1 &  2.94e-03 &  3.00e-01 &      \\ 
     &           &    2 &  1.71e-13 &  5.88e-15 &      \\ 
 394 &  3.93e+02 &    2 &           &           & forces  \\ 
 \hdashline 
     &           &    1 &  5.78e+00 &  3.00e-01 &      \\ 
     &           &    2 &  1.14e-13 &  1.73e-14 &      \\ 
 395 &  3.94e+02 &    2 &           &           & forces  \\ 
 \hdashline 
     &           &    1 &  2.75e-03 &  3.00e-01 &      \\ 
     &           &    2 &  1.14e-13 &  3.94e-15 &      \\ 
 396 &  3.95e+02 &    2 &           &           & forces  \\ 
 \hdashline 
     &           &    1 &  2.66e-03 &  3.00e-01 &      \\ 
     &           &    2 &  1.14e-13 &  3.89e-15 &      \\ 
 397 &  3.96e+02 &    2 &           &           & forces  \\ 
 \hdashline 
     &           &    1 &  2.56e-03 &  3.00e-01 &      \\ 
     &           &    2 &  1.14e-13 &  3.89e-15 &      \\ 
 398 &  3.97e+02 &    2 &           &           & forces  \\ 
 \hdashline 
     &           &    1 &  2.47e-03 &  3.00e-01 &      \\ 
     &           &    2 &  1.14e-13 &  3.83e-15 &      \\ 
 399 &  3.98e+02 &    2 &           &           & forces  \\ 
 \hdashline 
     &           &    1 &  2.38e-03 &  3.00e-01 &      \\ 
     &           &    2 &  1.14e-13 &  3.83e-15 &      \\ 
 400 &  3.99e+02 &    2 &           &           & forces  \\ 
 \hdashline 
     &           &    1 &  2.28e-03 &  3.00e-01 &      \\ 
     &           &    2 &  8.53e-14 &  2.94e-15 &      \\ 
 401 &  4.00e+02 &    2 &           &           & forces  \\ 
 \hdashline 
     &           &    1 &  3.42e+00 &  3.00e-01 &      \\ 
     &           &    2 &  1.42e-13 &  4.77e-15 &      \\ 
 402 &  4.01e+02 &    2 &           &           & forces  \\ 
 \hdashline 
     &           &    1 &  2.09e-03 &  3.00e-01 &      \\ 
     &           &    2 &  1.14e-13 &  3.89e-15 &      \\ 
 403 &  4.02e+02 &    2 &           &           & forces  \\ 
 \hdashline 
     &           &    1 &  2.00e-03 &  3.00e-01 &      \\ 
     &           &    2 &  1.14e-13 &  3.89e-15 &      \\ 
 404 &  4.03e+02 &    2 &           &           & forces  \\ 
 \hdashline 
     &           &    1 &  1.90e-03 &  3.00e-01 &      \\ 
     &           &    2 &  1.14e-13 &  3.89e-15 &      \\ 
 405 &  4.04e+02 &    2 &           &           & forces  \\ 
 \hdashline 
     &           &    1 &  1.81e-03 &  3.00e-01 &      \\ 
     &           &    2 &  1.14e-13 &  3.94e-15 &      \\ 
 406 &  4.05e+02 &    2 &           &           & forces  \\ 
 \hdashline 
     &           &    1 &  1.71e-03 &  3.00e-01 &      \\ 
     &           &    2 &  1.14e-13 &  3.89e-15 &      \\ 
 407 &  4.06e+02 &    2 &           &           & forces  \\ 
 \hdashline 
     &           &    1 &  1.62e-03 &  3.00e-01 &      \\ 
     &           &    2 &  1.14e-13 &  3.89e-15 &      \\ 
 408 &  4.07e+02 &    2 &           &           & forces  \\ 
 \hdashline 
     &           &    1 &  1.68e+00 &  3.00e-01 &      \\ 
     &           &    2 &  1.42e-13 &  4.77e-15 &      \\ 
 409 &  4.08e+02 &    2 &           &           & forces  \\ 
 \hdashline 
     &           &    1 &  1.43e-03 &  3.00e-01 &      \\ 
     &           &    2 &  1.42e-13 &  4.83e-15 &      \\ 
 410 &  4.09e+02 &    2 &           &           & forces  \\ 
 \hdashline 
     &           &    1 &  1.33e-03 &  3.00e-01 &      \\ 
     &           &    2 &  1.42e-13 &  4.83e-15 &      \\ 
 411 &  4.10e+02 &    2 &           &           & forces  \\ 
 \hdashline 
     &           &    1 &  1.24e-03 &  3.00e-01 &      \\ 
     &           &    2 &  1.56e-13 &  5.33e-15 &      \\ 
 412 &  4.11e+02 &    2 &           &           & forces  \\ 
 \hdashline 
     &           &    1 &  1.14e-03 &  3.00e-01 &      \\ 
     &           &    2 &  1.42e-13 &  4.83e-15 &      \\ 
 413 &  4.12e+02 &    2 &           &           & forces  \\ 
 \hdashline 
     &           &    1 &  1.05e-03 &  3.00e-01 &      \\ 
     &           &    2 &  1.42e-13 &  4.83e-15 &      \\ 
 414 &  4.13e+02 &    2 &           &           & forces  \\ 
 \hdashline 
     &           &    1 &  6.72e-01 &  3.00e-01 &      \\ 
     &           &    2 &  1.56e-13 &  5.21e-15 &      \\ 
 415 &  4.14e+02 &    2 &           &           & forces  \\ 
 \hdashline 
     &           &    1 &  8.58e-04 &  3.00e-01 &      \\ 
     &           &    2 &  1.56e-13 &  5.33e-15 &      \\ 
 416 &  4.15e+02 &    2 &           &           & forces  \\ 
 \hdashline 
     &           &    1 &  7.62e-04 &  3.00e-01 &      \\ 
     &           &    2 &  1.56e-13 &  5.33e-15 &      \\ 
 417 &  4.16e+02 &    2 &           &           & forces  \\ 
 \hdashline 
     &           &    1 &  6.67e-04 &  3.00e-01 &      \\ 
     &           &    2 &  1.56e-13 &  5.33e-15 &      \\ 
 418 &  4.17e+02 &    2 &           &           & forces  \\ 
 \hdashline 
     &           &    1 &  5.72e-04 &  3.00e-01 &      \\ 
     &           &    2 &  1.56e-13 &  5.33e-15 &      \\ 
 419 &  4.18e+02 &    2 &           &           & forces  \\ 
 \hdashline 
     &           &    1 &  1.79e-01 &  3.00e-01 &      \\ 
     &           &    2 &  1.56e-13 &  5.44e-15 &      \\ 
 420 &  4.19e+02 &    2 &           &           & forces  \\ 
 \hdashline 
     &           &    1 &  3.81e-04 &  3.00e-01 &      \\ 
     &           &    2 &  1.56e-13 &  5.33e-15 &      \\ 
 421 &  4.20e+02 &    2 &           &           & forces  \\ 
 \hdashline 
     &           &    1 &  2.86e-04 &  3.00e-01 &      \\ 
     &           &    2 &  1.56e-13 &  5.33e-15 &      \\ 
 422 &  4.21e+02 &    2 &           &           & forces  \\ 
 \hdashline 
     &           &    1 &  3.44e-02 &  3.00e-01 &      \\ 
     &           &    2 &  1.56e-13 &  5.33e-15 &      \\ 
 423 &  4.22e+02 &    2 &           &           & forces  \\ 
 \hdashline 
     &           &    1 &  9.53e-05 &  3.00e-01 &      \\ 
     &           &    2 &  1.58e-13 &  5.38e-15 &      \\ 
 424 &  4.23e+02 &    2 &           &           & forces  \\ 
 \hdashline 
     &           &    1 &  6.98e-04 &  3.00e-01 &      \\ 
     &           &    2 &  1.58e-13 &  5.38e-15 &      \\ 
 425 &  4.24e+02 &    2 &           &           & forces  \\ 
 \hdashline 
     &           &    1 &  9.52e-03 &  3.00e-01 &      \\ 
     &           &    2 &  1.58e-13 &  5.38e-15 &      \\ 
 426 &  4.25e+02 &    2 &           &           & forces  \\ 
 \hdashline 
     &           &    1 &  1.90e-02 &  3.00e-01 &      \\ 
     &           &    2 &  1.56e-13 &  5.38e-15 &      \\ 
 427 &  4.26e+02 &    2 &           &           & forces  \\ 
 \hdashline 
     &           &    1 &  2.86e-02 &  3.00e-01 &      \\ 
     &           &    2 &  1.56e-13 &  5.38e-15 &      \\ 
 428 &  4.27e+02 &    2 &           &           & forces  \\ 
 \hdashline 
     &           &    1 &  3.81e-02 &  3.00e-01 &      \\ 
     &           &    2 &  1.56e-13 &  5.38e-15 &      \\ 
 429 &  4.28e+02 &    2 &           &           & forces  \\ 
 \hdashline 
     &           &    1 &  4.76e-02 &  3.00e-01 &      \\ 
     &           &    2 &  1.56e-13 &  5.44e-15 &      \\ 
 430 &  4.29e+02 &    2 &           &           & forces  \\ 
 \hdashline 
     &           &    1 &  5.71e-02 &  3.00e-01 &      \\ 
     &           &    2 &  1.56e-13 &  5.50e-15 &      \\ 
 431 &  4.30e+02 &    2 &           &           & forces  \\ 
 \hdashline 
     &           &    1 &  6.67e-02 &  3.00e-01 &      \\ 
     &           &    2 &  1.56e-13 &  5.61e-15 &      \\ 
 432 &  4.31e+02 &    2 &           &           & forces  \\ 
 \hdashline 
     &           &    1 &  7.62e-02 &  3.00e-01 &      \\ 
     &           &    2 &  1.56e-13 &  5.66e-15 &      \\ 
 433 &  4.32e+02 &    2 &           &           & forces  \\ 
 \hdashline 
     &           &    1 &  8.57e-02 &  3.00e-01 &      \\ 
     &           &    2 &  1.56e-13 &  5.77e-15 &      \\ 
 434 &  4.33e+02 &    2 &           &           & forces  \\ 
 \hdashline 
     &           &    1 &  9.52e-02 &  3.00e-01 &      \\ 
     &           &    2 &  1.56e-13 &  5.94e-15 &      \\ 
 435 &  4.34e+02 &    2 &           &           & forces  \\ 
 \hdashline 
     &           &    1 &  1.05e-01 &  3.00e-01 &      \\ 
     &           &    2 &  1.56e-13 &  6.05e-15 &      \\ 
 436 &  4.35e+02 &    2 &           &           & forces  \\ 
 \hdashline 
     &           &    1 &  1.14e-01 &  3.00e-01 &      \\ 
     &           &    2 &  1.56e-13 &  6.22e-15 &      \\ 
 437 &  4.36e+02 &    2 &           &           & forces  \\ 
 \hdashline 
     &           &    1 &  1.24e-01 &  3.00e-01 &      \\ 
     &           &    2 &  1.56e-13 &  6.44e-15 &      \\ 
 438 &  4.37e+02 &    2 &           &           & forces  \\ 
 \hdashline 
     &           &    1 &  1.33e-01 &  3.00e-01 &      \\ 
     &           &    2 &  1.56e-13 &  6.66e-15 &      \\ 
 439 &  4.38e+02 &    2 &           &           & forces  \\ 
 \hdashline 
     &           &    1 &  1.43e-01 &  3.00e-01 &      \\ 
     &           &    2 &  1.56e-13 &  6.94e-15 &      \\ 
 440 &  4.39e+02 &    2 &           &           & forces  \\ 
 \hdashline 
     &           &    1 &  1.52e-01 &  3.00e-01 &      \\ 
     &           &    2 &  1.42e-13 &  7.22e-15 &      \\ 
 441 &  4.40e+02 &    2 &           &           & forces  \\ 
 \hdashline 
     &           &    1 &  1.62e-01 &  3.00e-01 &      \\ 
     &           &    2 &  1.42e-13 &  6.88e-15 &      \\ 
 442 &  4.41e+02 &    2 &           &           & forces  \\ 
 \hdashline 
     &           &    1 &  1.71e-01 &  3.00e-01 &      \\ 
     &           &    2 &  1.42e-13 &  7.27e-15 &      \\ 
 443 &  4.42e+02 &    2 &           &           & forces  \\ 
 \hdashline 
     &           &    1 &  1.81e-01 &  3.00e-01 &      \\ 
     &           &    2 &  1.42e-13 &  7.77e-15 &      \\ 
 444 &  4.43e+02 &    2 &           &           & forces  \\ 
 \hdashline 
     &           &    1 &  1.90e-01 &  3.00e-01 &      \\ 
     &           &    2 &  1.42e-13 &  8.33e-15 &      \\ 
 445 &  4.44e+02 &    2 &           &           & forces  \\ 
 \hdashline 
     &           &    1 &  2.00e-01 &  3.00e-01 &      \\ 
     &           &    2 &  1.42e-13 &  8.99e-15 &      \\ 
 446 &  4.45e+02 &    2 &           &           & forces  \\ 
 \hdashline 
     &           &    1 &  2.09e-01 &  3.00e-01 &      \\ 
     &           &    2 &  1.42e-13 &  9.83e-15 &      \\ 
 447 &  4.46e+02 &    2 &           &           & forces  \\ 
 \hdashline 
     &           &    1 &  2.19e-01 &  3.00e-01 &      \\ 
     &           &    2 &  1.14e-13 &  1.08e-14 &      \\ 
 448 &  4.47e+02 &    2 &           &           & forces  \\ 
 \hdashline 
     &           &    1 &  2.28e-01 &  3.00e-01 &      \\ 
     &           &    2 &  8.53e-14 &  9.77e-15 &      \\ 
 449 &  4.48e+02 &    2 &           &           & forces  \\ 
 \hdashline 
     &           &    1 &  2.37e-01 &  3.00e-01 &      \\ 
     &           &    2 &  8.53e-14 &  8.44e-15 &      \\ 
 450 &  4.49e+02 &    2 &           &           & forces  \\ 
 \hdashline 
     &           &    1 &  2.47e-01 &  3.00e-01 &      \\ 
     &           &    2 &  5.68e-14 &  1.00e-14 &      \\ 
 451 &  4.50e+02 &    2 &           &           & forces  \\ 
 \hdashline 
     &           &    1 &  2.56e-01 &  3.00e-01 &      \\ 
     &           &    2 &  5.68e-14 &  8.32e-15 &      \\ 
 452 &  4.51e+02 &    2 &           &           & forces  \\ 
 \hdashline 
     &           &    1 &  2.66e-01 &  3.00e-01 &      \\ 
     &           &    2 &  2.84e-14 &  1.11e-14 &      \\ 
 453 &  4.52e+02 &    2 &           &           & forces  \\ 
 \hdashline 
     &           &    1 &  2.75e-01 &  3.00e-01 &      \\ 
     &           &    2 &  0.00e+00 &  8.49e-15 &      \\ 
 454 &  4.53e+02 &    2 &           &           & displac  \\ 
 \hdashline 
     &           &    1 &  2.85e-01 &  3.00e-01 &      \\ 
     &           &    2 &  0.00e+00 &  0.00e+00 &      \\ 
 455 &  4.54e+02 &    2 &           &           & displac  \\ 
 \hdashline 
     &           &    1 &  2.94e-01 &  3.00e-01 &      \\ 
     &           &    2 &  0.00e+00 &  1.39e-17 &      \\ 
 456 &  4.55e+02 &    2 &           &           & displac  \\ 
 \hdashline 
     &           &    1 &  3.03e-01 &  3.00e-01 &      \\ 
     &           &    2 &  0.00e+00 &  0.00e+00 &      \\ 
 457 &  4.56e+02 &    2 &           &           & displac  \\ 
 \hdashline 
     &           &    1 &  3.13e-01 &  3.00e-01 &      \\ 
     &           &    2 &  0.00e+00 &  6.94e-18 &      \\ 
 458 &  4.57e+02 &    2 &           &           & displac  \\ 
 \hdashline 
     &           &    1 &  3.22e-01 &  3.00e-01 &      \\ 
     &           &    2 &  0.00e+00 &  0.00e+00 &      \\ 
 459 &  4.58e+02 &    2 &           &           & displac  \\ 
 \hdashline 
     &           &    1 &  3.31e-01 &  3.00e-01 &      \\ 
     &           &    2 &  0.00e+00 &  0.00e+00 &      \\ 
 460 &  4.59e+02 &    2 &           &           & displac  \\ 
 \hdashline 
     &           &    1 &  3.41e-01 &  3.00e-01 &      \\ 
     &           &    2 &  0.00e+00 &  0.00e+00 &      \\ 
 461 &  4.60e+02 &    2 &           &           & displac  \\ 
 \hdashline 
     &           &    1 &  3.50e-01 &  3.00e-01 &      \\ 
     &           &    2 &  0.00e+00 &  0.00e+00 &      \\ 
 462 &  4.61e+02 &    2 &           &           & displac  \\ 
 \hdashline 
     &           &    1 &  3.60e-01 &  3.00e-01 &      \\ 
     &           &    2 &  0.00e+00 &  0.00e+00 &      \\ 
 463 &  4.62e+02 &    2 &           &           & displac  \\ 
 \hdashline 
     &           &    1 &  3.69e-01 &  3.00e-01 &      \\ 
     &           &    2 &  0.00e+00 &  0.00e+00 &      \\ 
 464 &  4.63e+02 &    2 &           &           & displac  \\ 
 \hdashline 
     &           &    1 &  3.78e-01 &  3.00e-01 &      \\ 
     &           &    2 &  0.00e+00 &  0.00e+00 &      \\ 
 465 &  4.64e+02 &    2 &           &           & displac  \\ 
 \hdashline 
     &           &    1 &  3.87e-01 &  3.00e-01 &      \\ 
     &           &    2 &  0.00e+00 &  1.73e-18 &      \\ 
 466 &  4.65e+02 &    2 &           &           & displac  \\ 
 \hdashline 
     &           &    1 &  3.97e-01 &  3.00e-01 &      \\ 
     &           &    2 &  0.00e+00 &  0.00e+00 &      \\ 
 467 &  4.66e+02 &    2 &           &           & displac  \\ 
 \hdashline 
     &           &    1 &  4.06e-01 &  3.00e-01 &      \\ 
     &           &    2 &  0.00e+00 &  0.00e+00 &      \\ 
 468 &  4.67e+02 &    2 &           &           & displac  \\ 
 \hdashline 
     &           &    1 &  4.15e-01 &  3.00e-01 &      \\ 
     &           &    2 &  0.00e+00 &  0.00e+00 &      \\ 
 469 &  4.68e+02 &    2 &           &           & displac  \\ 
 \hdashline 
     &           &    1 &  4.25e-01 &  3.00e-01 &      \\ 
     &           &    2 &  0.00e+00 &  0.00e+00 &      \\ 
 470 &  4.69e+02 &    2 &           &           & displac  \\ 
 \hdashline 
     &           &    1 &  4.34e-01 &  3.00e-01 &      \\ 
     &           &    2 &  0.00e+00 &  0.00e+00 &      \\ 
 471 &  4.70e+02 &    2 &           &           & displac  \\ 
 \hdashline 
     &           &    1 &  4.43e-01 &  3.00e-01 &      \\ 
     &           &    2 &  0.00e+00 &  0.00e+00 &      \\ 
 472 &  4.71e+02 &    2 &           &           & displac  \\ 
 \hdashline 
     &           &    1 &  4.52e-01 &  3.00e-01 &      \\ 
     &           &    2 &  0.00e+00 &  0.00e+00 &      \\ 
 473 &  4.72e+02 &    2 &           &           & displac  \\ 
 \hdashline 
     &           &    1 &  4.62e-01 &  3.00e-01 &      \\ 
     &           &    2 &  0.00e+00 &  0.00e+00 &      \\ 
 474 &  4.73e+02 &    2 &           &           & displac  \\ 
 \hdashline 
     &           &    1 &  4.71e-01 &  3.00e-01 &      \\ 
     &           &    2 &  0.00e+00 &  0.00e+00 &      \\ 
 475 &  4.74e+02 &    2 &           &           & displac  \\ 
 \hdashline 
     &           &    1 &  4.80e-01 &  3.00e-01 &      \\ 
     &           &    2 &  0.00e+00 &  0.00e+00 &      \\ 
 476 &  4.75e+02 &    2 &           &           & displac  \\ 
 \hdashline 
     &           &    1 &  4.89e-01 &  3.00e-01 &      \\ 
     &           &    2 &  0.00e+00 &  0.00e+00 &      \\ 
 477 &  4.76e+02 &    2 &           &           & displac  \\ 
 \hdashline 
     &           &    1 &  4.98e-01 &  3.00e-01 &      \\ 
     &           &    2 &  0.00e+00 &  0.00e+00 &      \\ 
 478 &  4.77e+02 &    2 &           &           & displac  \\ 
 \hdashline 
     &           &    1 &  5.07e-01 &  3.00e-01 &      \\ 
     &           &    2 &  0.00e+00 &  0.00e+00 &      \\ 
 479 &  4.78e+02 &    2 &           &           & displac  \\ 
 \hdashline 
     &           &    1 &  5.17e-01 &  3.00e-01 &      \\ 
     &           &    2 &  0.00e+00 &  0.00e+00 &      \\ 
 480 &  4.79e+02 &    2 &           &           & displac  \\ 
 \hdashline 
     &           &    1 &  5.26e-01 &  3.00e-01 &      \\ 
     &           &    2 &  0.00e+00 &  8.67e-19 &      \\ 
 481 &  4.80e+02 &    2 &           &           & displac  \\ 
 \hdashline 
     &           &    1 &  5.35e-01 &  3.00e-01 &      \\ 
     &           &    2 &  0.00e+00 &  0.00e+00 &      \\ 
 482 &  4.81e+02 &    2 &           &           & displac  \\ 
 \hdashline 
     &           &    1 &  5.44e-01 &  3.00e-01 &      \\ 
     &           &    2 &  0.00e+00 &  0.00e+00 &      \\ 
 483 &  4.82e+02 &    2 &           &           & displac  \\ 
 \hdashline 
     &           &    1 &  5.53e-01 &  3.00e-01 &      \\ 
     &           &    2 &  0.00e+00 &  8.67e-19 &      \\ 
 484 &  4.83e+02 &    2 &           &           & displac  \\ 
 \hdashline 
     &           &    1 &  5.62e-01 &  3.00e-01 &      \\ 
     &           &    2 &  0.00e+00 &  0.00e+00 &      \\ 
 485 &  4.84e+02 &    2 &           &           & displac  \\ 
 \hdashline 
     &           &    1 &  5.71e-01 &  3.00e-01 &      \\ 
     &           &    2 &  0.00e+00 &  0.00e+00 &      \\ 
 486 &  4.85e+02 &    2 &           &           & displac  \\ 
 \hdashline 
     &           &    1 &  5.80e-01 &  3.00e-01 &      \\ 
     &           &    2 &  0.00e+00 &  0.00e+00 &      \\ 
 487 &  4.86e+02 &    2 &           &           & displac  \\ 
 \hdashline 
     &           &    1 &  5.89e-01 &  3.00e-01 &      \\ 
     &           &    2 &  0.00e+00 &  0.00e+00 &      \\ 
 488 &  4.87e+02 &    2 &           &           & displac  \\ 
 \hdashline 
     &           &    1 &  5.98e-01 &  3.00e-01 &      \\ 
     &           &    2 &  0.00e+00 &  0.00e+00 &      \\ 
 489 &  4.88e+02 &    2 &           &           & displac  \\ 
 \hdashline 
     &           &    1 &  6.07e-01 &  3.00e-01 &      \\ 
     &           &    2 &  0.00e+00 &  0.00e+00 &      \\ 
 490 &  4.89e+02 &    2 &           &           & displac  \\ 
 \hdashline 
     &           &    1 &  6.16e-01 &  3.00e-01 &      \\ 
     &           &    2 &  0.00e+00 &  0.00e+00 &      \\ 
 491 &  4.90e+02 &    2 &           &           & displac  \\ 
 \hdashline 
     &           &    1 &  6.25e-01 &  3.00e-01 &      \\ 
     &           &    2 &  0.00e+00 &  0.00e+00 &      \\ 
 492 &  4.91e+02 &    2 &           &           & displac  \\ 
 \hdashline 
     &           &    1 &  6.34e-01 &  3.00e-01 &      \\ 
     &           &    2 &  0.00e+00 &  0.00e+00 &      \\ 
 493 &  4.92e+02 &    2 &           &           & displac  \\ 
 \hdashline 
     &           &    1 &  6.43e-01 &  3.00e-01 &      \\ 
     &           &    2 &  0.00e+00 &  0.00e+00 &      \\ 
 494 &  4.93e+02 &    2 &           &           & displac  \\ 
 \hdashline 
     &           &    1 &  6.52e-01 &  3.00e-01 &      \\ 
     &           &    2 &  0.00e+00 &  0.00e+00 &      \\ 
 495 &  4.94e+02 &    2 &           &           & displac  \\ 
 \hdashline 
     &           &    1 &  6.61e-01 &  3.00e-01 &      \\ 
     &           &    2 &  0.00e+00 &  0.00e+00 &      \\ 
 496 &  4.95e+02 &    2 &           &           & displac  \\ 
 \hdashline 
     &           &    1 &  6.70e-01 &  3.00e-01 &      \\ 
     &           &    2 &  0.00e+00 &  0.00e+00 &      \\ 
 497 &  4.96e+02 &    2 &           &           & displac  \\ 
 \hdashline 
     &           &    1 &  6.78e-01 &  3.00e-01 &      \\ 
     &           &    2 &  0.00e+00 &  0.00e+00 &      \\ 
 498 &  4.97e+02 &    2 &           &           & displac  \\ 
 \hdashline 
     &           &    1 &  6.87e-01 &  3.00e-01 &      \\ 
     &           &    2 &  0.00e+00 &  0.00e+00 &      \\ 
 499 &  4.98e+02 &    2 &           &           & displac  \\ 
 \hdashline 
     &           &    1 &  3.29e+01 &  3.00e-01 &      \\ 
     &           &    2 &  0.00e+00 &  0.00e+00 &      \\ 
 500 &  4.99e+02 &    2 &           &           & displac  \\ 
 \hdashline 
     &           &    1 &  3.44e-03 &  3.00e-01 &      \\ 
     &           &    2 &  0.00e+00 &  0.00e+00 &      \\ 
 501 &  5.00e+02 &    2 &           &           & displac  \\ 
 \hdashline 
     &           &    1 &  3.48e-03 &  3.00e-01 &      \\ 
     &           &    2 &  0.00e+00 &  0.00e+00 &      \\ 
 502 &  5.01e+02 &    2 &           &           & displac  \\ 
 \hdashline 
     &           &    1 &  3.53e-03 &  3.00e-01 &      \\ 
     &           &    2 &  0.00e+00 &  0.00e+00 &      \\ 
 503 &  5.02e+02 &    2 &           &           & displac  \\ 
 \hdashline 
     &           &    1 &  3.57e-03 &  3.00e-01 &      \\ 
     &           &    2 &  0.00e+00 &  0.00e+00 &      \\ 
 504 &  5.03e+02 &    2 &           &           & displac  \\ 
 \hdashline 
     &           &    1 &  3.61e-03 &  3.00e-01 &      \\ 
     &           &    2 &  0.00e+00 &  5.55e-17 &      \\ 
 505 &  5.04e+02 &    2 &           &           & displac  \\ 
 \hdashline 
     &           &    1 &  3.65e-03 &  3.00e-01 &      \\ 
     &           &    2 &  0.00e+00 &  0.00e+00 &      \\ 
 506 &  5.05e+02 &    2 &           &           & displac  \\ 
 \hdashline 
     &           &    1 &  4.14e+01 &  3.00e-01 &      \\ 
     &           &    2 &  0.00e+00 &  0.00e+00 &      \\ 
 507 &  5.06e+02 &    2 &           &           & displac  \\ 
 \hdashline 
     &           &    1 &  3.74e-03 &  3.00e-01 &      \\ 
     &           &    2 &  0.00e+00 &  0.00e+00 &      \\ 
 508 &  5.07e+02 &    2 &           &           & displac  \\ 
 \hdashline 
     &           &    1 &  3.78e-03 &  3.00e-01 &      \\ 
     &           &    2 &  0.00e+00 &  0.00e+00 &      \\ 
 509 &  5.08e+02 &    2 &           &           & displac  \\ 
 \hdashline 
     &           &    1 &  3.82e-03 &  3.00e-01 &      \\ 
     &           &    2 &  0.00e+00 &  0.00e+00 &      \\ 
 510 &  5.09e+02 &    2 &           &           & displac  \\ 
 \hdashline 
     &           &    1 &  3.86e-03 &  3.00e-01 &      \\ 
     &           &    2 &  0.00e+00 &  0.00e+00 &      \\ 
 511 &  5.10e+02 &    2 &           &           & displac  \\ 
 \hdashline 
     &           &    1 &  3.91e-03 &  3.00e-01 &      \\ 
     &           &    2 &  1.14e-13 &  3.77e-15 &      \\ 
 512 &  5.11e+02 &    2 &           &           & forces  \\ 
 \hdashline 
     &           &    1 &  4.76e+01 &  3.00e-01 &      \\ 
     &           &    2 &  0.00e+00 &  6.66e-16 &      \\ 
 513 &  5.12e+02 &    2 &           &           & displac  \\ 
 \hdashline 
     &           &    1 &  3.99e-03 &  3.00e-01 &      \\ 
     &           &    2 &  0.00e+00 &  0.00e+00 &      \\ 
 514 &  5.13e+02 &    2 &           &           & displac  \\ 
 \hdashline 
     &           &    1 &  4.03e-03 &  3.00e-01 &      \\ 
     &           &    2 &  0.00e+00 &  0.00e+00 &      \\ 
 515 &  5.14e+02 &    2 &           &           & displac  \\ 
 \hdashline 
     &           &    1 &  4.07e-03 &  3.00e-01 &      \\ 
     &           &    2 &  1.14e-13 &  3.77e-15 &      \\ 
 516 &  5.15e+02 &    2 &           &           & forces  \\ 
 \hdashline 
     &           &    1 &  5.20e+01 &  3.00e-01 &      \\ 
     &           &    2 &  0.00e+00 &  4.44e-16 &      \\ 
 517 &  5.16e+02 &    2 &           &           & displac  \\ 
 \hdashline 
     &           &    1 &  4.16e-03 &  3.00e-01 &      \\ 
     &           &    2 &  1.14e-13 &  3.77e-15 &      \\ 
 518 &  5.17e+02 &    2 &           &           & forces  \\ 
 \hdashline 
     &           &    1 &  4.20e-03 &  3.00e-01 &      \\ 
     &           &    2 &  1.14e-13 &  3.77e-15 &      \\ 
 519 &  5.18e+02 &    2 &           &           & forces  \\ 
 \hdashline 
     &           &    1 &  4.24e-03 &  3.00e-01 &      \\ 
     &           &    2 &  1.14e-13 &  3.77e-15 &      \\ 
 520 &  5.19e+02 &    2 &           &           & forces  \\ 
 \hdashline 
     &           &    1 &  4.28e-03 &  3.00e-01 &      \\ 
     &           &    2 &  0.00e+00 &  0.00e+00 &      \\ 
 521 &  5.20e+02 &    2 &           &           & displac  \\ 
 \hdashline 
     &           &    1 &  4.32e-03 &  3.00e-01 &      \\ 
     &           &    2 &  0.00e+00 &  0.00e+00 &      \\ 
 522 &  5.21e+02 &    2 &           &           & displac  \\ 
 \hdashline 
     &           &    1 &  4.36e-03 &  3.00e-01 &      \\ 
     &           &    2 &  0.00e+00 &  0.00e+00 &      \\ 
 523 &  5.22e+02 &    2 &           &           & displac  \\ 
 \hdashline 
     &           &    1 &  4.40e-03 &  3.00e-01 &      \\ 
     &           &    2 &  0.00e+00 &  0.00e+00 &      \\ 
 524 &  5.23e+02 &    2 &           &           & displac  \\ 
 \hdashline 
     &           &    1 &  4.44e-03 &  3.00e-01 &      \\ 
     &           &    2 &  0.00e+00 &  0.00e+00 &      \\ 
 525 &  5.24e+02 &    2 &           &           & displac  \\ 
 \hdashline 
     &           &    1 &  4.48e-03 &  3.00e-01 &      \\ 
     &           &    2 &  1.14e-13 &  3.72e-15 &      \\ 
 526 &  5.25e+02 &    2 &           &           & forces  \\ 
 \hdashline 
     &           &    1 &  6.37e+01 &  3.00e-01 &      \\ 
     &           &    2 &  0.00e+00 &  4.44e-16 &      \\ 
 527 &  5.26e+02 &    2 &           &           & displac  \\ 
 \hdashline 
     &           &    1 &  4.56e-03 &  3.00e-01 &      \\ 
     &           &    2 &  1.14e-13 &  3.72e-15 &      \\ 
 528 &  5.27e+02 &    2 &           &           & forces  \\ 
 \hdashline 
     &           &    1 &  4.60e-03 &  3.00e-01 &      \\ 
     &           &    2 &  1.14e-13 &  3.72e-15 &      \\ 
 529 &  5.28e+02 &    2 &           &           & forces  \\ 
 \hdashline 
     &           &    1 &  4.64e-03 &  3.00e-01 &      \\ 
     &           &    2 &  1.14e-13 &  3.72e-15 &      \\ 
 530 &  5.29e+02 &    2 &           &           & forces  \\ 
 \hdashline 
     &           &    1 &  4.68e-03 &  3.00e-01 &      \\ 
     &           &    2 &  1.14e-13 &  3.72e-15 &      \\ 
 531 &  5.30e+02 &    2 &           &           & forces  \\ 
 \hdashline 
     &           &    1 &  4.72e-03 &  3.00e-01 &      \\ 
     &           &    2 &  1.14e-13 &  3.72e-15 &      \\ 
 532 &  5.31e+02 &    2 &           &           & forces  \\ 
 \hdashline 
     &           &    1 &  4.76e-03 &  3.00e-01 &      \\ 
     &           &    2 &  1.14e-13 &  3.72e-15 &      \\ 
 533 &  5.32e+02 &    2 &           &           & forces  \\ 
 \hdashline 
     &           &    1 &  4.80e-03 &  3.00e-01 &      \\ 
     &           &    2 &  1.14e-13 &  3.72e-15 &      \\ 
 534 &  5.33e+02 &    2 &           &           & forces  \\ 
 \hdashline 
     &           &    1 &  4.84e-03 &  3.00e-01 &      \\ 
     &           &    2 &  1.14e-13 &  3.72e-15 &      \\ 
 535 &  5.34e+02 &    2 &           &           & forces  \\ 
 \hdashline 
     &           &    1 &  4.88e-03 &  3.00e-01 &      \\ 
     &           &    2 &  1.14e-13 &  3.72e-15 &      \\ 
 536 &  5.35e+02 &    2 &           &           & forces  \\ 
 \hdashline 
     &           &    1 &  4.92e-03 &  3.00e-01 &      \\ 
     &           &    2 &  1.14e-13 &  3.66e-15 &      \\ 
 537 &  5.36e+02 &    2 &           &           & forces  \\ 
 \hdashline 
     &           &    1 &  4.96e-03 &  3.00e-01 &      \\ 
     &           &    2 &  1.14e-13 &  3.66e-15 &      \\ 
 538 &  5.37e+02 &    2 &           &           & forces  \\ 
 \hdashline 
     &           &    1 &  7.93e+01 &  3.00e-01 &      \\ 
     &           &    2 &  0.00e+00 &  4.44e-16 &      \\ 
 539 &  5.38e+02 &    2 &           &           & displac  \\ 
 \hdashline 
     &           &    1 &  5.03e-03 &  3.00e-01 &      \\ 
     &           &    2 &  0.00e+00 &  0.00e+00 &      \\ 
 540 &  5.39e+02 &    2 &           &           & displac  \\ 
 \hdashline 
     &           &    1 &  5.07e-03 &  3.00e-01 &      \\ 
     &           &    2 &  0.00e+00 &  0.00e+00 &      \\ 
 541 &  5.40e+02 &    2 &           &           & displac  \\ 
 \hdashline 
     &           &    1 &  5.11e-03 &  3.00e-01 &      \\ 
     &           &    2 &  0.00e+00 &  0.00e+00 &      \\ 
 542 &  5.41e+02 &    2 &           &           & displac  \\ 
 \hdashline 
     &           &    1 &  8.48e+01 &  3.00e-01 &      \\ 
     &           &    2 &  0.00e+00 &  0.00e+00 &      \\ 
 543 &  5.42e+02 &    2 &           &           & displac  \\ 
 \hdashline 
     &           &    1 &  5.19e-03 &  3.00e-01 &      \\ 
     &           &    2 &  0.00e+00 &  0.00e+00 &      \\ 
 544 &  5.43e+02 &    2 &           &           & displac  \\ 
 \hdashline 
     &           &    1 &  5.22e-03 &  3.00e-01 &      \\ 
     &           &    2 &  0.00e+00 &  0.00e+00 &      \\ 
 545 &  5.44e+02 &    2 &           &           & displac  \\ 
 \hdashline 
     &           &    1 &  5.26e-03 &  3.00e-01 &      \\ 
     &           &    2 &  0.00e+00 &  0.00e+00 &      \\ 
 546 &  5.45e+02 &    2 &           &           & displac  \\ 
 \hdashline 
     &           &    1 &  5.30e-03 &  3.00e-01 &      \\ 
     &           &    2 &  2.27e-13 &  7.27e-15 &      \\ 
 547 &  5.46e+02 &    2 &           &           & forces  \\ 
 \hdashline 
     &           &    1 &  5.34e-03 &  3.00e-01 &      \\ 
     &           &    2 &  0.00e+00 &  0.00e+00 &      \\ 
 548 &  5.47e+02 &    2 &           &           & displac  \\ 
 \hdashline 
     &           &    1 &  9.35e+01 &  3.00e-01 &      \\ 
     &           &    2 &  0.00e+00 &  0.00e+00 &      \\ 
 549 &  5.48e+02 &    2 &           &           & displac  \\ 
 \hdashline 
     &           &    1 &  5.41e-03 &  3.00e-01 &      \\ 
     &           &    2 &  0.00e+00 &  0.00e+00 &      \\ 
 550 &  5.49e+02 &    2 &           &           & displac  \\ 
 \hdashline 
     &           &    1 &  9.64e+01 &  3.00e-01 &      \\ 
     &           &    2 &  0.00e+00 &  0.00e+00 &      \\ 
 551 &  5.50e+02 &    2 &           &           & displac  \\ 
 \hdashline 
     &           &    1 &  5.49e-03 &  3.00e-01 &      \\ 
     &           &    2 &  0.00e+00 &  0.00e+00 &      \\ 
 552 &  5.51e+02 &    2 &           &           & displac  \\ 
 \hdashline 
     &           &    1 &  5.52e-03 &  3.00e-01 &      \\ 
     &           &    2 &  2.27e-13 &  7.27e-15 &      \\ 
 553 &  5.52e+02 &    2 &           &           & forces  \\ 
 \hdashline 
     &           &    1 &  5.56e-03 &  3.00e-01 &      \\ 
     &           &    2 &  2.27e-13 &  7.22e-15 &      \\ 
 554 &  5.53e+02 &    2 &           &           & forces  \\ 
 \hdashline 
     &           &    1 &  5.60e-03 &  3.00e-01 &      \\ 
     &           &    2 &  2.27e-13 &  7.22e-15 &      \\ 
 555 &  5.54e+02 &    2 &           &           & forces  \\ 
 \hdashline 
     &           &    1 &  5.63e-03 &  3.00e-01 &      \\ 
     &           &    2 &  2.27e-13 &  7.22e-15 &      \\ 
 556 &  5.55e+02 &    2 &           &           & forces  \\ 
 \hdashline 
     &           &    1 &  5.67e-03 &  3.00e-01 &      \\ 
     &           &    2 &  2.27e-13 &  7.22e-15 &      \\ 
 557 &  5.56e+02 &    2 &           &           & forces  \\ 
 \hdashline 
     &           &    1 &  1.07e+02 &  3.00e-01 &      \\ 
     &           &    2 &  0.00e+00 &  4.44e-16 &      \\ 
 558 &  5.57e+02 &    2 &           &           & displac  \\ 
 \hdashline 
     &           &    1 &  5.74e-03 &  3.00e-01 &      \\ 
     &           &    2 &  2.27e-13 &  7.22e-15 &      \\ 
 559 &  5.58e+02 &    2 &           &           & forces  \\ 
 \hdashline 
     &           &    1 &  5.78e-03 &  3.00e-01 &      \\ 
     &           &    2 &  2.27e-13 &  7.16e-15 &      \\ 
 560 &  5.59e+02 &    2 &           &           & forces  \\ 
 \hdashline 
     &           &    1 &  5.81e-03 &  3.00e-01 &      \\ 
     &           &    2 &  2.27e-13 &  7.16e-15 &      \\ 
 561 &  5.60e+02 &    2 &           &           & forces  \\ 
 \hdashline 
     &           &    1 &  5.85e-03 &  3.00e-01 &      \\ 
     &           &    2 &  2.27e-13 &  7.16e-15 &      \\ 
 562 &  5.61e+02 &    2 &           &           & forces  \\ 
 \hdashline 
     &           &    1 &  5.88e-03 &  3.00e-01 &      \\ 
     &           &    2 &  2.27e-13 &  7.16e-15 &      \\ 
 563 &  5.62e+02 &    2 &           &           & forces  \\ 
 \hdashline 
     &           &    1 &  5.92e-03 &  3.00e-01 &      \\ 
     &           &    2 &  0.00e+00 &  0.00e+00 &      \\ 
 564 &  5.63e+02 &    2 &           &           & displac  \\ 
 \hdashline 
     &           &    1 &  5.95e-03 &  3.00e-01 &      \\ 
     &           &    2 &  0.00e+00 &  0.00e+00 &      \\ 
 565 &  5.64e+02 &    2 &           &           & displac  \\ 
 \hdashline 
     &           &    1 &  1.20e+02 &  3.00e-01 &      \\ 
     &           &    2 &  0.00e+00 &  0.00e+00 &      \\ 
 566 &  5.65e+02 &    2 &           &           & displac  \\ 
 \hdashline 
     &           &    1 &  6.02e-03 &  3.00e-01 &      \\ 
     &           &    2 &  0.00e+00 &  0.00e+00 &      \\ 
 567 &  5.66e+02 &    2 &           &           & displac  \\ 
 \hdashline 
     &           &    1 &  6.06e-03 &  3.00e-01 &      \\ 
     &           &    2 &  0.00e+00 &  0.00e+00 &      \\ 
 568 &  5.67e+02 &    2 &           &           & displac  \\ 
 \hdashline 
     &           &    1 &  6.09e-03 &  3.00e-01 &      \\ 
     &           &    2 &  0.00e+00 &  0.00e+00 &      \\ 
 569 &  5.68e+02 &    2 &           &           & displac  \\ 
 \hdashline 
     &           &    1 &  1.27e+02 &  3.00e-01 &      \\ 
     &           &    2 &  0.00e+00 &  0.00e+00 &      \\ 
 570 &  5.69e+02 &    2 &           &           & displac  \\ 
 \hdashline 
     &           &    1 &  6.16e-03 &  3.00e-01 &      \\ 
     &           &    2 &  2.27e-13 &  7.11e-15 &      \\ 
 571 &  5.70e+02 &    2 &           &           & forces  \\ 
 \hdashline 
     &           &    1 &  6.19e-03 &  3.00e-01 &      \\ 
     &           &    2 &  2.27e-13 &  7.11e-15 &      \\ 
 572 &  5.71e+02 &    2 &           &           & forces  \\ 
 \hdashline 
     &           &    1 &  6.23e-03 &  3.00e-01 &      \\ 
     &           &    2 &  2.27e-13 &  7.05e-15 &      \\ 
 573 &  5.72e+02 &    2 &           &           & forces  \\ 
 \hdashline 
     &           &    1 &  6.26e-03 &  3.00e-01 &      \\ 
     &           &    2 &  2.27e-13 &  7.05e-15 &      \\ 
 574 &  5.73e+02 &    2 &           &           & forces  \\ 
 \hdashline 
     &           &    1 &  6.30e-03 &  3.00e-01 &      \\ 
     &           &    2 &  4.55e-13 &  1.41e-14 &      \\ 
 575 &  5.74e+02 &    2 &           &           & forces  \\ 
 \hdashline 
     &           &    1 &  1.37e+02 &  3.00e-01 &      \\ 
     &           &    2 &  0.00e+00 &  8.88e-16 &      \\ 
 576 &  5.75e+02 &    2 &           &           & displac  \\ 
 \hdashline 
     &           &    1 &  6.36e-03 &  3.00e-01 &      \\ 
     &           &    2 &  0.00e+00 &  0.00e+00 &      \\ 
 577 &  5.76e+02 &    2 &           &           & displac  \\ 
 \hdashline 
     &           &    1 &  6.40e-03 &  3.00e-01 &      \\ 
     &           &    2 &  2.27e-13 &  7.05e-15 &      \\ 
 578 &  5.77e+02 &    2 &           &           & forces  \\ 
 \hdashline 
     &           &    1 &  1.42e+02 &  3.00e-01 &      \\ 
     &           &    2 &  0.00e+00 &  8.88e-16 &      \\ 
 579 &  5.78e+02 &    2 &           &           & displac  \\ 
 \hdashline 
     &           &    1 &  6.46e-03 &  3.00e-01 &      \\ 
     &           &    2 &  0.00e+00 &  0.00e+00 &      \\ 
 580 &  5.79e+02 &    2 &           &           & displac  \\ 
 \hdashline 
     &           &    1 &  6.49e-03 &  3.00e-01 &      \\ 
     &           &    2 &  0.00e+00 &  0.00e+00 &      \\ 
 581 &  5.80e+02 &    2 &           &           & displac  \\ 
 \hdashline 
     &           &    1 &  6.53e-03 &  3.00e-01 &      \\ 
     &           &    2 &  0.00e+00 &  0.00e+00 &      \\ 
 582 &  5.81e+02 &    2 &           &           & displac  \\ 
 \hdashline 
     &           &    1 &  6.56e-03 &  3.00e-01 &      \\ 
     &           &    2 &  0.00e+00 &  0.00e+00 &      \\ 
 583 &  5.82e+02 &    2 &           &           & displac  \\ 
 \hdashline 
     &           &    1 &  1.51e+02 &  3.00e-01 &      \\ 
     &           &    2 &  0.00e+00 &  0.00e+00 &      \\ 
 584 &  5.83e+02 &    2 &           &           & displac  \\ 
 \hdashline 
     &           &    1 &  6.62e-03 &  3.00e-01 &      \\ 
     &           &    2 &  0.00e+00 &  0.00e+00 &      \\ 
 585 &  5.84e+02 &    2 &           &           & displac  \\ 
 \hdashline 
     &           &    1 &  6.66e-03 &  3.00e-01 &      \\ 
     &           &    2 &  0.00e+00 &  0.00e+00 &      \\ 
 586 &  5.85e+02 &    2 &           &           & displac  \\ 
 \hdashline 
     &           &    1 &  1.57e+02 &  3.00e-01 &      \\ 
     &           &    2 &  0.00e+00 &  0.00e+00 &      \\ 
 587 &  5.86e+02 &    2 &           &           & displac  \\ 
 \hdashline 
     &           &    1 &  6.72e-03 &  3.00e-01 &      \\ 
     &           &    2 &  0.00e+00 &  0.00e+00 &      \\ 
 588 &  5.87e+02 &    2 &           &           & displac  \\ 
 \hdashline 
     &           &    1 &  6.75e-03 &  3.00e-01 &      \\ 
     &           &    2 &  0.00e+00 &  0.00e+00 &      \\ 
 589 &  5.88e+02 &    2 &           &           & displac  \\ 
 \hdashline 
     &           &    1 &  6.78e-03 &  3.00e-01 &      \\ 
     &           &    2 &  0.00e+00 &  0.00e+00 &      \\ 
 590 &  5.89e+02 &    2 &           &           & displac  \\ 
 \hdashline 
     &           &    1 &  6.81e-03 &  3.00e-01 &      \\ 
     &           &    2 &  2.27e-13 &  6.88e-15 &      \\ 
 591 &  5.90e+02 &    2 &           &           & forces  \\ 
 \hdashline 
     &           &    1 &  1.66e+02 &  3.00e-01 &      \\ 
     &           &    2 &  0.00e+00 &  0.00e+00 &      \\ 
 592 &  5.91e+02 &    2 &           &           & displac  \\ 
 \hdashline 
     &           &    1 &  6.87e-03 &  3.00e-01 &      \\ 
     &           &    2 &  0.00e+00 &  0.00e+00 &      \\ 
 593 &  5.92e+02 &    2 &           &           & displac  \\ 
 \hdashline 
     &           &    1 &  6.90e-03 &  3.00e-01 &      \\ 
     &           &    2 &  2.27e-13 &  6.88e-15 &      \\ 
 594 &  5.93e+02 &    2 &           &           & forces  \\ 
 \hdashline 
     &           &    1 &  6.94e-03 &  3.00e-01 &      \\ 
     &           &    2 &  2.27e-13 &  6.88e-15 &      \\ 
 595 &  5.94e+02 &    2 &           &           & forces  \\ 
 \hdashline 
     &           &    1 &  6.97e-03 &  3.00e-01 &      \\ 
     &           &    2 &  2.27e-13 &  6.83e-15 &      \\ 
 596 &  5.95e+02 &    2 &           &           & forces  \\ 
 \hdashline 
     &           &    1 &  7.00e-03 &  3.00e-01 &      \\ 
     &           &    2 &  2.27e-13 &  6.83e-15 &      \\ 
 597 &  5.96e+02 &    2 &           &           & forces  \\ 
 \hdashline 
     &           &    1 &  7.03e-03 &  3.00e-01 &      \\ 
     &           &    2 &  4.55e-13 &  1.37e-14 &      \\ 
 598 &  5.97e+02 &    2 &           &           & forces  \\ 
 \hdashline 
     &           &    1 &  7.06e-03 &  3.00e-01 &      \\ 
     &           &    2 &  4.55e-13 &  1.37e-14 &      \\ 
 599 &  5.98e+02 &    2 &           &           & forces  \\ 
 \hdashline 
     &           &    1 &  7.09e-03 &  3.00e-01 &      \\ 
     &           &    2 &  4.55e-13 &  1.37e-14 &      \\ 
 600 &  5.99e+02 &    2 &           &           & forces  \\ 
 \hdashline 
     &           &    1 &  7.11e-03 &  3.00e-01 &      \\ 
     &           &    2 &  2.27e-13 &  6.83e-15 &      \\ 
 601 &  6.00e+02 &    2 &           &           & forces  \\ 
 \hdashline 
     &           &    1 &  7.14e-03 &  3.00e-01 &      \\ 
     &           &    2 &  2.27e-13 &  6.77e-15 &      \\ 
 602 &  6.01e+02 &    2 &           &           & forces  \\ 
 \hdashline 
     &           &    1 &  7.17e-03 &  3.00e-01 &      \\ 
     &           &    2 &  2.27e-13 &  6.77e-15 &      \\ 
 603 &  6.02e+02 &    2 &           &           & forces  \\ 
 \hdashline 
     &           &    1 &  1.89e+02 &  3.00e-01 &      \\ 
     &           &    2 &  0.00e+00 &  0.00e+00 &      \\ 
 604 &  6.03e+02 &    2 &           &           & displac  \\ 
 \hdashline 
     &           &    1 &  7.23e-03 &  3.00e-01 &      \\ 
     &           &    2 &  0.00e+00 &  0.00e+00 &      \\ 
 605 &  6.04e+02 &    2 &           &           & displac  \\ 
 \hdashline 
     &           &    1 &  7.26e-03 &  3.00e-01 &      \\ 
     &           &    2 &  0.00e+00 &  0.00e+00 &      \\ 
 606 &  6.05e+02 &    2 &           &           & displac  \\ 
 \hdashline 
     &           &    1 &  7.29e-03 &  3.00e-01 &      \\ 
     &           &    2 &  0.00e+00 &  0.00e+00 &      \\ 
 607 &  6.06e+02 &    2 &           &           & displac  \\ 
 \hdashline 
     &           &    1 &  7.32e-03 &  3.00e-01 &      \\ 
     &           &    2 &  2.27e-13 &  6.72e-15 &      \\ 
 608 &  6.07e+02 &    2 &           &           & forces  \\ 
 \hdashline 
     &           &    1 &  7.34e-03 &  3.00e-01 &      \\ 
     &           &    2 &  2.27e-13 &  6.72e-15 &      \\ 
 609 &  6.08e+02 &    2 &           &           & forces  \\ 
 \hdashline 
     &           &    1 &  7.37e-03 &  3.00e-01 &      \\ 
     &           &    2 &  2.27e-13 &  6.72e-15 &      \\ 
 610 &  6.09e+02 &    2 &           &           & forces  \\ 
 \hdashline 
     &           &    1 &  7.40e-03 &  3.00e-01 &      \\ 
     &           &    2 &  2.27e-13 &  6.72e-15 &      \\ 
 611 &  6.10e+02 &    2 &           &           & forces  \\ 
 \hdashline 
     &           &    1 &  7.43e-03 &  3.00e-01 &      \\ 
     &           &    2 &  2.27e-13 &  6.72e-15 &      \\ 
 612 &  6.11e+02 &    2 &           &           & forces  \\ 
 \hdashline 
     &           &    1 &  7.46e-03 &  3.00e-01 &      \\ 
     &           &    2 &  2.27e-13 &  6.66e-15 &      \\ 
 613 &  6.12e+02 &    2 &           &           & forces  \\ 
 \hdashline 
     &           &    1 &  7.48e-03 &  3.00e-01 &      \\ 
     &           &    2 &  2.27e-13 &  6.66e-15 &      \\ 
 614 &  6.13e+02 &    2 &           &           & forces  \\ 
 \hdashline 
     &           &    1 &  2.12e+02 &  3.00e-01 &      \\ 
     &           &    2 &  0.00e+00 &  0.00e+00 &      \\ 
 615 &  6.14e+02 &    2 &           &           & displac  \\ 
 \hdashline 
     &           &    1 &  7.54e-03 &  3.00e-01 &      \\ 
     &           &    2 &  2.27e-13 &  6.66e-15 &      \\ 
 616 &  6.15e+02 &    2 &           &           & forces  \\ 
 \hdashline 
     &           &    1 &  7.56e-03 &  3.00e-01 &      \\ 
     &           &    2 &  2.27e-13 &  6.66e-15 &      \\ 
 617 &  6.16e+02 &    2 &           &           & forces  \\ 
 \hdashline 
     &           &    1 &  7.59e-03 &  3.00e-01 &      \\ 
     &           &    2 &  4.55e-13 &  1.33e-14 &      \\ 
 618 &  6.17e+02 &    2 &           &           & forces  \\ 
 \hdashline 
     &           &    1 &  7.62e-03 &  3.00e-01 &      \\ 
     &           &    2 &  4.55e-13 &  1.33e-14 &      \\ 
 619 &  6.18e+02 &    2 &           &           & forces  \\ 
 \hdashline 
     &           &    1 &  7.64e-03 &  3.00e-01 &      \\ 
     &           &    2 &  4.55e-13 &  1.33e-14 &      \\ 
 620 &  6.19e+02 &    2 &           &           & forces  \\ 
 \hdashline 
     &           &    1 &  7.67e-03 &  3.00e-01 &      \\ 
     &           &    2 &  4.55e-13 &  1.32e-14 &      \\ 
 621 &  6.20e+02 &    2 &           &           & forces  \\ 
 \hdashline 
     &           &    1 &  7.70e-03 &  3.00e-01 &      \\ 
     &           &    2 &  4.55e-13 &  1.32e-14 &      \\ 
 622 &  6.21e+02 &    2 &           &           & forces  \\ 
 \hdashline 
     &           &    1 &  7.72e-03 &  3.00e-01 &      \\ 
     &           &    2 &  2.27e-13 &  6.61e-15 &      \\ 
 623 &  6.22e+02 &    2 &           &           & forces  \\ 
 \hdashline 
     &           &    1 &  7.75e-03 &  3.00e-01 &      \\ 
     &           &    2 &  2.27e-13 &  6.61e-15 &      \\ 
 624 &  6.23e+02 &    2 &           &           & forces  \\ 
 \hdashline 
     &           &    1 &  7.77e-03 &  3.00e-01 &      \\ 
     &           &    2 &  2.27e-13 &  6.55e-15 &      \\ 
 625 &  6.24e+02 &    2 &           &           & forces  \\ 
 \hdashline 
     &           &    1 &  2.35e+02 &  3.00e-01 &      \\ 
     &           &    2 &  0.00e+00 &  0.00e+00 &      \\ 
 626 &  6.25e+02 &    2 &           &           & displac  \\ 
 \hdashline 
     &           &    1 &  7.82e-03 &  3.00e-01 &      \\ 
     &           &    2 &  2.27e-13 &  6.55e-15 &      \\ 
 627 &  6.26e+02 &    2 &           &           & forces  \\ 
 \hdashline 
     &           &    1 &  2.40e+02 &  3.00e-01 &      \\ 
     &           &    2 &  0.00e+00 &  0.00e+00 &      \\ 
 628 &  6.27e+02 &    2 &           &           & displac  \\ 
 \hdashline 
     &           &    1 &  7.87e-03 &  3.00e-01 &      \\ 
     &           &    2 &  0.00e+00 &  0.00e+00 &      \\ 
 629 &  6.28e+02 &    2 &           &           & displac  \\ 
 \hdashline 
     &           &    1 &  7.90e-03 &  3.00e-01 &      \\ 
     &           &    2 &  0.00e+00 &  0.00e+00 &      \\ 
 630 &  6.29e+02 &    2 &           &           & displac  \\ 
 \hdashline 
     &           &    1 &  7.92e-03 &  3.00e-01 &      \\ 
     &           &    2 &  0.00e+00 &  0.00e+00 &      \\ 
 631 &  6.30e+02 &    2 &           &           & displac  \\ 
 \hdashline 
     &           &    1 &  2.49e+02 &  3.00e-01 &      \\ 
     &           &    2 &  0.00e+00 &  0.00e+00 &      \\ 
 632 &  6.31e+02 &    2 &           &           & displac  \\ 
 \hdashline 
     &           &    1 &  7.97e-03 &  3.00e-01 &      \\ 
     &           &    2 &  4.55e-13 &  1.30e-14 &      \\ 
 633 &  6.32e+02 &    2 &           &           & forces  \\ 
 \hdashline 
     &           &    1 &  7.99e-03 &  3.00e-01 &      \\ 
     &           &    2 &  4.55e-13 &  1.30e-14 &      \\ 
 634 &  6.33e+02 &    2 &           &           & forces  \\ 
 \hdashline 
     &           &    1 &  2.55e+02 &  3.00e-01 &      \\ 
     &           &    2 &  0.00e+00 &  8.88e-16 &      \\ 
 635 &  6.34e+02 &    2 &           &           & displac  \\ 
 \hdashline 
     &           &    1 &  8.04e-03 &  3.00e-01 &      \\ 
     &           &    2 &  0.00e+00 &  0.00e+00 &      \\ 
 636 &  6.35e+02 &    2 &           &           & displac  \\ 
 \hdashline 
     &           &    1 &  2.60e+02 &  3.00e-01 &      \\ 
     &           &    2 &  0.00e+00 &  0.00e+00 &      \\ 
 637 &  6.36e+02 &    2 &           &           & displac  \\ 
 \hdashline 
     &           &    1 &  8.09e-03 &  3.00e-01 &      \\ 
     &           &    2 &  2.27e-13 &  6.44e-15 &      \\ 
 638 &  6.37e+02 &    2 &           &           & forces  \\ 
 \hdashline 
     &           &    1 &  8.11e-03 &  3.00e-01 &      \\ 
     &           &    2 &  2.27e-13 &  6.44e-15 &      \\ 
 639 &  6.38e+02 &    2 &           &           & forces  \\ 
 \hdashline 
     &           &    1 &  2.67e+02 &  3.00e-01 &      \\ 
     &           &    2 &  0.00e+00 &  0.00e+00 &      \\ 
 640 &  6.39e+02 &    2 &           &           & displac  \\ 
 \hdashline 
     &           &    1 &  8.16e-03 &  3.00e-01 &      \\ 
     &           &    2 &  2.27e-13 &  6.44e-15 &      \\ 
 641 &  6.40e+02 &    2 &           &           & forces  \\ 
 \hdashline 
     &           &    1 &  8.18e-03 &  3.00e-01 &      \\ 
     &           &    2 &  2.27e-13 &  6.38e-15 &      \\ 
 642 &  6.41e+02 &    2 &           &           & forces  \\ 
 \hdashline 
     &           &    1 &  2.74e+02 &  3.00e-01 &      \\ 
     &           &    2 &  0.00e+00 &  0.00e+00 &      \\ 
 643 &  6.42e+02 &    2 &           &           & displac  \\ 
 \hdashline 
     &           &    1 &  8.23e-03 &  3.00e-01 &      \\ 
     &           &    2 &  0.00e+00 &  0.00e+00 &      \\ 
 644 &  6.43e+02 &    2 &           &           & displac  \\ 
 \hdashline 
     &           &    1 &  8.25e-03 &  3.00e-01 &      \\ 
     &           &    2 &  0.00e+00 &  0.00e+00 &      \\ 
 645 &  6.44e+02 &    2 &           &           & displac  \\ 
 \hdashline 
     &           &    1 &  8.27e-03 &  3.00e-01 &      \\ 
     &           &    2 &  0.00e+00 &  0.00e+00 &      \\ 
 646 &  6.45e+02 &    2 &           &           & displac  \\ 
 \hdashline 
     &           &    1 &  8.29e-03 &  3.00e-01 &      \\ 
     &           &    2 &  0.00e+00 &  0.00e+00 &      \\ 
 647 &  6.46e+02 &    2 &           &           & displac  \\ 
 \hdashline 
     &           &    1 &  8.31e-03 &  3.00e-01 &      \\ 
     &           &    2 &  0.00e+00 &  0.00e+00 &      \\ 
 648 &  6.47e+02 &    2 &           &           & displac  \\ 
 \hdashline 
     &           &    1 &  8.33e-03 &  3.00e-01 &      \\ 
     &           &    2 &  0.00e+00 &  0.00e+00 &      \\ 
 649 &  6.48e+02 &    2 &           &           & displac  \\ 
 \hdashline 
     &           &    1 &  8.36e-03 &  3.00e-01 &      \\ 
     &           &    2 &  0.00e+00 &  0.00e+00 &      \\ 
 650 &  6.49e+02 &    2 &           &           & displac  \\ 
 \hdashline 
     &           &    1 &  8.38e-03 &  3.00e-01 &      \\ 
     &           &    2 &  0.00e+00 &  0.00e+00 &      \\ 
 651 &  6.50e+02 &    2 &           &           & displac  \\ 
 \hdashline 
     &           &    1 &  8.40e-03 &  3.00e-01 &      \\ 
     &           &    2 &  0.00e+00 &  2.71e-20 &      \\ 
 652 &  6.51e+02 &    2 &           &           & displac  \\ 
 \hdashline 
     &           &    1 &  8.42e-03 &  3.00e-01 &      \\ 
     &           &    2 &  0.00e+00 &  2.71e-20 &      \\ 
 653 &  6.52e+02 &    2 &           &           & displac  \\ 
 \hdashline 
     &           &    1 &  3.00e+02 &  3.00e-01 &      \\ 
     &           &    2 &  0.00e+00 &  0.00e+00 &      \\ 
 654 &  6.53e+02 &    2 &           &           & displac  \\ 
 \hdashline 
     &           &    1 &  8.46e-03 &  3.00e-01 &      \\ 
     &           &    2 &  0.00e+00 &  0.00e+00 &      \\ 
 655 &  6.54e+02 &    2 &           &           & displac  \\ 
 \hdashline 
     &           &    1 &  8.48e-03 &  3.00e-01 &      \\ 
     &           &    2 &  0.00e+00 &  0.00e+00 &      \\ 
 656 &  6.55e+02 &    2 &           &           & displac  \\ 
 \hdashline 
     &           &    1 &  8.50e-03 &  3.00e-01 &      \\ 
     &           &    2 &  4.55e-13 &  1.25e-14 &      \\ 
 657 &  6.56e+02 &    2 &           &           & forces  \\ 
 \hdashline 
     &           &    1 &  8.52e-03 &  3.00e-01 &      \\ 
     &           &    2 &  4.55e-13 &  1.25e-14 &      \\ 
 658 &  6.57e+02 &    2 &           &           & forces  \\ 
 \hdashline 
     &           &    1 &  3.12e+02 &  3.00e-01 &      \\ 
     &           &    2 &  0.00e+00 &  0.00e+00 &      \\ 
 659 &  6.58e+02 &    2 &           &           & displac  \\ 
 \hdashline 
     &           &    1 &  8.56e-03 &  3.00e-01 &      \\ 
     &           &    2 &  0.00e+00 &  2.71e-20 &      \\ 
 660 &  6.59e+02 &    2 &           &           & displac  \\ 
 \hdashline 
     &           &    1 &  8.58e-03 &  3.00e-01 &      \\ 
     &           &    2 &  0.00e+00 &  0.00e+00 &      \\ 
 661 &  6.60e+02 &    2 &           &           & displac  \\ 
 \hdashline 
     &           &    1 &  8.60e-03 &  3.00e-01 &      \\ 
     &           &    2 &  0.00e+00 &  0.00e+00 &      \\ 
 662 &  6.61e+02 &    2 &           &           & displac  \\ 
 \hdashline 
     &           &    1 &  8.62e-03 &  3.00e-01 &      \\ 
     &           &    2 &  0.00e+00 &  2.71e-20 &      \\ 
 663 &  6.62e+02 &    2 &           &           & displac  \\ 
 \hdashline 
     &           &    1 &  8.63e-03 &  3.00e-01 &      \\ 
     &           &    2 &  0.00e+00 &  0.00e+00 &      \\ 
 664 &  6.63e+02 &    2 &           &           & displac  \\ 
 \hdashline 
     &           &    1 &  8.65e-03 &  3.00e-01 &      \\ 
     &           &    2 &  0.00e+00 &  0.00e+00 &      \\ 
 665 &  6.64e+02 &    2 &           &           & displac  \\ 
 \hdashline 
     &           &    1 &  8.67e-03 &  3.00e-01 &      \\ 
     &           &    2 &  0.00e+00 &  0.00e+00 &      \\ 
 666 &  6.65e+02 &    2 &           &           & displac  \\ 
 \hdashline 
     &           &    1 &  8.69e-03 &  3.00e-01 &      \\ 
     &           &    2 &  0.00e+00 &  0.00e+00 &      \\ 
 667 &  6.66e+02 &    2 &           &           & displac  \\ 
 \hdashline 
     &           &    1 &  8.71e-03 &  3.00e-01 &      \\ 
     &           &    2 &  0.00e+00 &  0.00e+00 &      \\ 
 668 &  6.67e+02 &    2 &           &           & displac  \\ 
 \hdashline 
     &           &    1 &  8.73e-03 &  3.00e-01 &      \\ 
     &           &    2 &  0.00e+00 &  0.00e+00 &      \\ 
 669 &  6.68e+02 &    2 &           &           & displac  \\ 
 \hdashline 
     &           &    1 &  8.74e-03 &  3.00e-01 &      \\ 
     &           &    2 &  0.00e+00 &  0.00e+00 &      \\ 
 670 &  6.69e+02 &    2 &           &           & displac  \\ 
 \hdashline 
     &           &    1 &  3.41e+02 &  3.00e-01 &      \\ 
     &           &    2 &  0.00e+00 &  0.00e+00 &      \\ 
 671 &  6.70e+02 &    2 &           &           & displac  \\ 
 \hdashline 
     &           &    1 &  8.78e-03 &  3.00e-01 &      \\ 
     &           &    2 &  4.55e-13 &  1.22e-14 &      \\ 
 672 &  6.71e+02 &    2 &           &           & forces  \\ 
 \hdashline 
     &           &    1 &  8.80e-03 &  3.00e-01 &      \\ 
     &           &    2 &  4.55e-13 &  1.22e-14 &      \\ 
 673 &  6.72e+02 &    2 &           &           & forces  \\ 
 \hdashline 
     &           &    1 &  3.49e+02 &  3.00e-01 &      \\ 
     &           &    2 &  0.00e+00 &  0.00e+00 &      \\ 
 674 &  6.73e+02 &    2 &           &           & displac  \\ 
 \hdashline 
     &           &    1 &  8.83e-03 &  3.00e-01 &      \\ 
     &           &    2 &  0.00e+00 &  2.43e-14 &      \\ 
 675 &  6.74e+02 &    2 &           &           & displac  \\ 
 \hdashline 
     &           &    1 &  8.85e-03 &  3.00e-01 &      \\ 
     &           &    2 &  0.00e+00 &  0.00e+00 &      \\ 
 676 &  6.75e+02 &    2 &           &           & displac  \\ 
 \hdashline 
     &           &    1 &  8.86e-03 &  3.00e-01 &      \\ 
     &           &    2 &  0.00e+00 &  0.00e+00 &      \\ 
 677 &  6.76e+02 &    2 &           &           & displac  \\ 
 \hdashline 
     &           &    1 &  8.88e-03 &  3.00e-01 &      \\ 
     &           &    2 &  0.00e+00 &  0.00e+00 &      \\ 
 678 &  6.77e+02 &    2 &           &           & displac  \\ 
 \hdashline 
     &           &    1 &  8.90e-03 &  3.00e-01 &      \\ 
     &           &    2 &  4.55e-13 &  1.21e-14 &      \\ 
 679 &  6.78e+02 &    2 &           &           & forces  \\ 
 \hdashline 
     &           &    1 &  3.64e+02 &  3.00e-01 &      \\ 
     &           &    2 &  0.00e+00 &  0.00e+00 &      \\ 
 680 &  6.79e+02 &    2 &           &           & displac  \\ 
 \hdashline 
     &           &    1 &  8.93e-03 &  3.00e-01 &      \\ 
     &           &    2 &  0.00e+00 &  0.00e+00 &      \\ 
 681 &  6.80e+02 &    2 &           &           & displac  \\ 
 \hdashline 
     &           &    1 &  8.95e-03 &  3.00e-01 &      \\ 
     &           &    2 &  0.00e+00 &  0.00e+00 &      \\ 
 682 &  6.81e+02 &    2 &           &           & displac  \\ 
 \hdashline 
     &           &    1 &  8.96e-03 &  3.00e-01 &      \\ 
     &           &    2 &  4.55e-13 &  1.20e-14 &      \\ 
 683 &  6.82e+02 &    2 &           &           & forces  \\ 
 \hdashline 
     &           &    1 &  8.98e-03 &  3.00e-01 &      \\ 
     &           &    2 &  4.55e-13 &  1.20e-14 &      \\ 
 684 &  6.83e+02 &    2 &           &           & forces  \\ 
 \hdashline 
     &           &    1 &  3.77e+02 &  3.00e-01 &      \\ 
     &           &    2 &  0.00e+00 &  0.00e+00 &      \\ 
 685 &  6.84e+02 &    2 &           &           & displac  \\ 
 \hdashline 
     &           &    1 &  9.01e-03 &  3.00e-01 &      \\ 
     &           &    2 &  0.00e+00 &  2.71e-20 &      \\ 
 686 &  6.85e+02 &    2 &           &           & displac  \\ 
 \hdashline 
     &           &    1 &  9.02e-03 &  3.00e-01 &      \\ 
     &           &    2 &  0.00e+00 &  2.71e-20 &      \\ 
 687 &  6.86e+02 &    2 &           &           & displac  \\ 
 \hdashline 
     &           &    1 &  9.04e-03 &  3.00e-01 &      \\ 
     &           &    2 &  0.00e+00 &  0.00e+00 &      \\ 
 688 &  6.87e+02 &    2 &           &           & displac  \\ 
 \hdashline 
     &           &    1 &  3.87e+02 &  3.00e-01 &      \\ 
     &           &    2 &  0.00e+00 &  0.00e+00 &      \\ 
 689 &  6.88e+02 &    2 &           &           & displac  \\ 
 \hdashline 
     &           &    1 &  9.07e-03 &  3.00e-01 &      \\ 
     &           &    2 &  0.00e+00 &  0.00e+00 &      \\ 
 690 &  6.89e+02 &    2 &           &           & displac  \\ 
 \hdashline 
     &           &    1 &  9.08e-03 &  3.00e-01 &      \\ 
     &           &    2 &  0.00e+00 &  2.71e-20 &      \\ 
 691 &  6.90e+02 &    2 &           &           & displac  \\ 
 \hdashline 
     &           &    1 &  9.10e-03 &  3.00e-01 &      \\ 
     &           &    2 &  0.00e+00 &  2.71e-20 &      \\ 
 692 &  6.91e+02 &    2 &           &           & displac  \\ 
 \hdashline 
     &           &    1 &  9.11e-03 &  3.00e-01 &      \\ 
     &           &    2 &  0.00e+00 &  0.00e+00 &      \\ 
 693 &  6.92e+02 &    2 &           &           & displac  \\ 
 \hdashline 
     &           &    1 &  9.12e-03 &  3.00e-01 &      \\ 
     &           &    2 &  0.00e+00 &  0.00e+00 &      \\ 
 694 &  6.93e+02 &    2 &           &           & displac  \\ 
 \hdashline 
     &           &    1 &  9.14e-03 &  3.00e-01 &      \\ 
     &           &    2 &  0.00e+00 &  0.00e+00 &      \\ 
 695 &  6.94e+02 &    2 &           &           & displac  \\ 
 \hdashline 
     &           &    1 &  9.15e-03 &  3.00e-01 &      \\ 
     &           &    2 &  0.00e+00 &  2.71e-20 &      \\ 
 696 &  6.95e+02 &    2 &           &           & displac  \\ 
 \hdashline 
     &           &    1 &  4.08e+02 &  3.00e-01 &      \\ 
     &           &    2 &  0.00e+00 &  0.00e+00 &      \\ 
 697 &  6.96e+02 &    2 &           &           & displac  \\ 
 \hdashline 
     &           &    1 &  9.18e-03 &  3.00e-01 &      \\ 
     &           &    2 &  4.55e-13 &  1.17e-14 &      \\ 
 698 &  6.97e+02 &    2 &           &           & forces  \\ 
 \hdashline 
     &           &    1 &  9.19e-03 &  3.00e-01 &      \\ 
     &           &    2 &  4.55e-13 &  1.17e-14 &      \\ 
 699 &  6.98e+02 &    2 &           &           & forces  \\ 
 \hdashline 
     &           &    1 &  4.16e+02 &  3.00e-01 &      \\ 
     &           &    2 &  0.00e+00 &  0.00e+00 &      \\ 
 700 &  6.99e+02 &    2 &           &           & displac  \\ 
 \hdashline 
     &           &    1 &  9.22e-03 &  3.00e-01 &      \\ 
     &           &    2 &  0.00e+00 &  0.00e+00 &      \\ 
 701 &  7.00e+02 &    2 &           &           & displac  \\ 
 \hdashline 
     &           &    1 &  9.23e-03 &  3.00e-01 &      \\ 
     &           &    2 &  0.00e+00 &  0.00e+00 &      \\ 
 702 &  7.01e+02 &    2 &           &           & displac  \\ 
 \hdashline 
     &           &    1 &  4.24e+02 &  3.00e-01 &      \\ 
     &           &    2 &  0.00e+00 &  0.00e+00 &      \\ 
 703 &  7.02e+02 &    2 &           &           & displac  \\ 
 \hdashline 
     &           &    1 &  9.25e-03 &  3.00e-01 &      \\ 
     &           &    2 &  0.00e+00 &  2.33e-14 &      \\ 
 704 &  7.03e+02 &    2 &           &           & displac  \\ 
 \hdashline 
     &           &    1 &  9.27e-03 &  3.00e-01 &      \\ 
     &           &    2 &  0.00e+00 &  2.32e-14 &      \\ 
 705 &  7.04e+02 &    2 &           &           & displac  \\ 
 \hdashline 
     &           &    1 &  4.32e+02 &  3.00e-01 &      \\ 
     &           &    2 &  0.00e+00 &  0.00e+00 &      \\ 
 706 &  7.05e+02 &    2 &           &           & displac  \\ 
 \hdashline 
     &           &    1 &  9.29e-03 &  3.00e-01 &      \\ 
     &           &    2 &  0.00e+00 &  0.00e+00 &      \\ 
 707 &  7.06e+02 &    2 &           &           & displac  \\ 
 \hdashline 
     &           &    1 &  4.37e+02 &  3.00e-01 &      \\ 
     &           &    2 &  0.00e+00 &  0.00e+00 &      \\ 
 708 &  7.07e+02 &    2 &           &           & displac  \\ 
 \hdashline 
     &           &    1 &  9.31e-03 &  3.00e-01 &      \\ 
     &           &    2 &  0.00e+00 &  0.00e+00 &      \\ 
 709 &  7.08e+02 &    2 &           &           & displac  \\ 
 \hdashline 
     &           &    1 &  4.43e+02 &  3.00e-01 &      \\ 
     &           &    2 &  0.00e+00 &  0.00e+00 &      \\ 
 710 &  7.09e+02 &    2 &           &           & displac  \\ 
 \hdashline 
     &           &    1 &  9.33e-03 &  3.00e-01 &      \\ 
     &           &    2 &  4.55e-13 &  2.30e-14 &      \\ 
 711 &  7.10e+02 &    2 &           &           & forces  \\ 
 \hdashline 
     &           &    1 &  4.48e+02 &  3.00e-01 &      \\ 
     &           &    2 &  0.00e+00 &  0.00e+00 &      \\ 
 712 &  7.11e+02 &    2 &           &           & displac  \\ 
 \hdashline 
     &           &    1 &  9.36e-03 &  3.00e-01 &      \\ 
     &           &    2 &  0.00e+00 &  0.00e+00 &      \\ 
 713 &  7.12e+02 &    2 &           &           & displac  \\ 
 \hdashline 
     &           &    1 &  4.54e+02 &  3.00e-01 &      \\ 
     &           &    2 &  0.00e+00 &  0.00e+00 &      \\ 
 714 &  7.13e+02 &    2 &           &           & displac  \\ 
 \hdashline 
     &           &    1 &  9.38e-03 &  3.00e-01 &      \\ 
     &           &    2 &  0.00e+00 &  2.29e-14 &      \\ 
 715 &  7.14e+02 &    2 &           &           & displac  \\ 
 \hdashline 
     &           &    1 &  4.59e+02 &  3.00e-01 &      \\ 
     &           &    2 &  0.00e+00 &  0.00e+00 &      \\ 
 716 &  7.15e+02 &    2 &           &           & displac  \\ 
 \hdashline 
     &           &    1 &  9.40e-03 &  3.00e-01 &      \\ 
     &           &    2 &  0.00e+00 &  0.00e+00 &      \\ 
 717 &  7.16e+02 &    2 &           &           & displac  \\ 
 \hdashline 
     &           &    1 &  9.41e-03 &  3.00e-01 &      \\ 
     &           &    2 &  0.00e+00 &  2.28e-14 &      \\ 
 718 &  7.17e+02 &    2 &           &           & displac  \\ 
 \hdashline 
     &           &    1 &  9.42e-03 &  3.00e-01 &      \\ 
     &           &    2 &  0.00e+00 &  0.00e+00 &      \\ 
 719 &  7.18e+02 &    2 &           &           & displac  \\ 
 \hdashline 
     &           &    1 &  4.70e+02 &  3.00e-01 &      \\ 
     &           &    2 &  0.00e+00 &  0.00e+00 &      \\ 
 720 &  7.19e+02 &    2 &           &           & displac  \\ 
 \hdashline 
     &           &    1 &  9.44e-03 &  3.00e-01 &      \\ 
     &           &    2 &  0.00e+00 &  2.71e-20 &      \\ 
 721 &  7.20e+02 &    2 &           &           & displac  \\ 
 \hdashline 
     &           &    1 &  9.45e-03 &  3.00e-01 &      \\ 
     &           &    2 &  0.00e+00 &  0.00e+00 &      \\ 
 722 &  7.21e+02 &    2 &           &           & displac  \\ 
 \hdashline 
     &           &    1 &  9.46e-03 &  3.00e-01 &      \\ 
     &           &    2 &  0.00e+00 &  2.71e-20 &      \\ 
 723 &  7.22e+02 &    2 &           &           & displac  \\ 
 \hdashline 
     &           &    1 &  9.47e-03 &  3.00e-01 &      \\ 
     &           &    2 &  0.00e+00 &  2.71e-20 &      \\ 
 724 &  7.23e+02 &    2 &           &           & displac  \\ 
 \hdashline 
     &           &    1 &  4.84e+02 &  3.00e-01 &      \\ 
     &           &    2 &  0.00e+00 &  0.00e+00 &      \\ 
 725 &  7.24e+02 &    2 &           &           & displac  \\ 
 \hdashline 
     &           &    1 &  9.48e-03 &  3.00e-01 &      \\ 
     &           &    2 &  0.00e+00 &  2.25e-14 &      \\ 
 726 &  7.25e+02 &    2 &           &           & displac  \\ 
 \hdashline 
     &           &    1 &  9.49e-03 &  3.00e-01 &      \\ 
     &           &    2 &  4.55e-13 &  1.12e-14 &      \\ 
 727 &  7.26e+02 &    2 &           &           & forces  \\ 
 \hdashline 
     &           &    1 &  4.92e+02 &  3.00e-01 &      \\ 
     &           &    2 &  0.00e+00 &  0.00e+00 &      \\ 
 728 &  7.27e+02 &    2 &           &           & displac  \\ 
 \hdashline 
     &           &    1 &  9.51e-03 &  3.00e-01 &      \\ 
     &           &    2 &  0.00e+00 &  0.00e+00 &      \\ 
 729 &  7.28e+02 &    2 &           &           & displac  \\ 
 \hdashline 
     &           &    1 &  9.52e-03 &  3.00e-01 &      \\ 
     &           &    2 &  0.00e+00 &  0.00e+00 &      \\ 
 730 &  7.29e+02 &    2 &           &           & displac  \\ 
 \hdashline 
     &           &    1 &  5.00e+02 &  3.00e-01 &      \\ 
     &           &    2 &  0.00e+00 &  0.00e+00 &      \\ 
 731 &  7.30e+02 &    2 &           &           & displac  \\ 
 \hdashline 
     &           &    1 &  9.53e-03 &  3.00e-01 &      \\ 
     &           &    2 &  4.55e-13 &  2.23e-14 &      \\ 
 732 &  7.31e+02 &    2 &           &           & forces  \\ 
 \hdashline 
     &           &    1 &  5.06e+02 &  3.00e-01 &      \\ 
     &           &    2 &  0.00e+00 &  0.00e+00 &      \\ 
 733 &  7.32e+02 &    2 &           &           & displac  \\ 
 \hdashline 
     &           &    1 &  9.55e-03 &  3.00e-01 &      \\ 
     &           &    2 &  0.00e+00 &  0.00e+00 &      \\ 
 734 &  7.33e+02 &    2 &           &           & displac  \\ 
 \hdashline 
     &           &    1 &  9.56e-03 &  3.00e-01 &      \\ 
     &           &    2 &  0.00e+00 &  0.00e+00 &      \\ 
 735 &  7.34e+02 &    2 &           &           & displac  \\ 
 \hdashline 
     &           &    1 &  9.56e-03 &  3.00e-01 &      \\ 
     &           &    2 &  0.00e+00 &  0.00e+00 &      \\ 
 736 &  7.35e+02 &    2 &           &           & displac  \\ 
 \hdashline 
     &           &    1 &  9.57e-03 &  3.00e-01 &      \\ 
     &           &    2 &  0.00e+00 &  2.21e-14 &      \\ 
 737 &  7.36e+02 &    2 &           &           & displac  \\ 
 \hdashline 
     &           &    1 &  9.58e-03 &  3.00e-01 &      \\ 
     &           &    2 &  0.00e+00 &  0.00e+00 &      \\ 
 738 &  7.37e+02 &    2 &           &           & displac  \\ 
 \hdashline 
     &           &    1 &  9.59e-03 &  3.00e-01 &      \\ 
     &           &    2 &  0.00e+00 &  0.00e+00 &      \\ 
 739 &  7.38e+02 &    2 &           &           & displac  \\ 
 \hdashline 
     &           &    1 &  9.59e-03 &  3.00e-01 &      \\ 
     &           &    2 &  0.00e+00 &  0.00e+00 &      \\ 
 740 &  7.39e+02 &    2 &           &           & displac  \\ 
 \hdashline 
     &           &    1 &  9.60e-03 &  3.00e-01 &      \\ 
     &           &    2 &  0.00e+00 &  0.00e+00 &      \\ 
 741 &  7.40e+02 &    2 &           &           & displac  \\ 
 \hdashline 
     &           &    1 &  9.61e-03 &  3.00e-01 &      \\ 
     &           &    2 &  4.55e-13 &  1.09e-14 &      \\ 
 742 &  7.41e+02 &    2 &           &           & forces  \\ 
 \hdashline 
     &           &    1 &  9.61e-03 &  3.00e-01 &      \\ 
     &           &    2 &  4.55e-13 &  1.09e-14 &      \\ 
 743 &  7.42e+02 &    2 &           &           & forces  \\ 
 \hdashline 
     &           &    1 &  9.62e-03 &  3.00e-01 &      \\ 
     &           &    2 &  4.55e-13 &  1.09e-14 &      \\ 
 744 &  7.43e+02 &    2 &           &           & forces  \\ 
 \hdashline 
     &           &    1 &  9.62e-03 &  3.00e-01 &      \\ 
     &           &    2 &  4.55e-13 &  1.09e-14 &      \\ 
 745 &  7.44e+02 &    2 &           &           & forces  \\ 
 \hdashline 
     &           &    1 &  9.63e-03 &  3.00e-01 &      \\ 
     &           &    2 &  4.55e-13 &  1.09e-14 &      \\ 
 746 &  7.45e+02 &    2 &           &           & forces  \\ 
 \hdashline 
     &           &    1 &  9.64e-03 &  3.00e-01 &      \\ 
     &           &    2 &  4.55e-13 &  1.09e-14 &      \\ 
 747 &  7.46e+02 &    2 &           &           & forces  \\ 
 \hdashline 
     &           &    1 &  9.64e-03 &  3.00e-01 &      \\ 
     &           &    2 &  4.55e-13 &  1.08e-14 &      \\ 
 748 &  7.47e+02 &    2 &           &           & forces  \\ 
 \hdashline 
     &           &    1 &  5.51e+02 &  3.00e-01 &      \\ 
     &           &    2 &  0.00e+00 &  0.00e+00 &      \\ 
 749 &  7.48e+02 &    2 &           &           & displac  \\ 
 \hdashline 
     &           &    1 &  9.65e-03 &  3.00e-01 &      \\ 
     &           &    2 &  4.55e-13 &  2.16e-14 &      \\ 
 750 &  7.49e+02 &    2 &           &           & forces  \\ 
 \hdashline 
     &           &    1 &  9.66e-03 &  3.00e-01 &      \\ 
     &           &    2 &  4.55e-13 &  1.08e-14 &      \\ 
 751 &  7.50e+02 &    2 &           &           & forces  \\ 
 \hdashline 
     &           &    1 &  5.60e+02 &  3.00e-01 &      \\ 
     &           &    2 &  0.00e+00 &  0.00e+00 &      \\ 
 752 &  7.51e+02 &    2 &           &           & displac  \\ 
 \hdashline 
     &           &    1 &  9.67e-03 &  3.00e-01 &      \\ 
     &           &    2 &  0.00e+00 &  2.71e-20 &      \\ 
 753 &  7.52e+02 &    2 &           &           & displac  \\ 
 \hdashline 
     &           &    1 &  9.67e-03 &  3.00e-01 &      \\ 
     &           &    2 &  0.00e+00 &  0.00e+00 &      \\ 
 754 &  7.53e+02 &    2 &           &           & displac  \\ 
 \hdashline 
     &           &    1 &  5.68e+02 &  3.00e-01 &      \\ 
     &           &    2 &  0.00e+00 &  0.00e+00 &      \\ 
 755 &  7.54e+02 &    2 &           &           & displac  \\ 
 \hdashline 
     &           &    1 &  9.68e-03 &  3.00e-01 &      \\ 
     &           &    2 &  4.55e-13 &  1.07e-14 &      \\ 
 756 &  7.55e+02 &    2 &           &           & forces  \\ 
 \hdashline 
     &           &    1 &  9.69e-03 &  3.00e-01 &      \\ 
     &           &    2 &  4.55e-13 &  1.07e-14 &      \\ 
 757 &  7.56e+02 &    2 &           &           & forces  \\ 
 \hdashline 
     &           &    1 &  9.69e-03 &  3.00e-01 &      \\ 
     &           &    2 &  4.55e-13 &  1.07e-14 &      \\ 
 758 &  7.57e+02 &    2 &           &           & forces  \\ 
 \hdashline 
     &           &    1 &  9.69e-03 &  3.00e-01 &      \\ 
     &           &    2 &  4.55e-13 &  1.07e-14 &      \\ 
 759 &  7.58e+02 &    2 &           &           & forces  \\ 
 \hdashline 
     &           &    1 &  9.70e-03 &  3.00e-01 &      \\ 
     &           &    2 &  4.55e-13 &  1.07e-14 &      \\ 
 760 &  7.59e+02 &    2 &           &           & forces  \\ 
 \hdashline 
     &           &    1 &  9.70e-03 &  3.00e-01 &      \\ 
     &           &    2 &  4.55e-13 &  1.07e-14 &      \\ 
 761 &  7.60e+02 &    2 &           &           & forces  \\ 
 \hdashline 
     &           &    1 &  9.71e-03 &  3.00e-01 &      \\ 
     &           &    2 &  4.55e-13 &  1.06e-14 &      \\ 
 762 &  7.61e+02 &    2 &           &           & forces  \\ 
 \hdashline 
     &           &    1 &  9.71e-03 &  3.00e-01 &      \\ 
     &           &    2 &  4.55e-13 &  1.06e-14 &      \\ 
 763 &  7.62e+02 &    2 &           &           & forces  \\ 
 \hdashline 
     &           &    1 &  5.94e+02 &  3.00e-01 &      \\ 
     &           &    2 &  0.00e+00 &  0.00e+00 &      \\ 
 764 &  7.63e+02 &    2 &           &           & displac  \\ 
 \hdashline 
     &           &    1 &  9.72e-03 &  3.00e-01 &      \\ 
     &           &    2 &  0.00e+00 &  2.71e-20 &      \\ 
 765 &  7.64e+02 &    2 &           &           & displac  \\ 
 \hdashline 
     &           &    1 &  6.00e+02 &  3.00e-01 &      \\ 
     &           &    2 &  0.00e+00 &  0.00e+00 &      \\ 
 766 &  7.65e+02 &    2 &           &           & displac  \\ 
 \hdashline 
     &           &    1 &  9.72e-03 &  3.00e-01 &      \\ 
     &           &    2 &  0.00e+00 &  0.00e+00 &      \\ 
 767 &  7.66e+02 &    2 &           &           & displac  \\ 
 \hdashline 
     &           &    1 &  9.73e-03 &  3.00e-01 &      \\ 
     &           &    2 &  0.00e+00 &  2.71e-20 &      \\ 
 768 &  7.67e+02 &    2 &           &           & displac  \\ 
 \hdashline 
     &           &    1 &  9.73e-03 &  3.00e-01 &      \\ 
     &           &    2 &  0.00e+00 &  2.71e-20 &      \\ 
 769 &  7.68e+02 &    2 &           &           & displac  \\ 
 \hdashline 
     &           &    1 &  9.73e-03 &  3.00e-01 &      \\ 
     &           &    2 &  0.00e+00 &  2.71e-20 &      \\ 
 770 &  7.69e+02 &    2 &           &           & displac  \\ 
 \hdashline 
     &           &    1 &  9.73e-03 &  3.00e-01 &      \\ 
     &           &    2 &  0.00e+00 &  2.71e-20 &      \\ 
 771 &  7.70e+02 &    2 &           &           & displac  \\ 
 \hdashline 
     &           &    1 &  9.74e-03 &  3.00e-01 &      \\ 
     &           &    2 &  0.00e+00 &  0.00e+00 &      \\ 
 772 &  7.71e+02 &    2 &           &           & displac  \\ 
 \hdashline 
     &           &    1 &  9.74e-03 &  3.00e-01 &      \\ 
     &           &    2 &  0.00e+00 &  0.00e+00 &      \\ 
 773 &  7.72e+02 &    2 &           &           & displac  \\ 
 \hdashline 
     &           &    1 &  9.74e-03 &  3.00e-01 &      \\ 
     &           &    2 &  0.00e+00 &  0.00e+00 &      \\ 
 774 &  7.73e+02 &    2 &           &           & displac  \\ 
 \hdashline 
     &           &    1 &  9.74e-03 &  3.00e-01 &      \\ 
     &           &    2 &  0.00e+00 &  2.08e-14 &      \\ 
 775 &  7.74e+02 &    2 &           &           & displac  \\ 
 \hdashline 
     &           &    1 &  9.75e-03 &  3.00e-01 &      \\ 
     &           &    2 &  0.00e+00 &  0.00e+00 &      \\ 
 776 &  7.75e+02 &    2 &           &           & displac  \\ 
 \hdashline 
     &           &    1 &  6.32e+02 &  3.00e-01 &      \\ 
     &           &    2 &  0.00e+00 &  1.78e-15 &      \\ 
 777 &  7.76e+02 &    2 &           &           & displac  \\ 
 \hdashline 
     &           &    1 &  9.75e-03 &  3.00e-01 &      \\ 
     &           &    2 &  0.00e+00 &  2.08e-14 &      \\ 
 778 &  7.77e+02 &    2 &           &           & displac  \\ 
 \hdashline 
     &           &    1 &  6.38e+02 &  3.00e-01 &      \\ 
     &           &    2 &  0.00e+00 &  0.00e+00 &      \\ 
 779 &  7.78e+02 &    2 &           &           & displac  \\ 
 \hdashline 
     &           &    1 &  9.75e-03 &  3.00e-01 &      \\ 
     &           &    2 &  0.00e+00 &  0.00e+00 &      \\ 
 780 &  7.79e+02 &    2 &           &           & displac  \\ 
 \hdashline 
     &           &    1 &  9.75e-03 &  3.00e-01 &      \\ 
     &           &    2 &  0.00e+00 &  0.00e+00 &      \\ 
 781 &  7.80e+02 &    2 &           &           & displac  \\ 
 \hdashline 
     &           &    1 &  9.75e-03 &  3.00e-01 &      \\ 
     &           &    2 &  0.00e+00 &  0.00e+00 &      \\ 
 782 &  7.81e+02 &    2 &           &           & displac  \\ 
 \hdashline 
     &           &    1 &  9.76e-03 &  3.00e-01 &      \\ 
     &           &    2 &  0.00e+00 &  0.00e+00 &      \\ 
 783 &  7.82e+02 &    2 &           &           & displac  \\ 
 \hdashline 
     &           &    1 &  9.76e-03 &  3.00e-01 &      \\ 
     &           &    2 &  0.00e+00 &  0.00e+00 &      \\ 
 784 &  7.83e+02 &    2 &           &           & displac  \\ 
 \hdashline 
     &           &    1 &  9.76e-03 &  3.00e-01 &      \\ 
     &           &    2 &  0.00e+00 &  2.71e-20 &      \\ 
 785 &  7.84e+02 &    2 &           &           & displac  \\ 
 \hdashline 
     &           &    1 &  9.76e-03 &  3.00e-01 &      \\ 
     &           &    2 &  0.00e+00 &  0.00e+00 &      \\ 
 786 &  7.85e+02 &    2 &           &           & displac  \\ 
 \hdashline 
     &           &    1 &  9.76e-03 &  3.00e-01 &      \\ 
     &           &    2 &  4.55e-13 &  1.02e-14 &      \\ 
 787 &  7.86e+02 &    2 &           &           & forces  \\ 
 \hdashline 
     &           &    1 &  6.64e+02 &  3.00e-01 &      \\ 
     &           &    2 &  0.00e+00 &  0.00e+00 &      \\ 
 788 &  7.87e+02 &    2 &           &           & displac  \\ 
 \hdashline 
     &           &    1 &  9.76e-03 &  3.00e-01 &      \\ 
     &           &    2 &  0.00e+00 &  0.00e+00 &      \\ 
 789 &  7.88e+02 &    2 &           &           & displac  \\ 
 \hdashline 
     &           &    1 &  9.76e-03 &  3.00e-01 &      \\ 
     &           &    2 &  0.00e+00 &  0.00e+00 &      \\ 
 790 &  7.89e+02 &    2 &           &           & displac  \\ 
 \hdashline 
     &           &    1 &  9.76e-03 &  3.00e-01 &      \\ 
     &           &    2 &  0.00e+00 &  2.71e-20 &      \\ 
 791 &  7.90e+02 &    2 &           &           & displac  \\ 
 \hdashline 
     &           &    1 &  9.76e-03 &  3.00e-01 &      \\ 
     &           &    2 &  0.00e+00 &  0.00e+00 &      \\ 
 792 &  7.91e+02 &    2 &           &           & displac  \\ 
 \hdashline 
     &           &    1 &  9.76e-03 &  3.00e-01 &      \\ 
     &           &    2 &  0.00e+00 &  0.00e+00 &      \\ 
 793 &  7.92e+02 &    2 &           &           & displac  \\ 
 \hdashline 
     &           &    1 &  6.82e+02 &  3.00e-01 &      \\ 
     &           &    2 &  0.00e+00 &  0.00e+00 &      \\ 
 794 &  7.93e+02 &    2 &           &           & displac  \\ 
 \hdashline 
     &           &    1 &  9.76e-03 &  3.00e-01 &      \\ 
     &           &    2 &  0.00e+00 &  0.00e+00 &      \\ 
 795 &  7.94e+02 &    2 &           &           & displac  \\ 
 \hdashline 
     &           &    1 &  9.76e-03 &  3.00e-01 &      \\ 
     &           &    2 &  0.00e+00 &  0.00e+00 &      \\ 
 796 &  7.95e+02 &    2 &           &           & displac  \\ 
 \hdashline 
     &           &    1 &  9.76e-03 &  3.00e-01 &      \\ 
     &           &    2 &  0.00e+00 &  0.00e+00 &      \\ 
 797 &  7.96e+02 &    2 &           &           & displac  \\ 
 \hdashline 
     &           &    1 &  9.76e-03 &  3.00e-01 &      \\ 
     &           &    2 &  1.36e-12 &  2.02e-14 &      \\ 
 798 &  7.97e+02 &    2 &           &           & forces  \\ 
 \hdashline 
     &           &    1 &  6.97e+02 &  3.00e-01 &      \\ 
     &           &    2 &  0.00e+00 &  0.00e+00 &      \\ 
 799 &  7.98e+02 &    2 &           &           & displac  \\ 
 \hdashline 
     &           &    1 &  7.00e+02 &  3.00e-01 &      \\ 
     &           &    2 &  0.00e+00 &  0.00e+00 &      \\ 
 800 &  7.99e+02 &    2 &           &           & displac  \\ 
 \hdashline 
     &           &    1 &  9.76e-03 &  3.00e-01 &      \\ 
     &           &    2 &  0.00e+00 &  2.71e-20 &      \\ 
 801 &  8.00e+02 &    2 &           &           & displac  \\ 
 \hdashline 
     &           &    1 &  9.75e-03 &  3.00e-01 &      \\ 
     &           &    2 &  0.00e+00 &  0.00e+00 &      \\ 
 802 &  8.01e+02 &    2 &           &           & displac  \\ 
 \hdashline 
     &           &    1 &  9.75e-03 &  3.00e-01 &      \\ 
     &           &    2 &  0.00e+00 &  0.00e+00 &      \\ 
 803 &  8.02e+02 &    2 &           &           & displac  \\ 
 \hdashline 
     &           &    1 &  9.75e-03 &  3.00e-01 &      \\ 
     &           &    2 &  0.00e+00 &  0.00e+00 &      \\ 
 804 &  8.03e+02 &    2 &           &           & displac  \\ 
 \hdashline 
     &           &    1 &  9.75e-03 &  3.00e-01 &      \\ 
     &           &    2 &  0.00e+00 &  0.00e+00 &      \\ 
 805 &  8.04e+02 &    2 &           &           & displac  \\ 
 \hdashline 
     &           &    1 &  9.75e-03 &  3.00e-01 &      \\ 
     &           &    2 &  0.00e+00 &  2.71e-20 &      \\ 
 806 &  8.05e+02 &    2 &           &           & displac  \\ 
 \hdashline 
     &           &    1 &  9.75e-03 &  3.00e-01 &      \\ 
     &           &    2 &  0.00e+00 &  0.00e+00 &      \\ 
 807 &  8.06e+02 &    2 &           &           & displac  \\ 
 \hdashline 
     &           &    1 &  9.75e-03 &  3.00e-01 &      \\ 
     &           &    2 &  0.00e+00 &  0.00e+00 &      \\ 
 808 &  8.07e+02 &    2 &           &           & displac  \\ 
 \hdashline 
     &           &    1 &  9.74e-03 &  3.00e-01 &      \\ 
     &           &    2 &  0.00e+00 &  0.00e+00 &      \\ 
 809 &  8.08e+02 &    2 &           &           & displac  \\ 
 \hdashline 
     &           &    1 &  9.74e-03 &  3.00e-01 &      \\ 
     &           &    2 &  0.00e+00 &  0.00e+00 &      \\ 
 810 &  8.09e+02 &    2 &           &           & displac  \\ 
 \hdashline 
     &           &    1 &  9.74e-03 &  3.00e-01 &      \\ 
     &           &    2 &  0.00e+00 &  2.71e-20 &      \\ 
 811 &  8.10e+02 &    2 &           &           & displac  \\ 
 \hdashline 
     &           &    1 &  9.74e-03 &  3.00e-01 &      \\ 
     &           &    2 &  0.00e+00 &  0.00e+00 &      \\ 
 812 &  8.11e+02 &    2 &           &           & displac  \\ 
 \hdashline 
     &           &    1 &  9.73e-03 &  3.00e-01 &      \\ 
     &           &    2 &  0.00e+00 &  1.97e-14 &      \\ 
 813 &  8.12e+02 &    2 &           &           & displac  \\ 
 \hdashline 
     &           &    1 &  7.41e+02 &  3.00e-01 &      \\ 
     &           &    2 &  0.00e+00 &  0.00e+00 &      \\ 
 814 &  8.13e+02 &    2 &           &           & displac  \\ 
 \hdashline 
     &           &    1 &  9.73e-03 &  3.00e-01 &      \\ 
     &           &    2 &  0.00e+00 &  0.00e+00 &      \\ 
 815 &  8.14e+02 &    2 &           &           & displac  \\ 
 \hdashline 
     &           &    1 &  9.73e-03 &  3.00e-01 &      \\ 
     &           &    2 &  0.00e+00 &  0.00e+00 &      \\ 
 816 &  8.15e+02 &    2 &           &           & displac  \\ 
 \hdashline 
     &           &    1 &  9.72e-03 &  3.00e-01 &      \\ 
     &           &    2 &  0.00e+00 &  0.00e+00 &      \\ 
 817 &  8.16e+02 &    2 &           &           & displac  \\ 
 \hdashline 
     &           &    1 &  9.72e-03 &  3.00e-01 &      \\ 
     &           &    2 &  0.00e+00 &  0.00e+00 &      \\ 
 818 &  8.17e+02 &    2 &           &           & displac  \\ 
 \hdashline 
     &           &    1 &  7.56e+02 &  3.00e-01 &      \\ 
     &           &    2 &  0.00e+00 &  0.00e+00 &      \\ 
 819 &  8.18e+02 &    2 &           &           & displac  \\ 
 \hdashline 
     &           &    1 &  9.72e-03 &  3.00e-01 &      \\ 
     &           &    2 &  0.00e+00 &  2.71e-20 &      \\ 
 820 &  8.19e+02 &    2 &           &           & displac  \\ 
 \hdashline 
     &           &    1 &  9.71e-03 &  3.00e-01 &      \\ 
     &           &    2 &  9.09e-13 &  1.95e-14 &      \\ 
 821 &  8.20e+02 &    2 &           &           & forces  \\ 
 \hdashline 
     &           &    1 &  7.65e+02 &  3.00e-01 &      \\ 
     &           &    2 &  0.00e+00 &  0.00e+00 &      \\ 
 822 &  8.21e+02 &    2 &           &           & displac  \\ 
 \hdashline 
     &           &    1 &  9.71e-03 &  3.00e-01 &      \\ 
     &           &    2 &  0.00e+00 &  0.00e+00 &      \\ 
 823 &  8.22e+02 &    2 &           &           & displac  \\ 
 \hdashline 
     &           &    1 &  9.70e-03 &  3.00e-01 &      \\ 
     &           &    2 &  0.00e+00 &  0.00e+00 &      \\ 
 824 &  8.23e+02 &    2 &           &           & displac  \\ 
 \hdashline 
     &           &    1 &  9.70e-03 &  3.00e-01 &      \\ 
     &           &    2 &  9.09e-13 &  1.94e-14 &      \\ 
 825 &  8.24e+02 &    2 &           &           & forces  \\ 
 \hdashline 
     &           &    1 &  7.77e+02 &  3.00e-01 &      \\ 
     &           &    2 &  0.00e+00 &  0.00e+00 &      \\ 
 826 &  8.25e+02 &    2 &           &           & displac  \\ 
 \hdashline 
     &           &    1 &  9.69e-03 &  3.00e-01 &      \\ 
     &           &    2 &  9.09e-13 &  1.93e-14 &      \\ 
 827 &  8.26e+02 &    2 &           &           & forces  \\ 
 \hdashline 
     &           &    1 &  9.69e-03 &  3.00e-01 &      \\ 
     &           &    2 &  9.09e-13 &  1.93e-14 &      \\ 
 828 &  8.27e+02 &    2 &           &           & forces  \\ 
 \hdashline 
     &           &    1 &  9.68e-03 &  3.00e-01 &      \\ 
     &           &    2 &  0.00e+00 &  1.93e-14 &      \\ 
 829 &  8.28e+02 &    2 &           &           & displac  \\ 
 \hdashline 
     &           &    1 &  9.68e-03 &  3.00e-01 &      \\ 
     &           &    2 &  0.00e+00 &  0.00e+00 &      \\ 
 830 &  8.29e+02 &    2 &           &           & displac  \\ 
 \hdashline 
     &           &    1 &  9.68e-03 &  3.00e-01 &      \\ 
     &           &    2 &  0.00e+00 &  2.71e-20 &      \\ 
 831 &  8.30e+02 &    2 &           &           & displac  \\ 
 \hdashline 
     &           &    1 &  9.67e-03 &  3.00e-01 &      \\ 
     &           &    2 &  0.00e+00 &  0.00e+00 &      \\ 
 832 &  8.31e+02 &    2 &           &           & displac  \\ 
 \hdashline 
     &           &    1 &  9.67e-03 &  3.00e-01 &      \\ 
     &           &    2 &  9.09e-13 &  1.92e-14 &      \\ 
 833 &  8.32e+02 &    2 &           &           & forces  \\ 
 \hdashline 
     &           &    1 &  8.01e+02 &  3.00e-01 &      \\ 
     &           &    2 &  0.00e+00 &  0.00e+00 &      \\ 
 834 &  8.33e+02 &    2 &           &           & displac  \\ 
 \hdashline 
     &           &    1 &  9.66e-03 &  3.00e-01 &      \\ 
     &           &    2 &  9.09e-13 &  1.92e-14 &      \\ 
 835 &  8.34e+02 &    2 &           &           & forces  \\ 
 \hdashline 
     &           &    1 &  9.65e-03 &  3.00e-01 &      \\ 
     &           &    2 &  1.82e-12 &  1.91e-14 &      \\ 
 836 &  8.35e+02 &    2 &           &           & forces  \\ 
 \hdashline 
     &           &    1 &  9.65e-03 &  3.00e-01 &      \\ 
     &           &    2 &  0.00e+00 &  3.81e-14 &      \\ 
 837 &  8.36e+02 &    2 &           &           & displac  \\ 
 \hdashline 
     &           &    1 &  9.64e-03 &  3.00e-01 &      \\ 
     &           &    2 &  0.00e+00 &  0.00e+00 &      \\ 
 838 &  8.37e+02 &    2 &           &           & displac  \\ 
 \hdashline 
     &           &    1 &  9.64e-03 &  3.00e-01 &      \\ 
     &           &    2 &  0.00e+00 &  0.00e+00 &      \\ 
 839 &  8.38e+02 &    2 &           &           & displac  \\ 
 \hdashline 
     &           &    1 &  8.19e+02 &  3.00e-01 &      \\ 
     &           &    2 &  0.00e+00 &  0.00e+00 &      \\ 
 840 &  8.39e+02 &    2 &           &           & displac  \\ 
 \hdashline 
     &           &    1 &  9.63e-03 &  3.00e-01 &      \\ 
     &           &    2 &  0.00e+00 &  0.00e+00 &      \\ 
 841 &  8.40e+02 &    2 &           &           & displac  \\ 
 \hdashline 
     &           &    1 &  9.63e-03 &  3.00e-01 &      \\ 
     &           &    2 &  9.09e-13 &  1.89e-14 &      \\ 
 842 &  8.41e+02 &    2 &           &           & forces  \\ 
 \hdashline 
     &           &    1 &  9.62e-03 &  3.00e-01 &      \\ 
     &           &    2 &  9.09e-13 &  1.89e-14 &      \\ 
 843 &  8.42e+02 &    2 &           &           & forces  \\ 
 \hdashline 
     &           &    1 &  9.61e-03 &  3.00e-01 &      \\ 
     &           &    2 &  1.82e-12 &  1.89e-14 &      \\ 
 844 &  8.43e+02 &    2 &           &           & forces  \\ 
 \hdashline 
     &           &    1 &  8.34e+02 &  3.00e-01 &      \\ 
     &           &    2 &  0.00e+00 &  0.00e+00 &      \\ 
 845 &  8.44e+02 &    2 &           &           & displac  \\ 
 \hdashline 
     &           &    1 &  9.60e-03 &  3.00e-01 &      \\ 
     &           &    2 &  0.00e+00 &  1.89e-14 &      \\ 
 846 &  8.45e+02 &    2 &           &           & displac  \\ 
 \hdashline 
     &           &    1 &  9.60e-03 &  3.00e-01 &      \\ 
     &           &    2 &  0.00e+00 &  0.00e+00 &      \\ 
 847 &  8.46e+02 &    2 &           &           & displac  \\ 
 \hdashline 
     &           &    1 &  9.59e-03 &  3.00e-01 &      \\ 
     &           &    2 &  0.00e+00 &  0.00e+00 &      \\ 
 848 &  8.47e+02 &    2 &           &           & displac  \\ 
 \hdashline 
     &           &    1 &  9.59e-03 &  3.00e-01 &      \\ 
     &           &    2 &  0.00e+00 &  0.00e+00 &      \\ 
 849 &  8.48e+02 &    2 &           &           & displac  \\ 
 \hdashline 
     &           &    1 &  9.58e-03 &  3.00e-01 &      \\ 
     &           &    2 &  0.00e+00 &  0.00e+00 &      \\ 
 850 &  8.49e+02 &    2 &           &           & displac  \\ 
 \hdashline 
     &           &    1 &  9.58e-03 &  3.00e-01 &      \\ 
     &           &    2 &  0.00e+00 &  0.00e+00 &      \\ 
 851 &  8.50e+02 &    2 &           &           & displac  \\ 
 \hdashline 
     &           &    1 &  9.57e-03 &  3.00e-01 &      \\ 
     &           &    2 &  9.09e-13 &  1.87e-14 &      \\ 
 852 &  8.51e+02 &    2 &           &           & forces  \\ 
 \hdashline 
     &           &    1 &  8.58e+02 &  3.00e-01 &      \\ 
     &           &    2 &  0.00e+00 &  0.00e+00 &      \\ 
 853 &  8.52e+02 &    2 &           &           & displac  \\ 
 \hdashline 
     &           &    1 &  9.56e-03 &  3.00e-01 &      \\ 
     &           &    2 &  0.00e+00 &  0.00e+00 &      \\ 
 854 &  8.53e+02 &    2 &           &           & displac  \\ 
 \hdashline 
     &           &    1 &  9.55e-03 &  3.00e-01 &      \\ 
     &           &    2 &  0.00e+00 &  0.00e+00 &      \\ 
 855 &  8.54e+02 &    2 &           &           & displac  \\ 
 \hdashline 
     &           &    1 &  9.55e-03 &  3.00e-01 &      \\ 
     &           &    2 &  0.00e+00 &  0.00e+00 &      \\ 
 856 &  8.55e+02 &    2 &           &           & displac  \\ 
 \hdashline 
     &           &    1 &  8.70e+02 &  3.00e-01 &      \\ 
     &           &    2 &  0.00e+00 &  0.00e+00 &      \\ 
 857 &  8.56e+02 &    2 &           &           & displac  \\ 
 \hdashline 
     &           &    1 &  9.53e-03 &  3.00e-01 &      \\ 
     &           &    2 &  9.09e-13 &  1.85e-14 &      \\ 
 858 &  8.57e+02 &    2 &           &           & forces  \\ 
 \hdashline 
     &           &    1 &  8.76e+02 &  3.00e-01 &      \\ 
     &           &    2 &  0.00e+00 &  0.00e+00 &      \\ 
 859 &  8.58e+02 &    2 &           &           & displac  \\ 
 \hdashline 
     &           &    1 &  9.52e-03 &  3.00e-01 &      \\ 
     &           &    2 &  9.09e-13 &  1.85e-14 &      \\ 
 860 &  8.59e+02 &    2 &           &           & forces  \\ 
 \hdashline 
     &           &    1 &  9.51e-03 &  3.00e-01 &      \\ 
     &           &    2 &  9.09e-13 &  1.85e-14 &      \\ 
 861 &  8.60e+02 &    2 &           &           & forces  \\ 
 \hdashline 
     &           &    1 &  8.85e+02 &  3.00e-01 &      \\ 
     &           &    2 &  0.00e+00 &  0.00e+00 &      \\ 
 862 &  8.61e+02 &    2 &           &           & displac  \\ 
 \hdashline 
     &           &    1 &  9.50e-03 &  3.00e-01 &      \\ 
     &           &    2 &  0.00e+00 &  2.71e-20 &      \\ 
 863 &  8.62e+02 &    2 &           &           & displac  \\ 
 \hdashline 
     &           &    1 &  9.49e-03 &  3.00e-01 &      \\ 
     &           &    2 &  9.09e-13 &  1.84e-14 &      \\ 
 864 &  8.63e+02 &    2 &           &           & forces  \\ 
 \hdashline 
     &           &    1 &  9.49e-03 &  3.00e-01 &      \\ 
     &           &    2 &  9.09e-13 &  1.84e-14 &      \\ 
 865 &  8.64e+02 &    2 &           &           & forces  \\ 
 \hdashline 
     &           &    1 &  9.48e-03 &  3.00e-01 &      \\ 
     &           &    2 &  9.09e-13 &  1.84e-14 &      \\ 
 866 &  8.65e+02 &    2 &           &           & forces  \\ 
 \hdashline 
     &           &    1 &  9.00e+02 &  3.00e-01 &      \\ 
     &           &    2 &  0.00e+00 &  0.00e+00 &      \\ 
 867 &  8.66e+02 &    2 &           &           & displac  \\ 
 \hdashline 
     &           &    1 &  9.46e-03 &  3.00e-01 &      \\ 
     &           &    2 &  0.00e+00 &  2.71e-20 &      \\ 
 868 &  8.67e+02 &    2 &           &           & displac  \\ 
 \hdashline 
     &           &    1 &  9.06e+02 &  3.00e-01 &      \\ 
     &           &    2 &  0.00e+00 &  0.00e+00 &      \\ 
 869 &  8.68e+02 &    2 &           &           & displac  \\ 
 \hdashline 
     &           &    1 &  9.45e-03 &  3.00e-01 &      \\ 
     &           &    2 &  9.09e-13 &  1.83e-14 &      \\ 
 870 &  8.69e+02 &    2 &           &           & forces  \\ 
 \hdashline 
     &           &    1 &  9.44e-03 &  3.00e-01 &      \\ 
     &           &    2 &  9.09e-13 &  1.83e-14 &      \\ 
 871 &  8.70e+02 &    2 &           &           & forces  \\ 
 \hdashline 
     &           &    1 &  9.43e-03 &  3.00e-01 &      \\ 
     &           &    2 &  9.09e-13 &  1.82e-14 &      \\ 
 872 &  8.71e+02 &    2 &           &           & forces  \\ 
 \hdashline 
     &           &    1 &  9.43e-03 &  3.00e-01 &      \\ 
     &           &    2 &  9.09e-13 &  1.82e-14 &      \\ 
 873 &  8.72e+02 &    2 &           &           & forces  \\ 
 \hdashline 
     &           &    1 &  9.42e-03 &  3.00e-01 &      \\ 
     &           &    2 &  9.09e-13 &  1.82e-14 &      \\ 
 874 &  8.73e+02 &    2 &           &           & forces  \\ 
 \hdashline 
     &           &    1 &  9.41e-03 &  3.00e-01 &      \\ 
     &           &    2 &  9.09e-13 &  1.82e-14 &      \\ 
 875 &  8.74e+02 &    2 &           &           & forces  \\ 
 \hdashline 
     &           &    1 &  9.40e-03 &  3.00e-01 &      \\ 
     &           &    2 &  9.09e-13 &  1.82e-14 &      \\ 
 876 &  8.75e+02 &    2 &           &           & forces  \\ 
 \hdashline 
     &           &    1 &  9.40e-03 &  3.00e-01 &      \\ 
     &           &    2 &  9.09e-13 &  1.81e-14 &      \\ 
 877 &  8.76e+02 &    2 &           &           & forces  \\ 
 \hdashline 
     &           &    1 &  9.39e-03 &  3.00e-01 &      \\ 
     &           &    2 &  9.09e-13 &  1.81e-14 &      \\ 
 878 &  8.77e+02 &    2 &           &           & forces  \\ 
 \hdashline 
     &           &    1 &  9.38e-03 &  3.00e-01 &      \\ 
     &           &    2 &  9.09e-13 &  1.80e-14 &      \\ 
 879 &  8.78e+02 &    2 &           &           & forces  \\ 
 \hdashline 
     &           &    1 &  9.37e-03 &  3.00e-01 &      \\ 
     &           &    2 &  9.09e-13 &  1.80e-14 &      \\ 
 880 &  8.79e+02 &    2 &           &           & forces  \\ 
 \hdashline 
     &           &    1 &  9.36e-03 &  3.00e-01 &      \\ 
     &           &    2 &  9.09e-13 &  1.80e-14 &      \\ 
 881 &  8.80e+02 &    2 &           &           & forces  \\ 
 \hdashline 
     &           &    1 &  9.36e-03 &  3.00e-01 &      \\ 
     &           &    2 &  9.09e-13 &  1.80e-14 &      \\ 
 882 &  8.81e+02 &    2 &           &           & forces  \\ 
 \hdashline 
     &           &    1 &  9.35e-03 &  3.00e-01 &      \\ 
     &           &    2 &  0.00e+00 &  2.71e-20 &      \\ 
 883 &  8.82e+02 &    2 &           &           & displac  \\ 
 \hdashline 
     &           &    1 &  9.34e-03 &  3.00e-01 &      \\ 
     &           &    2 &  0.00e+00 &  0.00e+00 &      \\ 
 884 &  8.83e+02 &    2 &           &           & displac  \\ 
 \hdashline 
     &           &    1 &  9.33e-03 &  3.00e-01 &      \\ 
     &           &    2 &  0.00e+00 &  0.00e+00 &      \\ 
 885 &  8.84e+02 &    2 &           &           & displac  \\ 
 \hdashline 
     &           &    1 &  9.57e+02 &  3.00e-01 &      \\ 
     &           &    2 &  0.00e+00 &  0.00e+00 &      \\ 
 886 &  8.85e+02 &    2 &           &           & displac  \\ 
 \hdashline 
     &           &    1 &  9.32e-03 &  3.00e-01 &      \\ 
     &           &    2 &  9.09e-13 &  1.79e-14 &      \\ 
 887 &  8.86e+02 &    2 &           &           & forces  \\ 
 \hdashline 
     &           &    1 &  9.31e-03 &  3.00e-01 &      \\ 
     &           &    2 &  9.09e-13 &  1.79e-14 &      \\ 
 888 &  8.87e+02 &    2 &           &           & forces  \\ 
 \hdashline 
     &           &    1 &  9.30e-03 &  3.00e-01 &      \\ 
     &           &    2 &  9.09e-13 &  1.78e-14 &      \\ 
 889 &  8.88e+02 &    2 &           &           & forces  \\ 
 \hdashline 
     &           &    1 &  9.29e-03 &  3.00e-01 &      \\ 
     &           &    2 &  9.09e-13 &  1.78e-14 &      \\ 
 890 &  8.89e+02 &    2 &           &           & forces  \\ 
 \hdashline 
     &           &    1 &  9.28e-03 &  3.00e-01 &      \\ 
     &           &    2 &  9.09e-13 &  1.78e-14 &      \\ 
 891 &  8.90e+02 &    2 &           &           & forces  \\ 
 \hdashline 
     &           &    1 &  9.27e-03 &  3.00e-01 &      \\ 
     &           &    2 &  9.09e-13 &  1.78e-14 &      \\ 
 892 &  8.91e+02 &    2 &           &           & forces  \\ 
 \hdashline 
     &           &    1 &  9.77e+02 &  3.00e-01 &      \\ 
     &           &    2 &  0.00e+00 &  0.00e+00 &      \\ 
 893 &  8.92e+02 &    2 &           &           & displac  \\ 
 \hdashline 
     &           &    1 &  9.25e-03 &  3.00e-01 &      \\ 
     &           &    2 &  9.09e-13 &  1.78e-14 &      \\ 
 894 &  8.93e+02 &    2 &           &           & forces  \\ 
 \hdashline 
     &           &    1 &  9.25e-03 &  3.00e-01 &      \\ 
     &           &    2 &  9.09e-13 &  1.77e-14 &      \\ 
 895 &  8.94e+02 &    2 &           &           & forces  \\ 
 \hdashline 
     &           &    1 &  9.24e-03 &  3.00e-01 &      \\ 
     &           &    2 &  9.09e-13 &  1.77e-14 &      \\ 
 896 &  8.95e+02 &    2 &           &           & forces  \\ 
 \hdashline 
     &           &    1 &  9.89e+02 &  3.00e-01 &      \\ 
     &           &    2 &  0.00e+00 &  0.00e+00 &      \\ 
 897 &  8.96e+02 &    2 &           &           & displac  \\ 
 \hdashline 
     &           &    1 &  9.22e-03 &  3.00e-01 &      \\ 
     &           &    2 &  9.09e-13 &  1.77e-14 &      \\ 
 898 &  8.97e+02 &    2 &           &           & forces  \\ 
 \hdashline 
     &           &    1 &  9.21e-03 &  3.00e-01 &      \\ 
     &           &    2 &  9.09e-13 &  1.77e-14 &      \\ 
 899 &  8.98e+02 &    2 &           &           & forces  \\ 
 \hdashline 
     &           &    1 &  9.20e-03 &  3.00e-01 &      \\ 
     &           &    2 &  9.09e-13 &  1.76e-14 &      \\ 
 900 &  8.99e+02 &    2 &           &           & forces  \\ 
 \hdashline 
     &           &    1 &  9.19e-03 &  3.00e-01 &      \\ 
     &           &    2 &  9.09e-13 &  1.76e-14 &      \\ 
 901 &  9.00e+02 &    2 &           &           & forces  \\ 
 \hdashline 
     &           &    1 &  9.18e-03 &  3.00e-01 &      \\ 
     &           &    2 &  9.09e-13 &  3.51e-14 &      \\ 
 902 &  9.01e+02 &    2 &           &           & forces  \\ 
 \hdashline 
     &           &    1 &  9.17e-03 &  3.00e-01 &      \\ 
     &           &    2 &  9.09e-13 &  1.75e-14 &      \\ 
 903 &  9.02e+02 &    2 &           &           & forces  \\ 
 \hdashline 
     &           &    1 &  9.16e-03 &  3.00e-01 &      \\ 
     &           &    2 &  9.09e-13 &  1.75e-14 &      \\ 
 904 &  9.03e+02 &    2 &           &           & forces  \\ 
 \hdashline 
     &           &    1 &  1.01e+03 &  3.00e-01 &      \\ 
     &           &    2 &  0.00e+00 &  0.00e+00 &      \\ 
 905 &  9.04e+02 &    2 &           &           & displac  \\ 
 \hdashline 
     &           &    1 &  9.14e-03 &  3.00e-01 &      \\ 
     &           &    2 &  9.09e-13 &  1.75e-14 &      \\ 
 906 &  9.05e+02 &    2 &           &           & forces  \\ 
 \hdashline 
     &           &    1 &  9.13e-03 &  3.00e-01 &      \\ 
     &           &    2 &  9.09e-13 &  1.75e-14 &      \\ 
 907 &  9.06e+02 &    2 &           &           & forces  \\ 
 \hdashline 
     &           &    1 &  9.12e-03 &  3.00e-01 &      \\ 
     &           &    2 &  9.09e-13 &  5.23e-14 &      \\ 
 908 &  9.07e+02 &    2 &           &           & forces  \\ 
 \hdashline 
     &           &    1 &  9.11e-03 &  3.00e-01 &      \\ 
     &           &    2 &  9.09e-13 &  1.74e-14 &      \\ 
 909 &  9.08e+02 &    2 &           &           & forces  \\ 
 \hdashline 
     &           &    1 &  9.10e-03 &  3.00e-01 &      \\ 
     &           &    2 &  9.09e-13 &  1.74e-14 &      \\ 
 910 &  9.09e+02 &    2 &           &           & forces  \\ 
 \hdashline 
     &           &    1 &  9.09e-03 &  3.00e-01 &      \\ 
     &           &    2 &  1.82e-12 &  3.47e-14 &      \\ 
 911 &  9.10e+02 &    2 &           &           & forces  \\ 
 \hdashline 
     &           &    1 &  9.08e-03 &  3.00e-01 &      \\ 
     &           &    2 &  9.09e-13 &  3.47e-14 &      \\ 
 912 &  9.11e+02 &    2 &           &           & forces  \\ 
 \hdashline 
     &           &    1 &  9.07e-03 &  3.00e-01 &      \\ 
     &           &    2 &  9.09e-13 &  1.73e-14 &      \\ 
 913 &  9.12e+02 &    2 &           &           & forces  \\ 
 \hdashline 
     &           &    1 &  9.06e-03 &  3.00e-01 &      \\ 
     &           &    2 &  9.09e-13 &  1.73e-14 &      \\ 
 914 &  9.13e+02 &    2 &           &           & forces  \\ 
 \hdashline 
     &           &    1 &  9.05e-03 &  3.00e-01 &      \\ 
     &           &    2 &  9.09e-13 &  1.73e-14 &      \\ 
 915 &  9.14e+02 &    2 &           &           & forces  \\ 
 \hdashline 
     &           &    1 &  9.04e-03 &  3.00e-01 &      \\ 
     &           &    2 &  0.00e+00 &  0.00e+00 &      \\ 
 916 &  9.15e+02 &    2 &           &           & displac  \\ 
 \hdashline 
     &           &    1 &  9.03e-03 &  3.00e-01 &      \\ 
     &           &    2 &  0.00e+00 &  2.71e-20 &      \\ 
 917 &  9.16e+02 &    2 &           &           & displac  \\ 
 \hdashline 
     &           &    1 &  9.02e-03 &  3.00e-01 &      \\ 
     &           &    2 &  0.00e+00 &  0.00e+00 &      \\ 
 918 &  9.17e+02 &    2 &           &           & displac  \\ 
 \hdashline 
     &           &    1 &  9.01e-03 &  3.00e-01 &      \\ 
     &           &    2 &  0.00e+00 &  2.71e-20 &      \\ 
 919 &  9.18e+02 &    2 &           &           & displac  \\ 
 \hdashline 
     &           &    1 &  9.00e-03 &  3.00e-01 &      \\ 
     &           &    2 &  0.00e+00 &  0.00e+00 &      \\ 
 920 &  9.19e+02 &    2 &           &           & displac  \\ 
 \hdashline 
     &           &    1 &  8.99e-03 &  3.00e-01 &      \\ 
     &           &    2 &  0.00e+00 &  0.00e+00 &      \\ 
 921 &  9.20e+02 &    2 &           &           & displac  \\ 
 \hdashline 
     &           &    1 &  8.98e-03 &  3.00e-01 &      \\ 
     &           &    2 &  0.00e+00 &  0.00e+00 &      \\ 
 922 &  9.21e+02 &    2 &           &           & displac  \\ 
 \hdashline 
     &           &    1 &  8.97e-03 &  3.00e-01 &      \\ 
     &           &    2 &  9.09e-13 &  1.72e-14 &      \\ 
 923 &  9.22e+02 &    2 &           &           & forces  \\ 
 \hdashline 
     &           &    1 &  1.07e+03 &  3.00e-01 &      \\ 
     &           &    2 &  0.00e+00 &  3.55e-15 &      \\ 
 924 &  9.23e+02 &    2 &           &           & displac  \\ 
 \hdashline 
     &           &    1 &  8.95e-03 &  3.00e-01 &      \\ 
     &           &    2 &  9.09e-13 &  1.71e-14 &      \\ 
 925 &  9.24e+02 &    2 &           &           & forces  \\ 
 \hdashline 
     &           &    1 &  8.94e-03 &  3.00e-01 &      \\ 
     &           &    2 &  9.09e-13 &  1.71e-14 &      \\ 
 926 &  9.25e+02 &    2 &           &           & forces  \\ 
 \hdashline 
     &           &    1 &  8.93e-03 &  3.00e-01 &      \\ 
     &           &    2 &  9.09e-13 &  1.71e-14 &      \\ 
 927 &  9.26e+02 &    2 &           &           & forces  \\ 
 \hdashline 
     &           &    1 &  8.92e-03 &  3.00e-01 &      \\ 
     &           &    2 &  9.09e-13 &  1.70e-14 &      \\ 
 928 &  9.27e+02 &    2 &           &           & forces  \\ 
 \hdashline 
     &           &    1 &  8.91e-03 &  3.00e-01 &      \\ 
     &           &    2 &  9.09e-13 &  1.70e-14 &      \\ 
 929 &  9.28e+02 &    2 &           &           & forces  \\ 
 \hdashline 
     &           &    1 &  8.90e-03 &  3.00e-01 &      \\ 
     &           &    2 &  9.09e-13 &  1.70e-14 &      \\ 
 930 &  9.29e+02 &    2 &           &           & forces  \\ 
 \hdashline 
     &           &    1 &  8.89e-03 &  3.00e-01 &      \\ 
     &           &    2 &  9.09e-13 &  1.70e-14 &      \\ 
 931 &  9.30e+02 &    2 &           &           & forces  \\ 
 \hdashline 
     &           &    1 &  1.09e+03 &  3.00e-01 &      \\ 
     &           &    2 &  0.00e+00 &  0.00e+00 &      \\ 
 932 &  9.31e+02 &    2 &           &           & displac  \\ 
 \hdashline 
     &           &    1 &  8.87e-03 &  3.00e-01 &      \\ 
     &           &    2 &  0.00e+00 &  0.00e+00 &      \\ 
 933 &  9.32e+02 &    2 &           &           & displac  \\ 
 \hdashline 
     &           &    1 &  8.86e-03 &  3.00e-01 &      \\ 
     &           &    2 &  9.09e-13 &  1.69e-14 &      \\ 
 934 &  9.33e+02 &    2 &           &           & forces  \\ 
 \hdashline 
     &           &    1 &  1.10e+03 &  3.00e-01 &      \\ 
     &           &    2 &  0.00e+00 &  3.55e-15 &      \\ 
 935 &  9.34e+02 &    2 &           &           & displac  \\ 
 \hdashline 
     &           &    1 &  8.83e-03 &  3.00e-01 &      \\ 
     &           &    2 &  0.00e+00 &  0.00e+00 &      \\ 
 936 &  9.35e+02 &    2 &           &           & displac  \\ 
 \hdashline 
     &           &    1 &  8.82e-03 &  3.00e-01 &      \\ 
     &           &    2 &  0.00e+00 &  2.71e-20 &      \\ 
 937 &  9.36e+02 &    2 &           &           & displac  \\ 
 \hdashline 
     &           &    1 &  8.81e-03 &  3.00e-01 &      \\ 
     &           &    2 &  0.00e+00 &  0.00e+00 &      \\ 
 938 &  9.37e+02 &    2 &           &           & displac  \\ 
 \hdashline 
     &           &    1 &  1.11e+03 &  3.00e-01 &      \\ 
     &           &    2 &  0.00e+00 &  0.00e+00 &      \\ 
 939 &  9.38e+02 &    2 &           &           & displac  \\ 
 \hdashline 
     &           &    1 &  8.79e-03 &  3.00e-01 &      \\ 
     &           &    2 &  0.00e+00 &  0.00e+00 &      \\ 
 940 &  9.39e+02 &    2 &           &           & displac  \\ 
 \hdashline 
     &           &    1 &  8.78e-03 &  3.00e-01 &      \\ 
     &           &    2 &  0.00e+00 &  0.00e+00 &      \\ 
 941 &  9.40e+02 &    2 &           &           & displac  \\ 
 \hdashline 
     &           &    1 &  8.77e-03 &  3.00e-01 &      \\ 
     &           &    2 &  0.00e+00 &  0.00e+00 &      \\ 
 942 &  9.41e+02 &    2 &           &           & displac  \\ 
 \hdashline 
     &           &    1 &  8.76e-03 &  3.00e-01 &      \\ 
     &           &    2 &  9.09e-13 &  1.68e-14 &      \\ 
 943 &  9.42e+02 &    2 &           &           & forces  \\ 
 \hdashline 
     &           &    1 &  8.75e-03 &  3.00e-01 &      \\ 
     &           &    2 &  9.09e-13 &  1.68e-14 &      \\ 
 944 &  9.43e+02 &    2 &           &           & forces  \\ 
 \hdashline 
     &           &    1 &  8.73e-03 &  3.00e-01 &      \\ 
     &           &    2 &  9.09e-13 &  1.68e-14 &      \\ 
 945 &  9.44e+02 &    2 &           &           & forces  \\ 
 \hdashline 
     &           &    1 &  1.13e+03 &  3.00e-01 &      \\ 
     &           &    2 &  0.00e+00 &  3.55e-15 &      \\ 
 946 &  9.45e+02 &    2 &           &           & displac  \\ 
 \hdashline 
     &           &    1 &  8.71e-03 &  3.00e-01 &      \\ 
     &           &    2 &  0.00e+00 &  2.71e-20 &      \\ 
 947 &  9.46e+02 &    2 &           &           & displac  \\ 
 \hdashline 
     &           &    1 &  8.70e-03 &  3.00e-01 &      \\ 
     &           &    2 &  0.00e+00 &  0.00e+00 &      \\ 
 948 &  9.47e+02 &    2 &           &           & displac  \\ 
 \hdashline 
     &           &    1 &  8.69e-03 &  3.00e-01 &      \\ 
     &           &    2 &  0.00e+00 &  0.00e+00 &      \\ 
 949 &  9.48e+02 &    2 &           &           & displac  \\ 
 \hdashline 
     &           &    1 &  8.68e-03 &  3.00e-01 &      \\ 
     &           &    2 &  9.09e-13 &  1.67e-14 &      \\ 
 950 &  9.49e+02 &    2 &           &           & forces  \\ 
 \hdashline 
     &           &    1 &  8.67e-03 &  3.00e-01 &      \\ 
     &           &    2 &  9.09e-13 &  1.67e-14 &      \\ 
 951 &  9.50e+02 &    2 &           &           & forces  \\ 
 \hdashline 
     &           &    1 &  8.65e-03 &  3.00e-01 &      \\ 
     &           &    2 &  9.09e-13 &  1.66e-14 &      \\ 
 952 &  9.51e+02 &    2 &           &           & forces  \\ 
 \hdashline 
     &           &    1 &  8.64e-03 &  3.00e-01 &      \\ 
     &           &    2 &  9.09e-13 &  1.65e-14 &      \\ 
 953 &  9.52e+02 &    2 &           &           & forces  \\ 
 \hdashline 
     &           &    1 &  8.63e-03 &  3.00e-01 &      \\ 
     &           &    2 &  9.09e-13 &  1.66e-14 &      \\ 
 954 &  9.53e+02 &    2 &           &           & forces  \\ 
 \hdashline 
     &           &    1 &  1.16e+03 &  3.00e-01 &      \\ 
     &           &    2 &  0.00e+00 &  0.00e+00 &      \\ 
 955 &  9.54e+02 &    2 &           &           & displac  \\ 
 \hdashline 
     &           &    1 &  8.61e-03 &  3.00e-01 &      \\ 
     &           &    2 &  9.09e-13 &  1.65e-14 &      \\ 
 956 &  9.55e+02 &    2 &           &           & forces  \\ 
 \hdashline 
     &           &    1 &  1.16e+03 &  3.00e-01 &      \\ 
     &           &    2 &  0.00e+00 &  0.00e+00 &      \\ 
 957 &  9.56e+02 &    2 &           &           & displac  \\ 
 \hdashline 
     &           &    1 &  8.58e-03 &  3.00e-01 &      \\ 
     &           &    2 &  0.00e+00 &  0.00e+00 &      \\ 
 958 &  9.57e+02 &    2 &           &           & displac  \\ 
 \hdashline 
     &           &    1 &  8.57e-03 &  3.00e-01 &      \\ 
     &           &    2 &  0.00e+00 &  5.55e-17 &      \\ 
 959 &  9.58e+02 &    2 &           &           & displac  \\ 
 \hdashline 
     &           &    1 &  8.56e-03 &  3.00e-01 &      \\ 
     &           &    2 &  0.00e+00 &  2.71e-20 &      \\ 
 960 &  9.59e+02 &    2 &           &           & displac  \\ 
 \hdashline 
     &           &    1 &  8.55e-03 &  3.00e-01 &      \\ 
     &           &    2 &  0.00e+00 &  0.00e+00 &      \\ 
 961 &  9.60e+02 &    2 &           &           & displac  \\ 
 \hdashline 
     &           &    1 &  8.54e-03 &  3.00e-01 &      \\ 
     &           &    2 &  0.00e+00 &  0.00e+00 &      \\ 
 962 &  9.61e+02 &    2 &           &           & displac  \\ 
 \hdashline 
     &           &    1 &  1.18e+03 &  3.00e-01 &      \\ 
     &           &    2 &  0.00e+00 &  0.00e+00 &      \\ 
 963 &  9.62e+02 &    2 &           &           & displac  \\ 
 \hdashline 
     &           &    1 &  8.51e-03 &  3.00e-01 &      \\ 
     &           &    2 &  0.00e+00 &  2.71e-20 &      \\ 
 964 &  9.63e+02 &    2 &           &           & displac  \\ 
 \hdashline 
     &           &    1 &  1.19e+03 &  3.00e-01 &      \\ 
     &           &    2 &  0.00e+00 &  0.00e+00 &      \\ 
 965 &  9.64e+02 &    2 &           &           & displac  \\ 
 \hdashline 
     &           &    1 &  8.49e-03 &  3.00e-01 &      \\ 
     &           &    2 &  0.00e+00 &  0.00e+00 &      \\ 
 966 &  9.65e+02 &    2 &           &           & displac  \\ 
 \hdashline 
     &           &    1 &  8.48e-03 &  3.00e-01 &      \\ 
     &           &    2 &  0.00e+00 &  0.00e+00 &      \\ 
 967 &  9.66e+02 &    2 &           &           & displac  \\ 
 \hdashline 
     &           &    1 &  8.47e-03 &  3.00e-01 &      \\ 
     &           &    2 &  9.09e-13 &  1.63e-14 &      \\ 
 968 &  9.67e+02 &    2 &           &           & forces  \\ 
 \hdashline 
     &           &    1 &  1.20e+03 &  3.00e-01 &      \\ 
     &           &    2 &  0.00e+00 &  0.00e+00 &      \\ 
 969 &  9.68e+02 &    2 &           &           & displac  \\ 
 \hdashline 
     &           &    1 &  1.20e+03 &  3.00e-01 &      \\ 
     &           &    2 &  0.00e+00 &  3.55e-15 &      \\ 
 970 &  9.69e+02 &    2 &           &           & displac  \\ 
 \hdashline 
     &           &    1 &  8.43e-03 &  3.00e-01 &      \\ 
     &           &    2 &  3.64e-12 &  3.26e-14 &      \\ 
 971 &  9.70e+02 &    2 &           &           & forces  \\ 
 \hdashline 
     &           &    1 &  8.42e-03 &  3.00e-01 &      \\ 
     &           &    2 &  9.09e-13 &  6.51e-14 &      \\ 
 972 &  9.71e+02 &    2 &           &           & forces  \\ 
 \hdashline 
     &           &    1 &  1.21e+03 &  3.00e-01 &      \\ 
     &           &    2 &  0.00e+00 &  0.00e+00 &      \\ 
 973 &  9.72e+02 &    2 &           &           & displac  \\ 
 \hdashline 
     &           &    1 &  8.39e-03 &  3.00e-01 &      \\ 
     &           &    2 &  2.73e-12 &  3.25e-14 &      \\ 
 974 &  9.73e+02 &    2 &           &           & forces  \\ 
 \hdashline 
     &           &    1 &  1.22e+03 &  3.00e-01 &      \\ 
     &           &    2 &  0.00e+00 &  3.55e-15 &      \\ 
 975 &  9.74e+02 &    2 &           &           & displac  \\ 
 \hdashline 
     &           &    1 &  8.37e-03 &  3.00e-01 &      \\ 
     &           &    2 &  9.09e-13 &  3.24e-14 &      \\ 
 976 &  9.75e+02 &    2 &           &           & forces  \\ 
 \hdashline 
     &           &    1 &  1.22e+03 &  3.00e-01 &      \\ 
     &           &    2 &  0.00e+00 &  0.00e+00 &      \\ 
 977 &  9.76e+02 &    2 &           &           & displac  \\ 
 \hdashline 
     &           &    1 &  8.34e-03 &  3.00e-01 &      \\ 
     &           &    2 &  0.00e+00 &  0.00e+00 &      \\ 
 978 &  9.77e+02 &    2 &           &           & displac  \\ 
 \hdashline 
     &           &    1 &  8.33e-03 &  3.00e-01 &      \\ 
     &           &    2 &  0.00e+00 &  0.00e+00 &      \\ 
 979 &  9.78e+02 &    2 &           &           & displac  \\ 
 \hdashline 
     &           &    1 &  8.32e-03 &  3.00e-01 &      \\ 
     &           &    2 &  0.00e+00 &  0.00e+00 &      \\ 
 980 &  9.79e+02 &    2 &           &           & displac  \\ 
 \hdashline 
     &           &    1 &  8.31e-03 &  3.00e-01 &      \\ 
     &           &    2 &  0.00e+00 &  0.00e+00 &      \\ 
 981 &  9.80e+02 &    2 &           &           & displac  \\ 
 \hdashline 
     &           &    1 &  8.29e-03 &  3.00e-01 &      \\ 
     &           &    2 &  0.00e+00 &  0.00e+00 &      \\ 
 982 &  9.81e+02 &    2 &           &           & displac  \\ 
 \hdashline 
     &           &    1 &  1.24e+03 &  3.00e-01 &      \\ 
     &           &    2 &  0.00e+00 &  3.55e-15 &      \\ 
 983 &  9.82e+02 &    2 &           &           & displac  \\ 
 \hdashline 
     &           &    1 &  8.27e-03 &  3.00e-01 &      \\ 
     &           &    2 &  1.82e-12 &  3.21e-14 &      \\ 
 984 &  9.83e+02 &    2 &           &           & forces  \\ 
 \hdashline 
     &           &    1 &  1.24e+03 &  3.00e-01 &      \\ 
     &           &    2 &  0.00e+00 &  0.00e+00 &      \\ 
 985 &  9.84e+02 &    2 &           &           & displac  \\ 
 \hdashline 
     &           &    1 &  8.25e-03 &  3.00e-01 &      \\ 
     &           &    2 &  0.00e+00 &  0.00e+00 &      \\ 
 986 &  9.85e+02 &    2 &           &           & displac  \\ 
 \hdashline 
     &           &    1 &  8.23e-03 &  3.00e-01 &      \\ 
     &           &    2 &  0.00e+00 &  0.00e+00 &      \\ 
 987 &  9.86e+02 &    2 &           &           & displac  \\ 
 \hdashline 
     &           &    1 &  8.22e-03 &  3.00e-01 &      \\ 
     &           &    2 &  0.00e+00 &  0.00e+00 &      \\ 
 988 &  9.87e+02 &    2 &           &           & displac  \\ 
 \hdashline 
     &           &    1 &  8.21e-03 &  3.00e-01 &      \\ 
     &           &    2 &  0.00e+00 &  0.00e+00 &      \\ 
 989 &  9.88e+02 &    2 &           &           & displac  \\ 
 \hdashline 
     &           &    1 &  8.20e-03 &  3.00e-01 &      \\ 
     &           &    2 &  0.00e+00 &  0.00e+00 &      \\ 
 990 &  9.89e+02 &    2 &           &           & displac  \\ 
 \hdashline 
     &           &    1 &  8.18e-03 &  3.00e-01 &      \\ 
     &           &    2 &  0.00e+00 &  0.00e+00 &      \\ 
 991 &  9.90e+02 &    2 &           &           & displac  \\ 
 \hdashline 
     &           &    1 &  8.17e-03 &  3.00e-01 &      \\ 
     &           &    2 &  0.00e+00 &  0.00e+00 &      \\ 
 992 &  9.91e+02 &    2 &           &           & displac  \\ 
 \hdashline 
     &           &    1 &  8.16e-03 &  3.00e-01 &      \\ 
     &           &    2 &  0.00e+00 &  0.00e+00 &      \\ 
 993 &  9.92e+02 &    2 &           &           & displac  \\ 
 \hdashline 
     &           &    1 &  8.14e-03 &  3.00e-01 &      \\ 
     &           &    2 &  0.00e+00 &  0.00e+00 &      \\ 
 994 &  9.93e+02 &    2 &           &           & displac  \\ 
 \hdashline 
     &           &    1 &  8.13e-03 &  3.00e-01 &      \\ 
     &           &    2 &  0.00e+00 &  0.00e+00 &      \\ 
 995 &  9.94e+02 &    2 &           &           & displac  \\ 
 \hdashline 
     &           &    1 &  8.12e-03 &  3.00e-01 &      \\ 
     &           &    2 &  0.00e+00 &  0.00e+00 &      \\ 
 996 &  9.95e+02 &    2 &           &           & displac  \\ 
 \hdashline 
     &           &    1 &  1.28e+03 &  3.00e-01 &      \\ 
     &           &    2 &  0.00e+00 &  3.55e-15 &      \\ 
 997 &  9.96e+02 &    2 &           &           & displac  \\ 
 \hdashline 
     &           &    1 &  8.09e-03 &  3.00e-01 &      \\ 
     &           &    2 &  0.00e+00 &  0.00e+00 &      \\ 
 998 &  9.97e+02 &    2 &           &           & displac  \\ 
 \hdashline 
     &           &    1 &  8.08e-03 &  3.00e-01 &      \\ 
     &           &    2 &  0.00e+00 &  2.71e-20 &      \\ 
 999 &  9.98e+02 &    2 &           &           & displac  \\ 
 \hdashline 
     &           &    1 &  1.29e+03 &  3.00e-01 &      \\ 
     &           &    2 &  0.00e+00 &  3.55e-15 &      \\ 
1000 &  9.99e+02 &    2 &           &           & displac  \\ 
 \hdashline 
     &           &    1 &  8.06e-03 &  3.00e-01 &      \\ 
     &           &    2 &  0.00e+00 &  0.00e+00 &      \\ 
1001 &  1.00e+03 &    2 &           &           & displac  \\ 
 \hdashline 
 
\end{longtable}

\end{document}