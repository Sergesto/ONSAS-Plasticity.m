\documentclass[a4paper,10pt]{article} 
\usepackage[a4paper,margin=20mm]{geometry} 
\usepackage{longtable} 
\usepackage{float} 
\usepackage{adjustbox} 
\usepackage{graphicx} 
\usepackage{color} 
\usepackage{booktabs} 
\aboverulesep=0ex 
\belowrulesep=0ex 
\usepackage{array} 
\newcolumntype{?}{!{\vrule width 1pt}}
\usepackage{fancyhdr} 
\pagestyle{fancy} 
\fancyhf{} 
\rhead{Problem: Newton-RaphsonArc-LengthLogarithmicStrainJirasek } 
\lfoot{Date: \today} 
\rfoot{Page \thepage} 
\renewcommand{\footrulewidth}{1pt} 
\renewcommand{\headrulewidth}{1.5pt} 
\setlength{\parindent}{0pt} 
\usepackage[T1]{fontenc} 
\usepackage{libertine} 
\usepackage{arydshln} 
\definecolor{miblue}{rgb}{0,0.1,0.38} 
\usepackage{titlesec} 
\titleformat{\section}{\normalfont\Large\color{miblue}\bfseries}{\color{miblue}\sectionmark\thesection}{0.5em}{}[{\color{miblue}\titlerule[0.5pt]}] 

\begin{document} 
This is an ONSAS automatically-generated report with part of the results obtained after the analysis. The user can access other magnitudes and results through the GNU-Octave/MATLAB console. The code is provided AS IS \textbf{WITHOUT WARRANTY of any kind}, express or implied.

\section{Analysis results}

\begin{longtable}{cccccc} 
$\#t$ & $t$ & its & $\| RHS \|$ & $\| \Delta u \|$ & flagExit \\ \hline 
 \endhead 
   1 &  0.00e+00 &    0 &           &           &   \\ 
 \hdashline 
     &           &    1 &  4.42e-11 &  7.84e-07 &      \\ 
     &           &    2 &  1.16e-03 &  4.00e-02 &      \\ 
     &           &    3 &  0.00e+00 &  0.00e+00 &      \\ 
   2 &  1.00e+00 &    3 &           &           & displac  \\ 
 \hdashline 
     &           &    1 &  1.16e-03 &  4.00e-02 &      \\ 
     &           &    2 &  0.00e+00 &  0.00e+00 &      \\ 
   3 &  2.00e+00 &    2 &           &           & displac  \\ 
 \hdashline 
     &           &    1 &  1.16e-03 &  4.00e-02 &      \\ 
     &           &    2 &  0.00e+00 &  0.00e+00 &      \\ 
   4 &  3.00e+00 &    2 &           &           & displac  \\ 
 \hdashline 
     &           &    1 &  1.16e-03 &  4.00e-02 &      \\ 
     &           &    2 &  0.00e+00 &  0.00e+00 &      \\ 
   5 &  4.00e+00 &    2 &           &           & displac  \\ 
 \hdashline 
     &           &    1 &  1.16e-03 &  4.00e-02 &      \\ 
     &           &    2 &  0.00e+00 &  0.00e+00 &      \\ 
   6 &  5.00e+00 &    2 &           &           & displac  \\ 
 \hdashline 
     &           &    1 &  1.16e-03 &  4.00e-02 &      \\ 
     &           &    2 &  0.00e+00 &  0.00e+00 &      \\ 
   7 &  6.00e+00 &    2 &           &           & displac  \\ 
 \hdashline 
     &           &    1 &  1.16e-03 &  4.00e-02 &      \\ 
     &           &    2 &  0.00e+00 &  0.00e+00 &      \\ 
   8 &  7.00e+00 &    2 &           &           & displac  \\ 
 \hdashline 
     &           &    1 &  1.16e-03 &  4.00e-02 &      \\ 
     &           &    2 &  0.00e+00 &  0.00e+00 &      \\ 
   9 &  8.00e+00 &    2 &           &           & displac  \\ 
 \hdashline 
     &           &    1 &  1.16e-03 &  4.00e-02 &      \\ 
     &           &    2 &  0.00e+00 &  0.00e+00 &      \\ 
  10 &  9.00e+00 &    2 &           &           & displac  \\ 
 \hdashline 
     &           &    1 &  1.16e-03 &  4.00e-02 &      \\ 
     &           &    2 &  0.00e+00 &  0.00e+00 &      \\ 
  11 &  1.00e+01 &    2 &           &           & displac  \\ 
 \hdashline 
     &           &    1 &  1.16e-03 &  4.00e-02 &      \\ 
     &           &    2 &  0.00e+00 &  0.00e+00 &      \\ 
  12 &  1.10e+01 &    2 &           &           & displac  \\ 
 \hdashline 
     &           &    1 &  1.16e-03 &  4.00e-02 &      \\ 
     &           &    2 &  0.00e+00 &  0.00e+00 &      \\ 
  13 &  1.20e+01 &    2 &           &           & displac  \\ 
 \hdashline 
     &           &    1 &  1.16e-03 &  4.00e-02 &      \\ 
     &           &    2 &  0.00e+00 &  0.00e+00 &      \\ 
  14 &  1.30e+01 &    2 &           &           & displac  \\ 
 \hdashline 
     &           &    1 &  1.16e-03 &  4.00e-02 &      \\ 
     &           &    2 &  0.00e+00 &  0.00e+00 &      \\ 
  15 &  1.40e+01 &    2 &           &           & displac  \\ 
 \hdashline 
     &           &    1 &  1.16e-03 &  4.00e-02 &      \\ 
     &           &    2 &  0.00e+00 &  0.00e+00 &      \\ 
  16 &  1.50e+01 &    2 &           &           & displac  \\ 
 \hdashline 
     &           &    1 &  5.67e+00 &  4.00e-02 &      \\ 
     &           &    2 &  0.00e+00 &  0.00e+00 &      \\ 
  17 &  1.60e+01 &    2 &           &           & displac  \\ 
 \hdashline 
     &           &    1 &  1.16e-03 &  4.00e-02 &      \\ 
     &           &    2 &  0.00e+00 &  0.00e+00 &      \\ 
  18 &  1.70e+01 &    2 &           &           & displac  \\ 
 \hdashline 
     &           &    1 &  1.16e-03 &  4.00e-02 &      \\ 
     &           &    2 &  0.00e+00 &  0.00e+00 &      \\ 
  19 &  1.80e+01 &    2 &           &           & displac  \\ 
 \hdashline 
     &           &    1 &  1.16e-03 &  4.00e-02 &      \\ 
     &           &    2 &  0.00e+00 &  0.00e+00 &      \\ 
  20 &  1.90e+01 &    2 &           &           & displac  \\ 
 \hdashline 
     &           &    1 &  1.16e-03 &  4.00e-02 &      \\ 
     &           &    2 &  0.00e+00 &  0.00e+00 &      \\ 
  21 &  2.00e+01 &    2 &           &           & displac  \\ 
 \hdashline 
     &           &    1 &  3.13e+01 &  6.00e-02 &      \\ 
     &           &    2 &  0.00e+00 &  0.00e+00 &      \\ 
  22 &  2.10e+01 &    2 &           &           & displac  \\ 
 \hdashline 
     &           &    1 &  2.60e-03 &  6.00e-02 &      \\ 
     &           &    2 &  6.27e+01 &  1.20e-01 &      \\ 
     &           &    3 &  0.00e+00 &  0.00e+00 &      \\ 
  23 &  2.20e+01 &    3 &           &           & displac  \\ 
 \hdashline 
     &           &    1 &  2.60e-03 &  6.00e-02 &      \\ 
     &           &    2 &  1.04e-02 &  1.20e-01 &      \\ 
     &           &    3 &  0.00e+00 &  0.00e+00 &      \\ 
  24 &  2.30e+01 &    3 &           &           & displac  \\ 
 \hdashline 
     &           &    1 &  2.60e-03 &  6.00e-02 &      \\ 
     &           &    2 &  1.04e-02 &  1.20e-01 &      \\ 
     &           &    3 &  0.00e+00 &  0.00e+00 &      \\ 
  25 &  2.40e+01 &    3 &           &           & displac  \\ 
 \hdashline 
     &           &    1 &  2.60e-03 &  6.00e-02 &      \\ 
     &           &    2 &  1.04e-02 &  1.20e-01 &      \\ 
     &           &    3 &  0.00e+00 &  0.00e+00 &      \\ 
  26 &  2.50e+01 &    3 &           &           & displac  \\ 
 \hdashline 
     &           &    1 &  2.61e-03 &  6.00e-02 &      \\ 
     &           &    2 &  1.04e-02 &  1.20e-01 &      \\ 
     &           &    3 &  0.00e+00 &  0.00e+00 &      \\ 
  27 &  2.60e+01 &    3 &           &           & displac  \\ 
 \hdashline 
     &           &    1 &  2.61e-03 &  6.00e-02 &      \\ 
     &           &    2 &  1.04e-02 &  1.20e-01 &      \\ 
     &           &    3 &  0.00e+00 &  0.00e+00 &      \\ 
  28 &  2.70e+01 &    3 &           &           & displac  \\ 
 \hdashline 
     &           &    1 &  2.61e-03 &  6.00e-02 &      \\ 
     &           &    2 &  1.04e-02 &  1.20e-01 &      \\ 
     &           &    3 &  0.00e+00 &  0.00e+00 &      \\ 
  29 &  2.80e+01 &    3 &           &           & displac  \\ 
 \hdashline 
     &           &    1 &  2.61e-03 &  6.00e-02 &      \\ 
     &           &    2 &  1.04e-02 &  1.20e-01 &      \\ 
     &           &    3 &  0.00e+00 &  0.00e+00 &      \\ 
  30 &  2.90e+01 &    3 &           &           & displac  \\ 
 \hdashline 
     &           &    1 &  2.61e-03 &  6.00e-02 &      \\ 
     &           &    2 &  1.04e-02 &  1.20e-01 &      \\ 
     &           &    3 &  0.00e+00 &  0.00e+00 &      \\ 
  31 &  3.00e+01 &    3 &           &           & displac  \\ 
 \hdashline 
     &           &    1 &  2.61e-03 &  6.00e-02 &      \\ 
     &           &    2 &  1.04e-02 &  1.20e-01 &      \\ 
     &           &    3 &  0.00e+00 &  0.00e+00 &      \\ 
  32 &  3.10e+01 &    3 &           &           & displac  \\ 
 \hdashline 
     &           &    1 &  2.61e-03 &  6.00e-02 &      \\ 
     &           &    2 &  1.04e-02 &  1.20e-01 &      \\ 
     &           &    3 &  0.00e+00 &  0.00e+00 &      \\ 
  33 &  3.20e+01 &    3 &           &           & displac  \\ 
 \hdashline 
     &           &    1 &  2.61e-03 &  6.00e-02 &      \\ 
     &           &    2 &  1.04e-02 &  1.20e-01 &      \\ 
     &           &    3 &  0.00e+00 &  0.00e+00 &      \\ 
  34 &  3.30e+01 &    3 &           &           & displac  \\ 
 \hdashline 
     &           &    1 &  2.61e-03 &  6.00e-02 &      \\ 
     &           &    2 &  1.04e-02 &  1.20e-01 &      \\ 
     &           &    3 &  0.00e+00 &  0.00e+00 &      \\ 
  35 &  3.40e+01 &    3 &           &           & displac  \\ 
 \hdashline 
     &           &    1 &  2.61e-03 &  6.00e-02 &      \\ 
     &           &    2 &  1.05e-02 &  1.20e-01 &      \\ 
     &           &    3 &  0.00e+00 &  0.00e+00 &      \\ 
  36 &  3.50e+01 &    3 &           &           & displac  \\ 
 \hdashline 
     &           &    1 &  2.61e-03 &  6.00e-02 &      \\ 
     &           &    2 &  1.05e-02 &  1.20e-01 &      \\ 
     &           &    3 &  0.00e+00 &  0.00e+00 &      \\ 
  37 &  3.60e+01 &    3 &           &           & displac  \\ 
 \hdashline 
     &           &    1 &  2.61e-03 &  6.00e-02 &      \\ 
     &           &    2 &  1.05e-02 &  1.20e-01 &      \\ 
     &           &    3 &  0.00e+00 &  0.00e+00 &      \\ 
  38 &  3.70e+01 &    3 &           &           & displac  \\ 
 \hdashline 
     &           &    1 &  2.62e-03 &  6.00e-02 &      \\ 
     &           &    2 &  1.05e-02 &  1.20e-01 &      \\ 
     &           &    3 &  0.00e+00 &  0.00e+00 &      \\ 
  39 &  3.80e+01 &    3 &           &           & displac  \\ 
 \hdashline 
     &           &    1 &  2.62e-03 &  6.00e-02 &      \\ 
     &           &    2 &  1.05e-02 &  1.20e-01 &      \\ 
     &           &    3 &  0.00e+00 &  0.00e+00 &      \\ 
  40 &  3.90e+01 &    3 &           &           & displac  \\ 
 \hdashline 
     &           &    1 &  2.62e-03 &  6.00e-02 &      \\ 
     &           &    2 &  1.05e-02 &  1.20e-01 &      \\ 
     &           &    3 &  0.00e+00 &  0.00e+00 &      \\ 
  41 &  4.00e+01 &    3 &           &           & displac  \\ 
 \hdashline 
     &           &    1 &  2.62e-03 &  6.00e-02 &      \\ 
     &           &    2 &  1.05e-02 &  1.20e-01 &      \\ 
     &           &    3 &  0.00e+00 &  0.00e+00 &      \\ 
  42 &  4.10e+01 &    3 &           &           & displac  \\ 
 \hdashline 
     &           &    1 &  2.62e-03 &  6.00e-02 &      \\ 
     &           &    2 &  3.13e+01 &  1.20e-01 &      \\ 
     &           &    3 &  0.00e+00 &  0.00e+00 &      \\ 
  43 &  4.20e+01 &    3 &           &           & displac  \\ 
 \hdashline 
     &           &    1 &  3.13e+01 &  6.00e-02 &      \\ 
     &           &    2 &  3.13e+01 &  1.20e-01 &      \\ 
     &           &    3 &  0.00e+00 &  0.00e+00 &      \\ 
  44 &  4.30e+01 &    3 &           &           & displac  \\ 
 \hdashline 
     &           &    1 &  3.13e+01 &  6.00e-02 &      \\ 
     &           &    2 &  3.13e+01 &  1.20e-01 &      \\ 
     &           &    3 &  0.00e+00 &  6.94e-18 &      \\ 
  45 &  4.40e+01 &    3 &           &           & displac  \\ 
 \hdashline 
     &           &    1 &  3.13e+01 &  6.00e-02 &      \\ 
     &           &    2 &  3.13e+01 &  1.20e-01 &      \\ 
     &           &    3 &  0.00e+00 &  0.00e+00 &      \\ 
  46 &  4.50e+01 &    3 &           &           & displac  \\ 
 \hdashline 
     &           &    1 &  3.13e+01 &  6.00e-02 &      \\ 
     &           &    2 &  3.13e+01 &  1.20e-01 &      \\ 
     &           &    3 &  0.00e+00 &  6.94e-18 &      \\ 
  47 &  4.60e+01 &    3 &           &           & displac  \\ 
 \hdashline 
     &           &    1 &  3.13e+01 &  6.00e-02 &      \\ 
     &           &    2 &  3.13e+01 &  1.20e-01 &      \\ 
     &           &    3 &  0.00e+00 &  6.94e-18 &      \\ 
  48 &  4.70e+01 &    3 &           &           & displac  \\ 
 \hdashline 
     &           &    1 &  2.62e-03 &  6.00e-02 &      \\ 
     &           &    2 &  3.13e+01 &  1.20e-01 &      \\ 
     &           &    3 &  0.00e+00 &  0.00e+00 &      \\ 
  49 &  4.80e+01 &    3 &           &           & displac  \\ 
 \hdashline 
     &           &    1 &  3.13e+01 &  6.00e-02 &      \\ 
     &           &    2 &  3.13e+01 &  1.20e-01 &      \\ 
     &           &    3 &  0.00e+00 &  0.00e+00 &      \\ 
  50 &  4.90e+01 &    3 &           &           & displac  \\ 
 \hdashline 
     &           &    1 &  3.13e+01 &  6.00e-02 &      \\ 
     &           &    2 &  4.02e+01 &  1.20e-01 &      \\ 
     &           &    3 &  0.00e+00 &  0.00e+00 &      \\ 
  51 &  5.00e+01 &    3 &           &           & displac  \\ 
 \hdashline 
     &           &    1 &  6.34e+01 &  6.00e-02 &      \\ 
     &           &    2 &  6.34e+01 &  1.20e-01 &      \\ 
     &           &    3 &  0.00e+00 &  0.00e+00 &      \\ 
  52 &  5.10e+01 &    3 &           &           & displac  \\ 
 \hdashline 
     &           &    1 &  6.34e+01 &  6.00e-02 &      \\ 
     &           &    2 &  6.34e+01 &  1.20e-01 &      \\ 
     &           &    3 &  0.00e+00 &  0.00e+00 &      \\ 
  53 &  5.20e+01 &    3 &           &           & displac  \\ 
 \hdashline 
     &           &    1 &  6.34e+01 &  6.00e-02 &      \\ 
     &           &    2 &  6.34e+01 &  1.20e-01 &      \\ 
     &           &    3 &  0.00e+00 &  0.00e+00 &      \\ 
  54 &  5.30e+01 &    3 &           &           & displac  \\ 
 \hdashline 
     &           &    1 &  6.34e+01 &  6.00e-02 &      \\ 
     &           &    2 &  6.34e+01 &  1.20e-01 &      \\ 
     &           &    3 &  0.00e+00 &  0.00e+00 &      \\ 
  55 &  5.40e+01 &    3 &           &           & displac  \\ 
 \hdashline 
     &           &    1 &  6.34e+01 &  6.00e-02 &      \\ 
     &           &    2 &  6.34e+01 &  1.20e-01 &      \\ 
     &           &    3 &  0.00e+00 &  0.00e+00 &      \\ 
  56 &  5.50e+01 &    3 &           &           & displac  \\ 
 \hdashline 
     &           &    1 &  6.34e+01 &  6.00e-02 &      \\ 
     &           &    2 &  6.34e+01 &  1.20e-01 &      \\ 
     &           &    3 &  0.00e+00 &  0.00e+00 &      \\ 
  57 &  5.60e+01 &    3 &           &           & displac  \\ 
 \hdashline 
     &           &    1 &  6.33e+01 &  6.00e-02 &      \\ 
     &           &    2 &  6.34e+01 &  1.20e-01 &      \\ 
     &           &    3 &  0.00e+00 &  0.00e+00 &      \\ 
  58 &  5.70e+01 &    3 &           &           & displac  \\ 
 \hdashline 
     &           &    1 &  6.33e+01 &  6.00e-02 &      \\ 
     &           &    2 &  6.34e+01 &  1.20e-01 &      \\ 
     &           &    3 &  0.00e+00 &  0.00e+00 &      \\ 
  59 &  5.80e+01 &    3 &           &           & displac  \\ 
 \hdashline 
     &           &    1 &  3.13e+01 &  6.00e-02 &      \\ 
     &           &    2 &  6.33e+01 &  1.20e-01 &      \\ 
     &           &    3 &  0.00e+00 &  0.00e+00 &      \\ 
  60 &  5.90e+01 &    3 &           &           & displac  \\ 
 \hdashline 
     &           &    1 &  6.33e+01 &  6.00e-02 &      \\ 
     &           &    2 &  6.33e+01 &  1.20e-01 &      \\ 
     &           &    3 &  0.00e+00 &  0.00e+00 &      \\ 
  61 &  6.00e+01 &    3 &           &           & displac  \\ 
 \hdashline 
     &           &    1 &  3.51e-06 &  2.67e-02 &      \\ 
  62 &  6.10e+01 &    1 &           &           & forces  \\ 
 \hdashline 
     &           &    1 &  3.51e-06 &  2.67e-02 &      \\ 
  63 &  6.20e+01 &    1 &           &           & forces  \\ 
 \hdashline 
     &           &    1 &  3.51e-06 &  2.67e-02 &      \\ 
  64 &  6.30e+01 &    1 &           &           & forces  \\ 
 \hdashline 
     &           &    1 &  3.51e-06 &  2.67e-02 &      \\ 
  65 &  6.40e+01 &    1 &           &           & forces  \\ 
 \hdashline 
     &           &    1 &  3.51e-06 &  2.67e-02 &      \\ 
  66 &  6.50e+01 &    1 &           &           & forces  \\ 
 \hdashline 
     &           &    1 &  3.51e-06 &  2.67e-02 &      \\ 
  67 &  6.60e+01 &    1 &           &           & forces  \\ 
 \hdashline 
     &           &    1 &  1.42e+01 &  2.67e-02 &      \\ 
     &           &    2 &  2.81e+01 &  5.33e-02 &      \\ 
     &           &    3 &  2.27e-13 &  2.12e-16 &      \\ 
  68 &  6.70e+01 &    3 &           &           & forces  \\ 
 \hdashline 
     &           &    1 &  5.20e-04 &  2.67e-02 &      \\ 
     &           &    2 &  2.27e-13 &  2.15e-16 &      \\ 
  69 &  6.80e+01 &    2 &           &           & forces  \\ 
 \hdashline 
     &           &    1 &  5.20e-04 &  2.67e-02 &      \\ 
     &           &    2 &  2.27e-13 &  2.15e-16 &      \\ 
  70 &  6.90e+01 &    2 &           &           & forces  \\ 
 \hdashline 
     &           &    1 &  5.20e-04 &  2.67e-02 &      \\ 
     &           &    2 &  2.27e-13 &  2.15e-16 &      \\ 
  71 &  7.00e+01 &    2 &           &           & forces  \\ 
 \hdashline 
     &           &    1 &  5.20e-04 &  2.67e-02 &      \\ 
     &           &    2 &  2.27e-13 &  2.15e-16 &      \\ 
  72 &  7.10e+01 &    2 &           &           & forces  \\ 
 \hdashline 
     &           &    1 &  5.20e-04 &  2.67e-02 &      \\ 
     &           &    2 &  2.27e-13 &  2.15e-16 &      \\ 
  73 &  7.20e+01 &    2 &           &           & forces  \\ 
 \hdashline 
     &           &    1 &  5.20e-04 &  2.67e-02 &      \\ 
     &           &    2 &  2.27e-13 &  2.15e-16 &      \\ 
  74 &  7.30e+01 &    2 &           &           & forces  \\ 
 \hdashline 
     &           &    1 &  5.20e-04 &  2.67e-02 &      \\ 
     &           &    2 &  2.27e-13 &  2.15e-16 &      \\ 
  75 &  7.40e+01 &    2 &           &           & forces  \\ 
 \hdashline 
     &           &    1 &  5.20e-04 &  2.67e-02 &      \\ 
     &           &    2 &  2.27e-13 &  2.15e-16 &      \\ 
  76 &  7.50e+01 &    2 &           &           & forces  \\ 
 \hdashline 
     &           &    1 &  5.20e-04 &  2.67e-02 &      \\ 
     &           &    2 &  2.27e-13 &  2.15e-16 &      \\ 
  77 &  7.60e+01 &    2 &           &           & forces  \\ 
 \hdashline 
     &           &    1 &  5.20e-04 &  2.67e-02 &      \\ 
     &           &    2 &  2.27e-13 &  2.15e-16 &      \\ 
  78 &  7.70e+01 &    2 &           &           & forces  \\ 
 \hdashline 
     &           &    1 &  5.20e-04 &  2.67e-02 &      \\ 
     &           &    2 &  2.27e-13 &  2.15e-16 &      \\ 
  79 &  7.80e+01 &    2 &           &           & forces  \\ 
 \hdashline 
     &           &    1 &  5.20e-04 &  2.67e-02 &      \\ 
     &           &    2 &  2.27e-13 &  2.15e-16 &      \\ 
  80 &  7.90e+01 &    2 &           &           & forces  \\ 
 \hdashline 
     &           &    1 &  5.19e-04 &  2.67e-02 &      \\ 
     &           &    2 &  2.27e-13 &  2.15e-16 &      \\ 
  81 &  8.00e+01 &    2 &           &           & forces  \\ 
 \hdashline 
     &           &    1 &  5.19e-04 &  2.67e-02 &      \\ 
     &           &    2 &  2.27e-13 &  2.15e-16 &      \\ 
  82 &  8.10e+01 &    2 &           &           & forces  \\ 
 \hdashline 
     &           &    1 &  5.19e-04 &  2.67e-02 &      \\ 
     &           &    2 &  2.27e-13 &  2.15e-16 &      \\ 
  83 &  8.20e+01 &    2 &           &           & forces  \\ 
 \hdashline 
     &           &    1 &  5.19e-04 &  2.67e-02 &      \\ 
     &           &    2 &  2.27e-13 &  2.15e-16 &      \\ 
  84 &  8.30e+01 &    2 &           &           & forces  \\ 
 \hdashline 
     &           &    1 &  5.19e-04 &  2.67e-02 &      \\ 
     &           &    2 &  2.27e-13 &  2.15e-16 &      \\ 
  85 &  8.40e+01 &    2 &           &           & forces  \\ 
 \hdashline 
     &           &    1 &  5.19e-04 &  2.67e-02 &      \\ 
     &           &    2 &  2.27e-13 &  2.15e-16 &      \\ 
  86 &  8.50e+01 &    2 &           &           & forces  \\ 
 \hdashline 
     &           &    1 &  5.19e-04 &  2.67e-02 &      \\ 
     &           &    2 &  2.27e-13 &  2.15e-16 &      \\ 
  87 &  8.60e+01 &    2 &           &           & forces  \\ 
 \hdashline 
     &           &    1 &  5.19e-04 &  2.67e-02 &      \\ 
     &           &    2 &  2.27e-13 &  2.15e-16 &      \\ 
  88 &  8.70e+01 &    2 &           &           & forces  \\ 
 \hdashline 
     &           &    1 &  5.19e-04 &  2.67e-02 &      \\ 
     &           &    2 &  2.27e-13 &  2.15e-16 &      \\ 
  89 &  8.80e+01 &    2 &           &           & forces  \\ 
 \hdashline 
     &           &    1 &  5.19e-04 &  2.67e-02 &      \\ 
     &           &    2 &  2.27e-13 &  2.15e-16 &      \\ 
  90 &  8.90e+01 &    2 &           &           & forces  \\ 
 \hdashline 
     &           &    1 &  5.19e-04 &  2.67e-02 &      \\ 
     &           &    2 &  2.27e-13 &  2.15e-16 &      \\ 
  91 &  9.00e+01 &    2 &           &           & forces  \\ 
 \hdashline 
     &           &    1 &  5.19e-04 &  2.67e-02 &      \\ 
     &           &    2 &  2.27e-13 &  2.15e-16 &      \\ 
  92 &  9.10e+01 &    2 &           &           & forces  \\ 
 \hdashline 
     &           &    1 &  5.19e-04 &  2.67e-02 &      \\ 
     &           &    2 &  2.27e-13 &  2.15e-16 &      \\ 
  93 &  9.20e+01 &    2 &           &           & forces  \\ 
 \hdashline 
     &           &    1 &  5.18e-04 &  2.67e-02 &      \\ 
     &           &    2 &  2.27e-13 &  2.15e-16 &      \\ 
  94 &  9.30e+01 &    2 &           &           & forces  \\ 
 \hdashline 
     &           &    1 &  5.18e-04 &  2.67e-02 &      \\ 
     &           &    2 &  2.27e-13 &  2.15e-16 &      \\ 
  95 &  9.40e+01 &    2 &           &           & forces  \\ 
 \hdashline 
     &           &    1 &  5.18e-04 &  2.67e-02 &      \\ 
     &           &    2 &  2.27e-13 &  2.15e-16 &      \\ 
  96 &  9.50e+01 &    2 &           &           & forces  \\ 
 \hdashline 
     &           &    1 &  5.18e-04 &  2.67e-02 &      \\ 
     &           &    2 &  2.29e-13 &  2.15e-16 &      \\ 
  97 &  9.60e+01 &    2 &           &           & forces  \\ 
 \hdashline 
     &           &    1 &  5.18e-04 &  2.67e-02 &      \\ 
     &           &    2 &  2.29e-13 &  2.15e-16 &      \\ 
  98 &  9.70e+01 &    2 &           &           & forces  \\ 
 \hdashline 
     &           &    1 &  5.18e-04 &  2.67e-02 &      \\ 
     &           &    2 &  2.27e-13 &  2.15e-16 &      \\ 
  99 &  9.80e+01 &    2 &           &           & forces  \\ 
 \hdashline 
     &           &    1 &  5.18e-04 &  2.67e-02 &      \\ 
     &           &    2 &  2.27e-13 &  2.15e-16 &      \\ 
 100 &  9.90e+01 &    2 &           &           & forces  \\ 
 \hdashline 
     &           &    1 &  5.18e-04 &  2.67e-02 &      \\ 
     &           &    2 &  2.27e-13 &  2.15e-16 &      \\ 
 101 &  1.00e+02 &    2 &           &           & forces  \\ 
 \hdashline 
     &           &    1 &  5.18e-04 &  2.67e-02 &      \\ 
     &           &    2 &  2.27e-13 &  2.15e-16 &      \\ 
 102 &  1.01e+02 &    2 &           &           & forces  \\ 
 \hdashline 
     &           &    1 &  5.18e-04 &  2.67e-02 &      \\ 
     &           &    2 &  2.27e-13 &  2.15e-16 &      \\ 
 103 &  1.02e+02 &    2 &           &           & forces  \\ 
 \hdashline 
     &           &    1 &  5.18e-04 &  2.67e-02 &      \\ 
     &           &    2 &  2.27e-13 &  2.15e-16 &      \\ 
 104 &  1.03e+02 &    2 &           &           & forces  \\ 
 \hdashline 
     &           &    1 &  5.18e-04 &  2.67e-02 &      \\ 
     &           &    2 &  2.27e-13 &  2.15e-16 &      \\ 
 105 &  1.04e+02 &    2 &           &           & forces  \\ 
 \hdashline 
     &           &    1 &  5.18e-04 &  2.67e-02 &      \\ 
     &           &    2 &  2.27e-13 &  2.15e-16 &      \\ 
 106 &  1.05e+02 &    2 &           &           & forces  \\ 
 \hdashline 
     &           &    1 &  5.17e-04 &  2.67e-02 &      \\ 
     &           &    2 &  2.27e-13 &  2.15e-16 &      \\ 
 107 &  1.06e+02 &    2 &           &           & forces  \\ 
 \hdashline 
     &           &    1 &  5.17e-04 &  2.67e-02 &      \\ 
     &           &    2 &  2.27e-13 &  2.15e-16 &      \\ 
 108 &  1.07e+02 &    2 &           &           & forces  \\ 
 \hdashline 
     &           &    1 &  5.17e-04 &  2.67e-02 &      \\ 
     &           &    2 &  2.27e-13 &  2.15e-16 &      \\ 
 109 &  1.08e+02 &    2 &           &           & forces  \\ 
 \hdashline 
     &           &    1 &  5.17e-04 &  2.67e-02 &      \\ 
     &           &    2 &  2.27e-13 &  2.15e-16 &      \\ 
 110 &  1.09e+02 &    2 &           &           & forces  \\ 
 \hdashline 
     &           &    1 &  5.17e-04 &  2.67e-02 &      \\ 
     &           &    2 &  2.27e-13 &  2.15e-16 &      \\ 
 111 &  1.10e+02 &    2 &           &           & forces  \\ 
 \hdashline 
     &           &    1 &  5.17e-04 &  2.67e-02 &      \\ 
     &           &    2 &  2.27e-13 &  2.15e-16 &      \\ 
 112 &  1.11e+02 &    2 &           &           & forces  \\ 
 \hdashline 
     &           &    1 &  5.17e-04 &  2.67e-02 &      \\ 
     &           &    2 &  2.27e-13 &  2.15e-16 &      \\ 
 113 &  1.12e+02 &    2 &           &           & forces  \\ 
 \hdashline 
     &           &    1 &  5.17e-04 &  2.67e-02 &      \\ 
     &           &    2 &  2.27e-13 &  2.15e-16 &      \\ 
 114 &  1.13e+02 &    2 &           &           & forces  \\ 
 \hdashline 
     &           &    1 &  3.39e+00 &  2.67e-02 &      \\ 
     &           &    2 &  1.14e-13 &  2.15e-16 &      \\ 
 115 &  1.14e+02 &    2 &           &           & forces  \\ 
 \hdashline 
     &           &    1 &  1.39e+01 &  2.67e-02 &      \\ 
     &           &    2 &  5.68e-14 &  1.08e-16 &      \\ 
 116 &  1.15e+02 &    2 &           &           & forces  \\ 
 \hdashline 
     &           &    1 &  5.17e-04 &  2.67e-02 &      \\ 
     &           &    2 &  0.00e+00 &  1.04e-16 &      \\ 
 117 &  1.16e+02 &    2 &           &           & displac  \\ 
 \hdashline 
     &           &    1 &  5.17e-04 &  2.67e-02 &      \\ 
     &           &    2 &  0.00e+00 &  0.00e+00 &      \\ 
 118 &  1.17e+02 &    2 &           &           & displac  \\ 
 \hdashline 
     &           &    1 &  5.17e-04 &  2.67e-02 &      \\ 
     &           &    2 &  1.14e-13 &  2.12e-16 &      \\ 
 119 &  1.18e+02 &    2 &           &           & forces  \\ 
 \hdashline 
     &           &    1 &  5.16e-04 &  2.67e-02 &      \\ 
     &           &    2 &  1.14e-13 &  2.12e-16 &      \\ 
 120 &  1.19e+02 &    2 &           &           & forces  \\ 
 \hdashline 
     &           &    1 &  5.16e-04 &  2.67e-02 &      \\ 
     &           &    2 &  1.14e-13 &  2.12e-16 &      \\ 
 121 &  1.20e+02 &    2 &           &           & forces  \\ 
 \hdashline 
 
\end{longtable}

\end{document}